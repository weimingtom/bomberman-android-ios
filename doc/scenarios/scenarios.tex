\documentclass{report}

\usepackage[utf8]{inputenc}
\usepackage[francais]{babel}
\usepackage{a4wide}

\usepackage{pstricks}

\usepackage[colorlinks, linkcolor=blue]{hyperref}


\begin{document}

\chapter{Annexes}

	\section{Scenarios}
	
		Description de la partie \ldots
		
		\newpage

\newpage
	
	\subsection{Chargement de l'application}
	
		\hypertarget{Chargement de l'application}{}
		\label{Chargement de l'application}

		\begin{center}
			%LaTeX with PSTricks extensions
%%Creator: inkscape 0.48.0
%%Please note this file requires PSTricks extensions
\psset{xunit=.4pt,yunit=.4pt,runit=.4pt}
\begin{pspicture}(560,600)
{
\newrgbcolor{curcolor}{1 1 1}
\pscustom[linestyle=none,fillstyle=solid,fillcolor=curcolor]
{
\newpath
\moveto(133.12401716,597.5221417)
\lineto(426.87598554,597.5221417)
\curveto(443.85397304,597.5221417)(457.52217237,583.85394237)(457.52217237,566.87595487)
\lineto(457.52217237,33.124017)
\curveto(457.52217237,16.1460295)(443.85397304,2.47783017)(426.87598554,2.47783017)
\lineto(133.12401716,2.47783017)
\curveto(116.14602965,2.47783017)(102.47783033,16.1460295)(102.47783033,33.124017)
\lineto(102.47783033,566.87595487)
\curveto(102.47783033,583.85394237)(116.14602965,597.5221417)(133.12401716,597.5221417)
\closepath
}
}
{
\newrgbcolor{curcolor}{0 0 0}
\pscustom[linewidth=4.95566034,linecolor=curcolor]
{
\newpath
\moveto(133.12401716,597.5221417)
\lineto(426.87598554,597.5221417)
\curveto(443.85397304,597.5221417)(457.52217237,583.85394237)(457.52217237,566.87595487)
\lineto(457.52217237,33.124017)
\curveto(457.52217237,16.1460295)(443.85397304,2.47783017)(426.87598554,2.47783017)
\lineto(133.12401716,2.47783017)
\curveto(116.14602965,2.47783017)(102.47783033,16.1460295)(102.47783033,33.124017)
\lineto(102.47783033,566.87595487)
\curveto(102.47783033,583.85394237)(116.14602965,597.5221417)(133.12401716,597.5221417)
\closepath
}
}
{
\newrgbcolor{curcolor}{1 1 1}
\pscustom[linestyle=none,fillstyle=solid,fillcolor=curcolor]
{
\newpath
\moveto(260.00365964,158.92373777)
\lineto(309.96551649,158.92373777)
\lineto(309.96551649,110.05766797)
\lineto(260.00365964,110.05766797)
\closepath
}
}
{
\newrgbcolor{curcolor}{0 0 0}
\pscustom[linewidth=2,linecolor=curcolor]
{
\newpath
\moveto(260.00365964,158.92373777)
\lineto(309.96551649,158.92373777)
\lineto(309.96551649,110.05766797)
\lineto(260.00365964,110.05766797)
\closepath
}
}
{
\newrgbcolor{curcolor}{0 0 0}
\pscustom[linestyle=none,fillstyle=solid,fillcolor=curcolor]
{
\newpath
\moveto(43.96874944,502.65625119)
\lineto(49.12499944,502.65625119)
\lineto(49.12499944,520.45312619)
\lineto(43.51562444,519.32812619)
\lineto(43.51562444,522.20312619)
\lineto(49.09374944,523.32812619)
\lineto(52.24999944,523.32812619)
\lineto(52.24999944,502.65625119)
\lineto(57.40624944,502.65625119)
\lineto(57.40624944,500.00000119)
\lineto(43.96874944,500.00000119)
\lineto(43.96874944,502.65625119)
}
}
{
\newrgbcolor{curcolor}{0 0 0}
\pscustom[linestyle=none,fillstyle=solid,fillcolor=curcolor]
{
\newpath
\moveto(506.14062254,122.65622068)
\lineto(517.15624754,122.65622068)
\lineto(517.15624754,119.99997068)
\lineto(502.34374754,119.99997068)
\lineto(502.34374754,122.65622068)
\curveto(503.54166066,123.89580011)(505.17186736,125.55725679)(507.23437254,127.64059568)
\curveto(509.30727989,129.73433594)(510.60936193,131.08329293)(511.14062254,131.68747068)
\curveto(512.15102705,132.82287452)(512.85415135,133.78120689)(513.24999754,134.56247068)
\curveto(513.65623388,135.35412199)(513.85935868,136.13016288)(513.85937254,136.89059568)
\curveto(513.85935868,138.13016088)(513.42185911,139.14057654)(512.54687254,139.92184568)
\curveto(511.68227752,140.70307497)(510.55207032,141.09369958)(509.15624754,141.09372068)
\curveto(508.16665604,141.09369958)(507.11978208,140.92182475)(506.01562254,140.57809568)
\curveto(504.92186761,140.23432544)(503.74999379,139.71349263)(502.49999754,139.01559568)
\lineto(502.49999754,142.20309568)
\curveto(503.7708271,142.71348963)(504.95832591,143.09890591)(506.06249754,143.35934568)
\curveto(507.16665704,143.61973872)(508.17707269,143.74994693)(509.09374754,143.74997068)
\curveto(511.51040269,143.74994693)(513.4374841,143.14578086)(514.87499754,141.93747068)
\curveto(516.31248122,140.72911661)(517.0312305,139.11453489)(517.03124754,137.09372068)
\curveto(517.0312305,136.13537121)(516.84893902,135.22391379)(516.48437254,134.35934568)
\curveto(516.13018974,133.5051655)(515.47914872,132.49474985)(514.53124754,131.32809568)
\curveto(514.2708166,131.02600132)(513.44269243,130.15100219)(512.04687254,128.70309568)
\curveto(510.65102855,127.26558841)(508.68228052,125.24996543)(506.14062254,122.65622068)
}
}
{
\newrgbcolor{curcolor}{0 0 0}
\pscustom[linewidth=2,linecolor=curcolor,linestyle=dashed,dash=8 8]
{
\newpath
\moveto(150,510)
\lineto(60,510)
}
}
{
\newrgbcolor{curcolor}{0 0 0}
\pscustom[linestyle=none,fillstyle=solid,fillcolor=curcolor]
{
\newpath
\moveto(139.53769464,514.84048224)
\lineto(152.6487474,510.01921591)
\lineto(139.53769392,505.19795064)
\curveto(141.632292,508.04442372)(141.62022288,511.93889292)(139.53769464,514.84048224)
\lineto(139.53769464,514.84048224)
\closepath
}
}
{
\newrgbcolor{curcolor}{0 0 0}
\pscustom[linewidth=2,linecolor=curcolor,linestyle=dashed,dash=8 8]
{
\newpath
\moveto(290,130)
\lineto(500,130)
}
}
{
\newrgbcolor{curcolor}{0 0 0}
\pscustom[linestyle=none,fillstyle=solid,fillcolor=curcolor]
{
\newpath
\moveto(300.46230536,125.15951776)
\lineto(287.3512526,129.98078409)
\lineto(300.46230608,134.80204936)
\curveto(298.367708,131.95557628)(298.37977712,128.06110708)(300.46230536,125.15951776)
\lineto(300.46230536,125.15951776)
\closepath
}
}
\end{pspicture}

		\end{center}
		
		\begin{enumerate}
		  \item Fond d'écran (présent dans l'archive stocké sur le téléphone).
		  \item Spinner (présent dans l'archive stocké sur le téléphone).
		\end{enumerate}
		
		\subsubsection{Description des zones}

			\begin{tabular}{|c|c|c|c|c|} \hline
				Numéro de zone & Type  & Description & Evènement &	Règle \\\hline 
				2 & Barre de & Montre l'avancement du & Chargement de & RG0-01\\
				  & chargement & chargement de l'application & l'application &\\\hline
			\end{tabular}
		
		\subsubsection{Description des règles}
		Lancement de l’application 
		
		\underline{RG0-01 :}
		\begin{quote}
			Déclenchée au démarrage de l'application.\\
			Affichage du spinner montrant le chargement de l'application.\\
			\underline{Chargement de l'application :}
				\begin{quote}
					Chargement de la page d'accueil%
					\footnote[1]{
						\hyperlink{Page d'accueil}{Page d'accueil}
						\og voir section \ref{accueil}, page \pageref{accueil}.\fg
					}.\\
					Vérification d'un compte local existant (stocké sur le téléphone).\\
					Si aucun compte n'est trouvé alors la page de création du profil%
					\footnote[2]{
						\hyperlink{Création du profil}{Page de création du profil}
						\og voir section \ref{profil}, page \pageref{profil}.\fg
					}
					est chargée puis affichée.\\
					Sinon la page d'accueil%
					\footnotemark[1]
					est affichée directement. \\
					La page représentant le chargement de l'application est détruite une fois le chargement de l'application terminé.\\
				\end{quote}
		\end{quote}

	
\newpage

	\subsection{Création du profil}
		\hypertarget{profil}{}
		\label{profil}

		\begin{center}
			%LaTeX with PSTricks extensions
%%Creator: inkscape 0.48.0
%%Please note this file requires PSTricks extensions
\psset{xunit=.4pt,yunit=.4pt,runit=.4pt}
\begin{pspicture}(560,600)
{
\newrgbcolor{curcolor}{1 1 1}
\pscustom[linestyle=none,fillstyle=solid,fillcolor=curcolor]
{
\newpath
\moveto(133.17790741,598.5302021)
\lineto(426.82210296,598.5302021)
\curveto(443.82879388,598.5302021)(457.52010101,584.83889497)(457.52010101,567.83220405)
\lineto(457.52010101,33.17790718)
\curveto(457.52010101,16.17121626)(443.82879388,2.47990913)(426.82210296,2.47990913)
\lineto(133.17790741,2.47990913)
\curveto(116.17121649,2.47990913)(102.47990936,16.17121626)(102.47990936,33.17790718)
\lineto(102.47990936,567.83220405)
\curveto(102.47990936,584.83889497)(116.17121649,598.5302021)(133.17790741,598.5302021)
\closepath
}
}
{
\newrgbcolor{curcolor}{0 0 0}
\pscustom[linewidth=4.95981836,linecolor=curcolor]
{
\newpath
\moveto(133.17790741,598.5302021)
\lineto(426.82210296,598.5302021)
\curveto(443.82879388,598.5302021)(457.52010101,584.83889497)(457.52010101,567.83220405)
\lineto(457.52010101,33.17790718)
\curveto(457.52010101,16.17121626)(443.82879388,2.47990913)(426.82210296,2.47990913)
\lineto(133.17790741,2.47990913)
\curveto(116.17121649,2.47990913)(102.47990936,16.17121626)(102.47990936,33.17790718)
\lineto(102.47990936,567.83220405)
\curveto(102.47990936,584.83889497)(116.17121649,598.5302021)(133.17790741,598.5302021)
\closepath
}
}
{
\newrgbcolor{curcolor}{1 1 1}
\pscustom[linestyle=none,fillstyle=solid,fillcolor=curcolor]
{
\newpath
\moveto(164.38434976,370.09907455)
\lineto(396.60310358,370.09907455)
\curveto(409.61615135,370.09907455)(420.09235948,359.62286642)(420.09235948,346.60981865)
\lineto(420.09235948,243.48925514)
\curveto(420.09235948,230.47620737)(409.61615135,219.99999924)(396.60310358,219.99999924)
\lineto(164.38434976,219.99999924)
\curveto(151.37130199,219.99999924)(140.89509386,230.47620737)(140.89509386,243.48925514)
\lineto(140.89509386,346.60981865)
\curveto(140.89509386,359.62286642)(151.37130199,370.09907455)(164.38434976,370.09907455)
\closepath
}
}
{
\newrgbcolor{curcolor}{0 0 0}
\pscustom[linewidth=1.75071895,linecolor=curcolor]
{
\newpath
\moveto(164.38434976,370.09907455)
\lineto(396.60310358,370.09907455)
\curveto(409.61615135,370.09907455)(420.09235948,359.62286642)(420.09235948,346.60981865)
\lineto(420.09235948,243.48925514)
\curveto(420.09235948,230.47620737)(409.61615135,219.99999924)(396.60310358,219.99999924)
\lineto(164.38434976,219.99999924)
\curveto(151.37130199,219.99999924)(140.89509386,230.47620737)(140.89509386,243.48925514)
\lineto(140.89509386,346.60981865)
\curveto(140.89509386,359.62286642)(151.37130199,370.09907455)(164.38434976,370.09907455)
\closepath
}
}
{
\newrgbcolor{curcolor}{0 0 0}
\pscustom[linestyle=none,fillstyle=solid,fillcolor=curcolor]
{
\newpath
\moveto(164.7226581,345.55079574)
\lineto(164.7226581,338.97657699)
\lineto(167.6992206,338.97657699)
\curveto(168.8007743,338.97656802)(169.65233595,339.26172398)(170.2539081,339.83204574)
\curveto(170.85545974,340.40234784)(171.15624069,341.21484703)(171.15625185,342.26954574)
\curveto(171.15624069,343.31640743)(170.85545974,344.12500037)(170.2539081,344.69532699)
\curveto(169.65233595,345.26562423)(168.8007743,345.55078019)(167.6992206,345.55079574)
\lineto(164.7226581,345.55079574)
\moveto(162.3554706,347.49610824)
\lineto(167.6992206,347.49610824)
\curveto(169.66014844,347.49609075)(171.14061571,347.05077869)(172.14062685,346.16017074)
\curveto(173.1484262,345.27734297)(173.65233195,343.98046926)(173.6523456,342.26954574)
\curveto(173.65233195,340.5429727)(173.1484262,339.23828651)(172.14062685,338.35548324)
\curveto(171.14061571,337.47266327)(169.66014844,337.03125746)(167.6992206,337.03126449)
\lineto(164.7226581,337.03126449)
\lineto(164.7226581,330.00001449)
\lineto(162.3554706,330.00001449)
\lineto(162.3554706,347.49610824)
}
}
{
\newrgbcolor{curcolor}{0 0 0}
\pscustom[linestyle=none,fillstyle=solid,fillcolor=curcolor]
{
\newpath
\moveto(184.6914081,342.73829574)
\lineto(184.6914081,340.69923324)
\curveto(184.08202308,341.01172223)(183.44921121,341.246097)(182.7929706,341.40235824)
\curveto(182.13671252,341.55859669)(181.4570257,341.63672161)(180.7539081,341.63673324)
\curveto(179.68358998,341.63672161)(178.87890328,341.47265927)(178.3398456,341.14454574)
\curveto(177.80859185,340.81640993)(177.54296712,340.32422292)(177.5429706,339.66798324)
\curveto(177.54296712,339.16797408)(177.73437318,338.77344322)(178.11718935,338.48438949)
\curveto(178.49999741,338.20313129)(179.26952789,337.93360031)(180.4257831,337.67579574)
\lineto(181.16406435,337.51173324)
\curveto(182.69530571,337.18360106)(183.78124213,336.71875778)(184.42187685,336.11720199)
\curveto(185.07030334,335.52344647)(185.39452177,334.69141605)(185.3945331,333.62110824)
\curveto(185.39452177,332.40235584)(184.91014725,331.43751306)(183.9414081,330.72657699)
\curveto(182.98046168,330.01563948)(181.65624425,329.66017108)(179.96875185,329.66017074)
\curveto(179.26562164,329.66017108)(178.53124738,329.73048351)(177.76562685,329.87110824)
\curveto(177.0078114,330.00392074)(176.20703095,330.20704554)(175.3632831,330.48048324)
\lineto(175.3632831,332.70704574)
\curveto(176.160156,332.29298095)(176.94531146,331.98048126)(177.71875185,331.76954574)
\curveto(178.49218492,331.56641918)(179.25780915,331.46485678)(180.01562685,331.46485824)
\curveto(181.03124488,331.46485678)(181.8124941,331.63673161)(182.35937685,331.98048324)
\curveto(182.906243,332.33204341)(183.17968023,332.82423042)(183.17968935,333.45704574)
\curveto(183.17968023,334.0429792)(182.98046168,334.4921975)(182.5820331,334.80470199)
\curveto(182.19139997,335.11719688)(181.32811958,335.41797783)(179.99218935,335.70704574)
\lineto(179.24218935,335.88282699)
\curveto(177.906248,336.16407083)(176.94140522,336.5937579)(176.3476581,337.17188949)
\curveto(175.75390641,337.75781924)(175.4570317,338.55859969)(175.4570331,339.57423324)
\curveto(175.4570317,340.80859744)(175.89453127,341.76172148)(176.7695331,342.43360824)
\curveto(177.64452952,343.10547014)(178.88671577,343.4414073)(180.4960956,343.44142074)
\curveto(181.29296337,343.4414073)(182.04296262,343.38281361)(182.7460956,343.26563949)
\curveto(183.44921121,343.14843885)(184.09764806,342.97265777)(184.6914081,342.73829574)
}
}
{
\newrgbcolor{curcolor}{0 0 0}
\pscustom[linestyle=none,fillstyle=solid,fillcolor=curcolor]
{
\newpath
\moveto(200.0664081,337.10157699)
\lineto(200.0664081,336.04688949)
\lineto(190.1523456,336.04688949)
\curveto(190.24609193,334.56250993)(190.69140398,333.42969856)(191.4882831,332.64845199)
\curveto(192.29296488,331.87501262)(193.41015127,331.48829426)(194.8398456,331.48829574)
\curveto(195.66796151,331.48829426)(196.46874196,331.58985665)(197.24218935,331.79298324)
\curveto(198.0234279,331.99610625)(198.79686463,332.30079344)(199.56250185,332.70704574)
\lineto(199.56250185,330.66798324)
\curveto(198.78905214,330.3398579)(197.99608418,330.08985815)(197.1835956,329.91798324)
\curveto(196.3710858,329.7461085)(195.54686788,329.66017108)(194.71093935,329.66017074)
\curveto(192.61718331,329.66017108)(190.95702872,330.26954547)(189.7304706,331.48829574)
\curveto(188.51171866,332.70704304)(187.90234427,334.35547889)(187.9023456,336.43360824)
\curveto(187.90234427,338.58203716)(188.4804687,340.28516046)(189.6367206,341.54298324)
\curveto(190.80077887,342.80859544)(192.36718356,343.4414073)(194.33593935,343.44142074)
\curveto(196.10155482,343.4414073)(197.49608468,342.87109537)(198.5195331,341.73048324)
\curveto(199.55077012,340.59766015)(200.06639461,339.05469294)(200.0664081,337.10157699)
\moveto(197.9101581,337.73438949)
\curveto(197.89452178,338.91406808)(197.56249086,339.85547339)(196.91406435,340.55860824)
\curveto(196.27342965,341.26172198)(195.421868,341.61328413)(194.35937685,341.61329574)
\curveto(193.15624527,341.61328413)(192.19140248,341.27344072)(191.4648456,340.59376449)
\curveto(190.74609143,339.91406708)(190.33202934,338.95703679)(190.2226581,337.72267074)
\lineto(197.9101581,337.73438949)
}
}
{
\newrgbcolor{curcolor}{0 0 0}
\pscustom[linestyle=none,fillstyle=solid,fillcolor=curcolor]
{
\newpath
\moveto(203.38281435,335.17970199)
\lineto(203.38281435,343.12501449)
\lineto(205.53906435,343.12501449)
\lineto(205.53906435,335.26173324)
\curveto(205.53906015,334.01954172)(205.78124741,333.08594891)(206.26562685,332.46095199)
\curveto(206.74999644,331.84376265)(207.47655821,331.53516921)(208.44531435,331.53517074)
\curveto(209.60936858,331.53516921)(210.52733641,331.90626259)(211.1992206,332.64845199)
\curveto(211.87889756,333.3906361)(212.21874097,334.40235384)(212.21875185,335.68360824)
\lineto(212.21875185,343.12501449)
\lineto(214.37500185,343.12501449)
\lineto(214.37500185,330.00001449)
\lineto(212.21875185,330.00001449)
\lineto(212.21875185,332.01563949)
\curveto(211.695304,331.21876328)(211.08592961,330.62501387)(210.39062685,330.23438949)
\curveto(209.70311849,329.85157714)(208.90233804,329.66017108)(207.9882831,329.66017074)
\curveto(206.48046546,329.66017108)(205.33593536,330.12892062)(204.55468935,331.06642074)
\curveto(203.77343692,332.00391874)(203.38281231,333.37501112)(203.38281435,335.17970199)
\moveto(208.8085956,343.44142074)
\lineto(208.8085956,343.44142074)
}
}
{
\newrgbcolor{curcolor}{0 0 0}
\pscustom[linestyle=none,fillstyle=solid,fillcolor=curcolor]
{
\newpath
\moveto(227.47656435,341.13282699)
\lineto(227.47656435,348.23438949)
\lineto(229.63281435,348.23438949)
\lineto(229.63281435,330.00001449)
\lineto(227.47656435,330.00001449)
\lineto(227.47656435,331.96876449)
\curveto(227.0234289,331.18751331)(226.44921073,330.60548264)(225.7539081,330.22267074)
\curveto(225.06639961,329.8476709)(224.23827544,329.66017108)(223.2695331,329.66017074)
\curveto(221.68359049,329.66017108)(220.39062304,330.29298295)(219.39062685,331.55860824)
\curveto(218.39843753,332.82423042)(217.90234427,334.48829126)(217.9023456,336.55079574)
\curveto(217.90234427,338.61328713)(218.39843753,340.27734797)(219.39062685,341.54298324)
\curveto(220.39062304,342.80859544)(221.68359049,343.4414073)(223.2695331,343.44142074)
\curveto(224.23827544,343.4414073)(225.06639961,343.25000124)(225.7539081,342.86720199)
\curveto(226.44921073,342.4921895)(227.0234289,341.91406508)(227.47656435,341.13282699)
\moveto(220.1289081,336.55079574)
\curveto(220.12890455,334.96485328)(220.45312297,333.71876078)(221.10156435,332.81251449)
\curveto(221.75780917,331.91407508)(222.65624577,331.46485678)(223.79687685,331.46485824)
\curveto(224.93749349,331.46485678)(225.83593009,331.91407508)(226.49218935,332.81251449)
\curveto(227.14842878,333.71876078)(227.47655345,334.96485328)(227.47656435,336.55079574)
\curveto(227.47655345,338.13672511)(227.14842878,339.37891137)(226.49218935,340.27735824)
\curveto(225.83593009,341.18359706)(224.93749349,341.63672161)(223.79687685,341.63673324)
\curveto(222.65624577,341.63672161)(221.75780917,341.18359706)(221.10156435,340.27735824)
\curveto(220.45312297,339.37891137)(220.12890455,338.13672511)(220.1289081,336.55079574)
}
}
{
\newrgbcolor{curcolor}{0 0 0}
\pscustom[linestyle=none,fillstyle=solid,fillcolor=curcolor]
{
\newpath
\moveto(239.1601581,341.61329574)
\curveto(238.00390191,341.61328413)(237.08984032,341.16015958)(236.4179706,340.25392074)
\curveto(235.74609166,339.35547389)(235.4101545,338.12110012)(235.4101581,336.55079574)
\curveto(235.4101545,334.98047826)(235.74218542,333.74219825)(236.40625185,332.83595199)
\curveto(237.07812158,331.93751256)(237.99608941,331.48829426)(239.1601581,331.48829574)
\curveto(240.3085871,331.48829426)(241.21874244,331.9414188)(241.89062685,332.84767074)
\curveto(242.5624911,333.75391699)(242.89842826,334.98829076)(242.89843935,336.55079574)
\curveto(242.89842826,338.10547514)(242.5624911,339.33594266)(241.89062685,340.24220199)
\curveto(241.21874244,341.15625334)(240.3085871,341.61328413)(239.1601581,341.61329574)
\moveto(239.1601581,343.44142074)
\curveto(241.03514888,343.4414073)(242.50780365,342.83203291)(243.57812685,341.61329574)
\curveto(244.64842651,340.39453535)(245.18358223,338.70703704)(245.1835956,336.55079574)
\curveto(245.18358223,334.40235384)(244.64842651,332.71485553)(243.57812685,331.48829574)
\curveto(242.50780365,330.26954547)(241.03514888,329.66017108)(239.1601581,329.66017074)
\curveto(237.27734013,329.66017108)(235.80077911,330.26954547)(234.7304706,331.48829574)
\curveto(233.66796874,332.71485553)(233.13671927,334.40235384)(233.1367206,336.55079574)
\curveto(233.13671927,338.70703704)(233.66796874,340.39453535)(234.7304706,341.61329574)
\curveto(235.80077911,342.83203291)(237.27734013,343.4414073)(239.1601581,343.44142074)
}
}
{
\newrgbcolor{curcolor}{0 0 0}
\pscustom[linestyle=none,fillstyle=solid,fillcolor=curcolor,opacity=0.15686275]
{
\newpath
\moveto(265.28100675,349.97837753)
\lineto(384.74003881,349.97837753)
\curveto(393.21111723,349.97837753)(400.03079408,346.49264783)(400.03079408,342.16284008)
\lineto(400.03079408,327.93718166)
\curveto(400.03079408,323.60737391)(393.21111723,320.12164421)(384.74003881,320.12164421)
\lineto(265.28100675,320.12164421)
\curveto(256.80992833,320.12164421)(249.99025148,323.60737391)(249.99025148,327.93718166)
\lineto(249.99025148,342.16284008)
\curveto(249.99025148,346.49264783)(256.80992833,349.97837753)(265.28100675,349.97837753)
\closepath
}
}
{
\newrgbcolor{curcolor}{0 0 0}
\pscustom[linewidth=2,linecolor=curcolor]
{
\newpath
\moveto(265.28100675,349.97837753)
\lineto(384.74003881,349.97837753)
\curveto(393.21111723,349.97837753)(400.03079408,346.49264783)(400.03079408,342.16284008)
\lineto(400.03079408,327.93718166)
\curveto(400.03079408,323.60737391)(393.21111723,320.12164421)(384.74003881,320.12164421)
\lineto(265.28100675,320.12164421)
\curveto(256.80992833,320.12164421)(249.99025148,323.60737391)(249.99025148,327.93718166)
\lineto(249.99025148,342.16284008)
\curveto(249.99025148,346.49264783)(256.80992833,349.97837753)(265.28100675,349.97837753)
\closepath
}
}
{
\newrgbcolor{curcolor}{1 1 1}
\pscustom[linestyle=none,fillstyle=solid,fillcolor=curcolor]
{
\newpath
\moveto(234.64676946,269.59635849)
\lineto(325.16059202,269.59635849)
\curveto(333.3281575,269.59635849)(339.90349001,263.02102599)(339.90349001,254.8534605)
\curveto(339.90349001,246.68589502)(333.3281575,240.11056251)(325.16059202,240.11056251)
\lineto(234.64676946,240.11056251)
\curveto(226.47920398,240.11056251)(219.90387148,246.68589502)(219.90387148,254.8534605)
\curveto(219.90387148,263.02102599)(226.47920398,269.59635849)(234.64676946,269.59635849)
\closepath
}
}
{
\newrgbcolor{curcolor}{0 0 0}
\pscustom[linewidth=2,linecolor=curcolor]
{
\newpath
\moveto(234.64676946,269.59635849)
\lineto(325.16059202,269.59635849)
\curveto(333.3281575,269.59635849)(339.90349001,263.02102599)(339.90349001,254.8534605)
\curveto(339.90349001,246.68589502)(333.3281575,240.11056251)(325.16059202,240.11056251)
\lineto(234.64676946,240.11056251)
\curveto(226.47920398,240.11056251)(219.90387148,246.68589502)(219.90387148,254.8534605)
\curveto(219.90387148,263.02102599)(226.47920398,269.59635849)(234.64676946,269.59635849)
\closepath
}
}
{
\newrgbcolor{curcolor}{0 0 0}
\pscustom[linestyle=none,fillstyle=solid,fillcolor=curcolor]
{
\newpath
\moveto(246.86718935,250.00001449)
\lineto(240.18750185,267.49610824)
\lineto(242.6601581,267.49610824)
\lineto(248.20312685,252.76563949)
\lineto(253.75781435,267.49610824)
\lineto(256.21875185,267.49610824)
\lineto(249.5507831,250.00001449)
\lineto(246.86718935,250.00001449)
}
}
{
\newrgbcolor{curcolor}{0 0 0}
\pscustom[linestyle=none,fillstyle=solid,fillcolor=curcolor]
{
\newpath
\moveto(262.75781435,256.59767074)
\curveto(261.01562036,256.59766415)(259.80859032,256.3984456)(259.1367206,256.00001449)
\curveto(258.46484166,255.60157139)(258.1289045,254.92188457)(258.1289081,253.96095199)
\curveto(258.1289045,253.1953238)(258.37890425,252.58594941)(258.8789081,252.13282699)
\curveto(259.38671574,251.68751281)(260.07421505,251.46485678)(260.9414081,251.46485824)
\curveto(262.13671299,251.46485678)(263.09374329,251.88673136)(263.81250185,252.73048324)
\curveto(264.53905434,253.58204216)(264.90233523,254.71094728)(264.9023456,256.11720199)
\lineto(264.9023456,256.59767074)
\lineto(262.75781435,256.59767074)
\moveto(267.0585956,257.48829574)
\lineto(267.0585956,250.00001449)
\lineto(264.9023456,250.00001449)
\lineto(264.9023456,251.99220199)
\curveto(264.41014822,251.1953258)(263.79686758,250.60548264)(263.06250185,250.22267074)
\curveto(262.32811905,249.8476709)(261.42968245,249.66017108)(260.36718935,249.66017074)
\curveto(259.02343486,249.66017108)(257.95312343,250.03517071)(257.15625185,250.78517074)
\curveto(256.36718751,251.5429817)(255.97265666,252.55469944)(255.9726581,253.82032699)
\curveto(255.97265666,255.2968842)(256.46484366,256.41016433)(257.4492206,257.16017074)
\curveto(258.44140419,257.91016283)(259.91796521,258.28516246)(261.8789081,258.28517074)
\lineto(264.9023456,258.28517074)
\lineto(264.9023456,258.49610824)
\curveto(264.90233523,259.48828626)(264.57421055,260.25391049)(263.9179706,260.79298324)
\curveto(263.26952436,261.3398469)(262.35546277,261.61328413)(261.1757831,261.61329574)
\curveto(260.4257772,261.61328413)(259.69530918,261.52344047)(258.98437685,261.34376449)
\curveto(258.27343561,261.16406583)(257.58984254,260.89453485)(256.9335956,260.53517074)
\lineto(256.9335956,262.52735824)
\curveto(257.72265491,262.83203291)(258.48827914,263.05859519)(259.2304706,263.20704574)
\curveto(259.97265266,263.36328238)(260.69530818,263.4414073)(261.39843935,263.44142074)
\curveto(263.29686808,263.4414073)(264.71483541,262.94922029)(265.6523456,261.96485824)
\curveto(266.58983354,260.98047226)(267.05858307,259.48828626)(267.0585956,257.48829574)
}
}
{
\newrgbcolor{curcolor}{0 0 0}
\pscustom[linestyle=none,fillstyle=solid,fillcolor=curcolor]
{
\newpath
\moveto(271.5117206,268.23438949)
\lineto(273.6679706,268.23438949)
\lineto(273.6679706,250.00001449)
\lineto(271.5117206,250.00001449)
\lineto(271.5117206,268.23438949)
}
}
{
\newrgbcolor{curcolor}{0 0 0}
\pscustom[linestyle=none,fillstyle=solid,fillcolor=curcolor]
{
\newpath
\moveto(278.1679706,263.12501449)
\lineto(280.3242206,263.12501449)
\lineto(280.3242206,250.00001449)
\lineto(278.1679706,250.00001449)
\lineto(278.1679706,263.12501449)
\moveto(278.1679706,268.23438949)
\lineto(280.3242206,268.23438949)
\lineto(280.3242206,265.50392074)
\lineto(278.1679706,265.50392074)
\lineto(278.1679706,268.23438949)
}
}
{
\newrgbcolor{curcolor}{0 0 0}
\pscustom[linestyle=none,fillstyle=solid,fillcolor=curcolor]
{
\newpath
\moveto(293.46093935,261.13282699)
\lineto(293.46093935,268.23438949)
\lineto(295.61718935,268.23438949)
\lineto(295.61718935,250.00001449)
\lineto(293.46093935,250.00001449)
\lineto(293.46093935,251.96876449)
\curveto(293.0078039,251.18751331)(292.43358573,250.60548264)(291.7382831,250.22267074)
\curveto(291.05077461,249.8476709)(290.22265044,249.66017108)(289.2539081,249.66017074)
\curveto(287.66796549,249.66017108)(286.37499804,250.29298295)(285.37500185,251.55860824)
\curveto(284.38281253,252.82423042)(283.88671927,254.48829126)(283.8867206,256.55079574)
\curveto(283.88671927,258.61328713)(284.38281253,260.27734797)(285.37500185,261.54298324)
\curveto(286.37499804,262.80859544)(287.66796549,263.4414073)(289.2539081,263.44142074)
\curveto(290.22265044,263.4414073)(291.05077461,263.25000124)(291.7382831,262.86720199)
\curveto(292.43358573,262.4921895)(293.0078039,261.91406508)(293.46093935,261.13282699)
\moveto(286.1132831,256.55079574)
\curveto(286.11327955,254.96485328)(286.43749797,253.71876078)(287.08593935,252.81251449)
\curveto(287.74218417,251.91407508)(288.64062077,251.46485678)(289.78125185,251.46485824)
\curveto(290.92186849,251.46485678)(291.82030509,251.91407508)(292.47656435,252.81251449)
\curveto(293.13280378,253.71876078)(293.46092845,254.96485328)(293.46093935,256.55079574)
\curveto(293.46092845,258.13672511)(293.13280378,259.37891137)(292.47656435,260.27735824)
\curveto(291.82030509,261.18359706)(290.92186849,261.63672161)(289.78125185,261.63673324)
\curveto(288.64062077,261.63672161)(287.74218417,261.18359706)(287.08593935,260.27735824)
\curveto(286.43749797,259.37891137)(286.11327955,258.13672511)(286.1132831,256.55079574)
}
}
{
\newrgbcolor{curcolor}{0 0 0}
\pscustom[linestyle=none,fillstyle=solid,fillcolor=curcolor]
{
\newpath
\moveto(311.2851581,257.10157699)
\lineto(311.2851581,256.04688949)
\lineto(301.3710956,256.04688949)
\curveto(301.46484193,254.56250993)(301.91015398,253.42969856)(302.7070331,252.64845199)
\curveto(303.51171488,251.87501262)(304.62890127,251.48829426)(306.0585956,251.48829574)
\curveto(306.88671151,251.48829426)(307.68749196,251.58985665)(308.46093935,251.79298324)
\curveto(309.2421779,251.99610625)(310.01561463,252.30079344)(310.78125185,252.70704574)
\lineto(310.78125185,250.66798324)
\curveto(310.00780214,250.3398579)(309.21483418,250.08985815)(308.4023456,249.91798324)
\curveto(307.5898358,249.7461085)(306.76561788,249.66017108)(305.92968935,249.66017074)
\curveto(303.83593331,249.66017108)(302.17577872,250.26954547)(300.9492206,251.48829574)
\curveto(299.73046866,252.70704304)(299.12109427,254.35547889)(299.1210956,256.43360824)
\curveto(299.12109427,258.58203716)(299.6992187,260.28516046)(300.8554706,261.54298324)
\curveto(302.01952887,262.80859544)(303.58593356,263.4414073)(305.55468935,263.44142074)
\curveto(307.32030482,263.4414073)(308.71483468,262.87109537)(309.7382831,261.73048324)
\curveto(310.76952012,260.59766015)(311.28514461,259.05469294)(311.2851581,257.10157699)
\moveto(309.1289081,257.73438949)
\curveto(309.11327178,258.91406808)(308.78124086,259.85547339)(308.13281435,260.55860824)
\curveto(307.49217965,261.26172198)(306.640618,261.61328413)(305.57812685,261.61329574)
\curveto(304.37499527,261.61328413)(303.41015248,261.27344072)(302.6835956,260.59376449)
\curveto(301.96484143,259.91406708)(301.55077934,258.95703679)(301.4414081,257.72267074)
\lineto(309.1289081,257.73438949)
}
}
{
\newrgbcolor{curcolor}{0 0 0}
\pscustom[linestyle=none,fillstyle=solid,fillcolor=curcolor]
{
\newpath
\moveto(322.42968935,261.10938949)
\curveto(322.18749222,261.25000324)(321.92186749,261.35156564)(321.63281435,261.41407699)
\curveto(321.35155556,261.48437801)(321.03905587,261.51953422)(320.69531435,261.51954574)
\curveto(319.47655743,261.51953422)(318.53905837,261.12109712)(317.88281435,260.32423324)
\curveto(317.23437218,259.53516121)(316.91015375,258.3984436)(316.9101581,256.91407699)
\lineto(316.9101581,250.00001449)
\lineto(314.74218935,250.00001449)
\lineto(314.74218935,263.12501449)
\lineto(316.9101581,263.12501449)
\lineto(316.9101581,261.08595199)
\curveto(317.3632783,261.88281511)(317.95312146,262.47265827)(318.67968935,262.85548324)
\curveto(319.406245,263.246095)(320.28905662,263.4414073)(321.32812685,263.44142074)
\curveto(321.47655543,263.4414073)(321.64061777,263.42968856)(321.82031435,263.40626449)
\curveto(321.99999241,263.3906261)(322.19921096,263.36328238)(322.4179706,263.32423324)
\lineto(322.42968935,261.10938949)
}
}
{
\newrgbcolor{curcolor}{0 0 0}
\pscustom[linestyle=none,fillstyle=solid,fillcolor=curcolor]
{
\newpath
\moveto(506.14062685,322.65626449)
\lineto(517.15625185,322.65626449)
\lineto(517.15625185,320.00001449)
\lineto(502.34375185,320.00001449)
\lineto(502.34375185,322.65626449)
\curveto(503.54166497,323.89584393)(505.17187168,325.5573006)(507.23437685,327.64063949)
\curveto(509.30728421,329.73437976)(510.60936624,331.08333674)(511.14062685,331.68751449)
\curveto(512.15103136,332.82291834)(512.85415566,333.78125071)(513.25000185,334.56251449)
\curveto(513.65623819,335.35416581)(513.85936299,336.1302067)(513.85937685,336.89063949)
\curveto(513.85936299,338.1302047)(513.42186343,339.14062035)(512.54687685,339.92188949)
\curveto(511.68228183,340.70311879)(510.55207463,341.0937434)(509.15625185,341.09376449)
\curveto(508.16666035,341.0937434)(507.11978639,340.92186857)(506.01562685,340.57813949)
\curveto(504.92187193,340.23436926)(503.7499981,339.71353645)(502.50000185,339.01563949)
\lineto(502.50000185,342.20313949)
\curveto(503.77083141,342.71353345)(504.95833022,343.09894973)(506.06250185,343.35938949)
\curveto(507.16666135,343.61978254)(508.177077,343.74999074)(509.09375185,343.75001449)
\curveto(511.510407,343.74999074)(513.43748841,343.14582468)(514.87500185,341.93751449)
\curveto(516.31248554,340.72916043)(517.03123482,339.11457871)(517.03125185,337.09376449)
\curveto(517.03123482,336.13541503)(516.84894333,335.2239576)(516.48437685,334.35938949)
\curveto(516.13019405,333.50520932)(515.47915304,332.49479367)(514.53125185,331.32813949)
\curveto(514.27082091,331.02604513)(513.44269674,330.15104601)(512.04687685,328.70313949)
\curveto(510.65103286,327.26563223)(508.68228483,325.25000924)(506.14062685,322.65626449)
}
}
{
\newrgbcolor{curcolor}{0 0 0}
\pscustom[linewidth=2,linecolor=curcolor,linestyle=dashed,dash=8 8]
{
\newpath
\moveto(150,530)
\lineto(60,530)
}
}
{
\newrgbcolor{curcolor}{0 0 0}
\pscustom[linestyle=none,fillstyle=solid,fillcolor=curcolor]
{
\newpath
\moveto(139.53769464,534.84048224)
\lineto(152.6487474,530.01921591)
\lineto(139.53769392,525.19795064)
\curveto(141.632292,528.04442372)(141.62022288,531.93889292)(139.53769464,534.84048224)
\lineto(139.53769464,534.84048224)
\closepath
}
}
{
\newrgbcolor{curcolor}{0 0 0}
\pscustom[linewidth=2,linecolor=curcolor,linestyle=dashed,dash=8 8]
{
\newpath
\moveto(370,330)
\lineto(500,330)
}
}
{
\newrgbcolor{curcolor}{0 0 0}
\pscustom[linestyle=none,fillstyle=solid,fillcolor=curcolor]
{
\newpath
\moveto(380.46230536,325.15951776)
\lineto(367.3512526,329.98078409)
\lineto(380.46230608,334.80204936)
\curveto(378.367708,331.95557628)(378.37977712,328.06110708)(380.46230536,325.15951776)
\lineto(380.46230536,325.15951776)
\closepath
}
}
{
\newrgbcolor{curcolor}{0 0 0}
\pscustom[linewidth=2,linecolor=curcolor,linestyle=dashed,dash=8 8]
{
\newpath
\moveto(230,260)
\lineto(60,260)
}
}
{
\newrgbcolor{curcolor}{0 0 0}
\pscustom[linestyle=none,fillstyle=solid,fillcolor=curcolor]
{
\newpath
\moveto(219.53769464,264.84048224)
\lineto(232.6487474,260.01921591)
\lineto(219.53769392,255.19795064)
\curveto(221.632292,258.04442372)(221.62022288,261.93889292)(219.53769464,264.84048224)
\lineto(219.53769464,264.84048224)
\closepath
}
}
{
\newrgbcolor{curcolor}{0 0 0}
\pscustom[linestyle=none,fillstyle=solid,fillcolor=curcolor]
{
\newpath
\moveto(43.96875185,522.65624924)
\lineto(49.12500185,522.65624924)
\lineto(49.12500185,540.45312424)
\lineto(43.51562685,539.32812424)
\lineto(43.51562685,542.20312424)
\lineto(49.09375185,543.32812424)
\lineto(52.25000185,543.32812424)
\lineto(52.25000185,522.65624924)
\lineto(57.40625185,522.65624924)
\lineto(57.40625185,519.99999924)
\lineto(43.96875185,519.99999924)
\lineto(43.96875185,522.65624924)
}
}
{
\newrgbcolor{curcolor}{0 0 0}
\pscustom[linestyle=none,fillstyle=solid,fillcolor=curcolor]
{
\newpath
\moveto(512.98437685,292.57813949)
\curveto(514.49477902,292.25521057)(515.67186118,291.58333624)(516.51562685,290.56251449)
\curveto(517.36977614,289.54167162)(517.79685905,288.28125621)(517.79687685,286.78126449)
\curveto(517.79685905,284.47917668)(517.00519318,282.69792846)(515.42187685,281.43751449)
\curveto(513.83852968,280.17709765)(511.58853193,279.54688995)(508.67187685,279.54688949)
\curveto(507.69270249,279.54688995)(506.68228683,279.64584818)(505.64062685,279.84376449)
\curveto(504.60937224,280.03126446)(503.54166497,280.31772251)(502.43750185,280.70313949)
\lineto(502.43750185,283.75001449)
\curveto(503.31249854,283.23959459)(504.27083091,282.85417831)(505.31250185,282.59376449)
\curveto(506.35416216,282.33334549)(507.44270274,282.20313729)(508.57812685,282.20313949)
\curveto(510.55728296,282.20313729)(512.06248979,282.5937619)(513.09375185,283.37501449)
\curveto(514.13540438,284.15626034)(514.65623719,285.29167587)(514.65625185,286.78126449)
\curveto(514.65623719,288.15625634)(514.17186268,289.22917193)(513.20312685,290.00001449)
\curveto(512.24478127,290.78125371)(510.90624094,291.17187832)(509.18750185,291.17188949)
\lineto(506.46875185,291.17188949)
\lineto(506.46875185,293.76563949)
\lineto(509.31250185,293.76563949)
\curveto(510.86457432,293.76562573)(512.05207313,294.07291709)(512.87500185,294.68751449)
\curveto(513.69790482,295.31249918)(514.10936274,296.20833162)(514.10937685,297.37501449)
\curveto(514.10936274,298.57291259)(513.68227983,299.48957834)(512.82812685,300.12501449)
\curveto(511.98436486,300.77082706)(510.77082441,301.0937434)(509.18750185,301.09376449)
\curveto(508.32291019,301.0937434)(507.39582779,300.99999349)(506.40625185,300.81251449)
\curveto(505.4166631,300.62499387)(504.32812252,300.33332749)(503.14062685,299.93751449)
\lineto(503.14062685,302.75001449)
\curveto(504.33853918,303.08332474)(505.45832972,303.33332449)(506.50000185,303.50001449)
\curveto(507.55207763,303.66665749)(508.54165997,303.74999074)(509.46875185,303.75001449)
\curveto(511.86457332,303.74999074)(513.76040475,303.20311629)(515.15625185,302.10938949)
\curveto(516.55206863,301.02603513)(517.2499846,299.5572866)(517.25000185,297.70313949)
\curveto(517.2499846,296.41145642)(516.8801933,295.31770751)(516.14062685,294.42188949)
\curveto(515.40102811,293.53645929)(514.34894583,292.92187657)(512.98437685,292.57813949)
}
}
{
\newrgbcolor{curcolor}{1 1 1}
\pscustom[linestyle=none,fillstyle=solid,fillcolor=curcolor]
{
\newpath
\moveto(160.00001634,310.00001449)
\lineto(399.99998735,310.00001449)
\lineto(400.00000185,310)
\lineto(400.00000185,280.00001945)
\lineto(399.99998735,280.00000496)
\lineto(160.00001634,280.00000496)
\lineto(160.00000185,280.00001945)
\lineto(160.00000185,310)
\lineto(160.00001634,310.00001449)
\closepath
}
}
{
\newrgbcolor{curcolor}{1 0 0}
\pscustom[linewidth=2,linecolor=curcolor,linestyle=dashed,dash=8 8]
{
\newpath
\moveto(160.00001634,310.00001449)
\lineto(399.99998735,310.00001449)
\lineto(400.00000185,310)
\lineto(400.00000185,280.00001945)
\lineto(399.99998735,280.00000496)
\lineto(160.00001634,280.00000496)
\lineto(160.00000185,280.00001945)
\lineto(160.00000185,310)
\lineto(160.00001634,310.00001449)
\closepath
}
}
{
\newrgbcolor{curcolor}{0 0 0}
\pscustom[linewidth=2,linecolor=curcolor,linestyle=dashed,dash=8 8]
{
\newpath
\moveto(380,290)
\lineto(500,290)
}
}
{
\newrgbcolor{curcolor}{0 0 0}
\pscustom[linestyle=none,fillstyle=solid,fillcolor=curcolor]
{
\newpath
\moveto(390.46230536,285.15951776)
\lineto(377.3512526,289.98078409)
\lineto(390.46230608,294.80204936)
\curveto(388.367708,291.95557628)(388.37977712,288.06110708)(390.46230536,285.15951776)
\lineto(390.46230536,285.15951776)
\closepath
}
}
{
\newrgbcolor{curcolor}{0 0 0}
\pscustom[linestyle=none,fillstyle=solid,fillcolor=curcolor]
{
\newpath
\moveto(52.09375185,270.57813949)
\lineto(44.12500185,258.12501449)
\lineto(52.09375185,258.12501449)
\lineto(52.09375185,270.57813949)
\moveto(51.26562685,273.32813949)
\lineto(55.23437685,273.32813949)
\lineto(55.23437685,258.12501449)
\lineto(58.56250185,258.12501449)
\lineto(58.56250185,255.50001449)
\lineto(55.23437685,255.50001449)
\lineto(55.23437685,250.00001449)
\lineto(52.09375185,250.00001449)
\lineto(52.09375185,255.50001449)
\lineto(41.56250185,255.50001449)
\lineto(41.56250185,258.54688949)
\lineto(51.26562685,273.32813949)
}
}
\end{pspicture}

		\end{center}

		\begin{enumerate}
		  \item Fond d'écran (présent dans l'archive stocké sur le téléphone).
		  \item Zone de saisie.
		  \item Label d'erreur.
		  \item Bouton ``Valider".
		\end{enumerate}

		\subsubsection{Description des zones}
				
			\begin{tabular}{|c|c|c|c|c|} \hline
				Numéro de zone & Type  & Description & Evènement &	Règle \\\hline 
				2 & Zone de saisie & Définition du pseudonyme & Perte du focus & RG1-01 \\\hline
				4 & Bouton         & Bouton de validation     & Cliqué & RG1-02 \\\hline
			\end{tabular}

		\subsubsection{Description des règles}
		
		\underline{RG1-01 :}
			\begin{quote}
				Vérification de la zone de saisie :
				\begin{itemize}
				  \item Non vide.
				  \item Pseudonyme entré inexistant.
				\end{itemize}
				Si la vérification est correcte un nouveau compte hors ligne ayant pour
				pseudonyme celui entré dans la zone de saisie est créé dans la base de données
				du téléphone.\\
				Sinon une erreur est renvoyée et affichée via le label d'erreur (3) RG.
			\end{quote}
			
		\underline{RG1-02 :}
			\begin{quote}
				RG1-01.\\
				Afficher la page d'accueil%
					\footnote[1]{
						\hyperlink{Page d'accueil}{Page d'accueil}
						\og voir section \ref{accueil}, page \pageref{accueil}.\fg
					}.\\
				Cacher la page de création du profil.
			\end{quote}
	
\newpage

	\subsection{Page d'accueil}
		\hypertarget{Page d'accueil}{}
		\label{accueil}

		\begin{center}	
			%LaTeX with PSTricks extensions
%%Creator: inkscape 0.48.0
%%Please note this file requires PSTricks extensions
\psset{xunit=.4pt,yunit=.4pt,runit=.4pt}
\begin{pspicture}(560,600)
{
\newrgbcolor{curcolor}{1 1 1}
\pscustom[linestyle=none,fillstyle=solid,fillcolor=curcolor]
{
\newpath
\moveto(133.12418633,597.52237246)
\lineto(426.87615472,597.52237246)
\curveto(443.85414222,597.52237246)(457.52234155,583.85417314)(457.52234155,566.87618563)
\lineto(457.52234155,33.12418673)
\curveto(457.52234155,16.14619923)(443.85414222,2.4779999)(426.87615472,2.4779999)
\lineto(133.12418633,2.4779999)
\curveto(116.14619883,2.4779999)(102.4779995,16.14619923)(102.4779995,33.12418673)
\lineto(102.4779995,566.87618563)
\curveto(102.4779995,583.85417314)(116.14619883,597.52237246)(133.12418633,597.52237246)
\closepath
}
}
{
\newrgbcolor{curcolor}{0 0 0}
\pscustom[linewidth=4.95599985,linecolor=curcolor]
{
\newpath
\moveto(133.12418633,597.52237246)
\lineto(426.87615472,597.52237246)
\curveto(443.85414222,597.52237246)(457.52234155,583.85417314)(457.52234155,566.87618563)
\lineto(457.52234155,33.12418673)
\curveto(457.52234155,16.14619923)(443.85414222,2.4779999)(426.87615472,2.4779999)
\lineto(133.12418633,2.4779999)
\curveto(116.14619883,2.4779999)(102.4779995,16.14619923)(102.4779995,33.12418673)
\lineto(102.4779995,566.87618563)
\curveto(102.4779995,583.85417314)(116.14619883,597.52237246)(133.12418633,597.52237246)
\closepath
}
}
{
\newrgbcolor{curcolor}{1 1 1}
\pscustom[linestyle=none,fillstyle=solid,fillcolor=curcolor]
{
\newpath
\moveto(201.70414239,399.9999964)
\lineto(358.44696313,399.9999964)
\curveto(364.84734492,399.9999964)(369.99999886,394.84734246)(369.99999886,388.44696067)
\lineto(369.99999886,378.53398154)
\curveto(369.99999886,372.13359974)(364.84734492,366.9809458)(358.44696313,366.9809458)
\lineto(201.70414239,366.9809458)
\curveto(195.30376059,366.9809458)(190.15110665,372.13359974)(190.15110665,378.53398154)
\lineto(190.15110665,388.44696067)
\curveto(190.15110665,394.84734246)(195.30376059,399.9999964)(201.70414239,399.9999964)
\closepath
}
}
{
\newrgbcolor{curcolor}{0 0 0}
\pscustom[linewidth=2,linecolor=curcolor]
{
\newpath
\moveto(201.70414239,399.9999964)
\lineto(358.44696313,399.9999964)
\curveto(364.84734492,399.9999964)(369.99999886,394.84734246)(369.99999886,388.44696067)
\lineto(369.99999886,378.53398154)
\curveto(369.99999886,372.13359974)(364.84734492,366.9809458)(358.44696313,366.9809458)
\lineto(201.70414239,366.9809458)
\curveto(195.30376059,366.9809458)(190.15110665,372.13359974)(190.15110665,378.53398154)
\lineto(190.15110665,388.44696067)
\curveto(190.15110665,394.84734246)(195.30376059,399.9999964)(201.70414239,399.9999964)
\closepath
}
}
{
\newrgbcolor{curcolor}{0 0 0}
\pscustom[linestyle=none,fillstyle=solid,fillcolor=curcolor]
{
\newpath
\moveto(212.58688241,394.55735419)
\lineto(214.95406991,394.55735419)
\lineto(214.95406991,378.28001044)
\curveto(214.95406519,376.17063633)(214.55172184,374.63938786)(213.74703866,373.68626044)
\curveto(212.95016094,372.73313977)(211.66500597,372.25657775)(209.89156991,372.25657294)
\lineto(208.98922616,372.25657294)
\lineto(208.98922616,374.24876044)
\lineto(209.72750741,374.24876044)
\curveto(210.77438187,374.24876325)(211.51266238,374.54173171)(211.94235116,375.12766669)
\curveto(212.37203652,375.71360554)(212.58688005,376.76438574)(212.58688241,378.28001044)
\lineto(212.58688241,394.55735419)
}
}
{
\newrgbcolor{curcolor}{0 0 0}
\pscustom[linestyle=none,fillstyle=solid,fillcolor=curcolor]
{
\newpath
\moveto(224.65719491,388.67454169)
\curveto(223.50093872,388.67453008)(222.58687713,388.22140553)(221.91500741,387.31516669)
\curveto(221.24312847,386.41671983)(220.90719131,385.18234607)(220.90719491,383.61204169)
\curveto(220.90719131,382.04172421)(221.23922223,380.8034442)(221.90328866,379.89719794)
\curveto(222.57515839,378.9987585)(223.49312622,378.5495402)(224.65719491,378.54954169)
\curveto(225.80562391,378.5495402)(226.71577925,379.00266475)(227.38766366,379.90891669)
\curveto(228.05952791,380.81516294)(228.39546507,382.0495367)(228.39547616,383.61204169)
\curveto(228.39546507,385.16672108)(228.05952791,386.3971886)(227.38766366,387.30344794)
\curveto(226.71577925,388.21749928)(225.80562391,388.67453008)(224.65719491,388.67454169)
\moveto(224.65719491,390.50266669)
\curveto(226.53218569,390.50265325)(228.00484046,389.89327886)(229.07516366,388.67454169)
\curveto(230.14546332,387.4557813)(230.68061904,385.76828298)(230.68063241,383.61204169)
\curveto(230.68061904,381.46359979)(230.14546332,379.77610148)(229.07516366,378.54954169)
\curveto(228.00484046,377.33079142)(226.53218569,376.72141703)(224.65719491,376.72141669)
\curveto(222.77437694,376.72141703)(221.29781592,377.33079142)(220.22750741,378.54954169)
\curveto(219.16500555,379.77610148)(218.63375608,381.46359979)(218.63375741,383.61204169)
\curveto(218.63375608,385.76828298)(219.16500555,387.4557813)(220.22750741,388.67454169)
\curveto(221.29781592,389.89327886)(222.77437694,390.50265325)(224.65719491,390.50266669)
}
}
{
\newrgbcolor{curcolor}{0 0 0}
\pscustom[linestyle=none,fillstyle=solid,fillcolor=curcolor]
{
\newpath
\moveto(234.02047616,382.24094794)
\lineto(234.02047616,390.18626044)
\lineto(236.17672616,390.18626044)
\lineto(236.17672616,382.32297919)
\curveto(236.17672196,381.08078767)(236.41890922,380.14719485)(236.90328866,379.52219794)
\curveto(237.38765825,378.9050086)(238.11422003,378.59641516)(239.08297616,378.59641669)
\curveto(240.24703039,378.59641516)(241.16499822,378.96750853)(241.83688241,379.70969794)
\curveto(242.51655937,380.45188205)(242.85640278,381.46359979)(242.85641366,382.74485419)
\lineto(242.85641366,390.18626044)
\lineto(245.01266366,390.18626044)
\lineto(245.01266366,377.06126044)
\lineto(242.85641366,377.06126044)
\lineto(242.85641366,379.07688544)
\curveto(242.33296581,378.28000922)(241.72359142,377.68625982)(241.02828866,377.29563544)
\curveto(240.3407803,376.91282309)(239.53999985,376.72141703)(238.62594491,376.72141669)
\curveto(237.11812727,376.72141703)(235.97359717,377.19016656)(235.19235116,378.12766669)
\curveto(234.41109873,379.06516469)(234.02047412,380.43625707)(234.02047616,382.24094794)
\moveto(239.44625741,390.50266669)
\lineto(239.44625741,390.50266669)
}
}
{
\newrgbcolor{curcolor}{0 0 0}
\pscustom[linestyle=none,fillstyle=solid,fillcolor=curcolor]
{
\newpath
\moveto(260.70406991,384.16282294)
\lineto(260.70406991,383.10813544)
\lineto(250.79000741,383.10813544)
\curveto(250.88375374,381.62375588)(251.32906579,380.49094451)(252.12594491,379.70969794)
\curveto(252.93062669,378.93625857)(254.04781308,378.5495402)(255.47750741,378.54954169)
\curveto(256.30562332,378.5495402)(257.10640377,378.6511026)(257.87985116,378.85422919)
\curveto(258.66108971,379.05735219)(259.43452644,379.36203939)(260.20016366,379.76829169)
\lineto(260.20016366,377.72922919)
\curveto(259.42671395,377.40110385)(258.63374599,377.1511041)(257.82125741,376.97922919)
\curveto(257.00874762,376.80735444)(256.18452969,376.72141703)(255.34860116,376.72141669)
\curveto(253.25484512,376.72141703)(251.59469053,377.33079142)(250.36813241,378.54954169)
\curveto(249.14938047,379.76828898)(248.54000608,381.41672483)(248.54000741,383.49485419)
\curveto(248.54000608,385.64328311)(249.11813051,387.34640641)(250.27438241,388.60422919)
\curveto(251.43844069,389.86984138)(253.00484537,390.50265325)(254.97360116,390.50266669)
\curveto(256.73921663,390.50265325)(258.13374649,389.93234132)(259.15719491,388.79172919)
\curveto(260.18843194,387.65890609)(260.70405642,386.11593889)(260.70406991,384.16282294)
\moveto(258.54781991,384.79563544)
\curveto(258.53218359,385.97531403)(258.20015267,386.91671933)(257.55172616,387.61985419)
\curveto(256.91109146,388.32296793)(256.05952981,388.67453008)(254.99703866,388.67454169)
\curveto(253.79390708,388.67453008)(252.82906429,388.33468667)(252.10250741,387.65501044)
\curveto(251.38375324,386.97531303)(250.96969115,386.01828273)(250.86031991,384.78391669)
\lineto(258.54781991,384.79563544)
}
}
{
\newrgbcolor{curcolor}{0 0 0}
\pscustom[linestyle=none,fillstyle=solid,fillcolor=curcolor]
{
\newpath
\moveto(271.84860116,388.17063544)
\curveto(271.60640403,388.31124919)(271.3407793,388.41281159)(271.05172616,388.47532294)
\curveto(270.77046737,388.54562396)(270.45796768,388.58078017)(270.11422616,388.58079169)
\curveto(268.89546924,388.58078017)(267.95797018,388.18234307)(267.30172616,387.38547919)
\curveto(266.65328399,386.59640716)(266.32906556,385.45968954)(266.32906991,383.97532294)
\lineto(266.32906991,377.06126044)
\lineto(264.16110116,377.06126044)
\lineto(264.16110116,390.18626044)
\lineto(266.32906991,390.18626044)
\lineto(266.32906991,388.14719794)
\curveto(266.78219011,388.94406106)(267.37203327,389.53390422)(268.09860116,389.91672919)
\curveto(268.82515681,390.30734094)(269.70796843,390.50265325)(270.74703866,390.50266669)
\curveto(270.89546724,390.50265325)(271.05952958,390.49093451)(271.23922616,390.46751044)
\curveto(271.41890422,390.45187205)(271.61812277,390.42452833)(271.83688241,390.38547919)
\lineto(271.84860116,388.17063544)
}
}
{
\newrgbcolor{curcolor}{0 0 0}
\pscustom[linestyle=none,fillstyle=solid,fillcolor=curcolor]
{
}
}
{
\newrgbcolor{curcolor}{0 0 0}
\pscustom[linestyle=none,fillstyle=solid,fillcolor=curcolor]
{
\newpath
\moveto(293.00094491,384.16282294)
\lineto(293.00094491,383.10813544)
\lineto(283.08688241,383.10813544)
\curveto(283.18062874,381.62375588)(283.62594079,380.49094451)(284.42281991,379.70969794)
\curveto(285.22750169,378.93625857)(286.34468808,378.5495402)(287.77438241,378.54954169)
\curveto(288.60249832,378.5495402)(289.40327877,378.6511026)(290.17672616,378.85422919)
\curveto(290.95796471,379.05735219)(291.73140144,379.36203939)(292.49703866,379.76829169)
\lineto(292.49703866,377.72922919)
\curveto(291.72358895,377.40110385)(290.93062099,377.1511041)(290.11813241,376.97922919)
\curveto(289.30562262,376.80735444)(288.48140469,376.72141703)(287.64547616,376.72141669)
\curveto(285.55172012,376.72141703)(283.89156553,377.33079142)(282.66500741,378.54954169)
\curveto(281.44625547,379.76828898)(280.83688108,381.41672483)(280.83688241,383.49485419)
\curveto(280.83688108,385.64328311)(281.41500551,387.34640641)(282.57125741,388.60422919)
\curveto(283.73531569,389.86984138)(285.30172037,390.50265325)(287.27047616,390.50266669)
\curveto(289.03609163,390.50265325)(290.43062149,389.93234132)(291.45406991,388.79172919)
\curveto(292.48530694,387.65890609)(293.00093142,386.11593889)(293.00094491,384.16282294)
\moveto(290.84469491,384.79563544)
\curveto(290.82905859,385.97531403)(290.49702767,386.91671933)(289.84860116,387.61985419)
\curveto(289.20796646,388.32296793)(288.35640481,388.67453008)(287.29391366,388.67454169)
\curveto(286.09078208,388.67453008)(285.12593929,388.33468667)(284.39938241,387.65501044)
\curveto(283.68062824,386.97531303)(283.26656615,386.01828273)(283.15719491,384.78391669)
\lineto(290.84469491,384.79563544)
}
}
{
\newrgbcolor{curcolor}{0 0 0}
\pscustom[linestyle=none,fillstyle=solid,fillcolor=curcolor]
{
\newpath
\moveto(307.45016366,384.98313544)
\lineto(307.45016366,377.06126044)
\lineto(305.29391366,377.06126044)
\lineto(305.29391366,384.91282294)
\curveto(305.29390264,386.15500135)(305.05171538,387.08468792)(304.56735116,387.70188544)
\curveto(304.08296635,388.31906168)(303.35640458,388.62765512)(302.38766366,388.62766669)
\curveto(301.22359421,388.62765512)(300.30562638,388.25656175)(299.63375741,387.51438544)
\curveto(298.96187772,386.77218823)(298.62594056,385.76047049)(298.62594491,384.47922919)
\lineto(298.62594491,377.06126044)
\lineto(296.45797616,377.06126044)
\lineto(296.45797616,390.18626044)
\lineto(298.62594491,390.18626044)
\lineto(298.62594491,388.14719794)
\curveto(299.14156504,388.93624857)(299.74703319,389.52609173)(300.44235116,389.91672919)
\curveto(301.14546929,390.30734094)(301.95406223,390.50265325)(302.86813241,390.50266669)
\curveto(304.37593481,390.50265325)(305.51655867,390.03390372)(306.29000741,389.09641669)
\curveto(307.06343212,388.16671808)(307.45015049,386.79562571)(307.45016366,384.98313544)
}
}
{
\newrgbcolor{curcolor}{0 0 0}
\pscustom[linestyle=none,fillstyle=solid,fillcolor=curcolor]
{
}
}
{
\newrgbcolor{curcolor}{0 0 0}
\pscustom[linestyle=none,fillstyle=solid,fillcolor=curcolor]
{
\newpath
\moveto(327.78219491,389.79954169)
\lineto(327.78219491,387.76047919)
\curveto(327.17280989,388.07296818)(326.53999802,388.30734294)(325.88375741,388.46360419)
\curveto(325.22749933,388.61984263)(324.54781251,388.69796755)(323.84469491,388.69797919)
\curveto(322.77437679,388.69796755)(321.96969009,388.53390522)(321.43063241,388.20579169)
\curveto(320.89937866,387.87765587)(320.63375393,387.38546887)(320.63375741,386.72922919)
\curveto(320.63375393,386.22922002)(320.82515999,385.83468917)(321.20797616,385.54563544)
\curveto(321.59078422,385.26437724)(322.3603147,384.99484626)(323.51656991,384.73704169)
\lineto(324.25485116,384.57297919)
\curveto(325.78609253,384.24484701)(326.87202894,383.78000372)(327.51266366,383.17844794)
\curveto(328.16109015,382.58469242)(328.48530858,381.752662)(328.48531991,380.68235419)
\curveto(328.48530858,379.46360179)(328.00093406,378.498759)(327.03219491,377.78782294)
\curveto(326.07124849,377.07688542)(324.74703106,376.72141703)(323.05953866,376.72141669)
\curveto(322.35640846,376.72141703)(321.62203419,376.79172946)(320.85641366,376.93235419)
\curveto(320.09859821,377.06516669)(319.29781776,377.26829148)(318.45406991,377.54172919)
\lineto(318.45406991,379.76829169)
\curveto(319.25094281,379.3542269)(320.03609828,379.04172721)(320.80953866,378.83079169)
\curveto(321.58297173,378.62766512)(322.34859596,378.52610273)(323.10641366,378.52610419)
\curveto(324.12203169,378.52610273)(324.90328091,378.69797755)(325.45016366,379.04172919)
\curveto(325.99702981,379.39328936)(326.27046704,379.88547637)(326.27047616,380.51829169)
\curveto(326.27046704,381.10422515)(326.07124849,381.55344345)(325.67281991,381.86594794)
\curveto(325.28218678,382.17844282)(324.41890639,382.47922377)(323.08297616,382.76829169)
\lineto(322.33297616,382.94407294)
\curveto(320.99703481,383.22531678)(320.03219203,383.65500385)(319.43844491,384.23313544)
\curveto(318.84469322,384.81906518)(318.54781851,385.61984563)(318.54781991,386.63547919)
\curveto(318.54781851,387.86984338)(318.98531808,388.82296743)(319.86031991,389.49485419)
\curveto(320.73531633,390.16671608)(321.97750258,390.50265325)(323.58688241,390.50266669)
\curveto(324.38375018,390.50265325)(325.13374943,390.44405956)(325.83688241,390.32688544)
\curveto(326.53999802,390.20968479)(327.18843487,390.03390372)(327.78219491,389.79954169)
}
}
{
\newrgbcolor{curcolor}{0 0 0}
\pscustom[linestyle=none,fillstyle=solid,fillcolor=curcolor]
{
\newpath
\moveto(337.01656991,388.67454169)
\curveto(335.86031372,388.67453008)(334.94625213,388.22140553)(334.27438241,387.31516669)
\curveto(333.60250347,386.41671983)(333.26656631,385.18234607)(333.26656991,383.61204169)
\curveto(333.26656631,382.04172421)(333.59859723,380.8034442)(334.26266366,379.89719794)
\curveto(334.93453339,378.9987585)(335.85250122,378.5495402)(337.01656991,378.54954169)
\curveto(338.16499891,378.5495402)(339.07515425,379.00266475)(339.74703866,379.90891669)
\curveto(340.41890291,380.81516294)(340.75484007,382.0495367)(340.75485116,383.61204169)
\curveto(340.75484007,385.16672108)(340.41890291,386.3971886)(339.74703866,387.30344794)
\curveto(339.07515425,388.21749928)(338.16499891,388.67453008)(337.01656991,388.67454169)
\moveto(337.01656991,390.50266669)
\curveto(338.89156069,390.50265325)(340.36421546,389.89327886)(341.43453866,388.67454169)
\curveto(342.50483832,387.4557813)(343.03999404,385.76828298)(343.04000741,383.61204169)
\curveto(343.03999404,381.46359979)(342.50483832,379.77610148)(341.43453866,378.54954169)
\curveto(340.36421546,377.33079142)(338.89156069,376.72141703)(337.01656991,376.72141669)
\curveto(335.13375194,376.72141703)(333.65719092,377.33079142)(332.58688241,378.54954169)
\curveto(331.52438055,379.77610148)(330.99313108,381.46359979)(330.99313241,383.61204169)
\curveto(330.99313108,385.76828298)(331.52438055,387.4557813)(332.58688241,388.67454169)
\curveto(333.65719092,389.89327886)(335.13375194,390.50265325)(337.01656991,390.50266669)
}
}
{
\newrgbcolor{curcolor}{0 0 0}
\pscustom[linestyle=none,fillstyle=solid,fillcolor=curcolor]
{
\newpath
\moveto(346.60250741,395.29563544)
\lineto(348.75875741,395.29563544)
\lineto(348.75875741,377.06126044)
\lineto(346.60250741,377.06126044)
\lineto(346.60250741,395.29563544)
}
}
{
\newrgbcolor{curcolor}{0 0 0}
\pscustom[linestyle=none,fillstyle=solid,fillcolor=curcolor]
{
\newpath
\moveto(358.34469491,388.67454169)
\curveto(357.18843872,388.67453008)(356.27437713,388.22140553)(355.60250741,387.31516669)
\curveto(354.93062847,386.41671983)(354.59469131,385.18234607)(354.59469491,383.61204169)
\curveto(354.59469131,382.04172421)(354.92672223,380.8034442)(355.59078866,379.89719794)
\curveto(356.26265839,378.9987585)(357.18062622,378.5495402)(358.34469491,378.54954169)
\curveto(359.49312391,378.5495402)(360.40327925,379.00266475)(361.07516366,379.90891669)
\curveto(361.74702791,380.81516294)(362.08296507,382.0495367)(362.08297616,383.61204169)
\curveto(362.08296507,385.16672108)(361.74702791,386.3971886)(361.07516366,387.30344794)
\curveto(360.40327925,388.21749928)(359.49312391,388.67453008)(358.34469491,388.67454169)
\moveto(358.34469491,390.50266669)
\curveto(360.21968569,390.50265325)(361.69234046,389.89327886)(362.76266366,388.67454169)
\curveto(363.83296332,387.4557813)(364.36811904,385.76828298)(364.36813241,383.61204169)
\curveto(364.36811904,381.46359979)(363.83296332,379.77610148)(362.76266366,378.54954169)
\curveto(361.69234046,377.33079142)(360.21968569,376.72141703)(358.34469491,376.72141669)
\curveto(356.46187694,376.72141703)(354.98531592,377.33079142)(353.91500741,378.54954169)
\curveto(352.85250555,379.77610148)(352.32125608,381.46359979)(352.32125741,383.61204169)
\curveto(352.32125608,385.76828298)(352.85250555,387.4557813)(353.91500741,388.67454169)
\curveto(354.98531592,389.89327886)(356.46187694,390.50265325)(358.34469491,390.50266669)
}
}
{
\newrgbcolor{curcolor}{1 1 1}
\pscustom[linestyle=none,fillstyle=solid,fillcolor=curcolor]
{
\newpath
\moveto(151.8098362,349.9999964)
\lineto(401.00449258,349.9999964)
\curveto(406.79787809,349.9999964)(411.46186715,345.33600734)(411.46186715,339.54262183)
\lineto(411.46186715,330.68223021)
\curveto(411.46186715,324.8888447)(406.79787809,320.22485564)(401.00449258,320.22485564)
\lineto(151.8098362,320.22485564)
\curveto(146.01645069,320.22485564)(141.35246163,324.8888447)(141.35246163,330.68223021)
\lineto(141.35246163,339.54262183)
\curveto(141.35246163,345.33600734)(146.01645069,349.9999964)(151.8098362,349.9999964)
\closepath
}
}
{
\newrgbcolor{curcolor}{0 0 0}
\pscustom[linewidth=2,linecolor=curcolor]
{
\newpath
\moveto(151.8098362,349.9999964)
\lineto(401.00449258,349.9999964)
\curveto(406.79787809,349.9999964)(411.46186715,345.33600734)(411.46186715,339.54262183)
\lineto(411.46186715,330.68223021)
\curveto(411.46186715,324.8888447)(406.79787809,320.22485564)(401.00449258,320.22485564)
\lineto(151.8098362,320.22485564)
\curveto(146.01645069,320.22485564)(141.35246163,324.8888447)(141.35246163,330.68223021)
\lineto(141.35246163,339.54262183)
\curveto(141.35246163,345.33600734)(146.01645069,349.9999964)(151.8098362,349.9999964)
\closepath
}
}
{
\newrgbcolor{curcolor}{0 0 0}
\pscustom[linestyle=none,fillstyle=solid,fillcolor=curcolor]
{
\newpath
\moveto(152.35546761,347.49609015)
\lineto(154.72265511,347.49609015)
\lineto(154.72265511,331.2187464)
\curveto(154.72265039,329.10937229)(154.32030704,327.57812382)(153.51562386,326.6249964)
\curveto(152.71874614,325.67187573)(151.43359118,325.19531371)(149.66015511,325.1953089)
\lineto(148.75781136,325.1953089)
\lineto(148.75781136,327.1874964)
\lineto(149.49609261,327.1874964)
\curveto(150.54296707,327.18749921)(151.28124758,327.48046767)(151.71093636,328.06640265)
\curveto(152.14062172,328.6523415)(152.35546526,329.7031217)(152.35546761,331.2187464)
\lineto(152.35546761,347.49609015)
}
}
{
\newrgbcolor{curcolor}{0 0 0}
\pscustom[linestyle=none,fillstyle=solid,fillcolor=curcolor]
{
\newpath
\moveto(164.42578011,341.61327765)
\curveto(163.26952392,341.61326604)(162.35546234,341.16014149)(161.68359261,340.25390265)
\curveto(161.01171368,339.3554558)(160.67577652,338.12108203)(160.67578011,336.55077765)
\curveto(160.67577652,334.98046017)(161.00780743,333.74218016)(161.67187386,332.8359339)
\curveto(162.3437436,331.93749446)(163.26171143,331.48827616)(164.42578011,331.48827765)
\curveto(165.57420912,331.48827616)(166.48436446,331.94140071)(167.15624886,332.84765265)
\curveto(167.82811311,333.7538989)(168.16405028,334.98827266)(168.16406136,336.55077765)
\curveto(168.16405028,338.10545705)(167.82811311,339.33592457)(167.15624886,340.2421839)
\curveto(166.48436446,341.15623525)(165.57420912,341.61326604)(164.42578011,341.61327765)
\moveto(164.42578011,343.44140265)
\curveto(166.30077089,343.44138921)(167.77342567,342.83201482)(168.84374886,341.61327765)
\curveto(169.91404853,340.39451726)(170.44920424,338.70701895)(170.44921761,336.55077765)
\curveto(170.44920424,334.40233575)(169.91404853,332.71483744)(168.84374886,331.48827765)
\curveto(167.77342567,330.26952738)(166.30077089,329.66015299)(164.42578011,329.66015265)
\curveto(162.54296215,329.66015299)(161.06640112,330.26952738)(159.99609261,331.48827765)
\curveto(158.93359076,332.71483744)(158.40234129,334.40233575)(158.40234261,336.55077765)
\curveto(158.40234129,338.70701895)(158.93359076,340.39451726)(159.99609261,341.61327765)
\curveto(161.06640112,342.83201482)(162.54296215,343.44138921)(164.42578011,343.44140265)
}
}
{
\newrgbcolor{curcolor}{0 0 0}
\pscustom[linestyle=none,fillstyle=solid,fillcolor=curcolor]
{
\newpath
\moveto(173.78906136,335.1796839)
\lineto(173.78906136,343.1249964)
\lineto(175.94531136,343.1249964)
\lineto(175.94531136,335.26171515)
\curveto(175.94530717,334.01952363)(176.18749443,333.08593082)(176.67187386,332.4609339)
\curveto(177.15624346,331.84374456)(177.88280523,331.53515112)(178.85156136,331.53515265)
\curveto(180.0156156,331.53515112)(180.93358343,331.9062445)(181.60546761,332.6484339)
\curveto(182.28514458,333.39061801)(182.62498799,334.40233575)(182.62499886,335.68359015)
\lineto(182.62499886,343.1249964)
\lineto(184.78124886,343.1249964)
\lineto(184.78124886,329.9999964)
\lineto(182.62499886,329.9999964)
\lineto(182.62499886,332.0156214)
\curveto(182.10155101,331.21874518)(181.49217662,330.62499578)(180.79687386,330.2343714)
\curveto(180.1093655,329.85155905)(179.30858505,329.66015299)(178.39453011,329.66015265)
\curveto(176.88671248,329.66015299)(175.74218237,330.12890252)(174.96093636,331.06640265)
\curveto(174.17968393,332.00390065)(173.78905932,333.37499303)(173.78906136,335.1796839)
\moveto(179.21484261,343.44140265)
\lineto(179.21484261,343.44140265)
}
}
{
\newrgbcolor{curcolor}{0 0 0}
\pscustom[linestyle=none,fillstyle=solid,fillcolor=curcolor]
{
\newpath
\moveto(200.47265511,337.1015589)
\lineto(200.47265511,336.0468714)
\lineto(190.55859261,336.0468714)
\curveto(190.65233895,334.56249184)(191.097651,333.42968047)(191.89453011,332.6484339)
\curveto(192.6992119,331.87499453)(193.81639828,331.48827616)(195.24609261,331.48827765)
\curveto(196.07420852,331.48827616)(196.87498897,331.58983856)(197.64843636,331.79296515)
\curveto(198.42967492,331.99608816)(199.20311164,332.30077535)(199.96874886,332.70702765)
\lineto(199.96874886,330.66796515)
\curveto(199.19529915,330.33983981)(198.4023312,330.08984006)(197.58984261,329.91796515)
\curveto(196.77733282,329.74609041)(195.95311489,329.66015299)(195.11718636,329.66015265)
\curveto(193.02343032,329.66015299)(191.36327573,330.26952738)(190.13671761,331.48827765)
\curveto(188.91796568,332.70702495)(188.30859129,334.3554608)(188.30859261,336.43359015)
\curveto(188.30859129,338.58201907)(188.88671571,340.28514237)(190.04296761,341.54296515)
\curveto(191.20702589,342.80857734)(192.77343057,343.44138921)(194.74218636,343.44140265)
\curveto(196.50780184,343.44138921)(197.9023317,342.87107728)(198.92578011,341.73046515)
\curveto(199.95701714,340.59764205)(200.47264162,339.05467485)(200.47265511,337.1015589)
\moveto(198.31640511,337.7343714)
\curveto(198.3007688,338.91404999)(197.96873788,339.8554553)(197.32031136,340.55859015)
\curveto(196.67967667,341.26170389)(195.82811502,341.61326604)(194.76562386,341.61327765)
\curveto(193.56249229,341.61326604)(192.5976495,341.27342263)(191.87109261,340.5937464)
\curveto(191.15233845,339.91404899)(190.73827636,338.9570187)(190.62890511,337.72265265)
\lineto(198.31640511,337.7343714)
}
}
{
\newrgbcolor{curcolor}{0 0 0}
\pscustom[linestyle=none,fillstyle=solid,fillcolor=curcolor]
{
\newpath
\moveto(211.61718636,341.1093714)
\curveto(211.37498924,341.24998515)(211.1093645,341.35154755)(210.82031136,341.4140589)
\curveto(210.53905257,341.48435992)(210.22655289,341.51951613)(209.88281136,341.51952765)
\curveto(208.66405445,341.51951613)(207.72655539,341.12107903)(207.07031136,340.32421515)
\curveto(206.42186919,339.53514312)(206.09765077,338.3984255)(206.09765511,336.9140589)
\lineto(206.09765511,329.9999964)
\lineto(203.92968636,329.9999964)
\lineto(203.92968636,343.1249964)
\lineto(206.09765511,343.1249964)
\lineto(206.09765511,341.0859339)
\curveto(206.55077531,341.88279702)(207.14061847,342.47264018)(207.86718636,342.85546515)
\curveto(208.59374202,343.24607691)(209.47655364,343.44138921)(210.51562386,343.44140265)
\curveto(210.66405245,343.44138921)(210.82811479,343.42967047)(211.00781136,343.4062464)
\curveto(211.18748943,343.39060801)(211.38670798,343.36326429)(211.60546761,343.32421515)
\lineto(211.61718636,341.1093714)
}
}
{
\newrgbcolor{curcolor}{0 0 0}
\pscustom[linestyle=none,fillstyle=solid,fillcolor=curcolor]
{
}
}
{
\newrgbcolor{curcolor}{0 0 0}
\pscustom[linestyle=none,fillstyle=solid,fillcolor=curcolor]
{
\newpath
\moveto(232.76953011,337.1015589)
\lineto(232.76953011,336.0468714)
\lineto(222.85546761,336.0468714)
\curveto(222.94921395,334.56249184)(223.394526,333.42968047)(224.19140511,332.6484339)
\curveto(224.9960869,331.87499453)(226.11327328,331.48827616)(227.54296761,331.48827765)
\curveto(228.37108352,331.48827616)(229.17186397,331.58983856)(229.94531136,331.79296515)
\curveto(230.72654992,331.99608816)(231.49998664,332.30077535)(232.26562386,332.70702765)
\lineto(232.26562386,330.66796515)
\curveto(231.49217415,330.33983981)(230.6992062,330.08984006)(229.88671761,329.91796515)
\curveto(229.07420782,329.74609041)(228.24998989,329.66015299)(227.41406136,329.66015265)
\curveto(225.32030532,329.66015299)(223.66015073,330.26952738)(222.43359261,331.48827765)
\curveto(221.21484068,332.70702495)(220.60546629,334.3554608)(220.60546761,336.43359015)
\curveto(220.60546629,338.58201907)(221.18359071,340.28514237)(222.33984261,341.54296515)
\curveto(223.50390089,342.80857734)(225.07030557,343.44138921)(227.03906136,343.44140265)
\curveto(228.80467684,343.44138921)(230.1992067,342.87107728)(231.22265511,341.73046515)
\curveto(232.25389214,340.59764205)(232.76951662,339.05467485)(232.76953011,337.1015589)
\moveto(230.61328011,337.7343714)
\curveto(230.5976438,338.91404999)(230.26561288,339.8554553)(229.61718636,340.55859015)
\curveto(228.97655167,341.26170389)(228.12499002,341.61326604)(227.06249886,341.61327765)
\curveto(225.85936729,341.61326604)(224.8945245,341.27342263)(224.16796761,340.5937464)
\curveto(223.44921345,339.91404899)(223.03515136,338.9570187)(222.92578011,337.72265265)
\lineto(230.61328011,337.7343714)
}
}
{
\newrgbcolor{curcolor}{0 0 0}
\pscustom[linestyle=none,fillstyle=solid,fillcolor=curcolor]
{
\newpath
\moveto(247.21874886,337.9218714)
\lineto(247.21874886,329.9999964)
\lineto(245.06249886,329.9999964)
\lineto(245.06249886,337.8515589)
\curveto(245.06248785,339.09373731)(244.82030059,340.02342388)(244.33593636,340.6406214)
\curveto(243.85155156,341.25779764)(243.12498979,341.56639109)(242.15624886,341.56640265)
\curveto(240.99217942,341.56639109)(240.07421159,341.19529771)(239.40234261,340.4531214)
\curveto(238.73046293,339.71092419)(238.39452577,338.69920645)(238.39453011,337.41796515)
\lineto(238.39453011,329.9999964)
\lineto(236.22656136,329.9999964)
\lineto(236.22656136,343.1249964)
\lineto(238.39453011,343.1249964)
\lineto(238.39453011,341.0859339)
\curveto(238.91015025,341.87498453)(239.51561839,342.46482769)(240.21093636,342.85546515)
\curveto(240.9140545,343.24607691)(241.72264744,343.44138921)(242.63671761,343.44140265)
\curveto(244.14452002,343.44138921)(245.28514388,342.97263968)(246.05859261,342.03515265)
\curveto(246.83201733,341.10545405)(247.21873569,339.73436167)(247.21874886,337.9218714)
}
}
{
\newrgbcolor{curcolor}{0 0 0}
\pscustom[linestyle=none,fillstyle=solid,fillcolor=curcolor]
{
}
}
{
\newrgbcolor{curcolor}{0 0 0}
\pscustom[linestyle=none,fillstyle=solid,fillcolor=curcolor]
{
\newpath
\moveto(269.40234261,340.60546515)
\curveto(269.94139209,341.57420358)(270.5859227,342.28904661)(271.33593636,342.7499964)
\curveto(272.0859212,343.21092069)(272.96873282,343.44138921)(273.98437386,343.44140265)
\curveto(275.35154293,343.44138921)(276.40622938,342.96092094)(277.14843636,341.9999964)
\curveto(277.89060289,341.04686036)(278.26169627,339.68748671)(278.26171761,337.9218714)
\lineto(278.26171761,329.9999964)
\lineto(276.09374886,329.9999964)
\lineto(276.09374886,337.8515589)
\curveto(276.09372969,339.10936229)(275.87107366,340.04295511)(275.42578011,340.65234015)
\curveto(274.98044955,341.26170389)(274.30076273,341.56639109)(273.38671761,341.56640265)
\curveto(272.26951477,341.56639109)(271.38670315,341.19529771)(270.73828011,340.4531214)
\curveto(270.08982945,339.71092419)(269.76561102,338.69920645)(269.76562386,337.41796515)
\lineto(269.76562386,329.9999964)
\lineto(267.59765511,329.9999964)
\lineto(267.59765511,337.8515589)
\curveto(267.59764444,339.11717479)(267.37498841,340.0507676)(266.92968636,340.65234015)
\curveto(266.4843643,341.26170389)(265.79686499,341.56639109)(264.86718636,341.56640265)
\curveto(263.76561702,341.56639109)(262.89061789,341.19139146)(262.24218636,340.44140265)
\curveto(261.59374419,339.69920545)(261.26952577,338.69139396)(261.26953011,337.41796515)
\lineto(261.26953011,329.9999964)
\lineto(259.10156136,329.9999964)
\lineto(259.10156136,343.1249964)
\lineto(261.26953011,343.1249964)
\lineto(261.26953011,341.0859339)
\curveto(261.76171277,341.89060951)(262.35155593,342.48435892)(263.03906136,342.8671839)
\curveto(263.72655456,343.24998315)(264.54295999,343.44138921)(265.48828011,343.44140265)
\curveto(266.44139559,343.44138921)(267.24998854,343.19920195)(267.91406136,342.71484015)
\curveto(268.5859247,342.23045292)(269.08201795,341.52732862)(269.40234261,340.60546515)
}
}
{
\newrgbcolor{curcolor}{0 0 0}
\pscustom[linestyle=none,fillstyle=solid,fillcolor=curcolor]
{
\newpath
\moveto(282.35156136,335.1796839)
\lineto(282.35156136,343.1249964)
\lineto(284.50781136,343.1249964)
\lineto(284.50781136,335.26171515)
\curveto(284.50780717,334.01952363)(284.74999443,333.08593082)(285.23437386,332.4609339)
\curveto(285.71874346,331.84374456)(286.44530523,331.53515112)(287.41406136,331.53515265)
\curveto(288.5781156,331.53515112)(289.49608343,331.9062445)(290.16796761,332.6484339)
\curveto(290.84764458,333.39061801)(291.18748799,334.40233575)(291.18749886,335.68359015)
\lineto(291.18749886,343.1249964)
\lineto(293.34374886,343.1249964)
\lineto(293.34374886,329.9999964)
\lineto(291.18749886,329.9999964)
\lineto(291.18749886,332.0156214)
\curveto(290.66405101,331.21874518)(290.05467662,330.62499578)(289.35937386,330.2343714)
\curveto(288.6718655,329.85155905)(287.87108505,329.66015299)(286.95703011,329.66015265)
\curveto(285.44921248,329.66015299)(284.30468237,330.12890252)(283.52343636,331.06640265)
\curveto(282.74218393,332.00390065)(282.35155932,333.37499303)(282.35156136,335.1796839)
\moveto(287.77734261,343.44140265)
\lineto(287.77734261,343.44140265)
}
}
{
\newrgbcolor{curcolor}{0 0 0}
\pscustom[linestyle=none,fillstyle=solid,fillcolor=curcolor]
{
\newpath
\moveto(297.80859261,348.2343714)
\lineto(299.96484261,348.2343714)
\lineto(299.96484261,329.9999964)
\lineto(297.80859261,329.9999964)
\lineto(297.80859261,348.2343714)
}
}
{
\newrgbcolor{curcolor}{0 0 0}
\pscustom[linestyle=none,fillstyle=solid,fillcolor=curcolor]
{
\newpath
\moveto(306.59765511,346.8515589)
\lineto(306.59765511,343.1249964)
\lineto(311.03906136,343.1249964)
\lineto(311.03906136,341.44921515)
\lineto(306.59765511,341.44921515)
\lineto(306.59765511,334.32421515)
\curveto(306.59765072,333.2538994)(306.74218182,332.56640009)(307.03124886,332.26171515)
\curveto(307.32811874,331.9570257)(307.92577439,331.8046821)(308.82421761,331.8046839)
\lineto(311.03906136,331.8046839)
\lineto(311.03906136,329.9999964)
\lineto(308.82421761,329.9999964)
\curveto(307.16015016,329.9999964)(306.0117138,330.30858984)(305.37890511,330.92577765)
\curveto(304.74609007,331.5507761)(304.42968414,332.68358747)(304.42968636,334.32421515)
\lineto(304.42968636,341.44921515)
\lineto(302.84765511,341.44921515)
\lineto(302.84765511,343.1249964)
\lineto(304.42968636,343.1249964)
\lineto(304.42968636,346.8515589)
\lineto(306.59765511,346.8515589)
}
}
{
\newrgbcolor{curcolor}{0 0 0}
\pscustom[linestyle=none,fillstyle=solid,fillcolor=curcolor]
{
\newpath
\moveto(313.88671761,343.1249964)
\lineto(316.04296761,343.1249964)
\lineto(316.04296761,329.9999964)
\lineto(313.88671761,329.9999964)
\lineto(313.88671761,343.1249964)
\moveto(313.88671761,348.2343714)
\lineto(316.04296761,348.2343714)
\lineto(316.04296761,345.50390265)
\lineto(313.88671761,345.50390265)
\lineto(313.88671761,348.2343714)
}
}
{
\newrgbcolor{curcolor}{0 0 0}
\pscustom[linestyle=none,fillstyle=solid,fillcolor=curcolor]
{
\newpath
\moveto(320.54296761,343.1249964)
\lineto(322.69921761,343.1249964)
\lineto(322.69921761,329.7656214)
\curveto(322.6992132,328.09374831)(322.37890102,326.88281202)(321.73828011,326.1328089)
\curveto(321.10546479,325.38281352)(320.08202831,325.00781389)(318.66796761,325.0078089)
\lineto(317.84765511,325.0078089)
\lineto(317.84765511,326.8359339)
\lineto(318.42187386,326.8359339)
\curveto(319.2421854,326.83593707)(319.80077859,327.02734313)(320.09765511,327.41015265)
\curveto(320.394528,327.78515487)(320.54296535,328.57031033)(320.54296761,329.7656214)
\lineto(320.54296761,343.1249964)
\moveto(320.54296761,348.2343714)
\lineto(322.69921761,348.2343714)
\lineto(322.69921761,345.50390265)
\lineto(320.54296761,345.50390265)
\lineto(320.54296761,348.2343714)
}
}
{
\newrgbcolor{curcolor}{0 0 0}
\pscustom[linestyle=none,fillstyle=solid,fillcolor=curcolor]
{
\newpath
\moveto(332.28515511,341.61327765)
\curveto(331.12889892,341.61326604)(330.21483734,341.16014149)(329.54296761,340.25390265)
\curveto(328.87108868,339.3554558)(328.53515152,338.12108203)(328.53515511,336.55077765)
\curveto(328.53515152,334.98046017)(328.86718243,333.74218016)(329.53124886,332.8359339)
\curveto(330.2031186,331.93749446)(331.12108643,331.48827616)(332.28515511,331.48827765)
\curveto(333.43358412,331.48827616)(334.34373946,331.94140071)(335.01562386,332.84765265)
\curveto(335.68748811,333.7538989)(336.02342528,334.98827266)(336.02343636,336.55077765)
\curveto(336.02342528,338.10545705)(335.68748811,339.33592457)(335.01562386,340.2421839)
\curveto(334.34373946,341.15623525)(333.43358412,341.61326604)(332.28515511,341.61327765)
\moveto(332.28515511,343.44140265)
\curveto(334.16014589,343.44138921)(335.63280067,342.83201482)(336.70312386,341.61327765)
\curveto(337.77342353,340.39451726)(338.30857924,338.70701895)(338.30859261,336.55077765)
\curveto(338.30857924,334.40233575)(337.77342353,332.71483744)(336.70312386,331.48827765)
\curveto(335.63280067,330.26952738)(334.16014589,329.66015299)(332.28515511,329.66015265)
\curveto(330.40233715,329.66015299)(328.92577612,330.26952738)(327.85546761,331.48827765)
\curveto(326.79296576,332.71483744)(326.26171629,334.40233575)(326.26171761,336.55077765)
\curveto(326.26171629,338.70701895)(326.79296576,340.39451726)(327.85546761,341.61327765)
\curveto(328.92577612,342.83201482)(330.40233715,343.44138921)(332.28515511,343.44140265)
}
}
{
\newrgbcolor{curcolor}{0 0 0}
\pscustom[linestyle=none,fillstyle=solid,fillcolor=curcolor]
{
\newpath
\moveto(341.64843636,335.1796839)
\lineto(341.64843636,343.1249964)
\lineto(343.80468636,343.1249964)
\lineto(343.80468636,335.26171515)
\curveto(343.80468217,334.01952363)(344.04686943,333.08593082)(344.53124886,332.4609339)
\curveto(345.01561846,331.84374456)(345.74218023,331.53515112)(346.71093636,331.53515265)
\curveto(347.8749906,331.53515112)(348.79295843,331.9062445)(349.46484261,332.6484339)
\curveto(350.14451958,333.39061801)(350.48436299,334.40233575)(350.48437386,335.68359015)
\lineto(350.48437386,343.1249964)
\lineto(352.64062386,343.1249964)
\lineto(352.64062386,329.9999964)
\lineto(350.48437386,329.9999964)
\lineto(350.48437386,332.0156214)
\curveto(349.96092601,331.21874518)(349.35155162,330.62499578)(348.65624886,330.2343714)
\curveto(347.9687405,329.85155905)(347.16796005,329.66015299)(346.25390511,329.66015265)
\curveto(344.74608748,329.66015299)(343.60155737,330.12890252)(342.82031136,331.06640265)
\curveto(342.03905893,332.00390065)(341.64843432,333.37499303)(341.64843636,335.1796839)
\moveto(347.07421761,343.44140265)
\lineto(347.07421761,343.44140265)
}
}
{
\newrgbcolor{curcolor}{0 0 0}
\pscustom[linestyle=none,fillstyle=solid,fillcolor=curcolor]
{
\newpath
\moveto(368.33203011,337.1015589)
\lineto(368.33203011,336.0468714)
\lineto(358.41796761,336.0468714)
\curveto(358.51171395,334.56249184)(358.957026,333.42968047)(359.75390511,332.6484339)
\curveto(360.5585869,331.87499453)(361.67577328,331.48827616)(363.10546761,331.48827765)
\curveto(363.93358352,331.48827616)(364.73436397,331.58983856)(365.50781136,331.79296515)
\curveto(366.28904992,331.99608816)(367.06248664,332.30077535)(367.82812386,332.70702765)
\lineto(367.82812386,330.66796515)
\curveto(367.05467415,330.33983981)(366.2617062,330.08984006)(365.44921761,329.91796515)
\curveto(364.63670782,329.74609041)(363.81248989,329.66015299)(362.97656136,329.66015265)
\curveto(360.88280532,329.66015299)(359.22265073,330.26952738)(357.99609261,331.48827765)
\curveto(356.77734068,332.70702495)(356.16796629,334.3554608)(356.16796761,336.43359015)
\curveto(356.16796629,338.58201907)(356.74609071,340.28514237)(357.90234261,341.54296515)
\curveto(359.06640089,342.80857734)(360.63280557,343.44138921)(362.60156136,343.44140265)
\curveto(364.36717684,343.44138921)(365.7617067,342.87107728)(366.78515511,341.73046515)
\curveto(367.81639214,340.59764205)(368.33201662,339.05467485)(368.33203011,337.1015589)
\moveto(366.17578011,337.7343714)
\curveto(366.1601438,338.91404999)(365.82811288,339.8554553)(365.17968636,340.55859015)
\curveto(364.53905167,341.26170389)(363.68749002,341.61326604)(362.62499886,341.61327765)
\curveto(361.42186729,341.61326604)(360.4570245,341.27342263)(359.73046761,340.5937464)
\curveto(359.01171345,339.91404899)(358.59765136,338.9570187)(358.48828011,337.72265265)
\lineto(366.17578011,337.7343714)
}
}
{
\newrgbcolor{curcolor}{0 0 0}
\pscustom[linestyle=none,fillstyle=solid,fillcolor=curcolor]
{
\newpath
\moveto(371.64843636,335.1796839)
\lineto(371.64843636,343.1249964)
\lineto(373.80468636,343.1249964)
\lineto(373.80468636,335.26171515)
\curveto(373.80468217,334.01952363)(374.04686943,333.08593082)(374.53124886,332.4609339)
\curveto(375.01561846,331.84374456)(375.74218023,331.53515112)(376.71093636,331.53515265)
\curveto(377.8749906,331.53515112)(378.79295843,331.9062445)(379.46484261,332.6484339)
\curveto(380.14451958,333.39061801)(380.48436299,334.40233575)(380.48437386,335.68359015)
\lineto(380.48437386,343.1249964)
\lineto(382.64062386,343.1249964)
\lineto(382.64062386,329.9999964)
\lineto(380.48437386,329.9999964)
\lineto(380.48437386,332.0156214)
\curveto(379.96092601,331.21874518)(379.35155162,330.62499578)(378.65624886,330.2343714)
\curveto(377.9687405,329.85155905)(377.16796005,329.66015299)(376.25390511,329.66015265)
\curveto(374.74608748,329.66015299)(373.60155737,330.12890252)(372.82031136,331.06640265)
\curveto(372.03905893,332.00390065)(371.64843432,333.37499303)(371.64843636,335.1796839)
\moveto(377.07421761,343.44140265)
\lineto(377.07421761,343.44140265)
}
}
{
\newrgbcolor{curcolor}{0 0 0}
\pscustom[linestyle=none,fillstyle=solid,fillcolor=curcolor]
{
\newpath
\moveto(394.71093636,341.1093714)
\curveto(394.46873924,341.24998515)(394.2031145,341.35154755)(393.91406136,341.4140589)
\curveto(393.63280257,341.48435992)(393.32030289,341.51951613)(392.97656136,341.51952765)
\curveto(391.75780445,341.51951613)(390.82030539,341.12107903)(390.16406136,340.32421515)
\curveto(389.51561919,339.53514312)(389.19140077,338.3984255)(389.19140511,336.9140589)
\lineto(389.19140511,329.9999964)
\lineto(387.02343636,329.9999964)
\lineto(387.02343636,343.1249964)
\lineto(389.19140511,343.1249964)
\lineto(389.19140511,341.0859339)
\curveto(389.64452531,341.88279702)(390.23436847,342.47264018)(390.96093636,342.85546515)
\curveto(391.68749202,343.24607691)(392.57030364,343.44138921)(393.60937386,343.44140265)
\curveto(393.75780245,343.44138921)(393.92186479,343.42967047)(394.10156136,343.4062464)
\curveto(394.28123943,343.39060801)(394.48045798,343.36326429)(394.69921761,343.32421515)
\lineto(394.71093636,341.1093714)
}
}
{
\newrgbcolor{curcolor}{1 1 1}
\pscustom[linestyle=none,fillstyle=solid,fillcolor=curcolor]
{
\newpath
\moveto(230.75260812,299.9999964)
\lineto(330.59985238,299.9999964)
\curveto(335.80753353,299.9999964)(339.99999886,295.80753107)(339.99999886,290.59984992)
\lineto(339.99999886,279.61122916)
\curveto(339.99999886,274.40354801)(335.80753353,270.21108268)(330.59985238,270.21108268)
\lineto(230.75260812,270.21108268)
\curveto(225.54492696,270.21108268)(221.35246163,274.40354801)(221.35246163,279.61122916)
\lineto(221.35246163,290.59984992)
\curveto(221.35246163,295.80753107)(225.54492696,299.9999964)(230.75260812,299.9999964)
\closepath
}
}
{
\newrgbcolor{curcolor}{0 0 0}
\pscustom[linewidth=2,linecolor=curcolor]
{
\newpath
\moveto(230.75260812,299.9999964)
\lineto(330.59985238,299.9999964)
\curveto(335.80753353,299.9999964)(339.99999886,295.80753107)(339.99999886,290.59984992)
\lineto(339.99999886,279.61122916)
\curveto(339.99999886,274.40354801)(335.80753353,270.21108268)(330.59985238,270.21108268)
\lineto(230.75260812,270.21108268)
\curveto(225.54492696,270.21108268)(221.35246163,274.40354801)(221.35246163,279.61122916)
\lineto(221.35246163,290.59984992)
\curveto(221.35246163,295.80753107)(225.54492696,299.9999964)(230.75260812,299.9999964)
\closepath
}
}
{
\newrgbcolor{curcolor}{0 0 0}
\pscustom[linestyle=none,fillstyle=solid,fillcolor=curcolor]
{
\newpath
\moveto(245.21453744,293.55099128)
\curveto(243.4957797,293.55097538)(242.12859357,292.91035103)(241.11297494,291.62911628)
\curveto(240.10515809,290.34785359)(239.60125234,288.60176158)(239.60125619,286.39083503)
\curveto(239.60125234,284.1877035)(240.10515809,282.44551774)(241.11297494,281.16427253)
\curveto(242.12859357,279.8830203)(243.4957797,279.24239594)(245.21453744,279.24239753)
\curveto(246.93327626,279.24239594)(248.2926499,279.8830203)(249.29266244,281.16427253)
\curveto(250.30046039,282.44551774)(250.80436614,284.1877035)(250.80438119,286.39083503)
\curveto(250.80436614,288.60176158)(250.30046039,290.34785359)(249.29266244,291.62911628)
\curveto(248.2926499,292.91035103)(246.93327626,293.55097538)(245.21453744,293.55099128)
\moveto(245.21453744,295.47286628)
\curveto(247.66765053,295.47284846)(249.62858607,294.64863054)(251.09734994,293.00021003)
\curveto(252.56608313,291.35957133)(253.30045739,289.15644853)(253.30047494,286.39083503)
\curveto(253.30045739,283.63301655)(252.56608313,281.42989376)(251.09734994,279.78146003)
\curveto(249.62858607,278.14083454)(247.66765053,277.32052287)(245.21453744,277.32052253)
\curveto(242.75359294,277.32052287)(240.78484491,278.14083454)(239.30828744,279.78146003)
\curveto(237.83953536,281.42208126)(237.10516109,283.62520406)(237.10516244,286.39083503)
\curveto(237.10516109,289.15644853)(237.83953536,291.35957133)(239.30828744,293.00021003)
\curveto(240.78484491,294.64863054)(242.75359294,295.47284846)(245.21453744,295.47286628)
}
}
{
\newrgbcolor{curcolor}{0 0 0}
\pscustom[linestyle=none,fillstyle=solid,fillcolor=curcolor]
{
\newpath
\moveto(258.99578744,279.62911628)
\lineto(258.99578744,272.66817878)
\lineto(256.82781869,272.66817878)
\lineto(256.82781869,290.78536628)
\lineto(258.99578744,290.78536628)
\lineto(258.99578744,288.79317878)
\curveto(259.44890764,289.57441686)(260.01921957,290.15254128)(260.70672494,290.52755378)
\curveto(261.40203068,290.91035303)(262.23015486,291.10175908)(263.19109994,291.10177253)
\curveto(264.7848398,291.10175908)(266.07780726,290.46894722)(267.07000619,289.20333503)
\curveto(268.06999277,287.93769975)(268.56999227,286.27363891)(268.57000619,284.21114753)
\curveto(268.56999227,282.14864304)(268.06999277,280.4845822)(267.07000619,279.21896003)
\curveto(266.07780726,277.95333473)(264.7848398,277.32052287)(263.19109994,277.32052253)
\curveto(262.23015486,277.32052287)(261.40203068,277.50802268)(260.70672494,277.88302253)
\curveto(260.01921957,278.26583442)(259.44890764,278.84786509)(258.99578744,279.62911628)
\moveto(266.33172494,284.21114753)
\curveto(266.33171325,285.79707689)(266.00358858,287.03926315)(265.34734994,287.93771003)
\curveto(264.69890239,288.84394884)(263.80437203,289.29707339)(262.66375619,289.29708503)
\curveto(261.52312431,289.29707339)(260.62468771,288.84394884)(259.96844369,287.93771003)
\curveto(259.32000152,287.03926315)(258.99578309,285.79707689)(258.99578744,284.21114753)
\curveto(258.99578309,282.62520506)(259.32000152,281.37911256)(259.96844369,280.47286628)
\curveto(260.62468771,279.57442686)(261.52312431,279.12520856)(262.66375619,279.12521003)
\curveto(263.80437203,279.12520856)(264.69890239,279.57442686)(265.34734994,280.47286628)
\curveto(266.00358858,281.37911256)(266.33171325,282.62520506)(266.33172494,284.21114753)
}
}
{
\newrgbcolor{curcolor}{0 0 0}
\pscustom[linestyle=none,fillstyle=solid,fillcolor=curcolor]
{
\newpath
\moveto(274.27703744,294.51192878)
\lineto(274.27703744,290.78536628)
\lineto(278.71844369,290.78536628)
\lineto(278.71844369,289.10958503)
\lineto(274.27703744,289.10958503)
\lineto(274.27703744,281.98458503)
\curveto(274.27703304,280.91426927)(274.42156415,280.22676996)(274.71063119,279.92208503)
\curveto(275.00750106,279.61739557)(275.60515671,279.46505197)(276.50359994,279.46505378)
\lineto(278.71844369,279.46505378)
\lineto(278.71844369,277.66036628)
\lineto(276.50359994,277.66036628)
\curveto(274.83953248,277.66036628)(273.69109613,277.96895972)(273.05828744,278.58614753)
\curveto(272.42547239,279.21114597)(272.10906646,280.34395734)(272.10906869,281.98458503)
\lineto(272.10906869,289.10958503)
\lineto(270.52703744,289.10958503)
\lineto(270.52703744,290.78536628)
\lineto(272.10906869,290.78536628)
\lineto(272.10906869,294.51192878)
\lineto(274.27703744,294.51192878)
}
}
{
\newrgbcolor{curcolor}{0 0 0}
\pscustom[linestyle=none,fillstyle=solid,fillcolor=curcolor]
{
\newpath
\moveto(281.56609994,290.78536628)
\lineto(283.72234994,290.78536628)
\lineto(283.72234994,277.66036628)
\lineto(281.56609994,277.66036628)
\lineto(281.56609994,290.78536628)
\moveto(281.56609994,295.89474128)
\lineto(283.72234994,295.89474128)
\lineto(283.72234994,293.16427253)
\lineto(281.56609994,293.16427253)
\lineto(281.56609994,295.89474128)
}
}
{
\newrgbcolor{curcolor}{0 0 0}
\pscustom[linestyle=none,fillstyle=solid,fillcolor=curcolor]
{
\newpath
\moveto(293.30828744,289.27364753)
\curveto(292.15203125,289.27363591)(291.23796966,288.82051137)(290.56609994,287.91427253)
\curveto(289.894221,287.01582567)(289.55828384,285.7814519)(289.55828744,284.21114753)
\curveto(289.55828384,282.64083004)(289.89031476,281.40255003)(290.55438119,280.49630378)
\curveto(291.22625092,279.59786434)(292.14421875,279.14864604)(293.30828744,279.14864753)
\curveto(294.45671644,279.14864604)(295.36687178,279.60177058)(296.03875619,280.50802253)
\curveto(296.71062044,281.41426877)(297.0465576,282.64864254)(297.04656869,284.21114753)
\curveto(297.0465576,285.76582692)(296.71062044,286.99629444)(296.03875619,287.90255378)
\curveto(295.36687178,288.81660512)(294.45671644,289.27363591)(293.30828744,289.27364753)
\moveto(293.30828744,291.10177253)
\curveto(295.18327821,291.10175908)(296.65593299,290.49238469)(297.72625619,289.27364753)
\curveto(298.79655585,288.05488713)(299.33171157,286.36738882)(299.33172494,284.21114753)
\curveto(299.33171157,282.06270562)(298.79655585,280.37520731)(297.72625619,279.14864753)
\curveto(296.65593299,277.92989726)(295.18327821,277.32052287)(293.30828744,277.32052253)
\curveto(291.42546947,277.32052287)(289.94890845,277.92989726)(288.87859994,279.14864753)
\curveto(287.81609808,280.37520731)(287.28484861,282.06270562)(287.28484994,284.21114753)
\curveto(287.28484861,286.36738882)(287.81609808,288.05488713)(288.87859994,289.27364753)
\curveto(289.94890845,290.49238469)(291.42546947,291.10175908)(293.30828744,291.10177253)
}
}
{
\newrgbcolor{curcolor}{0 0 0}
\pscustom[linestyle=none,fillstyle=solid,fillcolor=curcolor]
{
\newpath
\moveto(313.80438119,285.58224128)
\lineto(313.80438119,277.66036628)
\lineto(311.64813119,277.66036628)
\lineto(311.64813119,285.51192878)
\curveto(311.64812017,286.75410718)(311.40593291,287.68379375)(310.92156869,288.30099128)
\curveto(310.43718388,288.91816752)(309.71062211,289.22676096)(308.74188119,289.22677253)
\curveto(307.57781174,289.22676096)(306.65984391,288.85566758)(305.98797494,288.11349128)
\curveto(305.31609525,287.37129406)(304.98015809,286.35957633)(304.98016244,285.07833503)
\lineto(304.98016244,277.66036628)
\lineto(302.81219369,277.66036628)
\lineto(302.81219369,290.78536628)
\lineto(304.98016244,290.78536628)
\lineto(304.98016244,288.74630378)
\curveto(305.49578257,289.5353544)(306.10125072,290.12519756)(306.79656869,290.51583503)
\curveto(307.49968682,290.90644678)(308.30827976,291.10175908)(309.22234994,291.10177253)
\curveto(310.73015234,291.10175908)(311.8707762,290.63300955)(312.64422494,289.69552253)
\curveto(313.41764965,288.76582392)(313.80436802,287.39473154)(313.80438119,285.58224128)
}
}
{
\newrgbcolor{curcolor}{0 0 0}
\pscustom[linestyle=none,fillstyle=solid,fillcolor=curcolor]
{
\newpath
\moveto(326.49578744,290.39864753)
\lineto(326.49578744,288.35958503)
\curveto(325.88640242,288.67207401)(325.25359055,288.90644878)(324.59734994,289.06271003)
\curveto(323.94109186,289.21894847)(323.26140504,289.29707339)(322.55828744,289.29708503)
\curveto(321.48796932,289.29707339)(320.68328262,289.13301105)(320.14422494,288.80489753)
\curveto(319.61297119,288.47676171)(319.34734646,287.9845747)(319.34734994,287.32833503)
\curveto(319.34734646,286.82832586)(319.53875252,286.433795)(319.92156869,286.14474128)
\curveto(320.30437675,285.86348307)(321.07390723,285.59395209)(322.23016244,285.33614753)
\lineto(322.96844369,285.17208503)
\curveto(324.49968505,284.84395284)(325.58562147,284.37910956)(326.22625619,283.77755378)
\curveto(326.87468268,283.18379825)(327.19890111,282.35176783)(327.19891244,281.28146003)
\curveto(327.19890111,280.06270762)(326.71452659,279.09786484)(325.74578744,278.38692878)
\curveto(324.78484102,277.67599126)(323.46062359,277.32052287)(321.77313119,277.32052253)
\curveto(321.07000098,277.32052287)(320.33562672,277.39083529)(319.57000619,277.53146003)
\curveto(318.81219074,277.66427252)(318.01141029,277.86739732)(317.16766244,278.14083503)
\lineto(317.16766244,280.36739753)
\curveto(317.96453534,279.95333273)(318.7496908,279.64083304)(319.52313119,279.42989753)
\curveto(320.29656426,279.22677096)(321.06218849,279.12520856)(321.82000619,279.12521003)
\curveto(322.83562422,279.12520856)(323.61687344,279.29708339)(324.16375619,279.64083503)
\curveto(324.71062234,279.99239519)(324.98405957,280.4845822)(324.98406869,281.11739753)
\curveto(324.98405957,281.70333098)(324.78484102,282.15254928)(324.38641244,282.46505378)
\curveto(323.99577931,282.77754866)(323.13249892,283.07832961)(321.79656869,283.36739753)
\lineto(321.04656869,283.54317878)
\curveto(319.71062734,283.82442261)(318.74578456,284.25410968)(318.15203744,284.83224128)
\curveto(317.55828575,285.41817102)(317.26141104,286.21895147)(317.26141244,287.23458503)
\curveto(317.26141104,288.46894922)(317.69891061,289.42207326)(318.57391244,290.09396003)
\curveto(319.44890886,290.76582192)(320.69109511,291.10175908)(322.30047494,291.10177253)
\curveto(323.09734271,291.10175908)(323.84734196,291.04316539)(324.55047494,290.92599128)
\curveto(325.25359055,290.80879063)(325.9020274,290.63300955)(326.49578744,290.39864753)
}
}
{
\newrgbcolor{curcolor}{1 1 1}
\pscustom[linestyle=none,fillstyle=solid,fillcolor=curcolor]
{
\newpath
\moveto(182.63554459,249.84527228)
\lineto(378.59052735,249.84527228)
\curveto(384.94255905,249.84527228)(390.05628854,244.73154279)(390.05628854,238.3795111)
\lineto(390.05628854,231.39312384)
\curveto(390.05628854,225.04109215)(384.94255905,219.92736266)(378.59052735,219.92736266)
\lineto(182.63554459,219.92736266)
\curveto(176.2835129,219.92736266)(171.16978341,225.04109215)(171.16978341,231.39312384)
\lineto(171.16978341,238.3795111)
\curveto(171.16978341,244.73154279)(176.2835129,249.84527228)(182.63554459,249.84527228)
\closepath
}
}
{
\newrgbcolor{curcolor}{0 0 0}
\pscustom[linewidth=1.5483532,linecolor=curcolor]
{
\newpath
\moveto(182.63554459,249.84527228)
\lineto(378.59052735,249.84527228)
\curveto(384.94255905,249.84527228)(390.05628854,244.73154279)(390.05628854,238.3795111)
\lineto(390.05628854,231.39312384)
\curveto(390.05628854,225.04109215)(384.94255905,219.92736266)(378.59052735,219.92736266)
\lineto(182.63554459,219.92736266)
\curveto(176.2835129,219.92736266)(171.16978341,225.04109215)(171.16978341,231.39312384)
\lineto(171.16978341,238.3795111)
\curveto(171.16978341,244.73154279)(176.2835129,249.84527228)(182.63554459,249.84527228)
\closepath
}
}
{
\newrgbcolor{curcolor}{0 0 0}
\pscustom[linestyle=none,fillstyle=solid,fillcolor=curcolor]
{
\newpath
\moveto(182.35546761,247.49609015)
\lineto(185.88281136,247.49609015)
\lineto(190.34765511,235.58984015)
\lineto(194.83593636,247.49609015)
\lineto(198.36328011,247.49609015)
\lineto(198.36328011,229.9999964)
\lineto(196.05468636,229.9999964)
\lineto(196.05468636,245.36327765)
\lineto(191.54296761,233.36327765)
\lineto(189.16406136,233.36327765)
\lineto(184.65234261,245.36327765)
\lineto(184.65234261,229.9999964)
\lineto(182.35546761,229.9999964)
\lineto(182.35546761,247.49609015)
}
}
{
\newrgbcolor{curcolor}{0 0 0}
\pscustom[linestyle=none,fillstyle=solid,fillcolor=curcolor]
{
\newpath
\moveto(214.20703011,237.1015589)
\lineto(214.20703011,236.0468714)
\lineto(204.29296761,236.0468714)
\curveto(204.38671395,234.56249184)(204.832026,233.42968047)(205.62890511,232.6484339)
\curveto(206.4335869,231.87499453)(207.55077328,231.48827616)(208.98046761,231.48827765)
\curveto(209.80858352,231.48827616)(210.60936397,231.58983856)(211.38281136,231.79296515)
\curveto(212.16404992,231.99608816)(212.93748664,232.30077535)(213.70312386,232.70702765)
\lineto(213.70312386,230.66796515)
\curveto(212.92967415,230.33983981)(212.1367062,230.08984006)(211.32421761,229.91796515)
\curveto(210.51170782,229.74609041)(209.68748989,229.66015299)(208.85156136,229.66015265)
\curveto(206.75780532,229.66015299)(205.09765073,230.26952738)(203.87109261,231.48827765)
\curveto(202.65234068,232.70702495)(202.04296629,234.3554608)(202.04296761,236.43359015)
\curveto(202.04296629,238.58201907)(202.62109071,240.28514237)(203.77734261,241.54296515)
\curveto(204.94140089,242.80857734)(206.50780557,243.44138921)(208.47656136,243.44140265)
\curveto(210.24217684,243.44138921)(211.6367067,242.87107728)(212.66015511,241.73046515)
\curveto(213.69139214,240.59764205)(214.20701662,239.05467485)(214.20703011,237.1015589)
\moveto(212.05078011,237.7343714)
\curveto(212.0351438,238.91404999)(211.70311288,239.8554553)(211.05468636,240.55859015)
\curveto(210.41405167,241.26170389)(209.56249002,241.61326604)(208.49999886,241.61327765)
\curveto(207.29686729,241.61326604)(206.3320245,241.27342263)(205.60546761,240.5937464)
\curveto(204.88671345,239.91404899)(204.47265136,238.9570187)(204.36328011,237.72265265)
\lineto(212.05078011,237.7343714)
}
}
{
\newrgbcolor{curcolor}{0 0 0}
\pscustom[linestyle=none,fillstyle=solid,fillcolor=curcolor]
{
\newpath
\moveto(226.11328011,242.73827765)
\lineto(226.11328011,240.69921515)
\curveto(225.50389509,241.01170414)(224.87108323,241.24607891)(224.21484261,241.40234015)
\curveto(223.55858454,241.55857859)(222.87889772,241.63670352)(222.17578011,241.63671515)
\curveto(221.10546199,241.63670352)(220.3007753,241.47264118)(219.76171761,241.14452765)
\curveto(219.23046387,240.81639184)(218.96483913,240.32420483)(218.96484261,239.66796515)
\curveto(218.96483913,239.16795598)(219.15624519,238.77342513)(219.53906136,238.4843714)
\curveto(219.92186943,238.2031132)(220.69139991,237.93358222)(221.84765511,237.67577765)
\lineto(222.58593636,237.51171515)
\curveto(224.11717773,237.18358297)(225.20311414,236.71873968)(225.84374886,236.1171839)
\curveto(226.49217536,235.52342838)(226.81639378,234.69139796)(226.81640511,233.62109015)
\curveto(226.81639378,232.40233775)(226.33201927,231.43749496)(225.36328011,230.7265589)
\curveto(224.4023337,230.01562139)(223.07811627,229.66015299)(221.39062386,229.66015265)
\curveto(220.68749366,229.66015299)(219.95311939,229.73046542)(219.18749886,229.87109015)
\curveto(218.42968342,230.00390265)(217.62890297,230.20702745)(216.78515511,230.48046515)
\lineto(216.78515511,232.70702765)
\curveto(217.58202802,232.29296286)(218.36718348,231.98046317)(219.14062386,231.76952765)
\curveto(219.91405693,231.56640109)(220.67968117,231.46483869)(221.43749886,231.46484015)
\curveto(222.45311689,231.46483869)(223.23436611,231.63671352)(223.78124886,231.98046515)
\curveto(224.32811502,232.33202532)(224.60155225,232.82421233)(224.60156136,233.45702765)
\curveto(224.60155225,234.04296111)(224.4023337,234.49217941)(224.00390511,234.8046839)
\curveto(223.61327198,235.11717879)(222.7499916,235.41795973)(221.41406136,235.70702765)
\lineto(220.66406136,235.8828089)
\curveto(219.32812002,236.16405274)(218.36327723,236.59373981)(217.76953011,237.1718714)
\curveto(217.17577842,237.75780114)(216.87890372,238.55858159)(216.87890511,239.57421515)
\curveto(216.87890372,240.80857934)(217.31640328,241.76170339)(218.19140511,242.43359015)
\curveto(219.06640153,243.10545205)(220.30858779,243.44138921)(221.91796761,243.44140265)
\curveto(222.71483538,243.44138921)(223.46483463,243.38279552)(224.16796761,243.2656214)
\curveto(224.87108323,243.14842075)(225.51952008,242.97263968)(226.11328011,242.73827765)
}
}
{
\newrgbcolor{curcolor}{0 0 0}
\pscustom[linestyle=none,fillstyle=solid,fillcolor=curcolor]
{
}
}
{
\newrgbcolor{curcolor}{0 0 0}
\pscustom[linestyle=none,fillstyle=solid,fillcolor=curcolor]
{
\newpath
\moveto(246.26953011,242.73827765)
\lineto(246.26953011,240.69921515)
\curveto(245.66014509,241.01170414)(245.02733323,241.24607891)(244.37109261,241.40234015)
\curveto(243.71483454,241.55857859)(243.03514772,241.63670352)(242.33203011,241.63671515)
\curveto(241.26171199,241.63670352)(240.4570253,241.47264118)(239.91796761,241.14452765)
\curveto(239.38671387,240.81639184)(239.12108913,240.32420483)(239.12109261,239.66796515)
\curveto(239.12108913,239.16795598)(239.31249519,238.77342513)(239.69531136,238.4843714)
\curveto(240.07811943,238.2031132)(240.84764991,237.93358222)(242.00390511,237.67577765)
\lineto(242.74218636,237.51171515)
\curveto(244.27342773,237.18358297)(245.35936414,236.71873968)(245.99999886,236.1171839)
\curveto(246.64842536,235.52342838)(246.97264378,234.69139796)(246.97265511,233.62109015)
\curveto(246.97264378,232.40233775)(246.48826927,231.43749496)(245.51953011,230.7265589)
\curveto(244.5585837,230.01562139)(243.23436627,229.66015299)(241.54687386,229.66015265)
\curveto(240.84374366,229.66015299)(240.10936939,229.73046542)(239.34374886,229.87109015)
\curveto(238.58593342,230.00390265)(237.78515297,230.20702745)(236.94140511,230.48046515)
\lineto(236.94140511,232.70702765)
\curveto(237.73827802,232.29296286)(238.52343348,231.98046317)(239.29687386,231.76952765)
\curveto(240.07030693,231.56640109)(240.83593117,231.46483869)(241.59374886,231.46484015)
\curveto(242.60936689,231.46483869)(243.39061611,231.63671352)(243.93749886,231.98046515)
\curveto(244.48436502,232.33202532)(244.75780225,232.82421233)(244.75781136,233.45702765)
\curveto(244.75780225,234.04296111)(244.5585837,234.49217941)(244.16015511,234.8046839)
\curveto(243.76952198,235.11717879)(242.9062416,235.41795973)(241.57031136,235.70702765)
\lineto(240.82031136,235.8828089)
\curveto(239.48437002,236.16405274)(238.51952723,236.59373981)(237.92578011,237.1718714)
\curveto(237.33202842,237.75780114)(237.03515372,238.55858159)(237.03515511,239.57421515)
\curveto(237.03515372,240.80857934)(237.47265328,241.76170339)(238.34765511,242.43359015)
\curveto(239.22265153,243.10545205)(240.46483779,243.44138921)(242.07421761,243.44140265)
\curveto(242.87108538,243.44138921)(243.62108463,243.38279552)(244.32421761,243.2656214)
\curveto(245.02733323,243.14842075)(245.67577008,242.97263968)(246.26953011,242.73827765)
}
}
{
\newrgbcolor{curcolor}{0 0 0}
\pscustom[linestyle=none,fillstyle=solid,fillcolor=curcolor]
{
\newpath
\moveto(252.55078011,246.8515589)
\lineto(252.55078011,243.1249964)
\lineto(256.99218636,243.1249964)
\lineto(256.99218636,241.44921515)
\lineto(252.55078011,241.44921515)
\lineto(252.55078011,234.32421515)
\curveto(252.55077572,233.2538994)(252.69530682,232.56640009)(252.98437386,232.26171515)
\curveto(253.28124374,231.9570257)(253.87889939,231.8046821)(254.77734261,231.8046839)
\lineto(256.99218636,231.8046839)
\lineto(256.99218636,229.9999964)
\lineto(254.77734261,229.9999964)
\curveto(253.11327516,229.9999964)(251.9648388,230.30858984)(251.33203011,230.92577765)
\curveto(250.69921507,231.5507761)(250.38280914,232.68358747)(250.38281136,234.32421515)
\lineto(250.38281136,241.44921515)
\lineto(248.80078011,241.44921515)
\lineto(248.80078011,243.1249964)
\lineto(250.38281136,243.1249964)
\lineto(250.38281136,246.8515589)
\lineto(252.55078011,246.8515589)
}
}
{
\newrgbcolor{curcolor}{0 0 0}
\pscustom[linestyle=none,fillstyle=solid,fillcolor=curcolor]
{
\newpath
\moveto(265.80468636,236.59765265)
\curveto(264.06249238,236.59764605)(262.85546234,236.3984275)(262.18359261,235.9999964)
\curveto(261.51171368,235.6015533)(261.17577652,234.92186648)(261.17578011,233.9609339)
\curveto(261.17577652,233.19530571)(261.42577627,232.58593132)(261.92578011,232.1328089)
\curveto(262.43358776,231.68749471)(263.12108707,231.46483869)(263.98828011,231.46484015)
\curveto(265.18358501,231.46483869)(266.1406153,231.88671327)(266.85937386,232.73046515)
\curveto(267.58592636,233.58202407)(267.94920724,234.71092919)(267.94921761,236.1171839)
\lineto(267.94921761,236.59765265)
\lineto(265.80468636,236.59765265)
\moveto(270.10546761,237.48827765)
\lineto(270.10546761,229.9999964)
\lineto(267.94921761,229.9999964)
\lineto(267.94921761,231.9921839)
\curveto(267.45702023,231.19530771)(266.8437396,230.60546455)(266.10937386,230.22265265)
\curveto(265.37499107,229.8476528)(264.47655446,229.66015299)(263.41406136,229.66015265)
\curveto(262.07030687,229.66015299)(260.99999544,230.03515262)(260.20312386,230.78515265)
\curveto(259.41405953,231.54296361)(259.01952867,232.55468135)(259.01953011,233.8203089)
\curveto(259.01952867,235.29686611)(259.51171568,236.41014624)(260.49609261,237.16015265)
\curveto(261.4882762,237.91014474)(262.96483723,238.28514437)(264.92578011,238.28515265)
\lineto(267.94921761,238.28515265)
\lineto(267.94921761,238.49609015)
\curveto(267.94920724,239.48826816)(267.62108257,240.2538924)(266.96484261,240.79296515)
\curveto(266.31639638,241.33982881)(265.40233479,241.61326604)(264.22265511,241.61327765)
\curveto(263.47264922,241.61326604)(262.7421812,241.52342238)(262.03124886,241.3437464)
\curveto(261.32030762,241.16404774)(260.63671455,240.89451676)(259.98046761,240.53515265)
\lineto(259.98046761,242.52734015)
\curveto(260.76952692,242.83201482)(261.53515116,243.05857709)(262.27734261,243.20702765)
\curveto(263.01952467,243.36326429)(263.7421802,243.44138921)(264.44531136,243.44140265)
\curveto(266.3437401,243.44138921)(267.76170743,242.9492022)(268.69921761,241.96484015)
\curveto(269.63670555,240.98045417)(270.10545509,239.48826816)(270.10546761,237.48827765)
}
}
{
\newrgbcolor{curcolor}{0 0 0}
\pscustom[linestyle=none,fillstyle=solid,fillcolor=curcolor]
{
\newpath
\moveto(276.69140511,246.8515589)
\lineto(276.69140511,243.1249964)
\lineto(281.13281136,243.1249964)
\lineto(281.13281136,241.44921515)
\lineto(276.69140511,241.44921515)
\lineto(276.69140511,234.32421515)
\curveto(276.69140072,233.2538994)(276.83593182,232.56640009)(277.12499886,232.26171515)
\curveto(277.42186874,231.9570257)(278.01952439,231.8046821)(278.91796761,231.8046839)
\lineto(281.13281136,231.8046839)
\lineto(281.13281136,229.9999964)
\lineto(278.91796761,229.9999964)
\curveto(277.25390016,229.9999964)(276.1054638,230.30858984)(275.47265511,230.92577765)
\curveto(274.83984007,231.5507761)(274.52343414,232.68358747)(274.52343636,234.32421515)
\lineto(274.52343636,241.44921515)
\lineto(272.94140511,241.44921515)
\lineto(272.94140511,243.1249964)
\lineto(274.52343636,243.1249964)
\lineto(274.52343636,246.8515589)
\lineto(276.69140511,246.8515589)
}
}
{
\newrgbcolor{curcolor}{0 0 0}
\pscustom[linestyle=none,fillstyle=solid,fillcolor=curcolor]
{
\newpath
\moveto(283.98046761,243.1249964)
\lineto(286.13671761,243.1249964)
\lineto(286.13671761,229.9999964)
\lineto(283.98046761,229.9999964)
\lineto(283.98046761,243.1249964)
\moveto(283.98046761,248.2343714)
\lineto(286.13671761,248.2343714)
\lineto(286.13671761,245.50390265)
\lineto(283.98046761,245.50390265)
\lineto(283.98046761,248.2343714)
}
}
{
\newrgbcolor{curcolor}{0 0 0}
\pscustom[linestyle=none,fillstyle=solid,fillcolor=curcolor]
{
\newpath
\moveto(299.00390511,242.73827765)
\lineto(299.00390511,240.69921515)
\curveto(298.39452009,241.01170414)(297.76170823,241.24607891)(297.10546761,241.40234015)
\curveto(296.44920954,241.55857859)(295.76952272,241.63670352)(295.06640511,241.63671515)
\curveto(293.99608699,241.63670352)(293.1914003,241.47264118)(292.65234261,241.14452765)
\curveto(292.12108887,240.81639184)(291.85546413,240.32420483)(291.85546761,239.66796515)
\curveto(291.85546413,239.16795598)(292.04687019,238.77342513)(292.42968636,238.4843714)
\curveto(292.81249443,238.2031132)(293.58202491,237.93358222)(294.73828011,237.67577765)
\lineto(295.47656136,237.51171515)
\curveto(297.00780273,237.18358297)(298.09373914,236.71873968)(298.73437386,236.1171839)
\curveto(299.38280036,235.52342838)(299.70701878,234.69139796)(299.70703011,233.62109015)
\curveto(299.70701878,232.40233775)(299.22264427,231.43749496)(298.25390511,230.7265589)
\curveto(297.2929587,230.01562139)(295.96874127,229.66015299)(294.28124886,229.66015265)
\curveto(293.57811866,229.66015299)(292.84374439,229.73046542)(292.07812386,229.87109015)
\curveto(291.32030842,230.00390265)(290.51952797,230.20702745)(289.67578011,230.48046515)
\lineto(289.67578011,232.70702765)
\curveto(290.47265302,232.29296286)(291.25780848,231.98046317)(292.03124886,231.76952765)
\curveto(292.80468193,231.56640109)(293.57030617,231.46483869)(294.32812386,231.46484015)
\curveto(295.34374189,231.46483869)(296.12499111,231.63671352)(296.67187386,231.98046515)
\curveto(297.21874002,232.33202532)(297.49217725,232.82421233)(297.49218636,233.45702765)
\curveto(297.49217725,234.04296111)(297.2929587,234.49217941)(296.89453011,234.8046839)
\curveto(296.50389698,235.11717879)(295.6406166,235.41795973)(294.30468636,235.70702765)
\lineto(293.55468636,235.8828089)
\curveto(292.21874502,236.16405274)(291.25390223,236.59373981)(290.66015511,237.1718714)
\curveto(290.06640342,237.75780114)(289.76952872,238.55858159)(289.76953011,239.57421515)
\curveto(289.76952872,240.80857934)(290.20702828,241.76170339)(291.08203011,242.43359015)
\curveto(291.95702653,243.10545205)(293.19921279,243.44138921)(294.80859261,243.44140265)
\curveto(295.60546038,243.44138921)(296.35545963,243.38279552)(297.05859261,243.2656214)
\curveto(297.76170823,243.14842075)(298.41014508,242.97263968)(299.00390511,242.73827765)
}
}
{
\newrgbcolor{curcolor}{0 0 0}
\pscustom[linestyle=none,fillstyle=solid,fillcolor=curcolor]
{
\newpath
\moveto(305.28515511,246.8515589)
\lineto(305.28515511,243.1249964)
\lineto(309.72656136,243.1249964)
\lineto(309.72656136,241.44921515)
\lineto(305.28515511,241.44921515)
\lineto(305.28515511,234.32421515)
\curveto(305.28515072,233.2538994)(305.42968182,232.56640009)(305.71874886,232.26171515)
\curveto(306.01561874,231.9570257)(306.61327439,231.8046821)(307.51171761,231.8046839)
\lineto(309.72656136,231.8046839)
\lineto(309.72656136,229.9999964)
\lineto(307.51171761,229.9999964)
\curveto(305.84765016,229.9999964)(304.6992138,230.30858984)(304.06640511,230.92577765)
\curveto(303.43359007,231.5507761)(303.11718414,232.68358747)(303.11718636,234.32421515)
\lineto(303.11718636,241.44921515)
\lineto(301.53515511,241.44921515)
\lineto(301.53515511,243.1249964)
\lineto(303.11718636,243.1249964)
\lineto(303.11718636,246.8515589)
\lineto(305.28515511,246.8515589)
}
}
{
\newrgbcolor{curcolor}{0 0 0}
\pscustom[linestyle=none,fillstyle=solid,fillcolor=curcolor]
{
\newpath
\moveto(312.57421761,243.1249964)
\lineto(314.73046761,243.1249964)
\lineto(314.73046761,229.9999964)
\lineto(312.57421761,229.9999964)
\lineto(312.57421761,243.1249964)
\moveto(312.57421761,248.2343714)
\lineto(314.73046761,248.2343714)
\lineto(314.73046761,245.50390265)
\lineto(312.57421761,245.50390265)
\lineto(312.57421761,248.2343714)
}
}
{
\newrgbcolor{curcolor}{0 0 0}
\pscustom[linestyle=none,fillstyle=solid,fillcolor=curcolor]
{
\newpath
\moveto(320.51953011,236.55077765)
\curveto(320.51952656,234.96483519)(320.84374499,233.71874268)(321.49218636,232.8124964)
\curveto(322.14843118,231.91405699)(323.04686779,231.46483869)(324.18749886,231.46484015)
\curveto(325.3281155,231.46483869)(326.22655211,231.91405699)(326.88281136,232.8124964)
\curveto(327.53905079,233.71874268)(327.86717546,234.96483519)(327.86718636,236.55077765)
\curveto(327.86717546,238.13670702)(327.53905079,239.37889327)(326.88281136,240.27734015)
\curveto(326.22655211,241.18357897)(325.3281155,241.63670352)(324.18749886,241.63671515)
\curveto(323.04686779,241.63670352)(322.14843118,241.18357897)(321.49218636,240.27734015)
\curveto(320.84374499,239.37889327)(320.51952656,238.13670702)(320.51953011,236.55077765)
\moveto(327.86718636,231.9687464)
\curveto(327.41405092,231.18749521)(326.83983274,230.60546455)(326.14453011,230.22265265)
\curveto(325.45702162,229.8476528)(324.62889745,229.66015299)(323.66015511,229.66015265)
\curveto(322.07421251,229.66015299)(320.78124505,230.29296486)(319.78124886,231.55859015)
\curveto(318.78905954,232.82421233)(318.29296629,234.48827316)(318.29296761,236.55077765)
\curveto(318.29296629,238.61326904)(318.78905954,240.27732988)(319.78124886,241.54296515)
\curveto(320.78124505,242.80857734)(322.07421251,243.44138921)(323.66015511,243.44140265)
\curveto(324.62889745,243.44138921)(325.45702162,243.24998315)(326.14453011,242.8671839)
\curveto(326.83983274,242.49217141)(327.41405092,241.91404699)(327.86718636,241.1328089)
\lineto(327.86718636,243.1249964)
\lineto(330.02343636,243.1249964)
\lineto(330.02343636,225.0078089)
\lineto(327.86718636,225.0078089)
\lineto(327.86718636,231.9687464)
}
}
{
\newrgbcolor{curcolor}{0 0 0}
\pscustom[linestyle=none,fillstyle=solid,fillcolor=curcolor]
{
\newpath
\moveto(334.24218636,235.1796839)
\lineto(334.24218636,243.1249964)
\lineto(336.39843636,243.1249964)
\lineto(336.39843636,235.26171515)
\curveto(336.39843217,234.01952363)(336.64061943,233.08593082)(337.12499886,232.4609339)
\curveto(337.60936846,231.84374456)(338.33593023,231.53515112)(339.30468636,231.53515265)
\curveto(340.4687406,231.53515112)(341.38670843,231.9062445)(342.05859261,232.6484339)
\curveto(342.73826958,233.39061801)(343.07811299,234.40233575)(343.07812386,235.68359015)
\lineto(343.07812386,243.1249964)
\lineto(345.23437386,243.1249964)
\lineto(345.23437386,229.9999964)
\lineto(343.07812386,229.9999964)
\lineto(343.07812386,232.0156214)
\curveto(342.55467601,231.21874518)(341.94530162,230.62499578)(341.24999886,230.2343714)
\curveto(340.5624905,229.85155905)(339.76171005,229.66015299)(338.84765511,229.66015265)
\curveto(337.33983748,229.66015299)(336.19530737,230.12890252)(335.41406136,231.06640265)
\curveto(334.63280893,232.00390065)(334.24218432,233.37499303)(334.24218636,235.1796839)
\moveto(339.66796761,243.44140265)
\lineto(339.66796761,243.44140265)
}
}
{
\newrgbcolor{curcolor}{0 0 0}
\pscustom[linestyle=none,fillstyle=solid,fillcolor=curcolor]
{
\newpath
\moveto(360.92578011,237.1015589)
\lineto(360.92578011,236.0468714)
\lineto(351.01171761,236.0468714)
\curveto(351.10546395,234.56249184)(351.550776,233.42968047)(352.34765511,232.6484339)
\curveto(353.1523369,231.87499453)(354.26952328,231.48827616)(355.69921761,231.48827765)
\curveto(356.52733352,231.48827616)(357.32811397,231.58983856)(358.10156136,231.79296515)
\curveto(358.88279992,231.99608816)(359.65623664,232.30077535)(360.42187386,232.70702765)
\lineto(360.42187386,230.66796515)
\curveto(359.64842415,230.33983981)(358.8554562,230.08984006)(358.04296761,229.91796515)
\curveto(357.23045782,229.74609041)(356.40623989,229.66015299)(355.57031136,229.66015265)
\curveto(353.47655532,229.66015299)(351.81640073,230.26952738)(350.58984261,231.48827765)
\curveto(349.37109068,232.70702495)(348.76171629,234.3554608)(348.76171761,236.43359015)
\curveto(348.76171629,238.58201907)(349.33984071,240.28514237)(350.49609261,241.54296515)
\curveto(351.66015089,242.80857734)(353.22655557,243.44138921)(355.19531136,243.44140265)
\curveto(356.96092684,243.44138921)(358.3554567,242.87107728)(359.37890511,241.73046515)
\curveto(360.41014214,240.59764205)(360.92576662,239.05467485)(360.92578011,237.1015589)
\moveto(358.76953011,237.7343714)
\curveto(358.7538938,238.91404999)(358.42186288,239.8554553)(357.77343636,240.55859015)
\curveto(357.13280167,241.26170389)(356.28124002,241.61326604)(355.21874886,241.61327765)
\curveto(354.01561729,241.61326604)(353.0507745,241.27342263)(352.32421761,240.5937464)
\curveto(351.60546345,239.91404899)(351.19140136,238.9570187)(351.08203011,237.72265265)
\lineto(358.76953011,237.7343714)
}
}
{
\newrgbcolor{curcolor}{0 0 0}
\pscustom[linestyle=none,fillstyle=solid,fillcolor=curcolor]
{
\newpath
\moveto(372.83203011,242.73827765)
\lineto(372.83203011,240.69921515)
\curveto(372.22264509,241.01170414)(371.58983323,241.24607891)(370.93359261,241.40234015)
\curveto(370.27733454,241.55857859)(369.59764772,241.63670352)(368.89453011,241.63671515)
\curveto(367.82421199,241.63670352)(367.0195253,241.47264118)(366.48046761,241.14452765)
\curveto(365.94921387,240.81639184)(365.68358913,240.32420483)(365.68359261,239.66796515)
\curveto(365.68358913,239.16795598)(365.87499519,238.77342513)(366.25781136,238.4843714)
\curveto(366.64061943,238.2031132)(367.41014991,237.93358222)(368.56640511,237.67577765)
\lineto(369.30468636,237.51171515)
\curveto(370.83592773,237.18358297)(371.92186414,236.71873968)(372.56249886,236.1171839)
\curveto(373.21092536,235.52342838)(373.53514378,234.69139796)(373.53515511,233.62109015)
\curveto(373.53514378,232.40233775)(373.05076927,231.43749496)(372.08203011,230.7265589)
\curveto(371.1210837,230.01562139)(369.79686627,229.66015299)(368.10937386,229.66015265)
\curveto(367.40624366,229.66015299)(366.67186939,229.73046542)(365.90624886,229.87109015)
\curveto(365.14843342,230.00390265)(364.34765297,230.20702745)(363.50390511,230.48046515)
\lineto(363.50390511,232.70702765)
\curveto(364.30077802,232.29296286)(365.08593348,231.98046317)(365.85937386,231.76952765)
\curveto(366.63280693,231.56640109)(367.39843117,231.46483869)(368.15624886,231.46484015)
\curveto(369.17186689,231.46483869)(369.95311611,231.63671352)(370.49999886,231.98046515)
\curveto(371.04686502,232.33202532)(371.32030225,232.82421233)(371.32031136,233.45702765)
\curveto(371.32030225,234.04296111)(371.1210837,234.49217941)(370.72265511,234.8046839)
\curveto(370.33202198,235.11717879)(369.4687416,235.41795973)(368.13281136,235.70702765)
\lineto(367.38281136,235.8828089)
\curveto(366.04687002,236.16405274)(365.08202723,236.59373981)(364.48828011,237.1718714)
\curveto(363.89452842,237.75780114)(363.59765372,238.55858159)(363.59765511,239.57421515)
\curveto(363.59765372,240.80857934)(364.03515328,241.76170339)(364.91015511,242.43359015)
\curveto(365.78515153,243.10545205)(367.02733779,243.44138921)(368.63671761,243.44140265)
\curveto(369.43358538,243.44138921)(370.18358463,243.38279552)(370.88671761,243.2656214)
\curveto(371.58983323,243.14842075)(372.23827008,242.97263968)(372.83203011,242.73827765)
}
}
{
\newrgbcolor{curcolor}{1 1 1}
\pscustom[linestyle=none,fillstyle=solid,fillcolor=curcolor]
{
\newpath
\moveto(203.99147492,199.9999964)
\lineto(357.5801323,199.9999964)
\curveto(364.46073838,199.9999964)(369.99999886,194.46073592)(369.99999886,187.58012984)
\lineto(369.99999886,182.59711669)
\curveto(369.99999886,175.71651061)(364.46073838,170.17725013)(357.5801323,170.17725013)
\lineto(203.99147492,170.17725013)
\curveto(197.11086885,170.17725013)(191.57160836,175.71651061)(191.57160836,182.59711669)
\lineto(191.57160836,187.58012984)
\curveto(191.57160836,194.46073592)(197.11086885,199.9999964)(203.99147492,199.9999964)
\closepath
}
}
{
\newrgbcolor{curcolor}{0 0 0}
\pscustom[linewidth=2,linecolor=curcolor]
{
\newpath
\moveto(203.99147492,199.9999964)
\lineto(357.5801323,199.9999964)
\curveto(364.46073838,199.9999964)(369.99999886,194.46073592)(369.99999886,187.58012984)
\lineto(369.99999886,182.59711669)
\curveto(369.99999886,175.71651061)(364.46073838,170.17725013)(357.5801323,170.17725013)
\lineto(203.99147492,170.17725013)
\curveto(197.11086885,170.17725013)(191.57160836,175.71651061)(191.57160836,182.59711669)
\lineto(191.57160836,187.58012984)
\curveto(191.57160836,194.46073592)(197.11086885,199.9999964)(203.99147492,199.9999964)
\closepath
}
}
{
\newrgbcolor{curcolor}{0 0 0}
\pscustom[linestyle=none,fillstyle=solid,fillcolor=curcolor]
{
\newpath
\moveto(215.45703011,196.1484339)
\lineto(215.45703011,193.65234015)
\curveto(214.66014045,194.39451326)(213.8085788,194.9492002)(212.90234261,195.31640265)
\curveto(212.00389311,195.68357447)(211.04686282,195.86716804)(210.03124886,195.8671839)
\curveto(208.03124083,195.86716804)(206.49999236,195.2538874)(205.43749886,194.02734015)
\curveto(204.37499449,192.80857734)(203.84374502,191.04295411)(203.84374886,188.73046515)
\curveto(203.84374502,186.42577123)(204.37499449,184.66014799)(205.43749886,183.43359015)
\curveto(206.49999236,182.21483794)(208.03124083,181.60546355)(210.03124886,181.60546515)
\curveto(211.04686282,181.60546355)(212.00389311,181.78905711)(212.90234261,182.1562464)
\curveto(213.8085788,182.52343138)(214.66014045,183.07811832)(215.45703011,183.8203089)
\lineto(215.45703011,181.34765265)
\curveto(214.62889048,180.78515187)(213.74998511,180.36327729)(212.82031136,180.08202765)
\curveto(211.89842446,179.80077785)(210.92186294,179.66015299)(209.89062386,179.66015265)
\curveto(207.24217912,179.66015299)(205.15624371,180.46874593)(203.63281136,182.0859339)
\curveto(202.10937175,183.71093019)(201.34765377,185.92577173)(201.34765511,188.73046515)
\curveto(201.34765377,191.54295361)(202.10937175,193.75779514)(203.63281136,195.3749964)
\curveto(205.15624371,196.9999794)(207.24217912,197.81247859)(209.89062386,197.8124964)
\curveto(210.93748793,197.81247859)(211.92186194,197.67185373)(212.84374886,197.3906214)
\curveto(213.77342259,197.11716679)(214.64451547,196.7031047)(215.45703011,196.1484339)
}
}
{
\newrgbcolor{curcolor}{0 0 0}
\pscustom[linestyle=none,fillstyle=solid,fillcolor=curcolor]
{
\newpath
\moveto(226.64843636,191.1093714)
\curveto(226.40623924,191.24998515)(226.1406145,191.35154755)(225.85156136,191.4140589)
\curveto(225.57030257,191.48435992)(225.25780289,191.51951613)(224.91406136,191.51952765)
\curveto(223.69530445,191.51951613)(222.75780539,191.12107903)(222.10156136,190.32421515)
\curveto(221.45311919,189.53514312)(221.12890077,188.3984255)(221.12890511,186.9140589)
\lineto(221.12890511,179.9999964)
\lineto(218.96093636,179.9999964)
\lineto(218.96093636,193.1249964)
\lineto(221.12890511,193.1249964)
\lineto(221.12890511,191.0859339)
\curveto(221.58202531,191.88279702)(222.17186847,192.47264018)(222.89843636,192.85546515)
\curveto(223.62499202,193.24607691)(224.50780364,193.44138921)(225.54687386,193.44140265)
\curveto(225.69530245,193.44138921)(225.85936479,193.42967047)(226.03906136,193.4062464)
\curveto(226.21873943,193.39060801)(226.41795798,193.36326429)(226.63671761,193.32421515)
\lineto(226.64843636,191.1093714)
}
}
{
\newrgbcolor{curcolor}{0 0 0}
\pscustom[linestyle=none,fillstyle=solid,fillcolor=curcolor]
{
\newpath
\moveto(239.64453011,187.1015589)
\lineto(239.64453011,186.0468714)
\lineto(229.73046761,186.0468714)
\curveto(229.82421395,184.56249184)(230.269526,183.42968047)(231.06640511,182.6484339)
\curveto(231.8710869,181.87499453)(232.98827328,181.48827616)(234.41796761,181.48827765)
\curveto(235.24608352,181.48827616)(236.04686397,181.58983856)(236.82031136,181.79296515)
\curveto(237.60154992,181.99608816)(238.37498664,182.30077535)(239.14062386,182.70702765)
\lineto(239.14062386,180.66796515)
\curveto(238.36717415,180.33983981)(237.5742062,180.08984006)(236.76171761,179.91796515)
\curveto(235.94920782,179.74609041)(235.12498989,179.66015299)(234.28906136,179.66015265)
\curveto(232.19530532,179.66015299)(230.53515073,180.26952738)(229.30859261,181.48827765)
\curveto(228.08984068,182.70702495)(227.48046629,184.3554608)(227.48046761,186.43359015)
\curveto(227.48046629,188.58201907)(228.05859071,190.28514237)(229.21484261,191.54296515)
\curveto(230.37890089,192.80857734)(231.94530557,193.44138921)(233.91406136,193.44140265)
\curveto(235.67967684,193.44138921)(237.0742067,192.87107728)(238.09765511,191.73046515)
\curveto(239.12889214,190.59764205)(239.64451662,189.05467485)(239.64453011,187.1015589)
\moveto(237.48828011,187.7343714)
\curveto(237.4726438,188.91404999)(237.14061288,189.8554553)(236.49218636,190.55859015)
\curveto(235.85155167,191.26170389)(234.99999002,191.61326604)(233.93749886,191.61327765)
\curveto(232.73436729,191.61326604)(231.7695245,191.27342263)(231.04296761,190.5937464)
\curveto(230.32421345,189.91404899)(229.91015136,188.9570187)(229.80078011,187.72265265)
\lineto(237.48828011,187.7343714)
\moveto(235.41406136,199.1953089)
\lineto(237.74609261,199.1953089)
\lineto(233.92578011,194.7890589)
\lineto(232.13281136,194.7890589)
\lineto(235.41406136,199.1953089)
}
}
{
\newrgbcolor{curcolor}{0 0 0}
\pscustom[linestyle=none,fillstyle=solid,fillcolor=curcolor]
{
\newpath
\moveto(254.41015511,187.1015589)
\lineto(254.41015511,186.0468714)
\lineto(244.49609261,186.0468714)
\curveto(244.58983895,184.56249184)(245.035151,183.42968047)(245.83203011,182.6484339)
\curveto(246.6367119,181.87499453)(247.75389828,181.48827616)(249.18359261,181.48827765)
\curveto(250.01170852,181.48827616)(250.81248897,181.58983856)(251.58593636,181.79296515)
\curveto(252.36717492,181.99608816)(253.14061164,182.30077535)(253.90624886,182.70702765)
\lineto(253.90624886,180.66796515)
\curveto(253.13279915,180.33983981)(252.3398312,180.08984006)(251.52734261,179.91796515)
\curveto(250.71483282,179.74609041)(249.89061489,179.66015299)(249.05468636,179.66015265)
\curveto(246.96093032,179.66015299)(245.30077573,180.26952738)(244.07421761,181.48827765)
\curveto(242.85546568,182.70702495)(242.24609129,184.3554608)(242.24609261,186.43359015)
\curveto(242.24609129,188.58201907)(242.82421571,190.28514237)(243.98046761,191.54296515)
\curveto(245.14452589,192.80857734)(246.71093057,193.44138921)(248.67968636,193.44140265)
\curveto(250.44530184,193.44138921)(251.8398317,192.87107728)(252.86328011,191.73046515)
\curveto(253.89451714,190.59764205)(254.41014162,189.05467485)(254.41015511,187.1015589)
\moveto(252.25390511,187.7343714)
\curveto(252.2382688,188.91404999)(251.90623788,189.8554553)(251.25781136,190.55859015)
\curveto(250.61717667,191.26170389)(249.76561502,191.61326604)(248.70312386,191.61327765)
\curveto(247.49999229,191.61326604)(246.5351495,191.27342263)(245.80859261,190.5937464)
\curveto(245.08983845,189.91404899)(244.67577636,188.9570187)(244.56640511,187.72265265)
\lineto(252.25390511,187.7343714)
}
}
{
\newrgbcolor{curcolor}{0 0 0}
\pscustom[linestyle=none,fillstyle=solid,fillcolor=curcolor]
{
\newpath
\moveto(265.55468636,191.1093714)
\curveto(265.31248924,191.24998515)(265.0468645,191.35154755)(264.75781136,191.4140589)
\curveto(264.47655257,191.48435992)(264.16405289,191.51951613)(263.82031136,191.51952765)
\curveto(262.60155445,191.51951613)(261.66405539,191.12107903)(261.00781136,190.32421515)
\curveto(260.35936919,189.53514312)(260.03515077,188.3984255)(260.03515511,186.9140589)
\lineto(260.03515511,179.9999964)
\lineto(257.86718636,179.9999964)
\lineto(257.86718636,193.1249964)
\lineto(260.03515511,193.1249964)
\lineto(260.03515511,191.0859339)
\curveto(260.48827531,191.88279702)(261.07811847,192.47264018)(261.80468636,192.85546515)
\curveto(262.53124202,193.24607691)(263.41405364,193.44138921)(264.45312386,193.44140265)
\curveto(264.60155245,193.44138921)(264.76561479,193.42967047)(264.94531136,193.4062464)
\curveto(265.12498943,193.39060801)(265.32420798,193.36326429)(265.54296761,193.32421515)
\lineto(265.55468636,191.1093714)
}
}
{
\newrgbcolor{curcolor}{0 0 0}
\pscustom[linestyle=none,fillstyle=solid,fillcolor=curcolor]
{
}
}
{
\newrgbcolor{curcolor}{0 0 0}
\pscustom[linestyle=none,fillstyle=solid,fillcolor=curcolor]
{
\newpath
\moveto(286.39062386,187.9218714)
\lineto(286.39062386,179.9999964)
\lineto(284.23437386,179.9999964)
\lineto(284.23437386,187.8515589)
\curveto(284.23436285,189.09373731)(283.99217559,190.02342388)(283.50781136,190.6406214)
\curveto(283.02342656,191.25779764)(282.29686479,191.56639109)(281.32812386,191.56640265)
\curveto(280.16405442,191.56639109)(279.24608659,191.19529771)(278.57421761,190.4531214)
\curveto(277.90233793,189.71092419)(277.56640077,188.69920645)(277.56640511,187.41796515)
\lineto(277.56640511,179.9999964)
\lineto(275.39843636,179.9999964)
\lineto(275.39843636,193.1249964)
\lineto(277.56640511,193.1249964)
\lineto(277.56640511,191.0859339)
\curveto(278.08202525,191.87498453)(278.68749339,192.46482769)(279.38281136,192.85546515)
\curveto(280.0859295,193.24607691)(280.89452244,193.44138921)(281.80859261,193.44140265)
\curveto(283.31639502,193.44138921)(284.45701888,192.97263968)(285.23046761,192.03515265)
\curveto(286.00389233,191.10545405)(286.39061069,189.73436167)(286.39062386,187.9218714)
}
}
{
\newrgbcolor{curcolor}{0 0 0}
\pscustom[linestyle=none,fillstyle=solid,fillcolor=curcolor]
{
\newpath
\moveto(290.71484261,193.1249964)
\lineto(292.87109261,193.1249964)
\lineto(292.87109261,179.9999964)
\lineto(290.71484261,179.9999964)
\lineto(290.71484261,193.1249964)
\moveto(290.71484261,198.2343714)
\lineto(292.87109261,198.2343714)
\lineto(292.87109261,195.50390265)
\lineto(290.71484261,195.50390265)
\lineto(290.71484261,198.2343714)
}
}
{
\newrgbcolor{curcolor}{0 0 0}
\pscustom[linestyle=none,fillstyle=solid,fillcolor=curcolor]
{
\newpath
\moveto(295.82421761,193.1249964)
\lineto(298.10937386,193.1249964)
\lineto(302.21093636,182.1093714)
\lineto(306.31249886,193.1249964)
\lineto(308.59765511,193.1249964)
\lineto(303.67578011,179.9999964)
\lineto(300.74609261,179.9999964)
\lineto(295.82421761,193.1249964)
}
}
{
\newrgbcolor{curcolor}{0 0 0}
\pscustom[linestyle=none,fillstyle=solid,fillcolor=curcolor]
{
\newpath
\moveto(322.80078011,187.1015589)
\lineto(322.80078011,186.0468714)
\lineto(312.88671761,186.0468714)
\curveto(312.98046395,184.56249184)(313.425776,183.42968047)(314.22265511,182.6484339)
\curveto(315.0273369,181.87499453)(316.14452328,181.48827616)(317.57421761,181.48827765)
\curveto(318.40233352,181.48827616)(319.20311397,181.58983856)(319.97656136,181.79296515)
\curveto(320.75779992,181.99608816)(321.53123664,182.30077535)(322.29687386,182.70702765)
\lineto(322.29687386,180.66796515)
\curveto(321.52342415,180.33983981)(320.7304562,180.08984006)(319.91796761,179.91796515)
\curveto(319.10545782,179.74609041)(318.28123989,179.66015299)(317.44531136,179.66015265)
\curveto(315.35155532,179.66015299)(313.69140073,180.26952738)(312.46484261,181.48827765)
\curveto(311.24609068,182.70702495)(310.63671629,184.3554608)(310.63671761,186.43359015)
\curveto(310.63671629,188.58201907)(311.21484071,190.28514237)(312.37109261,191.54296515)
\curveto(313.53515089,192.80857734)(315.10155557,193.44138921)(317.07031136,193.44140265)
\curveto(318.83592684,193.44138921)(320.2304567,192.87107728)(321.25390511,191.73046515)
\curveto(322.28514214,190.59764205)(322.80076662,189.05467485)(322.80078011,187.1015589)
\moveto(320.64453011,187.7343714)
\curveto(320.6288938,188.91404999)(320.29686288,189.8554553)(319.64843636,190.55859015)
\curveto(319.00780167,191.26170389)(318.15624002,191.61326604)(317.09374886,191.61327765)
\curveto(315.89061729,191.61326604)(314.9257745,191.27342263)(314.19921761,190.5937464)
\curveto(313.48046345,189.91404899)(313.06640136,188.9570187)(312.95703011,187.72265265)
\lineto(320.64453011,187.7343714)
}
}
{
\newrgbcolor{curcolor}{0 0 0}
\pscustom[linestyle=none,fillstyle=solid,fillcolor=curcolor]
{
\newpath
\moveto(332.30468636,186.59765265)
\curveto(330.56249238,186.59764605)(329.35546234,186.3984275)(328.68359261,185.9999964)
\curveto(328.01171368,185.6015533)(327.67577652,184.92186648)(327.67578011,183.9609339)
\curveto(327.67577652,183.19530571)(327.92577627,182.58593132)(328.42578011,182.1328089)
\curveto(328.93358776,181.68749471)(329.62108707,181.46483869)(330.48828011,181.46484015)
\curveto(331.68358501,181.46483869)(332.6406153,181.88671327)(333.35937386,182.73046515)
\curveto(334.08592636,183.58202407)(334.44920724,184.71092919)(334.44921761,186.1171839)
\lineto(334.44921761,186.59765265)
\lineto(332.30468636,186.59765265)
\moveto(336.60546761,187.48827765)
\lineto(336.60546761,179.9999964)
\lineto(334.44921761,179.9999964)
\lineto(334.44921761,181.9921839)
\curveto(333.95702023,181.19530771)(333.3437396,180.60546455)(332.60937386,180.22265265)
\curveto(331.87499107,179.8476528)(330.97655446,179.66015299)(329.91406136,179.66015265)
\curveto(328.57030687,179.66015299)(327.49999544,180.03515262)(326.70312386,180.78515265)
\curveto(325.91405953,181.54296361)(325.51952867,182.55468135)(325.51953011,183.8203089)
\curveto(325.51952867,185.29686611)(326.01171568,186.41014624)(326.99609261,187.16015265)
\curveto(327.9882762,187.91014474)(329.46483723,188.28514437)(331.42578011,188.28515265)
\lineto(334.44921761,188.28515265)
\lineto(334.44921761,188.49609015)
\curveto(334.44920724,189.48826816)(334.12108257,190.2538924)(333.46484261,190.79296515)
\curveto(332.81639638,191.33982881)(331.90233479,191.61326604)(330.72265511,191.61327765)
\curveto(329.97264922,191.61326604)(329.2421812,191.52342238)(328.53124886,191.3437464)
\curveto(327.82030762,191.16404774)(327.13671455,190.89451676)(326.48046761,190.53515265)
\lineto(326.48046761,192.52734015)
\curveto(327.26952692,192.83201482)(328.03515116,193.05857709)(328.77734261,193.20702765)
\curveto(329.51952467,193.36326429)(330.2421802,193.44138921)(330.94531136,193.44140265)
\curveto(332.8437401,193.44138921)(334.26170743,192.9492022)(335.19921761,191.96484015)
\curveto(336.13670555,190.98045417)(336.60545509,189.48826816)(336.60546761,187.48827765)
}
}
{
\newrgbcolor{curcolor}{0 0 0}
\pscustom[linestyle=none,fillstyle=solid,fillcolor=curcolor]
{
\newpath
\moveto(340.83593636,185.1796839)
\lineto(340.83593636,193.1249964)
\lineto(342.99218636,193.1249964)
\lineto(342.99218636,185.26171515)
\curveto(342.99218217,184.01952363)(343.23436943,183.08593082)(343.71874886,182.4609339)
\curveto(344.20311846,181.84374456)(344.92968023,181.53515112)(345.89843636,181.53515265)
\curveto(347.0624906,181.53515112)(347.98045843,181.9062445)(348.65234261,182.6484339)
\curveto(349.33201958,183.39061801)(349.67186299,184.40233575)(349.67187386,185.68359015)
\lineto(349.67187386,193.1249964)
\lineto(351.82812386,193.1249964)
\lineto(351.82812386,179.9999964)
\lineto(349.67187386,179.9999964)
\lineto(349.67187386,182.0156214)
\curveto(349.14842601,181.21874518)(348.53905162,180.62499578)(347.84374886,180.2343714)
\curveto(347.1562405,179.85155905)(346.35546005,179.66015299)(345.44140511,179.66015265)
\curveto(343.93358748,179.66015299)(342.78905737,180.12890252)(342.00781136,181.06640265)
\curveto(341.22655893,182.00390065)(340.83593432,183.37499303)(340.83593636,185.1796839)
\moveto(346.26171761,193.44140265)
\lineto(346.26171761,193.44140265)
}
}
{
\newrgbcolor{curcolor}{1 1 1}
\pscustom[linestyle=none,fillstyle=solid,fillcolor=curcolor]
{
\newpath
\moveto(254.69320374,150.5329859)
\lineto(315.21460801,150.5329859)
\curveto(323.33513314,150.5329859)(329.8725956,143.99552344)(329.8725956,135.87499831)
\lineto(329.8725956,134.83607695)
\curveto(329.8725956,126.71555183)(323.33513314,120.17808936)(315.21460801,120.17808936)
\lineto(254.69320374,120.17808936)
\curveto(246.57267862,120.17808936)(240.03521615,126.71555183)(240.03521615,134.83607695)
\lineto(240.03521615,135.87499831)
\curveto(240.03521615,143.99552344)(246.57267862,150.5329859)(254.69320374,150.5329859)
\closepath
}
}
{
\newrgbcolor{curcolor}{0 0 0}
\pscustom[linewidth=2,linecolor=curcolor]
{
\newpath
\moveto(254.69320374,150.5329859)
\lineto(315.21460801,150.5329859)
\curveto(323.33513314,150.5329859)(329.8725956,143.99552344)(329.8725956,135.87499831)
\lineto(329.8725956,134.83607695)
\curveto(329.8725956,126.71555183)(323.33513314,120.17808936)(315.21460801,120.17808936)
\lineto(254.69320374,120.17808936)
\curveto(246.57267862,120.17808936)(240.03521615,126.71555183)(240.03521615,134.83607695)
\lineto(240.03521615,135.87499831)
\curveto(240.03521615,143.99552344)(246.57267862,150.5329859)(254.69320374,150.5329859)
\closepath
}
}
{
\newrgbcolor{curcolor}{0 0 0}
\pscustom[linestyle=none,fillstyle=solid,fillcolor=curcolor]
{
\newpath
\moveto(268.20312386,145.1640589)
\lineto(264.99218636,136.45702765)
\lineto(271.42578011,136.45702765)
\lineto(268.20312386,145.1640589)
\moveto(266.86718636,147.49609015)
\lineto(269.55078011,147.49609015)
\lineto(276.21874886,129.9999964)
\lineto(273.75781136,129.9999964)
\lineto(272.16406136,134.48827765)
\lineto(264.27734261,134.48827765)
\lineto(262.68359261,129.9999964)
\lineto(260.18749886,129.9999964)
\lineto(266.86718636,147.49609015)
}
}
{
\newrgbcolor{curcolor}{0 0 0}
\pscustom[linestyle=none,fillstyle=solid,fillcolor=curcolor]
{
\newpath
\moveto(278.66796761,143.1249964)
\lineto(280.82421761,143.1249964)
\lineto(280.82421761,129.9999964)
\lineto(278.66796761,129.9999964)
\lineto(278.66796761,143.1249964)
\moveto(278.66796761,148.2343714)
\lineto(280.82421761,148.2343714)
\lineto(280.82421761,145.50390265)
\lineto(278.66796761,145.50390265)
\lineto(278.66796761,148.2343714)
}
}
{
\newrgbcolor{curcolor}{0 0 0}
\pscustom[linestyle=none,fillstyle=solid,fillcolor=curcolor]
{
\newpath
\moveto(293.96093636,141.1328089)
\lineto(293.96093636,148.2343714)
\lineto(296.11718636,148.2343714)
\lineto(296.11718636,129.9999964)
\lineto(293.96093636,129.9999964)
\lineto(293.96093636,131.9687464)
\curveto(293.50780092,131.18749521)(292.93358274,130.60546455)(292.23828011,130.22265265)
\curveto(291.55077162,129.8476528)(290.72264745,129.66015299)(289.75390511,129.66015265)
\curveto(288.16796251,129.66015299)(286.87499505,130.29296486)(285.87499886,131.55859015)
\curveto(284.88280954,132.82421233)(284.38671629,134.48827316)(284.38671761,136.55077765)
\curveto(284.38671629,138.61326904)(284.88280954,140.27732988)(285.87499886,141.54296515)
\curveto(286.87499505,142.80857734)(288.16796251,143.44138921)(289.75390511,143.44140265)
\curveto(290.72264745,143.44138921)(291.55077162,143.24998315)(292.23828011,142.8671839)
\curveto(292.93358274,142.49217141)(293.50780092,141.91404699)(293.96093636,141.1328089)
\moveto(286.61328011,136.55077765)
\curveto(286.61327656,134.96483519)(286.93749499,133.71874268)(287.58593636,132.8124964)
\curveto(288.24218118,131.91405699)(289.14061779,131.46483869)(290.28124886,131.46484015)
\curveto(291.4218655,131.46483869)(292.32030211,131.91405699)(292.97656136,132.8124964)
\curveto(293.63280079,133.71874268)(293.96092546,134.96483519)(293.96093636,136.55077765)
\curveto(293.96092546,138.13670702)(293.63280079,139.37889327)(292.97656136,140.27734015)
\curveto(292.32030211,141.18357897)(291.4218655,141.63670352)(290.28124886,141.63671515)
\curveto(289.14061779,141.63670352)(288.24218118,141.18357897)(287.58593636,140.27734015)
\curveto(286.93749499,139.37889327)(286.61327656,138.13670702)(286.61328011,136.55077765)
}
}
{
\newrgbcolor{curcolor}{0 0 0}
\pscustom[linestyle=none,fillstyle=solid,fillcolor=curcolor]
{
\newpath
\moveto(311.78515511,137.1015589)
\lineto(311.78515511,136.0468714)
\lineto(301.87109261,136.0468714)
\curveto(301.96483895,134.56249184)(302.410151,133.42968047)(303.20703011,132.6484339)
\curveto(304.0117119,131.87499453)(305.12889828,131.48827616)(306.55859261,131.48827765)
\curveto(307.38670852,131.48827616)(308.18748897,131.58983856)(308.96093636,131.79296515)
\curveto(309.74217492,131.99608816)(310.51561164,132.30077535)(311.28124886,132.70702765)
\lineto(311.28124886,130.66796515)
\curveto(310.50779915,130.33983981)(309.7148312,130.08984006)(308.90234261,129.91796515)
\curveto(308.08983282,129.74609041)(307.26561489,129.66015299)(306.42968636,129.66015265)
\curveto(304.33593032,129.66015299)(302.67577573,130.26952738)(301.44921761,131.48827765)
\curveto(300.23046568,132.70702495)(299.62109129,134.3554608)(299.62109261,136.43359015)
\curveto(299.62109129,138.58201907)(300.19921571,140.28514237)(301.35546761,141.54296515)
\curveto(302.51952589,142.80857734)(304.08593057,143.44138921)(306.05468636,143.44140265)
\curveto(307.82030184,143.44138921)(309.2148317,142.87107728)(310.23828011,141.73046515)
\curveto(311.26951714,140.59764205)(311.78514162,139.05467485)(311.78515511,137.1015589)
\moveto(309.62890511,137.7343714)
\curveto(309.6132688,138.91404999)(309.28123788,139.8554553)(308.63281136,140.55859015)
\curveto(307.99217667,141.26170389)(307.14061502,141.61326604)(306.07812386,141.61327765)
\curveto(304.87499229,141.61326604)(303.9101495,141.27342263)(303.18359261,140.5937464)
\curveto(302.46483845,139.91404899)(302.05077636,138.9570187)(301.94140511,137.72265265)
\lineto(309.62890511,137.7343714)
}
}
{
\newrgbcolor{curcolor}{0 0 0}
\pscustom[linestyle=none,fillstyle=solid,fillcolor=curcolor,opacity=0.11935484]
{
\newpath
\moveto(168.69516545,59.81200812)
\lineto(341.30483228,59.81200812)
\curveto(346.12195457,59.81200812)(349.99999886,55.93396382)(349.99999886,51.11684153)
\lineto(349.99999886,38.69517634)
\curveto(349.99999886,33.87805405)(346.12195457,30.00000975)(341.30483228,30.00000975)
\lineto(168.69516545,30.00000975)
\curveto(163.87804316,30.00000975)(159.99999886,33.87805405)(159.99999886,38.69517634)
\lineto(159.99999886,51.11684153)
\curveto(159.99999886,55.93396382)(163.87804316,59.81200812)(168.69516545,59.81200812)
\closepath
}
}
{
\newrgbcolor{curcolor}{0 0 0}
\pscustom[linewidth=1.87494767,linecolor=curcolor]
{
\newpath
\moveto(168.69516545,59.81200812)
\lineto(341.30483228,59.81200812)
\curveto(346.12195457,59.81200812)(349.99999886,55.93396382)(349.99999886,51.11684153)
\lineto(349.99999886,38.69517634)
\curveto(349.99999886,33.87805405)(346.12195457,30.00000975)(341.30483228,30.00000975)
\lineto(168.69516545,30.00000975)
\curveto(163.87804316,30.00000975)(159.99999886,33.87805405)(159.99999886,38.69517634)
\lineto(159.99999886,51.11684153)
\curveto(159.99999886,55.93396382)(163.87804316,59.81200812)(168.69516545,59.81200812)
\closepath
}
}
{
\newrgbcolor{curcolor}{0 0 0}
\pscustom[linestyle=none,fillstyle=solid,fillcolor=curcolor]
{
\newpath
\moveto(43.96874886,512.6562464)
\lineto(49.12499886,512.6562464)
\lineto(49.12499886,530.4531214)
\lineto(43.51562386,529.3281214)
\lineto(43.51562386,532.2031214)
\lineto(49.09374886,533.3281214)
\lineto(52.24999886,533.3281214)
\lineto(52.24999886,512.6562464)
\lineto(57.40624886,512.6562464)
\lineto(57.40624886,509.9999964)
\lineto(43.96874886,509.9999964)
\lineto(43.96874886,512.6562464)
}
}
{
\newrgbcolor{curcolor}{0 0 0}
\pscustom[linestyle=none,fillstyle=solid,fillcolor=curcolor]
{
\newpath
\moveto(46.14062386,382.6562464)
\lineto(57.15624886,382.6562464)
\lineto(57.15624886,379.9999964)
\lineto(42.34374886,379.9999964)
\lineto(42.34374886,382.6562464)
\curveto(43.54166199,383.89582584)(45.17186869,385.55728251)(47.23437386,387.6406214)
\curveto(49.30728122,389.73436167)(50.60936325,391.08331865)(51.14062386,391.6874964)
\curveto(52.15102838,392.82290025)(52.85415268,393.78123262)(53.24999886,394.5624964)
\curveto(53.65623521,395.35414771)(53.85936,396.13018861)(53.85937386,396.8906214)
\curveto(53.85936,398.13018661)(53.42186044,399.14060226)(52.54687386,399.9218714)
\curveto(51.68227885,400.7031007)(50.55207164,401.09372531)(49.15624886,401.0937464)
\curveto(48.16665736,401.09372531)(47.11978341,400.92185048)(46.01562386,400.5781214)
\curveto(44.92186894,400.23435117)(43.74999511,399.71351836)(42.49999886,399.0156214)
\lineto(42.49999886,402.2031214)
\curveto(43.77082843,402.71351536)(44.95832724,403.09893164)(46.06249886,403.3593714)
\curveto(47.16665836,403.61976445)(48.17707402,403.74997265)(49.09374886,403.7499964)
\curveto(51.51040402,403.74997265)(53.43748543,403.14580659)(54.87499886,401.9374964)
\curveto(56.31248255,400.72914234)(57.03123183,399.11456062)(57.03124886,397.0937464)
\curveto(57.03123183,396.13539693)(56.84894035,395.22393951)(56.48437386,394.3593714)
\curveto(56.13019107,393.50519123)(55.47915005,392.49477557)(54.53124886,391.3281214)
\curveto(54.27081793,391.02602704)(53.44269375,390.15102792)(52.04687386,388.7031214)
\curveto(50.65102988,387.26561414)(48.68228185,385.24999115)(46.14062386,382.6562464)
}
}
{
\newrgbcolor{curcolor}{0 0 0}
\pscustom[linestyle=none,fillstyle=solid,fillcolor=curcolor]
{
\newpath
\moveto(512.98437386,342.5781214)
\curveto(514.49477604,342.25519248)(515.67185819,341.58331815)(516.51562386,340.5624964)
\curveto(517.36977316,339.54165353)(517.79685607,338.28123812)(517.79687386,336.7812464)
\curveto(517.79685607,334.47915859)(517.00519019,332.69791037)(515.42187386,331.4374964)
\curveto(513.83852669,330.17707956)(511.58852894,329.54687186)(508.67187386,329.5468714)
\curveto(507.6926995,329.54687186)(506.68228385,329.64583009)(505.64062386,329.8437464)
\curveto(504.60936925,330.03124637)(503.54166199,330.31770442)(502.43749886,330.7031214)
\lineto(502.43749886,333.7499964)
\curveto(503.31249555,333.2395765)(504.27082793,332.85416021)(505.31249886,332.5937464)
\curveto(506.35415918,332.3333274)(507.44269975,332.2031192)(508.57812386,332.2031214)
\curveto(510.55727997,332.2031192)(512.0624868,332.59374381)(513.09374886,333.3749964)
\curveto(514.13540139,334.15624225)(514.65623421,335.29165778)(514.65624886,336.7812464)
\curveto(514.65623421,338.15623825)(514.17185969,339.22915384)(513.20312386,339.9999964)
\curveto(512.24477829,340.78123562)(510.90623796,341.17186023)(509.18749886,341.1718714)
\lineto(506.46874886,341.1718714)
\lineto(506.46874886,343.7656214)
\lineto(509.31249886,343.7656214)
\curveto(510.86457133,343.76560764)(512.05207014,344.072899)(512.87499886,344.6874964)
\curveto(513.69790183,345.31248109)(514.10935975,346.20831353)(514.10937386,347.3749964)
\curveto(514.10935975,348.5728945)(513.68227685,349.48956025)(512.82812386,350.1249964)
\curveto(511.98436188,350.77080896)(510.77082143,351.09372531)(509.18749886,351.0937464)
\curveto(508.32290721,351.09372531)(507.3958248,350.9999754)(506.40624886,350.8124964)
\curveto(505.41666011,350.62497578)(504.32811954,350.3333094)(503.14062386,349.9374964)
\lineto(503.14062386,352.7499964)
\curveto(504.33853619,353.08330665)(505.45832674,353.3333064)(506.49999886,353.4999964)
\curveto(507.55207464,353.6666394)(508.54165699,353.74997265)(509.46874886,353.7499964)
\curveto(511.86457033,353.74997265)(513.76040177,353.2030982)(515.15624886,352.1093714)
\curveto(516.55206564,351.02601704)(517.24998161,349.55726851)(517.24999886,347.7031214)
\curveto(517.24998161,346.41143832)(516.88019032,345.31768942)(516.14062386,344.4218714)
\curveto(515.40102513,343.5364412)(514.34894285,342.92185848)(512.98437386,342.5781214)
}
}
{
\newrgbcolor{curcolor}{0 0 0}
\pscustom[linestyle=none,fillstyle=solid,fillcolor=curcolor]
{
\newpath
\moveto(512.09374886,300.5781214)
\lineto(504.12499886,288.1249964)
\lineto(512.09374886,288.1249964)
\lineto(512.09374886,300.5781214)
\moveto(511.26562386,303.3281214)
\lineto(515.23437386,303.3281214)
\lineto(515.23437386,288.1249964)
\lineto(518.56249886,288.1249964)
\lineto(518.56249886,285.4999964)
\lineto(515.23437386,285.4999964)
\lineto(515.23437386,279.9999964)
\lineto(512.09374886,279.9999964)
\lineto(512.09374886,285.4999964)
\lineto(501.56249886,285.4999964)
\lineto(501.56249886,288.5468714)
\lineto(511.26562386,303.3281214)
}
}
{
\newrgbcolor{curcolor}{0 0 0}
\pscustom[linestyle=none,fillstyle=solid,fillcolor=curcolor]
{
\newpath
\moveto(503.45312386,253.3281214)
\lineto(515.84374886,253.3281214)
\lineto(515.84374886,250.6718714)
\lineto(506.34374886,250.6718714)
\lineto(506.34374886,244.9531214)
\curveto(506.80207539,245.10935629)(507.26040827,245.22393951)(507.71874886,245.2968714)
\curveto(508.17707402,245.38018936)(508.63540689,245.42185598)(509.09374886,245.4218714)
\curveto(511.69790383,245.42185598)(513.76040177,244.70831503)(515.28124886,243.2812464)
\curveto(516.80206539,241.85415121)(517.5624813,239.92186148)(517.56249886,237.4843714)
\curveto(517.5624813,234.97394976)(516.78123208,233.02082671)(515.21874886,231.6249964)
\curveto(513.65623521,230.2395795)(511.45311241,229.54687186)(508.60937386,229.5468714)
\curveto(507.63019957,229.54687186)(506.63020057,229.63020511)(505.60937386,229.7968714)
\curveto(504.5989526,229.96353811)(503.55207864,230.21353786)(502.46874886,230.5468714)
\lineto(502.46874886,233.7187464)
\curveto(503.40624546,233.20832653)(504.37499449,232.82811857)(505.37499886,232.5781214)
\curveto(506.37499249,232.32811907)(507.4322831,232.2031192)(508.54687386,232.2031214)
\curveto(510.34894685,232.2031192)(511.77602875,232.67707706)(512.82812386,233.6249964)
\curveto(513.88019332,234.5729085)(514.40623446,235.85936554)(514.40624886,237.4843714)
\curveto(514.40623446,239.10936229)(513.88019332,240.39581934)(512.82812386,241.3437464)
\curveto(511.77602875,242.29165078)(510.34894685,242.76560864)(508.54687386,242.7656214)
\curveto(507.70311616,242.76560864)(506.859367,242.67185873)(506.01562386,242.4843714)
\curveto(505.18228535,242.29685911)(504.32811954,242.00519273)(503.45312386,241.6093714)
\lineto(503.45312386,253.3281214)
}
}
{
\newrgbcolor{curcolor}{0 0 0}
\pscustom[linestyle=none,fillstyle=solid,fillcolor=curcolor]
{
\newpath
\moveto(510.56249886,192.9218714)
\curveto(509.14582305,192.92185848)(508.02082418,192.43748396)(507.18749886,191.4687464)
\curveto(506.36457583,190.4999859)(505.95311791,189.17186223)(505.95312386,187.4843714)
\curveto(505.95311791,185.80728226)(506.36457583,184.47915859)(507.18749886,183.4999964)
\curveto(508.02082418,182.53124387)(509.14582305,182.04686936)(510.56249886,182.0468714)
\curveto(511.97915355,182.04686936)(513.0989441,182.53124387)(513.92187386,183.4999964)
\curveto(514.75519244,184.47915859)(515.17185869,185.80728226)(515.17187386,187.4843714)
\curveto(515.17185869,189.17186223)(514.75519244,190.4999859)(513.92187386,191.4687464)
\curveto(513.0989441,192.43748396)(511.97915355,192.92185848)(510.56249886,192.9218714)
\moveto(516.82812386,202.8124964)
\lineto(516.82812386,199.9374964)
\curveto(516.03644116,200.31247609)(515.23435863,200.59893414)(514.42187386,200.7968714)
\curveto(513.61977691,200.99476707)(512.82290271,201.09372531)(512.03124886,201.0937464)
\curveto(509.94790558,201.09372531)(508.35415718,200.39060101)(507.24999886,198.9843714)
\curveto(506.15624271,197.57810382)(505.53124333,195.45310595)(505.37499886,192.6093714)
\curveto(505.98957621,193.51560789)(506.76040877,194.20831553)(507.68749886,194.6874964)
\curveto(508.61457358,195.17706456)(509.63540589,195.42185598)(510.74999886,195.4218714)
\curveto(513.09373577,195.42185598)(514.94269225,194.70831503)(516.29687386,193.2812464)
\curveto(517.66143954,191.86456787)(518.34373052,189.93227814)(518.34374886,187.4843714)
\curveto(518.34373052,185.08853298)(517.63539789,183.1666599)(516.21874886,181.7187464)
\curveto(514.80206739,180.27082946)(512.91665261,179.54687186)(510.56249886,179.5468714)
\curveto(507.86457433,179.54687186)(505.80207639,180.57812082)(504.37499886,182.6406214)
\curveto(502.94791258,184.71353336)(502.23437163,187.71353036)(502.23437386,191.6406214)
\curveto(502.23437163,195.32810607)(503.10937075,198.26560314)(504.85937386,200.4531214)
\curveto(506.60936725,202.65101542)(508.95832324,203.74997265)(511.90624886,203.7499964)
\curveto(512.69790283,203.74997265)(513.49477704,203.67184773)(514.29687386,203.5156214)
\curveto(515.10935875,203.35934804)(515.95310791,203.12497328)(516.82812386,202.8124964)
}
}
{
\newrgbcolor{curcolor}{0 0 0}
\pscustom[linestyle=none,fillstyle=solid,fillcolor=curcolor]
{
\newpath
\moveto(42.62499886,153.3281214)
\lineto(57.62499886,153.3281214)
\lineto(57.62499886,151.9843714)
\lineto(49.15624886,129.9999964)
\lineto(45.85937386,129.9999964)
\lineto(53.82812386,150.6718714)
\lineto(42.62499886,150.6718714)
\lineto(42.62499886,153.3281214)
}
}
{
\newrgbcolor{curcolor}{0 0 0}
\pscustom[linestyle=none,fillstyle=solid,fillcolor=curcolor]
{
\newpath
\moveto(50.17187386,51.0781214)
\curveto(48.67186519,51.07811032)(47.48957471,50.67706906)(46.62499886,49.8749964)
\curveto(45.77082643,49.072904)(45.34374352,47.96873843)(45.34374886,46.5624964)
\curveto(45.34374352,45.15624125)(45.77082643,44.05207568)(46.62499886,43.2499964)
\curveto(47.48957471,42.44791062)(48.67186519,42.04686936)(50.17187386,42.0468714)
\curveto(51.67186219,42.04686936)(52.85415268,42.44791062)(53.71874886,43.2499964)
\curveto(54.58331761,44.06249234)(55.01560885,45.1666579)(55.01562386,46.5624964)
\curveto(55.01560885,47.96873843)(54.58331761,49.072904)(53.71874886,49.8749964)
\curveto(52.86456933,50.67706906)(51.68227885,51.07811032)(50.17187386,51.0781214)
\moveto(47.01562386,52.4218714)
\curveto(45.66145154,52.75519198)(44.60416093,53.38539968)(43.84374886,54.3124964)
\curveto(43.09374577,55.2395645)(42.71874614,56.3697717)(42.71874886,57.7031214)
\curveto(42.71874614,59.56768517)(43.38020382,61.04164203)(44.70312386,62.1249964)
\curveto(46.03645116,63.20830653)(47.859366,63.74997265)(50.17187386,63.7499964)
\curveto(52.49477804,63.74997265)(54.31769288,63.20830653)(55.64062386,62.1249964)
\curveto(56.96352357,61.04164203)(57.62498124,59.56768517)(57.62499886,57.7031214)
\curveto(57.62498124,56.3697717)(57.24477329,55.2395645)(56.48437386,54.3124964)
\curveto(55.73435813,53.38539968)(54.68748418,52.75519198)(53.34374886,52.4218714)
\curveto(54.86456733,52.06769267)(56.04685782,51.37498503)(56.89062386,50.3437464)
\curveto(57.74477279,49.31248709)(58.17185569,48.05207168)(58.17187386,46.5624964)
\curveto(58.17185569,44.30207543)(57.47914805,42.56770217)(56.09374886,41.3593714)
\curveto(54.71873414,40.15103792)(52.74477779,39.54687186)(50.17187386,39.5468714)
\curveto(47.5989496,39.54687186)(45.61978491,40.15103792)(44.23437386,41.3593714)
\curveto(42.859371,42.56770217)(42.17187169,44.30207543)(42.17187386,46.5624964)
\curveto(42.17187169,48.05207168)(42.5989546,49.31248709)(43.45312386,50.3437464)
\curveto(44.30728622,51.37498503)(45.49478504,52.06769267)(47.01562386,52.4218714)
\moveto(45.85937386,57.4062464)
\curveto(45.859368,56.19789687)(46.23436763,55.25518948)(46.98437386,54.5781214)
\curveto(47.74478279,53.90102417)(48.80728172,53.56248284)(50.17187386,53.5624964)
\curveto(51.526029,53.56248284)(52.58331961,53.90102417)(53.34374886,54.5781214)
\curveto(54.11456808,55.25518948)(54.49998436,56.19789687)(54.49999886,57.4062464)
\curveto(54.49998436,58.61456112)(54.11456808,59.55726851)(53.34374886,60.2343714)
\curveto(52.58331961,60.91143382)(51.526029,61.24997515)(50.17187386,61.2499964)
\curveto(48.80728172,61.24997515)(47.74478279,60.91143382)(46.98437386,60.2343714)
\curveto(46.23436763,59.55726851)(45.859368,58.61456112)(45.85937386,57.4062464)
}
}
{
\newrgbcolor{curcolor}{0 0 0}
\pscustom[linewidth=2,linecolor=curcolor,linestyle=dashed,dash=8 8]
{
\newpath
\moveto(390.091779,339.999999)
\lineto(499.999999,339.999999)
}
}
{
\newrgbcolor{curcolor}{0 0 0}
\pscustom[linestyle=none,fillstyle=solid,fillcolor=curcolor]
{
\newpath
\moveto(400.55408436,335.15951676)
\lineto(387.4430316,339.98078309)
\lineto(400.55408508,344.80204836)
\curveto(398.459487,341.95557528)(398.47155612,338.06110608)(400.55408436,335.15951676)
\lineto(400.55408436,335.15951676)
\closepath
}
}
{
\newrgbcolor{curcolor}{0 0 0}
\pscustom[linewidth=2,linecolor=curcolor,linestyle=dashed,dash=8 8]
{
\newpath
\moveto(322.129899,289.999819)
\lineto(500.262979,289.999819)
}
}
{
\newrgbcolor{curcolor}{0 0 0}
\pscustom[linestyle=none,fillstyle=solid,fillcolor=curcolor]
{
\newpath
\moveto(332.59220436,285.15933676)
\lineto(319.4811516,289.98060309)
\lineto(332.59220508,294.80186836)
\curveto(330.497607,291.95539528)(330.50967612,288.06092608)(332.59220436,285.15933676)
\lineto(332.59220436,285.15933676)
\closepath
}
}
{
\newrgbcolor{curcolor}{0 0 0}
\pscustom[linewidth=2,linecolor=curcolor,linestyle=dashed,dash=8 8]
{
\newpath
\moveto(350.207469,189.999789)
\lineto(500.322879,189.999789)
}
}
{
\newrgbcolor{curcolor}{0 0 0}
\pscustom[linestyle=none,fillstyle=solid,fillcolor=curcolor]
{
\newpath
\moveto(360.66977436,185.15930676)
\lineto(347.5587216,189.98057309)
\lineto(360.66977508,194.80183836)
\curveto(358.575177,191.95536528)(358.58724612,188.06089608)(360.66977436,185.15930676)
\lineto(360.66977436,185.15930676)
\closepath
}
}
{
\newrgbcolor{curcolor}{0 0 0}
\pscustom[linewidth=2,linecolor=curcolor,linestyle=dashed,dash=8 8]
{
\newpath
\moveto(370.862939,239.998149)
\lineto(501.401259,239.833959)
}
}
{
\newrgbcolor{curcolor}{0 0 0}
\pscustom[linestyle=none,fillstyle=solid,fillcolor=curcolor]
{
\newpath
\moveto(381.31914777,235.1445112)
\lineto(368.21416953,239.98226467)
\lineto(381.33127678,244.78703517)
\curveto(379.23310009,241.94319891)(379.24027077,238.04871761)(381.31914777,235.1445112)
\lineto(381.31914777,235.1445112)
\closepath
}
}
{
\newrgbcolor{curcolor}{0 0 0}
\pscustom[linewidth=2,linecolor=curcolor,linestyle=dashed,dash=8 8]
{
\newpath
\moveto(210.624469,389.999999)
\lineto(60,389.999999)
}
}
{
\newrgbcolor{curcolor}{0 0 0}
\pscustom[linestyle=none,fillstyle=solid,fillcolor=curcolor]
{
\newpath
\moveto(200.16216364,394.84048124)
\lineto(213.2732164,390.01921491)
\lineto(200.16216292,385.19794964)
\curveto(202.256761,388.04442272)(202.24469188,391.93889192)(200.16216364,394.84048124)
\lineto(200.16216364,394.84048124)
\closepath
}
}
{
\newrgbcolor{curcolor}{0 0 0}
\pscustom[linewidth=2,linecolor=curcolor,linestyle=dashed,dash=8 8]
{
\newpath
\moveto(260,140)
\lineto(60,140)
}
}
{
\newrgbcolor{curcolor}{0 0 0}
\pscustom[linestyle=none,fillstyle=solid,fillcolor=curcolor]
{
\newpath
\moveto(249.53769464,144.84048224)
\lineto(262.6487474,140.01921591)
\lineto(249.53769392,135.19795064)
\curveto(251.632292,138.04442372)(251.62022288,141.93889292)(249.53769464,144.84048224)
\lineto(249.53769464,144.84048224)
\closepath
}
}
{
\newrgbcolor{curcolor}{0 0 0}
\pscustom[linewidth=2,linecolor=curcolor,linestyle=dashed,dash=8 8]
{
\newpath
\moveto(190,50)
\lineto(60,50)
}
}
{
\newrgbcolor{curcolor}{0 0 0}
\pscustom[linestyle=none,fillstyle=solid,fillcolor=curcolor]
{
\newpath
\moveto(179.53769464,54.84048224)
\lineto(192.6487474,50.01921591)
\lineto(179.53769392,45.19795064)
\curveto(181.632292,48.04442372)(181.62022288,51.93889292)(179.53769464,54.84048224)
\lineto(179.53769464,54.84048224)
\closepath
}
}
{
\newrgbcolor{curcolor}{0 0 0}
\pscustom[linewidth=2,linecolor=curcolor,linestyle=dashed,dash=8 8]
{
\newpath
\moveto(60,520)
\lineto(160,520)
}
}
{
\newrgbcolor{curcolor}{0 0 0}
\pscustom[linestyle=none,fillstyle=solid,fillcolor=curcolor]
{
\newpath
\moveto(149.53769464,524.84048224)
\lineto(162.6487474,520.01921591)
\lineto(149.53769392,515.19795064)
\curveto(151.632292,518.04442372)(151.62022288,521.93889292)(149.53769464,524.84048224)
\closepath
}
}
{
\newrgbcolor{curcolor}{1 1 1}
\pscustom[linestyle=none,fillstyle=solid,fillcolor=curcolor]
{
\newpath
\moveto(369.99999886,59.9999964)
\lineto(389.99999886,59.9999964)
\curveto(395.53999886,59.9999964)(399.99999886,55.5399964)(399.99999886,49.9999964)
\lineto(399.99999886,39.9999964)
\curveto(399.99999886,34.4599964)(395.53999886,29.9999964)(389.99999886,29.9999964)
\lineto(369.99999886,29.9999964)
\curveto(364.45999886,29.9999964)(359.99999886,34.4599964)(359.99999886,39.9999964)
\lineto(359.99999886,49.9999964)
\curveto(359.99999886,55.5399964)(364.45999886,59.9999964)(369.99999886,59.9999964)
\closepath
}
}
{
\newrgbcolor{curcolor}{0 0 0}
\pscustom[linewidth=2,linecolor=curcolor]
{
\newpath
\moveto(369.99999886,59.9999964)
\lineto(389.99999886,59.9999964)
\curveto(395.53999886,59.9999964)(399.99999886,55.5399964)(399.99999886,49.9999964)
\lineto(399.99999886,39.9999964)
\curveto(399.99999886,34.4599964)(395.53999886,29.9999964)(389.99999886,29.9999964)
\lineto(369.99999886,29.9999964)
\curveto(364.45999886,29.9999964)(359.99999886,34.4599964)(359.99999886,39.9999964)
\lineto(359.99999886,49.9999964)
\curveto(359.99999886,55.5399964)(364.45999886,59.9999964)(369.99999886,59.9999964)
\closepath
}
}
{
\newrgbcolor{curcolor}{0 0 0}
\pscustom[linestyle=none,fillstyle=solid,fillcolor=curcolor]
{
\newpath
\moveto(380.59091073,54.92437384)
\lineto(380.59091073,46.22124884)
\lineto(389.29403573,46.22124884)
\lineto(389.29403573,43.56499884)
\lineto(380.59091073,43.56499884)
\lineto(380.59091073,34.86187384)
\lineto(377.96591073,34.86187384)
\lineto(377.96591073,43.56499884)
\lineto(369.26278573,43.56499884)
\lineto(369.26278573,46.22124884)
\lineto(377.96591073,46.22124884)
\lineto(377.96591073,54.92437384)
\lineto(380.59091073,54.92437384)
}
}
\end{pspicture}

		\end{center}

		\begin{enumerate}
		  \item Fond d'écran (présent dans l'archive stocké sur le téléphone).
		  \item Bouton ``\hyperlink{Creer partie solo}{Jouer en solo}''
		  \item Bouton ``\hyperlink{Connexion multi-joueurs}{Jouer en multi-joueurs}''
		  \item Bouton ``\hyperlink{Options}{Options}''
		  \item Bouton ``\hyperlink{Statistiques}{Mes statistiques}''
		  \item Bouton ``\hyperlink{Creer niveau}{Créer niveau}''
		  \item Bouton ``\hyperlink{Aide}{Aide}''
		  \item Liste déroulante ``Comptes hors lignes''
		  \item Bouton ``\hyperlink{profil}{+}'' 
		\end{enumerate}

		\subsubsection{Description des zones}
		
		\begin{tabular}{|c|c|c|c|c|} \hline
			Numéro de zone & Type  & Description & Evènement &	Règle \\\hline 
			2 & Bouton & Affiche l'écran de création d'un partie solitaire & Cliqué & RG2-01 \\\hline
			3 & Bouton & Affiche l'écran de connexion multi-joueurs & Cliqué & RG2-02 \\\hline
			4 & Bouton & Affiche l'écran des options & Cliqué & RG2-03 \\\hline
			5 & Bouton & Affiche l'écran des statistiques & Cliqué & RG2-04 \\\hline
			6 & Bouton & Affiche l'écran de création de niveaux & Cliqué & RG2-05 \\\hline
			7 & Bouton & Affiche l'écran d'aide & Cliqué & RG2-06 \\\hline
			8 & Liste & Permet de sélectionner le compte hors ligne voulu & Cliqué & RG2-07 \\ 
			  & déroulante & Affiche par défaut le compte en cours d'utilisation & & \\
			  & & Taille : ??? & & \\\hline
			9 & Bouton & Affiche l'écran de création du profil & Cliqué & RG2-09 \\\hline
			
		\end{tabular}
		
\newpage

		\subsubsection{Description des règles}
		
		\underline{RG2-01 :}
			\begin{quote}
				Charger la page de création d'une partie solitaire%
					\footnote[1]{
						\hyperlink{Creer partie solo}{Création d'une partie solitaire}
						\og voir section \ref{Creer partie solo}, page \pageref{Creer partie solo}.\fg
					}
				et l'afficher.\\
				Cacher la page d'accueil%
					\footnote[2]{
						\hyperlink{Page d'accueil}{Page d'accueil}
						\og voir section \ref{accueil}, page \pageref{accueil}.\fg
					}.\\ 
			\end{quote}


		\underline{RG2-02 :}
			\begin{quote}
				Charger la page de connexion multi-joueurs%
					\footnote[3]{
						\hyperlink{Connexion multi-joueurs}{Connexion multi-joueurs}
						\og voir section \ref{Connexion multi-joueurs}, page \pageref{Connexion multi-joueurs}.\fg
					}
				et l'afficher.\\			
				Cacher la page d'accueil%
					\footnotemark[2]
				.\\
			\end{quote}

		
		\underline{RG2-03 :}
			\begin{quote}
				Charger la page des options%
					\footnote[4]{
						\hyperlink{Options}{Options}
						\og voir section \ref{Options}, page \pageref{Options}.\fg
					}
				et l'afficher.\\			
				Cacher la page d'accueil%
					\footnotemark[2]
				.\\	
			\end{quote}


		\underline{RG2-04 :}
			\begin{quote}
				Charger la page des statistiques%
					\footnote[5]{
						\hyperlink{Statistiques}{Statistiques}
						\og voir section \ref{Statistiques}, page \pageref{Statistiques}.\fg
					}
				et l'afficher.\\				
				Cacher la page d'accueil%
					\footnotemark[2]
				.\\		
			\end{quote}


		\underline{RG2-05 :}
			\begin{quote}
				Charger la page de création de niveaux%
					\footnote[6]{
						\hyperlink{Creation de niveaux}{Création de niveaux}
						\og voir section \ref{Creation de niveaux}, page \pageref{Creation de niveaux}.\fg
					}
				et l'afficher.\\
				Cacher la page d'accueil%
					\footnotemark[2]
				.\\		
			\end{quote}


		\underline{RG2-06 :}
			\begin{quote}
				Charger la page d'aide%
					\footnote[7]{
						\hyperlink{Aide}{Aide}
						\og voir section \ref{Aide}, page \pageref{Aide}.\fg
					}
				et l'afficher.\\
				Cacher la page d'accueil%
					\footnotemark[2]	
				.\\		
			\end{quote}
			
			
		\underline{RG2-07 :}
			\begin{quote}
				Récupérer la liste des comptes hors ligne présents sur le téléphone.
				Les afficher en premier plan.	
			\end{quote}			


		\underline{RG2-08 :}
			\begin{quote}
				Charger la page de création du profil%
					\footnote[8]{
						\hyperlink{profil}{Création du profil}
						\og voir section \ref{profil}, page \pageref{profil}.\fg
					}
				et l'afficher.\\
				Cacher la page d'accueil%
					\footnotemark[2]
				.\\			
			\end{quote}

	
\newpage

	\subsection{Création d'une partie solitaire}
	
		\hypertarget{Creer partie solo}{}
		\label{Creer partie solo}

		\begin{center}
			%LaTeX with PSTricks extensions
%%Creator: inkscape 0.48.0
%%Please note this file requires PSTricks extensions
\psset{xunit=.4pt,yunit=.4pt,runit=.4pt}
\begin{pspicture}(560,600)
{
\newrgbcolor{curcolor}{1 1 1}
\pscustom[linestyle=none,fillstyle=solid,fillcolor=curcolor]
{
\newpath
\moveto(133.12401581,597.52220273)
\lineto(426.87598419,597.52220273)
\curveto(443.85397169,597.52220273)(457.52217102,583.8540034)(457.52217102,566.8760159)
\lineto(457.52217102,33.124017)
\curveto(457.52217102,16.1460295)(443.85397169,2.47783017)(426.87598419,2.47783017)
\lineto(133.12401581,2.47783017)
\curveto(116.14602831,2.47783017)(102.47782898,16.1460295)(102.47782898,33.124017)
\lineto(102.47782898,566.8760159)
\curveto(102.47782898,583.8540034)(116.14602831,597.52220273)(133.12401581,597.52220273)
\closepath
}
}
{
\newrgbcolor{curcolor}{0 0 0}
\pscustom[linewidth=4.95566034,linecolor=curcolor]
{
\newpath
\moveto(133.12401581,597.52220273)
\lineto(426.87598419,597.52220273)
\curveto(443.85397169,597.52220273)(457.52217102,583.8540034)(457.52217102,566.8760159)
\lineto(457.52217102,33.124017)
\curveto(457.52217102,16.1460295)(443.85397169,2.47783017)(426.87598419,2.47783017)
\lineto(133.12401581,2.47783017)
\curveto(116.14602831,2.47783017)(102.47782898,16.1460295)(102.47782898,33.124017)
\lineto(102.47782898,566.8760159)
\curveto(102.47782898,583.8540034)(116.14602831,597.52220273)(133.12401581,597.52220273)
\closepath
}
}
{
\newrgbcolor{curcolor}{1 1 1}
\pscustom[linestyle=none,fillstyle=solid,fillcolor=curcolor]
{
\newpath
\moveto(135.10877228,299.91494751)
\lineto(423.47055817,299.91494751)
\curveto(437.43929419,299.91494751)(448.68488312,288.66935858)(448.68488312,274.70062256)
\lineto(448.68488312,165.27957153)
\curveto(448.68488312,151.31083551)(437.43929419,140.06524658)(423.47055817,140.06524658)
\lineto(135.10877228,140.06524658)
\curveto(121.14003625,140.06524658)(109.89444733,151.31083551)(109.89444733,165.27957153)
\lineto(109.89444733,274.70062256)
\curveto(109.89444733,288.66935858)(121.14003625,299.91494751)(135.10877228,299.91494751)
\closepath
}
}
{
\newrgbcolor{curcolor}{0 0 0}
\pscustom[linewidth=2,linecolor=curcolor]
{
\newpath
\moveto(135.10877228,299.91494751)
\lineto(423.47055817,299.91494751)
\curveto(437.43929419,299.91494751)(448.68488312,288.66935858)(448.68488312,274.70062256)
\lineto(448.68488312,165.27957153)
\curveto(448.68488312,151.31083551)(437.43929419,140.06524658)(423.47055817,140.06524658)
\lineto(135.10877228,140.06524658)
\curveto(121.14003625,140.06524658)(109.89444733,151.31083551)(109.89444733,165.27957153)
\lineto(109.89444733,274.70062256)
\curveto(109.89444733,288.66935858)(121.14003625,299.91494751)(135.10877228,299.91494751)
\closepath
}
}
{
\newrgbcolor{curcolor}{0 0 0}
\pscustom[linestyle=none,fillstyle=solid,fillcolor=curcolor,opacity=0.11935484]
{
\newpath
\moveto(312.75479603,289.96047974)
\lineto(417.33941174,289.96047974)
\curveto(424.44983058,289.96047974)(430.17410278,285.79401281)(430.17410278,280.61862564)
\lineto(430.17410278,269.59705544)
\curveto(430.17410278,264.42166827)(424.44983058,260.25520134)(417.33941174,260.25520134)
\lineto(312.75479603,260.25520134)
\curveto(305.64437719,260.25520134)(299.92010498,264.42166827)(299.92010498,269.59705544)
\lineto(299.92010498,280.61862564)
\curveto(299.92010498,285.79401281)(305.64437719,289.96047974)(312.75479603,289.96047974)
\closepath
}
}
{
\newrgbcolor{curcolor}{0 0 0}
\pscustom[linewidth=2,linecolor=curcolor]
{
\newpath
\moveto(312.75479603,289.96047974)
\lineto(417.33941174,289.96047974)
\curveto(424.44983058,289.96047974)(430.17410278,285.79401281)(430.17410278,280.61862564)
\lineto(430.17410278,269.59705544)
\curveto(430.17410278,264.42166827)(424.44983058,260.25520134)(417.33941174,260.25520134)
\lineto(312.75479603,260.25520134)
\curveto(305.64437719,260.25520134)(299.92010498,264.42166827)(299.92010498,269.59705544)
\lineto(299.92010498,280.61862564)
\curveto(299.92010498,285.79401281)(305.64437719,289.96047974)(312.75479603,289.96047974)
\closepath
}
}
{
\newrgbcolor{curcolor}{1 1 1}
\pscustom[linestyle=none,fillstyle=solid,fillcolor=curcolor]
{
\newpath
\moveto(190.0634613,350.19837952)
\lineto(369.92247009,350.19837952)
\lineto(369.92247009,319.90265656)
\lineto(190.0634613,319.90265656)
\closepath
}
}
{
\newrgbcolor{curcolor}{0 0 0}
\pscustom[linewidth=1.57047963,linecolor=curcolor]
{
\newpath
\moveto(190.0634613,350.19837952)
\lineto(369.92247009,350.19837952)
\lineto(369.92247009,319.90265656)
\lineto(190.0634613,319.90265656)
\closepath
}
}
{
\newrgbcolor{curcolor}{1 1 1}
\pscustom[linestyle=none,fillstyle=solid,fillcolor=curcolor]
{
\newpath
\moveto(220.09861755,479.90138245)
\lineto(340.15664673,479.90138245)
\lineto(340.15664673,359.90190125)
\lineto(220.09861755,359.90190125)
\closepath
}
}
{
\newrgbcolor{curcolor}{0 0 0}
\pscustom[linewidth=2,linecolor=curcolor]
{
\newpath
\moveto(220.09861755,479.90138245)
\lineto(340.15664673,479.90138245)
\lineto(340.15664673,359.90190125)
\lineto(220.09861755,359.90190125)
\closepath
}
}
{
\newrgbcolor{curcolor}{1 1 1}
\pscustom[linestyle=none,fillstyle=solid,fillcolor=curcolor]
{
\newpath
\moveto(360.10479736,460.10528564)
\lineto(440.10527039,460.10528564)
\lineto(440.10527039,380.14836884)
\lineto(360.10479736,380.14836884)
\closepath
}
}
{
\newrgbcolor{curcolor}{0 0 0}
\pscustom[linewidth=2,linecolor=curcolor]
{
\newpath
\moveto(360.10479736,460.10528564)
\lineto(440.10527039,460.10528564)
\lineto(440.10527039,380.14836884)
\lineto(360.10479736,380.14836884)
\closepath
}
}
{
\newrgbcolor{curcolor}{1 1 1}
\pscustom[linestyle=none,fillstyle=solid,fillcolor=curcolor]
{
\newpath
\moveto(119.89472198,460.10528564)
\lineto(199.89520264,460.10528564)
\lineto(199.89520264,380.14836884)
\lineto(119.89472198,380.14836884)
\closepath
}
}
{
\newrgbcolor{curcolor}{0 0 0}
\pscustom[linewidth=2,linecolor=curcolor]
{
\newpath
\moveto(119.89472198,460.10528564)
\lineto(199.89520264,460.10528564)
\lineto(199.89520264,380.14836884)
\lineto(119.89472198,380.14836884)
\closepath
}
}
{
\newrgbcolor{curcolor}{0 0 0}
\pscustom[linestyle=none,fillstyle=solid,fillcolor=curcolor]
{
\newpath
\moveto(119.93554688,286.03808594)
\lineto(133.50292969,286.03808594)
\lineto(133.50292969,284.21191406)
\lineto(127.80957031,284.21191406)
\lineto(127.80957031,270)
\lineto(125.62890625,270)
\lineto(125.62890625,284.21191406)
\lineto(119.93554688,284.21191406)
\lineto(119.93554688,286.03808594)
}
}
{
\newrgbcolor{curcolor}{0 0 0}
\pscustom[linestyle=none,fillstyle=solid,fillcolor=curcolor]
{
\newpath
\moveto(137.09082031,268.8828125)
\curveto(136.53222004,267.45052338)(135.98794975,266.515954)(135.45800781,266.07910156)
\curveto(134.92805498,265.64225696)(134.21907131,265.4238327)(133.33105469,265.42382812)
\lineto(131.75195312,265.42382812)
\lineto(131.75195312,267.078125)
\lineto(132.91210938,267.078125)
\curveto(133.45637676,267.07812792)(133.87890238,267.20703404)(134.1796875,267.46484375)
\curveto(134.48046428,267.72265853)(134.81347176,268.33138188)(135.17871094,269.29101562)
\lineto(135.53320312,270.19335938)
\lineto(130.66699219,282.03125)
\lineto(132.76171875,282.03125)
\lineto(136.52148438,272.62109375)
\lineto(140.28125,282.03125)
\lineto(142.37597656,282.03125)
\lineto(137.09082031,268.8828125)
}
}
{
\newrgbcolor{curcolor}{0 0 0}
\pscustom[linestyle=none,fillstyle=solid,fillcolor=curcolor]
{
\newpath
\moveto(147.01660156,271.8046875)
\lineto(147.01660156,265.42382812)
\lineto(145.02929688,265.42382812)
\lineto(145.02929688,282.03125)
\lineto(147.01660156,282.03125)
\lineto(147.01660156,280.20507812)
\curveto(147.43196175,280.92121304)(147.95474768,281.45116042)(148.58496094,281.79492188)
\curveto(149.22232454,282.14582119)(149.98143836,282.32127674)(150.86230469,282.32128906)
\curveto(152.3232329,282.32127674)(153.50845306,281.7411992)(154.41796875,280.58105469)
\curveto(155.33462311,279.42088902)(155.79295599,277.89549992)(155.79296875,276.00488281)
\curveto(155.79295599,274.1142537)(155.33462311,272.5888646)(154.41796875,271.42871094)
\curveto(153.50845306,270.26855442)(152.3232329,269.68847687)(150.86230469,269.68847656)
\curveto(149.98143836,269.68847687)(149.22232454,269.8603517)(148.58496094,270.20410156)
\curveto(147.95474768,270.55501247)(147.43196175,271.08854058)(147.01660156,271.8046875)
\moveto(153.74121094,276.00488281)
\curveto(153.74120023,277.4586514)(153.44041928,278.59732213)(152.83886719,279.42089844)
\curveto(152.24445693,280.25161735)(151.42447077,280.66698152)(150.37890625,280.66699219)
\curveto(149.33332703,280.66698152)(148.50976015,280.25161735)(147.90820312,279.42089844)
\curveto(147.3137978,278.59732213)(147.01659758,277.4586514)(147.01660156,276.00488281)
\curveto(147.01659758,274.55110222)(147.3137978,273.40885076)(147.90820312,272.578125)
\curveto(148.50976015,271.75455554)(149.33332703,271.34277209)(150.37890625,271.34277344)
\curveto(151.42447077,271.34277209)(152.24445693,271.75455554)(152.83886719,272.578125)
\curveto(153.44041928,273.40885076)(153.74120023,274.55110222)(153.74121094,276.00488281)
}
}
{
\newrgbcolor{curcolor}{0 0 0}
\pscustom[linestyle=none,fillstyle=solid,fillcolor=curcolor]
{
\newpath
\moveto(169.36035156,276.50976562)
\lineto(169.36035156,275.54296875)
\lineto(160.27246094,275.54296875)
\curveto(160.35839508,274.18228748)(160.76659779,273.14387706)(161.49707031,272.42773438)
\curveto(162.23469528,271.71874828)(163.2587828,271.36425645)(164.56933594,271.36425781)
\curveto(165.32844219,271.36425645)(166.06249093,271.45735531)(166.77148438,271.64355469)
\curveto(167.48761972,271.82975077)(168.19660338,272.10904737)(168.8984375,272.48144531)
\lineto(168.8984375,270.61230469)
\curveto(168.18944193,270.31152313)(167.46255464,270.08235669)(166.71777344,269.92480469)
\curveto(165.97297279,269.76725284)(165.2174397,269.68847687)(164.45117188,269.68847656)
\curveto(162.53189551,269.68847687)(161.01008713,270.24707007)(159.88574219,271.36425781)
\curveto(158.76855292,272.48144283)(158.20995972,273.99250903)(158.20996094,275.89746094)
\curveto(158.20995972,277.86685411)(158.73990711,279.42805047)(159.79980469,280.58105469)
\curveto(160.86685811,281.7411992)(162.30272907,282.32127674)(164.10742188,282.32128906)
\curveto(165.72590273,282.32127674)(167.00422176,281.79849081)(167.94238281,280.75292969)
\curveto(168.88768342,279.71450851)(169.3603392,278.30012191)(169.36035156,276.50976562)
\moveto(167.38378906,277.08984375)
\curveto(167.36945577,278.17121579)(167.0650941,279.03417065)(166.47070312,279.67871094)
\curveto(165.88345465,280.32323186)(165.10285648,280.64549717)(164.12890625,280.64550781)
\curveto(163.02603564,280.64549717)(162.14159642,280.33397404)(161.47558594,279.7109375)
\curveto(160.81672795,279.08788154)(160.43717104,278.21060377)(160.33691406,277.07910156)
\lineto(167.38378906,277.08984375)
}
}
{
\newrgbcolor{curcolor}{0 0 0}
\pscustom[linestyle=none,fillstyle=solid,fillcolor=curcolor]
{
}
}
{
\newrgbcolor{curcolor}{0 0 0}
\pscustom[linestyle=none,fillstyle=solid,fillcolor=curcolor]
{
\newpath
\moveto(187.52539062,280.20507812)
\lineto(187.52539062,286.71484375)
\lineto(189.50195312,286.71484375)
\lineto(189.50195312,270)
\lineto(187.52539062,270)
\lineto(187.52539062,271.8046875)
\curveto(187.11001647,271.08854058)(186.58364981,270.55501247)(185.94628906,270.20410156)
\curveto(185.31607295,269.8603517)(184.55695912,269.68847687)(183.66894531,269.68847656)
\curveto(182.21516459,269.68847687)(181.02994442,270.26855442)(180.11328125,271.42871094)
\curveto(179.20377437,272.5888646)(178.74902222,274.1142537)(178.74902344,276.00488281)
\curveto(178.74902222,277.89549992)(179.20377437,279.42088902)(180.11328125,280.58105469)
\curveto(181.02994442,281.7411992)(182.21516459,282.32127674)(183.66894531,282.32128906)
\curveto(184.55695912,282.32127674)(185.31607295,282.14582119)(185.94628906,281.79492188)
\curveto(186.58364981,281.45116042)(187.11001647,280.92121304)(187.52539062,280.20507812)
\moveto(180.79003906,276.00488281)
\curveto(180.79003581,274.55110222)(181.08723603,273.40885076)(181.68164062,272.578125)
\curveto(182.28319838,271.75455554)(183.10676526,271.34277209)(184.15234375,271.34277344)
\curveto(185.197909,271.34277209)(186.02147589,271.75455554)(186.62304688,272.578125)
\curveto(187.22459969,273.40885076)(187.52538063,274.55110222)(187.52539062,276.00488281)
\curveto(187.52538063,277.4586514)(187.22459969,278.59732213)(186.62304688,279.42089844)
\curveto(186.02147589,280.25161735)(185.197909,280.66698152)(184.15234375,280.66699219)
\curveto(183.10676526,280.66698152)(182.28319838,280.25161735)(181.68164062,279.42089844)
\curveto(181.08723603,278.59732213)(180.79003581,277.4586514)(180.79003906,276.00488281)
}
}
{
\newrgbcolor{curcolor}{0 0 0}
\pscustom[linestyle=none,fillstyle=solid,fillcolor=curcolor]
{
\newpath
\moveto(203.86425781,276.50976562)
\lineto(203.86425781,275.54296875)
\lineto(194.77636719,275.54296875)
\curveto(194.86230133,274.18228748)(195.27050404,273.14387706)(196.00097656,272.42773438)
\curveto(196.73860153,271.71874828)(197.76268905,271.36425645)(199.07324219,271.36425781)
\curveto(199.83234844,271.36425645)(200.56639718,271.45735531)(201.27539062,271.64355469)
\curveto(201.99152597,271.82975077)(202.70050963,272.10904737)(203.40234375,272.48144531)
\lineto(203.40234375,270.61230469)
\curveto(202.69334818,270.31152313)(201.96646089,270.08235669)(201.22167969,269.92480469)
\curveto(200.47687904,269.76725284)(199.72134595,269.68847687)(198.95507812,269.68847656)
\curveto(197.03580176,269.68847687)(195.51399338,270.24707007)(194.38964844,271.36425781)
\curveto(193.27245917,272.48144283)(192.71386597,273.99250903)(192.71386719,275.89746094)
\curveto(192.71386597,277.86685411)(193.24381336,279.42805047)(194.30371094,280.58105469)
\curveto(195.37076436,281.7411992)(196.80663532,282.32127674)(198.61132812,282.32128906)
\curveto(200.22980898,282.32127674)(201.50812801,281.79849081)(202.44628906,280.75292969)
\curveto(203.39158967,279.71450851)(203.86424545,278.30012191)(203.86425781,276.50976562)
\moveto(201.88769531,277.08984375)
\curveto(201.87336202,278.17121579)(201.56900035,279.03417065)(200.97460938,279.67871094)
\curveto(200.3873609,280.32323186)(199.60676273,280.64549717)(198.6328125,280.64550781)
\curveto(197.52994189,280.64549717)(196.64550267,280.33397404)(195.97949219,279.7109375)
\curveto(195.3206342,279.08788154)(194.94107729,278.21060377)(194.84082031,277.07910156)
\lineto(201.88769531,277.08984375)
}
}
{
\newrgbcolor{curcolor}{0 0 0}
\pscustom[linestyle=none,fillstyle=solid,fillcolor=curcolor]
{
}
}
{
\newrgbcolor{curcolor}{0 0 0}
\pscustom[linestyle=none,fillstyle=solid,fillcolor=curcolor]
{
\newpath
\moveto(216.02441406,271.8046875)
\lineto(216.02441406,265.42382812)
\lineto(214.03710938,265.42382812)
\lineto(214.03710938,282.03125)
\lineto(216.02441406,282.03125)
\lineto(216.02441406,280.20507812)
\curveto(216.43977425,280.92121304)(216.96256018,281.45116042)(217.59277344,281.79492188)
\curveto(218.23013704,282.14582119)(218.98925086,282.32127674)(219.87011719,282.32128906)
\curveto(221.3310454,282.32127674)(222.51626556,281.7411992)(223.42578125,280.58105469)
\curveto(224.34243561,279.42088902)(224.80076849,277.89549992)(224.80078125,276.00488281)
\curveto(224.80076849,274.1142537)(224.34243561,272.5888646)(223.42578125,271.42871094)
\curveto(222.51626556,270.26855442)(221.3310454,269.68847687)(219.87011719,269.68847656)
\curveto(218.98925086,269.68847687)(218.23013704,269.8603517)(217.59277344,270.20410156)
\curveto(216.96256018,270.55501247)(216.43977425,271.08854058)(216.02441406,271.8046875)
\moveto(222.74902344,276.00488281)
\curveto(222.74901273,277.4586514)(222.44823178,278.59732213)(221.84667969,279.42089844)
\curveto(221.25226943,280.25161735)(220.43228327,280.66698152)(219.38671875,280.66699219)
\curveto(218.34113953,280.66698152)(217.51757265,280.25161735)(216.91601562,279.42089844)
\curveto(216.3216103,278.59732213)(216.02441008,277.4586514)(216.02441406,276.00488281)
\curveto(216.02441008,274.55110222)(216.3216103,273.40885076)(216.91601562,272.578125)
\curveto(217.51757265,271.75455554)(218.34113953,271.34277209)(219.38671875,271.34277344)
\curveto(220.43228327,271.34277209)(221.25226943,271.75455554)(221.84667969,272.578125)
\curveto(222.44823178,273.40885076)(222.74901273,274.55110222)(222.74902344,276.00488281)
}
}
{
\newrgbcolor{curcolor}{0 0 0}
\pscustom[linestyle=none,fillstyle=solid,fillcolor=curcolor]
{
\newpath
\moveto(233.54492188,276.04785156)
\curveto(231.94791072,276.04784551)(230.84146652,275.86522851)(230.22558594,275.5)
\curveto(229.60969692,275.13476049)(229.30175451,274.51171424)(229.30175781,273.63085938)
\curveto(229.30175451,272.92903353)(229.53092095,272.37044034)(229.98925781,271.95507812)
\curveto(230.45474815,271.54687345)(231.08495586,271.34277209)(231.87988281,271.34277344)
\curveto(232.97557897,271.34277209)(233.85285673,271.72949046)(234.51171875,272.50292969)
\curveto(235.1777252,273.28352536)(235.51073268,274.31835506)(235.51074219,275.60742188)
\lineto(235.51074219,276.04785156)
\lineto(233.54492188,276.04785156)
\moveto(237.48730469,276.86425781)
\lineto(237.48730469,270)
\lineto(235.51074219,270)
\lineto(235.51074219,271.82617188)
\curveto(235.05956126,271.09570203)(234.49738734,270.55501247)(233.82421875,270.20410156)
\curveto(233.15103452,269.8603517)(232.32746763,269.68847687)(231.35351562,269.68847656)
\curveto(230.12174067,269.68847687)(229.14062186,270.03222653)(228.41015625,270.71972656)
\curveto(227.68684728,271.41438661)(227.32519399,272.34179453)(227.32519531,273.50195312)
\curveto(227.32519399,274.85546389)(227.77636542,275.87597069)(228.67871094,276.56347656)
\curveto(229.58821256,277.25096931)(230.94172683,277.59471897)(232.73925781,277.59472656)
\lineto(235.51074219,277.59472656)
\lineto(235.51074219,277.78808594)
\curveto(235.51073268,278.69758245)(235.20995173,279.39940466)(234.60839844,279.89355469)
\curveto(234.01398939,280.39484638)(233.1760996,280.64549717)(232.09472656,280.64550781)
\curveto(231.40722116,280.64549717)(230.73762547,280.56314048)(230.0859375,280.3984375)
\curveto(229.43424136,280.23371372)(228.80761438,279.98664366)(228.20605469,279.65722656)
\lineto(228.20605469,281.48339844)
\curveto(228.92935905,281.76268355)(229.63118127,281.97036563)(230.31152344,282.10644531)
\curveto(230.99185699,282.24966223)(231.65429122,282.32127674)(232.29882812,282.32128906)
\curveto(234.03905446,282.32127674)(235.33885785,281.87010532)(236.19824219,280.96777344)
\curveto(237.05760613,280.06541962)(237.4872932,278.69758245)(237.48730469,276.86425781)
}
}
{
\newrgbcolor{curcolor}{0 0 0}
\pscustom[linestyle=none,fillstyle=solid,fillcolor=curcolor]
{
\newpath
\moveto(248.54101562,280.18359375)
\curveto(248.31900159,280.31248969)(248.07551225,280.40558855)(247.81054688,280.46289062)
\curveto(247.55272632,280.52733322)(247.26626827,280.55955975)(246.95117188,280.55957031)
\curveto(245.83397804,280.55955975)(244.9746039,280.19432574)(244.37304688,279.46386719)
\curveto(243.77864155,278.74055116)(243.48144133,277.69856001)(243.48144531,276.33789062)
\lineto(243.48144531,270)
\lineto(241.49414062,270)
\lineto(241.49414062,282.03125)
\lineto(243.48144531,282.03125)
\lineto(243.48144531,280.16210938)
\curveto(243.8968055,280.89256723)(244.43749506,281.4332568)(245.10351562,281.78417969)
\curveto(245.76952498,282.14224046)(246.57876896,282.32127674)(247.53125,282.32128906)
\curveto(247.66730954,282.32127674)(247.81770001,282.31053456)(247.98242188,282.2890625)
\curveto(248.14712677,282.27472731)(248.32974377,282.24966223)(248.53027344,282.21386719)
\lineto(248.54101562,280.18359375)
}
}
{
\newrgbcolor{curcolor}{0 0 0}
\pscustom[linestyle=none,fillstyle=solid,fillcolor=curcolor]
{
\newpath
\moveto(252.59082031,285.44726562)
\lineto(252.59082031,282.03125)
\lineto(256.66210938,282.03125)
\lineto(256.66210938,280.49511719)
\lineto(252.59082031,280.49511719)
\lineto(252.59082031,273.96386719)
\curveto(252.59081628,272.98274441)(252.72330313,272.35253671)(252.98828125,272.07324219)
\curveto(253.26041197,271.79394352)(253.80826298,271.65429522)(254.63183594,271.65429688)
\lineto(256.66210938,271.65429688)
\lineto(256.66210938,270)
\lineto(254.63183594,270)
\curveto(253.10644077,270)(252.05370745,270.28287732)(251.47363281,270.84863281)
\curveto(250.89355236,271.42154806)(250.60351358,272.45995848)(250.60351562,273.96386719)
\lineto(250.60351562,280.49511719)
\lineto(249.15332031,280.49511719)
\lineto(249.15332031,282.03125)
\lineto(250.60351562,282.03125)
\lineto(250.60351562,285.44726562)
\lineto(252.59082031,285.44726562)
}
}
{
\newrgbcolor{curcolor}{0 0 0}
\pscustom[linestyle=none,fillstyle=solid,fillcolor=curcolor]
{
\newpath
\moveto(259.27246094,282.03125)
\lineto(261.24902344,282.03125)
\lineto(261.24902344,270)
\lineto(259.27246094,270)
\lineto(259.27246094,282.03125)
\moveto(259.27246094,286.71484375)
\lineto(261.24902344,286.71484375)
\lineto(261.24902344,284.21191406)
\lineto(259.27246094,284.21191406)
\lineto(259.27246094,286.71484375)
}
}
{
\newrgbcolor{curcolor}{0 0 0}
\pscustom[linestyle=none,fillstyle=solid,fillcolor=curcolor]
{
\newpath
\moveto(275.66503906,276.50976562)
\lineto(275.66503906,275.54296875)
\lineto(266.57714844,275.54296875)
\curveto(266.66308258,274.18228748)(267.07128529,273.14387706)(267.80175781,272.42773438)
\curveto(268.53938278,271.71874828)(269.5634703,271.36425645)(270.87402344,271.36425781)
\curveto(271.63312969,271.36425645)(272.36717843,271.45735531)(273.07617188,271.64355469)
\curveto(273.79230722,271.82975077)(274.50129088,272.10904737)(275.203125,272.48144531)
\lineto(275.203125,270.61230469)
\curveto(274.49412943,270.31152313)(273.76724214,270.08235669)(273.02246094,269.92480469)
\curveto(272.27766029,269.76725284)(271.5221272,269.68847687)(270.75585938,269.68847656)
\curveto(268.83658301,269.68847687)(267.31477463,270.24707007)(266.19042969,271.36425781)
\curveto(265.07324042,272.48144283)(264.51464722,273.99250903)(264.51464844,275.89746094)
\curveto(264.51464722,277.86685411)(265.04459461,279.42805047)(266.10449219,280.58105469)
\curveto(267.17154561,281.7411992)(268.60741657,282.32127674)(270.41210938,282.32128906)
\curveto(272.03059023,282.32127674)(273.30890926,281.79849081)(274.24707031,280.75292969)
\curveto(275.19237092,279.71450851)(275.6650267,278.30012191)(275.66503906,276.50976562)
\moveto(273.68847656,277.08984375)
\curveto(273.67414327,278.17121579)(273.3697816,279.03417065)(272.77539062,279.67871094)
\curveto(272.18814215,280.32323186)(271.40754398,280.64549717)(270.43359375,280.64550781)
\curveto(269.33072314,280.64549717)(268.44628392,280.33397404)(267.78027344,279.7109375)
\curveto(267.12141545,279.08788154)(266.74185854,278.21060377)(266.64160156,277.07910156)
\lineto(273.68847656,277.08984375)
}
}
{
\newrgbcolor{curcolor}{0 0 0}
\pscustom[linestyle=none,fillstyle=solid,fillcolor=curcolor]
{
\newpath
\moveto(112.111327,235.68264286)
\lineto(114.96844613,235.68264286)
\lineto(121.92216989,222.56300392)
\lineto(121.92216989,235.68264286)
\lineto(123.98097632,235.68264286)
\lineto(123.98097632,220)
\lineto(121.12385719,220)
\lineto(114.17013343,233.11963894)
\lineto(114.17013343,220)
\lineto(112.111327,220)
\lineto(112.111327,235.68264286)
}
}
{
\newrgbcolor{curcolor}{0 0 0}
\pscustom[linestyle=none,fillstyle=solid,fillcolor=curcolor]
{
\newpath
\moveto(132.67838398,230.40957741)
\curveto(131.64197248,230.409567)(130.82265237,230.00340832)(130.22042121,229.19110014)
\curveto(129.61818178,228.38577631)(129.31706414,227.27934403)(129.31706736,225.87179997)
\curveto(129.31706414,224.46424417)(129.61468042,223.35431052)(130.20991709,222.54199569)
\curveto(130.81214827,221.73667852)(131.63496974,221.3340212)(132.67838398,221.33402253)
\curveto(133.70777958,221.3340212)(134.52359832,221.74017989)(135.12584265,222.55249981)
\curveto(135.72806891,223.36481463)(136.02918655,224.47124691)(136.02919649,225.87179997)
\curveto(136.02918655,227.26533856)(135.72806891,228.36826947)(135.12584265,229.18059602)
\curveto(134.52359832,229.99990695)(133.70777958,230.409567)(132.67838398,230.40957741)
\moveto(132.67838398,232.04821926)
\curveto(134.35903403,232.04820722)(135.67904976,231.50199381)(136.63843513,230.40957741)
\curveto(137.59779941,229.31714019)(138.07748682,227.80454923)(138.07749881,225.87179997)
\curveto(138.07748682,223.94604171)(137.59779941,222.43345075)(136.63843513,221.33402253)
\curveto(135.67904976,220.24159439)(134.35903403,219.69538099)(132.67838398,219.69538068)
\curveto(130.99071803,219.69538099)(129.66720094,220.24159439)(128.70782873,221.33402253)
\curveto(127.75545402,222.43345075)(127.27926797,223.94604171)(127.27926916,225.87179997)
\curveto(127.27926797,227.80454923)(127.75545402,229.31714019)(128.70782873,230.40957741)
\curveto(129.66720094,231.50199381)(130.99071803,232.04820722)(132.67838398,232.04821926)
}
}
{
\newrgbcolor{curcolor}{0 0 0}
\pscustom[linestyle=none,fillstyle=solid,fillcolor=curcolor]
{
\newpath
\moveto(150.43033777,229.50622357)
\curveto(150.91351536,230.37455332)(151.49124108,231.01530366)(152.16351665,231.42847651)
\curveto(152.83576638,231.84162651)(153.62707555,232.04820722)(154.53744651,232.04821926)
\curveto(155.76291001,232.04820722)(156.70827937,231.61753895)(157.37355741,230.75621319)
\curveto(158.0387992,229.90186865)(158.37142916,228.68339259)(158.37144828,227.10078136)
\lineto(158.37144828,220)
\lineto(156.42818711,220)
\lineto(156.42818711,227.03775668)
\curveto(156.42816993,228.16519013)(156.22859195,229.00201707)(155.82945259,229.54824003)
\curveto(155.43028005,230.09444388)(154.82104203,230.36755059)(154.00173668,230.36756095)
\curveto(153.00033068,230.36755059)(152.20902151,230.03492063)(151.62780681,229.36967008)
\curveto(151.04656734,228.7044008)(150.7559538,227.79754649)(150.75596531,226.64910444)
\lineto(150.75596531,220)
\lineto(148.81270414,220)
\lineto(148.81270414,227.03775668)
\curveto(148.81269457,228.17219286)(148.6131166,229.00901981)(148.21396962,229.54824003)
\curveto(147.8148047,230.09444388)(147.19856394,230.36755059)(146.36524548,230.36756095)
\curveto(145.37785259,230.36755059)(144.59354616,230.03141926)(144.01232384,229.35916597)
\curveto(143.43109199,228.69389669)(143.14047845,227.79054375)(143.14048234,226.64910444)
\lineto(143.14048234,220)
\lineto(141.19722117,220)
\lineto(141.19722117,231.76460817)
\lineto(143.14048234,231.76460817)
\lineto(143.14048234,229.93689226)
\curveto(143.58165081,230.65816413)(144.11035738,231.19037206)(144.72660363,231.53351766)
\curveto(145.34283891,231.87664019)(146.07462481,232.04820722)(146.92196354,232.04821926)
\curveto(147.77628965,232.04820722)(148.50107282,231.8311224)(149.09631523,231.39696417)
\curveto(149.69854067,230.96278314)(150.14321441,230.33253691)(150.43033777,229.50622357)
}
}
{
\newrgbcolor{curcolor}{0 0 0}
\pscustom[linestyle=none,fillstyle=solid,fillcolor=curcolor]
{
\newpath
\moveto(170.68227312,225.87179997)
\curveto(170.68226265,227.2933495)(170.38814774,228.40678452)(169.79992751,229.21210837)
\curveto(169.21869083,230.02441652)(168.41687756,230.43057521)(167.3944853,230.43058564)
\curveto(166.37207866,230.43057521)(165.56676403,230.02441652)(164.97853898,229.21210837)
\curveto(164.39730712,228.40678452)(164.10669358,227.2933495)(164.10669748,225.87179997)
\curveto(164.10669358,224.4502387)(164.39730712,223.33330232)(164.97853898,222.52098747)
\curveto(165.56676403,221.71567031)(166.37207866,221.31301299)(167.3944853,221.31301431)
\curveto(168.41687756,221.31301299)(169.21869083,221.71567031)(169.79992751,222.52098747)
\curveto(170.38814774,223.33330232)(170.68226265,224.4502387)(170.68227312,225.87179997)
\moveto(164.10669748,229.97890872)
\curveto(164.51285227,230.67917234)(165.02405199,231.1973748)(165.64029819,231.53351766)
\curveto(166.26353626,231.87664019)(167.00582627,232.04820722)(167.86717045,232.04821926)
\curveto(169.29572093,232.04820722)(170.45467373,231.4809856)(171.34403233,230.34655272)
\curveto(172.2403714,229.21209915)(172.6885465,227.72051639)(172.68855898,225.87179997)
\curveto(172.6885465,224.02307181)(172.2403714,222.53148905)(171.34403233,221.39704722)
\curveto(170.45467373,220.2626026)(169.29572093,219.69538099)(167.86717045,219.69538068)
\curveto(167.00582627,219.69538099)(166.26353626,219.86344665)(165.64029819,220.19957817)
\curveto(165.02405199,220.54271204)(164.51285227,221.06441587)(164.10669748,221.76469123)
\lineto(164.10669748,220)
\lineto(162.16343631,220)
\lineto(162.16343631,236.34440207)
\lineto(164.10669748,236.34440207)
\lineto(164.10669748,229.97890872)
}
}
{
\newrgbcolor{curcolor}{0 0 0}
\pscustom[linestyle=none,fillstyle=solid,fillcolor=curcolor]
{
\newpath
\moveto(182.7094813,229.95790049)
\curveto(182.49238764,230.08393978)(182.25429462,230.17497535)(181.99520152,230.23100747)
\curveto(181.74309489,230.29402186)(181.46298545,230.32553417)(181.15487236,230.3255445)
\curveto(180.06243826,230.32553417)(179.22210995,229.96839464)(178.63388489,229.25412482)
\curveto(178.05265304,228.54683924)(177.7620395,227.52794116)(177.7620434,226.19742752)
\lineto(177.7620434,220)
\lineto(175.81878222,220)
\lineto(175.81878222,231.76460817)
\lineto(177.7620434,231.76460817)
\lineto(177.7620434,229.93689226)
\curveto(178.16819818,230.65116139)(178.69690475,231.17986796)(179.34816468,231.52301354)
\curveto(179.99941364,231.87313882)(180.7907228,232.04820722)(181.72209454,232.04821926)
\curveto(181.85513867,232.04820722)(182.00219612,232.03770311)(182.16326735,232.01670692)
\curveto(182.32432197,232.00268943)(182.50289174,231.97817986)(182.69897718,231.94317812)
\lineto(182.7094813,229.95790049)
}
}
{
\newrgbcolor{curcolor}{0 0 0}
\pscustom[linestyle=none,fillstyle=solid,fillcolor=curcolor]
{
\newpath
\moveto(194.35854627,226.36549335)
\lineto(194.35854627,225.42012305)
\lineto(185.47206545,225.42012305)
\curveto(185.55609508,224.0895978)(185.95525103,223.07420109)(186.6695345,222.37392986)
\curveto(187.3908119,221.68065663)(188.39220314,221.3340212)(189.67371123,221.33402253)
\curveto(190.41599383,221.3340212)(191.13377427,221.42505677)(191.82705469,221.60712951)
\curveto(192.52731873,221.78919904)(193.22058959,222.06230574)(193.90686935,222.42645044)
\lineto(193.90686935,220.59873452)
\curveto(193.21358685,220.30461901)(192.50280915,220.08053146)(191.77453412,219.9264712)
\curveto(191.04624007,219.77241108)(190.30745143,219.69538099)(189.55816597,219.69538068)
\curveto(187.68142544,219.69538099)(186.19334405,220.24159439)(185.09391733,221.33402253)
\curveto(184.0014877,222.42644801)(183.45527429,223.9040253)(183.45527548,225.76675883)
\curveto(183.45527429,227.69250545)(183.97347675,229.21910189)(185.00988442,230.34655272)
\curveto(186.05328933,231.4809856)(187.45733789,232.04820722)(189.22203431,232.04821926)
\curveto(190.80464568,232.04820722)(192.05463405,231.53700749)(192.97200316,230.51461856)
\curveto(193.89635361,229.49921133)(194.35853418,228.11617098)(194.35854627,226.36549335)
\moveto(192.42578921,226.93271553)
\curveto(192.41177358,227.99012173)(192.1141573,228.83395141)(191.53293948,229.46420711)
\curveto(190.95870587,230.09444388)(190.19540765,230.409567)(189.24304254,230.40957741)
\curveto(188.16461422,230.409567)(187.29977633,230.10494799)(186.64852627,229.49571946)
\curveto(186.00427018,228.88647193)(185.63312517,228.02863678)(185.53509014,226.92221142)
\lineto(192.42578921,226.93271553)
}
}
{
\newrgbcolor{curcolor}{0 0 0}
\pscustom[linestyle=none,fillstyle=solid,fillcolor=curcolor]
{
}
}
{
\newrgbcolor{curcolor}{0 0 0}
\pscustom[linestyle=none,fillstyle=solid,fillcolor=curcolor]
{
\newpath
\moveto(212.12099928,229.97890872)
\lineto(212.12099928,236.34440207)
\lineto(214.05375634,236.34440207)
\lineto(214.05375634,220)
\lineto(212.12099928,220)
\lineto(212.12099928,221.76469123)
\curveto(211.71483083,221.06441587)(211.20012973,220.54271204)(210.57689446,220.19957817)
\curveto(209.96064547,219.86344665)(209.21835546,219.69538099)(208.3500222,219.69538068)
\curveto(206.9284608,219.69538099)(205.769508,220.2626026)(204.87316032,221.39704722)
\curveto(203.98381033,222.53148905)(203.5391366,224.02307181)(203.53913778,225.87179997)
\curveto(203.5391366,227.72051639)(203.98381033,229.21209915)(204.87316032,230.34655272)
\curveto(205.769508,231.4809856)(206.9284608,232.04820722)(208.3500222,232.04821926)
\curveto(209.21835546,232.04820722)(209.96064547,231.87664019)(210.57689446,231.53351766)
\curveto(211.20012973,231.1973748)(211.71483083,230.67917234)(212.12099928,229.97890872)
\moveto(205.53491953,225.87179997)
\curveto(205.53491634,224.4502387)(205.82552989,223.33330232)(206.40676103,222.52098747)
\curveto(206.99498679,221.71567031)(207.80030143,221.31301299)(208.82270735,221.31301431)
\curveto(209.84510033,221.31301299)(210.65041496,221.71567031)(211.23865367,222.52098747)
\curveto(211.8268746,223.33330232)(212.12098951,224.4502387)(212.12099928,225.87179997)
\curveto(212.12098951,227.2933495)(211.8268746,228.40678452)(211.23865367,229.21210837)
\curveto(210.65041496,230.02441652)(209.84510033,230.43057521)(208.82270735,230.43058564)
\curveto(207.80030143,230.43057521)(206.99498679,230.02441652)(206.40676103,229.21210837)
\curveto(205.82552989,228.40678452)(205.53491634,227.2933495)(205.53491953,225.87179997)
}
}
{
\newrgbcolor{curcolor}{0 0 0}
\pscustom[linestyle=none,fillstyle=solid,fillcolor=curcolor]
{
\newpath
\moveto(219.86253258,235.68264286)
\lineto(219.86253258,229.85285935)
\lineto(218.07683313,229.85285935)
\lineto(218.07683313,235.68264286)
\lineto(219.86253258,235.68264286)
}
}
{
\newrgbcolor{curcolor}{0 0 0}
\pscustom[linestyle=none,fillstyle=solid,fillcolor=curcolor]
{
\newpath
\moveto(234.02208234,226.36549335)
\lineto(234.02208234,225.42012305)
\lineto(225.13560152,225.42012305)
\curveto(225.21963115,224.0895978)(225.6187871,223.07420109)(226.33307057,222.37392986)
\curveto(227.05434797,221.68065663)(228.05573921,221.3340212)(229.3372473,221.33402253)
\curveto(230.07952991,221.3340212)(230.79731034,221.42505677)(231.49059076,221.60712951)
\curveto(232.1908548,221.78919904)(232.88412566,222.06230574)(233.57040542,222.42645044)
\lineto(233.57040542,220.59873452)
\curveto(232.87712292,220.30461901)(232.16634522,220.08053146)(231.43807019,219.9264712)
\curveto(230.70977614,219.77241108)(229.9709875,219.69538099)(229.22170204,219.69538068)
\curveto(227.34496151,219.69538099)(225.85688012,220.24159439)(224.7574534,221.33402253)
\curveto(223.66502377,222.42644801)(223.11881036,223.9040253)(223.11881155,225.76675883)
\curveto(223.11881036,227.69250545)(223.63701282,229.21910189)(224.67342049,230.34655272)
\curveto(225.7168254,231.4809856)(227.12087396,232.04820722)(228.88557038,232.04821926)
\curveto(230.46818175,232.04820722)(231.71817012,231.53700749)(232.63553923,230.51461856)
\curveto(233.55988968,229.49921133)(234.02207025,228.11617098)(234.02208234,226.36549335)
\moveto(232.08932528,226.93271553)
\curveto(232.07530965,227.99012173)(231.77769338,228.83395141)(231.19647556,229.46420711)
\curveto(230.62224194,230.09444388)(229.85894372,230.409567)(228.90657861,230.40957741)
\curveto(227.8281503,230.409567)(226.9633124,230.10494799)(226.31206234,229.49571946)
\curveto(225.66780625,228.88647193)(225.29666125,228.02863678)(225.19862621,226.92221142)
\lineto(232.08932528,226.93271553)
}
}
{
\newrgbcolor{curcolor}{0 0 0}
\pscustom[linestyle=none,fillstyle=solid,fillcolor=curcolor]
{
\newpath
\moveto(246.97365497,227.10078136)
\lineto(246.97365497,220)
\lineto(245.04089791,220)
\lineto(245.04089791,227.03775668)
\curveto(245.04088804,228.15118466)(244.82380322,228.98451023)(244.38964281,229.53773591)
\curveto(243.95546396,230.09094252)(243.30420952,230.36755059)(242.43587753,230.36756095)
\curveto(241.3924626,230.36755059)(240.56964113,230.03492063)(239.96741063,229.36967008)
\curveto(239.36517054,228.7044008)(239.0640529,227.79754649)(239.06405679,226.64910444)
\lineto(239.06405679,220)
\lineto(237.12079562,220)
\lineto(237.12079562,231.76460817)
\lineto(239.06405679,231.76460817)
\lineto(239.06405679,229.93689226)
\curveto(239.52623347,230.64415866)(240.06894551,231.17286522)(240.69219453,231.52301354)
\curveto(241.32243524,231.87313882)(242.04721841,232.04820722)(242.86654622,232.04821926)
\curveto(244.21806656,232.04820722)(245.24046601,231.62804306)(245.93374764,230.78772553)
\curveto(246.62700773,229.95438916)(246.97364316,228.725409)(246.97365497,227.10078136)
}
}
{
\newrgbcolor{curcolor}{0 0 0}
\pscustom[linestyle=none,fillstyle=solid,fillcolor=curcolor]
{
\newpath
\moveto(260.62899707,227.10078136)
\lineto(260.62899707,220)
\lineto(258.69624001,220)
\lineto(258.69624001,227.03775668)
\curveto(258.69623014,228.15118466)(258.47914532,228.98451023)(258.04498492,229.53773591)
\curveto(257.61080606,230.09094252)(256.95955162,230.36755059)(256.09121963,230.36756095)
\curveto(255.0478047,230.36755059)(254.22498323,230.03492063)(253.62275274,229.36967008)
\curveto(253.02051264,228.7044008)(252.719395,227.79754649)(252.71939889,226.64910444)
\lineto(252.71939889,220)
\lineto(250.77613772,220)
\lineto(250.77613772,231.76460817)
\lineto(252.71939889,231.76460817)
\lineto(252.71939889,229.93689226)
\curveto(253.18157557,230.64415866)(253.72428761,231.17286522)(254.34753663,231.52301354)
\curveto(254.97777734,231.87313882)(255.70256052,232.04820722)(256.52188832,232.04821926)
\curveto(257.87340866,232.04820722)(258.89580811,231.62804306)(259.58908974,230.78772553)
\curveto(260.28234983,229.95438916)(260.62898526,228.725409)(260.62899707,227.10078136)
}
}
{
\newrgbcolor{curcolor}{0 0 0}
\pscustom[linestyle=none,fillstyle=solid,fillcolor=curcolor]
{
\newpath
\moveto(274.56795789,226.36549335)
\lineto(274.56795789,225.42012305)
\lineto(265.68147707,225.42012305)
\curveto(265.7655067,224.0895978)(266.16466265,223.07420109)(266.87894612,222.37392986)
\curveto(267.60022352,221.68065663)(268.60161476,221.3340212)(269.88312285,221.33402253)
\curveto(270.62540546,221.3340212)(271.34318589,221.42505677)(272.03646631,221.60712951)
\curveto(272.73673035,221.78919904)(273.43000121,222.06230574)(274.11628097,222.42645044)
\lineto(274.11628097,220.59873452)
\curveto(273.42299847,220.30461901)(272.71222077,220.08053146)(271.98394574,219.9264712)
\curveto(271.25565169,219.77241108)(270.51686305,219.69538099)(269.76757759,219.69538068)
\curveto(267.89083706,219.69538099)(266.40275567,220.24159439)(265.30332895,221.33402253)
\curveto(264.21089932,222.42644801)(263.66468591,223.9040253)(263.6646871,225.76675883)
\curveto(263.66468591,227.69250545)(264.18288837,229.21910189)(265.21929604,230.34655272)
\curveto(266.26270095,231.4809856)(267.66674951,232.04820722)(269.43144593,232.04821926)
\curveto(271.0140573,232.04820722)(272.26404567,231.53700749)(273.18141478,230.51461856)
\curveto(274.10576523,229.49921133)(274.5679458,228.11617098)(274.56795789,226.36549335)
\moveto(272.63520083,226.93271553)
\curveto(272.6211852,227.99012173)(272.32356893,228.83395141)(271.74235111,229.46420711)
\curveto(271.16811749,230.09444388)(270.40481927,230.409567)(269.45245416,230.40957741)
\curveto(268.37402584,230.409567)(267.50918795,230.10494799)(266.85793789,229.49571946)
\curveto(266.2136818,228.88647193)(265.8425368,228.02863678)(265.74450176,226.92221142)
\lineto(272.63520083,226.93271553)
}
}
{
\newrgbcolor{curcolor}{0 0 0}
\pscustom[linestyle=none,fillstyle=solid,fillcolor=curcolor]
{
\newpath
\moveto(286.89978776,229.50622357)
\curveto(287.38296536,230.37455332)(287.96069107,231.01530366)(288.63296665,231.42847651)
\curveto(289.30521638,231.84162651)(290.09652554,232.04820722)(291.00689651,232.04821926)
\curveto(292.23236001,232.04820722)(293.17772936,231.61753895)(293.84300741,230.75621319)
\curveto(294.5082492,229.90186865)(294.84087915,228.68339259)(294.84089828,227.10078136)
\lineto(294.84089828,220)
\lineto(292.89763711,220)
\lineto(292.89763711,227.03775668)
\curveto(292.89761993,228.16519013)(292.69804195,229.00201707)(292.29890259,229.54824003)
\curveto(291.89973005,230.09444388)(291.29049202,230.36755059)(290.47118667,230.36756095)
\curveto(289.46978067,230.36755059)(288.67847151,230.03492063)(288.09725681,229.36967008)
\curveto(287.51601734,228.7044008)(287.2254038,227.79754649)(287.22541531,226.64910444)
\lineto(287.22541531,220)
\lineto(285.28215414,220)
\lineto(285.28215414,227.03775668)
\curveto(285.28214457,228.17219286)(285.0825666,229.00901981)(284.68341962,229.54824003)
\curveto(284.2842547,230.09444388)(283.66801393,230.36755059)(282.83469548,230.36756095)
\curveto(281.84730258,230.36755059)(281.06299616,230.03141926)(280.48177384,229.35916597)
\curveto(279.90054199,228.69389669)(279.60992844,227.79054375)(279.60993234,226.64910444)
\lineto(279.60993234,220)
\lineto(277.66667117,220)
\lineto(277.66667117,231.76460817)
\lineto(279.60993234,231.76460817)
\lineto(279.60993234,229.93689226)
\curveto(280.05110081,230.65816413)(280.57980737,231.19037206)(281.19605362,231.53351766)
\curveto(281.8122889,231.87664019)(282.54407481,232.04820722)(283.39141354,232.04821926)
\curveto(284.24573965,232.04820722)(284.97052282,231.8311224)(285.56576523,231.39696417)
\curveto(286.16799067,230.96278314)(286.6126644,230.33253691)(286.89978776,229.50622357)
}
}
{
\newrgbcolor{curcolor}{0 0 0}
\pscustom[linestyle=none,fillstyle=solid,fillcolor=curcolor]
{
\newpath
\moveto(298.70641701,231.76460817)
\lineto(300.63917407,231.76460817)
\lineto(300.63917407,220)
\lineto(298.70641701,220)
\lineto(298.70641701,231.76460817)
\moveto(298.70641701,236.34440207)
\lineto(300.63917407,236.34440207)
\lineto(300.63917407,233.89694341)
\lineto(298.70641701,233.89694341)
\lineto(298.70641701,236.34440207)
}
}
{
\newrgbcolor{curcolor}{0 0 0}
\pscustom[linestyle=none,fillstyle=solid,fillcolor=curcolor]
{
\newpath
\moveto(312.17269383,231.4179724)
\lineto(312.17269383,229.59025649)
\curveto(311.6264709,229.87035633)(311.05924929,230.08043841)(310.4710273,230.22050335)
\curveto(309.88278965,230.36054785)(309.27355162,230.43057521)(308.64331138,230.43058564)
\curveto(307.68393056,230.43057521)(306.96264875,230.28351775)(306.47946381,229.98941283)
\curveto(306.00327393,229.69528793)(305.76518091,229.25411557)(305.76518403,228.66589441)
\curveto(305.76518091,228.21771065)(305.93674794,227.86407248)(306.27988563,227.60497886)
\curveto(306.62301606,227.35287276)(307.31278555,227.11127837)(308.34919618,226.88019496)
\lineto(309.01095539,226.73313736)
\curveto(310.38348527,226.43901571)(311.35686557,226.02235292)(311.9310992,225.48314774)
\curveto(312.512317,224.95093432)(312.80293054,224.20514294)(312.8029407,223.24577136)
\curveto(312.80293054,222.15334131)(312.36876091,221.28850342)(311.50043051,220.6512551)
\curveto(310.63908513,220.01400547)(309.45212139,219.69538099)(307.93953571,219.69538068)
\curveto(307.30928418,219.69538099)(306.651027,219.75840561)(305.9647622,219.88445474)
\curveto(305.28549349,220.00350137)(304.56771306,220.1855725)(303.81141874,220.43066869)
\lineto(303.81141874,222.42645044)
\curveto(304.52569664,222.055303)(305.2294716,221.77519357)(305.92274574,221.58612128)
\curveto(306.61601332,221.40404856)(307.30228145,221.31301299)(307.98155217,221.31301431)
\curveto(308.89190251,221.31301299)(309.59217611,221.46707318)(310.08237506,221.77519534)
\curveto(310.57255914,222.09031668)(310.8176549,222.53148905)(310.81766307,223.09871376)
\curveto(310.8176549,223.62391586)(310.63908513,224.02657318)(310.28195324,224.30668692)
\curveto(309.9318088,224.58679205)(309.15800648,224.85639739)(307.96054394,225.11550373)
\lineto(307.28828062,225.27306545)
\curveto(306.09080813,225.52515867)(305.22597024,225.91030915)(304.69376435,226.42851804)
\curveto(304.16155437,226.95371681)(303.8954504,227.67149724)(303.89545165,228.5818615)
\curveto(303.8954504,229.6882852)(304.28760362,230.54261899)(305.07191247,231.14486542)
\curveto(305.85621647,231.74708957)(306.96965149,232.04820722)(308.41222086,232.04821926)
\curveto(309.12649417,232.04820722)(309.79875682,231.9956867)(310.42901084,231.89065755)
\curveto(311.05924929,231.78560462)(311.64047637,231.62804306)(312.17269383,231.4179724)
}
}
{
\newrgbcolor{curcolor}{0 0 0}
\pscustom[linestyle=none,fillstyle=solid,fillcolor=curcolor,opacity=0.11935484]
{
\newpath
\moveto(333.90655899,240)
\lineto(419.49635696,240)
\curveto(425.31537098,240)(429.99999237,235.80657851)(429.99999237,230.59770966)
\lineto(429.99999237,219.50483704)
\curveto(429.99999237,214.29596819)(425.31537098,210.10254669)(419.49635696,210.10254669)
\lineto(333.90655899,210.10254669)
\curveto(328.08754498,210.10254669)(323.40292358,214.29596819)(323.40292358,219.50483704)
\lineto(323.40292358,230.59770966)
\curveto(323.40292358,235.80657851)(328.08754498,240)(333.90655899,240)
\closepath
}
}
{
\newrgbcolor{curcolor}{0 0 0}
\pscustom[linewidth=2,linecolor=curcolor]
{
\newpath
\moveto(333.90655899,240)
\lineto(419.49635696,240)
\curveto(425.31537098,240)(429.99999237,235.80657851)(429.99999237,230.59770966)
\lineto(429.99999237,219.50483704)
\curveto(429.99999237,214.29596819)(425.31537098,210.10254669)(419.49635696,210.10254669)
\lineto(333.90655899,210.10254669)
\curveto(328.08754498,210.10254669)(323.40292358,214.29596819)(323.40292358,219.50483704)
\lineto(323.40292358,230.59770966)
\curveto(323.40292358,235.80657851)(328.08754498,240)(333.90655899,240)
\closepath
}
}
{
\newrgbcolor{curcolor}{0 0 0}
\pscustom[linestyle=none,fillstyle=solid,fillcolor=curcolor,opacity=0.11935484]
{
\newpath
\moveto(312.70388412,189.96066284)
\lineto(417.25014114,189.96066284)
\curveto(424.35795159,189.96066284)(430.0801239,185.79414402)(430.0801239,180.6186924)
\lineto(430.0801239,169.59698677)
\curveto(430.0801239,164.42153514)(424.35795159,160.25501633)(417.25014114,160.25501633)
\lineto(312.70388412,160.25501633)
\curveto(305.59607368,160.25501633)(299.87390137,164.42153514)(299.87390137,169.59698677)
\lineto(299.87390137,180.6186924)
\curveto(299.87390137,185.79414402)(305.59607368,189.96066284)(312.70388412,189.96066284)
\closepath
}
}
{
\newrgbcolor{curcolor}{0 0 0}
\pscustom[linewidth=2,linecolor=curcolor]
{
\newpath
\moveto(312.70388412,189.96066284)
\lineto(417.25014114,189.96066284)
\curveto(424.35795159,189.96066284)(430.0801239,185.79414402)(430.0801239,180.6186924)
\lineto(430.0801239,169.59698677)
\curveto(430.0801239,164.42153514)(424.35795159,160.25501633)(417.25014114,160.25501633)
\lineto(312.70388412,160.25501633)
\curveto(305.59607368,160.25501633)(299.87390137,164.42153514)(299.87390137,169.59698677)
\lineto(299.87390137,180.6186924)
\curveto(299.87390137,185.79414402)(305.59607368,189.96066284)(312.70388412,189.96066284)
\closepath
}
}
{
\newrgbcolor{curcolor}{1 1 1}
\pscustom[linestyle=none,fillstyle=solid,fillcolor=curcolor]
{
\newpath
\moveto(329.26973534,49.81427002)
\lineto(420.7846241,49.81427002)
\curveto(425.93425914,49.81427002)(430.0799942,46.91140156)(430.0799942,43.30559635)
\lineto(430.0799942,26.42860699)
\curveto(430.0799942,22.82280177)(425.93425914,19.91993332)(420.7846241,19.91993332)
\lineto(329.26973534,19.91993332)
\curveto(324.1201003,19.91993332)(319.97436523,22.82280177)(319.97436523,26.42860699)
\lineto(319.97436523,43.30559635)
\curveto(319.97436523,46.91140156)(324.1201003,49.81427002)(329.26973534,49.81427002)
\closepath
}
}
{
\newrgbcolor{curcolor}{0 0 0}
\pscustom[linewidth=2,linecolor=curcolor]
{
\newpath
\moveto(329.26973534,49.81427002)
\lineto(420.7846241,49.81427002)
\curveto(425.93425914,49.81427002)(430.0799942,46.91140156)(430.0799942,43.30559635)
\lineto(430.0799942,26.42860699)
\curveto(430.0799942,22.82280177)(425.93425914,19.91993332)(420.7846241,19.91993332)
\lineto(329.26973534,19.91993332)
\curveto(324.1201003,19.91993332)(319.97436523,22.82280177)(319.97436523,26.42860699)
\lineto(319.97436523,43.30559635)
\curveto(319.97436523,46.91140156)(324.1201003,49.81427002)(329.26973534,49.81427002)
\closepath
}
}
{
\newrgbcolor{curcolor}{0 0 0}
\pscustom[linestyle=none,fillstyle=solid,fillcolor=curcolor]
{
\newpath
\moveto(342.35546875,47.49609375)
\lineto(344.72265625,47.49609375)
\lineto(344.72265625,31.9921875)
\lineto(353.2421875,31.9921875)
\lineto(353.2421875,30)
\lineto(342.35546875,30)
\lineto(342.35546875,47.49609375)
}
}
{
\newrgbcolor{curcolor}{0 0 0}
\pscustom[linestyle=none,fillstyle=solid,fillcolor=curcolor]
{
\newpath
\moveto(361.5859375,36.59765625)
\curveto(359.84374352,36.59764965)(358.63671347,36.3984311)(357.96484375,36)
\curveto(357.29296482,35.6015569)(356.95702765,34.92187008)(356.95703125,33.9609375)
\curveto(356.95702765,33.1953093)(357.2070274,32.58593491)(357.70703125,32.1328125)
\curveto(358.21483889,31.68749831)(358.90233821,31.46484229)(359.76953125,31.46484375)
\curveto(360.96483614,31.46484229)(361.92186644,31.88671686)(362.640625,32.73046875)
\curveto(363.36717749,33.58202767)(363.73045838,34.71093279)(363.73046875,36.1171875)
\lineto(363.73046875,36.59765625)
\lineto(361.5859375,36.59765625)
\moveto(365.88671875,37.48828125)
\lineto(365.88671875,30)
\lineto(363.73046875,30)
\lineto(363.73046875,31.9921875)
\curveto(363.23827137,31.1953113)(362.62499073,30.60546814)(361.890625,30.22265625)
\curveto(361.1562422,29.8476564)(360.2578056,29.66015659)(359.1953125,29.66015625)
\curveto(357.85155801,29.66015659)(356.78124658,30.03515621)(355.984375,30.78515625)
\curveto(355.19531066,31.54296721)(354.80077981,32.55468495)(354.80078125,33.8203125)
\curveto(354.80077981,35.2968697)(355.29296682,36.41014984)(356.27734375,37.16015625)
\curveto(357.26952734,37.91014834)(358.74608836,38.28514796)(360.70703125,38.28515625)
\lineto(363.73046875,38.28515625)
\lineto(363.73046875,38.49609375)
\curveto(363.73045838,39.48827176)(363.40233371,40.253896)(362.74609375,40.79296875)
\curveto(362.09764751,41.33983241)(361.18358593,41.61326964)(360.00390625,41.61328125)
\curveto(359.25390036,41.61326964)(358.52343234,41.52342598)(357.8125,41.34375)
\curveto(357.10155876,41.16405134)(356.41796569,40.89452036)(355.76171875,40.53515625)
\lineto(355.76171875,42.52734375)
\curveto(356.55077806,42.83201842)(357.31640229,43.05858069)(358.05859375,43.20703125)
\curveto(358.80077581,43.36326789)(359.52343134,43.44139281)(360.2265625,43.44140625)
\curveto(362.12499123,43.44139281)(363.54295857,42.9492058)(364.48046875,41.96484375)
\curveto(365.41795669,40.98045777)(365.88670622,39.48827176)(365.88671875,37.48828125)
}
}
{
\newrgbcolor{curcolor}{0 0 0}
\pscustom[linestyle=none,fillstyle=solid,fillcolor=curcolor]
{
\newpath
\moveto(381.25,37.921875)
\lineto(381.25,30)
\lineto(379.09375,30)
\lineto(379.09375,37.8515625)
\curveto(379.09373898,39.09374091)(378.85155173,40.02342748)(378.3671875,40.640625)
\curveto(377.8828027,41.25780124)(377.15624092,41.56639468)(376.1875,41.56640625)
\curveto(375.02343055,41.56639468)(374.10546272,41.1953013)(373.43359375,40.453125)
\curveto(372.76171407,39.71092779)(372.4257769,38.69921005)(372.42578125,37.41796875)
\lineto(372.42578125,30)
\lineto(370.2578125,30)
\lineto(370.2578125,43.125)
\lineto(372.42578125,43.125)
\lineto(372.42578125,41.0859375)
\curveto(372.94140139,41.87498812)(373.54686953,42.46483129)(374.2421875,42.85546875)
\curveto(374.94530563,43.2460805)(375.75389857,43.44139281)(376.66796875,43.44140625)
\curveto(378.17577115,43.44139281)(379.31639501,42.97264328)(380.08984375,42.03515625)
\curveto(380.86326846,41.10545764)(381.24998683,39.73436527)(381.25,37.921875)
}
}
{
\newrgbcolor{curcolor}{0 0 0}
\pscustom[linestyle=none,fillstyle=solid,fillcolor=curcolor]
{
\newpath
\moveto(395.01953125,42.62109375)
\lineto(395.01953125,40.60546875)
\curveto(394.41014515,40.94139531)(393.79686452,41.19139506)(393.1796875,41.35546875)
\curveto(392.57030324,41.52733222)(391.95311636,41.61326964)(391.328125,41.61328125)
\curveto(389.92968088,41.61326964)(388.84374447,41.16795758)(388.0703125,40.27734375)
\curveto(387.29687102,39.39452186)(386.91015265,38.1523356)(386.91015625,36.55078125)
\curveto(386.91015265,34.9492138)(387.29687102,33.7031213)(388.0703125,32.8125)
\curveto(388.84374447,31.92968557)(389.92968088,31.48827976)(391.328125,31.48828125)
\curveto(391.95311636,31.48827976)(392.57030324,31.57031093)(393.1796875,31.734375)
\curveto(393.79686452,31.90624809)(394.41014515,32.16015409)(395.01953125,32.49609375)
\lineto(395.01953125,30.50390625)
\curveto(394.41795764,30.22265603)(393.79295827,30.01171874)(393.14453125,29.87109375)
\curveto(392.50389706,29.73046902)(391.82030399,29.66015659)(391.09375,29.66015625)
\curveto(389.1171817,29.66015659)(387.54687077,30.28124972)(386.3828125,31.5234375)
\curveto(385.21874809,32.76562223)(384.63671743,34.44140181)(384.63671875,36.55078125)
\curveto(384.63671743,38.69139756)(385.22265434,40.37498963)(386.39453125,41.6015625)
\curveto(387.57421449,42.82811217)(389.18749413,43.44139281)(391.234375,43.44140625)
\curveto(391.89842891,43.44139281)(392.54686577,43.37108038)(393.1796875,43.23046875)
\curveto(393.8124895,43.09764315)(394.42577014,42.89451836)(395.01953125,42.62109375)
}
}
{
\newrgbcolor{curcolor}{0 0 0}
\pscustom[linestyle=none,fillstyle=solid,fillcolor=curcolor]
{
\newpath
\moveto(410.01953125,37.1015625)
\lineto(410.01953125,36.046875)
\lineto(400.10546875,36.046875)
\curveto(400.19921508,34.56249544)(400.64452714,33.42968407)(401.44140625,32.6484375)
\curveto(402.24608804,31.87499813)(403.36327442,31.48827976)(404.79296875,31.48828125)
\curveto(405.62108466,31.48827976)(406.42186511,31.58984216)(407.1953125,31.79296875)
\curveto(407.97655105,31.99609175)(408.74998778,32.30077895)(409.515625,32.70703125)
\lineto(409.515625,30.66796875)
\curveto(408.74217529,30.33984341)(407.94920733,30.08984366)(407.13671875,29.91796875)
\curveto(406.32420896,29.746094)(405.49999103,29.66015659)(404.6640625,29.66015625)
\curveto(402.57030646,29.66015659)(400.91015187,30.26953098)(399.68359375,31.48828125)
\curveto(398.46484182,32.70702854)(397.85546743,34.35546439)(397.85546875,36.43359375)
\curveto(397.85546743,38.58202267)(398.43359185,40.28514596)(399.58984375,41.54296875)
\curveto(400.75390203,42.80858094)(402.32030671,43.44139281)(404.2890625,43.44140625)
\curveto(406.05467798,43.44139281)(407.44920783,42.87108088)(408.47265625,41.73046875)
\curveto(409.50389328,40.59764565)(410.01951776,39.05467845)(410.01953125,37.1015625)
\moveto(407.86328125,37.734375)
\curveto(407.84764493,38.91405359)(407.51561402,39.85545889)(406.8671875,40.55859375)
\curveto(406.2265528,41.26170749)(405.37499116,41.61326964)(404.3125,41.61328125)
\curveto(403.10936842,41.61326964)(402.14452564,41.27342623)(401.41796875,40.59375)
\curveto(400.69921458,39.91405259)(400.2851525,38.95702229)(400.17578125,37.72265625)
\lineto(407.86328125,37.734375)
}
}
{
\newrgbcolor{curcolor}{0 0 0}
\pscustom[linestyle=none,fillstyle=solid,fillcolor=curcolor]
{
\newpath
\moveto(421.1640625,41.109375)
\curveto(420.92186537,41.24998875)(420.65624064,41.35155115)(420.3671875,41.4140625)
\curveto(420.08592871,41.48436352)(419.77342902,41.51951973)(419.4296875,41.51953125)
\curveto(418.21093059,41.51951973)(417.27343152,41.12108263)(416.6171875,40.32421875)
\curveto(415.96874533,39.53514671)(415.6445269,38.3984291)(415.64453125,36.9140625)
\lineto(415.64453125,30)
\lineto(413.4765625,30)
\lineto(413.4765625,43.125)
\lineto(415.64453125,43.125)
\lineto(415.64453125,41.0859375)
\curveto(416.09765145,41.88280062)(416.68749461,42.47264378)(417.4140625,42.85546875)
\curveto(418.14061816,43.2460805)(419.02342977,43.44139281)(420.0625,43.44140625)
\curveto(420.21092859,43.44139281)(420.37499092,43.42967407)(420.5546875,43.40625)
\curveto(420.73436556,43.39061161)(420.93358411,43.36326789)(421.15234375,43.32421875)
\lineto(421.1640625,41.109375)
}
}
{
\newrgbcolor{curcolor}{1 1 1}
\pscustom[linestyle=none,fillstyle=solid,fillcolor=curcolor]
{
\newpath
\moveto(139.20729923,49.62908936)
\lineto(230.67112637,49.62908936)
\curveto(235.81788884,49.62908936)(239.96131134,46.74393137)(239.96131134,43.16012526)
\lineto(239.96131134,26.38610125)
\curveto(239.96131134,22.80229513)(235.81788884,19.91713715)(230.67112637,19.91713715)
\lineto(139.20729923,19.91713715)
\curveto(134.06053676,19.91713715)(129.91711426,22.80229513)(129.91711426,26.38610125)
\lineto(129.91711426,43.16012526)
\curveto(129.91711426,46.74393137)(134.06053676,49.62908936)(139.20729923,49.62908936)
\closepath
}
}
{
\newrgbcolor{curcolor}{0 0 0}
\pscustom[linewidth=1.72500002,linecolor=curcolor]
{
\newpath
\moveto(139.20729923,49.62908936)
\lineto(230.67112637,49.62908936)
\curveto(235.81788884,49.62908936)(239.96131134,46.74393137)(239.96131134,43.16012526)
\lineto(239.96131134,26.38610125)
\curveto(239.96131134,22.80229513)(235.81788884,19.91713715)(230.67112637,19.91713715)
\lineto(139.20729923,19.91713715)
\curveto(134.06053676,19.91713715)(129.91711426,22.80229513)(129.91711426,26.38610125)
\lineto(129.91711426,43.16012526)
\curveto(129.91711426,46.74393137)(134.06053676,49.62908936)(139.20729923,49.62908936)
\closepath
}
}
{
\newrgbcolor{curcolor}{0 0 0}
\pscustom[linestyle=none,fillstyle=solid,fillcolor=curcolor]
{
\newpath
\moveto(160.65234375,38.203125)
\curveto(161.16014509,38.03124197)(161.6523321,37.66405484)(162.12890625,37.1015625)
\curveto(162.61326864,36.53905596)(163.09764315,35.76561923)(163.58203125,34.78125)
\lineto(165.984375,30)
\lineto(163.44140625,30)
\lineto(161.203125,34.48828125)
\curveto(160.62498938,35.66015059)(160.06248994,36.43749356)(159.515625,36.8203125)
\curveto(158.97655352,37.2031178)(158.23827301,37.39452386)(157.30078125,37.39453125)
\lineto(154.72265625,37.39453125)
\lineto(154.72265625,30)
\lineto(152.35546875,30)
\lineto(152.35546875,47.49609375)
\lineto(157.69921875,47.49609375)
\curveto(159.69920905,47.49607625)(161.19139506,47.07810792)(162.17578125,46.2421875)
\curveto(163.16014309,45.40623459)(163.6523301,44.14451711)(163.65234375,42.45703125)
\curveto(163.6523301,41.35545739)(163.39451786,40.44139581)(162.87890625,39.71484375)
\curveto(162.37108138,38.98827226)(161.62889462,38.48436652)(160.65234375,38.203125)
\moveto(154.72265625,45.55078125)
\lineto(154.72265625,39.33984375)
\lineto(157.69921875,39.33984375)
\curveto(158.83983491,39.33983441)(159.69920905,39.6015529)(160.27734375,40.125)
\curveto(160.86327039,40.65623934)(161.15623884,41.43358232)(161.15625,42.45703125)
\curveto(161.15623884,43.48045527)(160.86327039,44.24998575)(160.27734375,44.765625)
\curveto(159.69920905,45.28904721)(158.83983491,45.5507657)(157.69921875,45.55078125)
\lineto(154.72265625,45.55078125)
}
}
{
\newrgbcolor{curcolor}{0 0 0}
\pscustom[linestyle=none,fillstyle=solid,fillcolor=curcolor]
{
\newpath
\moveto(179.09765625,37.1015625)
\lineto(179.09765625,36.046875)
\lineto(169.18359375,36.046875)
\curveto(169.27734008,34.56249544)(169.72265214,33.42968407)(170.51953125,32.6484375)
\curveto(171.32421304,31.87499813)(172.44139942,31.48827976)(173.87109375,31.48828125)
\curveto(174.69920966,31.48827976)(175.49999011,31.58984216)(176.2734375,31.79296875)
\curveto(177.05467605,31.99609175)(177.82811278,32.30077895)(178.59375,32.70703125)
\lineto(178.59375,30.66796875)
\curveto(177.82030029,30.33984341)(177.02733233,30.08984366)(176.21484375,29.91796875)
\curveto(175.40233396,29.746094)(174.57811603,29.66015659)(173.7421875,29.66015625)
\curveto(171.64843146,29.66015659)(169.98827687,30.26953098)(168.76171875,31.48828125)
\curveto(167.54296682,32.70702854)(166.93359243,34.35546439)(166.93359375,36.43359375)
\curveto(166.93359243,38.58202267)(167.51171685,40.28514596)(168.66796875,41.54296875)
\curveto(169.83202703,42.80858094)(171.39843171,43.44139281)(173.3671875,43.44140625)
\curveto(175.13280298,43.44139281)(176.52733283,42.87108088)(177.55078125,41.73046875)
\curveto(178.58201828,40.59764565)(179.09764276,39.05467845)(179.09765625,37.1015625)
\moveto(176.94140625,37.734375)
\curveto(176.92576993,38.91405359)(176.59373902,39.85545889)(175.9453125,40.55859375)
\curveto(175.3046778,41.26170749)(174.45311616,41.61326964)(173.390625,41.61328125)
\curveto(172.18749342,41.61326964)(171.22265064,41.27342623)(170.49609375,40.59375)
\curveto(169.77733958,39.91405259)(169.3632775,38.95702229)(169.25390625,37.72265625)
\lineto(176.94140625,37.734375)
}
}
{
\newrgbcolor{curcolor}{0 0 0}
\pscustom[linestyle=none,fillstyle=solid,fillcolor=curcolor]
{
\newpath
\moveto(184.76953125,46.8515625)
\lineto(184.76953125,43.125)
\lineto(189.2109375,43.125)
\lineto(189.2109375,41.44921875)
\lineto(184.76953125,41.44921875)
\lineto(184.76953125,34.32421875)
\curveto(184.76952686,33.253903)(184.91405796,32.56640368)(185.203125,32.26171875)
\curveto(185.49999487,31.95702929)(186.09765053,31.8046857)(186.99609375,31.8046875)
\lineto(189.2109375,31.8046875)
\lineto(189.2109375,30)
\lineto(186.99609375,30)
\curveto(185.33202629,30)(184.18358994,30.30859344)(183.55078125,30.92578125)
\curveto(182.91796621,31.5507797)(182.60156027,32.68359107)(182.6015625,34.32421875)
\lineto(182.6015625,41.44921875)
\lineto(181.01953125,41.44921875)
\lineto(181.01953125,43.125)
\lineto(182.6015625,43.125)
\lineto(182.6015625,46.8515625)
\lineto(184.76953125,46.8515625)
}
}
{
\newrgbcolor{curcolor}{0 0 0}
\pscustom[linestyle=none,fillstyle=solid,fillcolor=curcolor]
{
\newpath
\moveto(197.14453125,41.61328125)
\curveto(195.98827506,41.61326964)(195.07421347,41.16014509)(194.40234375,40.25390625)
\curveto(193.73046482,39.35545939)(193.39452765,38.12108563)(193.39453125,36.55078125)
\curveto(193.39452765,34.98046377)(193.72655857,33.74218376)(194.390625,32.8359375)
\curveto(195.06249473,31.93749806)(195.98046257,31.48827976)(197.14453125,31.48828125)
\curveto(198.29296025,31.48827976)(199.20311559,31.94140431)(199.875,32.84765625)
\curveto(200.54686425,33.7539025)(200.88280141,34.98827626)(200.8828125,36.55078125)
\curveto(200.88280141,38.10546064)(200.54686425,39.33592816)(199.875,40.2421875)
\curveto(199.20311559,41.15623884)(198.29296025,41.61326964)(197.14453125,41.61328125)
\moveto(197.14453125,43.44140625)
\curveto(199.01952203,43.44139281)(200.4921768,42.83201842)(201.5625,41.61328125)
\curveto(202.63279966,40.39452086)(203.16795538,38.70702254)(203.16796875,36.55078125)
\curveto(203.16795538,34.40233935)(202.63279966,32.71484104)(201.5625,31.48828125)
\curveto(200.4921768,30.26953098)(199.01952203,29.66015659)(197.14453125,29.66015625)
\curveto(195.26171329,29.66015659)(193.78515226,30.26953098)(192.71484375,31.48828125)
\curveto(191.65234189,32.71484104)(191.12109243,34.40233935)(191.12109375,36.55078125)
\curveto(191.12109243,38.70702254)(191.65234189,40.39452086)(192.71484375,41.61328125)
\curveto(193.78515226,42.83201842)(195.26171329,43.44139281)(197.14453125,43.44140625)
}
}
{
\newrgbcolor{curcolor}{0 0 0}
\pscustom[linestyle=none,fillstyle=solid,fillcolor=curcolor]
{
\newpath
\moveto(206.5078125,35.1796875)
\lineto(206.5078125,43.125)
\lineto(208.6640625,43.125)
\lineto(208.6640625,35.26171875)
\curveto(208.6640583,34.01952723)(208.90624556,33.08593441)(209.390625,32.4609375)
\curveto(209.87499459,31.84374816)(210.60155637,31.53515471)(211.5703125,31.53515625)
\curveto(212.73436673,31.53515471)(213.65233457,31.90624809)(214.32421875,32.6484375)
\curveto(215.00389571,33.39062161)(215.34373913,34.40233935)(215.34375,35.68359375)
\lineto(215.34375,43.125)
\lineto(217.5,43.125)
\lineto(217.5,30)
\lineto(215.34375,30)
\lineto(215.34375,32.015625)
\curveto(214.82030215,31.21874878)(214.21092776,30.62499938)(213.515625,30.234375)
\curveto(212.82811664,29.85156265)(212.02733619,29.66015659)(211.11328125,29.66015625)
\curveto(209.60546361,29.66015659)(208.46093351,30.12890612)(207.6796875,31.06640625)
\curveto(206.89843507,32.00390425)(206.50781046,33.37499662)(206.5078125,35.1796875)
\moveto(211.93359375,43.44140625)
\lineto(211.93359375,43.44140625)
}
}
{
\newrgbcolor{curcolor}{0 0 0}
\pscustom[linestyle=none,fillstyle=solid,fillcolor=curcolor]
{
\newpath
\moveto(229.5703125,41.109375)
\curveto(229.32811538,41.24998875)(229.06249064,41.35155115)(228.7734375,41.4140625)
\curveto(228.49217871,41.48436352)(228.17967902,41.51951973)(227.8359375,41.51953125)
\curveto(226.61718059,41.51951973)(225.67968152,41.12108263)(225.0234375,40.32421875)
\curveto(224.37499533,39.53514671)(224.0507769,38.3984291)(224.05078125,36.9140625)
\lineto(224.05078125,30)
\lineto(221.8828125,30)
\lineto(221.8828125,43.125)
\lineto(224.05078125,43.125)
\lineto(224.05078125,41.0859375)
\curveto(224.50390145,41.88280062)(225.09374461,42.47264378)(225.8203125,42.85546875)
\curveto(226.54686816,43.2460805)(227.42967977,43.44139281)(228.46875,43.44140625)
\curveto(228.61717859,43.44139281)(228.78124092,43.42967407)(228.9609375,43.40625)
\curveto(229.14061556,43.39061161)(229.33983411,43.36326789)(229.55859375,43.32421875)
\lineto(229.5703125,41.109375)
}
}
{
\newrgbcolor{curcolor}{0 0 0}
\pscustom[linewidth=2,linecolor=curcolor,linestyle=dashed,dash=8 8]
{
\newpath
\moveto(299.8133,460.48668)
\lineto(500.32522,499.8184)
}
}
{
\newrgbcolor{curcolor}{0 0 0}
\pscustom[linestyle=none,fillstyle=solid,fillcolor=curcolor]
{
\newpath
\moveto(311.01168865,457.75058911)
\lineto(297.21778492,459.95797069)
\lineto(309.15561533,467.21279855)
\curveto(307.64810037,464.01637106)(308.40958335,460.1970544)(311.01168865,457.75058911)
\lineto(311.01168865,457.75058911)
\closepath
}
}
{
\newrgbcolor{curcolor}{0 0 0}
\pscustom[linewidth=2,linecolor=curcolor,linestyle=dashed,dash=8 8]
{
\newpath
\moveto(140,40)
\lineto(70.11406,39.73606)
}
}
{
\newrgbcolor{curcolor}{0 0 0}
\pscustom[linestyle=none,fillstyle=solid,fillcolor=curcolor]
{
\newpath
\moveto(129.51948821,44.80093475)
\lineto(142.64865594,40.0292193)
\lineto(129.55590443,35.15847191)
\curveto(131.6397373,38.01283536)(131.61296003,41.9072312)(129.51948821,44.80093475)
\lineto(129.51948821,44.80093475)
\closepath
}
}
{
\newrgbcolor{curcolor}{0 0 0}
\pscustom[linewidth=2,linecolor=curcolor,linestyle=dashed,dash=8 8]
{
\newpath
\moveto(220,330)
\lineto(71.232745,330)
}
}
{
\newrgbcolor{curcolor}{0 0 0}
\pscustom[linestyle=none,fillstyle=solid,fillcolor=curcolor]
{
\newpath
\moveto(209.53769464,334.84048224)
\lineto(222.6487474,330.01921591)
\lineto(209.53769392,325.19795064)
\curveto(211.632292,328.04442372)(211.62022288,331.93889292)(209.53769464,334.84048224)
\lineto(209.53769464,334.84048224)
\closepath
}
}
{
\newrgbcolor{curcolor}{0 0 0}
\pscustom[linewidth=2.21782422,linecolor=curcolor,linestyle=dashed,dash=8.87129678 8.87129678]
{
\newpath
\moveto(63.318808,420)
\lineto(160,420)
}
}
{
\newrgbcolor{curcolor}{0 0 0}
\pscustom[linestyle=none,fillstyle=solid,fillcolor=curcolor]
{
\newpath
\moveto(148.39822288,425.36766938)
\lineto(162.93722807,420.02130876)
\lineto(148.39822209,414.67494931)
\curveto(150.72094726,417.83143778)(150.70756367,422.15006184)(148.39822288,425.36766938)
\closepath
}
}
{
\newrgbcolor{curcolor}{0 0 0}
\pscustom[linewidth=2,linecolor=curcolor,linestyle=dashed,dash=8 8]
{
\newpath
\moveto(500,420)
\lineto(399.70808,420)
}
}
{
\newrgbcolor{curcolor}{0 0 0}
\pscustom[linestyle=none,fillstyle=solid,fillcolor=curcolor]
{
\newpath
\moveto(410.17038536,415.15951776)
\lineto(397.0593326,419.98078409)
\lineto(410.17038608,424.80204936)
\curveto(408.075788,421.95557628)(408.08785712,418.06110708)(410.17038536,415.15951776)
\closepath
}
}
{
\newrgbcolor{curcolor}{0 0 0}
\pscustom[linewidth=2,linecolor=curcolor,linestyle=dashed,dash=8 8]
{
\newpath
\moveto(500,270)
\lineto(399.86062,270)
}
}
{
\newrgbcolor{curcolor}{0 0 0}
\pscustom[linestyle=none,fillstyle=solid,fillcolor=curcolor]
{
\newpath
\moveto(410.32292536,265.15951776)
\lineto(397.2118726,269.98078409)
\lineto(410.32292608,274.80204936)
\curveto(408.228328,271.95557628)(408.24039712,268.06110708)(410.32292536,265.15951776)
\closepath
}
}
{
\newrgbcolor{curcolor}{0 0 0}
\pscustom[linewidth=2.12666917,linecolor=curcolor,linestyle=dashed,dash=8.50667644 8.50667644]
{
\newpath
\moveto(500.47505,219.83946)
\lineto(413.6734,219.83946)
}
}
{
\newrgbcolor{curcolor}{0 0 0}
\pscustom[linestyle=none,fillstyle=solid,fillcolor=curcolor]
{
\newpath
\moveto(424.79833112,214.69240783)
\lineto(410.85689528,219.81902706)
\lineto(424.79833189,224.94564516)
\curveto(422.57107331,221.91889189)(422.58390682,217.7777681)(424.79833112,214.69240783)
\closepath
}
}
{
\newrgbcolor{curcolor}{0 0 0}
\pscustom[linewidth=2.10198331,linecolor=curcolor,linestyle=dashed,dash=8.40793346 8.40793346]
{
\newpath
\moveto(499.6268,171.30557)
\lineto(400,170)
}
}
{
\newrgbcolor{curcolor}{0 0 0}
\pscustom[linestyle=none,fillstyle=solid,fillcolor=curcolor]
{
\newpath
\moveto(411.06151301,165.05721353)
\lineto(397.21669222,169.94332835)
\lineto(410.9287202,175.19056371)
\curveto(408.76670473,172.17035504)(408.83302145,168.07781803)(411.06151301,165.05721353)
\closepath
}
}
{
\newrgbcolor{curcolor}{0 0 0}
\pscustom[linewidth=2,linecolor=curcolor,linestyle=dashed,dash=8 8]
{
\newpath
\moveto(490,40)
\lineto(411.0551,41.27788)
}
}
{
\newrgbcolor{curcolor}{0 0 0}
\pscustom[linestyle=none,fillstyle=solid,fillcolor=curcolor]
{
\newpath
\moveto(421.43769241,36.26870077)
\lineto(408.40638854,41.30153623)
\lineto(421.59375622,45.90996935)
\curveto(419.45336271,43.0977699)(419.40239879,39.20361547)(421.43769241,36.26870077)
\closepath
}
}
{
\newrgbcolor{curcolor}{0 0 0}
\pscustom[linestyle=none,fillstyle=solid,fillcolor=curcolor]
{
\newpath
\moveto(52.984375,422.578125)
\curveto(54.49477717,422.25519608)(55.67185933,421.58332175)(56.515625,420.5625)
\curveto(57.3697743,419.54165713)(57.7968572,418.28124172)(57.796875,416.78125)
\curveto(57.7968572,414.47916219)(57.00519133,412.69791397)(55.421875,411.4375)
\curveto(53.83852783,410.17708316)(51.58853008,409.54687545)(48.671875,409.546875)
\curveto(47.69270064,409.54687545)(46.68228498,409.64583369)(45.640625,409.84375)
\curveto(44.60937039,410.03124997)(43.54166312,410.31770802)(42.4375,410.703125)
\lineto(42.4375,413.75)
\curveto(43.31249669,413.23958009)(44.27082906,412.85416381)(45.3125,412.59375)
\curveto(46.35416031,412.333331)(47.44270089,412.2031228)(48.578125,412.203125)
\curveto(50.55728111,412.2031228)(52.06248794,412.59374741)(53.09375,413.375)
\curveto(54.13540253,414.15624584)(54.65623534,415.29166137)(54.65625,416.78125)
\curveto(54.65623534,418.15624184)(54.17186083,419.22915744)(53.203125,420)
\curveto(52.24477942,420.78123922)(50.90623909,421.17186383)(49.1875,421.171875)
\lineto(46.46875,421.171875)
\lineto(46.46875,423.765625)
\lineto(49.3125,423.765625)
\curveto(50.86457247,423.76561123)(52.05207128,424.07290259)(52.875,424.6875)
\curveto(53.69790297,425.31248469)(54.10936089,426.20831713)(54.109375,427.375)
\curveto(54.10936089,428.57289809)(53.68227798,429.48956384)(52.828125,430.125)
\curveto(51.98436302,430.77081256)(50.77082256,431.09372891)(49.1875,431.09375)
\curveto(48.32290834,431.09372891)(47.39582594,430.999979)(46.40625,430.8125)
\curveto(45.41666125,430.62497938)(44.32812067,430.333313)(43.140625,429.9375)
\lineto(43.140625,432.75)
\curveto(44.33853733,433.08331025)(45.45832788,433.33331)(46.5,433.5)
\curveto(47.55207578,433.666643)(48.54165812,433.74997625)(49.46875,433.75)
\curveto(51.86457147,433.74997625)(53.76040291,433.2031018)(55.15625,432.109375)
\curveto(56.55206678,431.02602064)(57.24998275,429.55727211)(57.25,427.703125)
\curveto(57.24998275,426.41144192)(56.88019145,425.31769302)(56.140625,424.421875)
\curveto(55.40102627,423.5364448)(54.34894398,422.92186208)(52.984375,422.578125)
}
}
{
\newrgbcolor{curcolor}{0 0 0}
\pscustom[linestyle=none,fillstyle=solid,fillcolor=curcolor]
{
\newpath
\moveto(512.09375,430.578125)
\lineto(504.125,418.125)
\lineto(512.09375,418.125)
\lineto(512.09375,430.578125)
\moveto(511.265625,433.328125)
\lineto(515.234375,433.328125)
\lineto(515.234375,418.125)
\lineto(518.5625,418.125)
\lineto(518.5625,415.5)
\lineto(515.234375,415.5)
\lineto(515.234375,410)
\lineto(512.09375,410)
\lineto(512.09375,415.5)
\lineto(501.5625,415.5)
\lineto(501.5625,418.546875)
\lineto(511.265625,433.328125)
}
}
{
\newrgbcolor{curcolor}{0 0 0}
\pscustom[linestyle=none,fillstyle=solid,fillcolor=curcolor]
{
\newpath
\moveto(53.453125,343.328125)
\lineto(65.84375,343.328125)
\lineto(65.84375,340.671875)
\lineto(56.34375,340.671875)
\lineto(56.34375,334.953125)
\curveto(56.80207653,335.10935989)(57.26040941,335.22394311)(57.71875,335.296875)
\curveto(58.17707516,335.38019295)(58.63540803,335.42185958)(59.09375,335.421875)
\curveto(61.69790497,335.42185958)(63.76040291,334.70831863)(65.28125,333.28125)
\curveto(66.80206653,331.85415481)(67.56248244,329.92186508)(67.5625,327.484375)
\curveto(67.56248244,324.97395336)(66.78123322,323.02083031)(65.21875,321.625)
\curveto(63.65623634,320.23958309)(61.45311355,319.54687545)(58.609375,319.546875)
\curveto(57.6302007,319.54687545)(56.6302017,319.6302087)(55.609375,319.796875)
\curveto(54.59895373,319.9635417)(53.55207978,320.21354145)(52.46875,320.546875)
\lineto(52.46875,323.71875)
\curveto(53.40624659,323.20833012)(54.37499562,322.82812217)(55.375,322.578125)
\curveto(56.37499363,322.32812267)(57.43228423,322.2031228)(58.546875,322.203125)
\curveto(60.34894798,322.2031228)(61.77602989,322.67708066)(62.828125,323.625)
\curveto(63.88019445,324.57291209)(64.40623559,325.85936914)(64.40625,327.484375)
\curveto(64.40623559,329.10936589)(63.88019445,330.39582294)(62.828125,331.34375)
\curveto(61.77602989,332.29165438)(60.34894798,332.76561223)(58.546875,332.765625)
\curveto(57.7031173,332.76561223)(56.85936814,332.67186233)(56.015625,332.484375)
\curveto(55.18228648,332.2968627)(54.32812067,332.00519633)(53.453125,331.609375)
\lineto(53.453125,343.328125)
}
}
{
\newrgbcolor{curcolor}{0 0 0}
\pscustom[linestyle=none,fillstyle=solid,fillcolor=curcolor]
{
\newpath
\moveto(510.5625,272.921875)
\curveto(509.14582419,272.92186208)(508.02082531,272.43748756)(507.1875,271.46875)
\curveto(506.36457697,270.4999895)(505.95311905,269.17186583)(505.953125,267.484375)
\curveto(505.95311905,265.80728586)(506.36457697,264.47916219)(507.1875,263.5)
\curveto(508.02082531,262.53124747)(509.14582419,262.04687295)(510.5625,262.046875)
\curveto(511.97915469,262.04687295)(513.09894523,262.53124747)(513.921875,263.5)
\curveto(514.75519358,264.47916219)(515.17185983,265.80728586)(515.171875,267.484375)
\curveto(515.17185983,269.17186583)(514.75519358,270.4999895)(513.921875,271.46875)
\curveto(513.09894523,272.43748756)(511.97915469,272.92186208)(510.5625,272.921875)
\moveto(516.828125,282.8125)
\lineto(516.828125,279.9375)
\curveto(516.0364423,280.31247969)(515.23435977,280.59893773)(514.421875,280.796875)
\curveto(513.61977805,280.99477067)(512.82290384,281.09372891)(512.03125,281.09375)
\curveto(509.94790672,281.09372891)(508.35415831,280.39060461)(507.25,278.984375)
\curveto(506.15624384,277.57810742)(505.53124447,275.45310955)(505.375,272.609375)
\curveto(505.98957734,273.51561148)(506.76040991,274.20831913)(507.6875,274.6875)
\curveto(508.61457472,275.17706816)(509.63540703,275.42185958)(510.75,275.421875)
\curveto(513.09373691,275.42185958)(514.94269339,274.70831863)(516.296875,273.28125)
\curveto(517.66144067,271.86457147)(518.34373166,269.93228173)(518.34375,267.484375)
\curveto(518.34373166,265.08853658)(517.63539903,263.1666635)(516.21875,261.71875)
\curveto(514.80206853,260.27083306)(512.91665375,259.54687545)(510.5625,259.546875)
\curveto(507.86457547,259.54687545)(505.80207753,260.57812442)(504.375,262.640625)
\curveto(502.94791372,264.71353695)(502.23437277,267.71353395)(502.234375,271.640625)
\curveto(502.23437277,275.32810967)(503.10937189,278.26560673)(504.859375,280.453125)
\curveto(506.60936839,282.65101902)(508.95832438,283.74997625)(511.90625,283.75)
\curveto(512.69790397,283.74997625)(513.49477817,283.67185133)(514.296875,283.515625)
\curveto(515.10935989,283.35935164)(515.95310905,283.12497688)(516.828125,282.8125)
}
}
{
\newrgbcolor{curcolor}{0 0 0}
\pscustom[linestyle=none,fillstyle=solid,fillcolor=curcolor]
{
\newpath
\moveto(502.625,233.328125)
\lineto(517.625,233.328125)
\lineto(517.625,231.984375)
\lineto(509.15625,210)
\lineto(505.859375,210)
\lineto(513.828125,230.671875)
\lineto(502.625,230.671875)
\lineto(502.625,233.328125)
}
}
{
\newrgbcolor{curcolor}{0 0 0}
\pscustom[linestyle=none,fillstyle=solid,fillcolor=curcolor]
{
\newpath
\moveto(510.171875,171.078125)
\curveto(508.67186633,171.07811392)(507.48957584,170.67707266)(506.625,169.875)
\curveto(505.77082756,169.07290759)(505.34374466,167.96874203)(505.34375,166.5625)
\curveto(505.34374466,165.15624484)(505.77082756,164.05207928)(506.625,163.25)
\curveto(507.48957584,162.44791422)(508.67186633,162.04687295)(510.171875,162.046875)
\curveto(511.67186333,162.04687295)(512.85415381,162.44791422)(513.71875,163.25)
\curveto(514.58331875,164.06249594)(515.01560998,165.1666615)(515.015625,166.5625)
\curveto(515.01560998,167.96874203)(514.58331875,169.07290759)(513.71875,169.875)
\curveto(512.86457047,170.67707266)(511.68227998,171.07811392)(510.171875,171.078125)
\moveto(507.015625,172.421875)
\curveto(505.66145267,172.75519558)(504.60416206,173.38540328)(503.84375,174.3125)
\curveto(503.09374691,175.23956809)(502.71874728,176.3697753)(502.71875,177.703125)
\curveto(502.71874728,179.56768877)(503.38020495,181.04164563)(504.703125,182.125)
\curveto(506.0364523,183.20831013)(507.85936714,183.74997625)(510.171875,183.75)
\curveto(512.49477917,183.74997625)(514.31769402,183.20831013)(515.640625,182.125)
\curveto(516.9635247,181.04164563)(517.62498237,179.56768877)(517.625,177.703125)
\curveto(517.62498237,176.3697753)(517.24477442,175.23956809)(516.484375,174.3125)
\curveto(515.73435927,173.38540328)(514.68748531,172.75519558)(513.34375,172.421875)
\curveto(514.86456847,172.06769627)(516.04685895,171.37498863)(516.890625,170.34375)
\curveto(517.74477392,169.31249069)(518.17185683,168.05207528)(518.171875,166.5625)
\curveto(518.17185683,164.30207903)(517.47914919,162.56770577)(516.09375,161.359375)
\curveto(514.71873528,160.15104152)(512.74477892,159.54687545)(510.171875,159.546875)
\curveto(507.59895073,159.54687545)(505.61978605,160.15104152)(504.234375,161.359375)
\curveto(502.85937214,162.56770577)(502.17187283,164.30207903)(502.171875,166.5625)
\curveto(502.17187283,168.05207528)(502.59895573,169.31249069)(503.453125,170.34375)
\curveto(504.30728736,171.37498863)(505.49478617,172.06769627)(507.015625,172.421875)
\moveto(505.859375,177.40625)
\curveto(505.85936914,176.19790047)(506.23436877,175.25519308)(506.984375,174.578125)
\curveto(507.74478392,173.90102777)(508.80728286,173.56248644)(510.171875,173.5625)
\curveto(511.52603014,173.56248644)(512.58332075,173.90102777)(513.34375,174.578125)
\curveto(514.11456922,175.25519308)(514.4999855,176.19790047)(514.5,177.40625)
\curveto(514.4999855,178.61456472)(514.11456922,179.55727211)(513.34375,180.234375)
\curveto(512.58332075,180.91143742)(511.52603014,181.24997875)(510.171875,181.25)
\curveto(508.80728286,181.24997875)(507.74478392,180.91143742)(506.984375,180.234375)
\curveto(506.23436877,179.55727211)(505.85936914,178.61456472)(505.859375,177.40625)
}
}
{
\newrgbcolor{curcolor}{0 0 0}
\pscustom[linestyle=none,fillstyle=solid,fillcolor=curcolor]
{
\newpath
\moveto(53.515625,30.484375)
\lineto(53.515625,33.359375)
\curveto(54.30728736,32.98437202)(55.10936989,32.69791397)(55.921875,32.5)
\curveto(56.73436827,32.30208103)(57.53124247,32.2031228)(58.3125,32.203125)
\curveto(60.39582294,32.2031228)(61.98436302,32.90103877)(63.078125,34.296875)
\curveto(64.18227748,35.7031193)(64.81248519,37.8333255)(64.96875,40.6875)
\curveto(64.36456897,39.79165687)(63.59894473,39.10415756)(62.671875,38.625)
\curveto(61.74477992,38.14582519)(60.71873928,37.90624209)(59.59375,37.90625)
\curveto(57.26040941,37.90624209)(55.41145292,38.60936639)(54.046875,40.015625)
\curveto(52.69270564,41.43228023)(52.01562298,43.36456997)(52.015625,45.8125)
\curveto(52.01562298,48.20831513)(52.72395561,50.1301882)(54.140625,51.578125)
\curveto(55.55728611,53.02601864)(57.44270089,53.74997625)(59.796875,53.75)
\curveto(62.49477917,53.74997625)(64.55206878,52.71351895)(65.96875,50.640625)
\curveto(67.39581594,48.57810642)(68.10935689,45.57810942)(68.109375,41.640625)
\curveto(68.10935689,37.9635337)(67.23435777,35.02603664)(65.484375,32.828125)
\curveto(63.74477792,30.64062436)(61.40103027,29.54687545)(58.453125,29.546875)
\curveto(57.66145067,29.54687545)(56.85936814,29.62500038)(56.046875,29.78125)
\curveto(55.23436977,29.93750006)(54.39062061,30.17187483)(53.515625,30.484375)
\moveto(59.796875,40.375)
\curveto(61.21353045,40.37498963)(62.333321,40.85936414)(63.15625,41.828125)
\curveto(63.98956934,42.7968622)(64.40623559,44.12498587)(64.40625,45.8125)
\curveto(64.40623559,47.48956584)(63.98956934,48.81248119)(63.15625,49.78125)
\curveto(62.333321,50.76039591)(61.21353045,51.24997875)(59.796875,51.25)
\curveto(58.38019995,51.24997875)(57.25520108,50.76039591)(56.421875,49.78125)
\curveto(55.59895273,48.81248119)(55.18749481,47.48956584)(55.1875,45.8125)
\curveto(55.18749481,44.12498587)(55.59895273,42.7968622)(56.421875,41.828125)
\curveto(57.25520108,40.85936414)(58.38019995,40.37498963)(59.796875,40.375)
}
}
{
\newrgbcolor{curcolor}{0 0 0}
\pscustom[linewidth=2,linecolor=curcolor,linestyle=dashed,dash=8 8]
{
\newpath
\moveto(59.189408,539.999593)
\lineto(139.51115,539.999593)
}
}
{
\newrgbcolor{curcolor}{0 0 0}
\pscustom[linestyle=none,fillstyle=solid,fillcolor=curcolor]
{
\newpath
\moveto(129.04884464,544.84007524)
\lineto(142.1598974,540.01880891)
\lineto(129.04884392,535.19754364)
\curveto(131.143442,538.04401672)(131.13137288,541.93848592)(129.04884464,544.84007524)
\closepath
}
}
{
\newrgbcolor{curcolor}{0 0 0}
\pscustom[linestyle=none,fillstyle=solid,fillcolor=curcolor]
{
\newpath
\moveto(43.96875,532.65625)
\lineto(49.125,532.65625)
\lineto(49.125,550.453125)
\lineto(43.515625,549.328125)
\lineto(43.515625,552.203125)
\lineto(49.09375,553.328125)
\lineto(52.25,553.328125)
\lineto(52.25,532.65625)
\lineto(57.40625,532.65625)
\lineto(57.40625,530)
\lineto(43.96875,530)
\lineto(43.96875,532.65625)
}
}
{
\newrgbcolor{curcolor}{0 0 0}
\pscustom[linestyle=none,fillstyle=solid,fillcolor=curcolor]
{
\newpath
\moveto(506.140625,492.65625)
\lineto(517.15625,492.65625)
\lineto(517.15625,490)
\lineto(502.34375,490)
\lineto(502.34375,492.65625)
\curveto(503.54166313,493.89582944)(505.17186983,495.55728611)(507.234375,497.640625)
\curveto(509.30728236,499.73436527)(510.60936439,501.08332225)(511.140625,501.6875)
\curveto(512.15102952,502.82290384)(512.85415381,503.78123622)(513.25,504.5625)
\curveto(513.65623634,505.35415131)(513.85936114,506.1301922)(513.859375,506.890625)
\curveto(513.85936114,508.1301902)(513.42186158,509.14060586)(512.546875,509.921875)
\curveto(511.68227998,510.7031043)(510.55207278,511.09372891)(509.15625,511.09375)
\curveto(508.1666585,511.09372891)(507.11978455,510.92185408)(506.015625,510.578125)
\curveto(504.92187008,510.23435477)(503.74999625,509.71352195)(502.5,509.015625)
\lineto(502.5,512.203125)
\curveto(503.77082956,512.71351895)(504.95832838,513.09893523)(506.0625,513.359375)
\curveto(507.1666595,513.61976805)(508.17707516,513.74997625)(509.09375,513.75)
\curveto(511.51040516,513.74997625)(513.43748656,513.14581019)(514.875,511.9375)
\curveto(516.31248369,510.72914594)(517.03123297,509.11456422)(517.03125,507.09375)
\curveto(517.03123297,506.13540053)(516.84894148,505.22394311)(516.484375,504.359375)
\curveto(516.1301922,503.50519483)(515.47915119,502.49477917)(514.53125,501.328125)
\curveto(514.27081906,501.02603064)(513.44269489,500.15103152)(512.046875,498.703125)
\curveto(510.65103102,497.26561773)(508.68228298,495.24999475)(506.140625,492.65625)
}
}
{
\newrgbcolor{curcolor}{0 0 0}
\pscustom[linestyle=none,fillstyle=solid,fillcolor=curcolor]
{
\newpath
\moveto(493.96875,32.65625)
\lineto(499.125,32.65625)
\lineto(499.125,50.453125)
\lineto(493.515625,49.328125)
\lineto(493.515625,52.203125)
\lineto(499.09375,53.328125)
\lineto(502.25,53.328125)
\lineto(502.25,32.65625)
\lineto(507.40625,32.65625)
\lineto(507.40625,30)
\lineto(493.96875,30)
\lineto(493.96875,32.65625)
}
}
{
\newrgbcolor{curcolor}{0 0 0}
\pscustom[linestyle=none,fillstyle=solid,fillcolor=curcolor]
{
\newpath
\moveto(520.546875,51.25)
\curveto(518.92186645,51.24997875)(517.69790934,50.44789622)(516.875,48.84375)
\curveto(516.06249431,47.24998275)(515.65624472,44.84894348)(515.65625,41.640625)
\curveto(515.65624472,38.44269989)(516.06249431,36.04166062)(516.875,34.4375)
\curveto(517.69790934,32.84374716)(518.92186645,32.04687295)(520.546875,32.046875)
\curveto(522.18227986,32.04687295)(523.40623697,32.84374716)(524.21875,34.4375)
\curveto(525.041652,36.04166062)(525.45310992,38.44269989)(525.453125,41.640625)
\curveto(525.45310992,44.84894348)(525.041652,47.24998275)(524.21875,48.84375)
\curveto(523.40623697,50.44789622)(522.18227986,51.24997875)(520.546875,51.25)
\moveto(520.546875,53.75)
\curveto(523.16144555,53.74997625)(525.15623522,52.71351895)(526.53125,50.640625)
\curveto(527.91664913,48.57810642)(528.60935677,45.57810942)(528.609375,41.640625)
\curveto(528.60935677,37.71353395)(527.91664913,34.71353695)(526.53125,32.640625)
\curveto(525.15623522,30.57812442)(523.16144555,29.54687545)(520.546875,29.546875)
\curveto(517.93228411,29.54687545)(515.93228611,30.57812442)(514.546875,32.640625)
\curveto(513.1718722,34.71353695)(512.48437289,37.71353395)(512.484375,41.640625)
\curveto(512.48437289,45.57810942)(513.1718722,48.57810642)(514.546875,50.640625)
\curveto(515.93228611,52.71351895)(517.93228411,53.74997625)(520.546875,53.75)
}
}
{
\newrgbcolor{curcolor}{0 0 0}
\pscustom[linestyle=none,fillstyle=solid,fillcolor=curcolor]
{
\newpath
\moveto(142.01269531,194.25488281)
\lineto(142.01269531,181.78320312)
\lineto(144.63378906,181.78320312)
\curveto(146.84667052,181.78320134)(148.46515849,182.28450292)(149.48925781,183.28710938)
\curveto(150.52049498,184.28970925)(151.03611946,185.87238996)(151.03613281,188.03515625)
\curveto(151.03611946,190.18358357)(150.52049498,191.7555221)(149.48925781,192.75097656)
\curveto(148.46515849,193.75356698)(146.84667052,194.25486856)(144.63378906,194.25488281)
\lineto(142.01269531,194.25488281)
\moveto(139.84277344,196.03808594)
\lineto(144.30078125,196.03808594)
\curveto(147.40884444,196.0380699)(149.68976664,195.38995857)(151.14355469,194.09375)
\curveto(152.59731582,192.8046747)(153.32420311,190.78514546)(153.32421875,188.03515625)
\curveto(153.32420311,185.27082806)(152.59373509,183.24055666)(151.1328125,181.94433594)
\curveto(149.67186301,180.64811133)(147.39452154,180)(144.30078125,180)
\lineto(139.84277344,180)
\lineto(139.84277344,196.03808594)
}
}
{
\newrgbcolor{curcolor}{0 0 0}
\pscustom[linestyle=none,fillstyle=solid,fillcolor=curcolor]
{
\newpath
\moveto(156.68652344,192.03125)
\lineto(158.66308594,192.03125)
\lineto(158.66308594,180)
\lineto(156.68652344,180)
\lineto(156.68652344,192.03125)
\moveto(156.68652344,196.71484375)
\lineto(158.66308594,196.71484375)
\lineto(158.66308594,194.21191406)
\lineto(156.68652344,194.21191406)
\lineto(156.68652344,196.71484375)
}
}
{
\newrgbcolor{curcolor}{0 0 0}
\pscustom[linestyle=none,fillstyle=solid,fillcolor=curcolor]
{
\newpath
\moveto(168.87890625,196.71484375)
\lineto(168.87890625,195.07128906)
\lineto(166.98828125,195.07128906)
\curveto(166.27929131,195.07127399)(165.78515118,194.92804497)(165.50585938,194.64160156)
\curveto(165.23371944,194.35512887)(165.09765187,193.83950439)(165.09765625,193.09472656)
\lineto(165.09765625,192.03125)
\lineto(170.52246094,192.03125)
\lineto(170.52246094,192.86914062)
\curveto(170.52245113,194.20831913)(170.83397426,195.18227648)(171.45703125,195.79101562)
\curveto(171.65754114,195.99152047)(171.89744975,196.15981457)(172.17675781,196.29589844)
\curveto(172.73533954,196.57517874)(173.48371119,196.71482704)(174.421875,196.71484375)
\lineto(176.29101562,196.71484375)
\lineto(176.29101562,195.07128906)
\lineto(174.40039062,195.07128906)
\curveto(173.69139327,195.07127399)(173.19725314,194.92804497)(172.91796875,194.64160156)
\curveto(172.6458214,194.35512887)(172.50975383,193.83950439)(172.50976562,193.09472656)
\lineto(172.50976562,192.03125)
\lineto(179.921875,192.03125)
\lineto(179.921875,180)
\lineto(177.93457031,180)
\lineto(177.93457031,190.49511719)
\lineto(172.50976562,190.49511719)
\lineto(172.50976562,180)
\lineto(170.52246094,180)
\lineto(170.52246094,190.49511719)
\lineto(165.09765625,190.49511719)
\lineto(165.09765625,180)
\lineto(163.11035156,180)
\lineto(163.11035156,190.49511719)
\lineto(161.21972656,190.49511719)
\lineto(161.21972656,192.03125)
\lineto(163.11035156,192.03125)
\lineto(163.11035156,192.86914062)
\curveto(163.11034917,194.20831913)(163.42187229,195.18227648)(164.04492188,195.79101562)
\curveto(164.6679648,196.40688463)(165.65624506,196.71482704)(167.00976562,196.71484375)
\lineto(168.87890625,196.71484375)
\moveto(177.93457031,196.69335938)
\lineto(179.921875,196.69335938)
\lineto(179.921875,194.19042969)
\lineto(177.93457031,194.19042969)
\lineto(177.93457031,196.69335938)
}
}
{
\newrgbcolor{curcolor}{0 0 0}
\pscustom[linestyle=none,fillstyle=solid,fillcolor=curcolor]
{
\newpath
\moveto(192.71582031,191.56933594)
\lineto(192.71582031,189.72167969)
\curveto(192.15721639,190.02961237)(191.59504247,190.2587788)(191.02929688,190.40917969)
\curveto(190.47069464,190.5667212)(189.90494,190.64549717)(189.33203125,190.64550781)
\curveto(188.05012414,190.64549717)(187.05468243,190.23729445)(186.34570312,189.42089844)
\curveto(185.6367151,188.61164503)(185.28222326,187.4729743)(185.28222656,186.00488281)
\curveto(185.28222326,184.53677932)(185.6367151,183.39452786)(186.34570312,182.578125)
\curveto(187.05468243,181.76887844)(188.05012414,181.36425645)(189.33203125,181.36425781)
\curveto(189.90494,181.36425645)(190.47069464,181.43945169)(191.02929688,181.58984375)
\curveto(191.59504247,181.74739409)(192.15721639,181.98014125)(192.71582031,182.28808594)
\lineto(192.71582031,180.46191406)
\curveto(192.16437784,180.20410136)(191.59146175,180.01074218)(190.99707031,179.88183594)
\curveto(190.4098223,179.75292993)(189.78319533,179.68847687)(189.1171875,179.68847656)
\curveto(187.30533322,179.68847687)(185.86588154,180.25781224)(184.79882812,181.39648438)
\curveto(183.73176909,182.53515371)(183.19824097,184.07128499)(183.19824219,186.00488281)
\curveto(183.19824097,187.96711443)(183.73534981,189.51040716)(184.80957031,190.63476562)
\curveto(185.89094661,191.75910282)(187.36978628,192.32127674)(189.24609375,192.32128906)
\curveto(189.85480984,192.32127674)(190.44921029,192.25682368)(191.02929688,192.12792969)
\curveto(191.60936538,192.00617289)(192.17153929,191.81997516)(192.71582031,191.56933594)
}
}
{
\newrgbcolor{curcolor}{0 0 0}
\pscustom[linestyle=none,fillstyle=solid,fillcolor=curcolor]
{
\newpath
\moveto(195.97070312,184.74804688)
\lineto(195.97070312,192.03125)
\lineto(197.94726562,192.03125)
\lineto(197.94726562,184.82324219)
\curveto(197.94726178,183.68456663)(198.16926677,182.82877321)(198.61328125,182.25585938)
\curveto(199.05728671,181.69010248)(199.72330167,181.40722516)(200.61132812,181.40722656)
\curveto(201.67837784,181.40722516)(202.51984835,181.74739409)(203.13574219,182.42773438)
\curveto(203.75877941,183.10806981)(204.07030253,184.03547774)(204.0703125,185.20996094)
\lineto(204.0703125,192.03125)
\lineto(206.046875,192.03125)
\lineto(206.046875,180)
\lineto(204.0703125,180)
\lineto(204.0703125,181.84765625)
\curveto(203.5904853,181.11718638)(203.03189211,180.57291609)(202.39453125,180.21484375)
\curveto(201.76431525,179.86393243)(201.03026651,179.68847687)(200.19238281,179.68847656)
\curveto(198.81021665,179.68847687)(197.76106405,180.11816394)(197.04492188,180.97753906)
\curveto(196.32877381,181.83691223)(195.97070126,183.09374691)(195.97070312,184.74804688)
\moveto(200.94433594,192.32128906)
\lineto(200.94433594,192.32128906)
}
}
{
\newrgbcolor{curcolor}{0 0 0}
\pscustom[linestyle=none,fillstyle=solid,fillcolor=curcolor]
{
\newpath
\moveto(210.13964844,196.71484375)
\lineto(212.11621094,196.71484375)
\lineto(212.11621094,180)
\lineto(210.13964844,180)
\lineto(210.13964844,196.71484375)
}
}
{
\newrgbcolor{curcolor}{0 0 0}
\pscustom[linestyle=none,fillstyle=solid,fillcolor=curcolor]
{
\newpath
\moveto(218.19628906,195.44726562)
\lineto(218.19628906,192.03125)
\lineto(222.26757812,192.03125)
\lineto(222.26757812,190.49511719)
\lineto(218.19628906,190.49511719)
\lineto(218.19628906,183.96386719)
\curveto(218.19628503,182.98274441)(218.32877188,182.35253671)(218.59375,182.07324219)
\curveto(218.86588072,181.79394352)(219.41373173,181.65429522)(220.23730469,181.65429688)
\lineto(222.26757812,181.65429688)
\lineto(222.26757812,180)
\lineto(220.23730469,180)
\curveto(218.71190952,180)(217.6591762,180.28287732)(217.07910156,180.84863281)
\curveto(216.49902111,181.42154806)(216.20898233,182.45995848)(216.20898438,183.96386719)
\lineto(216.20898438,190.49511719)
\lineto(214.75878906,190.49511719)
\lineto(214.75878906,192.03125)
\lineto(216.20898438,192.03125)
\lineto(216.20898438,195.44726562)
\lineto(218.19628906,195.44726562)
}
}
{
\newrgbcolor{curcolor}{0 0 0}
\pscustom[linestyle=none,fillstyle=solid,fillcolor=curcolor]
{
\newpath
\moveto(235.16894531,186.50976562)
\lineto(235.16894531,185.54296875)
\lineto(226.08105469,185.54296875)
\curveto(226.16698883,184.18228748)(226.57519154,183.14387706)(227.30566406,182.42773438)
\curveto(228.04328903,181.71874828)(229.06737655,181.36425645)(230.37792969,181.36425781)
\curveto(231.13703594,181.36425645)(231.87108468,181.45735531)(232.58007812,181.64355469)
\curveto(233.29621347,181.82975077)(234.00519713,182.10904737)(234.70703125,182.48144531)
\lineto(234.70703125,180.61230469)
\curveto(233.99803568,180.31152313)(233.27114839,180.08235669)(232.52636719,179.92480469)
\curveto(231.78156654,179.76725284)(231.02603345,179.68847687)(230.25976562,179.68847656)
\curveto(228.34048926,179.68847687)(226.81868088,180.24707007)(225.69433594,181.36425781)
\curveto(224.57714667,182.48144283)(224.01855347,183.99250903)(224.01855469,185.89746094)
\curveto(224.01855347,187.86685411)(224.54850086,189.42805047)(225.60839844,190.58105469)
\curveto(226.67545186,191.7411992)(228.11132282,192.32127674)(229.91601562,192.32128906)
\curveto(231.53449648,192.32127674)(232.81281551,191.79849081)(233.75097656,190.75292969)
\curveto(234.69627717,189.71450851)(235.16893295,188.30012191)(235.16894531,186.50976562)
\moveto(233.19238281,187.08984375)
\curveto(233.17804952,188.17121579)(232.87368785,189.03417065)(232.27929688,189.67871094)
\curveto(231.6920484,190.32323186)(230.91145023,190.64549717)(229.9375,190.64550781)
\curveto(228.83462939,190.64549717)(227.95019017,190.33397404)(227.28417969,189.7109375)
\curveto(226.6253217,189.08788154)(226.24576479,188.21060377)(226.14550781,187.07910156)
\lineto(233.19238281,187.08984375)
\moveto(231.29101562,197.59570312)
\lineto(233.42871094,197.59570312)
\lineto(229.92675781,193.55664062)
\lineto(228.28320312,193.55664062)
\lineto(231.29101562,197.59570312)
}
}
{
\newrgbcolor{curcolor}{0 0 0}
\pscustom[linestyle=none,fillstyle=solid,fillcolor=curcolor]
{
}
}
{
\newrgbcolor{curcolor}{0 0 0}
\pscustom[linestyle=none,fillstyle=solid,fillcolor=curcolor]
{
\newpath
\moveto(253.33398438,190.20507812)
\lineto(253.33398438,196.71484375)
\lineto(255.31054688,196.71484375)
\lineto(255.31054688,180)
\lineto(253.33398438,180)
\lineto(253.33398438,181.8046875)
\curveto(252.91861022,181.08854058)(252.39224356,180.55501247)(251.75488281,180.20410156)
\curveto(251.1246667,179.8603517)(250.36555287,179.68847687)(249.47753906,179.68847656)
\curveto(248.02375834,179.68847687)(246.83853817,180.26855442)(245.921875,181.42871094)
\curveto(245.01236812,182.5888646)(244.55761597,184.1142537)(244.55761719,186.00488281)
\curveto(244.55761597,187.89549992)(245.01236812,189.42088902)(245.921875,190.58105469)
\curveto(246.83853817,191.7411992)(248.02375834,192.32127674)(249.47753906,192.32128906)
\curveto(250.36555287,192.32127674)(251.1246667,192.14582119)(251.75488281,191.79492188)
\curveto(252.39224356,191.45116042)(252.91861022,190.92121304)(253.33398438,190.20507812)
\moveto(246.59863281,186.00488281)
\curveto(246.59862956,184.55110222)(246.89582978,183.40885076)(247.49023438,182.578125)
\curveto(248.09179213,181.75455554)(248.91535901,181.34277209)(249.9609375,181.34277344)
\curveto(251.00650275,181.34277209)(251.83006964,181.75455554)(252.43164062,182.578125)
\curveto(253.03319344,183.40885076)(253.33397438,184.55110222)(253.33398438,186.00488281)
\curveto(253.33397438,187.4586514)(253.03319344,188.59732213)(252.43164062,189.42089844)
\curveto(251.83006964,190.25161735)(251.00650275,190.66698152)(249.9609375,190.66699219)
\curveto(248.91535901,190.66698152)(248.09179213,190.25161735)(247.49023438,189.42089844)
\curveto(246.89582978,188.59732213)(246.59862956,187.4586514)(246.59863281,186.00488281)
}
}
{
\newrgbcolor{curcolor}{0 0 0}
\pscustom[linestyle=none,fillstyle=solid,fillcolor=curcolor]
{
\newpath
\moveto(269.67285156,186.50976562)
\lineto(269.67285156,185.54296875)
\lineto(260.58496094,185.54296875)
\curveto(260.67089508,184.18228748)(261.07909779,183.14387706)(261.80957031,182.42773438)
\curveto(262.54719528,181.71874828)(263.5712828,181.36425645)(264.88183594,181.36425781)
\curveto(265.64094219,181.36425645)(266.37499093,181.45735531)(267.08398438,181.64355469)
\curveto(267.80011972,181.82975077)(268.50910338,182.10904737)(269.2109375,182.48144531)
\lineto(269.2109375,180.61230469)
\curveto(268.50194193,180.31152313)(267.77505464,180.08235669)(267.03027344,179.92480469)
\curveto(266.28547279,179.76725284)(265.5299397,179.68847687)(264.76367188,179.68847656)
\curveto(262.84439551,179.68847687)(261.32258713,180.24707007)(260.19824219,181.36425781)
\curveto(259.08105292,182.48144283)(258.52245972,183.99250903)(258.52246094,185.89746094)
\curveto(258.52245972,187.86685411)(259.05240711,189.42805047)(260.11230469,190.58105469)
\curveto(261.17935811,191.7411992)(262.61522907,192.32127674)(264.41992188,192.32128906)
\curveto(266.03840273,192.32127674)(267.31672176,191.79849081)(268.25488281,190.75292969)
\curveto(269.20018342,189.71450851)(269.6728392,188.30012191)(269.67285156,186.50976562)
\moveto(267.69628906,187.08984375)
\curveto(267.68195577,188.17121579)(267.3775941,189.03417065)(266.78320312,189.67871094)
\curveto(266.19595465,190.32323186)(265.41535648,190.64549717)(264.44140625,190.64550781)
\curveto(263.33853564,190.64549717)(262.45409642,190.33397404)(261.78808594,189.7109375)
\curveto(261.12922795,189.08788154)(260.74967104,188.21060377)(260.64941406,187.07910156)
\lineto(267.69628906,187.08984375)
}
}
{
\newrgbcolor{curcolor}{0 0 0}
\pscustom[linestyle=none,fillstyle=solid,fillcolor=curcolor]
{
\newpath
\moveto(280.58691406,191.67675781)
\lineto(280.58691406,189.80761719)
\curveto(280.02831113,190.09406543)(279.44823358,190.30890896)(278.84667969,190.45214844)
\curveto(278.24510979,190.59536701)(277.62206353,190.66698152)(276.97753906,190.66699219)
\curveto(275.99641412,190.66698152)(275.25878465,190.51659105)(274.76464844,190.21582031)
\curveto(274.27766584,189.91502915)(274.0341765,189.46385772)(274.03417969,188.86230469)
\curveto(274.0341765,188.40396295)(274.20963205,188.04230967)(274.56054688,187.77734375)
\curveto(274.91145427,187.51952373)(275.61685721,187.27245367)(276.67675781,187.03613281)
\lineto(277.35351562,186.88574219)
\curveto(278.75715354,186.58495435)(279.75259526,186.15884801)(280.33984375,185.60742188)
\curveto(280.9342347,185.06314598)(281.23143492,184.30045143)(281.23144531,183.31933594)
\curveto(281.23143492,182.20214624)(280.78742495,181.31770702)(279.89941406,180.66601562)
\curveto(279.01854651,180.0143229)(277.80468054,179.68847687)(276.2578125,179.68847656)
\curveto(275.61327648,179.68847687)(274.94010007,179.75292993)(274.23828125,179.88183594)
\curveto(273.54361709,180.00358073)(272.80956835,180.18977846)(272.03613281,180.44042969)
\lineto(272.03613281,182.48144531)
\curveto(272.76659964,182.10188592)(273.48632548,181.81542787)(274.1953125,181.62207031)
\curveto(274.90429281,181.43587096)(275.60611503,181.34277209)(276.30078125,181.34277344)
\curveto(277.23176445,181.34277209)(277.94790956,181.50032402)(278.44921875,181.81542969)
\curveto(278.95051273,182.13769317)(279.20116352,182.5888646)(279.20117188,183.16894531)
\curveto(279.20116352,183.70605098)(279.01854651,184.11783442)(278.65332031,184.40429688)
\curveto(278.29523994,184.69075052)(277.50389959,184.96646639)(276.27929688,185.23144531)
\lineto(275.59179688,185.39257812)
\curveto(274.36718398,185.65038497)(273.48274476,186.04426479)(272.93847656,186.57421875)
\curveto(272.39420418,187.11132101)(272.12206903,187.84536976)(272.12207031,188.77636719)
\curveto(272.12206903,189.9078677)(272.5231103,190.78156474)(273.32519531,191.39746094)
\curveto(274.12727536,192.01333434)(275.2659461,192.32127674)(276.74121094,192.32128906)
\curveto(277.47167306,192.32127674)(278.15917237,192.26756586)(278.80371094,192.16015625)
\curveto(279.44823358,192.05272232)(280.04263403,191.89158967)(280.58691406,191.67675781)
}
}
{
\newrgbcolor{curcolor}{0 0 0}
\pscustom[linestyle=none,fillstyle=solid,fillcolor=curcolor]
{
\newpath
\moveto(175.35644531,159.00976562)
\lineto(175.35644531,158.04296875)
\lineto(166.26855469,158.04296875)
\curveto(166.35448883,156.68228748)(166.76269154,155.64387706)(167.49316406,154.92773438)
\curveto(168.23078903,154.21874828)(169.25487655,153.86425645)(170.56542969,153.86425781)
\curveto(171.32453594,153.86425645)(172.05858468,153.95735531)(172.76757812,154.14355469)
\curveto(173.48371347,154.32975077)(174.19269713,154.60904737)(174.89453125,154.98144531)
\lineto(174.89453125,153.11230469)
\curveto(174.18553568,152.81152313)(173.45864839,152.58235669)(172.71386719,152.42480469)
\curveto(171.96906654,152.26725284)(171.21353345,152.18847687)(170.44726562,152.18847656)
\curveto(168.52798926,152.18847687)(167.00618088,152.74707007)(165.88183594,153.86425781)
\curveto(164.76464667,154.98144283)(164.20605347,156.49250903)(164.20605469,158.39746094)
\curveto(164.20605347,160.36685411)(164.73600086,161.92805047)(165.79589844,163.08105469)
\curveto(166.86295186,164.2411992)(168.29882282,164.82127674)(170.10351562,164.82128906)
\curveto(171.72199648,164.82127674)(173.00031551,164.29849081)(173.93847656,163.25292969)
\curveto(174.88377717,162.21450851)(175.35643295,160.80012191)(175.35644531,159.00976562)
\moveto(173.37988281,159.58984375)
\curveto(173.36554952,160.67121579)(173.06118785,161.53417065)(172.46679688,162.17871094)
\curveto(171.8795484,162.82323186)(171.09895023,163.14549717)(170.125,163.14550781)
\curveto(169.02212939,163.14549717)(168.13769017,162.83397404)(167.47167969,162.2109375)
\curveto(166.8128217,161.58788154)(166.43326479,160.71060377)(166.33300781,159.57910156)
\lineto(173.37988281,159.58984375)
}
}
{
\newrgbcolor{curcolor}{0 0 0}
\pscustom[linestyle=none,fillstyle=solid,fillcolor=curcolor]
{
\newpath
\moveto(188.6015625,159.76171875)
\lineto(188.6015625,152.5)
\lineto(186.625,152.5)
\lineto(186.625,159.69726562)
\curveto(186.6249899,160.83592916)(186.40298492,161.68814185)(185.95898438,162.25390625)
\curveto(185.51496497,162.81965114)(184.84895001,163.10252846)(183.9609375,163.10253906)
\curveto(182.89387384,163.10252846)(182.05240333,162.76235953)(181.43652344,162.08203125)
\curveto(180.82063373,161.40168381)(180.51269133,160.47427588)(180.51269531,159.29980469)
\lineto(180.51269531,152.5)
\lineto(178.52539062,152.5)
\lineto(178.52539062,164.53125)
\lineto(180.51269531,164.53125)
\lineto(180.51269531,162.66210938)
\curveto(180.9853471,163.38540578)(181.54035957,163.92609534)(182.17773438,164.28417969)
\curveto(182.82225933,164.64224046)(183.56346953,164.82127674)(184.40136719,164.82128906)
\curveto(185.78351939,164.82127674)(186.82909126,164.39158967)(187.53808594,163.53222656)
\curveto(188.24705859,162.68000284)(188.60155043,161.42316816)(188.6015625,159.76171875)
}
}
{
\newrgbcolor{curcolor}{0 0 0}
\pscustom[linestyle=none,fillstyle=solid,fillcolor=curcolor]
{
\newpath
\moveto(202.56640625,159.76171875)
\lineto(202.56640625,152.5)
\lineto(200.58984375,152.5)
\lineto(200.58984375,159.69726562)
\curveto(200.58983365,160.83592916)(200.36782867,161.68814185)(199.92382812,162.25390625)
\curveto(199.47980872,162.81965114)(198.81379376,163.10252846)(197.92578125,163.10253906)
\curveto(196.85871759,163.10252846)(196.01724708,162.76235953)(195.40136719,162.08203125)
\curveto(194.78547748,161.40168381)(194.47753508,160.47427588)(194.47753906,159.29980469)
\lineto(194.47753906,152.5)
\lineto(192.49023438,152.5)
\lineto(192.49023438,164.53125)
\lineto(194.47753906,164.53125)
\lineto(194.47753906,162.66210938)
\curveto(194.95019085,163.38540578)(195.50520332,163.92609534)(196.14257812,164.28417969)
\curveto(196.78710308,164.64224046)(197.52831328,164.82127674)(198.36621094,164.82128906)
\curveto(199.74836314,164.82127674)(200.79393501,164.39158967)(201.50292969,163.53222656)
\curveto(202.21190234,162.68000284)(202.56639418,161.42316816)(202.56640625,159.76171875)
}
}
{
\newrgbcolor{curcolor}{0 0 0}
\pscustom[linestyle=none,fillstyle=solid,fillcolor=curcolor]
{
\newpath
\moveto(216.82128906,159.00976562)
\lineto(216.82128906,158.04296875)
\lineto(207.73339844,158.04296875)
\curveto(207.81933258,156.68228748)(208.22753529,155.64387706)(208.95800781,154.92773438)
\curveto(209.69563278,154.21874828)(210.7197203,153.86425645)(212.03027344,153.86425781)
\curveto(212.78937969,153.86425645)(213.52342843,153.95735531)(214.23242188,154.14355469)
\curveto(214.94855722,154.32975077)(215.65754088,154.60904737)(216.359375,154.98144531)
\lineto(216.359375,153.11230469)
\curveto(215.65037943,152.81152313)(214.92349214,152.58235669)(214.17871094,152.42480469)
\curveto(213.43391029,152.26725284)(212.6783772,152.18847687)(211.91210938,152.18847656)
\curveto(209.99283301,152.18847687)(208.47102463,152.74707007)(207.34667969,153.86425781)
\curveto(206.22949042,154.98144283)(205.67089722,156.49250903)(205.67089844,158.39746094)
\curveto(205.67089722,160.36685411)(206.20084461,161.92805047)(207.26074219,163.08105469)
\curveto(208.32779561,164.2411992)(209.76366657,164.82127674)(211.56835938,164.82128906)
\curveto(213.18684023,164.82127674)(214.46515926,164.29849081)(215.40332031,163.25292969)
\curveto(216.34862092,162.21450851)(216.8212767,160.80012191)(216.82128906,159.00976562)
\moveto(214.84472656,159.58984375)
\curveto(214.83039327,160.67121579)(214.5260316,161.53417065)(213.93164062,162.17871094)
\curveto(213.34439215,162.82323186)(212.56379398,163.14549717)(211.58984375,163.14550781)
\curveto(210.48697314,163.14549717)(209.60253392,162.83397404)(208.93652344,162.2109375)
\curveto(208.27766545,161.58788154)(207.89810854,160.71060377)(207.79785156,159.57910156)
\lineto(214.84472656,159.58984375)
}
}
{
\newrgbcolor{curcolor}{0 0 0}
\pscustom[linestyle=none,fillstyle=solid,fillcolor=curcolor]
{
\newpath
\moveto(229.43261719,162.22167969)
\curveto(229.92674588,163.10968991)(230.5175656,163.76496269)(231.20507812,164.1875)
\curveto(231.89256422,164.61001393)(232.70180821,164.82127674)(233.6328125,164.82128906)
\curveto(234.88605081,164.82127674)(235.85284672,164.38084749)(236.53320312,163.5)
\curveto(237.21352245,162.62629196)(237.55369138,161.38019945)(237.55371094,159.76171875)
\lineto(237.55371094,152.5)
\lineto(235.56640625,152.5)
\lineto(235.56640625,159.69726562)
\curveto(235.56638868,160.85025207)(235.36228732,161.70604548)(234.95410156,162.26464844)
\curveto(234.54588188,162.82323186)(233.92283563,163.10252846)(233.08496094,163.10253906)
\curveto(232.06085833,163.10252846)(231.25161434,162.76235953)(230.65722656,162.08203125)
\curveto(230.06281345,161.40168381)(229.76561323,160.47427588)(229.765625,159.29980469)
\lineto(229.765625,152.5)
\lineto(227.77832031,152.5)
\lineto(227.77832031,159.69726562)
\curveto(227.77831053,160.85741352)(227.57420917,161.71320693)(227.16601562,162.26464844)
\curveto(226.75780373,162.82323186)(226.12759603,163.10252846)(225.27539062,163.10253906)
\curveto(224.26561873,163.10252846)(223.4635362,162.7587788)(222.86914062,162.07128906)
\curveto(222.2747353,161.39094163)(221.97753508,160.46711443)(221.97753906,159.29980469)
\lineto(221.97753906,152.5)
\lineto(219.99023438,152.5)
\lineto(219.99023438,164.53125)
\lineto(221.97753906,164.53125)
\lineto(221.97753906,162.66210938)
\curveto(222.4287065,163.39972868)(222.96939606,163.94399897)(223.59960938,164.29492188)
\curveto(224.22981147,164.64582119)(224.97818312,164.82127674)(225.84472656,164.82128906)
\curveto(226.71841575,164.82127674)(227.45962595,164.59927175)(228.06835938,164.15527344)
\curveto(228.6842341,163.71125181)(229.13898625,163.0667212)(229.43261719,162.22167969)
}
}
{
\newrgbcolor{curcolor}{0 0 0}
\pscustom[linestyle=none,fillstyle=solid,fillcolor=curcolor]
{
\newpath
\moveto(241.50683594,164.53125)
\lineto(243.48339844,164.53125)
\lineto(243.48339844,152.5)
\lineto(241.50683594,152.5)
\lineto(241.50683594,164.53125)
\moveto(241.50683594,169.21484375)
\lineto(243.48339844,169.21484375)
\lineto(243.48339844,166.71191406)
\lineto(241.50683594,166.71191406)
\lineto(241.50683594,169.21484375)
}
}
{
\newrgbcolor{curcolor}{0 0 0}
\pscustom[linestyle=none,fillstyle=solid,fillcolor=curcolor]
{
\newpath
\moveto(255.27832031,164.17675781)
\lineto(255.27832031,162.30761719)
\curveto(254.71971738,162.59406543)(254.13963983,162.80890896)(253.53808594,162.95214844)
\curveto(252.93651604,163.09536701)(252.31346978,163.16698152)(251.66894531,163.16699219)
\curveto(250.68782037,163.16698152)(249.9501909,163.01659105)(249.45605469,162.71582031)
\curveto(248.96907209,162.41502915)(248.72558275,161.96385772)(248.72558594,161.36230469)
\curveto(248.72558275,160.90396295)(248.9010383,160.54230967)(249.25195312,160.27734375)
\curveto(249.60286052,160.01952373)(250.30826346,159.77245367)(251.36816406,159.53613281)
\lineto(252.04492188,159.38574219)
\curveto(253.44855979,159.08495435)(254.44400151,158.65884801)(255.03125,158.10742188)
\curveto(255.62564095,157.56314598)(255.92284117,156.80045143)(255.92285156,155.81933594)
\curveto(255.92284117,154.70214624)(255.4788312,153.81770702)(254.59082031,153.16601562)
\curveto(253.70995276,152.5143229)(252.49608679,152.18847687)(250.94921875,152.18847656)
\curveto(250.30468273,152.18847687)(249.63150632,152.25292993)(248.9296875,152.38183594)
\curveto(248.23502334,152.50358073)(247.5009746,152.68977846)(246.72753906,152.94042969)
\lineto(246.72753906,154.98144531)
\curveto(247.45800589,154.60188592)(248.17773173,154.31542787)(248.88671875,154.12207031)
\curveto(249.59569906,153.93587096)(250.29752128,153.84277209)(250.9921875,153.84277344)
\curveto(251.9231707,153.84277209)(252.63931581,154.00032402)(253.140625,154.31542969)
\curveto(253.64191898,154.63769317)(253.89256977,155.0888646)(253.89257812,155.66894531)
\curveto(253.89256977,156.20605098)(253.70995276,156.61783442)(253.34472656,156.90429688)
\curveto(252.98664619,157.19075052)(252.19530584,157.46646639)(250.97070312,157.73144531)
\lineto(250.28320312,157.89257812)
\curveto(249.05859023,158.15038497)(248.17415101,158.54426479)(247.62988281,159.07421875)
\curveto(247.08561043,159.61132101)(246.81347528,160.34536976)(246.81347656,161.27636719)
\curveto(246.81347528,162.4078677)(247.21451655,163.28156474)(248.01660156,163.89746094)
\curveto(248.81868161,164.51333434)(249.95735235,164.82127674)(251.43261719,164.82128906)
\curveto(252.16307931,164.82127674)(252.85057862,164.76756586)(253.49511719,164.66015625)
\curveto(254.13963983,164.55272232)(254.73404028,164.39158967)(255.27832031,164.17675781)
}
}
\end{pspicture}
		
		\end{center}
		
		\begin{enumerate}
		  \item Fond d'écran
		  \item Carte sélectionnée
		  \item Carte précédente
		  \item Carte suivante
		  \item Nom de la carte sélectionnée
		  \item Liste déroulante ``Type de partie''
		  \item Liste déroulante ``Nombre d'ennemis''
		  \item Liste déroulante ``Difficulté des ennemis''
		  \item Bouton ``\hyperlink{Page d'accueil}{Retour}''
		  \item Bouton ``Lancer'' 
		\end{enumerate}
		
		\subsubsection{Description des zones}
		
			\begin{tabular}{|c|c|c|c|c|} \hline
				Numéro de zone & Type  & Description & Evènement &	Règle \\\hline
				2,3,4 & Sélecteur & Permet de sélectionner la carte voulue & Cliqué & RG3-01 \\\hline
				5 & Label & Affiche le nom de la carte choisie & RG3-01 & RG3-02 \\\hline
				6 & Liste déroulante & Permet de choisir le type de partie & Cliqué & RG3-03 \\\hline
				7 & Liste déroulante & Permet de choisir le nombre d'ennemis & Cliqué & RG3-04 \\
				  &                  & (1, 2 ou 3). & & \\\hline
				8 & Liste      & Permet de choisir la difficulté  & Cliqué & RG3-05 \\
				  & déroulante & de l'intelligence artificielle   &        & \\
				  &            & (facile/moyenne/difficile). & & \\\hline				  
				9 & Bouton & Affiche l'écran d'accueil & Cliqué & RG3-06 \\\hline
				10& Bouton & Lance une partie selon les paramètres choisis & Cliqué & RG3-07 \\\hline
			\end{tabular}

\newpage
			
		\subsubsection{Description des règles}

			\underline{RG3-01 :}
				\begin{quote}
					Décaler les cartes vers la droite ou la gauche.\\
					RG3-02\\
				\end{quote}
				
				
			\underline{RG3-02 :}
				\begin{quote}
					Afficher le nom de la carte choisie.\\
				\end{quote}


			\underline{RG3-03 :}
				\begin{quote}
					Dérouler la liste.\\
				\end{quote}


			\underline{RG3-04 :}
				\begin{quote}
					Dérouler la liste.\\				
				\end{quote}	
				

			\underline{RG3-05 :}
				\begin{quote}
					Dérouler la liste.\\				
				\end{quote}	


			\underline{RG3-06 :}
				\begin{quote}
					Afficher la page d'accueil%
						\footnote[1]{
							\hyperlink{Page d'accueil}{Page d'accueil}
							\og voir section \ref{accueil}, page \pageref{accueil}.\fg
						}.\\
					Cacher la page de création d'une partie solitaire.\\
				\end{quote}


			\underline{RG3-03 :}
				\begin{quote}
					 Lancer une partie selon les paramètres choisis.\\
					 Supprimer la page de création d'une partie solitaire.\\
				\end{quote}	


\newpage

	\subsection{Création d'un compte multi-joueurs}
	
		\hypertarget{Creation compte multi-joueurs}{}
		\label{Creation compte multi-joueurs}
	
		\begin{center}
			%LaTeX with PSTricks extensions
%%Creator: inkscape 0.48.0
%%Please note this file requires PSTricks extensions
\psset{xunit=.4pt,yunit=.4pt,runit=.4pt}
\begin{pspicture}(560,600)
{
\newrgbcolor{curcolor}{1 1 1}
\pscustom[linestyle=none,fillstyle=solid,fillcolor=curcolor]
{
\newpath
\moveto(133.12401581,597.52220274)
\lineto(426.87598419,597.52220274)
\curveto(443.85397169,597.52220274)(457.52217102,583.85400341)(457.52217102,566.87601591)
\lineto(457.52217102,33.12401701)
\curveto(457.52217102,16.14602951)(443.85397169,2.47783018)(426.87598419,2.47783018)
\lineto(133.12401581,2.47783018)
\curveto(116.14602831,2.47783018)(102.47782898,16.14602951)(102.47782898,33.12401701)
\lineto(102.47782898,566.87601591)
\curveto(102.47782898,583.85400341)(116.14602831,597.52220274)(133.12401581,597.52220274)
\closepath
}
}
{
\newrgbcolor{curcolor}{0 0 0}
\pscustom[linewidth=4.95566034,linecolor=curcolor]
{
\newpath
\moveto(133.12401581,597.52220274)
\lineto(426.87598419,597.52220274)
\curveto(443.85397169,597.52220274)(457.52217102,583.85400341)(457.52217102,566.87601591)
\lineto(457.52217102,33.12401701)
\curveto(457.52217102,16.14602951)(443.85397169,2.47783018)(426.87598419,2.47783018)
\lineto(133.12401581,2.47783018)
\curveto(116.14602831,2.47783018)(102.47782898,16.14602951)(102.47782898,33.12401701)
\lineto(102.47782898,566.87601591)
\curveto(102.47782898,583.85400341)(116.14602831,597.52220274)(133.12401581,597.52220274)
\closepath
}
}
{
\newrgbcolor{curcolor}{1 1 1}
\pscustom[linestyle=none,fillstyle=solid,fillcolor=curcolor]
{
\newpath
\moveto(161.48632431,459.92251707)
\lineto(398.41664505,459.92251707)
\curveto(421.39583781,459.92251707)(439.89533234,441.42302254)(439.89533234,418.44382978)
\lineto(439.89533234,151.43260694)
\curveto(439.89533234,128.45341419)(421.39583781,109.95391966)(398.41664505,109.95391966)
\lineto(161.48632431,109.95391966)
\curveto(138.50713155,109.95391966)(120.00763702,128.45341419)(120.00763702,151.43260694)
\lineto(120.00763702,418.44382978)
\curveto(120.00763702,441.42302254)(138.50713155,459.92251707)(161.48632431,459.92251707)
\closepath
}
}
{
\newrgbcolor{curcolor}{0 0 0}
\pscustom[linewidth=2,linecolor=curcolor]
{
\newpath
\moveto(161.48632431,459.92251707)
\lineto(398.41664505,459.92251707)
\curveto(421.39583781,459.92251707)(439.89533234,441.42302254)(439.89533234,418.44382978)
\lineto(439.89533234,151.43260694)
\curveto(439.89533234,128.45341419)(421.39583781,109.95391966)(398.41664505,109.95391966)
\lineto(161.48632431,109.95391966)
\curveto(138.50713155,109.95391966)(120.00763702,128.45341419)(120.00763702,151.43260694)
\lineto(120.00763702,418.44382978)
\curveto(120.00763702,441.42302254)(138.50713155,459.92251707)(161.48632431,459.92251707)
\closepath
}
}
{
\newrgbcolor{curcolor}{0 0 0}
\pscustom[linestyle=none,fillstyle=solid,fillcolor=curcolor]
{
\newpath
\moveto(336.8671875,140.26532103)
\lineto(330.1875,157.76141478)
\lineto(332.66015625,157.76141478)
\lineto(338.203125,143.03094603)
\lineto(343.7578125,157.76141478)
\lineto(346.21875,157.76141478)
\lineto(339.55078125,140.26532103)
\lineto(336.8671875,140.26532103)
}
}
{
\newrgbcolor{curcolor}{0 0 0}
\pscustom[linestyle=none,fillstyle=solid,fillcolor=curcolor]
{
\newpath
\moveto(352.7578125,146.86297728)
\curveto(351.01561852,146.86297068)(349.80858847,146.66375213)(349.13671875,146.26532103)
\curveto(348.46483982,145.86687792)(348.12890265,145.1871911)(348.12890625,144.22625853)
\curveto(348.12890265,143.46063033)(348.3789024,142.85125594)(348.87890625,142.39813353)
\curveto(349.38671389,141.95281934)(350.07421321,141.73016331)(350.94140625,141.73016478)
\curveto(352.13671114,141.73016331)(353.09374144,142.15203789)(353.8125,142.99578978)
\curveto(354.53905249,143.84734869)(354.90233338,144.97625381)(354.90234375,146.38250853)
\lineto(354.90234375,146.86297728)
\lineto(352.7578125,146.86297728)
\moveto(357.05859375,147.75360228)
\lineto(357.05859375,140.26532103)
\lineto(354.90234375,140.26532103)
\lineto(354.90234375,142.25750853)
\curveto(354.41014637,141.46063233)(353.79686573,140.87078917)(353.0625,140.48797728)
\curveto(352.3281172,140.11297743)(351.4296806,139.92547762)(350.3671875,139.92547728)
\curveto(349.02343301,139.92547762)(347.95312158,140.30047724)(347.15625,141.05047728)
\curveto(346.36718566,141.80828823)(345.97265481,142.82000597)(345.97265625,144.08563353)
\curveto(345.97265481,145.56219073)(346.46484182,146.67547087)(347.44921875,147.42547728)
\curveto(348.44140234,148.17546937)(349.91796336,148.55046899)(351.87890625,148.55047728)
\lineto(354.90234375,148.55047728)
\lineto(354.90234375,148.76141478)
\curveto(354.90233338,149.75359279)(354.57420871,150.51921702)(353.91796875,151.05828978)
\curveto(353.26952251,151.60515344)(352.35546093,151.87859066)(351.17578125,151.87860228)
\curveto(350.42577536,151.87859066)(349.69530734,151.788747)(348.984375,151.60907103)
\curveto(348.27343376,151.42937236)(347.58984069,151.15984138)(346.93359375,150.80047728)
\lineto(346.93359375,152.79266478)
\curveto(347.72265306,153.09733944)(348.48827729,153.32390172)(349.23046875,153.47235228)
\curveto(349.97265081,153.62858891)(350.69530634,153.70671383)(351.3984375,153.70672728)
\curveto(353.29686623,153.70671383)(354.71483357,153.21452683)(355.65234375,152.23016478)
\curveto(356.58983169,151.24577879)(357.05858122,149.75359279)(357.05859375,147.75360228)
}
}
{
\newrgbcolor{curcolor}{0 0 0}
\pscustom[linestyle=none,fillstyle=solid,fillcolor=curcolor]
{
\newpath
\moveto(361.51171875,158.49969603)
\lineto(363.66796875,158.49969603)
\lineto(363.66796875,140.26532103)
\lineto(361.51171875,140.26532103)
\lineto(361.51171875,158.49969603)
}
}
{
\newrgbcolor{curcolor}{0 0 0}
\pscustom[linestyle=none,fillstyle=solid,fillcolor=curcolor]
{
\newpath
\moveto(368.16796875,153.39032103)
\lineto(370.32421875,153.39032103)
\lineto(370.32421875,140.26532103)
\lineto(368.16796875,140.26532103)
\lineto(368.16796875,153.39032103)
\moveto(368.16796875,158.49969603)
\lineto(370.32421875,158.49969603)
\lineto(370.32421875,155.76922728)
\lineto(368.16796875,155.76922728)
\lineto(368.16796875,158.49969603)
}
}
{
\newrgbcolor{curcolor}{0 0 0}
\pscustom[linestyle=none,fillstyle=solid,fillcolor=curcolor]
{
\newpath
\moveto(383.4609375,151.39813353)
\lineto(383.4609375,158.49969603)
\lineto(385.6171875,158.49969603)
\lineto(385.6171875,140.26532103)
\lineto(383.4609375,140.26532103)
\lineto(383.4609375,142.23407103)
\curveto(383.00780205,141.45281984)(382.43358388,140.87078917)(381.73828125,140.48797728)
\curveto(381.05077276,140.11297743)(380.22264859,139.92547762)(379.25390625,139.92547728)
\curveto(377.66796364,139.92547762)(376.37499619,140.55828948)(375.375,141.82391478)
\curveto(374.38281068,143.08953695)(373.88671743,144.75359779)(373.88671875,146.81610228)
\curveto(373.88671743,148.87859366)(374.38281068,150.5426545)(375.375,151.80828978)
\curveto(376.37499619,153.07390197)(377.66796364,153.70671383)(379.25390625,153.70672728)
\curveto(380.22264859,153.70671383)(381.05077276,153.51530778)(381.73828125,153.13250853)
\curveto(382.43358388,152.75749603)(383.00780205,152.17937161)(383.4609375,151.39813353)
\moveto(376.11328125,146.81610228)
\curveto(376.1132777,145.23015981)(376.43749612,143.98406731)(377.0859375,143.07782103)
\curveto(377.74218232,142.17938161)(378.64061892,141.73016331)(379.78125,141.73016478)
\curveto(380.92186664,141.73016331)(381.82030324,142.17938161)(382.4765625,143.07782103)
\curveto(383.13280193,143.98406731)(383.4609266,145.23015981)(383.4609375,146.81610228)
\curveto(383.4609266,148.40203164)(383.13280193,149.6442179)(382.4765625,150.54266478)
\curveto(381.82030324,151.44890359)(380.92186664,151.90202814)(379.78125,151.90203978)
\curveto(378.64061892,151.90202814)(377.74218232,151.44890359)(377.0859375,150.54266478)
\curveto(376.43749612,149.6442179)(376.1132777,148.40203164)(376.11328125,146.81610228)
}
}
{
\newrgbcolor{curcolor}{0 0 0}
\pscustom[linestyle=none,fillstyle=solid,fillcolor=curcolor]
{
\newpath
\moveto(401.28515625,147.36688353)
\lineto(401.28515625,146.31219603)
\lineto(391.37109375,146.31219603)
\curveto(391.46484008,144.82781646)(391.91015214,143.6950051)(392.70703125,142.91375853)
\curveto(393.51171304,142.14031915)(394.62889942,141.75360079)(396.05859375,141.75360228)
\curveto(396.88670966,141.75360079)(397.68749011,141.85516319)(398.4609375,142.05828978)
\curveto(399.24217605,142.26141278)(400.01561278,142.56609997)(400.78125,142.97235228)
\lineto(400.78125,140.93328978)
\curveto(400.00780029,140.60516444)(399.21483233,140.35516469)(398.40234375,140.18328978)
\curveto(397.58983396,140.01141503)(396.76561603,139.92547762)(395.9296875,139.92547728)
\curveto(393.83593146,139.92547762)(392.17577687,140.53485201)(390.94921875,141.75360228)
\curveto(389.73046682,142.97234957)(389.12109243,144.62078542)(389.12109375,146.69891478)
\curveto(389.12109243,148.84734369)(389.69921685,150.55046699)(390.85546875,151.80828978)
\curveto(392.01952703,153.07390197)(393.58593171,153.70671383)(395.5546875,153.70672728)
\curveto(397.32030298,153.70671383)(398.71483283,153.1364019)(399.73828125,151.99578978)
\curveto(400.76951828,150.86296668)(401.28514276,149.31999947)(401.28515625,147.36688353)
\moveto(399.12890625,147.99969603)
\curveto(399.11326993,149.17937461)(398.78123902,150.12077992)(398.1328125,150.82391478)
\curveto(397.4921778,151.52702851)(396.64061616,151.87859066)(395.578125,151.87860228)
\curveto(394.37499342,151.87859066)(393.41015064,151.53874725)(392.68359375,150.85907103)
\curveto(391.96483958,150.17937361)(391.5507775,149.22234332)(391.44140625,147.98797728)
\lineto(399.12890625,147.99969603)
}
}
{
\newrgbcolor{curcolor}{0 0 0}
\pscustom[linestyle=none,fillstyle=solid,fillcolor=curcolor]
{
\newpath
\moveto(412.4296875,151.37469603)
\curveto(412.18749037,151.51530978)(411.92186564,151.61687217)(411.6328125,151.67938353)
\curveto(411.35155371,151.74968454)(411.03905402,151.78484076)(410.6953125,151.78485228)
\curveto(409.47655559,151.78484076)(408.53905652,151.38640365)(407.8828125,150.58953978)
\curveto(407.23437033,149.80046774)(406.9101519,148.66375013)(406.91015625,147.17938353)
\lineto(406.91015625,140.26532103)
\lineto(404.7421875,140.26532103)
\lineto(404.7421875,153.39032103)
\lineto(406.91015625,153.39032103)
\lineto(406.91015625,151.35125853)
\curveto(407.36327645,152.14812164)(407.95311961,152.7379648)(408.6796875,153.12078978)
\curveto(409.40624316,153.51140153)(410.28905477,153.70671383)(411.328125,153.70672728)
\curveto(411.47655359,153.70671383)(411.64061592,153.6949951)(411.8203125,153.67157103)
\curveto(411.99999056,153.65593263)(412.19920911,153.62858891)(412.41796875,153.58953978)
\lineto(412.4296875,151.37469603)
}
}
{
\newrgbcolor{curcolor}{0 0 0}
\pscustom[linestyle=none,fillstyle=solid,fillcolor=curcolor]
{
\newpath
\moveto(158.28671265,155.30978514)
\lineto(155.07577515,146.60275389)
\lineto(161.5093689,146.60275389)
\lineto(158.28671265,155.30978514)
\moveto(156.95077515,157.64181639)
\lineto(159.6343689,157.64181639)
\lineto(166.30233765,140.14572264)
\lineto(163.84140015,140.14572264)
\lineto(162.24765015,144.63400389)
\lineto(154.3609314,144.63400389)
\lineto(152.7671814,140.14572264)
\lineto(150.27108765,140.14572264)
\lineto(156.95077515,157.64181639)
}
}
{
\newrgbcolor{curcolor}{0 0 0}
\pscustom[linestyle=none,fillstyle=solid,fillcolor=curcolor]
{
\newpath
\moveto(179.66171265,148.06759764)
\lineto(179.66171265,140.14572264)
\lineto(177.50546265,140.14572264)
\lineto(177.50546265,147.99728514)
\curveto(177.50545163,149.23946354)(177.26326437,150.16915011)(176.77890015,150.78634764)
\curveto(176.29451534,151.40352388)(175.56795357,151.71211732)(174.59921265,151.71212889)
\curveto(173.4351432,151.71211732)(172.51717537,151.34102394)(171.8453064,150.59884764)
\curveto(171.17342671,149.85665043)(170.83748955,148.84493269)(170.8374939,147.56369139)
\lineto(170.8374939,140.14572264)
\lineto(168.66952515,140.14572264)
\lineto(168.66952515,153.27072264)
\lineto(170.8374939,153.27072264)
\lineto(170.8374939,151.23166014)
\curveto(171.35311403,152.02071076)(171.95858218,152.61055392)(172.65390015,153.00119139)
\curveto(173.35701828,153.39180314)(174.16561122,153.58711545)(175.0796814,153.58712889)
\curveto(176.5874838,153.58711545)(177.72810766,153.11836591)(178.5015564,152.18087889)
\curveto(179.27498111,151.25118028)(179.66169947,149.8800879)(179.66171265,148.06759764)
}
}
{
\newrgbcolor{curcolor}{0 0 0}
\pscustom[linestyle=none,fillstyle=solid,fillcolor=curcolor]
{
\newpath
\moveto(194.89608765,148.06759764)
\lineto(194.89608765,140.14572264)
\lineto(192.73983765,140.14572264)
\lineto(192.73983765,147.99728514)
\curveto(192.73982663,149.23946354)(192.49763937,150.16915011)(192.01327515,150.78634764)
\curveto(191.52889034,151.40352388)(190.80232857,151.71211732)(189.83358765,151.71212889)
\curveto(188.6695182,151.71211732)(187.75155037,151.34102394)(187.0796814,150.59884764)
\curveto(186.40780171,149.85665043)(186.07186455,148.84493269)(186.0718689,147.56369139)
\lineto(186.0718689,140.14572264)
\lineto(183.90390015,140.14572264)
\lineto(183.90390015,153.27072264)
\lineto(186.0718689,153.27072264)
\lineto(186.0718689,151.23166014)
\curveto(186.58748903,152.02071076)(187.19295718,152.61055392)(187.88827515,153.00119139)
\curveto(188.59139328,153.39180314)(189.39998622,153.58711545)(190.3140564,153.58712889)
\curveto(191.8218588,153.58711545)(192.96248266,153.11836591)(193.7359314,152.18087889)
\curveto(194.50935611,151.25118028)(194.89607447,149.8800879)(194.89608765,148.06759764)
}
}
{
\newrgbcolor{curcolor}{0 0 0}
\pscustom[linestyle=none,fillstyle=solid,fillcolor=curcolor]
{
\newpath
\moveto(198.99765015,145.32541014)
\lineto(198.99765015,153.27072264)
\lineto(201.15390015,153.27072264)
\lineto(201.15390015,145.40744139)
\curveto(201.15389595,144.16524987)(201.39608321,143.23165705)(201.88046265,142.60666014)
\curveto(202.36483224,141.98947079)(203.09139401,141.68087735)(204.06015015,141.68087889)
\curveto(205.22420438,141.68087735)(206.14217221,142.05197073)(206.8140564,142.79416014)
\curveto(207.49373336,143.53634425)(207.83357677,144.54806198)(207.83358765,145.82931639)
\lineto(207.83358765,153.27072264)
\lineto(209.98983765,153.27072264)
\lineto(209.98983765,140.14572264)
\lineto(207.83358765,140.14572264)
\lineto(207.83358765,142.16134764)
\curveto(207.31013979,141.36447142)(206.7007654,140.77072201)(206.00546265,140.38009764)
\curveto(205.31795429,139.99728529)(204.51717384,139.80587923)(203.6031189,139.80587889)
\curveto(202.09530126,139.80587923)(200.95077115,140.27462876)(200.16952515,141.21212889)
\curveto(199.38827272,142.14962688)(198.99764811,143.52071926)(198.99765015,145.32541014)
\moveto(204.4234314,153.58712889)
\lineto(204.4234314,153.58712889)
}
}
{
\newrgbcolor{curcolor}{0 0 0}
\pscustom[linestyle=none,fillstyle=solid,fillcolor=curcolor]
{
\newpath
\moveto(214.4546814,158.38009764)
\lineto(216.6109314,158.38009764)
\lineto(216.6109314,140.14572264)
\lineto(214.4546814,140.14572264)
\lineto(214.4546814,158.38009764)
}
}
{
\newrgbcolor{curcolor}{0 0 0}
\pscustom[linestyle=none,fillstyle=solid,fillcolor=curcolor]
{
\newpath
\moveto(232.3374939,147.24728514)
\lineto(232.3374939,146.19259764)
\lineto(222.4234314,146.19259764)
\curveto(222.51717773,144.70821807)(222.96248978,143.57540671)(223.7593689,142.79416014)
\curveto(224.56405068,142.02072076)(225.68123706,141.6340024)(227.1109314,141.63400389)
\curveto(227.93904731,141.6340024)(228.73982776,141.7355648)(229.51327515,141.93869139)
\curveto(230.2945137,142.14181439)(231.06795043,142.44650159)(231.83358765,142.85275389)
\lineto(231.83358765,140.81369139)
\curveto(231.06013794,140.48556605)(230.26716998,140.2355663)(229.4546814,140.06369139)
\curveto(228.6421716,139.89181664)(227.81795368,139.80587923)(226.98202515,139.80587889)
\curveto(224.88826911,139.80587923)(223.22811452,140.41525362)(222.0015564,141.63400389)
\curveto(220.78280446,142.85275118)(220.17343007,144.50118703)(220.1734314,146.57931639)
\curveto(220.17343007,148.7277453)(220.75155449,150.4308686)(221.9078064,151.68869139)
\curveto(223.07186467,152.95430358)(224.63826936,153.58711545)(226.60702515,153.58712889)
\curveto(228.37264062,153.58711545)(229.76717048,153.01680352)(230.7906189,151.87619139)
\curveto(231.82185592,150.74336829)(232.33748041,149.20040108)(232.3374939,147.24728514)
\moveto(230.1812439,147.88009764)
\curveto(230.16560758,149.05977622)(229.83357666,150.00118153)(229.18515015,150.70431639)
\curveto(228.54451545,151.40743013)(227.6929538,151.75899227)(226.63046265,151.75900389)
\curveto(225.42733107,151.75899227)(224.46248828,151.41914886)(223.7359314,150.73947264)
\curveto(223.01717723,150.05977522)(222.60311514,149.10274493)(222.4937439,147.86837889)
\lineto(230.1812439,147.88009764)
}
}
{
\newrgbcolor{curcolor}{0 0 0}
\pscustom[linestyle=none,fillstyle=solid,fillcolor=curcolor]
{
\newpath
\moveto(243.48202515,151.25509764)
\curveto(243.23982802,151.39571139)(242.97420329,151.49727379)(242.68515015,151.55978514)
\curveto(242.40389136,151.63008615)(242.09139167,151.66524237)(241.74765015,151.66525389)
\curveto(240.52889323,151.66524237)(239.59139417,151.26680527)(238.93515015,150.46994139)
\curveto(238.28670797,149.68086935)(237.96248955,148.54415174)(237.9624939,147.05978514)
\lineto(237.9624939,140.14572264)
\lineto(235.79452515,140.14572264)
\lineto(235.79452515,153.27072264)
\lineto(237.9624939,153.27072264)
\lineto(237.9624939,151.23166014)
\curveto(238.4156141,152.02852325)(239.00545726,152.61836641)(239.73202515,153.00119139)
\curveto(240.4585808,153.39180314)(241.34139242,153.58711545)(242.38046265,153.58712889)
\curveto(242.52889123,153.58711545)(242.69295357,153.57539671)(242.87265015,153.55197264)
\curveto(243.05232821,153.53633425)(243.25154676,153.50899052)(243.4703064,153.46994139)
\lineto(243.48202515,151.25509764)
}
}
{
\newrgbcolor{curcolor}{1 1 1}
\pscustom[linestyle=none,fillstyle=solid,fillcolor=curcolor]
{
\newpath
\moveto(149.98284912,480.0171578)
\lineto(329.94728088,480.0171578)
\lineto(329.94728088,450.20041967)
\lineto(149.98284912,450.20041967)
\closepath
}
}
{
\newrgbcolor{curcolor}{0 0 0}
\pscustom[linewidth=2,linecolor=curcolor]
{
\newpath
\moveto(149.98284912,480.0171578)
\lineto(329.94728088,480.0171578)
\lineto(329.94728088,450.20041967)
\lineto(149.98284912,450.20041967)
\closepath
}
}
{
\newrgbcolor{curcolor}{0 0 0}
\pscustom[linestyle=none,fillstyle=solid,fillcolor=curcolor]
{
}
}
{
\newrgbcolor{curcolor}{0 0 0}
\pscustom[linestyle=none,fillstyle=solid,fillcolor=curcolor]
{
}
}
{
\newrgbcolor{curcolor}{0 0 0}
\pscustom[linestyle=none,fillstyle=solid,fillcolor=curcolor]
{
}
}
{
\newrgbcolor{curcolor}{0 0 0}
\pscustom[linestyle=none,fillstyle=solid,fillcolor=curcolor]
{
}
}
{
\newrgbcolor{curcolor}{0 0 0}
\pscustom[linestyle=none,fillstyle=solid,fillcolor=curcolor]
{
}
}
{
\newrgbcolor{curcolor}{0 0 0}
\pscustom[linestyle=none,fillstyle=solid,fillcolor=curcolor]
{
}
}
{
\newrgbcolor{curcolor}{0 0 0}
\pscustom[linestyle=none,fillstyle=solid,fillcolor=curcolor]
{
}
}
{
\newrgbcolor{curcolor}{0 0 0}
\pscustom[linestyle=none,fillstyle=solid,fillcolor=curcolor]
{
\newpath
\moveto(185.83984375,377.49609495)
\lineto(189.02734375,377.49609495)
\lineto(196.78515625,362.8593762)
\lineto(196.78515625,377.49609495)
\lineto(199.08203125,377.49609495)
\lineto(199.08203125,360.0000012)
\lineto(195.89453125,360.0000012)
\lineto(188.13671875,374.63671995)
\lineto(188.13671875,360.0000012)
\lineto(185.83984375,360.0000012)
\lineto(185.83984375,377.49609495)
}
}
{
\newrgbcolor{curcolor}{0 0 0}
\pscustom[linestyle=none,fillstyle=solid,fillcolor=curcolor]
{
\newpath
\moveto(208.78515625,371.61328245)
\curveto(207.62890006,371.61327084)(206.71483847,371.16014629)(206.04296875,370.25390745)
\curveto(205.37108982,369.3554606)(205.03515265,368.12108683)(205.03515625,366.55078245)
\curveto(205.03515265,364.98046497)(205.36718357,363.74218496)(206.03125,362.8359387)
\curveto(206.70311973,361.93749926)(207.62108757,361.48828096)(208.78515625,361.48828245)
\curveto(209.93358525,361.48828096)(210.84374059,361.94140551)(211.515625,362.84765745)
\curveto(212.18748925,363.7539037)(212.52342641,364.98827746)(212.5234375,366.55078245)
\curveto(212.52342641,368.10546185)(212.18748925,369.33592937)(211.515625,370.2421887)
\curveto(210.84374059,371.15624004)(209.93358525,371.61327084)(208.78515625,371.61328245)
\moveto(208.78515625,373.44140745)
\curveto(210.66014703,373.44139401)(212.1328018,372.83201962)(213.203125,371.61328245)
\curveto(214.27342466,370.39452206)(214.80858038,368.70702374)(214.80859375,366.55078245)
\curveto(214.80858038,364.40234055)(214.27342466,362.71484224)(213.203125,361.48828245)
\curveto(212.1328018,360.26953218)(210.66014703,359.66015779)(208.78515625,359.66015745)
\curveto(206.90233829,359.66015779)(205.42577726,360.26953218)(204.35546875,361.48828245)
\curveto(203.29296689,362.71484224)(202.76171743,364.40234055)(202.76171875,366.55078245)
\curveto(202.76171743,368.70702374)(203.29296689,370.39452206)(204.35546875,371.61328245)
\curveto(205.42577726,372.83201962)(206.90233829,373.44139401)(208.78515625,373.44140745)
}
}
{
\newrgbcolor{curcolor}{0 0 0}
\pscustom[linestyle=none,fillstyle=solid,fillcolor=curcolor]
{
\newpath
\moveto(228.58984375,370.60546995)
\curveto(229.12889323,371.57420838)(229.77342384,372.28905141)(230.5234375,372.7500012)
\curveto(231.27342234,373.21092549)(232.15623395,373.44139401)(233.171875,373.44140745)
\curveto(234.53904407,373.44139401)(235.59373052,372.96092574)(236.3359375,372.0000012)
\curveto(237.07810403,371.04686515)(237.44919741,369.68749151)(237.44921875,367.9218762)
\lineto(237.44921875,360.0000012)
\lineto(235.28125,360.0000012)
\lineto(235.28125,367.8515637)
\curveto(235.28123083,369.10936709)(235.0585748,370.04295991)(234.61328125,370.65234495)
\curveto(234.16795069,371.26170869)(233.48826387,371.56639588)(232.57421875,371.56640745)
\curveto(231.4570159,371.56639588)(230.57420429,371.19530251)(229.92578125,370.4531262)
\curveto(229.27733058,369.71092899)(228.95311216,368.69921125)(228.953125,367.41796995)
\lineto(228.953125,360.0000012)
\lineto(226.78515625,360.0000012)
\lineto(226.78515625,367.8515637)
\curveto(226.78514557,369.11717958)(226.56248955,370.0507724)(226.1171875,370.65234495)
\curveto(225.67186544,371.26170869)(224.98436613,371.56639588)(224.0546875,371.56640745)
\curveto(222.95311816,371.56639588)(222.07811903,371.19139626)(221.4296875,370.44140745)
\curveto(220.78124533,369.69921025)(220.4570269,368.69139876)(220.45703125,367.41796995)
\lineto(220.45703125,360.0000012)
\lineto(218.2890625,360.0000012)
\lineto(218.2890625,373.1250012)
\lineto(220.45703125,373.1250012)
\lineto(220.45703125,371.0859387)
\curveto(220.94921391,371.89061431)(221.53905707,372.48436372)(222.2265625,372.8671887)
\curveto(222.9140557,373.24998795)(223.73046113,373.44139401)(224.67578125,373.44140745)
\curveto(225.62889673,373.44139401)(226.43748967,373.19920675)(227.1015625,372.71484495)
\curveto(227.77342584,372.23045772)(228.26951909,371.52733342)(228.58984375,370.60546995)
}
}
{
\newrgbcolor{curcolor}{0 0 0}
\pscustom[linestyle=none,fillstyle=solid,fillcolor=curcolor]
{
}
}
{
\newrgbcolor{curcolor}{0 0 0}
\pscustom[linestyle=none,fillstyle=solid,fillcolor=curcolor]
{
\newpath
\moveto(140.8984375,341.1328137)
\lineto(140.8984375,348.2343762)
\lineto(143.0546875,348.2343762)
\lineto(143.0546875,330.0000012)
\lineto(140.8984375,330.0000012)
\lineto(140.8984375,331.9687512)
\curveto(140.44530205,331.18750001)(139.87108388,330.60546935)(139.17578125,330.22265745)
\curveto(138.48827276,329.8476576)(137.66014859,329.66015779)(136.69140625,329.66015745)
\curveto(135.10546364,329.66015779)(133.81249619,330.29296966)(132.8125,331.55859495)
\curveto(131.82031068,332.82421713)(131.32421743,334.48827796)(131.32421875,336.55078245)
\curveto(131.32421743,338.61327384)(131.82031068,340.27733467)(132.8125,341.54296995)
\curveto(133.81249619,342.80858214)(135.10546364,343.44139401)(136.69140625,343.44140745)
\curveto(137.66014859,343.44139401)(138.48827276,343.24998795)(139.17578125,342.8671887)
\curveto(139.87108388,342.49217621)(140.44530205,341.91405179)(140.8984375,341.1328137)
\moveto(133.55078125,336.55078245)
\curveto(133.5507777,334.96483999)(133.87499612,333.71874748)(134.5234375,332.8125012)
\curveto(135.17968232,331.91406179)(136.07811892,331.46484349)(137.21875,331.46484495)
\curveto(138.35936664,331.46484349)(139.25780324,331.91406179)(139.9140625,332.8125012)
\curveto(140.57030193,333.71874748)(140.8984266,334.96483999)(140.8984375,336.55078245)
\curveto(140.8984266,338.13671181)(140.57030193,339.37889807)(139.9140625,340.27734495)
\curveto(139.25780324,341.18358377)(138.35936664,341.63670831)(137.21875,341.63671995)
\curveto(136.07811892,341.63670831)(135.17968232,341.18358377)(134.5234375,340.27734495)
\curveto(133.87499612,339.37889807)(133.5507777,338.13671181)(133.55078125,336.55078245)
}
}
{
\newrgbcolor{curcolor}{0 0 0}
\pscustom[linestyle=none,fillstyle=solid,fillcolor=curcolor]
{
\newpath
\moveto(149.53515625,347.49609495)
\lineto(149.53515625,340.9921887)
\lineto(147.54296875,340.9921887)
\lineto(147.54296875,347.49609495)
\lineto(149.53515625,347.49609495)
}
}
{
\newrgbcolor{curcolor}{0 0 0}
\pscustom[linestyle=none,fillstyle=solid,fillcolor=curcolor]
{
\newpath
\moveto(153.8828125,335.1796887)
\lineto(153.8828125,343.1250012)
\lineto(156.0390625,343.1250012)
\lineto(156.0390625,335.26171995)
\curveto(156.0390583,334.01952843)(156.28124556,333.08593562)(156.765625,332.4609387)
\curveto(157.24999459,331.84374936)(157.97655637,331.53515592)(158.9453125,331.53515745)
\curveto(160.10936673,331.53515592)(161.02733457,331.90624929)(161.69921875,332.6484387)
\curveto(162.37889571,333.39062281)(162.71873913,334.40234055)(162.71875,335.68359495)
\lineto(162.71875,343.1250012)
\lineto(164.875,343.1250012)
\lineto(164.875,330.0000012)
\lineto(162.71875,330.0000012)
\lineto(162.71875,332.0156262)
\curveto(162.19530215,331.21874998)(161.58592776,330.62500058)(160.890625,330.2343762)
\curveto(160.20311664,329.85156385)(159.40233619,329.66015779)(158.48828125,329.66015745)
\curveto(156.98046361,329.66015779)(155.83593351,330.12890732)(155.0546875,331.06640745)
\curveto(154.27343507,332.00390545)(153.88281046,333.37499783)(153.8828125,335.1796887)
\moveto(159.30859375,343.44140745)
\lineto(159.30859375,343.44140745)
}
}
{
\newrgbcolor{curcolor}{0 0 0}
\pscustom[linestyle=none,fillstyle=solid,fillcolor=curcolor]
{
\newpath
\moveto(171.47265625,346.8515637)
\lineto(171.47265625,343.1250012)
\lineto(175.9140625,343.1250012)
\lineto(175.9140625,341.44921995)
\lineto(171.47265625,341.44921995)
\lineto(171.47265625,334.32421995)
\curveto(171.47265186,333.2539042)(171.61718296,332.56640488)(171.90625,332.26171995)
\curveto(172.20311987,331.95703049)(172.80077553,331.8046869)(173.69921875,331.8046887)
\lineto(175.9140625,331.8046887)
\lineto(175.9140625,330.0000012)
\lineto(173.69921875,330.0000012)
\curveto(172.03515129,330.0000012)(170.88671494,330.30859464)(170.25390625,330.92578245)
\curveto(169.62109121,331.5507809)(169.30468527,332.68359227)(169.3046875,334.32421995)
\lineto(169.3046875,341.44921995)
\lineto(167.72265625,341.44921995)
\lineto(167.72265625,343.1250012)
\lineto(169.3046875,343.1250012)
\lineto(169.3046875,346.8515637)
\lineto(171.47265625,346.8515637)
}
}
{
\newrgbcolor{curcolor}{0 0 0}
\pscustom[linestyle=none,fillstyle=solid,fillcolor=curcolor]
{
\newpath
\moveto(178.76171875,343.1250012)
\lineto(180.91796875,343.1250012)
\lineto(180.91796875,330.0000012)
\lineto(178.76171875,330.0000012)
\lineto(178.76171875,343.1250012)
\moveto(178.76171875,348.2343762)
\lineto(180.91796875,348.2343762)
\lineto(180.91796875,345.50390745)
\lineto(178.76171875,345.50390745)
\lineto(178.76171875,348.2343762)
}
}
{
\newrgbcolor{curcolor}{0 0 0}
\pscustom[linestyle=none,fillstyle=solid,fillcolor=curcolor]
{
\newpath
\moveto(185.41796875,348.2343762)
\lineto(187.57421875,348.2343762)
\lineto(187.57421875,330.0000012)
\lineto(185.41796875,330.0000012)
\lineto(185.41796875,348.2343762)
}
}
{
\newrgbcolor{curcolor}{0 0 0}
\pscustom[linestyle=none,fillstyle=solid,fillcolor=curcolor]
{
\newpath
\moveto(192.07421875,343.1250012)
\lineto(194.23046875,343.1250012)
\lineto(194.23046875,330.0000012)
\lineto(192.07421875,330.0000012)
\lineto(192.07421875,343.1250012)
\moveto(192.07421875,348.2343762)
\lineto(194.23046875,348.2343762)
\lineto(194.23046875,345.50390745)
\lineto(192.07421875,345.50390745)
\lineto(192.07421875,348.2343762)
}
}
{
\newrgbcolor{curcolor}{0 0 0}
\pscustom[linestyle=none,fillstyle=solid,fillcolor=curcolor]
{
\newpath
\moveto(207.09765625,342.73828245)
\lineto(207.09765625,340.69921995)
\curveto(206.48827123,341.01170894)(205.85545936,341.24608371)(205.19921875,341.40234495)
\curveto(204.54296068,341.55858339)(203.86327386,341.63670831)(203.16015625,341.63671995)
\curveto(202.08983813,341.63670831)(201.28515143,341.47264598)(200.74609375,341.14453245)
\curveto(200.21484,340.81639663)(199.94921527,340.32420963)(199.94921875,339.66796995)
\curveto(199.94921527,339.16796078)(200.14062133,338.77342993)(200.5234375,338.4843762)
\curveto(200.90624556,338.203118)(201.67577604,337.93358702)(202.83203125,337.67578245)
\lineto(203.5703125,337.51171995)
\curveto(205.10155387,337.18358777)(206.18749028,336.71874448)(206.828125,336.1171887)
\curveto(207.47655149,335.52343318)(207.80076992,334.69140276)(207.80078125,333.62109495)
\curveto(207.80076992,332.40234255)(207.3163954,331.43749976)(206.34765625,330.7265637)
\curveto(205.38670983,330.01562619)(204.06249241,329.66015779)(202.375,329.66015745)
\curveto(201.6718698,329.66015779)(200.93749553,329.73047022)(200.171875,329.87109495)
\curveto(199.41405955,330.00390745)(198.61327911,330.20703224)(197.76953125,330.48046995)
\lineto(197.76953125,332.70703245)
\curveto(198.56640415,332.29296766)(199.35155962,331.98046797)(200.125,331.76953245)
\curveto(200.89843307,331.56640588)(201.6640573,331.46484349)(202.421875,331.46484495)
\curveto(203.43749303,331.46484349)(204.21874225,331.63671831)(204.765625,331.98046995)
\curveto(205.31249116,332.33203012)(205.58592838,332.82421713)(205.5859375,333.45703245)
\curveto(205.58592838,334.04296591)(205.38670983,334.49218421)(204.98828125,334.8046887)
\curveto(204.59764812,335.11718358)(203.73436773,335.41796453)(202.3984375,335.70703245)
\lineto(201.6484375,335.8828137)
\curveto(200.31249616,336.16405754)(199.34765337,336.59374461)(198.75390625,337.1718762)
\curveto(198.16015456,337.75780594)(197.86327986,338.55858639)(197.86328125,339.57421995)
\curveto(197.86327986,340.80858414)(198.30077942,341.76170819)(199.17578125,342.43359495)
\curveto(200.05077767,343.10545685)(201.29296393,343.44139401)(202.90234375,343.44140745)
\curveto(203.69921152,343.44139401)(204.44921077,343.38280032)(205.15234375,343.2656262)
\curveto(205.85545936,343.14842555)(206.50389621,342.97264448)(207.09765625,342.73828245)
}
}
{
\newrgbcolor{curcolor}{0 0 0}
\pscustom[linestyle=none,fillstyle=solid,fillcolor=curcolor]
{
\newpath
\moveto(217.2109375,336.59765745)
\curveto(215.46874352,336.59765085)(214.26171347,336.3984323)(213.58984375,336.0000012)
\curveto(212.91796482,335.6015581)(212.58202765,334.92187128)(212.58203125,333.9609387)
\curveto(212.58202765,333.19531051)(212.8320274,332.58593612)(213.33203125,332.1328137)
\curveto(213.83983889,331.68749951)(214.52733821,331.46484349)(215.39453125,331.46484495)
\curveto(216.58983614,331.46484349)(217.54686644,331.88671806)(218.265625,332.73046995)
\curveto(218.99217749,333.58202887)(219.35545838,334.71093399)(219.35546875,336.1171887)
\lineto(219.35546875,336.59765745)
\lineto(217.2109375,336.59765745)
\moveto(221.51171875,337.48828245)
\lineto(221.51171875,330.0000012)
\lineto(219.35546875,330.0000012)
\lineto(219.35546875,331.9921887)
\curveto(218.86327137,331.19531251)(218.24999073,330.60546935)(217.515625,330.22265745)
\curveto(216.7812422,329.8476576)(215.8828056,329.66015779)(214.8203125,329.66015745)
\curveto(213.47655801,329.66015779)(212.40624658,330.03515742)(211.609375,330.78515745)
\curveto(210.82031066,331.54296841)(210.42577981,332.55468615)(210.42578125,333.8203137)
\curveto(210.42577981,335.2968709)(210.91796682,336.41015104)(211.90234375,337.16015745)
\curveto(212.89452734,337.91014954)(214.37108836,338.28514917)(216.33203125,338.28515745)
\lineto(219.35546875,338.28515745)
\lineto(219.35546875,338.49609495)
\curveto(219.35545838,339.48827296)(219.02733371,340.2538972)(218.37109375,340.79296995)
\curveto(217.72264751,341.33983361)(216.80858593,341.61327084)(215.62890625,341.61328245)
\curveto(214.87890036,341.61327084)(214.14843234,341.52342718)(213.4375,341.3437512)
\curveto(212.72655876,341.16405254)(212.04296569,340.89452156)(211.38671875,340.53515745)
\lineto(211.38671875,342.52734495)
\curveto(212.17577806,342.83201962)(212.94140229,343.05858189)(213.68359375,343.20703245)
\curveto(214.42577581,343.36326909)(215.14843134,343.44139401)(215.8515625,343.44140745)
\curveto(217.74999123,343.44139401)(219.16795857,342.949207)(220.10546875,341.96484495)
\curveto(221.04295669,340.98045897)(221.51170622,339.48827296)(221.51171875,337.48828245)
}
}
{
\newrgbcolor{curcolor}{0 0 0}
\pscustom[linestyle=none,fillstyle=solid,fillcolor=curcolor]
{
\newpath
\moveto(228.09765625,346.8515637)
\lineto(228.09765625,343.1250012)
\lineto(232.5390625,343.1250012)
\lineto(232.5390625,341.44921995)
\lineto(228.09765625,341.44921995)
\lineto(228.09765625,334.32421995)
\curveto(228.09765186,333.2539042)(228.24218296,332.56640488)(228.53125,332.26171995)
\curveto(228.82811987,331.95703049)(229.42577553,331.8046869)(230.32421875,331.8046887)
\lineto(232.5390625,331.8046887)
\lineto(232.5390625,330.0000012)
\lineto(230.32421875,330.0000012)
\curveto(228.66015129,330.0000012)(227.51171494,330.30859464)(226.87890625,330.92578245)
\curveto(226.24609121,331.5507809)(225.92968527,332.68359227)(225.9296875,334.32421995)
\lineto(225.9296875,341.44921995)
\lineto(224.34765625,341.44921995)
\lineto(224.34765625,343.1250012)
\lineto(225.9296875,343.1250012)
\lineto(225.9296875,346.8515637)
\lineto(228.09765625,346.8515637)
}
}
{
\newrgbcolor{curcolor}{0 0 0}
\pscustom[linestyle=none,fillstyle=solid,fillcolor=curcolor]
{
\newpath
\moveto(246.61328125,337.1015637)
\lineto(246.61328125,336.0468762)
\lineto(236.69921875,336.0468762)
\curveto(236.79296508,334.56249664)(237.23827714,333.42968527)(238.03515625,332.6484387)
\curveto(238.83983804,331.87499933)(239.95702442,331.48828096)(241.38671875,331.48828245)
\curveto(242.21483466,331.48828096)(243.01561511,331.58984336)(243.7890625,331.79296995)
\curveto(244.57030105,331.99609296)(245.34373778,332.30078015)(246.109375,332.70703245)
\lineto(246.109375,330.66796995)
\curveto(245.33592529,330.33984461)(244.54295733,330.08984486)(243.73046875,329.91796995)
\curveto(242.91795896,329.74609521)(242.09374103,329.66015779)(241.2578125,329.66015745)
\curveto(239.16405646,329.66015779)(237.50390187,330.26953218)(236.27734375,331.48828245)
\curveto(235.05859182,332.70702974)(234.44921743,334.3554656)(234.44921875,336.43359495)
\curveto(234.44921743,338.58202387)(235.02734185,340.28514717)(236.18359375,341.54296995)
\curveto(237.34765203,342.80858214)(238.91405671,343.44139401)(240.8828125,343.44140745)
\curveto(242.64842798,343.44139401)(244.04295783,342.87108208)(245.06640625,341.73046995)
\curveto(246.09764328,340.59764685)(246.61326776,339.05467965)(246.61328125,337.1015637)
\moveto(244.45703125,337.7343762)
\curveto(244.44139493,338.91405479)(244.10936402,339.8554601)(243.4609375,340.55859495)
\curveto(242.8203028,341.26170869)(241.96874116,341.61327084)(240.90625,341.61328245)
\curveto(239.70311842,341.61327084)(238.73827564,341.27342743)(238.01171875,340.5937512)
\curveto(237.29296458,339.91405379)(236.8789025,338.95702349)(236.76953125,337.72265745)
\lineto(244.45703125,337.7343762)
}
}
{
\newrgbcolor{curcolor}{0 0 0}
\pscustom[linestyle=none,fillstyle=solid,fillcolor=curcolor]
{
\newpath
\moveto(249.9296875,335.1796887)
\lineto(249.9296875,343.1250012)
\lineto(252.0859375,343.1250012)
\lineto(252.0859375,335.26171995)
\curveto(252.0859333,334.01952843)(252.32812056,333.08593562)(252.8125,332.4609387)
\curveto(253.29686959,331.84374936)(254.02343137,331.53515592)(254.9921875,331.53515745)
\curveto(256.15624173,331.53515592)(257.07420957,331.90624929)(257.74609375,332.6484387)
\curveto(258.42577071,333.39062281)(258.76561413,334.40234055)(258.765625,335.68359495)
\lineto(258.765625,343.1250012)
\lineto(260.921875,343.1250012)
\lineto(260.921875,330.0000012)
\lineto(258.765625,330.0000012)
\lineto(258.765625,332.0156262)
\curveto(258.24217715,331.21874998)(257.63280276,330.62500058)(256.9375,330.2343762)
\curveto(256.24999164,329.85156385)(255.44921119,329.66015779)(254.53515625,329.66015745)
\curveto(253.02733861,329.66015779)(251.88280851,330.12890732)(251.1015625,331.06640745)
\curveto(250.32031007,332.00390545)(249.92968546,333.37499783)(249.9296875,335.1796887)
\moveto(255.35546875,343.44140745)
\lineto(255.35546875,343.44140745)
}
}
{
\newrgbcolor{curcolor}{0 0 0}
\pscustom[linestyle=none,fillstyle=solid,fillcolor=curcolor]
{
\newpath
\moveto(272.9921875,341.1093762)
\curveto(272.74999037,341.24998995)(272.48436564,341.35155235)(272.1953125,341.4140637)
\curveto(271.91405371,341.48436472)(271.60155402,341.51952093)(271.2578125,341.51953245)
\curveto(270.03905559,341.51952093)(269.10155652,341.12108383)(268.4453125,340.32421995)
\curveto(267.79687033,339.53514792)(267.4726519,338.3984303)(267.47265625,336.9140637)
\lineto(267.47265625,330.0000012)
\lineto(265.3046875,330.0000012)
\lineto(265.3046875,343.1250012)
\lineto(267.47265625,343.1250012)
\lineto(267.47265625,341.0859387)
\curveto(267.92577645,341.88280182)(268.51561961,342.47264498)(269.2421875,342.85546995)
\curveto(269.96874316,343.24608171)(270.85155477,343.44139401)(271.890625,343.44140745)
\curveto(272.03905359,343.44139401)(272.20311592,343.42967527)(272.3828125,343.4062512)
\curveto(272.56249056,343.39061281)(272.76170911,343.36326909)(272.98046875,343.32421995)
\lineto(272.9921875,341.1093762)
}
}
{
\newrgbcolor{curcolor}{0 0 0}
\pscustom[linestyle=none,fillstyle=solid,fillcolor=curcolor]
{
\newpath
\moveto(132.1669813,295.61288791)
\lineto(135.41206724,295.61288791)
\lineto(139.51963449,284.98817159)
\lineto(143.64876376,295.61288791)
\lineto(146.8938497,295.61288791)
\lineto(146.8938497,279.99999276)
\lineto(144.76998947,279.99999276)
\lineto(144.76998947,293.70964148)
\lineto(140.61929816,283.00126598)
\lineto(138.43075183,283.00126598)
\lineto(134.28006052,293.70964148)
\lineto(134.28006052,279.99999276)
\lineto(132.1669813,279.99999276)
\lineto(132.1669813,295.61288791)
}
}
{
\newrgbcolor{curcolor}{0 0 0}
\pscustom[linestyle=none,fillstyle=solid,fillcolor=curcolor]
{
\newpath
\moveto(155.82053207,290.36327414)
\curveto(154.75679942,290.36326377)(153.91588098,289.95891146)(153.29777423,289.15021597)
\curveto(152.67965901,288.34847378)(152.37060351,287.2469623)(152.37060682,285.84567822)
\curveto(152.37060351,284.44438245)(152.67606534,283.33938517)(153.28699321,282.53068307)
\curveto(153.90509998,281.7289475)(154.74961208,281.32808098)(155.82053207,281.3280823)
\curveto(156.87706386,281.32808098)(157.71438863,281.73243329)(158.33250889,282.54114047)
\curveto(158.9506106,283.34984256)(159.2596661,284.45135404)(159.2596763,285.84567822)
\curveto(159.2596661,287.23301912)(158.9506106,288.3310448)(158.33250889,289.13975858)
\curveto(157.71438863,289.95542566)(156.87706386,290.36326377)(155.82053207,290.36327414)
\moveto(155.82053207,291.99462822)
\curveto(157.54548621,291.99461622)(158.90029924,291.45083207)(159.88497525,290.36327414)
\curveto(160.8696296,289.27569547)(161.36196219,287.76983167)(161.36197449,285.84567822)
\curveto(161.36196219,283.92848467)(160.8696296,282.42262087)(159.88497525,281.3280823)
\curveto(158.90029924,280.24051268)(157.54548621,279.69672853)(155.82053207,279.69672822)
\curveto(154.08837707,279.69672853)(152.72997036,280.24051268)(151.74530787,281.3280823)
\curveto(150.76782734,282.42262087)(150.27908842,283.92848467)(150.27908964,285.84567822)
\curveto(150.27908842,287.76983167)(150.76782734,289.27569547)(151.74530787,290.36327414)
\curveto(152.72997036,291.45083207)(154.08837707,291.99461622)(155.82053207,291.99462822)
}
}
{
\newrgbcolor{curcolor}{0 0 0}
\pscustom[linestyle=none,fillstyle=solid,fillcolor=curcolor]
{
\newpath
\moveto(166.60154975,295.03773102)
\lineto(166.60154975,291.71227847)
\lineto(170.68755496,291.71227847)
\lineto(170.68755496,290.21687056)
\lineto(166.60154975,290.21687056)
\lineto(166.60154975,283.85877261)
\curveto(166.60154571,282.90366069)(166.73451144,282.29016063)(167.00044736,282.01827057)
\curveto(167.27356172,281.74637648)(167.82339301,281.61043044)(168.64994287,281.61043205)
\lineto(170.68755496,281.61043205)
\lineto(170.68755496,279.99999276)
\lineto(168.64994287,279.99999276)
\curveto(167.11903398,279.99999276)(166.06249543,280.27537063)(165.48032405,280.8261272)
\curveto(164.89814682,281.38385371)(164.60705967,282.3947345)(164.60706172,283.85877261)
\lineto(164.60706172,290.21687056)
\lineto(163.1516245,290.21687056)
\lineto(163.1516245,291.71227847)
\lineto(164.60706172,291.71227847)
\lineto(164.60706172,295.03773102)
\lineto(166.60154975,295.03773102)
}
}
{
\newrgbcolor{curcolor}{0 0 0}
\pscustom[linestyle=none,fillstyle=solid,fillcolor=curcolor]
{
}
}
{
\newrgbcolor{curcolor}{0 0 0}
\pscustom[linestyle=none,fillstyle=solid,fillcolor=curcolor]
{
\newpath
\moveto(188.28217346,289.93452082)
\lineto(188.28217346,296.27170398)
\lineto(190.26588048,296.27170398)
\lineto(190.26588048,279.99999276)
\lineto(188.28217346,279.99999276)
\lineto(188.28217346,281.75683562)
\curveto(187.86529788,281.0596747)(187.33702861,280.54029112)(186.69736405,280.19868333)
\curveto(186.06486995,279.86404673)(185.30301222,279.69672853)(184.41178858,279.69672822)
\curveto(182.95275299,279.69672853)(181.76324871,280.26142745)(180.84327215,281.39082669)
\curveto(179.93047775,282.52022315)(179.47408185,284.00517217)(179.47408307,285.84567822)
\curveto(179.47408185,287.68617257)(179.93047775,289.1711216)(180.84327215,290.30052975)
\curveto(181.76324871,291.4299173)(182.95275299,291.99461622)(184.41178858,291.99462822)
\curveto(185.30301222,291.99461622)(186.06486995,291.82381223)(186.69736405,291.48221572)
\curveto(187.33702861,291.14756783)(187.86529788,290.63167005)(188.28217346,289.93452082)
\moveto(181.52247618,285.84567822)
\curveto(181.52247292,284.43043927)(181.8207474,283.3184704)(182.41730054,282.50976827)
\curveto(183.0210327,281.70803272)(183.84757646,281.3071662)(184.89693431,281.30716751)
\curveto(185.94627889,281.3071662)(186.77282265,281.70803272)(187.37656808,282.50976827)
\curveto(187.98029528,283.3184704)(188.28216343,284.43043927)(188.28217346,285.84567822)
\curveto(188.28216343,287.26090548)(187.98029528,288.36938856)(187.37656808,289.17113077)
\curveto(186.77282265,289.97982623)(185.94627889,290.38417855)(184.89693431,290.38418893)
\curveto(183.84757646,290.38417855)(183.0210327,289.97982623)(182.41730054,289.17113077)
\curveto(181.8207474,288.36938856)(181.52247292,287.26090548)(181.52247618,285.84567822)
}
}
{
\newrgbcolor{curcolor}{0 0 0}
\pscustom[linestyle=none,fillstyle=solid,fillcolor=curcolor]
{
\newpath
\moveto(204.68010055,286.33717592)
\lineto(204.68010055,285.39601011)
\lineto(195.55936068,285.39601011)
\curveto(195.64560544,284.0714023)(196.05528365,283.0605215)(196.78839655,282.3633647)
\curveto(197.52868775,281.67317476)(198.55647695,281.32808098)(199.87176724,281.3280823)
\curveto(200.63361737,281.32808098)(201.37031942,281.41871167)(202.0818756,281.59997465)
\curveto(202.8005995,281.78123443)(203.51214587,282.05312651)(204.21651684,282.41565169)
\lineto(204.21651684,280.59606445)
\curveto(203.50495853,280.303257)(202.77544382,280.08016607)(202.02797052,279.92679098)
\curveto(201.2804777,279.77341603)(200.52221364,279.69672853)(199.75317606,279.69672822)
\curveto(197.82696224,279.69672853)(196.29965311,280.24051268)(195.17124409,281.3280823)
\curveto(194.05001661,282.41564928)(193.48940431,283.88665512)(193.48940553,285.74110424)
\curveto(193.48940431,287.65828621)(194.02126726,289.17809319)(195.08499596,290.30052975)
\curveto(196.15590637,291.4299173)(197.59696745,291.99461622)(199.40818353,291.99462822)
\curveto(201.03251457,291.99461622)(202.31545424,291.48569003)(203.25700639,290.46784812)
\curveto(204.20572389,289.45695686)(204.68008814,288.0800675)(204.68010055,286.33717592)
\moveto(202.69639353,286.90187541)
\curveto(202.68200843,287.95457885)(202.37654661,288.79465565)(201.78000714,289.42210832)
\curveto(201.19063599,290.04954215)(200.40721625,290.36326377)(199.42974557,290.36327414)
\curveto(198.3228885,290.36326377)(197.43525237,290.05999953)(196.76683452,289.45348051)
\curveto(196.10559501,288.84694259)(195.72466615,287.99292261)(195.62404678,286.89141802)
\lineto(202.69639353,286.90187541)
}
}
{
\newrgbcolor{curcolor}{0 0 0}
\pscustom[linestyle=none,fillstyle=solid,fillcolor=curcolor]
{
}
}
{
\newrgbcolor{curcolor}{0 0 0}
\pscustom[linestyle=none,fillstyle=solid,fillcolor=curcolor]
{
\newpath
\moveto(216.88421149,281.75683562)
\lineto(216.88421149,275.54514124)
\lineto(214.88972346,275.54514124)
\lineto(214.88972346,291.71227847)
\lineto(216.88421149,291.71227847)
\lineto(216.88421149,289.93452082)
\curveto(217.30107304,290.63167005)(217.82574865,291.14756783)(218.45823989,291.48221572)
\curveto(219.09790731,291.82381223)(219.85976504,291.99461622)(220.74381536,291.99462822)
\curveto(222.21002427,291.99461622)(223.39952855,291.4299173)(224.31233179,290.30052975)
\curveto(225.23229951,289.1711216)(225.69228908,287.68617257)(225.69230189,285.84567822)
\curveto(225.69228908,284.00517217)(225.23229951,282.52022315)(224.31233179,281.39082669)
\curveto(223.39952855,280.26142745)(222.21002427,279.69672853)(220.74381536,279.69672822)
\curveto(219.85976504,279.69672853)(219.09790731,279.86404673)(218.45823989,280.19868333)
\curveto(217.82574865,280.54029112)(217.30107304,281.0596747)(216.88421149,281.75683562)
\moveto(223.63312776,285.84567822)
\curveto(223.63311701,287.26090548)(223.33124885,288.36938856)(222.72752238,289.17113077)
\curveto(222.13096356,289.97982623)(221.30801346,290.38417855)(220.25866962,290.38418893)
\curveto(219.20931104,290.38417855)(218.38276727,289.97982623)(217.77903585,289.17113077)
\curveto(217.18248198,288.36938856)(216.88420749,287.26090548)(216.88421149,285.84567822)
\curveto(216.88420749,284.43043927)(217.18248198,283.3184704)(217.77903585,282.50976827)
\curveto(218.38276727,281.70803272)(219.20931104,281.3071662)(220.25866962,281.30716751)
\curveto(221.30801346,281.3071662)(222.13096356,281.70803272)(222.72752238,282.50976827)
\curveto(223.33124885,283.3184704)(223.63311701,284.43043927)(223.63312776,285.84567822)
}
}
{
\newrgbcolor{curcolor}{0 0 0}
\pscustom[linestyle=none,fillstyle=solid,fillcolor=curcolor]
{
\newpath
\moveto(234.46805038,285.88750781)
\curveto(232.86526665,285.88750192)(231.75482307,285.70972634)(231.13671632,285.35418052)
\curveto(230.51860109,284.99862399)(230.2095456,284.39209551)(230.20954891,283.53459327)
\curveto(230.2095456,282.85137375)(230.43954039,282.3075896)(230.89953396,281.90323919)
\curveto(231.36670687,281.50585656)(231.99919253,281.3071662)(232.79699284,281.30716751)
\curveto(233.89664952,281.3071662)(234.77709831,281.68363215)(235.43834186,282.43656649)
\curveto(236.10675567,283.19646754)(236.44096684,284.20386254)(236.44097638,285.4587545)
\lineto(236.44097638,285.88750781)
\lineto(234.46805038,285.88750781)
\moveto(238.4246834,286.68227006)
\lineto(238.4246834,279.99999276)
\lineto(236.44097638,279.99999276)
\lineto(236.44097638,281.77775042)
\curveto(235.98816461,281.06664629)(235.42395865,280.54029112)(234.74835681,280.19868333)
\curveto(234.07273928,279.86404673)(233.24619551,279.69672853)(232.26872304,279.69672822)
\curveto(231.03249569,279.69672853)(230.04783052,280.03136493)(229.31472455,280.70063843)
\curveto(228.58880109,281.37688212)(228.22584057,282.27970324)(228.22584189,283.4091045)
\curveto(228.22584057,284.72673191)(228.6786428,285.72018372)(229.58424996,286.38946291)
\curveto(230.49703908,287.05872932)(231.85544579,287.39336572)(233.65947416,287.39337312)
\lineto(236.44097638,287.39337312)
\lineto(236.44097638,287.58160628)
\curveto(236.44096684,288.46699084)(236.13909869,289.15020682)(235.53537101,289.63125628)
\curveto(234.9388134,290.11925806)(234.09789496,290.36326377)(233.01261317,290.36327414)
\curveto(232.3226227,290.36326377)(231.65060668,290.28309047)(230.99656311,290.12275398)
\curveto(230.34251134,289.96239725)(229.71361934,289.72187734)(229.10988524,289.40119352)
\lineto(229.10988524,291.17895118)
\curveto(229.83580407,291.45083207)(230.54016311,291.65300823)(231.22296445,291.78548026)
\curveto(231.90575715,291.9249003)(232.57058583,291.99461622)(233.21745248,291.99462822)
\curveto(234.96396907,291.99461622)(236.26847075,291.55540595)(237.13096143,290.67699607)
\curveto(237.99343165,289.79856485)(238.42467188,288.46699084)(238.4246834,286.68227006)
}
}
{
\newrgbcolor{curcolor}{0 0 0}
\pscustom[linestyle=none,fillstyle=solid,fillcolor=curcolor]
{
\newpath
\moveto(250.21911371,291.36718434)
\lineto(250.21911371,289.5475971)
\curveto(249.65849164,289.82645121)(249.07631734,290.03559896)(248.47258905,290.17504097)
\curveto(247.86884471,290.31446263)(247.24354638,290.38417855)(246.5966922,290.38418893)
\curveto(245.61202087,290.38417855)(244.87172515,290.23777512)(244.37580282,289.94497822)
\curveto(243.88705997,289.65216142)(243.64269051,289.21295115)(243.64269371,288.62734608)
\curveto(243.64269051,288.18115558)(243.81878027,287.8290902)(244.17096351,287.57114888)
\curveto(244.5231393,287.32016401)(245.231092,287.0796441)(246.29482374,286.84958842)
\lineto(246.97402778,286.70318485)
\curveto(248.38273931,286.4103713)(249.38177916,285.99556159)(249.97115033,285.4587545)
\curveto(250.56768978,284.92890807)(250.86596427,284.18643356)(250.86597469,283.23132873)
\curveto(250.86596427,282.1437572)(250.42034937,281.28276563)(249.52912866,280.64835144)
\curveto(248.64507711,280.01393595)(247.42682348,279.69672853)(245.8743641,279.69672822)
\curveto(245.22749833,279.69672853)(244.55188865,279.75947285)(243.84753302,279.88496139)
\curveto(243.15035792,280.00347856)(242.41365587,280.18473994)(241.63742466,280.42874608)
\lineto(241.63742466,282.41565169)
\curveto(242.37053185,282.04615492)(243.09285922,281.76729125)(243.80440896,281.57905985)
\curveto(244.51595196,281.39779689)(245.220311,281.3071662)(245.91748817,281.30716751)
\curveto(246.85183651,281.3071662)(247.57057022,281.46054122)(248.07369145,281.76729302)
\curveto(248.57679741,282.08101288)(248.82835421,282.52022315)(248.82836259,283.08492516)
\curveto(248.82835421,283.60779145)(248.64507711,284.00865797)(248.27853076,284.28752592)
\curveto(247.91915607,284.5663853)(247.12495532,284.83479158)(245.89592614,285.09274557)
\lineto(245.20594109,285.24960654)
\curveto(243.97690168,285.50057859)(243.08926555,285.88401613)(242.54303004,286.39992031)
\curveto(241.99679032,286.92278329)(241.72367151,287.63737143)(241.72367279,288.54368689)
\curveto(241.72367151,289.64518983)(242.12616238,290.49572401)(242.93114663,291.09529199)
\curveto(243.73612589,291.69483778)(244.87891248,291.99461622)(246.35950984,291.99462822)
\curveto(247.0926123,291.99461622)(247.78259666,291.94232928)(248.42946499,291.83776725)
\curveto(249.07631734,291.73318153)(249.67286631,291.57632072)(250.21911371,291.36718434)
}
}
{
\newrgbcolor{curcolor}{0 0 0}
\pscustom[linestyle=none,fillstyle=solid,fillcolor=curcolor]
{
\newpath
\moveto(261.73323909,291.36718434)
\lineto(261.73323909,289.5475971)
\curveto(261.17261702,289.82645121)(260.59044272,290.03559896)(259.98671444,290.17504097)
\curveto(259.38297009,290.31446263)(258.75767177,290.38417855)(258.11081758,290.38418893)
\curveto(257.12614625,290.38417855)(256.38585053,290.23777512)(255.88992821,289.94497822)
\curveto(255.40118535,289.65216142)(255.15681589,289.21295115)(255.15681909,288.62734608)
\curveto(255.15681589,288.18115558)(255.33290565,287.8290902)(255.6850889,287.57114888)
\curveto(256.03726468,287.32016401)(256.74521738,287.0796441)(257.80894913,286.84958842)
\lineto(258.48815316,286.70318485)
\curveto(259.89686469,286.4103713)(260.89590455,285.99556159)(261.48527572,285.4587545)
\curveto(262.08181516,284.92890807)(262.38008965,284.18643356)(262.38010008,283.23132873)
\curveto(262.38008965,282.1437572)(261.93447475,281.28276563)(261.04325404,280.64835144)
\curveto(260.1592025,280.01393595)(258.94094886,279.69672853)(257.38848949,279.69672822)
\curveto(256.74162372,279.69672853)(256.06601403,279.75947285)(255.3616584,279.88496139)
\curveto(254.6644833,280.00347856)(253.92778125,280.18473994)(253.15155004,280.42874608)
\lineto(253.15155004,282.41565169)
\curveto(253.88465723,282.04615492)(254.6069846,281.76729125)(255.31853434,281.57905985)
\curveto(256.03007735,281.39779689)(256.73443638,281.3071662)(257.43161355,281.30716751)
\curveto(258.3659619,281.3071662)(259.0846956,281.46054122)(259.58781683,281.76729302)
\curveto(260.09092279,282.08101288)(260.34247959,282.52022315)(260.34248798,283.08492516)
\curveto(260.34247959,283.60779145)(260.1592025,284.00865797)(259.79265614,284.28752592)
\curveto(259.43328145,284.5663853)(258.6390807,284.83479158)(257.41005152,285.09274557)
\lineto(256.72006647,285.24960654)
\curveto(255.49102707,285.50057859)(254.60339094,285.88401613)(254.05715542,286.39992031)
\curveto(253.5109157,286.92278329)(253.23779689,287.63737143)(253.23779818,288.54368689)
\curveto(253.23779689,289.64518983)(253.64028777,290.49572401)(254.44527201,291.09529199)
\curveto(255.25025127,291.69483778)(256.39303787,291.99461622)(257.87363522,291.99462822)
\curveto(258.60673769,291.99461622)(259.29672205,291.94232928)(259.94359037,291.83776725)
\curveto(260.59044272,291.73318153)(261.1869917,291.57632072)(261.73323909,291.36718434)
}
}
{
\newrgbcolor{curcolor}{0 0 0}
\pscustom[linestyle=none,fillstyle=solid,fillcolor=curcolor]
{
\newpath
\moveto(275.87792473,286.33717592)
\lineto(275.87792473,285.39601011)
\lineto(266.75718486,285.39601011)
\curveto(266.84342962,284.0714023)(267.25310783,283.0605215)(267.98622073,282.3633647)
\curveto(268.72651193,281.67317476)(269.75430113,281.32808098)(271.06959142,281.3280823)
\curveto(271.83144155,281.32808098)(272.5681436,281.41871167)(273.27969978,281.59997465)
\curveto(273.99842368,281.78123443)(274.70997005,282.05312651)(275.41434103,282.41565169)
\lineto(275.41434103,280.59606445)
\curveto(274.70278271,280.303257)(273.973268,280.08016607)(273.2257947,279.92679098)
\curveto(272.47830188,279.77341603)(271.72003782,279.69672853)(270.95100024,279.69672822)
\curveto(269.02478642,279.69672853)(267.49747729,280.24051268)(266.36906827,281.3280823)
\curveto(265.24784079,282.41564928)(264.6872285,283.88665512)(264.68722971,285.74110424)
\curveto(264.6872285,287.65828621)(265.21909144,289.17809319)(266.28282014,290.30052975)
\curveto(267.35373055,291.4299173)(268.79479163,291.99461622)(270.60600771,291.99462822)
\curveto(272.23033876,291.99461622)(273.51327842,291.48569003)(274.45483057,290.46784812)
\curveto(275.40354807,289.45695686)(275.87791232,288.0800675)(275.87792473,286.33717592)
\moveto(273.89421771,286.90187541)
\curveto(273.87983261,287.95457885)(273.57437079,288.79465565)(272.97783132,289.42210832)
\curveto(272.38846017,290.04954215)(271.60504043,290.36326377)(270.62756975,290.36327414)
\curveto(269.52071268,290.36326377)(268.63307655,290.05999953)(267.9646587,289.45348051)
\curveto(267.30341919,288.84694259)(266.92249033,287.99292261)(266.82187096,286.89141802)
\lineto(273.89421771,286.90187541)
}
}
{
\newrgbcolor{curcolor}{0 0 0}
\pscustom[linestyle=none,fillstyle=solid,fillcolor=curcolor]
{
\newpath
\moveto(146.8671875,240.0000012)
\lineto(140.1875,257.49609495)
\lineto(142.66015625,257.49609495)
\lineto(148.203125,242.7656262)
\lineto(153.7578125,257.49609495)
\lineto(156.21875,257.49609495)
\lineto(149.55078125,240.0000012)
\lineto(146.8671875,240.0000012)
}
}
{
\newrgbcolor{curcolor}{0 0 0}
\pscustom[linestyle=none,fillstyle=solid,fillcolor=curcolor]
{
\newpath
\moveto(168.01953125,247.1015637)
\lineto(168.01953125,246.0468762)
\lineto(158.10546875,246.0468762)
\curveto(158.19921508,244.56249664)(158.64452714,243.42968527)(159.44140625,242.6484387)
\curveto(160.24608804,241.87499933)(161.36327442,241.48828096)(162.79296875,241.48828245)
\curveto(163.62108466,241.48828096)(164.42186511,241.58984336)(165.1953125,241.79296995)
\curveto(165.97655105,241.99609296)(166.74998778,242.30078015)(167.515625,242.70703245)
\lineto(167.515625,240.66796995)
\curveto(166.74217529,240.33984461)(165.94920733,240.08984486)(165.13671875,239.91796995)
\curveto(164.32420896,239.74609521)(163.49999103,239.66015779)(162.6640625,239.66015745)
\curveto(160.57030646,239.66015779)(158.91015187,240.26953218)(157.68359375,241.48828245)
\curveto(156.46484182,242.70702974)(155.85546743,244.3554656)(155.85546875,246.43359495)
\curveto(155.85546743,248.58202387)(156.43359185,250.28514717)(157.58984375,251.54296995)
\curveto(158.75390203,252.80858214)(160.32030671,253.44139401)(162.2890625,253.44140745)
\curveto(164.05467798,253.44139401)(165.44920783,252.87108208)(166.47265625,251.73046995)
\curveto(167.50389328,250.59764685)(168.01951776,249.05467965)(168.01953125,247.1015637)
\moveto(165.86328125,247.7343762)
\curveto(165.84764493,248.91405479)(165.51561402,249.8554601)(164.8671875,250.55859495)
\curveto(164.2265528,251.26170869)(163.37499116,251.61327084)(162.3125,251.61328245)
\curveto(161.10936842,251.61327084)(160.14452564,251.27342743)(159.41796875,250.5937512)
\curveto(158.69921458,249.91405379)(158.2851525,248.95702349)(158.17578125,247.72265745)
\lineto(165.86328125,247.7343762)
\moveto(163.7890625,259.1953137)
\lineto(166.12109375,259.1953137)
\lineto(162.30078125,254.7890637)
\lineto(160.5078125,254.7890637)
\lineto(163.7890625,259.1953137)
}
}
{
\newrgbcolor{curcolor}{0 0 0}
\pscustom[linestyle=none,fillstyle=solid,fillcolor=curcolor]
{
\newpath
\moveto(179.1640625,251.1093762)
\curveto(178.92186538,251.24998995)(178.65624064,251.35155235)(178.3671875,251.4140637)
\curveto(178.08592871,251.48436472)(177.77342902,251.51952093)(177.4296875,251.51953245)
\curveto(176.21093059,251.51952093)(175.27343152,251.12108383)(174.6171875,250.32421995)
\curveto(173.96874533,249.53514792)(173.6445269,248.3984303)(173.64453125,246.9140637)
\lineto(173.64453125,240.0000012)
\lineto(171.4765625,240.0000012)
\lineto(171.4765625,253.1250012)
\lineto(173.64453125,253.1250012)
\lineto(173.64453125,251.0859387)
\curveto(174.09765145,251.88280182)(174.68749461,252.47264498)(175.4140625,252.85546995)
\curveto(176.14061816,253.24608171)(177.02342977,253.44139401)(178.0625,253.44140745)
\curveto(178.21092859,253.44139401)(178.37499092,253.42967527)(178.5546875,253.4062512)
\curveto(178.73436556,253.39061281)(178.93358411,253.36326909)(179.15234375,253.32421995)
\lineto(179.1640625,251.1093762)
}
}
{
\newrgbcolor{curcolor}{0 0 0}
\pscustom[linestyle=none,fillstyle=solid,fillcolor=curcolor]
{
\newpath
\moveto(181.44921875,253.1250012)
\lineto(183.60546875,253.1250012)
\lineto(183.60546875,240.0000012)
\lineto(181.44921875,240.0000012)
\lineto(181.44921875,253.1250012)
\moveto(181.44921875,258.2343762)
\lineto(183.60546875,258.2343762)
\lineto(183.60546875,255.50390745)
\lineto(181.44921875,255.50390745)
\lineto(181.44921875,258.2343762)
}
}
{
\newrgbcolor{curcolor}{0 0 0}
\pscustom[linestyle=none,fillstyle=solid,fillcolor=curcolor]
{
\newpath
\moveto(198.7109375,253.1250012)
\lineto(198.7109375,240.0000012)
\lineto(196.54296875,240.0000012)
\lineto(196.54296875,251.44921995)
\lineto(190.625,251.44921995)
\lineto(190.625,240.0000012)
\lineto(188.45703125,240.0000012)
\lineto(188.45703125,251.44921995)
\lineto(186.39453125,251.44921995)
\lineto(186.39453125,253.1250012)
\lineto(188.45703125,253.1250012)
\lineto(188.45703125,254.0390637)
\curveto(188.45702864,255.46873573)(188.7929658,256.52342218)(189.46484375,257.2031262)
\curveto(190.14452695,257.89060831)(191.18358841,258.23435797)(192.58203125,258.2343762)
\lineto(194.75,258.2343762)
\lineto(194.75,256.44140745)
\lineto(192.6875,256.44140745)
\curveto(191.91405643,256.44139101)(191.37499447,256.28514117)(191.0703125,255.97265745)
\curveto(190.77343257,255.66014179)(190.62499522,255.09764235)(190.625,254.28515745)
\lineto(190.625,253.1250012)
\lineto(198.7109375,253.1250012)
\moveto(196.54296875,258.2109387)
\lineto(198.7109375,258.2109387)
\lineto(198.7109375,255.48046995)
\lineto(196.54296875,255.48046995)
\lineto(196.54296875,258.2109387)
}
}
{
\newrgbcolor{curcolor}{0 0 0}
\pscustom[linestyle=none,fillstyle=solid,fillcolor=curcolor]
{
\newpath
\moveto(212.69140625,252.62109495)
\lineto(212.69140625,250.60546995)
\curveto(212.08202015,250.94139651)(211.46873952,251.19139626)(210.8515625,251.35546995)
\curveto(210.24217824,251.52733342)(209.62499136,251.61327084)(209,251.61328245)
\curveto(207.60155588,251.61327084)(206.51561947,251.16795878)(205.7421875,250.27734495)
\curveto(204.96874602,249.39452306)(204.58202765,248.1523368)(204.58203125,246.55078245)
\curveto(204.58202765,244.949215)(204.96874602,243.7031225)(205.7421875,242.8125012)
\curveto(206.51561947,241.92968677)(207.60155588,241.48828096)(209,241.48828245)
\curveto(209.62499136,241.48828096)(210.24217824,241.57031213)(210.8515625,241.7343762)
\curveto(211.46873952,241.90624929)(212.08202015,242.16015529)(212.69140625,242.49609495)
\lineto(212.69140625,240.50390745)
\curveto(212.08983264,240.22265723)(211.46483327,240.01171994)(210.81640625,239.87109495)
\curveto(210.17577206,239.73047022)(209.49217899,239.66015779)(208.765625,239.66015745)
\curveto(206.7890567,239.66015779)(205.21874577,240.28125092)(204.0546875,241.5234387)
\curveto(202.89062309,242.76562344)(202.30859243,244.44140301)(202.30859375,246.55078245)
\curveto(202.30859243,248.69139876)(202.89452934,250.37499083)(204.06640625,251.6015637)
\curveto(205.24608949,252.82811337)(206.85936913,253.44139401)(208.90625,253.44140745)
\curveto(209.57030391,253.44139401)(210.21874077,253.37108158)(210.8515625,253.23046995)
\curveto(211.4843645,253.09764435)(212.09764514,252.89451956)(212.69140625,252.62109495)
}
}
{
\newrgbcolor{curcolor}{0 0 0}
\pscustom[linestyle=none,fillstyle=solid,fillcolor=curcolor]
{
\newpath
\moveto(222.4296875,246.59765745)
\curveto(220.68749352,246.59765085)(219.48046347,246.3984323)(218.80859375,246.0000012)
\curveto(218.13671482,245.6015581)(217.80077765,244.92187128)(217.80078125,243.9609387)
\curveto(217.80077765,243.19531051)(218.0507774,242.58593612)(218.55078125,242.1328137)
\curveto(219.05858889,241.68749951)(219.74608821,241.46484349)(220.61328125,241.46484495)
\curveto(221.80858614,241.46484349)(222.76561644,241.88671806)(223.484375,242.73046995)
\curveto(224.21092749,243.58202887)(224.57420838,244.71093399)(224.57421875,246.1171887)
\lineto(224.57421875,246.59765745)
\lineto(222.4296875,246.59765745)
\moveto(226.73046875,247.48828245)
\lineto(226.73046875,240.0000012)
\lineto(224.57421875,240.0000012)
\lineto(224.57421875,241.9921887)
\curveto(224.08202137,241.19531251)(223.46874073,240.60546935)(222.734375,240.22265745)
\curveto(221.9999922,239.8476576)(221.1015556,239.66015779)(220.0390625,239.66015745)
\curveto(218.69530801,239.66015779)(217.62499658,240.03515742)(216.828125,240.78515745)
\curveto(216.03906066,241.54296841)(215.64452981,242.55468615)(215.64453125,243.8203137)
\curveto(215.64452981,245.2968709)(216.13671682,246.41015104)(217.12109375,247.16015745)
\curveto(218.11327734,247.91014954)(219.58983836,248.28514917)(221.55078125,248.28515745)
\lineto(224.57421875,248.28515745)
\lineto(224.57421875,248.49609495)
\curveto(224.57420838,249.48827296)(224.24608371,250.2538972)(223.58984375,250.79296995)
\curveto(222.94139751,251.33983361)(222.02733593,251.61327084)(220.84765625,251.61328245)
\curveto(220.09765036,251.61327084)(219.36718234,251.52342718)(218.65625,251.3437512)
\curveto(217.94530876,251.16405254)(217.26171569,250.89452156)(216.60546875,250.53515745)
\lineto(216.60546875,252.52734495)
\curveto(217.39452806,252.83201962)(218.16015229,253.05858189)(218.90234375,253.20703245)
\curveto(219.64452581,253.36326909)(220.36718134,253.44139401)(221.0703125,253.44140745)
\curveto(222.96874123,253.44139401)(224.38670857,252.949207)(225.32421875,251.96484495)
\curveto(226.26170669,250.98045897)(226.73045622,249.48827296)(226.73046875,247.48828245)
}
}
{
\newrgbcolor{curcolor}{0 0 0}
\pscustom[linestyle=none,fillstyle=solid,fillcolor=curcolor]
{
\newpath
\moveto(233.31640625,256.8515637)
\lineto(233.31640625,253.1250012)
\lineto(237.7578125,253.1250012)
\lineto(237.7578125,251.44921995)
\lineto(233.31640625,251.44921995)
\lineto(233.31640625,244.32421995)
\curveto(233.31640186,243.2539042)(233.46093296,242.56640488)(233.75,242.26171995)
\curveto(234.04686987,241.95703049)(234.64452553,241.8046869)(235.54296875,241.8046887)
\lineto(237.7578125,241.8046887)
\lineto(237.7578125,240.0000012)
\lineto(235.54296875,240.0000012)
\curveto(233.87890129,240.0000012)(232.73046494,240.30859464)(232.09765625,240.92578245)
\curveto(231.46484121,241.5507809)(231.14843527,242.68359227)(231.1484375,244.32421995)
\lineto(231.1484375,251.44921995)
\lineto(229.56640625,251.44921995)
\lineto(229.56640625,253.1250012)
\lineto(231.1484375,253.1250012)
\lineto(231.1484375,256.8515637)
\lineto(233.31640625,256.8515637)
}
}
{
\newrgbcolor{curcolor}{0 0 0}
\pscustom[linestyle=none,fillstyle=solid,fillcolor=curcolor]
{
\newpath
\moveto(240.60546875,253.1250012)
\lineto(242.76171875,253.1250012)
\lineto(242.76171875,240.0000012)
\lineto(240.60546875,240.0000012)
\lineto(240.60546875,253.1250012)
\moveto(240.60546875,258.2343762)
\lineto(242.76171875,258.2343762)
\lineto(242.76171875,255.50390745)
\lineto(240.60546875,255.50390745)
\lineto(240.60546875,258.2343762)
}
}
{
\newrgbcolor{curcolor}{0 0 0}
\pscustom[linestyle=none,fillstyle=solid,fillcolor=curcolor]
{
\newpath
\moveto(252.34765625,251.61328245)
\curveto(251.19140006,251.61327084)(250.27733847,251.16014629)(249.60546875,250.25390745)
\curveto(248.93358982,249.3554606)(248.59765265,248.12108683)(248.59765625,246.55078245)
\curveto(248.59765265,244.98046497)(248.92968357,243.74218496)(249.59375,242.8359387)
\curveto(250.26561973,241.93749926)(251.18358757,241.48828096)(252.34765625,241.48828245)
\curveto(253.49608525,241.48828096)(254.40624059,241.94140551)(255.078125,242.84765745)
\curveto(255.74998925,243.7539037)(256.08592641,244.98827746)(256.0859375,246.55078245)
\curveto(256.08592641,248.10546185)(255.74998925,249.33592937)(255.078125,250.2421887)
\curveto(254.40624059,251.15624004)(253.49608525,251.61327084)(252.34765625,251.61328245)
\moveto(252.34765625,253.44140745)
\curveto(254.22264703,253.44139401)(255.6953018,252.83201962)(256.765625,251.61328245)
\curveto(257.83592466,250.39452206)(258.37108038,248.70702374)(258.37109375,246.55078245)
\curveto(258.37108038,244.40234055)(257.83592466,242.71484224)(256.765625,241.48828245)
\curveto(255.6953018,240.26953218)(254.22264703,239.66015779)(252.34765625,239.66015745)
\curveto(250.46483829,239.66015779)(248.98827726,240.26953218)(247.91796875,241.48828245)
\curveto(246.85546689,242.71484224)(246.32421743,244.40234055)(246.32421875,246.55078245)
\curveto(246.32421743,248.70702374)(246.85546689,250.39452206)(247.91796875,251.61328245)
\curveto(248.98827726,252.83201962)(250.46483829,253.44139401)(252.34765625,253.44140745)
}
}
{
\newrgbcolor{curcolor}{0 0 0}
\pscustom[linestyle=none,fillstyle=solid,fillcolor=curcolor]
{
\newpath
\moveto(272.84375,247.9218762)
\lineto(272.84375,240.0000012)
\lineto(270.6875,240.0000012)
\lineto(270.6875,247.8515637)
\curveto(270.68748898,249.09374211)(270.44530173,250.02342868)(269.9609375,250.6406262)
\curveto(269.4765527,251.25780244)(268.74999092,251.56639588)(267.78125,251.56640745)
\curveto(266.61718055,251.56639588)(265.69921272,251.19530251)(265.02734375,250.4531262)
\curveto(264.35546407,249.71092899)(264.0195269,248.69921125)(264.01953125,247.41796995)
\lineto(264.01953125,240.0000012)
\lineto(261.8515625,240.0000012)
\lineto(261.8515625,253.1250012)
\lineto(264.01953125,253.1250012)
\lineto(264.01953125,251.0859387)
\curveto(264.53515139,251.87498933)(265.14061953,252.46483249)(265.8359375,252.85546995)
\curveto(266.53905563,253.24608171)(267.34764857,253.44139401)(268.26171875,253.44140745)
\curveto(269.76952115,253.44139401)(270.91014501,252.97264448)(271.68359375,252.03515745)
\curveto(272.45701846,251.10545885)(272.84373683,249.73436647)(272.84375,247.9218762)
}
}
{
\newrgbcolor{curcolor}{0 0 0}
\pscustom[linestyle=none,fillstyle=solid,fillcolor=curcolor]
{
\newpath
\moveto(43.96875,542.6562512)
\lineto(49.125,542.6562512)
\lineto(49.125,560.4531262)
\lineto(43.515625,559.3281262)
\lineto(43.515625,562.2031262)
\lineto(49.09375,563.3281262)
\lineto(52.25,563.3281262)
\lineto(52.25,542.6562512)
\lineto(57.40625,542.6562512)
\lineto(57.40625,540.0000012)
\lineto(43.96875,540.0000012)
\lineto(43.96875,542.6562512)
}
}
{
\newrgbcolor{curcolor}{0 0 0}
\pscustom[linestyle=none,fillstyle=solid,fillcolor=curcolor]
{
\newpath
\moveto(46.140625,462.6562512)
\lineto(57.15625,462.6562512)
\lineto(57.15625,460.0000012)
\lineto(42.34375,460.0000012)
\lineto(42.34375,462.6562512)
\curveto(43.54166312,463.89583064)(45.17186983,465.55728731)(47.234375,467.6406262)
\curveto(49.30728236,469.73436647)(50.60936439,471.08332345)(51.140625,471.6875012)
\curveto(52.15102952,472.82290504)(52.85415381,473.78123742)(53.25,474.5625012)
\curveto(53.65623634,475.35415251)(53.85936114,476.1301934)(53.859375,476.8906262)
\curveto(53.85936114,478.1301914)(53.42186158,479.14060706)(52.546875,479.9218762)
\curveto(51.68227998,480.7031055)(50.55207278,481.09373011)(49.15625,481.0937512)
\curveto(48.1666585,481.09373011)(47.11978455,480.92185528)(46.015625,480.5781262)
\curveto(44.92187008,480.23435597)(43.74999625,479.71352315)(42.5,479.0156262)
\lineto(42.5,482.2031262)
\curveto(43.77082956,482.71352015)(44.95832838,483.09893644)(46.0625,483.3593762)
\curveto(47.1666595,483.61976925)(48.17707516,483.74997745)(49.09375,483.7500012)
\curveto(51.51040516,483.74997745)(53.43748656,483.14581139)(54.875,481.9375012)
\curveto(56.31248369,480.72914714)(57.03123297,479.11456542)(57.03125,477.0937512)
\curveto(57.03123297,476.13540173)(56.84894148,475.22394431)(56.484375,474.3593762)
\curveto(56.1301922,473.50519603)(55.47915119,472.49478037)(54.53125,471.3281262)
\curveto(54.27081906,471.02603184)(53.44269489,470.15103272)(52.046875,468.7031262)
\curveto(50.65103102,467.26561894)(48.68228298,465.24999595)(46.140625,462.6562512)
}
}
{
\newrgbcolor{curcolor}{0 0 0}
\pscustom[linestyle=none,fillstyle=solid,fillcolor=curcolor]
{
\newpath
\moveto(52.984375,412.5781262)
\curveto(54.49477717,412.25519728)(55.67185933,411.58332295)(56.515625,410.5625012)
\curveto(57.3697743,409.54165833)(57.7968572,408.28124292)(57.796875,406.7812512)
\curveto(57.7968572,404.47916339)(57.00519133,402.69791517)(55.421875,401.4375012)
\curveto(53.83852783,400.17708436)(51.58853008,399.54687665)(48.671875,399.5468762)
\curveto(47.69270064,399.54687665)(46.68228498,399.64583489)(45.640625,399.8437512)
\curveto(44.60937039,400.03125117)(43.54166312,400.31770922)(42.4375,400.7031262)
\lineto(42.4375,403.7500012)
\curveto(43.31249669,403.23958129)(44.27082906,402.85416501)(45.3125,402.5937512)
\curveto(46.35416031,402.3333322)(47.44270089,402.203124)(48.578125,402.2031262)
\curveto(50.55728111,402.203124)(52.06248794,402.59374861)(53.09375,403.3750012)
\curveto(54.13540253,404.15624704)(54.65623534,405.29166258)(54.65625,406.7812512)
\curveto(54.65623534,408.15624304)(54.17186083,409.22915864)(53.203125,410.0000012)
\curveto(52.24477942,410.78124042)(50.90623909,411.17186503)(49.1875,411.1718762)
\lineto(46.46875,411.1718762)
\lineto(46.46875,413.7656262)
\lineto(49.3125,413.7656262)
\curveto(50.86457247,413.76561244)(52.05207128,414.07290379)(52.875,414.6875012)
\curveto(53.69790297,415.31248589)(54.10936089,416.20831833)(54.109375,417.3750012)
\curveto(54.10936089,418.57289929)(53.68227798,419.48956504)(52.828125,420.1250012)
\curveto(51.98436302,420.77081376)(50.77082256,421.09373011)(49.1875,421.0937512)
\curveto(48.32290834,421.09373011)(47.39582594,420.9999802)(46.40625,420.8125012)
\curveto(45.41666125,420.62498058)(44.32812067,420.3333142)(43.140625,419.9375012)
\lineto(43.140625,422.7500012)
\curveto(44.33853733,423.08331145)(45.45832788,423.3333112)(46.5,423.5000012)
\curveto(47.55207578,423.6666442)(48.54165812,423.74997745)(49.46875,423.7500012)
\curveto(51.86457147,423.74997745)(53.76040291,423.203103)(55.15625,422.1093762)
\curveto(56.55206678,421.02602184)(57.24998275,419.55727331)(57.25,417.7031262)
\curveto(57.24998275,416.41144312)(56.88019145,415.31769422)(56.140625,414.4218762)
\curveto(55.40102627,413.536446)(54.34894398,412.92186328)(52.984375,412.5781262)
}
}
{
\newrgbcolor{curcolor}{0 0 0}
\pscustom[linestyle=none,fillstyle=solid,fillcolor=curcolor]
{
\newpath
\moveto(512.09375,370.5781262)
\lineto(504.125,358.1250012)
\lineto(512.09375,358.1250012)
\lineto(512.09375,370.5781262)
\moveto(511.265625,373.3281262)
\lineto(515.234375,373.3281262)
\lineto(515.234375,358.1250012)
\lineto(518.5625,358.1250012)
\lineto(518.5625,355.5000012)
\lineto(515.234375,355.5000012)
\lineto(515.234375,350.0000012)
\lineto(512.09375,350.0000012)
\lineto(512.09375,355.5000012)
\lineto(501.5625,355.5000012)
\lineto(501.5625,358.5468762)
\lineto(511.265625,373.3281262)
}
}
{
\newrgbcolor{curcolor}{1 1 1}
\pscustom[linestyle=none,fillstyle=solid,fillcolor=curcolor]
{
\newpath
\moveto(149.99850464,429.91890837)
\lineto(409.9803772,429.91890837)
\lineto(409.9803772,389.8669026)
\lineto(149.99850464,389.8669026)
\closepath
}
}
{
\newrgbcolor{curcolor}{1 0 0}
\pscustom[linewidth=2,linecolor=curcolor,linestyle=dashed,dash=8 8]
{
\newpath
\moveto(149.99850464,429.91890837)
\lineto(409.9803772,429.91890837)
\lineto(409.9803772,389.8669026)
\lineto(149.99850464,389.8669026)
\closepath
}
}
{
\newrgbcolor{curcolor}{0 0 0}
\pscustom[linestyle=none,fillstyle=solid,fillcolor=curcolor,opacity=0.11636364]
{
\newpath
\moveto(280.76009941,340.41146971)
\lineto(419.70257759,340.41146971)
\curveto(425.69124755,340.41146971)(430.51245117,345.23267332)(430.51245117,351.22134329)
\lineto(430.51245117,358.7882588)
\curveto(430.51245117,364.77692876)(425.69124755,369.59813238)(419.70257759,369.59813238)
\lineto(280.76009941,369.59813238)
\curveto(274.77142945,369.59813238)(269.95022583,364.77692876)(269.95022583,358.7882588)
\lineto(269.95022583,351.22134329)
\curveto(269.95022583,345.23267332)(274.77142945,340.41146971)(280.76009941,340.41146971)
\closepath
}
}
{
\newrgbcolor{curcolor}{0 0 0}
\pscustom[linewidth=2,linecolor=curcolor]
{
\newpath
\moveto(280.76009941,340.41146971)
\lineto(419.70257759,340.41146971)
\curveto(425.69124755,340.41146971)(430.51245117,345.23267332)(430.51245117,351.22134329)
\lineto(430.51245117,358.7882588)
\curveto(430.51245117,364.77692876)(425.69124755,369.59813238)(419.70257759,369.59813238)
\lineto(280.76009941,369.59813238)
\curveto(274.77142945,369.59813238)(269.95022583,364.77692876)(269.95022583,358.7882588)
\lineto(269.95022583,351.22134329)
\curveto(269.95022583,345.23267332)(274.77142945,340.41146971)(280.76009941,340.41146971)
\closepath
}
}
{
\newrgbcolor{curcolor}{0 0 0}
\pscustom[linestyle=none,fillstyle=solid,fillcolor=curcolor]
{
\newpath
\moveto(503.453125,303.3281262)
\lineto(515.84375,303.3281262)
\lineto(515.84375,300.6718762)
\lineto(506.34375,300.6718762)
\lineto(506.34375,294.9531262)
\curveto(506.80207653,295.10936109)(507.26040941,295.22394431)(507.71875,295.2968762)
\curveto(508.17707516,295.38019415)(508.63540803,295.42186078)(509.09375,295.4218762)
\curveto(511.69790497,295.42186078)(513.76040291,294.70831983)(515.28125,293.2812512)
\curveto(516.80206653,291.85415601)(517.56248244,289.92186628)(517.5625,287.4843762)
\curveto(517.56248244,284.97395456)(516.78123322,283.02083151)(515.21875,281.6250012)
\curveto(513.65623634,280.23958429)(511.45311355,279.54687665)(508.609375,279.5468762)
\curveto(507.6302007,279.54687665)(506.6302017,279.6302099)(505.609375,279.7968762)
\curveto(504.59895373,279.9635429)(503.55207978,280.21354265)(502.46875,280.5468762)
\lineto(502.46875,283.7187512)
\curveto(503.40624659,283.20833133)(504.37499562,282.82812337)(505.375,282.5781262)
\curveto(506.37499363,282.32812387)(507.43228423,282.203124)(508.546875,282.2031262)
\curveto(510.34894798,282.203124)(511.77602989,282.67708186)(512.828125,283.6250012)
\curveto(513.88019445,284.57291329)(514.40623559,285.85937034)(514.40625,287.4843762)
\curveto(514.40623559,289.10936709)(513.88019445,290.39582414)(512.828125,291.3437512)
\curveto(511.77602989,292.29165558)(510.34894798,292.76561344)(508.546875,292.7656262)
\curveto(507.7031173,292.76561344)(506.85936814,292.67186353)(506.015625,292.4843762)
\curveto(505.18228648,292.2968639)(504.32812067,292.00519753)(503.453125,291.6093762)
\lineto(503.453125,303.3281262)
}
}
{
\newrgbcolor{curcolor}{0 0 0}
\pscustom[linestyle=none,fillstyle=solid,fillcolor=curcolor,opacity=0.11636364]
{
\newpath
\moveto(290.2822094,272.23803831)
\lineto(420.36839485,272.23803831)
\curveto(426.06473886,272.23803831)(430.65060425,276.8239037)(430.65060425,282.52024771)
\lineto(430.65060425,289.71779753)
\curveto(430.65060425,295.41414153)(426.06473886,300.00000692)(420.36839485,300.00000692)
\lineto(290.2822094,300.00000692)
\curveto(284.58586539,300.00000692)(280,295.41414153)(280,289.71779753)
\lineto(280,282.52024771)
\curveto(280,276.8239037)(284.58586539,272.23803831)(290.2822094,272.23803831)
\closepath
}
}
{
\newrgbcolor{curcolor}{0 0 0}
\pscustom[linewidth=2,linecolor=curcolor]
{
\newpath
\moveto(290.2822094,272.23803831)
\lineto(420.36839485,272.23803831)
\curveto(426.06473886,272.23803831)(430.65060425,276.8239037)(430.65060425,282.52024771)
\lineto(430.65060425,289.71779753)
\curveto(430.65060425,295.41414153)(426.06473886,300.00000692)(420.36839485,300.00000692)
\lineto(290.2822094,300.00000692)
\curveto(284.58586539,300.00000692)(280,295.41414153)(280,289.71779753)
\lineto(280,282.52024771)
\curveto(280,276.8239037)(284.58586539,272.23803831)(290.2822094,272.23803831)
\closepath
}
}
{
\newrgbcolor{curcolor}{0 0 0}
\pscustom[linestyle=none,fillstyle=solid,fillcolor=curcolor,opacity=0.11636364]
{
\newpath
\moveto(291.03757477,230.0000012)
\lineto(419.01493073,230.0000012)
\curveto(425.12974715,230.0000012)(430.05250549,234.92275955)(430.05250549,241.03757597)
\lineto(430.05250549,248.76388098)
\curveto(430.05250549,254.8786974)(425.12974715,259.80145575)(419.01493073,259.80145575)
\lineto(291.03757477,259.80145575)
\curveto(284.92275835,259.80145575)(280,254.8786974)(280,248.76388098)
\lineto(280,241.03757597)
\curveto(280,234.92275955)(284.92275835,230.0000012)(291.03757477,230.0000012)
\closepath
}
}
{
\newrgbcolor{curcolor}{0 0 0}
\pscustom[linewidth=2,linecolor=curcolor]
{
\newpath
\moveto(291.03757477,230.0000012)
\lineto(419.01493073,230.0000012)
\curveto(425.12974715,230.0000012)(430.05250549,234.92275955)(430.05250549,241.03757597)
\lineto(430.05250549,248.76388098)
\curveto(430.05250549,254.8786974)(425.12974715,259.80145575)(419.01493073,259.80145575)
\lineto(291.03757477,259.80145575)
\curveto(284.92275835,259.80145575)(280,254.8786974)(280,248.76388098)
\lineto(280,241.03757597)
\curveto(280,234.92275955)(284.92275835,230.0000012)(291.03757477,230.0000012)
\closepath
}
}
{
\newrgbcolor{curcolor}{0 0 0}
\pscustom[linestyle=none,fillstyle=solid,fillcolor=curcolor,opacity=0]
{
\newpath
\moveto(141.62518787,130.26532103)
\lineto(248.98716259,130.26532103)
\curveto(255.08827452,130.26532103)(260,135.17704651)(260,141.27815844)
\lineto(260,148.98715044)
\curveto(260,155.08826236)(255.08827452,159.99998785)(248.98716259,159.99998785)
\lineto(141.62518787,159.99998785)
\curveto(135.52407595,159.99998785)(130.61235046,155.08826236)(130.61235046,148.98715044)
\lineto(130.61235046,141.27815844)
\curveto(130.61235046,135.17704651)(135.52407595,130.26532103)(141.62518787,130.26532103)
\closepath
}
}
{
\newrgbcolor{curcolor}{0 0 0}
\pscustom[linewidth=2,linecolor=curcolor]
{
\newpath
\moveto(141.62518787,130.26532103)
\lineto(248.98716259,130.26532103)
\curveto(255.08827452,130.26532103)(260,135.17704651)(260,141.27815844)
\lineto(260,148.98715044)
\curveto(260,155.08826236)(255.08827452,159.99998785)(248.98716259,159.99998785)
\lineto(141.62518787,159.99998785)
\curveto(135.52407595,159.99998785)(130.61235046,155.08826236)(130.61235046,148.98715044)
\lineto(130.61235046,141.27815844)
\curveto(130.61235046,135.17704651)(135.52407595,130.26532103)(141.62518787,130.26532103)
\closepath
}
}
{
\newrgbcolor{curcolor}{0 0 0}
\pscustom[linestyle=none,fillstyle=solid,fillcolor=curcolor,opacity=0]
{
\newpath
\moveto(322.16531563,130.0000012)
\lineto(418.9267025,130.0000012)
\curveto(425.40054725,130.0000012)(430.61234283,135.21179679)(430.61234283,141.68564154)
\lineto(430.61234283,149.86559416)
\curveto(430.61234283,156.3394389)(425.40054725,161.55123449)(418.9267025,161.55123449)
\lineto(322.16531563,161.55123449)
\curveto(315.69147088,161.55123449)(310.47967529,156.3394389)(310.47967529,149.86559416)
\lineto(310.47967529,141.68564154)
\curveto(310.47967529,135.21179679)(315.69147088,130.0000012)(322.16531563,130.0000012)
\closepath
}
}
{
\newrgbcolor{curcolor}{0 0 0}
\pscustom[linewidth=2,linecolor=curcolor]
{
\newpath
\moveto(322.16531563,130.0000012)
\lineto(418.9267025,130.0000012)
\curveto(425.40054725,130.0000012)(430.61234283,135.21179679)(430.61234283,141.68564154)
\lineto(430.61234283,149.86559416)
\curveto(430.61234283,156.3394389)(425.40054725,161.55123449)(418.9267025,161.55123449)
\lineto(322.16531563,161.55123449)
\curveto(315.69147088,161.55123449)(310.47967529,156.3394389)(310.47967529,149.86559416)
\lineto(310.47967529,141.68564154)
\curveto(310.47967529,135.21179679)(315.69147088,130.0000012)(322.16531563,130.0000012)
\closepath
}
}
{
\newrgbcolor{curcolor}{0 0 0}
\pscustom[linestyle=none,fillstyle=solid,fillcolor=curcolor]
{
\newpath
\moveto(510.5625,252.9218762)
\curveto(509.14582419,252.92186328)(508.02082531,252.43748876)(507.1875,251.4687512)
\curveto(506.36457697,250.4999907)(505.95311905,249.17186703)(505.953125,247.4843762)
\curveto(505.95311905,245.80728706)(506.36457697,244.47916339)(507.1875,243.5000012)
\curveto(508.02082531,242.53124867)(509.14582419,242.04687415)(510.5625,242.0468762)
\curveto(511.97915469,242.04687415)(513.09894523,242.53124867)(513.921875,243.5000012)
\curveto(514.75519358,244.47916339)(515.17185983,245.80728706)(515.171875,247.4843762)
\curveto(515.17185983,249.17186703)(514.75519358,250.4999907)(513.921875,251.4687512)
\curveto(513.09894523,252.43748876)(511.97915469,252.92186328)(510.5625,252.9218762)
\moveto(516.828125,262.8125012)
\lineto(516.828125,259.9375012)
\curveto(516.0364423,260.31248089)(515.23435977,260.59893894)(514.421875,260.7968762)
\curveto(513.61977805,260.99477187)(512.82290384,261.09373011)(512.03125,261.0937512)
\curveto(509.94790672,261.09373011)(508.35415831,260.39060581)(507.25,258.9843762)
\curveto(506.15624384,257.57810862)(505.53124447,255.45311075)(505.375,252.6093762)
\curveto(505.98957734,253.51561269)(506.76040991,254.20832033)(507.6875,254.6875012)
\curveto(508.61457472,255.17706936)(509.63540703,255.42186078)(510.75,255.4218762)
\curveto(513.09373691,255.42186078)(514.94269339,254.70831983)(516.296875,253.2812512)
\curveto(517.66144067,251.86457267)(518.34373166,249.93228294)(518.34375,247.4843762)
\curveto(518.34373166,245.08853778)(517.63539903,243.1666647)(516.21875,241.7187512)
\curveto(514.80206853,240.27083426)(512.91665375,239.54687665)(510.5625,239.5468762)
\curveto(507.86457547,239.54687665)(505.80207753,240.57812562)(504.375,242.6406262)
\curveto(502.94791372,244.71353815)(502.23437277,247.71353515)(502.234375,251.6406262)
\curveto(502.23437277,255.32811087)(503.10937189,258.26560794)(504.859375,260.4531262)
\curveto(506.60936839,262.65102022)(508.95832438,263.74997745)(511.90625,263.7500012)
\curveto(512.69790397,263.74997745)(513.49477817,263.67185253)(514.296875,263.5156262)
\curveto(515.10935989,263.35935284)(515.95310905,263.12497808)(516.828125,262.8125012)
}
}
{
\newrgbcolor{curcolor}{0 0 0}
\pscustom[linestyle=none,fillstyle=solid,fillcolor=curcolor]
{
\newpath
\moveto(502.625,213.3281262)
\lineto(517.625,213.3281262)
\lineto(517.625,211.9843762)
\lineto(509.15625,190.0000012)
\lineto(505.859375,190.0000012)
\lineto(513.828125,210.6718762)
\lineto(502.625,210.6718762)
\lineto(502.625,213.3281262)
}
}
{
\newrgbcolor{curcolor}{0 0 0}
\pscustom[linestyle=none,fillstyle=solid,fillcolor=curcolor]
{
\newpath
\moveto(50.171875,151.0781262)
\curveto(48.67186633,151.07811512)(47.48957584,150.67707386)(46.625,149.8750012)
\curveto(45.77082756,149.07290879)(45.34374466,147.96874323)(45.34375,146.5625012)
\curveto(45.34374466,145.15624604)(45.77082756,144.05208048)(46.625,143.2500012)
\curveto(47.48957584,142.44791542)(48.67186633,142.04687415)(50.171875,142.0468762)
\curveto(51.67186333,142.04687415)(52.85415381,142.44791542)(53.71875,143.2500012)
\curveto(54.58331875,144.06249714)(55.01560998,145.1666627)(55.015625,146.5625012)
\curveto(55.01560998,147.96874323)(54.58331875,149.07290879)(53.71875,149.8750012)
\curveto(52.86457047,150.67707386)(51.68227998,151.07811512)(50.171875,151.0781262)
\moveto(47.015625,152.4218762)
\curveto(45.66145267,152.75519678)(44.60416206,153.38540448)(43.84375,154.3125012)
\curveto(43.09374691,155.23956929)(42.71874728,156.3697765)(42.71875,157.7031262)
\curveto(42.71874728,159.56768997)(43.38020495,161.04164683)(44.703125,162.1250012)
\curveto(46.0364523,163.20831133)(47.85936714,163.74997745)(50.171875,163.7500012)
\curveto(52.49477917,163.74997745)(54.31769402,163.20831133)(55.640625,162.1250012)
\curveto(56.9635247,161.04164683)(57.62498238,159.56768997)(57.625,157.7031262)
\curveto(57.62498238,156.3697765)(57.24477442,155.23956929)(56.484375,154.3125012)
\curveto(55.73435927,153.38540448)(54.68748531,152.75519678)(53.34375,152.4218762)
\curveto(54.86456847,152.06769747)(56.04685895,151.37498983)(56.890625,150.3437512)
\curveto(57.74477392,149.31249189)(58.17185683,148.05207648)(58.171875,146.5625012)
\curveto(58.17185683,144.30208023)(57.47914919,142.56770697)(56.09375,141.3593762)
\curveto(54.71873528,140.15104272)(52.74477892,139.54687665)(50.171875,139.5468762)
\curveto(47.59895073,139.54687665)(45.61978605,140.15104272)(44.234375,141.3593762)
\curveto(42.85937214,142.56770697)(42.17187283,144.30208023)(42.171875,146.5625012)
\curveto(42.17187283,148.05207648)(42.59895573,149.31249189)(43.453125,150.3437512)
\curveto(44.30728736,151.37498983)(45.49478617,152.06769747)(47.015625,152.4218762)
\moveto(45.859375,157.4062512)
\curveto(45.85936914,156.19790167)(46.23436877,155.25519428)(46.984375,154.5781262)
\curveto(47.74478392,153.90102897)(48.80728286,153.56248764)(50.171875,153.5625012)
\curveto(51.52603014,153.56248764)(52.58332075,153.90102897)(53.34375,154.5781262)
\curveto(54.11456922,155.25519428)(54.4999855,156.19790167)(54.5,157.4062512)
\curveto(54.4999855,158.61456592)(54.11456922,159.55727331)(53.34375,160.2343762)
\curveto(52.58332075,160.91143862)(51.52603014,161.24997995)(50.171875,161.2500012)
\curveto(48.80728286,161.24997995)(47.74478392,160.91143862)(46.984375,160.2343762)
\curveto(46.23436877,159.55727331)(45.85936914,158.61456592)(45.859375,157.4062512)
}
}
{
\newrgbcolor{curcolor}{0 0 0}
\pscustom[linewidth=2,linecolor=curcolor,linestyle=dashed,dash=8 8]
{
\newpath
\moveto(150,550)
\lineto(60,550)
}
}
{
\newrgbcolor{curcolor}{0 0 0}
\pscustom[linestyle=none,fillstyle=solid,fillcolor=curcolor]
{
\newpath
\moveto(139.53769464,554.84048224)
\lineto(152.6487474,550.01921591)
\lineto(139.53769392,545.19795064)
\curveto(141.632292,548.04442372)(141.62022288,551.93889292)(139.53769464,554.84048224)
\lineto(139.53769464,554.84048224)
\lineto(139.53769464,554.84048224)
\closepath
}
}
{
\newrgbcolor{curcolor}{0 0 0}
\pscustom[linewidth=2,linecolor=curcolor,linestyle=dashed,dash=8 8]
{
\newpath
\moveto(170,470)
\lineto(60,470)
}
}
{
\newrgbcolor{curcolor}{0 0 0}
\pscustom[linestyle=none,fillstyle=solid,fillcolor=curcolor]
{
\newpath
\moveto(159.53769464,474.84048224)
\lineto(172.6487474,470.01921591)
\lineto(159.53769392,465.19795064)
\curveto(161.632292,468.04442372)(161.62022288,471.93889292)(159.53769464,474.84048224)
\lineto(159.53769464,474.84048224)
\lineto(159.53769464,474.84048224)
\closepath
}
}
{
\newrgbcolor{curcolor}{0 0 0}
\pscustom[linewidth=2,linecolor=curcolor,linestyle=dashed,dash=8 8]
{
\newpath
\moveto(170,410)
\lineto(60,410)
}
}
{
\newrgbcolor{curcolor}{0 0 0}
\pscustom[linestyle=none,fillstyle=solid,fillcolor=curcolor]
{
\newpath
\moveto(159.53769464,414.84048224)
\lineto(172.6487474,410.01921591)
\lineto(159.53769392,405.19795064)
\curveto(161.632292,408.04442372)(161.62022288,411.93889292)(159.53769464,414.84048224)
\lineto(159.53769464,414.84048224)
\lineto(159.53769464,414.84048224)
\closepath
}
}
{
\newrgbcolor{curcolor}{0 0 0}
\pscustom[linewidth=2,linecolor=curcolor,linestyle=dashed,dash=8 8]
{
\newpath
\moveto(410,360)
\lineto(500,360)
}
}
{
\newrgbcolor{curcolor}{0 0 0}
\pscustom[linestyle=none,fillstyle=solid,fillcolor=curcolor]
{
\newpath
\moveto(420.46230536,355.15951776)
\lineto(407.3512526,359.98078409)
\lineto(420.46230608,364.80204936)
\curveto(418.367708,361.95557628)(418.37977712,358.06110708)(420.46230536,355.15951776)
\lineto(420.46230536,355.15951776)
\lineto(420.46230536,355.15951776)
\closepath
}
}
{
\newrgbcolor{curcolor}{0 0 0}
\pscustom[linewidth=2,linecolor=curcolor,linestyle=dashed,dash=8 8]
{
\newpath
\moveto(410,290)
\lineto(500,290)
}
}
{
\newrgbcolor{curcolor}{0 0 0}
\pscustom[linestyle=none,fillstyle=solid,fillcolor=curcolor]
{
\newpath
\moveto(420.46230536,285.15951776)
\lineto(407.3512526,289.98078409)
\lineto(420.46230608,294.80204936)
\curveto(418.367708,291.95557628)(418.37977712,288.06110708)(420.46230536,285.15951776)
\lineto(420.46230536,285.15951776)
\lineto(420.46230536,285.15951776)
\closepath
}
}
{
\newrgbcolor{curcolor}{0 0 0}
\pscustom[linewidth=2,linecolor=curcolor,linestyle=dashed,dash=8 8]
{
\newpath
\moveto(410,250)
\lineto(500,250)
}
}
{
\newrgbcolor{curcolor}{0 0 0}
\pscustom[linestyle=none,fillstyle=solid,fillcolor=curcolor]
{
\newpath
\moveto(420.46230536,245.15951776)
\lineto(407.3512526,249.98078409)
\lineto(420.46230608,254.80204936)
\curveto(418.367708,251.95557628)(418.37977712,248.06110708)(420.46230536,245.15951776)
\lineto(420.46230536,245.15951776)
\lineto(420.46230536,245.15951776)
\closepath
}
}
{
\newrgbcolor{curcolor}{0 0 0}
\pscustom[linewidth=2,linecolor=curcolor,linestyle=dashed,dash=8 8]
{
\newpath
\moveto(410,150.26534)
\lineto(500,150.26534)
}
}
{
\newrgbcolor{curcolor}{0 0 0}
\pscustom[linestyle=none,fillstyle=solid,fillcolor=curcolor]
{
\newpath
\moveto(420.46230536,145.42485776)
\lineto(407.3512526,150.24612409)
\lineto(420.46230608,155.06738936)
\curveto(418.367708,152.22091628)(418.37977712,148.32644708)(420.46230536,145.42485776)
\lineto(420.46230536,145.42485776)
\lineto(420.46230536,145.42485776)
\closepath
}
}
{
\newrgbcolor{curcolor}{0 0 0}
\pscustom[linewidth=2,linecolor=curcolor,linestyle=dashed,dash=8 8]
{
\newpath
\moveto(150,150.26534)
\lineto(60,150.26534)
}
}
{
\newrgbcolor{curcolor}{0 0 0}
\pscustom[linestyle=none,fillstyle=solid,fillcolor=curcolor]
{
\newpath
\moveto(139.53769464,155.10582224)
\lineto(152.6487474,150.28455591)
\lineto(139.53769392,145.46329064)
\curveto(141.632292,148.30976372)(141.62022288,152.20423292)(139.53769464,155.10582224)
\lineto(139.53769464,155.10582224)
\lineto(139.53769464,155.10582224)
\closepath
}
}
{
\newrgbcolor{curcolor}{0 0 0}
\pscustom[linestyle=none,fillstyle=solid,fillcolor=curcolor]
{
\newpath
\moveto(155.45703125,206.1484387)
\lineto(155.45703125,203.65234495)
\curveto(154.66014159,204.39451806)(153.80857994,204.949205)(152.90234375,205.31640745)
\curveto(152.00389425,205.68357927)(151.04686395,205.86717283)(150.03125,205.8671887)
\curveto(148.03124197,205.86717283)(146.4999935,205.2538922)(145.4375,204.02734495)
\curveto(144.37499562,202.80858214)(143.84374616,201.04295891)(143.84375,198.73046995)
\curveto(143.84374616,196.42577603)(144.37499562,194.66015279)(145.4375,193.43359495)
\curveto(146.4999935,192.21484274)(148.03124197,191.60546835)(150.03125,191.60546995)
\curveto(151.04686395,191.60546835)(152.00389425,191.78906191)(152.90234375,192.1562512)
\curveto(153.80857994,192.52343618)(154.66014159,193.07812312)(155.45703125,193.8203137)
\lineto(155.45703125,191.34765745)
\curveto(154.62889162,190.78515667)(153.74998625,190.36328209)(152.8203125,190.08203245)
\curveto(151.8984256,189.80078265)(150.92186408,189.66015779)(149.890625,189.66015745)
\curveto(147.24218026,189.66015779)(145.15624484,190.46875073)(143.6328125,192.0859387)
\curveto(142.10937289,193.71093499)(141.3476549,195.92577653)(141.34765625,198.73046995)
\curveto(141.3476549,201.54295841)(142.10937289,203.75779994)(143.6328125,205.3750012)
\curveto(145.15624484,206.9999842)(147.24218026,207.81248339)(149.890625,207.8125012)
\curveto(150.93748906,207.81248339)(151.92186308,207.67185853)(152.84375,207.3906262)
\curveto(153.77342373,207.11717158)(154.64451661,206.7031095)(155.45703125,206.1484387)
}
}
{
\newrgbcolor{curcolor}{0 0 0}
\pscustom[linestyle=none,fillstyle=solid,fillcolor=curcolor]
{
\newpath
\moveto(164.12890625,201.61328245)
\curveto(162.97265006,201.61327084)(162.05858847,201.16014629)(161.38671875,200.25390745)
\curveto(160.71483982,199.3554606)(160.37890265,198.12108683)(160.37890625,196.55078245)
\curveto(160.37890265,194.98046497)(160.71093357,193.74218496)(161.375,192.8359387)
\curveto(162.04686973,191.93749926)(162.96483757,191.48828096)(164.12890625,191.48828245)
\curveto(165.27733525,191.48828096)(166.18749059,191.94140551)(166.859375,192.84765745)
\curveto(167.53123925,193.7539037)(167.86717641,194.98827746)(167.8671875,196.55078245)
\curveto(167.86717641,198.10546185)(167.53123925,199.33592937)(166.859375,200.2421887)
\curveto(166.18749059,201.15624004)(165.27733525,201.61327084)(164.12890625,201.61328245)
\moveto(164.12890625,203.44140745)
\curveto(166.00389703,203.44139401)(167.4765518,202.83201962)(168.546875,201.61328245)
\curveto(169.61717466,200.39452206)(170.15233038,198.70702374)(170.15234375,196.55078245)
\curveto(170.15233038,194.40234055)(169.61717466,192.71484224)(168.546875,191.48828245)
\curveto(167.4765518,190.26953218)(166.00389703,189.66015779)(164.12890625,189.66015745)
\curveto(162.24608829,189.66015779)(160.76952726,190.26953218)(159.69921875,191.48828245)
\curveto(158.63671689,192.71484224)(158.10546743,194.40234055)(158.10546875,196.55078245)
\curveto(158.10546743,198.70702374)(158.63671689,200.39452206)(159.69921875,201.61328245)
\curveto(160.76952726,202.83201962)(162.24608829,203.44139401)(164.12890625,203.44140745)
}
}
{
\newrgbcolor{curcolor}{0 0 0}
\pscustom[linestyle=none,fillstyle=solid,fillcolor=curcolor]
{
\newpath
\moveto(184.625,197.9218762)
\lineto(184.625,190.0000012)
\lineto(182.46875,190.0000012)
\lineto(182.46875,197.8515637)
\curveto(182.46873898,199.09374211)(182.22655173,200.02342868)(181.7421875,200.6406262)
\curveto(181.2578027,201.25780244)(180.53124092,201.56639588)(179.5625,201.56640745)
\curveto(178.39843055,201.56639588)(177.48046272,201.19530251)(176.80859375,200.4531262)
\curveto(176.13671407,199.71092899)(175.8007769,198.69921125)(175.80078125,197.41796995)
\lineto(175.80078125,190.0000012)
\lineto(173.6328125,190.0000012)
\lineto(173.6328125,203.1250012)
\lineto(175.80078125,203.1250012)
\lineto(175.80078125,201.0859387)
\curveto(176.31640139,201.87498933)(176.92186953,202.46483249)(177.6171875,202.85546995)
\curveto(178.32030563,203.24608171)(179.12889857,203.44139401)(180.04296875,203.44140745)
\curveto(181.55077115,203.44139401)(182.69139501,202.97264448)(183.46484375,202.03515745)
\curveto(184.23826846,201.10545885)(184.62498683,199.73436647)(184.625,197.9218762)
}
}
{
\newrgbcolor{curcolor}{0 0 0}
\pscustom[linestyle=none,fillstyle=solid,fillcolor=curcolor]
{
\newpath
\moveto(199.859375,197.9218762)
\lineto(199.859375,190.0000012)
\lineto(197.703125,190.0000012)
\lineto(197.703125,197.8515637)
\curveto(197.70311398,199.09374211)(197.46092673,200.02342868)(196.9765625,200.6406262)
\curveto(196.4921777,201.25780244)(195.76561592,201.56639588)(194.796875,201.56640745)
\curveto(193.63280555,201.56639588)(192.71483772,201.19530251)(192.04296875,200.4531262)
\curveto(191.37108907,199.71092899)(191.0351519,198.69921125)(191.03515625,197.41796995)
\lineto(191.03515625,190.0000012)
\lineto(188.8671875,190.0000012)
\lineto(188.8671875,203.1250012)
\lineto(191.03515625,203.1250012)
\lineto(191.03515625,201.0859387)
\curveto(191.55077639,201.87498933)(192.15624453,202.46483249)(192.8515625,202.85546995)
\curveto(193.55468063,203.24608171)(194.36327357,203.44139401)(195.27734375,203.44140745)
\curveto(196.78514615,203.44139401)(197.92577001,202.97264448)(198.69921875,202.03515745)
\curveto(199.47264346,201.10545885)(199.85936183,199.73436647)(199.859375,197.9218762)
}
}
{
\newrgbcolor{curcolor}{0 0 0}
\pscustom[linestyle=none,fillstyle=solid,fillcolor=curcolor]
{
\newpath
\moveto(215.41015625,197.1015637)
\lineto(215.41015625,196.0468762)
\lineto(205.49609375,196.0468762)
\curveto(205.58984008,194.56249664)(206.03515214,193.42968527)(206.83203125,192.6484387)
\curveto(207.63671304,191.87499933)(208.75389942,191.48828096)(210.18359375,191.48828245)
\curveto(211.01170966,191.48828096)(211.81249011,191.58984336)(212.5859375,191.79296995)
\curveto(213.36717605,191.99609296)(214.14061278,192.30078015)(214.90625,192.70703245)
\lineto(214.90625,190.66796995)
\curveto(214.13280029,190.33984461)(213.33983233,190.08984486)(212.52734375,189.91796995)
\curveto(211.71483396,189.74609521)(210.89061603,189.66015779)(210.0546875,189.66015745)
\curveto(207.96093146,189.66015779)(206.30077687,190.26953218)(205.07421875,191.48828245)
\curveto(203.85546682,192.70702974)(203.24609243,194.3554656)(203.24609375,196.43359495)
\curveto(203.24609243,198.58202387)(203.82421685,200.28514717)(204.98046875,201.54296995)
\curveto(206.14452703,202.80858214)(207.71093171,203.44139401)(209.6796875,203.44140745)
\curveto(211.44530298,203.44139401)(212.83983283,202.87108208)(213.86328125,201.73046995)
\curveto(214.89451828,200.59764685)(215.41014276,199.05467965)(215.41015625,197.1015637)
\moveto(213.25390625,197.7343762)
\curveto(213.23826993,198.91405479)(212.90623902,199.8554601)(212.2578125,200.55859495)
\curveto(211.6171778,201.26170869)(210.76561616,201.61327084)(209.703125,201.61328245)
\curveto(208.49999342,201.61327084)(207.53515064,201.27342743)(206.80859375,200.5937512)
\curveto(206.08983958,199.91405379)(205.6757775,198.95702349)(205.56640625,197.72265745)
\lineto(213.25390625,197.7343762)
}
}
{
\newrgbcolor{curcolor}{0 0 0}
\pscustom[linestyle=none,fillstyle=solid,fillcolor=curcolor]
{
\newpath
\moveto(229.4375,203.1250012)
\lineto(224.69140625,196.73828245)
\lineto(229.68359375,190.0000012)
\lineto(227.140625,190.0000012)
\lineto(223.3203125,195.1562512)
\lineto(219.5,190.0000012)
\lineto(216.95703125,190.0000012)
\lineto(222.0546875,196.8671887)
\lineto(217.390625,203.1250012)
\lineto(219.93359375,203.1250012)
\lineto(223.4140625,198.44921995)
\lineto(226.89453125,203.1250012)
\lineto(229.4375,203.1250012)
}
}
{
\newrgbcolor{curcolor}{0 0 0}
\pscustom[linestyle=none,fillstyle=solid,fillcolor=curcolor]
{
\newpath
\moveto(232.73046875,203.1250012)
\lineto(234.88671875,203.1250012)
\lineto(234.88671875,190.0000012)
\lineto(232.73046875,190.0000012)
\lineto(232.73046875,203.1250012)
\moveto(232.73046875,208.2343762)
\lineto(234.88671875,208.2343762)
\lineto(234.88671875,205.50390745)
\lineto(232.73046875,205.50390745)
\lineto(232.73046875,208.2343762)
}
}
{
\newrgbcolor{curcolor}{0 0 0}
\pscustom[linestyle=none,fillstyle=solid,fillcolor=curcolor]
{
\newpath
\moveto(244.47265625,201.61328245)
\curveto(243.31640006,201.61327084)(242.40233847,201.16014629)(241.73046875,200.25390745)
\curveto(241.05858982,199.3554606)(240.72265265,198.12108683)(240.72265625,196.55078245)
\curveto(240.72265265,194.98046497)(241.05468357,193.74218496)(241.71875,192.8359387)
\curveto(242.39061973,191.93749926)(243.30858757,191.48828096)(244.47265625,191.48828245)
\curveto(245.62108525,191.48828096)(246.53124059,191.94140551)(247.203125,192.84765745)
\curveto(247.87498925,193.7539037)(248.21092641,194.98827746)(248.2109375,196.55078245)
\curveto(248.21092641,198.10546185)(247.87498925,199.33592937)(247.203125,200.2421887)
\curveto(246.53124059,201.15624004)(245.62108525,201.61327084)(244.47265625,201.61328245)
\moveto(244.47265625,203.44140745)
\curveto(246.34764703,203.44139401)(247.8203018,202.83201962)(248.890625,201.61328245)
\curveto(249.96092466,200.39452206)(250.49608038,198.70702374)(250.49609375,196.55078245)
\curveto(250.49608038,194.40234055)(249.96092466,192.71484224)(248.890625,191.48828245)
\curveto(247.8203018,190.26953218)(246.34764703,189.66015779)(244.47265625,189.66015745)
\curveto(242.58983829,189.66015779)(241.11327726,190.26953218)(240.04296875,191.48828245)
\curveto(238.98046689,192.71484224)(238.44921743,194.40234055)(238.44921875,196.55078245)
\curveto(238.44921743,198.70702374)(238.98046689,200.39452206)(240.04296875,201.61328245)
\curveto(241.11327726,202.83201962)(242.58983829,203.44139401)(244.47265625,203.44140745)
}
}
{
\newrgbcolor{curcolor}{0 0 0}
\pscustom[linestyle=none,fillstyle=solid,fillcolor=curcolor]
{
\newpath
\moveto(264.96875,197.9218762)
\lineto(264.96875,190.0000012)
\lineto(262.8125,190.0000012)
\lineto(262.8125,197.8515637)
\curveto(262.81248898,199.09374211)(262.57030173,200.02342868)(262.0859375,200.6406262)
\curveto(261.6015527,201.25780244)(260.87499092,201.56639588)(259.90625,201.56640745)
\curveto(258.74218055,201.56639588)(257.82421272,201.19530251)(257.15234375,200.4531262)
\curveto(256.48046407,199.71092899)(256.1445269,198.69921125)(256.14453125,197.41796995)
\lineto(256.14453125,190.0000012)
\lineto(253.9765625,190.0000012)
\lineto(253.9765625,203.1250012)
\lineto(256.14453125,203.1250012)
\lineto(256.14453125,201.0859387)
\curveto(256.66015139,201.87498933)(257.26561953,202.46483249)(257.9609375,202.85546995)
\curveto(258.66405563,203.24608171)(259.47264857,203.44139401)(260.38671875,203.44140745)
\curveto(261.89452115,203.44139401)(263.03514501,202.97264448)(263.80859375,202.03515745)
\curveto(264.58201846,201.10545885)(264.96873683,199.73436647)(264.96875,197.9218762)
}
}
{
\newrgbcolor{curcolor}{0 0 0}
\pscustom[linestyle=none,fillstyle=solid,fillcolor=curcolor]
{
}
}
{
\newrgbcolor{curcolor}{0 0 0}
\pscustom[linestyle=none,fillstyle=solid,fillcolor=curcolor]
{
\newpath
\moveto(282.8984375,196.59765745)
\curveto(281.15624352,196.59765085)(279.94921347,196.3984323)(279.27734375,196.0000012)
\curveto(278.60546482,195.6015581)(278.26952765,194.92187128)(278.26953125,193.9609387)
\curveto(278.26952765,193.19531051)(278.5195274,192.58593612)(279.01953125,192.1328137)
\curveto(279.52733889,191.68749951)(280.21483821,191.46484349)(281.08203125,191.46484495)
\curveto(282.27733614,191.46484349)(283.23436644,191.88671806)(283.953125,192.73046995)
\curveto(284.67967749,193.58202887)(285.04295838,194.71093399)(285.04296875,196.1171887)
\lineto(285.04296875,196.59765745)
\lineto(282.8984375,196.59765745)
\moveto(287.19921875,197.48828245)
\lineto(287.19921875,190.0000012)
\lineto(285.04296875,190.0000012)
\lineto(285.04296875,191.9921887)
\curveto(284.55077137,191.19531251)(283.93749073,190.60546935)(283.203125,190.22265745)
\curveto(282.4687422,189.8476576)(281.5703056,189.66015779)(280.5078125,189.66015745)
\curveto(279.16405801,189.66015779)(278.09374658,190.03515742)(277.296875,190.78515745)
\curveto(276.50781066,191.54296841)(276.11327981,192.55468615)(276.11328125,193.8203137)
\curveto(276.11327981,195.2968709)(276.60546682,196.41015104)(277.58984375,197.16015745)
\curveto(278.58202734,197.91014954)(280.05858836,198.28514917)(282.01953125,198.28515745)
\lineto(285.04296875,198.28515745)
\lineto(285.04296875,198.49609495)
\curveto(285.04295838,199.48827296)(284.71483371,200.2538972)(284.05859375,200.79296995)
\curveto(283.41014751,201.33983361)(282.49608593,201.61327084)(281.31640625,201.61328245)
\curveto(280.56640036,201.61327084)(279.83593234,201.52342718)(279.125,201.3437512)
\curveto(278.41405876,201.16405254)(277.73046569,200.89452156)(277.07421875,200.53515745)
\lineto(277.07421875,202.52734495)
\curveto(277.86327806,202.83201962)(278.62890229,203.05858189)(279.37109375,203.20703245)
\curveto(280.11327581,203.36326909)(280.83593134,203.44139401)(281.5390625,203.44140745)
\curveto(283.43749123,203.44139401)(284.85545857,202.949207)(285.79296875,201.96484495)
\curveto(286.73045669,200.98045897)(287.19920622,199.48827296)(287.19921875,197.48828245)
}
}
{
\newrgbcolor{curcolor}{0 0 0}
\pscustom[linestyle=none,fillstyle=solid,fillcolor=curcolor]
{
\newpath
\moveto(291.4296875,195.1796887)
\lineto(291.4296875,203.1250012)
\lineto(293.5859375,203.1250012)
\lineto(293.5859375,195.26171995)
\curveto(293.5859333,194.01952843)(293.82812056,193.08593562)(294.3125,192.4609387)
\curveto(294.79686959,191.84374936)(295.52343137,191.53515592)(296.4921875,191.53515745)
\curveto(297.65624173,191.53515592)(298.57420957,191.90624929)(299.24609375,192.6484387)
\curveto(299.92577071,193.39062281)(300.26561413,194.40234055)(300.265625,195.68359495)
\lineto(300.265625,203.1250012)
\lineto(302.421875,203.1250012)
\lineto(302.421875,190.0000012)
\lineto(300.265625,190.0000012)
\lineto(300.265625,192.0156262)
\curveto(299.74217715,191.21874998)(299.13280276,190.62500058)(298.4375,190.2343762)
\curveto(297.74999164,189.85156385)(296.94921119,189.66015779)(296.03515625,189.66015745)
\curveto(294.52733861,189.66015779)(293.38280851,190.12890732)(292.6015625,191.06640745)
\curveto(291.82031007,192.00390545)(291.42968546,193.37499783)(291.4296875,195.1796887)
\moveto(296.85546875,203.44140745)
\lineto(296.85546875,203.44140745)
}
}
{
\newrgbcolor{curcolor}{0 0 0}
\pscustom[linestyle=none,fillstyle=solid,fillcolor=curcolor]
{
\newpath
\moveto(309.01953125,206.8515637)
\lineto(309.01953125,203.1250012)
\lineto(313.4609375,203.1250012)
\lineto(313.4609375,201.44921995)
\lineto(309.01953125,201.44921995)
\lineto(309.01953125,194.32421995)
\curveto(309.01952686,193.2539042)(309.16405796,192.56640488)(309.453125,192.26171995)
\curveto(309.74999488,191.95703049)(310.34765053,191.8046869)(311.24609375,191.8046887)
\lineto(313.4609375,191.8046887)
\lineto(313.4609375,190.0000012)
\lineto(311.24609375,190.0000012)
\curveto(309.58202629,190.0000012)(308.43358994,190.30859464)(307.80078125,190.92578245)
\curveto(307.16796621,191.5507809)(306.85156027,192.68359227)(306.8515625,194.32421995)
\lineto(306.8515625,201.44921995)
\lineto(305.26953125,201.44921995)
\lineto(305.26953125,203.1250012)
\lineto(306.8515625,203.1250012)
\lineto(306.8515625,206.8515637)
\lineto(309.01953125,206.8515637)
}
}
{
\newrgbcolor{curcolor}{0 0 0}
\pscustom[linestyle=none,fillstyle=solid,fillcolor=curcolor]
{
\newpath
\moveto(321.39453125,201.61328245)
\curveto(320.23827506,201.61327084)(319.32421347,201.16014629)(318.65234375,200.25390745)
\curveto(317.98046482,199.3554606)(317.64452765,198.12108683)(317.64453125,196.55078245)
\curveto(317.64452765,194.98046497)(317.97655857,193.74218496)(318.640625,192.8359387)
\curveto(319.31249473,191.93749926)(320.23046257,191.48828096)(321.39453125,191.48828245)
\curveto(322.54296025,191.48828096)(323.45311559,191.94140551)(324.125,192.84765745)
\curveto(324.79686425,193.7539037)(325.13280141,194.98827746)(325.1328125,196.55078245)
\curveto(325.13280141,198.10546185)(324.79686425,199.33592937)(324.125,200.2421887)
\curveto(323.45311559,201.15624004)(322.54296025,201.61327084)(321.39453125,201.61328245)
\moveto(321.39453125,203.44140745)
\curveto(323.26952203,203.44139401)(324.7421768,202.83201962)(325.8125,201.61328245)
\curveto(326.88279966,200.39452206)(327.41795538,198.70702374)(327.41796875,196.55078245)
\curveto(327.41795538,194.40234055)(326.88279966,192.71484224)(325.8125,191.48828245)
\curveto(324.7421768,190.26953218)(323.26952203,189.66015779)(321.39453125,189.66015745)
\curveto(319.51171329,189.66015779)(318.03515226,190.26953218)(316.96484375,191.48828245)
\curveto(315.90234189,192.71484224)(315.37109243,194.40234055)(315.37109375,196.55078245)
\curveto(315.37109243,198.70702374)(315.90234189,200.39452206)(316.96484375,201.61328245)
\curveto(318.03515226,202.83201962)(319.51171329,203.44139401)(321.39453125,203.44140745)
}
}
{
\newrgbcolor{curcolor}{1 1 1}
\pscustom[linestyle=none,fillstyle=solid,fillcolor=curcolor]
{
\newpath
\moveto(330,210.0000012)
\lineto(350,210.0000012)
\lineto(350,190.0000012)
\lineto(330,190.0000012)
\closepath
}
}
{
\newrgbcolor{curcolor}{0 0 0}
\pscustom[linewidth=2,linecolor=curcolor]
{
\newpath
\moveto(330,210.0000012)
\lineto(350,210.0000012)
\lineto(350,190.0000012)
\lineto(330,190.0000012)
\closepath
}
}
{
\newrgbcolor{curcolor}{0 0 0}
\pscustom[linewidth=2,linecolor=curcolor,linestyle=dashed,dash=8 8]
{
\newpath
\moveto(340,200)
\lineto(500,200)
}
}
{
\newrgbcolor{curcolor}{0 0 0}
\pscustom[linestyle=none,fillstyle=solid,fillcolor=curcolor]
{
\newpath
\moveto(350.46230536,195.15951776)
\lineto(337.3512526,199.98078409)
\lineto(350.46230608,204.80204936)
\curveto(348.367708,201.95557628)(348.37977712,198.06110708)(350.46230536,195.15951776)
\lineto(350.46230536,195.15951776)
\lineto(350.46230536,195.15951776)
\closepath
}
}
{
\newrgbcolor{curcolor}{0 0 0}
\pscustom[linestyle=none,fillstyle=solid,fillcolor=curcolor]
{
\newpath
\moveto(503.515625,140.4843762)
\lineto(503.515625,143.3593762)
\curveto(504.30728736,142.98437322)(505.10936989,142.69791517)(505.921875,142.5000012)
\curveto(506.73436827,142.30208223)(507.53124247,142.203124)(508.3125,142.2031262)
\curveto(510.39582294,142.203124)(511.98436302,142.90103997)(513.078125,144.2968762)
\curveto(514.18227748,145.7031205)(514.81248519,147.8333267)(514.96875,150.6875012)
\curveto(514.36456897,149.79165808)(513.59894473,149.10415876)(512.671875,148.6250012)
\curveto(511.74477992,148.14582639)(510.71873928,147.90624329)(509.59375,147.9062512)
\curveto(507.26040941,147.90624329)(505.41145292,148.60936759)(504.046875,150.0156262)
\curveto(502.69270564,151.43228144)(502.01562298,153.36457117)(502.015625,155.8125012)
\curveto(502.01562298,158.20831633)(502.72395561,160.1301894)(504.140625,161.5781262)
\curveto(505.55728611,163.02601984)(507.44270089,163.74997745)(509.796875,163.7500012)
\curveto(512.49477917,163.74997745)(514.55206878,162.71352015)(515.96875,160.6406262)
\curveto(517.39581594,158.57810762)(518.10935689,155.57811062)(518.109375,151.6406262)
\curveto(518.10935689,147.9635349)(517.23435777,145.02603784)(515.484375,142.8281262)
\curveto(513.74477792,140.64062556)(511.40103027,139.54687665)(508.453125,139.5468762)
\curveto(507.66145067,139.54687665)(506.85936814,139.62500158)(506.046875,139.7812512)
\curveto(505.23436977,139.93750126)(504.39062061,140.17187603)(503.515625,140.4843762)
\moveto(509.796875,150.3750012)
\curveto(511.21353045,150.37499083)(512.333321,150.85936534)(513.15625,151.8281262)
\curveto(513.98956934,152.7968634)(514.40623559,154.12498708)(514.40625,155.8125012)
\curveto(514.40623559,157.48956704)(513.98956934,158.81248239)(513.15625,159.7812512)
\curveto(512.333321,160.76039711)(511.21353045,161.24997995)(509.796875,161.2500012)
\curveto(508.38019995,161.24997995)(507.25520108,160.76039711)(506.421875,159.7812512)
\curveto(505.59895273,158.81248239)(505.18749481,157.48956704)(505.1875,155.8125012)
\curveto(505.18749481,154.12498708)(505.59895273,152.7968634)(506.421875,151.8281262)
\curveto(507.25520108,150.85936534)(508.38019995,150.37499083)(509.796875,150.3750012)
}
}
\end{pspicture}

		\end{center}
		
		\begin{enumerate}
		  \item Fond d'écran
		  \item Pseudonyme du compte hors ligne utilisé
		  \item Cadre contenant l'erreur survenue
		  \item Champs de texte ``Nom utilisateur''
		  \item Champs de texte ``Mot de passe''
		  \item Champs de texte ``Vérification''
		  \item Check Box ``Connexion auto''
		  \item Bouton ``\hyperlink{Connexion multi-joueurs}{Retour}''
		  \item Bouton ``\hyperlink{Accueil multi-joueurs}{Valider}''
		\end{enumerate}
		
		\subsubsection{Description des zones}
		
			\begin{tabular}{|c|c|c|c|c|} \hline
				Numéro de zone & Type  & Description & Evènement &	Règle \\\hline
				2 & Label & Affiche le pseudonyme du compte & Chargement de la page & RG4-01 \\
				  &       & hors ligne en cours d'utilisation & & \\\hline
				3 & Label & Affiche l'erreur & RG4-03 & RG4-02 \\
				  &       & rencontré par l'utilisateur & RG4-04 & \\\hline
				4 & Champs de texte & & Perte du focus & RG4-03 \\
				 & & & RG4-07 & \\\hline 
				6 & Champs de texte & & Perte du focus & RG4-04 \\
				 & & & RG4-07 & \\\hline 
				7 & Check box & Permet la connexion automatique & Cliqué & RG4-05 \\
				  &           & lors des prochaines utilisations&        & \\				
				  &           & du multi-joueurs                &        & \\\hline
				8 & Bouton & Permet de revenir à la page & Cliqué & RG4-06 \\
				  &        & de connexion multi-joueurs \footnotemark[1] & & \\\hline
				9 & Bouton & Valide les paramètres entrés & Cliqué & RG4-07 \\\hline
			\end{tabular}
			
		\subsubsection{Description des règles}

			\underline{RG4-01 :}
				\begin{quote}
					Récupérer le pseudonyme du compte hors ligne en cours d'utilisation stocké sur le téléphone.\\
					Afficher le pseudonyme.\\
				\end{quote}

			\underline{RG4-02 :}
				\begin{quote}
					Afficher l'erreur rencontrée.\\
				\end{quote}
				
			\underline{RG4-03 :}
				\begin{quote}
					Vérifier les caractères entrés.\\
					Si une erreur est rencontrée RG4-02.\\
				\end{quote}
				
			\underline{RG4-04 :}
				\begin{quote}
					Vérifier que les caractères entrés soient identiques à ceux entrés dans la zone 5.\\
					Si ce n'est pas le cas RG4-02.\\					
				\end{quote}	
				
			\underline{RG4-05 :}
				\begin{quote}
					
				\end{quote}
				
			\underline{RG4-06 :}
				\begin{quote}
					Afficher la page de connexion multi-joueurs%
						\footnote[1]{
							\hyperlink{Connexion multi-joueurs}{Connexion multi-joueurs}
							\og voir section \ref{Connexion multi-joueurs}, page \pageref{Connexion multi-joueurs}.\fg
						}.\\
					Cacher la page de création d'un compte multi-joueurs.\\
				\end{quote}

			\underline{RG4-07 :}
				\begin{quote}
					RG4-03.\\
					RG4-04.\\
					Si aucune erreur n'a été détectée alors une connexion avec le serveur distant
					a lieu afin d'ajouter le nouveau compte.\\
					Il est possible que le serveur renvoie une erreur.\\
					Si c'est le cas alors elle sera affichée sous la forme d'un pop-up.\\
					Sinon l'utilisateur sera redirigé vers l'accueil multi-joueurs%
						\footnote[2]{
							\hyperlink{Accueil multi-joueurs}{Accueil multi-joueurs}
							\og voir section \ref{Accueil multi-joueurs}, page \pageref{Accueil multi-joueurs}.\fg
						}.\\
					RG4-05.\\
				\end{quote}
	
\newpage

	\subsection{Connexion multi-joueurs}
	
		\hypertarget{Connexion multi-joueurs}{}
		\label{Connexion multi-joueurs}
	
		\begin{center}
			%LaTeX with PSTricks extensions
%%Creator: inkscape 0.48.0
%%Please note this file requires PSTricks extensions
\psset{xunit=.5pt,yunit=.5pt,runit=.5pt}
\begin{pspicture}(560,600)
{
\newrgbcolor{curcolor}{1 1 1}
\pscustom[linestyle=none,fillstyle=solid,fillcolor=curcolor]
{
\newpath
\moveto(133.12401581,597.52220317)
\lineto(426.87598419,597.52220317)
\curveto(443.85397169,597.52220317)(457.52217102,583.85400385)(457.52217102,566.87601635)
\lineto(457.52217102,33.12401744)
\curveto(457.52217102,16.14602994)(443.85397169,2.47783062)(426.87598419,2.47783062)
\lineto(133.12401581,2.47783062)
\curveto(116.14602831,2.47783062)(102.47782898,16.14602994)(102.47782898,33.12401744)
\lineto(102.47782898,566.87601635)
\curveto(102.47782898,583.85400385)(116.14602831,597.52220317)(133.12401581,597.52220317)
\closepath
}
}
{
\newrgbcolor{curcolor}{0 0 0}
\pscustom[linewidth=4.95566034,linecolor=curcolor]
{
\newpath
\moveto(133.12401581,597.52220317)
\lineto(426.87598419,597.52220317)
\curveto(443.85397169,597.52220317)(457.52217102,583.85400385)(457.52217102,566.87601635)
\lineto(457.52217102,33.12401744)
\curveto(457.52217102,16.14602994)(443.85397169,2.47783062)(426.87598419,2.47783062)
\lineto(133.12401581,2.47783062)
\curveto(116.14602831,2.47783062)(102.47782898,16.14602994)(102.47782898,33.12401744)
\lineto(102.47782898,566.87601635)
\curveto(102.47782898,583.85400385)(116.14602831,597.52220317)(133.12401581,597.52220317)
\closepath
}
}
{
\newrgbcolor{curcolor}{0 0 0}
\pscustom[linestyle=none,fillstyle=solid,fillcolor=curcolor,opacity=0]
{
\newpath
\moveto(150.43106842,457.69565174)
\lineto(407.85224152,457.69565174)
\curveto(426.47576608,457.69565174)(441.46871185,442.70270597)(441.46871185,424.0791814)
\lineto(441.46871185,133.61644336)
\curveto(441.46871185,114.9929188)(426.47576608,99.99997303)(407.85224152,99.99997303)
\lineto(150.43106842,99.99997303)
\curveto(131.80754385,99.99997303)(116.81459808,114.9929188)(116.81459808,133.61644336)
\lineto(116.81459808,424.0791814)
\curveto(116.81459808,442.70270597)(131.80754385,457.69565174)(150.43106842,457.69565174)
\closepath
}
}
{
\newrgbcolor{curcolor}{0 0 0}
\pscustom[linewidth=2.23000407,linecolor=curcolor]
{
\newpath
\moveto(150.43106842,457.69565174)
\lineto(407.85224152,457.69565174)
\curveto(426.47576608,457.69565174)(441.46871185,442.70270597)(441.46871185,424.0791814)
\lineto(441.46871185,133.61644336)
\curveto(441.46871185,114.9929188)(426.47576608,99.99997303)(407.85224152,99.99997303)
\lineto(150.43106842,99.99997303)
\curveto(131.80754385,99.99997303)(116.81459808,114.9929188)(116.81459808,133.61644336)
\lineto(116.81459808,424.0791814)
\curveto(116.81459808,442.70270597)(131.80754385,457.69565174)(150.43106842,457.69565174)
\closepath
}
}
{
\newrgbcolor{curcolor}{0 0 0}
\pscustom[linestyle=none,fillstyle=solid,fillcolor=curcolor,opacity=0.11636364]
{
\newpath
\moveto(290.35070133,330.00074932)
\lineto(419.84194088,330.00074932)
\curveto(425.53073339,330.00074932)(430.11051941,325.42096329)(430.11051941,319.73217079)
\lineto(430.11051941,310.37671157)
\curveto(430.11051941,304.68791907)(425.53073339,300.10813304)(419.84194088,300.10813304)
\lineto(290.35070133,300.10813304)
\curveto(284.66190883,300.10813304)(280.0821228,304.68791907)(280.0821228,310.37671157)
\lineto(280.0821228,319.73217079)
\curveto(280.0821228,325.42096329)(284.66190883,330.00074932)(290.35070133,330.00074932)
\closepath
}
}
{
\newrgbcolor{curcolor}{0 0 0}
\pscustom[linewidth=2,linecolor=curcolor]
{
\newpath
\moveto(290.35070133,330.00074932)
\lineto(419.84194088,330.00074932)
\curveto(425.53073339,330.00074932)(430.11051941,325.42096329)(430.11051941,319.73217079)
\lineto(430.11051941,310.37671157)
\curveto(430.11051941,304.68791907)(425.53073339,300.10813304)(419.84194088,300.10813304)
\lineto(290.35070133,300.10813304)
\curveto(284.66190883,300.10813304)(280.0821228,304.68791907)(280.0821228,310.37671157)
\lineto(280.0821228,319.73217079)
\curveto(280.0821228,325.42096329)(284.66190883,330.00074932)(290.35070133,330.00074932)
\closepath
}
}
{
\newrgbcolor{curcolor}{0 0 0}
\pscustom[linestyle=none,fillstyle=solid,fillcolor=curcolor,opacity=0.11636364]
{
\newpath
\moveto(300.28209114,280.21498271)
\lineto(419.86942863,280.21498271)
\curveto(425.52066758,280.21498271)(430.07022095,275.66542934)(430.07022095,270.0141904)
\lineto(430.07022095,260.72049495)
\curveto(430.07022095,255.06925601)(425.52066758,250.51970264)(419.86942863,250.51970264)
\lineto(300.28209114,250.51970264)
\curveto(294.6308522,250.51970264)(290.08129883,255.06925601)(290.08129883,260.72049495)
\lineto(290.08129883,270.0141904)
\curveto(290.08129883,275.66542934)(294.6308522,280.21498271)(300.28209114,280.21498271)
\closepath
}
}
{
\newrgbcolor{curcolor}{0 0 0}
\pscustom[linewidth=2,linecolor=curcolor]
{
\newpath
\moveto(300.28209114,280.21498271)
\lineto(419.86942863,280.21498271)
\curveto(425.52066758,280.21498271)(430.07022095,275.66542934)(430.07022095,270.0141904)
\lineto(430.07022095,260.72049495)
\curveto(430.07022095,255.06925601)(425.52066758,250.51970264)(419.86942863,250.51970264)
\lineto(300.28209114,250.51970264)
\curveto(294.6308522,250.51970264)(290.08129883,255.06925601)(290.08129883,260.72049495)
\lineto(290.08129883,270.0141904)
\curveto(290.08129883,275.66542934)(294.6308522,280.21498271)(300.28209114,280.21498271)
\closepath
}
}
{
\newrgbcolor{curcolor}{0 0 0}
\pscustom[linestyle=none,fillstyle=solid,fillcolor=curcolor,opacity=0]
{
\newpath
\moveto(294.88945961,140.00001689)
\lineto(415.27087975,140.00001689)
\curveto(423.51964037,140.00001689)(430.16033936,133.35931791)(430.16033936,125.11055728)
\curveto(430.16033936,116.86179666)(423.51964037,110.22109767)(415.27087975,110.22109767)
\lineto(294.88945961,110.22109767)
\curveto(286.64069899,110.22109767)(280,116.86179666)(280,125.11055728)
\curveto(280,133.35931791)(286.64069899,140.00001689)(294.88945961,140.00001689)
\closepath
}
}
{
\newrgbcolor{curcolor}{0 0 0}
\pscustom[linewidth=2,linecolor=curcolor]
{
\newpath
\moveto(294.88945961,140.00001689)
\lineto(415.27087975,140.00001689)
\curveto(423.51964037,140.00001689)(430.16033936,133.35931791)(430.16033936,125.11055728)
\curveto(430.16033936,116.86179666)(423.51964037,110.22109767)(415.27087975,110.22109767)
\lineto(294.88945961,110.22109767)
\curveto(286.64069899,110.22109767)(280,116.86179666)(280,125.11055728)
\curveto(280,133.35931791)(286.64069899,140.00001689)(294.88945961,140.00001689)
\closepath
}
}
{
\newrgbcolor{curcolor}{0 0 0}
\pscustom[linestyle=none,fillstyle=solid,fillcolor=curcolor,opacity=0]
{
\newpath
\moveto(144.83261013,139.6113145)
\lineto(245.30122471,139.6113145)
\curveto(253.51662253,139.6113145)(260.13046265,132.99747481)(260.13046265,124.78207752)
\curveto(260.13046265,116.56668023)(253.51662253,109.95284053)(245.30122471,109.95284053)
\lineto(144.83261013,109.95284053)
\curveto(136.61721231,109.95284053)(130.00337219,116.56668023)(130.00337219,124.78207752)
\curveto(130.00337219,132.99747481)(136.61721231,139.6113145)(144.83261013,139.6113145)
\closepath
}
}
{
\newrgbcolor{curcolor}{0 0 0}
\pscustom[linewidth=2,linecolor=curcolor]
{
\newpath
\moveto(144.83261013,139.6113145)
\lineto(245.30122471,139.6113145)
\curveto(253.51662253,139.6113145)(260.13046265,132.99747481)(260.13046265,124.78207752)
\curveto(260.13046265,116.56668023)(253.51662253,109.95284053)(245.30122471,109.95284053)
\lineto(144.83261013,109.95284053)
\curveto(136.61721231,109.95284053)(130.00337219,116.56668023)(130.00337219,124.78207752)
\curveto(130.00337219,132.99747481)(136.61721231,139.6113145)(144.83261013,139.6113145)
\closepath
}
}
{
\newrgbcolor{curcolor}{0 0 0}
\pscustom[linestyle=none,fillstyle=solid,fillcolor=curcolor]
{
\newpath
\moveto(158.30810547,135.10872051)
\lineto(155.09716797,126.40168926)
\lineto(161.53076172,126.40168926)
\lineto(158.30810547,135.10872051)
\moveto(156.97216797,137.44075176)
\lineto(159.65576172,137.44075176)
\lineto(166.32373047,119.94465801)
\lineto(163.86279297,119.94465801)
\lineto(162.26904297,124.43293926)
\lineto(154.38232422,124.43293926)
\lineto(152.78857422,119.94465801)
\lineto(150.29248047,119.94465801)
\lineto(156.97216797,137.44075176)
}
}
{
\newrgbcolor{curcolor}{0 0 0}
\pscustom[linestyle=none,fillstyle=solid,fillcolor=curcolor]
{
\newpath
\moveto(179.68310547,127.86653301)
\lineto(179.68310547,119.94465801)
\lineto(177.52685547,119.94465801)
\lineto(177.52685547,127.79622051)
\curveto(177.52684445,129.03839891)(177.2846572,129.96808548)(176.80029297,130.58528301)
\curveto(176.31590816,131.20245925)(175.58934639,131.51105269)(174.62060547,131.51106426)
\curveto(173.45653602,131.51105269)(172.53856819,131.13995931)(171.86669922,130.39778301)
\curveto(171.19481954,129.6555858)(170.85888237,128.64386806)(170.85888672,127.36262676)
\lineto(170.85888672,119.94465801)
\lineto(168.69091797,119.94465801)
\lineto(168.69091797,133.06965801)
\lineto(170.85888672,133.06965801)
\lineto(170.85888672,131.03059551)
\curveto(171.37450686,131.81964613)(171.979975,132.40948929)(172.67529297,132.80012676)
\curveto(173.3784111,133.19073851)(174.18700404,133.38605082)(175.10107422,133.38606426)
\curveto(176.60887662,133.38605082)(177.74950048,132.91730129)(178.52294922,131.97981426)
\curveto(179.29637393,131.05011565)(179.6830923,129.67902327)(179.68310547,127.86653301)
}
}
{
\newrgbcolor{curcolor}{0 0 0}
\pscustom[linestyle=none,fillstyle=solid,fillcolor=curcolor]
{
\newpath
\moveto(194.91748047,127.86653301)
\lineto(194.91748047,119.94465801)
\lineto(192.76123047,119.94465801)
\lineto(192.76123047,127.79622051)
\curveto(192.76121945,129.03839891)(192.5190322,129.96808548)(192.03466797,130.58528301)
\curveto(191.55028316,131.20245925)(190.82372139,131.51105269)(189.85498047,131.51106426)
\curveto(188.69091102,131.51105269)(187.77294319,131.13995931)(187.10107422,130.39778301)
\curveto(186.42919454,129.6555858)(186.09325737,128.64386806)(186.09326172,127.36262676)
\lineto(186.09326172,119.94465801)
\lineto(183.92529297,119.94465801)
\lineto(183.92529297,133.06965801)
\lineto(186.09326172,133.06965801)
\lineto(186.09326172,131.03059551)
\curveto(186.60888186,131.81964613)(187.21435,132.40948929)(187.90966797,132.80012676)
\curveto(188.6127861,133.19073851)(189.42137904,133.38605082)(190.33544922,133.38606426)
\curveto(191.84325162,133.38605082)(192.98387548,132.91730129)(193.75732422,131.97981426)
\curveto(194.53074893,131.05011565)(194.9174673,129.67902327)(194.91748047,127.86653301)
}
}
{
\newrgbcolor{curcolor}{0 0 0}
\pscustom[linestyle=none,fillstyle=solid,fillcolor=curcolor]
{
\newpath
\moveto(199.01904297,125.12434551)
\lineto(199.01904297,133.06965801)
\lineto(201.17529297,133.06965801)
\lineto(201.17529297,125.20637676)
\curveto(201.17528877,123.96418524)(201.41747603,123.03059242)(201.90185547,122.40559551)
\curveto(202.38622506,121.78840616)(203.11278684,121.47981272)(204.08154297,121.47981426)
\curveto(205.2455972,121.47981272)(206.16356504,121.8509061)(206.83544922,122.59309551)
\curveto(207.51512618,123.33527962)(207.85496959,124.34699736)(207.85498047,125.62825176)
\lineto(207.85498047,133.06965801)
\lineto(210.01123047,133.06965801)
\lineto(210.01123047,119.94465801)
\lineto(207.85498047,119.94465801)
\lineto(207.85498047,121.96028301)
\curveto(207.33153262,121.16340679)(206.72215823,120.56965738)(206.02685547,120.17903301)
\curveto(205.33934711,119.79622066)(204.53856666,119.6048146)(203.62451172,119.60481426)
\curveto(202.11669408,119.6048146)(200.97216398,120.07356413)(200.19091797,121.01106426)
\curveto(199.40966554,121.94856225)(199.01904093,123.31965463)(199.01904297,125.12434551)
\moveto(204.44482422,133.38606426)
\lineto(204.44482422,133.38606426)
}
}
{
\newrgbcolor{curcolor}{0 0 0}
\pscustom[linestyle=none,fillstyle=solid,fillcolor=curcolor]
{
\newpath
\moveto(214.47607422,138.17903301)
\lineto(216.63232422,138.17903301)
\lineto(216.63232422,119.94465801)
\lineto(214.47607422,119.94465801)
\lineto(214.47607422,138.17903301)
}
}
{
\newrgbcolor{curcolor}{0 0 0}
\pscustom[linestyle=none,fillstyle=solid,fillcolor=curcolor]
{
\newpath
\moveto(232.35888672,127.04622051)
\lineto(232.35888672,125.99153301)
\lineto(222.44482422,125.99153301)
\curveto(222.53857055,124.50715345)(222.98388261,123.37434208)(223.78076172,122.59309551)
\curveto(224.5854435,121.81965613)(225.70262989,121.43293777)(227.13232422,121.43293926)
\curveto(227.96044013,121.43293777)(228.76122058,121.53450017)(229.53466797,121.73762676)
\curveto(230.31590652,121.94074976)(231.08934325,122.24543696)(231.85498047,122.65168926)
\lineto(231.85498047,120.61262676)
\curveto(231.08153076,120.28450142)(230.2885628,120.03450167)(229.47607422,119.86262676)
\curveto(228.66356443,119.69075201)(227.8393465,119.6048146)(227.00341797,119.60481426)
\curveto(224.90966193,119.6048146)(223.24950734,120.21418899)(222.02294922,121.43293926)
\curveto(220.80419729,122.65168655)(220.19482289,124.3001224)(220.19482422,126.37825176)
\curveto(220.19482289,128.52668068)(220.77294732,130.22980397)(221.92919922,131.48762676)
\curveto(223.0932575,132.75323895)(224.65966218,133.38605082)(226.62841797,133.38606426)
\curveto(228.39403345,133.38605082)(229.7885633,132.81573889)(230.81201172,131.67512676)
\curveto(231.84324875,130.54230366)(232.35887323,128.99933645)(232.35888672,127.04622051)
\moveto(230.20263672,127.67903301)
\curveto(230.1870004,128.85871159)(229.85496948,129.8001169)(229.20654297,130.50325176)
\curveto(228.56590827,131.2063655)(227.71434662,131.55792764)(226.65185547,131.55793926)
\curveto(225.44872389,131.55792764)(224.48388111,131.21808423)(223.75732422,130.53840801)
\curveto(223.03857005,129.85871059)(222.62450796,128.9016803)(222.51513672,127.66731426)
\lineto(230.20263672,127.67903301)
}
}
{
\newrgbcolor{curcolor}{0 0 0}
\pscustom[linestyle=none,fillstyle=solid,fillcolor=curcolor]
{
\newpath
\moveto(243.50341797,131.05403301)
\curveto(243.26122084,131.19464676)(242.99559611,131.29620916)(242.70654297,131.35872051)
\curveto(242.42528418,131.42902152)(242.11278449,131.46417774)(241.76904297,131.46418926)
\curveto(240.55028605,131.46417774)(239.61278699,131.06574064)(238.95654297,130.26887676)
\curveto(238.3081008,129.47980472)(237.98388237,128.34308711)(237.98388672,126.85872051)
\lineto(237.98388672,119.94465801)
\lineto(235.81591797,119.94465801)
\lineto(235.81591797,133.06965801)
\lineto(237.98388672,133.06965801)
\lineto(237.98388672,131.03059551)
\curveto(238.43700692,131.82745863)(239.02685008,132.41730179)(239.75341797,132.80012676)
\curveto(240.47997363,133.19073851)(241.36278524,133.38605082)(242.40185547,133.38606426)
\curveto(242.55028405,133.38605082)(242.71434639,133.37433208)(242.89404297,133.35090801)
\curveto(243.07372103,133.33526962)(243.27293958,133.30792589)(243.49169922,133.26887676)
\lineto(243.50341797,131.05403301)
}
}
{
\newrgbcolor{curcolor}{0 0 0}
\pscustom[linestyle=none,fillstyle=solid,fillcolor=curcolor]
{
\newpath
\moveto(304.36921127,135.01194477)
\lineto(304.36921127,132.69151869)
\curveto(303.62840434,133.38145986)(302.83677307,133.89710958)(301.99431509,134.23846941)
\curveto(301.1590958,134.57980077)(300.26941846,134.75047356)(299.3252804,134.75048832)
\curveto(297.466027,134.75047356)(296.04254325,134.1803538)(295.05482489,133.04012731)
\curveto(294.06709642,131.90713741)(293.57323471,130.2657735)(293.57323828,128.11603065)
\curveto(293.57323471,125.97353424)(294.06709642,124.33217033)(295.05482489,123.19193399)
\curveto(296.04254325,122.05895394)(297.466027,121.49246551)(299.3252804,121.492467)
\curveto(300.26941846,121.49246551)(301.1590958,121.66313831)(301.99431509,122.0044859)
\curveto(302.83677307,122.34582949)(303.62840434,122.86147921)(304.36921127,123.55143662)
\lineto(304.36921127,121.25279858)
\curveto(303.59935365,120.72988493)(302.78230303,120.33770063)(301.91805696,120.07624451)
\curveto(301.06104972,119.81478824)(300.1532157,119.68406014)(299.19455217,119.68405982)
\curveto(296.73249711,119.68406014)(294.79336364,120.43574671)(293.37714594,121.93912179)
\curveto(291.96092149,123.44975566)(291.25281095,125.50872322)(291.2528122,128.11603065)
\curveto(291.25281095,130.73058452)(291.96092149,132.78955208)(293.37714594,134.29293951)
\curveto(294.79336364,135.80356103)(296.73249711,136.55887894)(299.19455217,136.5588955)
\curveto(300.16774105,136.55887894)(301.08283774,136.42815084)(301.939845,136.16671081)
\curveto(302.80409104,135.91250111)(303.61387899,135.52757949)(304.36921127,135.01194477)
}
}
{
\newrgbcolor{curcolor}{0 0 0}
\pscustom[linestyle=none,fillstyle=solid,fillcolor=curcolor]
{
\newpath
\moveto(312.430785,130.79595936)
\curveto(311.35590268,130.79594857)(310.50617004,130.37471358)(309.88158451,129.53225314)
\curveto(309.25699043,128.69703631)(308.94469552,127.5495341)(308.94469887,126.08974309)
\curveto(308.94469552,124.62993989)(309.25335909,123.47880635)(309.87069049,122.63633901)
\curveto(310.49527603,121.80112908)(311.34864001,121.38352543)(312.430785,121.38352681)
\curveto(313.49839098,121.38352543)(314.34449228,121.80476041)(314.96909146,122.64723303)
\curveto(315.5936719,123.48970036)(315.9059668,124.63720256)(315.90597711,126.08974309)
\curveto(315.9059668,127.53500876)(315.5936719,128.67887963)(314.96909146,129.52135912)
\curveto(314.34449228,130.37108224)(313.49839098,130.79594857)(312.430785,130.79595936)
\moveto(312.430785,132.49542635)
\curveto(314.17381949,132.49541385)(315.54283319,131.92892542)(316.53783022,130.79595936)
\curveto(317.53280537,129.66297171)(318.03029841,128.09423452)(318.03031084,126.08974309)
\curveto(318.03029841,124.09250215)(317.53280537,122.52376496)(316.53783022,121.38352681)
\curveto(315.54283319,120.25054857)(314.17381949,119.68406014)(312.430785,119.68405982)
\curveto(310.68047417,119.68406014)(309.30782913,120.25054857)(308.31284576,121.38352681)
\curveto(307.32511963,122.52376496)(306.83125792,124.09250215)(306.83125915,126.08974309)
\curveto(306.83125792,128.09423452)(307.32511963,129.66297171)(308.31284576,130.79595936)
\curveto(309.30782913,131.92892542)(310.68047417,132.49541385)(312.430785,132.49542635)
}
}
{
\newrgbcolor{curcolor}{0 0 0}
\pscustom[linestyle=none,fillstyle=solid,fillcolor=curcolor]
{
\newpath
\moveto(331.48442526,127.36434333)
\lineto(331.48442526,119.99998638)
\lineto(329.47992574,119.99998638)
\lineto(329.47992574,127.29897921)
\curveto(329.4799155,128.45373679)(329.25477266,129.31799478)(328.80449655,129.89175577)
\curveto(328.35420131,130.46549698)(327.6787728,130.75237253)(326.77820899,130.75238328)
\curveto(325.69606329,130.75237253)(324.84269931,130.4073956)(324.21811448,129.71745146)
\curveto(323.5935197,129.02748789)(323.28122479,128.08697184)(323.28122884,126.8959005)
\lineto(323.28122884,119.99998638)
\lineto(321.26583529,119.99998638)
\lineto(321.26583529,132.20128783)
\lineto(323.28122884,132.20128783)
\lineto(323.28122884,130.3057285)
\curveto(323.76056116,131.03924808)(324.32341825,131.58757983)(324.96980181,131.95072539)
\curveto(325.62343657,132.31384705)(326.37512314,132.49541385)(327.22486377,132.49542635)
\curveto(328.62655152,132.49541385)(329.68690165,132.05965352)(330.40591737,131.18814405)
\curveto(331.12491074,130.32387487)(331.48441302,129.04927591)(331.48442526,127.36434333)
}
}
{
\newrgbcolor{curcolor}{0 0 0}
\pscustom[linestyle=none,fillstyle=solid,fillcolor=curcolor]
{
\newpath
\moveto(345.64664908,127.36434333)
\lineto(345.64664908,119.99998638)
\lineto(343.64214955,119.99998638)
\lineto(343.64214955,127.29897921)
\curveto(343.64213931,128.45373679)(343.41699647,129.31799478)(342.96672036,129.89175577)
\curveto(342.51642513,130.46549698)(341.84099661,130.75237253)(340.9404328,130.75238328)
\curveto(339.85828711,130.75237253)(339.00492313,130.4073956)(338.3803383,129.71745146)
\curveto(337.75574351,129.02748789)(337.44344861,128.08697184)(337.44345265,126.8959005)
\lineto(337.44345265,119.99998638)
\lineto(335.42805911,119.99998638)
\lineto(335.42805911,132.20128783)
\lineto(337.44345265,132.20128783)
\lineto(337.44345265,130.3057285)
\curveto(337.92278497,131.03924808)(338.48564207,131.58757983)(339.13202562,131.95072539)
\curveto(339.78566039,132.31384705)(340.53734696,132.49541385)(341.38708759,132.49542635)
\curveto(342.78877533,132.49541385)(343.84912547,132.05965352)(344.56814118,131.18814405)
\curveto(345.28713456,130.32387487)(345.64663683,129.04927591)(345.64664908,127.36434333)
}
}
{
\newrgbcolor{curcolor}{0 0 0}
\pscustom[linestyle=none,fillstyle=solid,fillcolor=curcolor]
{
\newpath
\moveto(360.10301141,126.60176199)
\lineto(360.10301141,125.62130026)
\lineto(350.8866712,125.62130026)
\curveto(350.97381995,124.24138693)(351.38779226,123.18829946)(352.12858939,122.46203471)
\curveto(352.87664006,121.7430277)(353.91520218,121.38352543)(355.24427887,121.38352681)
\curveto(356.01411444,121.38352543)(356.75853834,121.47794016)(357.47755279,121.66677131)
\curveto(358.2038101,121.85559912)(358.92281464,122.13884333)(359.63456859,122.5165048)
\lineto(359.63456859,120.62094547)
\curveto(358.91555197,120.31591262)(358.17839075,120.08350711)(357.4230827,119.92372824)
\curveto(356.66775493,119.76394953)(355.90154302,119.68406014)(355.12444466,119.68405982)
\curveto(353.17804095,119.68406014)(351.63472312,120.25054857)(350.49448651,121.38352681)
\curveto(349.36150672,122.51650229)(348.7950183,124.04892611)(348.79501953,125.98080289)
\curveto(348.7950183,127.97803176)(349.33245604,129.5612943)(350.40733436,130.73059525)
\curveto(351.48946967,131.90713741)(352.94563544,132.49541385)(354.77583604,132.49542635)
\curveto(356.41719274,132.49541385)(357.71357973,131.96523879)(358.66500088,130.90489955)
\curveto(359.62366251,129.85180118)(360.10299887,128.41742343)(360.10301141,126.60176199)
\moveto(358.09851189,127.19003902)
\curveto(358.08397601,128.28669533)(357.77531244,129.16184733)(357.17252026,129.81549764)
\curveto(356.57697153,130.46912832)(355.78534026,130.79594857)(354.79762408,130.79595936)
\curveto(353.67916533,130.79594857)(352.78222532,130.48002233)(352.10680135,129.84817969)
\curveto(351.43863097,129.21631737)(351.05370934,128.32664003)(350.95203532,127.179145)
\lineto(358.09851189,127.19003902)
}
}
{
\newrgbcolor{curcolor}{0 0 0}
\pscustom[linestyle=none,fillstyle=solid,fillcolor=curcolor]
{
\newpath
\moveto(373.14315481,132.20128783)
\lineto(368.73107706,126.26404739)
\lineto(373.37192922,119.99998638)
\lineto(371.00792706,119.99998638)
\lineto(367.45647682,124.79335481)
\lineto(363.90502657,119.99998638)
\lineto(361.54102441,119.99998638)
\lineto(366.27992275,126.3838816)
\lineto(361.94410312,132.20128783)
\lineto(364.30810528,132.20128783)
\lineto(367.54362897,127.85457419)
\lineto(370.77915266,132.20128783)
\lineto(373.14315481,132.20128783)
}
}
{
\newrgbcolor{curcolor}{0 0 0}
\pscustom[linestyle=none,fillstyle=solid,fillcolor=curcolor]
{
\newpath
\moveto(376.20437523,132.20128783)
\lineto(378.20887475,132.20128783)
\lineto(378.20887475,119.99998638)
\lineto(376.20437523,119.99998638)
\lineto(376.20437523,132.20128783)
\moveto(376.20437523,136.95108018)
\lineto(378.20887475,136.95108018)
\lineto(378.20887475,134.41277372)
\lineto(376.20437523,134.41277372)
\lineto(376.20437523,136.95108018)
}
}
{
\newrgbcolor{curcolor}{0 0 0}
\pscustom[linestyle=none,fillstyle=solid,fillcolor=curcolor]
{
\newpath
\moveto(387.12017709,130.79595936)
\curveto(386.04529477,130.79594857)(385.19556213,130.37471358)(384.5709766,129.53225314)
\curveto(383.94638252,128.69703631)(383.63408761,127.5495341)(383.63409096,126.08974309)
\curveto(383.63408761,124.62993989)(383.94275118,123.47880635)(384.56008258,122.63633901)
\curveto(385.18466812,121.80112908)(386.0380321,121.38352543)(387.12017709,121.38352681)
\curveto(388.18778307,121.38352543)(389.03388437,121.80476041)(389.65848355,122.64723303)
\curveto(390.28306399,123.48970036)(390.59535889,124.63720256)(390.5953692,126.08974309)
\curveto(390.59535889,127.53500876)(390.28306399,128.67887963)(389.65848355,129.52135912)
\curveto(389.03388437,130.37108224)(388.18778307,130.79594857)(387.12017709,130.79595936)
\moveto(387.12017709,132.49542635)
\curveto(388.86321158,132.49541385)(390.23222528,131.92892542)(391.22722231,130.79595936)
\curveto(392.22219746,129.66297171)(392.7196905,128.09423452)(392.71970293,126.08974309)
\curveto(392.7196905,124.09250215)(392.22219746,122.52376496)(391.22722231,121.38352681)
\curveto(390.23222528,120.25054857)(388.86321158,119.68406014)(387.12017709,119.68405982)
\curveto(385.36986626,119.68406014)(383.99722122,120.25054857)(383.00223785,121.38352681)
\curveto(382.01451172,122.52376496)(381.52065001,124.09250215)(381.52065124,126.08974309)
\curveto(381.52065001,128.09423452)(382.01451172,129.66297171)(383.00223785,130.79595936)
\curveto(383.99722122,131.92892542)(385.36986626,132.49541385)(387.12017709,132.49542635)
}
}
{
\newrgbcolor{curcolor}{0 0 0}
\pscustom[linestyle=none,fillstyle=solid,fillcolor=curcolor]
{
\newpath
\moveto(406.17381735,127.36434333)
\lineto(406.17381735,119.99998638)
\lineto(404.16931783,119.99998638)
\lineto(404.16931783,127.29897921)
\curveto(404.16930759,128.45373679)(403.94416475,129.31799478)(403.49388864,129.89175577)
\curveto(403.0435934,130.46549698)(402.36816489,130.75237253)(401.46760108,130.75238328)
\curveto(400.38545538,130.75237253)(399.5320914,130.4073956)(398.90750657,129.71745146)
\curveto(398.28291179,129.02748789)(397.97061688,128.08697184)(397.97062093,126.8959005)
\lineto(397.97062093,119.99998638)
\lineto(395.95522738,119.99998638)
\lineto(395.95522738,132.20128783)
\lineto(397.97062093,132.20128783)
\lineto(397.97062093,130.3057285)
\curveto(398.44995325,131.03924808)(399.01281034,131.58757983)(399.6591939,131.95072539)
\curveto(400.31282866,132.31384705)(401.06451523,132.49541385)(401.91425586,132.49542635)
\curveto(403.31594361,132.49541385)(404.37629374,132.05965352)(405.09530945,131.18814405)
\curveto(405.81430283,130.32387487)(406.17380511,129.04927591)(406.17381735,127.36434333)
}
}
{
\newrgbcolor{curcolor}{1 1 1}
\pscustom[linestyle=none,fillstyle=solid,fillcolor=curcolor]
{
\newpath
\moveto(129.960495,479.98731395)
\lineto(279.98731995,479.98731395)
\lineto(279.98731995,450.11458942)
\lineto(129.960495,450.11458942)
\closepath
}
}
{
\newrgbcolor{curcolor}{0 0 0}
\pscustom[linewidth=2,linecolor=curcolor]
{
\newpath
\moveto(129.960495,479.98731395)
\lineto(279.98731995,479.98731395)
\lineto(279.98731995,450.11458942)
\lineto(129.960495,450.11458942)
\closepath
}
}
{
\newrgbcolor{curcolor}{0 0 0}
\pscustom[linestyle=none,fillstyle=solid,fillcolor=curcolor,opacity=0]
{
\newpath
\moveto(294.7643919,450.05025288)
\lineto(415.16532612,450.05025288)
\curveto(423.41158085,450.05025288)(430.05026245,443.41157128)(430.05026245,435.16531655)
\curveto(430.05026245,426.91906182)(423.41158085,420.28038022)(415.16532612,420.28038022)
\lineto(294.7643919,420.28038022)
\curveto(286.51813717,420.28038022)(279.87945557,426.91906182)(279.87945557,435.16531655)
\curveto(279.87945557,443.41157128)(286.51813717,450.05025288)(294.7643919,450.05025288)
\closepath
}
}
{
\newrgbcolor{curcolor}{0 0 0}
\pscustom[linewidth=2,linecolor=curcolor]
{
\newpath
\moveto(294.7643919,450.05025288)
\lineto(415.16532612,450.05025288)
\curveto(423.41158085,450.05025288)(430.05026245,443.41157128)(430.05026245,435.16531655)
\curveto(430.05026245,426.91906182)(423.41158085,420.28038022)(415.16532612,420.28038022)
\lineto(294.7643919,420.28038022)
\curveto(286.51813717,420.28038022)(279.87945557,426.91906182)(279.87945557,435.16531655)
\curveto(279.87945557,443.41157128)(286.51813717,450.05025288)(294.7643919,450.05025288)
\closepath
}
}
{
\newrgbcolor{curcolor}{0 0 0}
\pscustom[linestyle=none,fillstyle=solid,fillcolor=curcolor]
{
\newpath
\moveto(292.15917969,446.03808567)
\lineto(294.32910156,446.03808567)
\lineto(294.32910156,429.99999973)
\lineto(292.15917969,429.99999973)
\lineto(292.15917969,446.03808567)
}
}
{
\newrgbcolor{curcolor}{0 0 0}
\pscustom[linestyle=none,fillstyle=solid,fillcolor=curcolor]
{
\newpath
\moveto(308.5625,437.26171848)
\lineto(308.5625,429.99999973)
\lineto(306.5859375,429.99999973)
\lineto(306.5859375,437.19726535)
\curveto(306.5859274,438.33592889)(306.36392242,439.18814158)(305.91992188,439.75390598)
\curveto(305.47590247,440.31965087)(304.80988751,440.60252819)(303.921875,440.60253879)
\curveto(302.85481134,440.60252819)(302.01334083,440.26235926)(301.39746094,439.58203098)
\curveto(300.78157123,438.90168354)(300.47362883,437.97427561)(300.47363281,436.79980442)
\lineto(300.47363281,429.99999973)
\lineto(298.48632812,429.99999973)
\lineto(298.48632812,442.03124973)
\lineto(300.47363281,442.03124973)
\lineto(300.47363281,440.1621091)
\curveto(300.9462846,440.88540551)(301.50129707,441.42609507)(302.13867188,441.78417942)
\curveto(302.78319683,442.14224019)(303.52440703,442.32127647)(304.36230469,442.32128879)
\curveto(305.74445689,442.32127647)(306.79002876,441.8915894)(307.49902344,441.03222629)
\curveto(308.20799609,440.18000257)(308.56248793,438.92316789)(308.5625,437.26171848)
}
}
{
\newrgbcolor{curcolor}{0 0 0}
\pscustom[linestyle=none,fillstyle=solid,fillcolor=curcolor]
{
\newpath
\moveto(320.19628906,441.67675754)
\lineto(320.19628906,439.80761692)
\curveto(319.63768613,440.09406516)(319.05760858,440.30890869)(318.45605469,440.45214817)
\curveto(317.85448479,440.59536674)(317.23143853,440.66698125)(316.58691406,440.66699192)
\curveto(315.60578912,440.66698125)(314.86815965,440.51659077)(314.37402344,440.21582004)
\curveto(313.88704084,439.91502888)(313.6435515,439.46385745)(313.64355469,438.86230442)
\curveto(313.6435515,438.40396268)(313.81900705,438.04230939)(314.16992188,437.77734348)
\curveto(314.52082927,437.51952346)(315.22623221,437.27245339)(316.28613281,437.03613254)
\lineto(316.96289062,436.88574192)
\curveto(318.36652854,436.58495408)(319.36197026,436.15884774)(319.94921875,435.6074216)
\curveto(320.5436097,435.06314571)(320.84080992,434.30045116)(320.84082031,433.31933567)
\curveto(320.84080992,432.20214596)(320.39679995,431.31770674)(319.50878906,430.66601535)
\curveto(318.62792151,430.01432263)(317.41405554,429.6884766)(315.8671875,429.68847629)
\curveto(315.22265148,429.6884766)(314.54947507,429.75292966)(313.84765625,429.88183567)
\curveto(313.15299209,430.00358045)(312.41894335,430.18977818)(311.64550781,430.44042942)
\lineto(311.64550781,432.48144504)
\curveto(312.37597464,432.10188565)(313.09570048,431.8154276)(313.8046875,431.62207004)
\curveto(314.51366781,431.43587069)(315.21549003,431.34277182)(315.91015625,431.34277317)
\curveto(316.84113945,431.34277182)(317.55728456,431.50032375)(318.05859375,431.81542942)
\curveto(318.55988773,432.1376929)(318.81053852,432.58886433)(318.81054688,433.16894504)
\curveto(318.81053852,433.70605071)(318.62792151,434.11783415)(318.26269531,434.4042966)
\curveto(317.90461494,434.69075025)(317.11327459,434.96646612)(315.88867188,435.23144504)
\lineto(315.20117188,435.39257785)
\curveto(313.97655898,435.6503847)(313.09211976,436.04426452)(312.54785156,436.57421848)
\curveto(312.00357918,437.11132074)(311.73144403,437.84536949)(311.73144531,438.77636692)
\curveto(311.73144403,439.90786742)(312.1324853,440.78156447)(312.93457031,441.39746067)
\curveto(313.73665036,442.01333407)(314.8753211,442.32127647)(316.35058594,442.32128879)
\curveto(317.08104806,442.32127647)(317.76854737,442.26756559)(318.41308594,442.16015598)
\curveto(319.05760858,442.05272205)(319.65200903,441.8915894)(320.19628906,441.67675754)
}
}
{
\newrgbcolor{curcolor}{0 0 0}
\pscustom[linestyle=none,fillstyle=solid,fillcolor=curcolor]
{
\newpath
\moveto(332.65722656,441.56933567)
\lineto(332.65722656,439.72167942)
\curveto(332.09862264,440.02961209)(331.53644872,440.25877853)(330.97070312,440.40917942)
\curveto(330.41210089,440.56672093)(329.84634625,440.6454969)(329.2734375,440.64550754)
\curveto(327.99153039,440.6454969)(326.99608868,440.23729418)(326.28710938,439.42089817)
\curveto(325.57812135,438.61164476)(325.22362951,437.47297403)(325.22363281,436.00488254)
\curveto(325.22362951,434.53677905)(325.57812135,433.39452758)(326.28710938,432.57812473)
\curveto(326.99608868,431.76887817)(327.99153039,431.36425618)(329.2734375,431.36425754)
\curveto(329.84634625,431.36425618)(330.41210089,431.43945141)(330.97070312,431.58984348)
\curveto(331.53644872,431.74739381)(332.09862264,431.98014098)(332.65722656,432.28808567)
\lineto(332.65722656,430.46191379)
\curveto(332.10578409,430.20410109)(331.532868,430.01074191)(330.93847656,429.88183567)
\curveto(330.35122855,429.75292966)(329.72460158,429.6884766)(329.05859375,429.68847629)
\curveto(327.24673947,429.6884766)(325.80728779,430.25781197)(324.74023438,431.3964841)
\curveto(323.67317534,432.53515344)(323.13964722,434.07128472)(323.13964844,436.00488254)
\curveto(323.13964722,437.96711416)(323.67675606,439.51040688)(324.75097656,440.63476535)
\curveto(325.83235286,441.75910255)(327.31119253,442.32127647)(329.1875,442.32128879)
\curveto(329.79621609,442.32127647)(330.39061654,442.25682341)(330.97070312,442.12792942)
\curveto(331.55077162,442.00617262)(332.11294554,441.81997489)(332.65722656,441.56933567)
}
}
{
\newrgbcolor{curcolor}{0 0 0}
\pscustom[linestyle=none,fillstyle=solid,fillcolor=curcolor]
{
\newpath
\moveto(343.08789062,440.18359348)
\curveto(342.86587659,440.31248942)(342.62238725,440.40558828)(342.35742188,440.46289035)
\curveto(342.09960132,440.52733295)(341.81314327,440.55955948)(341.49804688,440.55957004)
\curveto(340.38085304,440.55955948)(339.5214789,440.19432547)(338.91992188,439.46386692)
\curveto(338.32551655,438.74055088)(338.02831633,437.69855974)(338.02832031,436.33789035)
\lineto(338.02832031,429.99999973)
\lineto(336.04101562,429.99999973)
\lineto(336.04101562,442.03124973)
\lineto(338.02832031,442.03124973)
\lineto(338.02832031,440.1621091)
\curveto(338.4436805,440.89256696)(338.98437006,441.43325652)(339.65039062,441.78417942)
\curveto(340.31639998,442.14224019)(341.12564396,442.32127647)(342.078125,442.32128879)
\curveto(342.21418454,442.32127647)(342.36457501,442.31053429)(342.52929688,442.28906223)
\curveto(342.69400177,442.27472704)(342.87661877,442.24966196)(343.07714844,442.21386692)
\lineto(343.08789062,440.18359348)
}
}
{
\newrgbcolor{curcolor}{0 0 0}
\pscustom[linestyle=none,fillstyle=solid,fillcolor=curcolor]
{
\newpath
\moveto(345.18261719,442.03124973)
\lineto(347.15917969,442.03124973)
\lineto(347.15917969,429.99999973)
\lineto(345.18261719,429.99999973)
\lineto(345.18261719,442.03124973)
\moveto(345.18261719,446.71484348)
\lineto(347.15917969,446.71484348)
\lineto(347.15917969,444.21191379)
\lineto(345.18261719,444.21191379)
\lineto(345.18261719,446.71484348)
}
}
{
\newrgbcolor{curcolor}{0 0 0}
\pscustom[linestyle=none,fillstyle=solid,fillcolor=curcolor]
{
\newpath
\moveto(353.19628906,431.80468723)
\lineto(353.19628906,425.42382785)
\lineto(351.20898438,425.42382785)
\lineto(351.20898438,442.03124973)
\lineto(353.19628906,442.03124973)
\lineto(353.19628906,440.20507785)
\curveto(353.61164925,440.92121277)(354.13443518,441.45116015)(354.76464844,441.7949216)
\curveto(355.40201204,442.14582092)(356.16112586,442.32127647)(357.04199219,442.32128879)
\curveto(358.5029204,442.32127647)(359.68814056,441.74119892)(360.59765625,440.58105442)
\curveto(361.51431061,439.42088874)(361.97264349,437.89549965)(361.97265625,436.00488254)
\curveto(361.97264349,434.11425343)(361.51431061,432.58886433)(360.59765625,431.42871067)
\curveto(359.68814056,430.26855415)(358.5029204,429.6884766)(357.04199219,429.68847629)
\curveto(356.16112586,429.6884766)(355.40201204,429.86035143)(354.76464844,430.20410129)
\curveto(354.13443518,430.55501219)(353.61164925,431.08854031)(353.19628906,431.80468723)
\moveto(359.92089844,436.00488254)
\curveto(359.92088773,437.45865112)(359.62010678,438.59732186)(359.01855469,439.42089817)
\curveto(358.42414443,440.25161708)(357.60415827,440.66698125)(356.55859375,440.66699192)
\curveto(355.51301453,440.66698125)(354.68944765,440.25161708)(354.08789062,439.42089817)
\curveto(353.4934853,438.59732186)(353.19628508,437.45865112)(353.19628906,436.00488254)
\curveto(353.19628508,434.55110195)(353.4934853,433.40885049)(354.08789062,432.57812473)
\curveto(354.68944765,431.75455527)(355.51301453,431.34277182)(356.55859375,431.34277317)
\curveto(357.60415827,431.34277182)(358.42414443,431.75455527)(359.01855469,432.57812473)
\curveto(359.62010678,433.40885049)(359.92088773,434.55110195)(359.92089844,436.00488254)
}
}
{
\newrgbcolor{curcolor}{0 0 0}
\pscustom[linestyle=none,fillstyle=solid,fillcolor=curcolor]
{
\newpath
\moveto(367.20410156,445.44726535)
\lineto(367.20410156,442.03124973)
\lineto(371.27539062,442.03124973)
\lineto(371.27539062,440.49511692)
\lineto(367.20410156,440.49511692)
\lineto(367.20410156,433.96386692)
\curveto(367.20409753,432.98274414)(367.33658438,432.35253644)(367.6015625,432.07324192)
\curveto(367.87369322,431.79394325)(368.42154423,431.65429495)(369.24511719,431.6542966)
\lineto(371.27539062,431.6542966)
\lineto(371.27539062,429.99999973)
\lineto(369.24511719,429.99999973)
\curveto(367.71972202,429.99999973)(366.6669887,430.28287705)(366.08691406,430.84863254)
\curveto(365.50683361,431.42154779)(365.21679483,432.45995821)(365.21679688,433.96386692)
\lineto(365.21679688,440.49511692)
\lineto(363.76660156,440.49511692)
\lineto(363.76660156,442.03124973)
\lineto(365.21679688,442.03124973)
\lineto(365.21679688,445.44726535)
\lineto(367.20410156,445.44726535)
}
}
{
\newrgbcolor{curcolor}{0 0 0}
\pscustom[linestyle=none,fillstyle=solid,fillcolor=curcolor]
{
\newpath
\moveto(373.88574219,442.03124973)
\lineto(375.86230469,442.03124973)
\lineto(375.86230469,429.99999973)
\lineto(373.88574219,429.99999973)
\lineto(373.88574219,442.03124973)
\moveto(373.88574219,446.71484348)
\lineto(375.86230469,446.71484348)
\lineto(375.86230469,444.21191379)
\lineto(373.88574219,444.21191379)
\lineto(373.88574219,446.71484348)
}
}
{
\newrgbcolor{curcolor}{0 0 0}
\pscustom[linestyle=none,fillstyle=solid,fillcolor=curcolor]
{
\newpath
\moveto(384.64941406,440.64550754)
\curveto(383.58951255,440.6454969)(382.75162277,440.23013273)(382.13574219,439.39941379)
\curveto(381.51985317,438.57583751)(381.21191076,437.44432822)(381.21191406,436.00488254)
\curveto(381.21191076,434.56542485)(381.51627244,433.43033484)(382.125,432.5996091)
\curveto(382.74088059,431.77603962)(383.5823511,431.36425618)(384.64941406,431.36425754)
\curveto(385.70214065,431.36425618)(386.53644971,431.77962034)(387.15234375,432.61035129)
\curveto(387.76821931,433.44107702)(388.07616171,434.5725863)(388.07617188,436.00488254)
\curveto(388.07616171,437.43000532)(387.76821931,438.55793388)(387.15234375,439.3886716)
\curveto(386.53644971,440.226552)(385.70214065,440.6454969)(384.64941406,440.64550754)
\moveto(384.64941406,442.32128879)
\curveto(386.36815561,442.32127647)(387.71808915,441.76268328)(388.69921875,440.64550754)
\curveto(389.68032678,439.52831051)(390.17088618,437.98143706)(390.17089844,436.00488254)
\curveto(390.17088618,434.03547746)(389.68032678,432.48860401)(388.69921875,431.36425754)
\curveto(387.71808915,430.24706979)(386.36815561,429.6884766)(384.64941406,429.68847629)
\curveto(382.92349759,429.6884766)(381.56998332,430.24706979)(380.58886719,431.36425754)
\curveto(379.61490715,432.48860401)(379.12792847,434.03547746)(379.12792969,436.00488254)
\curveto(379.12792847,437.98143706)(379.61490715,439.52831051)(380.58886719,440.64550754)
\curveto(381.56998332,441.76268328)(382.92349759,442.32127647)(384.64941406,442.32128879)
}
}
{
\newrgbcolor{curcolor}{0 0 0}
\pscustom[linestyle=none,fillstyle=solid,fillcolor=curcolor]
{
\newpath
\moveto(403.4375,437.26171848)
\lineto(403.4375,429.99999973)
\lineto(401.4609375,429.99999973)
\lineto(401.4609375,437.19726535)
\curveto(401.4609274,438.33592889)(401.23892242,439.18814158)(400.79492188,439.75390598)
\curveto(400.35090247,440.31965087)(399.68488751,440.60252819)(398.796875,440.60253879)
\curveto(397.72981134,440.60252819)(396.88834083,440.26235926)(396.27246094,439.58203098)
\curveto(395.65657123,438.90168354)(395.34862883,437.97427561)(395.34863281,436.79980442)
\lineto(395.34863281,429.99999973)
\lineto(393.36132812,429.99999973)
\lineto(393.36132812,442.03124973)
\lineto(395.34863281,442.03124973)
\lineto(395.34863281,440.1621091)
\curveto(395.8212846,440.88540551)(396.37629707,441.42609507)(397.01367188,441.78417942)
\curveto(397.65819683,442.14224019)(398.39940703,442.32127647)(399.23730469,442.32128879)
\curveto(400.61945689,442.32127647)(401.66502876,441.8915894)(402.37402344,441.03222629)
\curveto(403.08299609,440.18000257)(403.43748793,438.92316789)(403.4375,437.26171848)
}
}
{
\newrgbcolor{curcolor}{1 1 1}
\pscustom[linestyle=none,fillstyle=solid,fillcolor=curcolor]
{
\newpath
\moveto(149.97337341,399.69833919)
\lineto(410.03535461,399.69833919)
\lineto(410.03535461,360.11021587)
\lineto(149.97337341,360.11021587)
\closepath
}
}
{
\newrgbcolor{curcolor}{1 0 0}
\pscustom[linewidth=2,linecolor=curcolor,linestyle=dashed,dash=8 8]
{
\newpath
\moveto(149.97337341,399.69833919)
\lineto(410.03535461,399.69833919)
\lineto(410.03535461,360.11021587)
\lineto(149.97337341,360.11021587)
\closepath
}
}
{
\newrgbcolor{curcolor}{0 0 0}
\pscustom[linestyle=none,fillstyle=solid,fillcolor=curcolor]
{
\newpath
\moveto(43.96875,542.65625164)
\lineto(49.125,542.65625164)
\lineto(49.125,560.45312664)
\lineto(43.515625,559.32812664)
\lineto(43.515625,562.20312664)
\lineto(49.09375,563.32812664)
\lineto(52.25,563.32812664)
\lineto(52.25,542.65625164)
\lineto(57.40625,542.65625164)
\lineto(57.40625,540.00000164)
\lineto(43.96875,540.00000164)
\lineto(43.96875,542.65625164)
}
}
{
\newrgbcolor{curcolor}{0 0 0}
\pscustom[linestyle=none,fillstyle=solid,fillcolor=curcolor]
{
\newpath
\moveto(46.140625,462.65625021)
\lineto(57.15625,462.65625021)
\lineto(57.15625,460.00000021)
\lineto(42.34375,460.00000021)
\lineto(42.34375,462.65625021)
\curveto(43.54166312,463.89582964)(45.17186983,465.55728631)(47.234375,467.64062521)
\curveto(49.30728236,469.73436547)(50.60936439,471.08332246)(51.140625,471.68750021)
\curveto(52.15102952,472.82290405)(52.85415381,473.78123642)(53.25,474.56250021)
\curveto(53.65623634,475.35415152)(53.85936114,476.13019241)(53.859375,476.89062521)
\curveto(53.85936114,478.13019041)(53.42186158,479.14060606)(52.546875,479.92187521)
\curveto(51.68227998,480.7031045)(50.55207278,481.09372911)(49.15625,481.09375021)
\curveto(48.1666585,481.09372911)(47.11978455,480.92185428)(46.015625,480.57812521)
\curveto(44.92187008,480.23435497)(43.74999625,479.71352216)(42.5,479.01562521)
\lineto(42.5,482.20312521)
\curveto(43.77082956,482.71351916)(44.95832838,483.09893544)(46.0625,483.35937521)
\curveto(47.1666595,483.61976825)(48.17707516,483.74997646)(49.09375,483.75000021)
\curveto(51.51040516,483.74997646)(53.43748656,483.14581039)(54.875,481.93750021)
\curveto(56.31248369,480.72914614)(57.03123297,479.11456442)(57.03125,477.09375021)
\curveto(57.03123297,476.13540074)(56.84894148,475.22394331)(56.484375,474.35937521)
\curveto(56.1301922,473.50519503)(55.47915119,472.49477938)(54.53125,471.32812521)
\curveto(54.27081906,471.02603085)(53.44269489,470.15103172)(52.046875,468.70312521)
\curveto(50.65103102,467.26561794)(48.68228298,465.24999496)(46.140625,462.65625021)
}
}
{
\newrgbcolor{curcolor}{0 0 0}
\pscustom[linestyle=none,fillstyle=solid,fillcolor=curcolor]
{
\newpath
\moveto(52.09375,390.57812664)
\lineto(44.125,378.12500164)
\lineto(52.09375,378.12500164)
\lineto(52.09375,390.57812664)
\moveto(51.265625,393.32812664)
\lineto(55.234375,393.32812664)
\lineto(55.234375,378.12500164)
\lineto(58.5625,378.12500164)
\lineto(58.5625,375.50000164)
\lineto(55.234375,375.50000164)
\lineto(55.234375,370.00000164)
\lineto(52.09375,370.00000164)
\lineto(52.09375,375.50000164)
\lineto(41.5625,375.50000164)
\lineto(41.5625,378.54687664)
\lineto(51.265625,393.32812664)
}
}
{
\newrgbcolor{curcolor}{0 0 0}
\pscustom[linestyle=none,fillstyle=solid,fillcolor=curcolor]
{
\newpath
\moveto(503.453125,333.32812664)
\lineto(515.84375,333.32812664)
\lineto(515.84375,330.67187664)
\lineto(506.34375,330.67187664)
\lineto(506.34375,324.95312664)
\curveto(506.80207653,325.10936153)(507.26040941,325.22394475)(507.71875,325.29687664)
\curveto(508.17707516,325.38019459)(508.63540803,325.42186121)(509.09375,325.42187664)
\curveto(511.69790497,325.42186121)(513.76040291,324.70832026)(515.28125,323.28125164)
\curveto(516.80206653,321.85415645)(517.56248244,319.92186671)(517.5625,317.48437664)
\curveto(517.56248244,314.973955)(516.78123322,313.02083195)(515.21875,311.62500164)
\curveto(513.65623634,310.23958473)(511.45311355,309.54687709)(508.609375,309.54687664)
\curveto(507.6302007,309.54687709)(506.6302017,309.63021034)(505.609375,309.79687664)
\curveto(504.59895373,309.96354334)(503.55207978,310.21354309)(502.46875,310.54687664)
\lineto(502.46875,313.71875164)
\curveto(503.40624659,313.20833176)(504.37499562,312.82812381)(505.375,312.57812664)
\curveto(506.37499363,312.32812431)(507.43228423,312.20312443)(508.546875,312.20312664)
\curveto(510.34894798,312.20312443)(511.77602989,312.67708229)(512.828125,313.62500164)
\curveto(513.88019445,314.57291373)(514.40623559,315.85937078)(514.40625,317.48437664)
\curveto(514.40623559,319.10936753)(513.88019445,320.39582457)(512.828125,321.34375164)
\curveto(511.77602989,322.29165601)(510.34894798,322.76561387)(508.546875,322.76562664)
\curveto(507.7031173,322.76561387)(506.85936814,322.67186396)(506.015625,322.48437664)
\curveto(505.18228648,322.29686434)(504.32812067,322.00519796)(503.453125,321.60937664)
\lineto(503.453125,333.32812664)
}
}
{
\newrgbcolor{curcolor}{0 0 0}
\pscustom[linestyle=none,fillstyle=solid,fillcolor=curcolor]
{
\newpath
\moveto(510.5625,272.92187664)
\curveto(509.14582419,272.92186371)(508.02082531,272.4374892)(507.1875,271.46875164)
\curveto(506.36457697,270.49999114)(505.95311905,269.17186746)(505.953125,267.48437664)
\curveto(505.95311905,265.8072875)(506.36457697,264.47916382)(507.1875,263.50000164)
\curveto(508.02082531,262.5312491)(509.14582419,262.04687459)(510.5625,262.04687664)
\curveto(511.97915469,262.04687459)(513.09894523,262.5312491)(513.921875,263.50000164)
\curveto(514.75519358,264.47916382)(515.17185983,265.8072875)(515.171875,267.48437664)
\curveto(515.17185983,269.17186746)(514.75519358,270.49999114)(513.921875,271.46875164)
\curveto(513.09894523,272.4374892)(511.97915469,272.92186371)(510.5625,272.92187664)
\moveto(516.828125,282.81250164)
\lineto(516.828125,279.93750164)
\curveto(516.0364423,280.31248132)(515.23435977,280.59893937)(514.421875,280.79687664)
\curveto(513.61977805,280.99477231)(512.82290384,281.09373054)(512.03125,281.09375164)
\curveto(509.94790672,281.09373054)(508.35415831,280.39060625)(507.25,278.98437664)
\curveto(506.15624384,277.57810906)(505.53124447,275.45311118)(505.375,272.60937664)
\curveto(505.98957734,273.51561312)(506.76040991,274.20832076)(507.6875,274.68750164)
\curveto(508.61457472,275.17706979)(509.63540703,275.42186121)(510.75,275.42187664)
\curveto(513.09373691,275.42186121)(514.94269339,274.70832026)(516.296875,273.28125164)
\curveto(517.66144067,271.8645731)(518.34373166,269.93228337)(518.34375,267.48437664)
\curveto(518.34373166,265.08853821)(517.63539903,263.16666514)(516.21875,261.71875164)
\curveto(514.80206853,260.2708347)(512.91665375,259.54687709)(510.5625,259.54687664)
\curveto(507.86457547,259.54687709)(505.80207753,260.57812606)(504.375,262.64062664)
\curveto(502.94791372,264.71353859)(502.23437277,267.71353559)(502.234375,271.64062664)
\curveto(502.23437277,275.32811131)(503.10937189,278.26560837)(504.859375,280.45312664)
\curveto(506.60936839,282.65102065)(508.95832438,283.74997789)(511.90625,283.75000164)
\curveto(512.69790397,283.74997789)(513.49477817,283.67185296)(514.296875,283.51562664)
\curveto(515.10935989,283.35935328)(515.95310905,283.12497851)(516.828125,282.81250164)
}
}
{
\newrgbcolor{curcolor}{0 0 0}
\pscustom[linestyle=none,fillstyle=solid,fillcolor=curcolor]
{
\newpath
\moveto(512.984375,442.57812473)
\curveto(514.49477717,442.25519581)(515.67185933,441.58332148)(516.515625,440.56249973)
\curveto(517.3697743,439.54165685)(517.7968572,438.28124145)(517.796875,436.78124973)
\curveto(517.7968572,434.47916192)(517.00519133,432.6979137)(515.421875,431.43749973)
\curveto(513.83852783,430.17708288)(511.58853008,429.54687518)(508.671875,429.54687473)
\curveto(507.69270064,429.54687518)(506.68228498,429.64583342)(505.640625,429.84374973)
\curveto(504.60937039,430.0312497)(503.54166313,430.31770774)(502.4375,430.70312473)
\lineto(502.4375,433.74999973)
\curveto(503.31249669,433.23957982)(504.27082906,432.85416354)(505.3125,432.59374973)
\curveto(506.35416031,432.33333073)(507.44270089,432.20312253)(508.578125,432.20312473)
\curveto(510.55728111,432.20312253)(512.06248794,432.59374713)(513.09375,433.37499973)
\curveto(514.13540253,434.15624557)(514.65623534,435.2916611)(514.65625,436.78124973)
\curveto(514.65623534,438.15624157)(514.17186083,439.22915717)(513.203125,439.99999973)
\curveto(512.24477942,440.78123895)(510.90623909,441.17186356)(509.1875,441.17187473)
\lineto(506.46875,441.17187473)
\lineto(506.46875,443.76562473)
\lineto(509.3125,443.76562473)
\curveto(510.86457247,443.76561096)(512.05207128,444.07290232)(512.875,444.68749973)
\curveto(513.69790297,445.31248442)(514.10936089,446.20831685)(514.109375,447.37499973)
\curveto(514.10936089,448.57289782)(513.68227798,449.48956357)(512.828125,450.12499973)
\curveto(511.98436302,450.77081229)(510.77082256,451.09372863)(509.1875,451.09374973)
\curveto(508.32290834,451.09372863)(507.39582594,450.99997873)(506.40625,450.81249973)
\curveto(505.41666125,450.6249791)(504.32812067,450.33331273)(503.140625,449.93749973)
\lineto(503.140625,452.74999973)
\curveto(504.33853733,453.08330998)(505.45832788,453.33330973)(506.5,453.49999973)
\curveto(507.55207578,453.66664273)(508.54165813,453.74997598)(509.46875,453.74999973)
\curveto(511.86457147,453.74997598)(513.76040291,453.20310153)(515.15625,452.10937473)
\curveto(516.55206678,451.02602037)(517.24998275,449.55727184)(517.25,447.70312473)
\curveto(517.24998275,446.41144165)(516.88019145,445.31769274)(516.140625,444.42187473)
\curveto(515.40102627,443.53644453)(514.34894398,442.92186181)(512.984375,442.57812473)
}
}
{
\newrgbcolor{curcolor}{0 0 0}
\pscustom[linestyle=none,fillstyle=solid,fillcolor=curcolor]
{
\newpath
\moveto(502.625,233.32812664)
\lineto(517.625,233.32812664)
\lineto(517.625,231.98437664)
\lineto(509.15625,210.00000164)
\lineto(505.859375,210.00000164)
\lineto(513.828125,230.67187664)
\lineto(502.625,230.67187664)
\lineto(502.625,233.32812664)
}
}
{
\newrgbcolor{curcolor}{0 0 0}
\pscustom[linestyle=none,fillstyle=solid,fillcolor=curcolor]
{
\newpath
\moveto(510.171875,181.07811138)
\curveto(508.67186633,181.0781003)(507.48957584,180.67705903)(506.625,179.87498638)
\curveto(505.77082756,179.07289397)(505.34374466,177.96872841)(505.34375,176.56248638)
\curveto(505.34374466,175.15623122)(505.77082756,174.05206566)(506.625,173.24998638)
\curveto(507.48957584,172.4479006)(508.67186633,172.04685933)(510.171875,172.04686138)
\curveto(511.67186333,172.04685933)(512.85415381,172.4479006)(513.71875,173.24998638)
\curveto(514.58331875,174.06248231)(515.01560998,175.16664788)(515.015625,176.56248638)
\curveto(515.01560998,177.96872841)(514.58331875,179.07289397)(513.71875,179.87498638)
\curveto(512.86457047,180.67705903)(511.68227998,181.0781003)(510.171875,181.07811138)
\moveto(507.015625,182.42186138)
\curveto(505.66145267,182.75518196)(504.60416206,183.38538966)(503.84375,184.31248638)
\curveto(503.09374691,185.23955447)(502.71874728,186.36976167)(502.71875,187.70311138)
\curveto(502.71874728,189.56767514)(503.38020495,191.041632)(504.703125,192.12498638)
\curveto(506.0364523,193.2082965)(507.85936714,193.74996263)(510.171875,193.74998638)
\curveto(512.49477917,193.74996263)(514.31769402,193.2082965)(515.640625,192.12498638)
\curveto(516.9635247,191.041632)(517.62498237,189.56767514)(517.625,187.70311138)
\curveto(517.62498237,186.36976167)(517.24477442,185.23955447)(516.484375,184.31248638)
\curveto(515.73435927,183.38538966)(514.68748531,182.75518196)(513.34375,182.42186138)
\curveto(514.86456847,182.06768264)(516.04685895,181.374975)(516.890625,180.34373638)
\curveto(517.74477392,179.31247706)(518.17185683,178.05206166)(518.171875,176.56248638)
\curveto(518.17185683,174.30206541)(517.47914919,172.56769214)(516.09375,171.35936138)
\curveto(514.71873528,170.15102789)(512.74477892,169.54686183)(510.171875,169.54686138)
\curveto(507.59895073,169.54686183)(505.61978605,170.15102789)(504.234375,171.35936138)
\curveto(502.85937214,172.56769214)(502.17187283,174.30206541)(502.171875,176.56248638)
\curveto(502.17187283,178.05206166)(502.59895573,179.31247706)(503.453125,180.34373638)
\curveto(504.30728736,181.374975)(505.49478617,182.06768264)(507.015625,182.42186138)
\moveto(505.859375,187.40623638)
\curveto(505.85936914,186.19788685)(506.23436877,185.25517946)(506.984375,184.57811138)
\curveto(507.74478392,183.90101414)(508.80728286,183.56247281)(510.171875,183.56248638)
\curveto(511.52603014,183.56247281)(512.58332075,183.90101414)(513.34375,184.57811138)
\curveto(514.11456922,185.25517946)(514.4999855,186.19788685)(514.5,187.40623638)
\curveto(514.4999855,188.6145511)(514.11456922,189.55725849)(513.34375,190.23436138)
\curveto(512.58332075,190.9114238)(511.52603014,191.24996513)(510.171875,191.24998638)
\curveto(508.80728286,191.24996513)(507.74478392,190.9114238)(506.984375,190.23436138)
\curveto(506.23436877,189.55725849)(505.85936914,188.6145511)(505.859375,187.40623638)
}
}
{
\newrgbcolor{curcolor}{0 0 0}
\pscustom[linewidth=2,linecolor=curcolor,linestyle=dashed,dash=8 8]
{
\newpath
\moveto(150,550)
\lineto(60,550)
}
}
{
\newrgbcolor{curcolor}{0 0 0}
\pscustom[linestyle=none,fillstyle=solid,fillcolor=curcolor]
{
\newpath
\moveto(139.53769464,554.84048224)
\lineto(152.6487474,550.01921591)
\lineto(139.53769392,545.19795064)
\curveto(141.632292,548.04442372)(141.62022288,551.93889292)(139.53769464,554.84048224)
\lineto(139.53769464,554.84048224)
\lineto(139.53769464,554.84048224)
\closepath
}
}
{
\newrgbcolor{curcolor}{0 0 0}
\pscustom[linewidth=2,linecolor=curcolor,linestyle=dashed,dash=8 8]
{
\newpath
\moveto(150,469.999997)
\lineto(60,469.999997)
}
}
{
\newrgbcolor{curcolor}{0 0 0}
\pscustom[linestyle=none,fillstyle=solid,fillcolor=curcolor]
{
\newpath
\moveto(139.53769464,474.84047924)
\lineto(152.6487474,470.01921291)
\lineto(139.53769392,465.19794764)
\curveto(141.632292,468.04442072)(141.62022288,471.93888992)(139.53769464,474.84047924)
\lineto(139.53769464,474.84047924)
\lineto(139.53769464,474.84047924)
\closepath
}
}
{
\newrgbcolor{curcolor}{0 0 0}
\pscustom[linewidth=2,linecolor=curcolor,linestyle=dashed,dash=8 8]
{
\newpath
\moveto(150,130)
\lineto(60,130)
}
}
{
\newrgbcolor{curcolor}{0 0 0}
\pscustom[linestyle=none,fillstyle=solid,fillcolor=curcolor]
{
\newpath
\moveto(139.53769464,134.84048224)
\lineto(152.6487474,130.01921591)
\lineto(139.53769392,125.19795064)
\curveto(141.632292,128.04442372)(141.62022288,131.93889292)(139.53769464,134.84048224)
\lineto(139.53769464,134.84048224)
\lineto(139.53769464,134.84048224)
\closepath
}
}
{
\newrgbcolor{curcolor}{0 0 0}
\pscustom[linewidth=2,linecolor=curcolor,linestyle=dashed,dash=8 8]
{
\newpath
\moveto(409.9375,439.999996)
\lineto(499.9375,439.999996)
}
}
{
\newrgbcolor{curcolor}{0 0 0}
\pscustom[linestyle=none,fillstyle=solid,fillcolor=curcolor]
{
\newpath
\moveto(420.39980536,435.15951376)
\lineto(407.2887526,439.98078009)
\lineto(420.39980608,444.80204536)
\curveto(418.305208,441.95557228)(418.31727712,438.06110308)(420.39980536,435.15951376)
\lineto(420.39980536,435.15951376)
\lineto(420.39980536,435.15951376)
\closepath
}
}
{
\newrgbcolor{curcolor}{0 0 0}
\pscustom[linewidth=2,linecolor=curcolor,linestyle=dashed,dash=8 8]
{
\newpath
\moveto(60,380)
\lineto(180,380)
}
}
{
\newrgbcolor{curcolor}{0 0 0}
\pscustom[linestyle=none,fillstyle=solid,fillcolor=curcolor]
{
\newpath
\moveto(169.53769464,384.84048224)
\lineto(182.6487474,380.01921591)
\lineto(169.53769392,375.19795064)
\curveto(171.632292,378.04442372)(171.62022288,381.93889292)(169.53769464,384.84048224)
\closepath
}
}
{
\newrgbcolor{curcolor}{0 0 0}
\pscustom[linewidth=2,linecolor=curcolor,linestyle=dashed,dash=8 8]
{
\newpath
\moveto(400,320)
\lineto(500,320)
}
}
{
\newrgbcolor{curcolor}{0 0 0}
\pscustom[linestyle=none,fillstyle=solid,fillcolor=curcolor]
{
\newpath
\moveto(410.46230536,315.15951776)
\lineto(397.3512526,319.98078409)
\lineto(410.46230608,324.80204936)
\curveto(408.367708,321.95557628)(408.37977712,318.06110708)(410.46230536,315.15951776)
\lineto(410.46230536,315.15951776)
\lineto(410.46230536,315.15951776)
\closepath
}
}
{
\newrgbcolor{curcolor}{0 0 0}
\pscustom[linewidth=2,linecolor=curcolor,linestyle=dashed,dash=8 8]
{
\newpath
\moveto(400.04545,270)
\lineto(500.04545,270)
}
}
{
\newrgbcolor{curcolor}{0 0 0}
\pscustom[linestyle=none,fillstyle=solid,fillcolor=curcolor]
{
\newpath
\moveto(410.50775536,265.15951776)
\lineto(397.3967026,269.98078409)
\lineto(410.50775608,274.80204936)
\curveto(408.413158,271.95557628)(408.42522712,268.06110708)(410.50775536,265.15951776)
\lineto(410.50775536,265.15951776)
\lineto(410.50775536,265.15951776)
\closepath
}
}
{
\newrgbcolor{curcolor}{0 0 0}
\pscustom[linewidth=2,linecolor=curcolor,linestyle=dashed,dash=8 8]
{
\newpath
\moveto(410,130)
\lineto(500,130)
}
}
{
\newrgbcolor{curcolor}{0 0 0}
\pscustom[linestyle=none,fillstyle=solid,fillcolor=curcolor]
{
\newpath
\moveto(420.46230536,125.15951776)
\lineto(407.3512526,129.98078409)
\lineto(420.46230608,134.80204936)
\curveto(418.367708,131.95557628)(418.37977712,128.06110708)(420.46230536,125.15951776)
\lineto(420.46230536,125.15951776)
\lineto(420.46230536,125.15951776)
\closepath
}
}
{
\newrgbcolor{curcolor}{1 1 1}
\pscustom[linestyle=none,fillstyle=solid,fillcolor=curcolor]
{
\newpath
\moveto(320,230)
\lineto(340,230)
\lineto(340,210)
\lineto(320,210)
\closepath
}
}
{
\newrgbcolor{curcolor}{0 0 0}
\pscustom[linewidth=2,linecolor=curcolor]
{
\newpath
\moveto(320,230)
\lineto(340,230)
\lineto(340,210)
\lineto(320,210)
\closepath
}
}
{
\newrgbcolor{curcolor}{0 0 0}
\pscustom[linewidth=2,linecolor=curcolor,linestyle=dashed,dash=8 8]
{
\newpath
\moveto(330,220)
\lineto(500,220)
}
}
{
\newrgbcolor{curcolor}{0 0 0}
\pscustom[linestyle=none,fillstyle=solid,fillcolor=curcolor]
{
\newpath
\moveto(340.46230536,215.15951776)
\lineto(327.3512526,219.98078409)
\lineto(340.46230608,224.80204936)
\curveto(338.367708,221.95557628)(338.37977712,218.06110708)(340.46230536,215.15951776)
\lineto(340.46230536,215.15951776)
\lineto(340.46230536,215.15951776)
\closepath
}
}
{
\newrgbcolor{curcolor}{0 0 0}
\pscustom[linestyle=none,fillstyle=solid,fillcolor=curcolor]
{
\newpath
\moveto(43.515625,120.48436138)
\lineto(43.515625,123.35936138)
\curveto(44.30728736,122.98435839)(45.10936989,122.69790035)(45.921875,122.49998638)
\curveto(46.73436827,122.30206741)(47.53124247,122.20310917)(48.3125,122.20311138)
\curveto(50.39582294,122.20310917)(51.98436302,122.90102514)(53.078125,124.29686138)
\curveto(54.18227748,125.70310567)(54.81248519,127.83331188)(54.96875,130.68748638)
\curveto(54.36456897,129.79164325)(53.59894473,129.10414394)(52.671875,128.62498638)
\curveto(51.74477992,128.14581156)(50.71873928,127.90622847)(49.59375,127.90623638)
\curveto(47.26040941,127.90622847)(45.41145292,128.60935277)(44.046875,130.01561138)
\curveto(42.69270564,131.43226661)(42.01562298,133.36455635)(42.015625,135.81248638)
\curveto(42.01562298,138.2083015)(42.72395561,140.13017458)(44.140625,141.57811138)
\curveto(45.55728611,143.02600502)(47.44270089,143.74996263)(49.796875,143.74998638)
\curveto(52.49477917,143.74996263)(54.55206878,142.71350533)(55.96875,140.64061138)
\curveto(57.39581594,138.5780928)(58.10935689,135.5780958)(58.109375,131.64061138)
\curveto(58.10935689,127.96352008)(57.23435777,125.02602302)(55.484375,122.82811138)
\curveto(53.74477792,120.64061074)(51.40103027,119.54686183)(48.453125,119.54686138)
\curveto(47.66145067,119.54686183)(46.85936814,119.62498675)(46.046875,119.78123638)
\curveto(45.23436977,119.93748644)(44.39062061,120.17186121)(43.515625,120.48436138)
\moveto(49.796875,130.37498638)
\curveto(51.21353045,130.374976)(52.333321,130.85935052)(53.15625,131.82811138)
\curveto(53.98956934,132.79684858)(54.40623559,134.12497225)(54.40625,135.81248638)
\curveto(54.40623559,137.48955222)(53.98956934,138.81246756)(53.15625,139.78123638)
\curveto(52.333321,140.76038228)(51.21353045,141.24996513)(49.796875,141.24998638)
\curveto(48.38019995,141.24996513)(47.25520108,140.76038228)(46.421875,139.78123638)
\curveto(45.59895273,138.81246756)(45.18749481,137.48955222)(45.1875,135.81248638)
\curveto(45.18749481,134.12497225)(45.59895273,132.79684858)(46.421875,131.82811138)
\curveto(47.25520108,130.85935052)(48.38019995,130.374976)(49.796875,130.37498638)
}
}
{
\newrgbcolor{curcolor}{1 1 1}
\pscustom[linestyle=none,fillstyle=solid,fillcolor=curcolor]
{
\newpath
\moveto(410,189.99998638)
\lineto(430,189.99998638)
\lineto(430,169.99998638)
\lineto(410,169.99998638)
\closepath
}
}
{
\newrgbcolor{curcolor}{0 0 0}
\pscustom[linewidth=2,linecolor=curcolor]
{
\newpath
\moveto(410,189.99998638)
\lineto(430,189.99998638)
\lineto(430,169.99998638)
\lineto(410,169.99998638)
\closepath
}
}
{
\newrgbcolor{curcolor}{0 0 0}
\pscustom[linewidth=2,linecolor=curcolor,linestyle=dashed,dash=8 8]
{
\newpath
\moveto(420,180)
\lineto(500,180)
}
}
{
\newrgbcolor{curcolor}{0 0 0}
\pscustom[linestyle=none,fillstyle=solid,fillcolor=curcolor]
{
\newpath
\moveto(430.46230536,175.15951776)
\lineto(417.3512526,179.98078409)
\lineto(430.46230608,184.80204936)
\curveto(428.367708,181.95557628)(428.37977712,178.06110708)(430.46230536,175.15951776)
\lineto(430.46230536,175.15951776)
\lineto(430.46230536,175.15951776)
\closepath
}
}
{
\newrgbcolor{curcolor}{0 0 0}
\pscustom[linestyle=none,fillstyle=solid,fillcolor=curcolor]
{
\newpath
\moveto(503.96875,122.65623638)
\lineto(509.125,122.65623638)
\lineto(509.125,140.45311138)
\lineto(503.515625,139.32811138)
\lineto(503.515625,142.20311138)
\lineto(509.09375,143.32811138)
\lineto(512.25,143.32811138)
\lineto(512.25,122.65623638)
\lineto(517.40625,122.65623638)
\lineto(517.40625,119.99998638)
\lineto(503.96875,119.99998638)
\lineto(503.96875,122.65623638)
}
}
{
\newrgbcolor{curcolor}{0 0 0}
\pscustom[linestyle=none,fillstyle=solid,fillcolor=curcolor]
{
\newpath
\moveto(530.546875,141.24998638)
\curveto(528.92186645,141.24996513)(527.69790934,140.4478826)(526.875,138.84373638)
\curveto(526.06249431,137.24996913)(525.65624472,134.84892986)(525.65625,131.64061138)
\curveto(525.65624472,128.44268627)(526.06249431,126.041647)(526.875,124.43748638)
\curveto(527.69790934,122.84373353)(528.92186645,122.04685933)(530.546875,122.04686138)
\curveto(532.18227986,122.04685933)(533.40623697,122.84373353)(534.21875,124.43748638)
\curveto(535.041652,126.041647)(535.45310992,128.44268627)(535.453125,131.64061138)
\curveto(535.45310992,134.84892986)(535.041652,137.24996913)(534.21875,138.84373638)
\curveto(533.40623697,140.4478826)(532.18227986,141.24996513)(530.546875,141.24998638)
\moveto(530.546875,143.74998638)
\curveto(533.16144555,143.74996263)(535.15623522,142.71350533)(536.53125,140.64061138)
\curveto(537.91664913,138.5780928)(538.60935677,135.5780958)(538.609375,131.64061138)
\curveto(538.60935677,127.71352033)(537.91664913,124.71352333)(536.53125,122.64061138)
\curveto(535.15623522,120.5781108)(533.16144555,119.54686183)(530.546875,119.54686138)
\curveto(527.93228411,119.54686183)(525.93228611,120.5781108)(524.546875,122.64061138)
\curveto(523.1718722,124.71352333)(522.48437289,127.71352033)(522.484375,131.64061138)
\curveto(522.48437289,135.5780958)(523.1718722,138.5780928)(524.546875,140.64061138)
\curveto(525.93228611,142.71350533)(527.93228411,143.74996263)(530.546875,143.74998638)
}
}
{
\newrgbcolor{curcolor}{0 0 0}
\pscustom[linestyle=none,fillstyle=solid,fillcolor=curcolor]
{
\newpath
\moveto(176.48535156,336.03808757)
\lineto(179.40722656,336.03808757)
\lineto(186.51855469,322.62109539)
\lineto(186.51855469,336.03808757)
\lineto(188.62402344,336.03808757)
\lineto(188.62402344,320.00000164)
\lineto(185.70214844,320.00000164)
\lineto(178.59082031,333.41699382)
\lineto(178.59082031,320.00000164)
\lineto(176.48535156,320.00000164)
\lineto(176.48535156,336.03808757)
}
}
{
\newrgbcolor{curcolor}{0 0 0}
\pscustom[linestyle=none,fillstyle=solid,fillcolor=curcolor]
{
\newpath
\moveto(197.51855469,330.64550945)
\curveto(196.45865318,330.6454988)(195.62076339,330.23013463)(195.00488281,329.3994157)
\curveto(194.38899379,328.57583941)(194.08105139,327.44433013)(194.08105469,326.00488445)
\curveto(194.08105139,324.56542676)(194.38541306,323.43033675)(194.99414062,322.59961101)
\curveto(195.61002121,321.77604153)(196.45149173,321.36425808)(197.51855469,321.36425945)
\curveto(198.57128127,321.36425808)(199.40559034,321.77962225)(200.02148438,322.6103532)
\curveto(200.63735994,323.44107892)(200.94530234,324.57258821)(200.9453125,326.00488445)
\curveto(200.94530234,327.43000723)(200.63735994,328.55793579)(200.02148438,329.38867351)
\curveto(199.40559034,330.22655391)(198.57128127,330.6454988)(197.51855469,330.64550945)
\moveto(197.51855469,332.3212907)
\curveto(199.23729623,332.32127838)(200.58722978,331.76268519)(201.56835938,330.64550945)
\curveto(202.5494674,329.52831242)(203.04002681,327.98143897)(203.04003906,326.00488445)
\curveto(203.04002681,324.03547937)(202.5494674,322.48860592)(201.56835938,321.36425945)
\curveto(200.58722978,320.2470717)(199.23729623,319.68847851)(197.51855469,319.6884782)
\curveto(195.79263822,319.68847851)(194.43912395,320.2470717)(193.45800781,321.36425945)
\curveto(192.48404778,322.48860592)(191.9970691,324.03547937)(191.99707031,326.00488445)
\curveto(191.9970691,327.98143897)(192.48404778,329.52831242)(193.45800781,330.64550945)
\curveto(194.43912395,331.76268519)(195.79263822,332.32127838)(197.51855469,332.3212907)
}
}
{
\newrgbcolor{curcolor}{0 0 0}
\pscustom[linestyle=none,fillstyle=solid,fillcolor=curcolor]
{
\newpath
\moveto(215.67285156,329.72168132)
\curveto(216.16698025,330.60969155)(216.75779997,331.26496433)(217.4453125,331.68750164)
\curveto(218.1327986,332.11001557)(218.94204258,332.32127838)(219.87304688,332.3212907)
\curveto(221.12628519,332.32127838)(222.0930811,331.88084913)(222.7734375,331.00000164)
\curveto(223.45375682,330.12629359)(223.79392575,328.88020109)(223.79394531,327.26172039)
\lineto(223.79394531,320.00000164)
\lineto(221.80664062,320.00000164)
\lineto(221.80664062,327.19726726)
\curveto(221.80662305,328.3502537)(221.60252169,329.20604712)(221.19433594,329.76465007)
\curveto(220.78611626,330.3232335)(220.16307001,330.6025301)(219.32519531,330.6025407)
\curveto(218.3010927,330.6025301)(217.49184872,330.26236117)(216.89746094,329.58203289)
\curveto(216.30304783,328.90168544)(216.0058476,327.97427752)(216.00585938,326.79980632)
\lineto(216.00585938,320.00000164)
\lineto(214.01855469,320.00000164)
\lineto(214.01855469,327.19726726)
\curveto(214.0185449,328.35741515)(213.81444354,329.21320857)(213.40625,329.76465007)
\curveto(212.99803811,330.3232335)(212.36783041,330.6025301)(211.515625,330.6025407)
\curveto(210.5058531,330.6025301)(209.70377057,330.25878044)(209.109375,329.5712907)
\curveto(208.51496968,328.89094327)(208.21776945,327.96711606)(208.21777344,326.79980632)
\lineto(208.21777344,320.00000164)
\lineto(206.23046875,320.00000164)
\lineto(206.23046875,332.03125164)
\lineto(208.21777344,332.03125164)
\lineto(208.21777344,330.16211101)
\curveto(208.66894088,330.89973032)(209.20963044,331.44400061)(209.83984375,331.79492351)
\curveto(210.47004585,332.14582282)(211.21841749,332.32127838)(212.08496094,332.3212907)
\curveto(212.95865013,332.32127838)(213.69986032,332.09927339)(214.30859375,331.65527507)
\curveto(214.92446847,331.21125345)(215.37922062,330.56672284)(215.67285156,329.72168132)
}
}
{
\newrgbcolor{curcolor}{0 0 0}
\pscustom[linestyle=none,fillstyle=solid,fillcolor=curcolor]
{
\newpath
\moveto(144.44140625,302.70507976)
\lineto(144.44140625,309.21484539)
\lineto(146.41796875,309.21484539)
\lineto(146.41796875,292.50000164)
\lineto(144.44140625,292.50000164)
\lineto(144.44140625,294.30468914)
\curveto(144.02603209,293.58854221)(143.49966543,293.0550141)(142.86230469,292.7041032)
\curveto(142.23208857,292.36035334)(141.47297475,292.18847851)(140.58496094,292.1884782)
\curveto(139.13118022,292.18847851)(137.94596005,292.76855605)(137.02929688,293.92871257)
\curveto(136.11979,295.08886623)(135.66503785,296.61425533)(135.66503906,298.50488445)
\curveto(135.66503785,300.39550155)(136.11979,301.92089065)(137.02929688,303.08105632)
\curveto(137.94596005,304.24120083)(139.13118022,304.82127838)(140.58496094,304.8212907)
\curveto(141.47297475,304.82127838)(142.23208857,304.64582282)(142.86230469,304.29492351)
\curveto(143.49966543,303.95116206)(144.02603209,303.42121467)(144.44140625,302.70507976)
\moveto(137.70605469,298.50488445)
\curveto(137.70605143,297.05110386)(138.00325166,295.90885239)(138.59765625,295.07812664)
\curveto(139.199214,294.25455717)(140.02278089,293.84277373)(141.06835938,293.84277507)
\curveto(142.11392463,293.84277373)(142.93749151,294.25455717)(143.5390625,295.07812664)
\curveto(144.14061531,295.90885239)(144.44139626,297.05110386)(144.44140625,298.50488445)
\curveto(144.44139626,299.95865303)(144.14061531,301.09732377)(143.5390625,301.92090007)
\curveto(142.93749151,302.75161899)(142.11392463,303.16698316)(141.06835938,303.16699382)
\curveto(140.02278089,303.16698316)(139.199214,302.75161899)(138.59765625,301.92090007)
\curveto(138.00325166,301.09732377)(137.70605143,299.95865303)(137.70605469,298.50488445)
}
}
{
\newrgbcolor{curcolor}{0 0 0}
\pscustom[linestyle=none,fillstyle=solid,fillcolor=curcolor]
{
\newpath
\moveto(152.35839844,308.53808757)
\lineto(152.35839844,302.57617351)
\lineto(150.53222656,302.57617351)
\lineto(150.53222656,308.53808757)
\lineto(152.35839844,308.53808757)
}
}
{
\newrgbcolor{curcolor}{0 0 0}
\pscustom[linestyle=none,fillstyle=solid,fillcolor=curcolor]
{
\newpath
\moveto(156.34375,297.24804851)
\lineto(156.34375,304.53125164)
\lineto(158.3203125,304.53125164)
\lineto(158.3203125,297.32324382)
\curveto(158.32030865,296.18456826)(158.54231364,295.32877485)(158.98632812,294.75586101)
\curveto(159.43033359,294.19010411)(160.09634854,293.90722679)(160.984375,293.9072282)
\curveto(162.05142471,293.90722679)(162.89289523,294.24739572)(163.50878906,294.92773601)
\curveto(164.13182628,295.60807144)(164.44334941,296.53547937)(164.44335938,297.70996257)
\lineto(164.44335938,304.53125164)
\lineto(166.41992188,304.53125164)
\lineto(166.41992188,292.50000164)
\lineto(164.44335938,292.50000164)
\lineto(164.44335938,294.34765789)
\curveto(163.96353218,293.61718802)(163.40493899,293.07291773)(162.76757812,292.71484539)
\curveto(162.13736213,292.36393406)(161.40331338,292.18847851)(160.56542969,292.1884782)
\curveto(159.18326352,292.18847851)(158.13411092,292.61816558)(157.41796875,293.4775407)
\curveto(156.70182069,294.33691386)(156.34374813,295.59374854)(156.34375,297.24804851)
\moveto(161.31738281,304.8212907)
\lineto(161.31738281,304.8212907)
}
}
{
\newrgbcolor{curcolor}{0 0 0}
\pscustom[linestyle=none,fillstyle=solid,fillcolor=curcolor]
{
\newpath
\moveto(172.46777344,307.94726726)
\lineto(172.46777344,304.53125164)
\lineto(176.5390625,304.53125164)
\lineto(176.5390625,302.99511882)
\lineto(172.46777344,302.99511882)
\lineto(172.46777344,296.46386882)
\curveto(172.46776941,295.48274605)(172.60025626,294.85253835)(172.86523438,294.57324382)
\curveto(173.13736509,294.29394515)(173.68521611,294.15429686)(174.50878906,294.15429851)
\lineto(176.5390625,294.15429851)
\lineto(176.5390625,292.50000164)
\lineto(174.50878906,292.50000164)
\curveto(172.98339389,292.50000164)(171.93066057,292.78287896)(171.35058594,293.34863445)
\curveto(170.77050548,293.92154969)(170.48046671,294.95996011)(170.48046875,296.46386882)
\lineto(170.48046875,302.99511882)
\lineto(169.03027344,302.99511882)
\lineto(169.03027344,304.53125164)
\lineto(170.48046875,304.53125164)
\lineto(170.48046875,307.94726726)
\lineto(172.46777344,307.94726726)
}
}
{
\newrgbcolor{curcolor}{0 0 0}
\pscustom[linestyle=none,fillstyle=solid,fillcolor=curcolor]
{
\newpath
\moveto(179.14941406,304.53125164)
\lineto(181.12597656,304.53125164)
\lineto(181.12597656,292.50000164)
\lineto(179.14941406,292.50000164)
\lineto(179.14941406,304.53125164)
\moveto(179.14941406,309.21484539)
\lineto(181.12597656,309.21484539)
\lineto(181.12597656,306.7119157)
\lineto(179.14941406,306.7119157)
\lineto(179.14941406,309.21484539)
}
}
{
\newrgbcolor{curcolor}{0 0 0}
\pscustom[linestyle=none,fillstyle=solid,fillcolor=curcolor]
{
\newpath
\moveto(185.25097656,309.21484539)
\lineto(187.22753906,309.21484539)
\lineto(187.22753906,292.50000164)
\lineto(185.25097656,292.50000164)
\lineto(185.25097656,309.21484539)
}
}
{
\newrgbcolor{curcolor}{0 0 0}
\pscustom[linestyle=none,fillstyle=solid,fillcolor=curcolor]
{
\newpath
\moveto(191.35253906,304.53125164)
\lineto(193.32910156,304.53125164)
\lineto(193.32910156,292.50000164)
\lineto(191.35253906,292.50000164)
\lineto(191.35253906,304.53125164)
\moveto(191.35253906,309.21484539)
\lineto(193.32910156,309.21484539)
\lineto(193.32910156,306.7119157)
\lineto(191.35253906,306.7119157)
\lineto(191.35253906,309.21484539)
}
}
{
\newrgbcolor{curcolor}{0 0 0}
\pscustom[linestyle=none,fillstyle=solid,fillcolor=curcolor]
{
\newpath
\moveto(205.12402344,304.17675945)
\lineto(205.12402344,302.30761882)
\curveto(204.5654205,302.59406706)(203.98534296,302.8089106)(203.38378906,302.95215007)
\curveto(202.78221916,303.09536864)(202.15917291,303.16698316)(201.51464844,303.16699382)
\curveto(200.53352349,303.16698316)(199.79589402,303.01659268)(199.30175781,302.71582195)
\curveto(198.81477521,302.41503078)(198.57128587,301.96385936)(198.57128906,301.36230632)
\curveto(198.57128587,300.90396459)(198.74674143,300.5423113)(199.09765625,300.27734539)
\curveto(199.44856364,300.01952537)(200.15396658,299.7724553)(201.21386719,299.53613445)
\lineto(201.890625,299.38574382)
\curveto(203.29426292,299.08495599)(204.28970463,298.65884964)(204.87695312,298.10742351)
\curveto(205.47134408,297.56314761)(205.7685443,296.80045306)(205.76855469,295.81933757)
\curveto(205.7685443,294.70214787)(205.32453433,293.81770865)(204.43652344,293.16601726)
\curveto(203.55565589,292.51432454)(202.34178991,292.18847851)(200.79492188,292.1884782)
\curveto(200.15038586,292.18847851)(199.47720945,292.25293157)(198.77539062,292.38183757)
\curveto(198.08072647,292.50358236)(197.34667772,292.68978009)(196.57324219,292.94043132)
\lineto(196.57324219,294.98144695)
\curveto(197.30370901,294.60188755)(198.02343486,294.31542951)(198.73242188,294.12207195)
\curveto(199.44140219,293.9358726)(200.1432244,293.84277373)(200.83789062,293.84277507)
\curveto(201.76887382,293.84277373)(202.48501894,294.00032566)(202.98632812,294.31543132)
\curveto(203.4876221,294.63769481)(203.73827289,295.08886623)(203.73828125,295.66894695)
\curveto(203.73827289,296.20605262)(203.55565589,296.61783606)(203.19042969,296.90429851)
\curveto(202.83234932,297.19075215)(202.04100896,297.46646802)(200.81640625,297.73144695)
\lineto(200.12890625,297.89257976)
\curveto(198.90429335,298.15038661)(198.01985413,298.54426642)(197.47558594,299.07422039)
\curveto(196.93131355,299.61132265)(196.65917841,300.34537139)(196.65917969,301.27636882)
\curveto(196.65917841,302.40786933)(197.06021967,303.28156638)(197.86230469,303.89746257)
\curveto(198.66438474,304.51333598)(199.80305547,304.82127838)(201.27832031,304.8212907)
\curveto(202.00878243,304.82127838)(202.69628175,304.76756749)(203.34082031,304.66015789)
\curveto(203.98534296,304.55272396)(204.57974341,304.39159131)(205.12402344,304.17675945)
}
}
{
\newrgbcolor{curcolor}{0 0 0}
\pscustom[linestyle=none,fillstyle=solid,fillcolor=curcolor]
{
\newpath
\moveto(214.39453125,298.5478532)
\curveto(212.7975201,298.54784715)(211.69107589,298.36523015)(211.07519531,298.00000164)
\curveto(210.45930629,297.63476213)(210.15136389,297.01171587)(210.15136719,296.13086101)
\curveto(210.15136389,295.42903517)(210.38053033,294.87044197)(210.83886719,294.45507976)
\curveto(211.30435753,294.04687509)(211.93456523,293.84277373)(212.72949219,293.84277507)
\curveto(213.82518834,293.84277373)(214.70246611,294.22949209)(215.36132812,295.00293132)
\curveto(216.02733458,295.783527)(216.36034206,296.81835669)(216.36035156,298.10742351)
\lineto(216.36035156,298.5478532)
\lineto(214.39453125,298.5478532)
\moveto(218.33691406,299.36425945)
\lineto(218.33691406,292.50000164)
\lineto(216.36035156,292.50000164)
\lineto(216.36035156,294.32617351)
\curveto(215.90917063,293.59570367)(215.34699671,293.0550141)(214.67382812,292.7041032)
\curveto(214.00064389,292.36035334)(213.17707701,292.18847851)(212.203125,292.1884782)
\curveto(210.97135005,292.18847851)(209.99023124,292.53222817)(209.25976562,293.2197282)
\curveto(208.53645665,293.91438824)(208.17480337,294.84179617)(208.17480469,296.00195476)
\curveto(208.17480337,297.35546553)(208.62597479,298.37597232)(209.52832031,299.0634782)
\curveto(210.43782194,299.75097095)(211.79133621,300.0947206)(213.58886719,300.0947282)
\lineto(216.36035156,300.0947282)
\lineto(216.36035156,300.28808757)
\curveto(216.36034206,301.19758408)(216.05956111,301.8994063)(215.45800781,302.39355632)
\curveto(214.86359876,302.89484801)(214.02570897,303.1454988)(212.94433594,303.14550945)
\curveto(212.25683053,303.1454988)(211.58723485,303.06314211)(210.93554688,302.89843914)
\curveto(210.28385074,302.73371536)(209.65722376,302.48664529)(209.05566406,302.1572282)
\lineto(209.05566406,303.98340007)
\curveto(209.77896843,304.26268519)(210.48079064,304.47036727)(211.16113281,304.60644695)
\curveto(211.84146637,304.74966387)(212.5039006,304.82127838)(213.1484375,304.8212907)
\curveto(214.88866384,304.82127838)(216.18846723,304.37010695)(217.04785156,303.46777507)
\curveto(217.90721551,302.56542126)(218.33690258,301.19758408)(218.33691406,299.36425945)
}
}
{
\newrgbcolor{curcolor}{0 0 0}
\pscustom[linestyle=none,fillstyle=solid,fillcolor=curcolor]
{
\newpath
\moveto(224.37402344,307.94726726)
\lineto(224.37402344,304.53125164)
\lineto(228.4453125,304.53125164)
\lineto(228.4453125,302.99511882)
\lineto(224.37402344,302.99511882)
\lineto(224.37402344,296.46386882)
\curveto(224.37401941,295.48274605)(224.50650626,294.85253835)(224.77148438,294.57324382)
\curveto(225.04361509,294.29394515)(225.59146611,294.15429686)(226.41503906,294.15429851)
\lineto(228.4453125,294.15429851)
\lineto(228.4453125,292.50000164)
\lineto(226.41503906,292.50000164)
\curveto(224.88964389,292.50000164)(223.83691057,292.78287896)(223.25683594,293.34863445)
\curveto(222.67675548,293.92154969)(222.38671671,294.95996011)(222.38671875,296.46386882)
\lineto(222.38671875,302.99511882)
\lineto(220.93652344,302.99511882)
\lineto(220.93652344,304.53125164)
\lineto(222.38671875,304.53125164)
\lineto(222.38671875,307.94726726)
\lineto(224.37402344,307.94726726)
}
}
{
\newrgbcolor{curcolor}{0 0 0}
\pscustom[linestyle=none,fillstyle=solid,fillcolor=curcolor]
{
\newpath
\moveto(241.34667969,299.00976726)
\lineto(241.34667969,298.04297039)
\lineto(232.25878906,298.04297039)
\curveto(232.3447232,296.68228912)(232.75292592,295.6438787)(233.48339844,294.92773601)
\curveto(234.22102341,294.21874992)(235.24511092,293.86425808)(236.55566406,293.86425945)
\curveto(237.31477031,293.86425808)(238.04881906,293.95735695)(238.7578125,294.14355632)
\curveto(239.47394784,294.32975241)(240.18293151,294.60904901)(240.88476562,294.98144695)
\lineto(240.88476562,293.11230632)
\curveto(240.17577006,292.81152476)(239.44888276,292.58235832)(238.70410156,292.42480632)
\curveto(237.95930092,292.26725447)(237.20376782,292.18847851)(236.4375,292.1884782)
\curveto(234.51822363,292.18847851)(232.99641526,292.7470717)(231.87207031,293.86425945)
\curveto(230.75488104,294.98144447)(230.19628785,296.49251066)(230.19628906,298.39746257)
\curveto(230.19628785,300.36685575)(230.72623524,301.9280521)(231.78613281,303.08105632)
\curveto(232.85318623,304.24120083)(234.28905719,304.82127838)(236.09375,304.8212907)
\curveto(237.71223085,304.82127838)(238.99054989,304.29849244)(239.92871094,303.25293132)
\curveto(240.87401155,302.21451015)(241.34666732,300.80012354)(241.34667969,299.00976726)
\moveto(239.37011719,299.58984539)
\curveto(239.3557839,300.67121742)(239.05142222,301.53417229)(238.45703125,302.17871257)
\curveto(237.86978278,302.8232335)(237.0891846,303.1454988)(236.11523438,303.14550945)
\curveto(235.01236376,303.1454988)(234.12792454,302.83397568)(233.46191406,302.21093914)
\curveto(232.80305608,301.58788317)(232.42349916,300.7106054)(232.32324219,299.5791032)
\lineto(239.37011719,299.58984539)
}
}
{
\newrgbcolor{curcolor}{0 0 0}
\pscustom[linestyle=none,fillstyle=solid,fillcolor=curcolor]
{
\newpath
\moveto(244.38671875,297.24804851)
\lineto(244.38671875,304.53125164)
\lineto(246.36328125,304.53125164)
\lineto(246.36328125,297.32324382)
\curveto(246.3632774,296.18456826)(246.58528239,295.32877485)(247.02929688,294.75586101)
\curveto(247.47330234,294.19010411)(248.13931729,293.90722679)(249.02734375,293.9072282)
\curveto(250.09439346,293.90722679)(250.93586398,294.24739572)(251.55175781,294.92773601)
\curveto(252.17479503,295.60807144)(252.48631816,296.53547937)(252.48632812,297.70996257)
\lineto(252.48632812,304.53125164)
\lineto(254.46289062,304.53125164)
\lineto(254.46289062,292.50000164)
\lineto(252.48632812,292.50000164)
\lineto(252.48632812,294.34765789)
\curveto(252.00650093,293.61718802)(251.44790774,293.07291773)(250.81054688,292.71484539)
\curveto(250.18033088,292.36393406)(249.44628213,292.18847851)(248.60839844,292.1884782)
\curveto(247.22623227,292.18847851)(246.17707967,292.61816558)(245.4609375,293.4775407)
\curveto(244.74478944,294.33691386)(244.38671688,295.59374854)(244.38671875,297.24804851)
\moveto(249.36035156,304.8212907)
\lineto(249.36035156,304.8212907)
}
}
{
\newrgbcolor{curcolor}{0 0 0}
\pscustom[linestyle=none,fillstyle=solid,fillcolor=curcolor]
{
\newpath
\moveto(265.52734375,302.68359539)
\curveto(265.30532972,302.81249132)(265.06184038,302.90559019)(264.796875,302.96289226)
\curveto(264.53905444,303.02733486)(264.2525964,303.05956139)(263.9375,303.05957195)
\curveto(262.82030616,303.05956139)(261.96093202,302.69432738)(261.359375,301.96386882)
\curveto(260.76496968,301.24055279)(260.46776945,300.19856165)(260.46777344,298.83789226)
\lineto(260.46777344,292.50000164)
\lineto(258.48046875,292.50000164)
\lineto(258.48046875,304.53125164)
\lineto(260.46777344,304.53125164)
\lineto(260.46777344,302.66211101)
\curveto(260.88313362,303.39256887)(261.42382318,303.93325843)(262.08984375,304.28418132)
\curveto(262.7558531,304.6422421)(263.56509708,304.82127838)(264.51757812,304.8212907)
\curveto(264.65363766,304.82127838)(264.80402814,304.8105362)(264.96875,304.78906414)
\curveto(265.13345489,304.77472894)(265.3160719,304.74966387)(265.51660156,304.71386882)
\lineto(265.52734375,302.68359539)
}
}
{
\newrgbcolor{curcolor}{0 0 0}
\pscustom[linestyle=none,fillstyle=solid,fillcolor=curcolor]
{
\newpath
\moveto(132.15917969,276.03808757)
\lineto(135.39257812,276.03808757)
\lineto(139.48535156,265.12402507)
\lineto(143.59960938,276.03808757)
\lineto(146.83300781,276.03808757)
\lineto(146.83300781,260.00000164)
\lineto(144.71679688,260.00000164)
\lineto(144.71679688,274.08300945)
\lineto(140.58105469,263.08300945)
\lineto(138.40039062,263.08300945)
\lineto(134.26464844,274.08300945)
\lineto(134.26464844,260.00000164)
\lineto(132.15917969,260.00000164)
\lineto(132.15917969,276.03808757)
}
}
{
\newrgbcolor{curcolor}{0 0 0}
\pscustom[linestyle=none,fillstyle=solid,fillcolor=curcolor]
{
\newpath
\moveto(155.72753906,270.64550945)
\curveto(154.66763755,270.6454988)(153.82974777,270.23013463)(153.21386719,269.3994157)
\curveto(152.59797817,268.57583941)(152.29003576,267.44433013)(152.29003906,266.00488445)
\curveto(152.29003576,264.56542676)(152.59439744,263.43033675)(153.203125,262.59961101)
\curveto(153.81900559,261.77604153)(154.6604761,261.36425808)(155.72753906,261.36425945)
\curveto(156.78026565,261.36425808)(157.61457471,261.77962225)(158.23046875,262.6103532)
\curveto(158.84634431,263.44107892)(159.15428671,264.57258821)(159.15429688,266.00488445)
\curveto(159.15428671,267.43000723)(158.84634431,268.55793579)(158.23046875,269.38867351)
\curveto(157.61457471,270.22655391)(156.78026565,270.6454988)(155.72753906,270.64550945)
\moveto(155.72753906,272.3212907)
\curveto(157.44628061,272.32127838)(158.79621415,271.76268519)(159.77734375,270.64550945)
\curveto(160.75845178,269.52831242)(161.24901118,267.98143897)(161.24902344,266.00488445)
\curveto(161.24901118,264.03547937)(160.75845178,262.48860592)(159.77734375,261.36425945)
\curveto(158.79621415,260.2470717)(157.44628061,259.68847851)(155.72753906,259.6884782)
\curveto(154.00162259,259.68847851)(152.64810832,260.2470717)(151.66699219,261.36425945)
\curveto(150.69303215,262.48860592)(150.20605347,264.03547937)(150.20605469,266.00488445)
\curveto(150.20605347,267.98143897)(150.69303215,269.52831242)(151.66699219,270.64550945)
\curveto(152.64810832,271.76268519)(154.00162259,272.32127838)(155.72753906,272.3212907)
}
}
{
\newrgbcolor{curcolor}{0 0 0}
\pscustom[linestyle=none,fillstyle=solid,fillcolor=curcolor]
{
\newpath
\moveto(166.46972656,275.44726726)
\lineto(166.46972656,272.03125164)
\lineto(170.54101562,272.03125164)
\lineto(170.54101562,270.49511882)
\lineto(166.46972656,270.49511882)
\lineto(166.46972656,263.96386882)
\curveto(166.46972253,262.98274605)(166.60220938,262.35253835)(166.8671875,262.07324382)
\curveto(167.13931822,261.79394515)(167.68716923,261.65429686)(168.51074219,261.65429851)
\lineto(170.54101562,261.65429851)
\lineto(170.54101562,260.00000164)
\lineto(168.51074219,260.00000164)
\curveto(166.98534702,260.00000164)(165.9326137,260.28287896)(165.35253906,260.84863445)
\curveto(164.77245861,261.42154969)(164.48241983,262.45996011)(164.48242188,263.96386882)
\lineto(164.48242188,270.49511882)
\lineto(163.03222656,270.49511882)
\lineto(163.03222656,272.03125164)
\lineto(164.48242188,272.03125164)
\lineto(164.48242188,275.44726726)
\lineto(166.46972656,275.44726726)
}
}
{
\newrgbcolor{curcolor}{0 0 0}
\pscustom[linestyle=none,fillstyle=solid,fillcolor=curcolor]
{
}
}
{
\newrgbcolor{curcolor}{0 0 0}
\pscustom[linestyle=none,fillstyle=solid,fillcolor=curcolor]
{
\newpath
\moveto(188.07226562,270.20507976)
\lineto(188.07226562,276.71484539)
\lineto(190.04882812,276.71484539)
\lineto(190.04882812,260.00000164)
\lineto(188.07226562,260.00000164)
\lineto(188.07226562,261.80468914)
\curveto(187.65689147,261.08854221)(187.13052481,260.5550141)(186.49316406,260.2041032)
\curveto(185.86294795,259.86035334)(185.10383412,259.68847851)(184.21582031,259.6884782)
\curveto(182.76203959,259.68847851)(181.57681942,260.26855605)(180.66015625,261.42871257)
\curveto(179.75064937,262.58886623)(179.29589722,264.11425533)(179.29589844,266.00488445)
\curveto(179.29589722,267.89550155)(179.75064937,269.42089065)(180.66015625,270.58105632)
\curveto(181.57681942,271.74120083)(182.76203959,272.32127838)(184.21582031,272.3212907)
\curveto(185.10383412,272.32127838)(185.86294795,272.14582282)(186.49316406,271.79492351)
\curveto(187.13052481,271.45116206)(187.65689147,270.92121467)(188.07226562,270.20507976)
\moveto(181.33691406,266.00488445)
\curveto(181.33691081,264.55110386)(181.63411103,263.40885239)(182.22851562,262.57812664)
\curveto(182.83007338,261.75455717)(183.65364026,261.34277373)(184.69921875,261.34277507)
\curveto(185.744784,261.34277373)(186.56835089,261.75455717)(187.16992188,262.57812664)
\curveto(187.77147469,263.40885239)(188.07225563,264.55110386)(188.07226562,266.00488445)
\curveto(188.07225563,267.45865303)(187.77147469,268.59732377)(187.16992188,269.42090007)
\curveto(186.56835089,270.25161899)(185.744784,270.66698316)(184.69921875,270.66699382)
\curveto(183.65364026,270.66698316)(182.83007338,270.25161899)(182.22851562,269.42090007)
\curveto(181.63411103,268.59732377)(181.33691081,267.45865303)(181.33691406,266.00488445)
}
}
{
\newrgbcolor{curcolor}{0 0 0}
\pscustom[linestyle=none,fillstyle=solid,fillcolor=curcolor]
{
\newpath
\moveto(204.41113281,266.50976726)
\lineto(204.41113281,265.54297039)
\lineto(195.32324219,265.54297039)
\curveto(195.40917633,264.18228912)(195.81737904,263.1438787)(196.54785156,262.42773601)
\curveto(197.28547653,261.71874992)(198.30956405,261.36425808)(199.62011719,261.36425945)
\curveto(200.37922344,261.36425808)(201.11327218,261.45735695)(201.82226562,261.64355632)
\curveto(202.53840097,261.82975241)(203.24738463,262.10904901)(203.94921875,262.48144695)
\lineto(203.94921875,260.61230632)
\curveto(203.24022318,260.31152476)(202.51333589,260.08235832)(201.76855469,259.92480632)
\curveto(201.02375404,259.76725447)(200.26822095,259.68847851)(199.50195312,259.6884782)
\curveto(197.58267676,259.68847851)(196.06086838,260.2470717)(194.93652344,261.36425945)
\curveto(193.81933417,262.48144447)(193.26074097,263.99251066)(193.26074219,265.89746257)
\curveto(193.26074097,267.86685575)(193.79068836,269.4280521)(194.85058594,270.58105632)
\curveto(195.91763936,271.74120083)(197.35351032,272.32127838)(199.15820312,272.3212907)
\curveto(200.77668398,272.32127838)(202.05500301,271.79849244)(202.99316406,270.75293132)
\curveto(203.93846467,269.71451015)(204.41112045,268.30012354)(204.41113281,266.50976726)
\moveto(202.43457031,267.08984539)
\curveto(202.42023702,268.17121742)(202.11587535,269.03417229)(201.52148438,269.67871257)
\curveto(200.9342359,270.3232335)(200.15363773,270.6454988)(199.1796875,270.64550945)
\curveto(198.07681689,270.6454988)(197.19237767,270.33397568)(196.52636719,269.71093914)
\curveto(195.8675092,269.08788317)(195.48795229,268.2106054)(195.38769531,267.0791032)
\lineto(202.43457031,267.08984539)
}
}
{
\newrgbcolor{curcolor}{0 0 0}
\pscustom[linestyle=none,fillstyle=solid,fillcolor=curcolor]
{
}
}
{
\newrgbcolor{curcolor}{0 0 0}
\pscustom[linestyle=none,fillstyle=solid,fillcolor=curcolor]
{
\newpath
\moveto(216.57128906,261.80468914)
\lineto(216.57128906,255.42382976)
\lineto(214.58398438,255.42382976)
\lineto(214.58398438,272.03125164)
\lineto(216.57128906,272.03125164)
\lineto(216.57128906,270.20507976)
\curveto(216.98664925,270.92121467)(217.50943518,271.45116206)(218.13964844,271.79492351)
\curveto(218.77701204,272.14582282)(219.53612586,272.32127838)(220.41699219,272.3212907)
\curveto(221.8779204,272.32127838)(223.06314056,271.74120083)(223.97265625,270.58105632)
\curveto(224.88931061,269.42089065)(225.34764349,267.89550155)(225.34765625,266.00488445)
\curveto(225.34764349,264.11425533)(224.88931061,262.58886623)(223.97265625,261.42871257)
\curveto(223.06314056,260.26855605)(221.8779204,259.68847851)(220.41699219,259.6884782)
\curveto(219.53612586,259.68847851)(218.77701204,259.86035334)(218.13964844,260.2041032)
\curveto(217.50943518,260.5550141)(216.98664925,261.08854221)(216.57128906,261.80468914)
\moveto(223.29589844,266.00488445)
\curveto(223.29588773,267.45865303)(222.99510678,268.59732377)(222.39355469,269.42090007)
\curveto(221.79914443,270.25161899)(220.97915827,270.66698316)(219.93359375,270.66699382)
\curveto(218.88801453,270.66698316)(218.06444765,270.25161899)(217.46289062,269.42090007)
\curveto(216.8684853,268.59732377)(216.57128508,267.45865303)(216.57128906,266.00488445)
\curveto(216.57128508,264.55110386)(216.8684853,263.40885239)(217.46289062,262.57812664)
\curveto(218.06444765,261.75455717)(218.88801453,261.34277373)(219.93359375,261.34277507)
\curveto(220.97915827,261.34277373)(221.79914443,261.75455717)(222.39355469,262.57812664)
\curveto(222.99510678,263.40885239)(223.29588773,264.55110386)(223.29589844,266.00488445)
}
}
{
\newrgbcolor{curcolor}{0 0 0}
\pscustom[linestyle=none,fillstyle=solid,fillcolor=curcolor]
{
\newpath
\moveto(234.09179688,266.0478532)
\curveto(232.49478572,266.04784715)(231.38834152,265.86523015)(230.77246094,265.50000164)
\curveto(230.15657192,265.13476213)(229.84862951,264.51171587)(229.84863281,263.63086101)
\curveto(229.84862951,262.92903517)(230.07779595,262.37044197)(230.53613281,261.95507976)
\curveto(231.00162315,261.54687509)(231.63183086,261.34277373)(232.42675781,261.34277507)
\curveto(233.52245397,261.34277373)(234.39973173,261.72949209)(235.05859375,262.50293132)
\curveto(235.7246002,263.283527)(236.05760768,264.31835669)(236.05761719,265.60742351)
\lineto(236.05761719,266.0478532)
\lineto(234.09179688,266.0478532)
\moveto(238.03417969,266.86425945)
\lineto(238.03417969,260.00000164)
\lineto(236.05761719,260.00000164)
\lineto(236.05761719,261.82617351)
\curveto(235.60643626,261.09570367)(235.04426234,260.5550141)(234.37109375,260.2041032)
\curveto(233.69790952,259.86035334)(232.87434263,259.68847851)(231.90039062,259.6884782)
\curveto(230.66861567,259.68847851)(229.68749686,260.03222817)(228.95703125,260.7197282)
\curveto(228.23372228,261.41438824)(227.87206899,262.34179617)(227.87207031,263.50195476)
\curveto(227.87206899,264.85546553)(228.32324042,265.87597232)(229.22558594,266.5634782)
\curveto(230.13508756,267.25097095)(231.48860183,267.5947206)(233.28613281,267.5947282)
\lineto(236.05761719,267.5947282)
\lineto(236.05761719,267.78808757)
\curveto(236.05760768,268.69758408)(235.75682673,269.3994063)(235.15527344,269.89355632)
\curveto(234.56086439,270.39484801)(233.7229746,270.6454988)(232.64160156,270.64550945)
\curveto(231.95409616,270.6454988)(231.28450047,270.56314211)(230.6328125,270.39843914)
\curveto(229.98111636,270.23371536)(229.35448938,269.98664529)(228.75292969,269.6572282)
\lineto(228.75292969,271.48340007)
\curveto(229.47623405,271.76268519)(230.17805627,271.97036727)(230.85839844,272.10644695)
\curveto(231.53873199,272.24966387)(232.20116622,272.32127838)(232.84570312,272.3212907)
\curveto(234.58592946,272.32127838)(235.88573285,271.87010695)(236.74511719,270.96777507)
\curveto(237.60448113,270.06542126)(238.0341682,268.69758408)(238.03417969,266.86425945)
}
}
{
\newrgbcolor{curcolor}{0 0 0}
\pscustom[linestyle=none,fillstyle=solid,fillcolor=curcolor]
{
\newpath
\moveto(249.78613281,271.67675945)
\lineto(249.78613281,269.80761882)
\curveto(249.22752988,270.09406706)(248.64745233,270.3089106)(248.04589844,270.45215007)
\curveto(247.44432854,270.59536864)(246.82128228,270.66698316)(246.17675781,270.66699382)
\curveto(245.19563287,270.66698316)(244.4580034,270.51659268)(243.96386719,270.21582195)
\curveto(243.47688459,269.91503078)(243.23339525,269.46385936)(243.23339844,268.86230632)
\curveto(243.23339525,268.40396459)(243.4088508,268.0423113)(243.75976562,267.77734539)
\curveto(244.11067302,267.51952537)(244.81607596,267.2724553)(245.87597656,267.03613445)
\lineto(246.55273438,266.88574382)
\curveto(247.95637229,266.58495599)(248.95181401,266.15884964)(249.5390625,265.60742351)
\curveto(250.13345345,265.06314761)(250.43065367,264.30045306)(250.43066406,263.31933757)
\curveto(250.43065367,262.20214787)(249.9866437,261.31770865)(249.09863281,260.66601726)
\curveto(248.21776526,260.01432454)(247.00389929,259.68847851)(245.45703125,259.6884782)
\curveto(244.81249523,259.68847851)(244.13931882,259.75293157)(243.4375,259.88183757)
\curveto(242.74283584,260.00358236)(242.0087871,260.18978009)(241.23535156,260.44043132)
\lineto(241.23535156,262.48144695)
\curveto(241.96581839,262.10188755)(242.68554423,261.81542951)(243.39453125,261.62207195)
\curveto(244.10351156,261.4358726)(244.80533378,261.34277373)(245.5,261.34277507)
\curveto(246.4309832,261.34277373)(247.14712831,261.50032566)(247.6484375,261.81543132)
\curveto(248.14973148,262.13769481)(248.40038227,262.58886623)(248.40039062,263.16894695)
\curveto(248.40038227,263.70605262)(248.21776526,264.11783606)(247.85253906,264.40429851)
\curveto(247.49445869,264.69075215)(246.70311834,264.96646802)(245.47851562,265.23144695)
\lineto(244.79101562,265.39257976)
\curveto(243.56640273,265.65038661)(242.68196351,266.04426642)(242.13769531,266.57422039)
\curveto(241.59342293,267.11132265)(241.32128778,267.84537139)(241.32128906,268.77636882)
\curveto(241.32128778,269.90786933)(241.72232905,270.78156638)(242.52441406,271.39746257)
\curveto(243.32649411,272.01333598)(244.46516485,272.32127838)(245.94042969,272.3212907)
\curveto(246.67089181,272.32127838)(247.35839112,272.26756749)(248.00292969,272.16015789)
\curveto(248.64745233,272.05272396)(249.24185278,271.89159131)(249.78613281,271.67675945)
}
}
{
\newrgbcolor{curcolor}{0 0 0}
\pscustom[linestyle=none,fillstyle=solid,fillcolor=curcolor]
{
\newpath
\moveto(261.25878906,271.67675945)
\lineto(261.25878906,269.80761882)
\curveto(260.70018613,270.09406706)(260.12010858,270.3089106)(259.51855469,270.45215007)
\curveto(258.91698479,270.59536864)(258.29393853,270.66698316)(257.64941406,270.66699382)
\curveto(256.66828912,270.66698316)(255.93065965,270.51659268)(255.43652344,270.21582195)
\curveto(254.94954084,269.91503078)(254.7060515,269.46385936)(254.70605469,268.86230632)
\curveto(254.7060515,268.40396459)(254.88150705,268.0423113)(255.23242188,267.77734539)
\curveto(255.58332927,267.51952537)(256.28873221,267.2724553)(257.34863281,267.03613445)
\lineto(258.02539062,266.88574382)
\curveto(259.42902854,266.58495599)(260.42447026,266.15884964)(261.01171875,265.60742351)
\curveto(261.6061097,265.06314761)(261.90330992,264.30045306)(261.90332031,263.31933757)
\curveto(261.90330992,262.20214787)(261.45929995,261.31770865)(260.57128906,260.66601726)
\curveto(259.69042151,260.01432454)(258.47655554,259.68847851)(256.9296875,259.6884782)
\curveto(256.28515148,259.68847851)(255.61197507,259.75293157)(254.91015625,259.88183757)
\curveto(254.21549209,260.00358236)(253.48144335,260.18978009)(252.70800781,260.44043132)
\lineto(252.70800781,262.48144695)
\curveto(253.43847464,262.10188755)(254.15820048,261.81542951)(254.8671875,261.62207195)
\curveto(255.57616781,261.4358726)(256.27799003,261.34277373)(256.97265625,261.34277507)
\curveto(257.90363945,261.34277373)(258.61978456,261.50032566)(259.12109375,261.81543132)
\curveto(259.62238773,262.13769481)(259.87303852,262.58886623)(259.87304688,263.16894695)
\curveto(259.87303852,263.70605262)(259.69042151,264.11783606)(259.32519531,264.40429851)
\curveto(258.96711494,264.69075215)(258.17577459,264.96646802)(256.95117188,265.23144695)
\lineto(256.26367188,265.39257976)
\curveto(255.03905898,265.65038661)(254.15461976,266.04426642)(253.61035156,266.57422039)
\curveto(253.06607918,267.11132265)(252.79394403,267.84537139)(252.79394531,268.77636882)
\curveto(252.79394403,269.90786933)(253.1949853,270.78156638)(253.99707031,271.39746257)
\curveto(254.79915036,272.01333598)(255.9378211,272.32127838)(257.41308594,272.3212907)
\curveto(258.14354806,272.32127838)(258.83104737,272.26756749)(259.47558594,272.16015789)
\curveto(260.12010858,272.05272396)(260.71450903,271.89159131)(261.25878906,271.67675945)
}
}
{
\newrgbcolor{curcolor}{0 0 0}
\pscustom[linestyle=none,fillstyle=solid,fillcolor=curcolor]
{
\newpath
\moveto(275.35253906,266.50976726)
\lineto(275.35253906,265.54297039)
\lineto(266.26464844,265.54297039)
\curveto(266.35058258,264.18228912)(266.75878529,263.1438787)(267.48925781,262.42773601)
\curveto(268.22688278,261.71874992)(269.2509703,261.36425808)(270.56152344,261.36425945)
\curveto(271.32062969,261.36425808)(272.05467843,261.45735695)(272.76367188,261.64355632)
\curveto(273.47980722,261.82975241)(274.18879088,262.10904901)(274.890625,262.48144695)
\lineto(274.890625,260.61230632)
\curveto(274.18162943,260.31152476)(273.45474214,260.08235832)(272.70996094,259.92480632)
\curveto(271.96516029,259.76725447)(271.2096272,259.68847851)(270.44335938,259.6884782)
\curveto(268.52408301,259.68847851)(267.00227463,260.2470717)(265.87792969,261.36425945)
\curveto(264.76074042,262.48144447)(264.20214722,263.99251066)(264.20214844,265.89746257)
\curveto(264.20214722,267.86685575)(264.73209461,269.4280521)(265.79199219,270.58105632)
\curveto(266.85904561,271.74120083)(268.29491657,272.32127838)(270.09960938,272.3212907)
\curveto(271.71809023,272.32127838)(272.99640926,271.79849244)(273.93457031,270.75293132)
\curveto(274.87987092,269.71451015)(275.3525267,268.30012354)(275.35253906,266.50976726)
\moveto(273.37597656,267.08984539)
\curveto(273.36164327,268.17121742)(273.0572816,269.03417229)(272.46289062,269.67871257)
\curveto(271.87564215,270.3232335)(271.09504398,270.6454988)(270.12109375,270.64550945)
\curveto(269.01822314,270.6454988)(268.13378392,270.33397568)(267.46777344,269.71093914)
\curveto(266.80891545,269.08788317)(266.42935854,268.2106054)(266.32910156,267.0791032)
\lineto(273.37597656,267.08984539)
}
}
{
\newrgbcolor{curcolor}{0 0 0}
\pscustom[linestyle=none,fillstyle=solid,fillcolor=curcolor]
{
\newpath
\moveto(144.16894531,224.80273601)
\lineto(144.16894531,222.51465007)
\curveto(143.43846312,223.19497542)(142.65786495,223.70343845)(141.82714844,224.0400407)
\curveto(141.00356973,224.37661486)(140.12629196,224.54490897)(139.1953125,224.54492351)
\curveto(137.3619718,224.54490897)(135.95832738,223.98273505)(134.984375,222.85840007)
\curveto(134.01041266,221.74120083)(133.52343398,220.12271287)(133.5234375,218.00293132)
\curveto(133.52343398,215.89029522)(134.01041266,214.27180726)(134.984375,213.14746257)
\curveto(135.95832738,212.03027304)(137.3619718,211.47167985)(139.1953125,211.47168132)
\curveto(140.12629196,211.47167985)(141.00356973,211.63997395)(141.82714844,211.97656414)
\curveto(142.65786495,212.31315036)(143.43846312,212.8216134)(144.16894531,213.50195476)
\lineto(144.16894531,211.2353532)
\curveto(143.40981732,210.71972748)(142.60415406,210.33300912)(141.75195312,210.07519695)
\curveto(140.90689013,209.81738463)(140.01170874,209.68847851)(139.06640625,209.6884782)
\curveto(136.63866524,209.68847851)(134.72655777,210.42968871)(133.33007812,211.91211101)
\curveto(131.93359182,213.40169094)(131.23535033,215.43196235)(131.23535156,218.00293132)
\curveto(131.23535033,220.58104574)(131.93359182,222.61131715)(133.33007812,224.09375164)
\curveto(134.72655777,225.58331939)(136.63866524,226.32811031)(139.06640625,226.32812664)
\curveto(140.02603164,226.32811031)(140.92837449,226.19920419)(141.7734375,225.94140789)
\curveto(142.62563842,225.69074115)(143.42414022,225.31118424)(144.16894531,224.80273601)
}
}
{
\newrgbcolor{curcolor}{0 0 0}
\pscustom[linestyle=none,fillstyle=solid,fillcolor=curcolor]
{
\newpath
\moveto(152.11816406,220.64550945)
\curveto(151.05826255,220.6454988)(150.22037277,220.23013463)(149.60449219,219.3994157)
\curveto(148.98860317,218.57583941)(148.68066076,217.44433013)(148.68066406,216.00488445)
\curveto(148.68066076,214.56542676)(148.98502244,213.43033675)(149.59375,212.59961101)
\curveto(150.20963059,211.77604153)(151.0511011,211.36425808)(152.11816406,211.36425945)
\curveto(153.17089065,211.36425808)(154.00519971,211.77962225)(154.62109375,212.6103532)
\curveto(155.23696931,213.44107892)(155.54491171,214.57258821)(155.54492188,216.00488445)
\curveto(155.54491171,217.43000723)(155.23696931,218.55793579)(154.62109375,219.38867351)
\curveto(154.00519971,220.22655391)(153.17089065,220.6454988)(152.11816406,220.64550945)
\moveto(152.11816406,222.3212907)
\curveto(153.83690561,222.32127838)(155.18683915,221.76268519)(156.16796875,220.64550945)
\curveto(157.14907678,219.52831242)(157.63963618,217.98143897)(157.63964844,216.00488445)
\curveto(157.63963618,214.03547937)(157.14907678,212.48860592)(156.16796875,211.36425945)
\curveto(155.18683915,210.2470717)(153.83690561,209.68847851)(152.11816406,209.6884782)
\curveto(150.39224759,209.68847851)(149.03873332,210.2470717)(148.05761719,211.36425945)
\curveto(147.08365715,212.48860592)(146.59667847,214.03547937)(146.59667969,216.00488445)
\curveto(146.59667847,217.98143897)(147.08365715,219.52831242)(148.05761719,220.64550945)
\curveto(149.03873332,221.76268519)(150.39224759,222.32127838)(152.11816406,222.3212907)
}
}
{
\newrgbcolor{curcolor}{0 0 0}
\pscustom[linestyle=none,fillstyle=solid,fillcolor=curcolor]
{
\newpath
\moveto(170.90625,217.26172039)
\lineto(170.90625,210.00000164)
\lineto(168.9296875,210.00000164)
\lineto(168.9296875,217.19726726)
\curveto(168.9296774,218.3359308)(168.70767242,219.18814349)(168.26367188,219.75390789)
\curveto(167.81965247,220.31965277)(167.15363751,220.6025301)(166.265625,220.6025407)
\curveto(165.19856134,220.6025301)(164.35709083,220.26236117)(163.74121094,219.58203289)
\curveto(163.12532123,218.90168544)(162.81737883,217.97427752)(162.81738281,216.79980632)
\lineto(162.81738281,210.00000164)
\lineto(160.83007812,210.00000164)
\lineto(160.83007812,222.03125164)
\lineto(162.81738281,222.03125164)
\lineto(162.81738281,220.16211101)
\curveto(163.2900346,220.88540742)(163.84504707,221.42609698)(164.48242188,221.78418132)
\curveto(165.12694683,222.1422421)(165.86815703,222.32127838)(166.70605469,222.3212907)
\curveto(168.08820689,222.32127838)(169.13377876,221.89159131)(169.84277344,221.0322282)
\curveto(170.55174609,220.18000448)(170.90623793,218.9231698)(170.90625,217.26172039)
}
}
{
\newrgbcolor{curcolor}{0 0 0}
\pscustom[linestyle=none,fillstyle=solid,fillcolor=curcolor]
{
\newpath
\moveto(184.87109375,217.26172039)
\lineto(184.87109375,210.00000164)
\lineto(182.89453125,210.00000164)
\lineto(182.89453125,217.19726726)
\curveto(182.89452115,218.3359308)(182.67251617,219.18814349)(182.22851562,219.75390789)
\curveto(181.78449622,220.31965277)(181.11848126,220.6025301)(180.23046875,220.6025407)
\curveto(179.16340509,220.6025301)(178.32193458,220.26236117)(177.70605469,219.58203289)
\curveto(177.09016498,218.90168544)(176.78222258,217.97427752)(176.78222656,216.79980632)
\lineto(176.78222656,210.00000164)
\lineto(174.79492188,210.00000164)
\lineto(174.79492188,222.03125164)
\lineto(176.78222656,222.03125164)
\lineto(176.78222656,220.16211101)
\curveto(177.25487835,220.88540742)(177.80989082,221.42609698)(178.44726562,221.78418132)
\curveto(179.09179058,222.1422421)(179.83300078,222.32127838)(180.67089844,222.3212907)
\curveto(182.05305064,222.32127838)(183.09862251,221.89159131)(183.80761719,221.0322282)
\curveto(184.51658984,220.18000448)(184.87108168,218.9231698)(184.87109375,217.26172039)
}
}
{
\newrgbcolor{curcolor}{0 0 0}
\pscustom[linestyle=none,fillstyle=solid,fillcolor=curcolor]
{
\newpath
\moveto(199.12597656,216.50976726)
\lineto(199.12597656,215.54297039)
\lineto(190.03808594,215.54297039)
\curveto(190.12402008,214.18228912)(190.53222279,213.1438787)(191.26269531,212.42773601)
\curveto(192.00032028,211.71874992)(193.0244078,211.36425808)(194.33496094,211.36425945)
\curveto(195.09406719,211.36425808)(195.82811593,211.45735695)(196.53710938,211.64355632)
\curveto(197.25324472,211.82975241)(197.96222838,212.10904901)(198.6640625,212.48144695)
\lineto(198.6640625,210.61230632)
\curveto(197.95506693,210.31152476)(197.22817964,210.08235832)(196.48339844,209.92480632)
\curveto(195.73859779,209.76725447)(194.9830647,209.68847851)(194.21679688,209.6884782)
\curveto(192.29752051,209.68847851)(190.77571213,210.2470717)(189.65136719,211.36425945)
\curveto(188.53417792,212.48144447)(187.97558472,213.99251066)(187.97558594,215.89746257)
\curveto(187.97558472,217.86685575)(188.50553211,219.4280521)(189.56542969,220.58105632)
\curveto(190.63248311,221.74120083)(192.06835407,222.32127838)(193.87304688,222.3212907)
\curveto(195.49152773,222.32127838)(196.76984676,221.79849244)(197.70800781,220.75293132)
\curveto(198.65330842,219.71451015)(199.1259642,218.30012354)(199.12597656,216.50976726)
\moveto(197.14941406,217.08984539)
\curveto(197.13508077,218.17121742)(196.8307191,219.03417229)(196.23632812,219.67871257)
\curveto(195.64907965,220.3232335)(194.86848148,220.6454988)(193.89453125,220.64550945)
\curveto(192.79166064,220.6454988)(191.90722142,220.33397568)(191.24121094,219.71093914)
\curveto(190.58235295,219.08788317)(190.20279604,218.2106054)(190.10253906,217.0791032)
\lineto(197.14941406,217.08984539)
}
}
{
\newrgbcolor{curcolor}{0 0 0}
\pscustom[linestyle=none,fillstyle=solid,fillcolor=curcolor]
{
\newpath
\moveto(211.984375,222.03125164)
\lineto(207.63378906,216.17675945)
\lineto(212.20996094,210.00000164)
\lineto(209.87890625,210.00000164)
\lineto(206.37695312,214.72656414)
\lineto(202.875,210.00000164)
\lineto(200.54394531,210.00000164)
\lineto(205.21679688,216.29492351)
\lineto(200.94140625,222.03125164)
\lineto(203.27246094,222.03125164)
\lineto(206.46289062,217.74511882)
\lineto(209.65332031,222.03125164)
\lineto(211.984375,222.03125164)
}
}
{
\newrgbcolor{curcolor}{0 0 0}
\pscustom[linestyle=none,fillstyle=solid,fillcolor=curcolor]
{
\newpath
\moveto(215.00292969,222.03125164)
\lineto(216.97949219,222.03125164)
\lineto(216.97949219,210.00000164)
\lineto(215.00292969,210.00000164)
\lineto(215.00292969,222.03125164)
\moveto(215.00292969,226.71484539)
\lineto(216.97949219,226.71484539)
\lineto(216.97949219,224.2119157)
\lineto(215.00292969,224.2119157)
\lineto(215.00292969,226.71484539)
}
}
{
\newrgbcolor{curcolor}{0 0 0}
\pscustom[linestyle=none,fillstyle=solid,fillcolor=curcolor]
{
\newpath
\moveto(225.76660156,220.64550945)
\curveto(224.70670005,220.6454988)(223.86881027,220.23013463)(223.25292969,219.3994157)
\curveto(222.63704067,218.57583941)(222.32909826,217.44433013)(222.32910156,216.00488445)
\curveto(222.32909826,214.56542676)(222.63345994,213.43033675)(223.2421875,212.59961101)
\curveto(223.85806809,211.77604153)(224.6995386,211.36425808)(225.76660156,211.36425945)
\curveto(226.81932815,211.36425808)(227.65363721,211.77962225)(228.26953125,212.6103532)
\curveto(228.88540681,213.44107892)(229.19334921,214.57258821)(229.19335938,216.00488445)
\curveto(229.19334921,217.43000723)(228.88540681,218.55793579)(228.26953125,219.38867351)
\curveto(227.65363721,220.22655391)(226.81932815,220.6454988)(225.76660156,220.64550945)
\moveto(225.76660156,222.3212907)
\curveto(227.48534311,222.32127838)(228.83527665,221.76268519)(229.81640625,220.64550945)
\curveto(230.79751428,219.52831242)(231.28807368,217.98143897)(231.28808594,216.00488445)
\curveto(231.28807368,214.03547937)(230.79751428,212.48860592)(229.81640625,211.36425945)
\curveto(228.83527665,210.2470717)(227.48534311,209.68847851)(225.76660156,209.6884782)
\curveto(224.04068509,209.68847851)(222.68717082,210.2470717)(221.70605469,211.36425945)
\curveto(220.73209465,212.48860592)(220.24511597,214.03547937)(220.24511719,216.00488445)
\curveto(220.24511597,217.98143897)(220.73209465,219.52831242)(221.70605469,220.64550945)
\curveto(222.68717082,221.76268519)(224.04068509,222.32127838)(225.76660156,222.3212907)
}
}
{
\newrgbcolor{curcolor}{0 0 0}
\pscustom[linestyle=none,fillstyle=solid,fillcolor=curcolor]
{
\newpath
\moveto(244.5546875,217.26172039)
\lineto(244.5546875,210.00000164)
\lineto(242.578125,210.00000164)
\lineto(242.578125,217.19726726)
\curveto(242.5781149,218.3359308)(242.35610992,219.18814349)(241.91210938,219.75390789)
\curveto(241.46808997,220.31965277)(240.80207501,220.6025301)(239.9140625,220.6025407)
\curveto(238.84699884,220.6025301)(238.00552833,220.26236117)(237.38964844,219.58203289)
\curveto(236.77375873,218.90168544)(236.46581633,217.97427752)(236.46582031,216.79980632)
\lineto(236.46582031,210.00000164)
\lineto(234.47851562,210.00000164)
\lineto(234.47851562,222.03125164)
\lineto(236.46582031,222.03125164)
\lineto(236.46582031,220.16211101)
\curveto(236.9384721,220.88540742)(237.49348457,221.42609698)(238.13085938,221.78418132)
\curveto(238.77538433,222.1422421)(239.51659453,222.32127838)(240.35449219,222.3212907)
\curveto(241.73664439,222.32127838)(242.78221626,221.89159131)(243.49121094,221.0322282)
\curveto(244.20018359,220.18000448)(244.55467543,218.9231698)(244.5546875,217.26172039)
}
}
{
\newrgbcolor{curcolor}{0 0 0}
\pscustom[linestyle=none,fillstyle=solid,fillcolor=curcolor]
{
}
}
{
\newrgbcolor{curcolor}{0 0 0}
\pscustom[linestyle=none,fillstyle=solid,fillcolor=curcolor]
{
\newpath
\moveto(260.99023438,216.0478532)
\curveto(259.39322322,216.04784715)(258.28677902,215.86523015)(257.67089844,215.50000164)
\curveto(257.05500942,215.13476213)(256.74706701,214.51171587)(256.74707031,213.63086101)
\curveto(256.74706701,212.92903517)(256.97623345,212.37044197)(257.43457031,211.95507976)
\curveto(257.90006065,211.54687509)(258.53026836,211.34277373)(259.32519531,211.34277507)
\curveto(260.42089147,211.34277373)(261.29816923,211.72949209)(261.95703125,212.50293132)
\curveto(262.6230377,213.283527)(262.95604518,214.31835669)(262.95605469,215.60742351)
\lineto(262.95605469,216.0478532)
\lineto(260.99023438,216.0478532)
\moveto(264.93261719,216.86425945)
\lineto(264.93261719,210.00000164)
\lineto(262.95605469,210.00000164)
\lineto(262.95605469,211.82617351)
\curveto(262.50487376,211.09570367)(261.94269984,210.5550141)(261.26953125,210.2041032)
\curveto(260.59634702,209.86035334)(259.77278013,209.68847851)(258.79882812,209.6884782)
\curveto(257.56705317,209.68847851)(256.58593436,210.03222817)(255.85546875,210.7197282)
\curveto(255.13215978,211.41438824)(254.77050649,212.34179617)(254.77050781,213.50195476)
\curveto(254.77050649,214.85546553)(255.22167792,215.87597232)(256.12402344,216.5634782)
\curveto(257.03352506,217.25097095)(258.38703933,217.5947206)(260.18457031,217.5947282)
\lineto(262.95605469,217.5947282)
\lineto(262.95605469,217.78808757)
\curveto(262.95604518,218.69758408)(262.65526423,219.3994063)(262.05371094,219.89355632)
\curveto(261.45930189,220.39484801)(260.6214121,220.6454988)(259.54003906,220.64550945)
\curveto(258.85253366,220.6454988)(258.18293797,220.56314211)(257.53125,220.39843914)
\curveto(256.87955386,220.23371536)(256.25292688,219.98664529)(255.65136719,219.6572282)
\lineto(255.65136719,221.48340007)
\curveto(256.37467155,221.76268519)(257.07649377,221.97036727)(257.75683594,222.10644695)
\curveto(258.43716949,222.24966387)(259.09960372,222.32127838)(259.74414062,222.3212907)
\curveto(261.48436696,222.32127838)(262.78417035,221.87010695)(263.64355469,220.96777507)
\curveto(264.50291863,220.06542126)(264.9326057,218.69758408)(264.93261719,216.86425945)
}
}
{
\newrgbcolor{curcolor}{0 0 0}
\pscustom[linestyle=none,fillstyle=solid,fillcolor=curcolor]
{
\newpath
\moveto(268.81054688,214.74804851)
\lineto(268.81054688,222.03125164)
\lineto(270.78710938,222.03125164)
\lineto(270.78710938,214.82324382)
\curveto(270.78710553,213.68456826)(271.00911052,212.82877485)(271.453125,212.25586101)
\curveto(271.89713046,211.69010411)(272.56314542,211.40722679)(273.45117188,211.4072282)
\curveto(274.51822159,211.40722679)(275.3596921,211.74739572)(275.97558594,212.42773601)
\curveto(276.59862316,213.10807144)(276.91014628,214.03547937)(276.91015625,215.20996257)
\lineto(276.91015625,222.03125164)
\lineto(278.88671875,222.03125164)
\lineto(278.88671875,210.00000164)
\lineto(276.91015625,210.00000164)
\lineto(276.91015625,211.84765789)
\curveto(276.43032905,211.11718802)(275.87173586,210.57291773)(275.234375,210.21484539)
\curveto(274.604159,209.86393406)(273.87011026,209.68847851)(273.03222656,209.6884782)
\curveto(271.6500604,209.68847851)(270.6009078,210.11816558)(269.88476562,210.9775407)
\curveto(269.16861756,211.83691386)(268.81054501,213.09374854)(268.81054688,214.74804851)
\moveto(273.78417969,222.3212907)
\lineto(273.78417969,222.3212907)
}
}
{
\newrgbcolor{curcolor}{0 0 0}
\pscustom[linestyle=none,fillstyle=solid,fillcolor=curcolor]
{
\newpath
\moveto(284.93457031,225.44726726)
\lineto(284.93457031,222.03125164)
\lineto(289.00585938,222.03125164)
\lineto(289.00585938,220.49511882)
\lineto(284.93457031,220.49511882)
\lineto(284.93457031,213.96386882)
\curveto(284.93456628,212.98274605)(285.06705313,212.35253835)(285.33203125,212.07324382)
\curveto(285.60416197,211.79394515)(286.15201298,211.65429686)(286.97558594,211.65429851)
\lineto(289.00585938,211.65429851)
\lineto(289.00585938,210.00000164)
\lineto(286.97558594,210.00000164)
\curveto(285.45019077,210.00000164)(284.39745745,210.28287896)(283.81738281,210.84863445)
\curveto(283.23730236,211.42154969)(282.94726358,212.45996011)(282.94726562,213.96386882)
\lineto(282.94726562,220.49511882)
\lineto(281.49707031,220.49511882)
\lineto(281.49707031,222.03125164)
\lineto(282.94726562,222.03125164)
\lineto(282.94726562,225.44726726)
\lineto(284.93457031,225.44726726)
}
}
{
\newrgbcolor{curcolor}{0 0 0}
\pscustom[linestyle=none,fillstyle=solid,fillcolor=curcolor]
{
\newpath
\moveto(296.27832031,220.64550945)
\curveto(295.2184188,220.6454988)(294.38052902,220.23013463)(293.76464844,219.3994157)
\curveto(293.14875942,218.57583941)(292.84081701,217.44433013)(292.84082031,216.00488445)
\curveto(292.84081701,214.56542676)(293.14517869,213.43033675)(293.75390625,212.59961101)
\curveto(294.36978684,211.77604153)(295.21125735,211.36425808)(296.27832031,211.36425945)
\curveto(297.3310469,211.36425808)(298.16535596,211.77962225)(298.78125,212.6103532)
\curveto(299.39712556,213.44107892)(299.70506796,214.57258821)(299.70507812,216.00488445)
\curveto(299.70506796,217.43000723)(299.39712556,218.55793579)(298.78125,219.38867351)
\curveto(298.16535596,220.22655391)(297.3310469,220.6454988)(296.27832031,220.64550945)
\moveto(296.27832031,222.3212907)
\curveto(297.99706186,222.32127838)(299.3469954,221.76268519)(300.328125,220.64550945)
\curveto(301.30923303,219.52831242)(301.79979243,217.98143897)(301.79980469,216.00488445)
\curveto(301.79979243,214.03547937)(301.30923303,212.48860592)(300.328125,211.36425945)
\curveto(299.3469954,210.2470717)(297.99706186,209.68847851)(296.27832031,209.6884782)
\curveto(294.55240384,209.68847851)(293.19888957,210.2470717)(292.21777344,211.36425945)
\curveto(291.2438134,212.48860592)(290.75683472,214.03547937)(290.75683594,216.00488445)
\curveto(290.75683472,217.98143897)(291.2438134,219.52831242)(292.21777344,220.64550945)
\curveto(293.19888957,221.76268519)(294.55240384,222.32127838)(296.27832031,222.3212907)
}
}
{
\newrgbcolor{curcolor}{0 0 0}
\pscustom[linestyle=none,fillstyle=solid,fillcolor=curcolor]
{
\newpath
\moveto(132.15917969,186.03807231)
\lineto(135.39257812,186.03807231)
\lineto(139.48535156,175.12400981)
\lineto(143.59960938,186.03807231)
\lineto(146.83300781,186.03807231)
\lineto(146.83300781,169.99998638)
\lineto(144.71679688,169.99998638)
\lineto(144.71679688,184.08299419)
\lineto(140.58105469,173.08299419)
\lineto(138.40039062,173.08299419)
\lineto(134.26464844,184.08299419)
\lineto(134.26464844,169.99998638)
\lineto(132.15917969,169.99998638)
\lineto(132.15917969,186.03807231)
}
}
{
\newrgbcolor{curcolor}{0 0 0}
\pscustom[linestyle=none,fillstyle=solid,fillcolor=curcolor]
{
\newpath
\moveto(161.35644531,176.509752)
\lineto(161.35644531,175.54295513)
\lineto(152.26855469,175.54295513)
\curveto(152.35448883,174.18227386)(152.76269154,173.14386344)(153.49316406,172.42772075)
\curveto(154.23078903,171.71873466)(155.25487655,171.36424283)(156.56542969,171.36424419)
\curveto(157.32453594,171.36424283)(158.05858468,171.45734169)(158.76757812,171.64354106)
\curveto(159.48371347,171.82973715)(160.19269713,172.10903375)(160.89453125,172.48143169)
\lineto(160.89453125,170.61229106)
\curveto(160.18553568,170.3115095)(159.45864839,170.08234307)(158.71386719,169.92479106)
\curveto(157.96906654,169.76723921)(157.21353345,169.68846325)(156.44726562,169.68846294)
\curveto(154.52798926,169.68846325)(153.00618088,170.24705644)(151.88183594,171.36424419)
\curveto(150.76464667,172.48142921)(150.20605347,173.99249541)(150.20605469,175.89744731)
\curveto(150.20605347,177.86684049)(150.73600086,179.42803684)(151.79589844,180.58104106)
\curveto(152.86295186,181.74118557)(154.29882282,182.32126312)(156.10351562,182.32127544)
\curveto(157.72199648,182.32126312)(159.00031551,181.79847718)(159.93847656,180.75291606)
\curveto(160.88377717,179.71449489)(161.35643295,178.30010829)(161.35644531,176.509752)
\moveto(159.37988281,177.08983013)
\curveto(159.36554952,178.17120216)(159.06118785,179.03415703)(158.46679688,179.67869731)
\curveto(157.8795484,180.32321824)(157.09895023,180.64548354)(156.125,180.64549419)
\curveto(155.02212939,180.64548354)(154.13769017,180.33396042)(153.47167969,179.71092388)
\curveto(152.8128217,179.08786791)(152.43326479,178.21059015)(152.33300781,177.07908794)
\lineto(159.37988281,177.08983013)
\moveto(157.47851562,187.5956895)
\lineto(159.61621094,187.5956895)
\lineto(156.11425781,183.556627)
\lineto(154.47070312,183.556627)
\lineto(157.47851562,187.5956895)
}
}
{
\newrgbcolor{curcolor}{0 0 0}
\pscustom[linestyle=none,fillstyle=solid,fillcolor=curcolor]
{
\newpath
\moveto(173.96777344,179.72166606)
\curveto(174.46190213,180.60967629)(175.05272185,181.26494907)(175.74023438,181.68748638)
\curveto(176.42772047,182.11000031)(177.23696446,182.32126312)(178.16796875,182.32127544)
\curveto(179.42120706,182.32126312)(180.38800297,181.88083387)(181.06835938,180.99998638)
\curveto(181.7486787,180.12627833)(182.08884763,178.88018583)(182.08886719,177.26170513)
\lineto(182.08886719,169.99998638)
\lineto(180.1015625,169.99998638)
\lineto(180.1015625,177.197252)
\curveto(180.10154493,178.35023844)(179.89744357,179.20603186)(179.48925781,179.76463481)
\curveto(179.08103813,180.32321824)(178.45799188,180.60251484)(177.62011719,180.60252544)
\curveto(176.59601458,180.60251484)(175.78677059,180.26234591)(175.19238281,179.58201763)
\curveto(174.5979697,178.90167018)(174.30076948,177.97426226)(174.30078125,176.79979106)
\lineto(174.30078125,169.99998638)
\lineto(172.31347656,169.99998638)
\lineto(172.31347656,177.197252)
\curveto(172.31346678,178.35739989)(172.10936542,179.21319331)(171.70117188,179.76463481)
\curveto(171.29295998,180.32321824)(170.66275228,180.60251484)(169.81054688,180.60252544)
\curveto(168.80077498,180.60251484)(167.99869245,180.25876518)(167.40429688,179.57127544)
\curveto(166.80989155,178.89092801)(166.51269133,177.96710081)(166.51269531,176.79979106)
\lineto(166.51269531,169.99998638)
\lineto(164.52539062,169.99998638)
\lineto(164.52539062,182.03123638)
\lineto(166.51269531,182.03123638)
\lineto(166.51269531,180.16209575)
\curveto(166.96386275,180.89971506)(167.50455231,181.44398535)(168.13476562,181.79490825)
\curveto(168.76496772,182.14580756)(169.51333937,182.32126312)(170.37988281,182.32127544)
\curveto(171.253572,182.32126312)(171.9947822,182.09925813)(172.60351562,181.65525981)
\curveto(173.21939035,181.21123819)(173.6741425,180.56670758)(173.96777344,179.72166606)
}
}
{
\newrgbcolor{curcolor}{0 0 0}
\pscustom[linestyle=none,fillstyle=solid,fillcolor=curcolor]
{
\newpath
\moveto(190.70410156,180.64549419)
\curveto(189.64420005,180.64548354)(188.80631027,180.23011938)(188.19042969,179.39940044)
\curveto(187.57454067,178.57582416)(187.26659826,177.44431487)(187.26660156,176.00486919)
\curveto(187.26659826,174.5654115)(187.57095994,173.43032149)(188.1796875,172.59959575)
\curveto(188.79556809,171.77602627)(189.6370386,171.36424283)(190.70410156,171.36424419)
\curveto(191.75682815,171.36424283)(192.59113721,171.77960699)(193.20703125,172.61033794)
\curveto(193.82290681,173.44106367)(194.13084921,174.57257295)(194.13085938,176.00486919)
\curveto(194.13084921,177.42999197)(193.82290681,178.55792053)(193.20703125,179.38865825)
\curveto(192.59113721,180.22653865)(191.75682815,180.64548354)(190.70410156,180.64549419)
\moveto(190.70410156,182.32127544)
\curveto(192.42284311,182.32126312)(193.77277665,181.76266993)(194.75390625,180.64549419)
\curveto(195.73501428,179.52829716)(196.22557368,177.98142371)(196.22558594,176.00486919)
\curveto(196.22557368,174.03546411)(195.73501428,172.48859066)(194.75390625,171.36424419)
\curveto(193.77277665,170.24705644)(192.42284311,169.68846325)(190.70410156,169.68846294)
\curveto(188.97818509,169.68846325)(187.62467082,170.24705644)(186.64355469,171.36424419)
\curveto(185.66959465,172.48859066)(185.18261597,174.03546411)(185.18261719,176.00486919)
\curveto(185.18261597,177.98142371)(185.66959465,179.52829716)(186.64355469,180.64549419)
\curveto(187.62467082,181.76266993)(188.97818509,182.32126312)(190.70410156,182.32127544)
}
}
{
\newrgbcolor{curcolor}{0 0 0}
\pscustom[linestyle=none,fillstyle=solid,fillcolor=curcolor]
{
\newpath
\moveto(206.46289062,180.18358013)
\curveto(206.24087659,180.31247606)(205.99738725,180.40557493)(205.73242188,180.462877)
\curveto(205.47460132,180.5273196)(205.18814327,180.55954613)(204.87304688,180.55955669)
\curveto(203.75585304,180.55954613)(202.8964789,180.19431212)(202.29492188,179.46385356)
\curveto(201.70051655,178.74053753)(201.40331633,177.69854639)(201.40332031,176.337877)
\lineto(201.40332031,169.99998638)
\lineto(199.41601562,169.99998638)
\lineto(199.41601562,182.03123638)
\lineto(201.40332031,182.03123638)
\lineto(201.40332031,180.16209575)
\curveto(201.8186805,180.89255361)(202.35937006,181.43324317)(203.02539062,181.78416606)
\curveto(203.69139998,182.14222684)(204.50064396,182.32126312)(205.453125,182.32127544)
\curveto(205.58918454,182.32126312)(205.73957501,182.31052094)(205.90429688,182.28904888)
\curveto(206.06900177,182.27471369)(206.25161877,182.24964861)(206.45214844,182.21385356)
\lineto(206.46289062,180.18358013)
}
}
{
\newrgbcolor{curcolor}{0 0 0}
\pscustom[linestyle=none,fillstyle=solid,fillcolor=curcolor]
{
\newpath
\moveto(208.55761719,182.03123638)
\lineto(210.53417969,182.03123638)
\lineto(210.53417969,169.99998638)
\lineto(208.55761719,169.99998638)
\lineto(208.55761719,182.03123638)
\moveto(208.55761719,186.71483013)
\lineto(210.53417969,186.71483013)
\lineto(210.53417969,184.21190044)
\lineto(208.55761719,184.21190044)
\lineto(208.55761719,186.71483013)
}
}
{
\newrgbcolor{curcolor}{0 0 0}
\pscustom[linestyle=none,fillstyle=solid,fillcolor=curcolor]
{
\newpath
\moveto(222.32910156,181.67674419)
\lineto(222.32910156,179.80760356)
\curveto(221.77049863,180.0940518)(221.19042108,180.30889534)(220.58886719,180.45213481)
\curveto(219.98729729,180.59535339)(219.36425103,180.6669679)(218.71972656,180.66697856)
\curveto(217.73860162,180.6669679)(217.00097215,180.51657742)(216.50683594,180.21580669)
\curveto(216.01985334,179.91501552)(215.776364,179.4638441)(215.77636719,178.86229106)
\curveto(215.776364,178.40394933)(215.95181955,178.04229604)(216.30273438,177.77733013)
\curveto(216.65364177,177.51951011)(217.35904471,177.27244004)(218.41894531,177.03611919)
\lineto(219.09570312,176.88572856)
\curveto(220.49934104,176.58494073)(221.49478276,176.15883438)(222.08203125,175.60740825)
\curveto(222.6764222,175.06313236)(222.97362242,174.30043781)(222.97363281,173.31932231)
\curveto(222.97362242,172.20213261)(222.52961245,171.31769339)(221.64160156,170.666002)
\curveto(220.76073401,170.01430928)(219.54686804,169.68846325)(218,169.68846294)
\curveto(217.35546398,169.68846325)(216.68228757,169.75291631)(215.98046875,169.88182231)
\curveto(215.28580459,170.0035671)(214.55175585,170.18976483)(213.77832031,170.44041606)
\lineto(213.77832031,172.48143169)
\curveto(214.50878714,172.1018723)(215.22851298,171.81541425)(215.9375,171.62205669)
\curveto(216.64648031,171.43585734)(217.34830253,171.34275847)(218.04296875,171.34275981)
\curveto(218.97395195,171.34275847)(219.69009706,171.5003104)(220.19140625,171.81541606)
\curveto(220.69270023,172.13767955)(220.94335102,172.58885098)(220.94335938,173.16893169)
\curveto(220.94335102,173.70603736)(220.76073401,174.1178208)(220.39550781,174.40428325)
\curveto(220.03742744,174.69073689)(219.24608709,174.96645276)(218.02148438,175.23143169)
\lineto(217.33398438,175.3925645)
\curveto(216.10937148,175.65037135)(215.22493226,176.04425117)(214.68066406,176.57420513)
\curveto(214.13639168,177.11130739)(213.86425653,177.84535614)(213.86425781,178.77635356)
\curveto(213.86425653,179.90785407)(214.2652978,180.78155112)(215.06738281,181.39744731)
\curveto(215.86946286,182.01332072)(217.0081336,182.32126312)(218.48339844,182.32127544)
\curveto(219.21386056,182.32126312)(219.90135987,182.26755223)(220.54589844,182.16014263)
\curveto(221.19042108,182.0527087)(221.78482153,181.89157605)(222.32910156,181.67674419)
}
}
{
\newrgbcolor{curcolor}{0 0 0}
\pscustom[linestyle=none,fillstyle=solid,fillcolor=curcolor]
{
\newpath
\moveto(236.42285156,176.509752)
\lineto(236.42285156,175.54295513)
\lineto(227.33496094,175.54295513)
\curveto(227.42089508,174.18227386)(227.82909779,173.14386344)(228.55957031,172.42772075)
\curveto(229.29719528,171.71873466)(230.3212828,171.36424283)(231.63183594,171.36424419)
\curveto(232.39094219,171.36424283)(233.12499093,171.45734169)(233.83398438,171.64354106)
\curveto(234.55011972,171.82973715)(235.25910338,172.10903375)(235.9609375,172.48143169)
\lineto(235.9609375,170.61229106)
\curveto(235.25194193,170.3115095)(234.52505464,170.08234307)(233.78027344,169.92479106)
\curveto(233.03547279,169.76723921)(232.2799397,169.68846325)(231.51367188,169.68846294)
\curveto(229.59439551,169.68846325)(228.07258713,170.24705644)(226.94824219,171.36424419)
\curveto(225.83105292,172.48142921)(225.27245972,173.99249541)(225.27246094,175.89744731)
\curveto(225.27245972,177.86684049)(225.80240711,179.42803684)(226.86230469,180.58104106)
\curveto(227.92935811,181.74118557)(229.36522907,182.32126312)(231.16992188,182.32127544)
\curveto(232.78840273,182.32126312)(234.06672176,181.79847718)(235.00488281,180.75291606)
\curveto(235.95018342,179.71449489)(236.4228392,178.30010829)(236.42285156,176.509752)
\moveto(234.44628906,177.08983013)
\curveto(234.43195577,178.17120216)(234.1275941,179.03415703)(233.53320312,179.67869731)
\curveto(232.94595465,180.32321824)(232.16535648,180.64548354)(231.19140625,180.64549419)
\curveto(230.08853564,180.64548354)(229.20409642,180.33396042)(228.53808594,179.71092388)
\curveto(227.87922795,179.08786791)(227.49967104,178.21059015)(227.39941406,177.07908794)
\lineto(234.44628906,177.08983013)
}
}
{
\newrgbcolor{curcolor}{0 0 0}
\pscustom[linestyle=none,fillstyle=solid,fillcolor=curcolor]
{
\newpath
\moveto(246.63867188,180.18358013)
\curveto(246.41665784,180.31247606)(246.1731685,180.40557493)(245.90820312,180.462877)
\curveto(245.65038257,180.5273196)(245.36392452,180.55954613)(245.04882812,180.55955669)
\curveto(243.93163429,180.55954613)(243.07226015,180.19431212)(242.47070312,179.46385356)
\curveto(241.8762978,178.74053753)(241.57909758,177.69854639)(241.57910156,176.337877)
\lineto(241.57910156,169.99998638)
\lineto(239.59179688,169.99998638)
\lineto(239.59179688,182.03123638)
\lineto(241.57910156,182.03123638)
\lineto(241.57910156,180.16209575)
\curveto(241.99446175,180.89255361)(242.53515131,181.43324317)(243.20117188,181.78416606)
\curveto(243.86718123,182.14222684)(244.67642521,182.32126312)(245.62890625,182.32127544)
\curveto(245.76496579,182.32126312)(245.91535626,182.31052094)(246.08007812,182.28904888)
\curveto(246.24478302,182.27471369)(246.42740002,182.24964861)(246.62792969,182.21385356)
\lineto(246.63867188,180.18358013)
}
}
{
\newrgbcolor{curcolor}{0 0 0}
\pscustom[linestyle=none,fillstyle=solid,fillcolor=curcolor]
{
}
}
{
\newrgbcolor{curcolor}{0 0 0}
\pscustom[linestyle=none,fillstyle=solid,fillcolor=curcolor]
{
\newpath
\moveto(265.10449219,179.72166606)
\curveto(265.59862088,180.60967629)(266.1894406,181.26494907)(266.87695312,181.68748638)
\curveto(267.56443922,182.11000031)(268.37368321,182.32126312)(269.3046875,182.32127544)
\curveto(270.55792581,182.32126312)(271.52472172,181.88083387)(272.20507812,180.99998638)
\curveto(272.88539745,180.12627833)(273.22556638,178.88018583)(273.22558594,177.26170513)
\lineto(273.22558594,169.99998638)
\lineto(271.23828125,169.99998638)
\lineto(271.23828125,177.197252)
\curveto(271.23826368,178.35023844)(271.03416232,179.20603186)(270.62597656,179.76463481)
\curveto(270.21775688,180.32321824)(269.59471063,180.60251484)(268.75683594,180.60252544)
\curveto(267.73273333,180.60251484)(266.92348934,180.26234591)(266.32910156,179.58201763)
\curveto(265.73468845,178.90167018)(265.43748823,177.97426226)(265.4375,176.79979106)
\lineto(265.4375,169.99998638)
\lineto(263.45019531,169.99998638)
\lineto(263.45019531,177.197252)
\curveto(263.45018553,178.35739989)(263.24608417,179.21319331)(262.83789062,179.76463481)
\curveto(262.42967873,180.32321824)(261.79947103,180.60251484)(260.94726562,180.60252544)
\curveto(259.93749373,180.60251484)(259.1354112,180.25876518)(258.54101562,179.57127544)
\curveto(257.9466103,178.89092801)(257.64941008,177.96710081)(257.64941406,176.79979106)
\lineto(257.64941406,169.99998638)
\lineto(255.66210938,169.99998638)
\lineto(255.66210938,182.03123638)
\lineto(257.64941406,182.03123638)
\lineto(257.64941406,180.16209575)
\curveto(258.1005815,180.89971506)(258.64127106,181.44398535)(259.27148438,181.79490825)
\curveto(259.90168647,182.14580756)(260.65005812,182.32126312)(261.51660156,182.32127544)
\curveto(262.39029075,182.32126312)(263.13150095,182.09925813)(263.74023438,181.65525981)
\curveto(264.3561091,181.21123819)(264.81086125,180.56670758)(265.10449219,179.72166606)
}
}
{
\newrgbcolor{curcolor}{0 0 0}
\pscustom[linestyle=none,fillstyle=solid,fillcolor=curcolor]
{
\newpath
\moveto(281.84082031,180.64549419)
\curveto(280.7809188,180.64548354)(279.94302902,180.23011938)(279.32714844,179.39940044)
\curveto(278.71125942,178.57582416)(278.40331701,177.44431487)(278.40332031,176.00486919)
\curveto(278.40331701,174.5654115)(278.70767869,173.43032149)(279.31640625,172.59959575)
\curveto(279.93228684,171.77602627)(280.77375735,171.36424283)(281.84082031,171.36424419)
\curveto(282.8935469,171.36424283)(283.72785596,171.77960699)(284.34375,172.61033794)
\curveto(284.95962556,173.44106367)(285.26756796,174.57257295)(285.26757812,176.00486919)
\curveto(285.26756796,177.42999197)(284.95962556,178.55792053)(284.34375,179.38865825)
\curveto(283.72785596,180.22653865)(282.8935469,180.64548354)(281.84082031,180.64549419)
\moveto(281.84082031,182.32127544)
\curveto(283.55956186,182.32126312)(284.9094954,181.76266993)(285.890625,180.64549419)
\curveto(286.87173303,179.52829716)(287.36229243,177.98142371)(287.36230469,176.00486919)
\curveto(287.36229243,174.03546411)(286.87173303,172.48859066)(285.890625,171.36424419)
\curveto(284.9094954,170.24705644)(283.55956186,169.68846325)(281.84082031,169.68846294)
\curveto(280.11490384,169.68846325)(278.76138957,170.24705644)(277.78027344,171.36424419)
\curveto(276.8063134,172.48859066)(276.31933472,174.03546411)(276.31933594,176.00486919)
\curveto(276.31933472,177.98142371)(276.8063134,179.52829716)(277.78027344,180.64549419)
\curveto(278.76138957,181.76266993)(280.11490384,182.32126312)(281.84082031,182.32127544)
}
}
{
\newrgbcolor{curcolor}{0 0 0}
\pscustom[linestyle=none,fillstyle=solid,fillcolor=curcolor]
{
\newpath
\moveto(292.58300781,185.447252)
\lineto(292.58300781,182.03123638)
\lineto(296.65429688,182.03123638)
\lineto(296.65429688,180.49510356)
\lineto(292.58300781,180.49510356)
\lineto(292.58300781,173.96385356)
\curveto(292.58300378,172.98273079)(292.71549063,172.35252309)(292.98046875,172.07322856)
\curveto(293.25259947,171.7939299)(293.80045048,171.6542816)(294.62402344,171.65428325)
\lineto(296.65429688,171.65428325)
\lineto(296.65429688,169.99998638)
\lineto(294.62402344,169.99998638)
\curveto(293.09862827,169.99998638)(292.04589495,170.2828637)(291.46582031,170.84861919)
\curveto(290.88573986,171.42153443)(290.59570108,172.45994485)(290.59570312,173.96385356)
\lineto(290.59570312,180.49510356)
\lineto(289.14550781,180.49510356)
\lineto(289.14550781,182.03123638)
\lineto(290.59570312,182.03123638)
\lineto(290.59570312,185.447252)
\lineto(292.58300781,185.447252)
}
}
{
\newrgbcolor{curcolor}{0 0 0}
\pscustom[linestyle=none,fillstyle=solid,fillcolor=curcolor]
{
}
}
{
\newrgbcolor{curcolor}{0 0 0}
\pscustom[linestyle=none,fillstyle=solid,fillcolor=curcolor]
{
\newpath
\moveto(314.18554688,180.2050645)
\lineto(314.18554688,186.71483013)
\lineto(316.16210938,186.71483013)
\lineto(316.16210938,169.99998638)
\lineto(314.18554688,169.99998638)
\lineto(314.18554688,171.80467388)
\curveto(313.77017272,171.08852696)(313.24380606,170.55499884)(312.60644531,170.20408794)
\curveto(311.9762292,169.86033808)(311.21711537,169.68846325)(310.32910156,169.68846294)
\curveto(308.87532084,169.68846325)(307.69010067,170.2685408)(306.7734375,171.42869731)
\curveto(305.86393062,172.58885098)(305.40917847,174.11424008)(305.40917969,176.00486919)
\curveto(305.40917847,177.89548629)(305.86393062,179.42087539)(306.7734375,180.58104106)
\curveto(307.69010067,181.74118557)(308.87532084,182.32126312)(310.32910156,182.32127544)
\curveto(311.21711537,182.32126312)(311.9762292,182.14580756)(312.60644531,181.79490825)
\curveto(313.24380606,181.4511468)(313.77017272,180.92119941)(314.18554688,180.2050645)
\moveto(307.45019531,176.00486919)
\curveto(307.45019206,174.5510886)(307.74739228,173.40883713)(308.34179688,172.57811138)
\curveto(308.94335463,171.75454191)(309.76692151,171.34275847)(310.8125,171.34275981)
\curveto(311.85806525,171.34275847)(312.68163214,171.75454191)(313.28320312,172.57811138)
\curveto(313.88475594,173.40883713)(314.18553688,174.5510886)(314.18554688,176.00486919)
\curveto(314.18553688,177.45863777)(313.88475594,178.59730851)(313.28320312,179.42088481)
\curveto(312.68163214,180.25160373)(311.85806525,180.6669679)(310.8125,180.66697856)
\curveto(309.76692151,180.6669679)(308.94335463,180.25160373)(308.34179688,179.42088481)
\curveto(307.74739228,178.59730851)(307.45019206,177.45863777)(307.45019531,176.00486919)
}
}
{
\newrgbcolor{curcolor}{0 0 0}
\pscustom[linestyle=none,fillstyle=solid,fillcolor=curcolor]
{
\newpath
\moveto(330.52441406,176.509752)
\lineto(330.52441406,175.54295513)
\lineto(321.43652344,175.54295513)
\curveto(321.52245758,174.18227386)(321.93066029,173.14386344)(322.66113281,172.42772075)
\curveto(323.39875778,171.71873466)(324.4228453,171.36424283)(325.73339844,171.36424419)
\curveto(326.49250469,171.36424283)(327.22655343,171.45734169)(327.93554688,171.64354106)
\curveto(328.65168222,171.82973715)(329.36066588,172.10903375)(330.0625,172.48143169)
\lineto(330.0625,170.61229106)
\curveto(329.35350443,170.3115095)(328.62661714,170.08234307)(327.88183594,169.92479106)
\curveto(327.13703529,169.76723921)(326.3815022,169.68846325)(325.61523438,169.68846294)
\curveto(323.69595801,169.68846325)(322.17414963,170.24705644)(321.04980469,171.36424419)
\curveto(319.93261542,172.48142921)(319.37402222,173.99249541)(319.37402344,175.89744731)
\curveto(319.37402222,177.86684049)(319.90396961,179.42803684)(320.96386719,180.58104106)
\curveto(322.03092061,181.74118557)(323.46679157,182.32126312)(325.27148438,182.32127544)
\curveto(326.88996523,182.32126312)(328.16828426,181.79847718)(329.10644531,180.75291606)
\curveto(330.05174592,179.71449489)(330.5244017,178.30010829)(330.52441406,176.509752)
\moveto(328.54785156,177.08983013)
\curveto(328.53351827,178.17120216)(328.2291566,179.03415703)(327.63476562,179.67869731)
\curveto(327.04751715,180.32321824)(326.26691898,180.64548354)(325.29296875,180.64549419)
\curveto(324.19009814,180.64548354)(323.30565892,180.33396042)(322.63964844,179.71092388)
\curveto(321.98079045,179.08786791)(321.60123354,178.21059015)(321.50097656,177.07908794)
\lineto(328.54785156,177.08983013)
}
}
{
\newrgbcolor{curcolor}{0 0 0}
\pscustom[linestyle=none,fillstyle=solid,fillcolor=curcolor]
{
}
}
{
\newrgbcolor{curcolor}{0 0 0}
\pscustom[linestyle=none,fillstyle=solid,fillcolor=curcolor]
{
\newpath
\moveto(342.68457031,171.80467388)
\lineto(342.68457031,165.4238145)
\lineto(340.69726562,165.4238145)
\lineto(340.69726562,182.03123638)
\lineto(342.68457031,182.03123638)
\lineto(342.68457031,180.2050645)
\curveto(343.0999305,180.92119941)(343.62271643,181.4511468)(344.25292969,181.79490825)
\curveto(344.89029329,182.14580756)(345.64940711,182.32126312)(346.53027344,182.32127544)
\curveto(347.99120165,182.32126312)(349.17642181,181.74118557)(350.0859375,180.58104106)
\curveto(351.00259186,179.42087539)(351.46092474,177.89548629)(351.4609375,176.00486919)
\curveto(351.46092474,174.11424008)(351.00259186,172.58885098)(350.0859375,171.42869731)
\curveto(349.17642181,170.2685408)(347.99120165,169.68846325)(346.53027344,169.68846294)
\curveto(345.64940711,169.68846325)(344.89029329,169.86033808)(344.25292969,170.20408794)
\curveto(343.62271643,170.55499884)(343.0999305,171.08852696)(342.68457031,171.80467388)
\moveto(349.40917969,176.00486919)
\curveto(349.40916898,177.45863777)(349.10838803,178.59730851)(348.50683594,179.42088481)
\curveto(347.91242568,180.25160373)(347.09243952,180.6669679)(346.046875,180.66697856)
\curveto(345.00129578,180.6669679)(344.1777289,180.25160373)(343.57617188,179.42088481)
\curveto(342.98176655,178.59730851)(342.68456633,177.45863777)(342.68457031,176.00486919)
\curveto(342.68456633,174.5510886)(342.98176655,173.40883713)(343.57617188,172.57811138)
\curveto(344.1777289,171.75454191)(345.00129578,171.34275847)(346.046875,171.34275981)
\curveto(347.09243952,171.34275847)(347.91242568,171.75454191)(348.50683594,172.57811138)
\curveto(349.10838803,173.40883713)(349.40916898,174.5510886)(349.40917969,176.00486919)
}
}
{
\newrgbcolor{curcolor}{0 0 0}
\pscustom[linestyle=none,fillstyle=solid,fillcolor=curcolor]
{
\newpath
\moveto(360.20507812,176.04783794)
\curveto(358.60806697,176.04783189)(357.50162277,175.86521489)(356.88574219,175.49998638)
\curveto(356.26985317,175.13474687)(355.96191076,174.51170062)(355.96191406,173.63084575)
\curveto(355.96191076,172.92901991)(356.1910772,172.37042671)(356.64941406,171.9550645)
\curveto(357.1149044,171.54685983)(357.74511211,171.34275847)(358.54003906,171.34275981)
\curveto(359.63573522,171.34275847)(360.51301298,171.72947683)(361.171875,172.50291606)
\curveto(361.83788145,173.28351174)(362.17088893,174.31834143)(362.17089844,175.60740825)
\lineto(362.17089844,176.04783794)
\lineto(360.20507812,176.04783794)
\moveto(364.14746094,176.86424419)
\lineto(364.14746094,169.99998638)
\lineto(362.17089844,169.99998638)
\lineto(362.17089844,171.82615825)
\curveto(361.71971751,171.09568841)(361.15754359,170.55499884)(360.484375,170.20408794)
\curveto(359.81119077,169.86033808)(358.98762388,169.68846325)(358.01367188,169.68846294)
\curveto(356.78189692,169.68846325)(355.80077811,170.03221291)(355.0703125,170.71971294)
\curveto(354.34700353,171.41437298)(353.98535024,172.34178091)(353.98535156,173.5019395)
\curveto(353.98535024,174.85545027)(354.43652167,175.87595706)(355.33886719,176.56346294)
\curveto(356.24836881,177.25095569)(357.60188308,177.59470534)(359.39941406,177.59471294)
\lineto(362.17089844,177.59471294)
\lineto(362.17089844,177.78807231)
\curveto(362.17088893,178.69756883)(361.87010798,179.39939104)(361.26855469,179.89354106)
\curveto(360.67414564,180.39483275)(359.83625585,180.64548354)(358.75488281,180.64549419)
\curveto(358.06737741,180.64548354)(357.39778172,180.56312686)(356.74609375,180.39842388)
\curveto(356.09439761,180.2337001)(355.46777063,179.98663004)(354.86621094,179.65721294)
\lineto(354.86621094,181.48338481)
\curveto(355.5895153,181.76266993)(356.29133752,181.97035201)(356.97167969,182.10643169)
\curveto(357.65201324,182.24964861)(358.31444747,182.32126312)(358.95898438,182.32127544)
\curveto(360.69921071,182.32126312)(361.9990141,181.87009169)(362.85839844,180.96775981)
\curveto(363.71776238,180.065406)(364.14744945,178.69756883)(364.14746094,176.86424419)
}
}
{
\newrgbcolor{curcolor}{0 0 0}
\pscustom[linestyle=none,fillstyle=solid,fillcolor=curcolor]
{
\newpath
\moveto(375.89941406,181.67674419)
\lineto(375.89941406,179.80760356)
\curveto(375.34081113,180.0940518)(374.76073358,180.30889534)(374.15917969,180.45213481)
\curveto(373.55760979,180.59535339)(372.93456353,180.6669679)(372.29003906,180.66697856)
\curveto(371.30891412,180.6669679)(370.57128465,180.51657742)(370.07714844,180.21580669)
\curveto(369.59016584,179.91501552)(369.3466765,179.4638441)(369.34667969,178.86229106)
\curveto(369.3466765,178.40394933)(369.52213205,178.04229604)(369.87304688,177.77733013)
\curveto(370.22395427,177.51951011)(370.92935721,177.27244004)(371.98925781,177.03611919)
\lineto(372.66601562,176.88572856)
\curveto(374.06965354,176.58494073)(375.06509526,176.15883438)(375.65234375,175.60740825)
\curveto(376.2467347,175.06313236)(376.54393492,174.30043781)(376.54394531,173.31932231)
\curveto(376.54393492,172.20213261)(376.09992495,171.31769339)(375.21191406,170.666002)
\curveto(374.33104651,170.01430928)(373.11718054,169.68846325)(371.5703125,169.68846294)
\curveto(370.92577648,169.68846325)(370.25260007,169.75291631)(369.55078125,169.88182231)
\curveto(368.85611709,170.0035671)(368.12206835,170.18976483)(367.34863281,170.44041606)
\lineto(367.34863281,172.48143169)
\curveto(368.07909964,172.1018723)(368.79882548,171.81541425)(369.5078125,171.62205669)
\curveto(370.21679281,171.43585734)(370.91861503,171.34275847)(371.61328125,171.34275981)
\curveto(372.54426445,171.34275847)(373.26040956,171.5003104)(373.76171875,171.81541606)
\curveto(374.26301273,172.13767955)(374.51366352,172.58885098)(374.51367188,173.16893169)
\curveto(374.51366352,173.70603736)(374.33104651,174.1178208)(373.96582031,174.40428325)
\curveto(373.60773994,174.69073689)(372.81639959,174.96645276)(371.59179688,175.23143169)
\lineto(370.90429688,175.3925645)
\curveto(369.67968398,175.65037135)(368.79524476,176.04425117)(368.25097656,176.57420513)
\curveto(367.70670418,177.11130739)(367.43456903,177.84535614)(367.43457031,178.77635356)
\curveto(367.43456903,179.90785407)(367.8356103,180.78155112)(368.63769531,181.39744731)
\curveto(369.43977536,182.01332072)(370.5784461,182.32126312)(372.05371094,182.32127544)
\curveto(372.78417306,182.32126312)(373.47167237,182.26755223)(374.11621094,182.16014263)
\curveto(374.76073358,182.0527087)(375.35513403,181.89157605)(375.89941406,181.67674419)
}
}
{
\newrgbcolor{curcolor}{0 0 0}
\pscustom[linestyle=none,fillstyle=solid,fillcolor=curcolor]
{
\newpath
\moveto(387.37207031,181.67674419)
\lineto(387.37207031,179.80760356)
\curveto(386.81346738,180.0940518)(386.23338983,180.30889534)(385.63183594,180.45213481)
\curveto(385.03026604,180.59535339)(384.40721978,180.6669679)(383.76269531,180.66697856)
\curveto(382.78157037,180.6669679)(382.0439409,180.51657742)(381.54980469,180.21580669)
\curveto(381.06282209,179.91501552)(380.81933275,179.4638441)(380.81933594,178.86229106)
\curveto(380.81933275,178.40394933)(380.9947883,178.04229604)(381.34570312,177.77733013)
\curveto(381.69661052,177.51951011)(382.40201346,177.27244004)(383.46191406,177.03611919)
\lineto(384.13867188,176.88572856)
\curveto(385.54230979,176.58494073)(386.53775151,176.15883438)(387.125,175.60740825)
\curveto(387.71939095,175.06313236)(388.01659117,174.30043781)(388.01660156,173.31932231)
\curveto(388.01659117,172.20213261)(387.5725812,171.31769339)(386.68457031,170.666002)
\curveto(385.80370276,170.01430928)(384.58983679,169.68846325)(383.04296875,169.68846294)
\curveto(382.39843273,169.68846325)(381.72525632,169.75291631)(381.0234375,169.88182231)
\curveto(380.32877334,170.0035671)(379.5947246,170.18976483)(378.82128906,170.44041606)
\lineto(378.82128906,172.48143169)
\curveto(379.55175589,172.1018723)(380.27148173,171.81541425)(380.98046875,171.62205669)
\curveto(381.68944906,171.43585734)(382.39127128,171.34275847)(383.0859375,171.34275981)
\curveto(384.0169207,171.34275847)(384.73306581,171.5003104)(385.234375,171.81541606)
\curveto(385.73566898,172.13767955)(385.98631977,172.58885098)(385.98632812,173.16893169)
\curveto(385.98631977,173.70603736)(385.80370276,174.1178208)(385.43847656,174.40428325)
\curveto(385.08039619,174.69073689)(384.28905584,174.96645276)(383.06445312,175.23143169)
\lineto(382.37695312,175.3925645)
\curveto(381.15234023,175.65037135)(380.26790101,176.04425117)(379.72363281,176.57420513)
\curveto(379.17936043,177.11130739)(378.90722528,177.84535614)(378.90722656,178.77635356)
\curveto(378.90722528,179.90785407)(379.30826655,180.78155112)(380.11035156,181.39744731)
\curveto(380.91243161,182.01332072)(382.05110235,182.32126312)(383.52636719,182.32127544)
\curveto(384.25682931,182.32126312)(384.94432862,182.26755223)(385.58886719,182.16014263)
\curveto(386.23338983,182.0527087)(386.82779028,181.89157605)(387.37207031,181.67674419)
}
}
{
\newrgbcolor{curcolor}{0 0 0}
\pscustom[linestyle=none,fillstyle=solid,fillcolor=curcolor]
{
\newpath
\moveto(401.46582031,176.509752)
\lineto(401.46582031,175.54295513)
\lineto(392.37792969,175.54295513)
\curveto(392.46386383,174.18227386)(392.87206654,173.14386344)(393.60253906,172.42772075)
\curveto(394.34016403,171.71873466)(395.36425155,171.36424283)(396.67480469,171.36424419)
\curveto(397.43391094,171.36424283)(398.16795968,171.45734169)(398.87695312,171.64354106)
\curveto(399.59308847,171.82973715)(400.30207213,172.10903375)(401.00390625,172.48143169)
\lineto(401.00390625,170.61229106)
\curveto(400.29491068,170.3115095)(399.56802339,170.08234307)(398.82324219,169.92479106)
\curveto(398.07844154,169.76723921)(397.32290845,169.68846325)(396.55664062,169.68846294)
\curveto(394.63736426,169.68846325)(393.11555588,170.24705644)(391.99121094,171.36424419)
\curveto(390.87402167,172.48142921)(390.31542847,173.99249541)(390.31542969,175.89744731)
\curveto(390.31542847,177.86684049)(390.84537586,179.42803684)(391.90527344,180.58104106)
\curveto(392.97232686,181.74118557)(394.40819782,182.32126312)(396.21289062,182.32127544)
\curveto(397.83137148,182.32126312)(399.10969051,181.79847718)(400.04785156,180.75291606)
\curveto(400.99315217,179.71449489)(401.46580795,178.30010829)(401.46582031,176.509752)
\moveto(399.48925781,177.08983013)
\curveto(399.47492452,178.17120216)(399.17056285,179.03415703)(398.57617188,179.67869731)
\curveto(397.9889234,180.32321824)(397.20832523,180.64548354)(396.234375,180.64549419)
\curveto(395.13150439,180.64548354)(394.24706517,180.33396042)(393.58105469,179.71092388)
\curveto(392.9221967,179.08786791)(392.54263979,178.21059015)(392.44238281,177.07908794)
\lineto(399.48925781,177.08983013)
}
}
\end{pspicture}
		
		\end{center}
		
		\begin{enumerate}
		  \item Fond d'écran
		  \item Pseudonyme du compte hors ligne utilisé
		  \item Bouton ``\hyperlink{Creation compte multi-joueurs}{Inscription}''
		  \item Cadre contenant l'erreur survenue
		  \item Champs de texte ``Nom utilisateur''
		  \item Champs de texte ``Mot de passe''
		  \item Check Box ``Connexion auto''
		  \item Bouton ``\hyperlink{Page d'accueil}{Retour}''
		  \item Bouton ``\hyperlink{Accueil multi-joueurs}{Connexion}''
		\end{enumerate}

		\subsubsection{Description des zones}
		
			\begin{tabular}{|c|c|c|c|c|} \hline
				Numéro de zone & Type  & Description & Evènement &	Règle \\\hline
				2 & Label & Affiche le pseudonyme du compte & Chargement de la page & RG5-01 \\
				  &       & hors ligne en cours d'utilisation & & \\\hline
				3 & Bouton & Permet à l'utilisateur d'être redirigé & Cliqué & RG5-02\\
				  &        & sur la page de création  & & \\
				  &        & s d'un compte multi-joueurs & & \\\hline
				4 & Label & Affiche l'erreur & Serveur distant & RG5-03 \\\hline
				7 & Check box & Permet la connexion automatique & Cliqué & RG5-04 \\
				  &           & lors des prochaines utilisations&        & \\				
				  &           & du multi-joueurs                &        & \\\hline
				8 & Bouton & Permet de revenir à la page & Cliqué & RG5-05 \\
				  &        & de connexion multi-joueurs \footnotemark[1] & & \\\hline
				9 & Bouton & Valide les paramètres entrés & Cliqué & RG5-06 \\\hline
			\end{tabular}
			
		\subsubsection{Description des règles}

			\underline{RG5-01 :}
				\begin{quote}
				
				\end{quote}		
		
\newpage

	\subsection{Accueil multi-joueurs}

		\hypertarget{Accueil multi-joueurs}{}
		\label{Accueil multi-joueurs}
	
		\begin{center}
			%LaTeX with PSTricks extensions
%%Creator: inkscape 0.48.0
%%Please note this file requires PSTricks extensions
\psset{xunit=.4pt,yunit=.4pt,runit=.4pt}
\begin{pspicture}(560,600)
{
\newrgbcolor{curcolor}{1 1 1}
\pscustom[linestyle=none,fillstyle=solid,fillcolor=curcolor]
{
\newpath
\moveto(133.12401581,597.52220317)
\lineto(426.87598419,597.52220317)
\curveto(443.85397169,597.52220317)(457.52217102,583.85400385)(457.52217102,566.87601635)
\lineto(457.52217102,33.12401744)
\curveto(457.52217102,16.14602994)(443.85397169,2.47783062)(426.87598419,2.47783062)
\lineto(133.12401581,2.47783062)
\curveto(116.14602831,2.47783062)(102.47782898,16.14602994)(102.47782898,33.12401744)
\lineto(102.47782898,566.87601635)
\curveto(102.47782898,583.85400385)(116.14602831,597.52220317)(133.12401581,597.52220317)
\closepath
}
}
{
\newrgbcolor{curcolor}{0 0 0}
\pscustom[linewidth=4.95566034,linecolor=curcolor]
{
\newpath
\moveto(133.12401581,597.52220317)
\lineto(426.87598419,597.52220317)
\curveto(443.85397169,597.52220317)(457.52217102,583.85400385)(457.52217102,566.87601635)
\lineto(457.52217102,33.12401744)
\curveto(457.52217102,16.14602994)(443.85397169,2.47783062)(426.87598419,2.47783062)
\lineto(133.12401581,2.47783062)
\curveto(116.14602831,2.47783062)(102.47782898,16.14602994)(102.47782898,33.12401744)
\lineto(102.47782898,566.87601635)
\curveto(102.47782898,583.85400385)(116.14602831,597.52220317)(133.12401581,597.52220317)
\closepath
}
}
{
\newrgbcolor{curcolor}{1 1 1}
\pscustom[linestyle=none,fillstyle=solid,fillcolor=curcolor]
{
\newpath
\moveto(284.99628735,49.69105693)
\lineto(425.16983509,49.69105693)
\curveto(433.42142812,49.69105693)(440.06440735,43.04807771)(440.06440735,34.79648468)
\curveto(440.06440735,26.54489164)(433.42142812,19.90191242)(425.16983509,19.90191242)
\lineto(284.99628735,19.90191242)
\curveto(276.74469431,19.90191242)(270.10171509,26.54489164)(270.10171509,34.79648468)
\curveto(270.10171509,43.04807771)(276.74469431,49.69105693)(284.99628735,49.69105693)
\closepath
}
}
{
\newrgbcolor{curcolor}{0 0 0}
\pscustom[linewidth=2,linecolor=curcolor]
{
\newpath
\moveto(284.99628735,49.69105693)
\lineto(425.16983509,49.69105693)
\curveto(433.42142812,49.69105693)(440.06440735,43.04807771)(440.06440735,34.79648468)
\curveto(440.06440735,26.54489164)(433.42142812,19.90191242)(425.16983509,19.90191242)
\lineto(284.99628735,19.90191242)
\curveto(276.74469431,19.90191242)(270.10171509,26.54489164)(270.10171509,34.79648468)
\curveto(270.10171509,43.04807771)(276.74469431,49.69105693)(284.99628735,49.69105693)
\closepath
}
}
{
\newrgbcolor{curcolor}{0 0 0}
\pscustom[linestyle=none,fillstyle=solid,fillcolor=curcolor]
{
\newpath
\moveto(295.45703125,46.14842388)
\lineto(295.45703125,43.65233013)
\curveto(294.66014159,44.39450323)(293.80857994,44.94919018)(292.90234375,45.31639263)
\curveto(292.00389425,45.68356444)(291.04686395,45.86715801)(290.03125,45.86717388)
\curveto(288.03124197,45.86715801)(286.4999935,45.25387737)(285.4375,44.02733013)
\curveto(284.37499562,42.80856732)(283.84374616,41.04294408)(283.84375,38.73045513)
\curveto(283.84374616,36.4257612)(284.37499562,34.66013797)(285.4375,33.43358013)
\curveto(286.4999935,32.21482791)(288.03124197,31.60545352)(290.03125,31.60545513)
\curveto(291.04686395,31.60545352)(292.00389425,31.78904709)(292.90234375,32.15623638)
\curveto(293.80857994,32.52342135)(294.66014159,33.0781083)(295.45703125,33.82029888)
\lineto(295.45703125,31.34764263)
\curveto(294.62889162,30.78514184)(293.74998625,30.36326726)(292.8203125,30.08201763)
\curveto(291.8984256,29.80076783)(290.92186408,29.66014297)(289.890625,29.66014263)
\curveto(287.24218026,29.66014297)(285.15624484,30.46873591)(283.6328125,32.08592388)
\curveto(282.10937289,33.71092017)(281.3476549,35.9257617)(281.34765625,38.73045513)
\curveto(281.3476549,41.54294358)(282.10937289,43.75778512)(283.6328125,45.37498638)
\curveto(285.15624484,46.99996938)(287.24218026,47.81246856)(289.890625,47.81248638)
\curveto(290.93748906,47.81246856)(291.92186308,47.67184371)(292.84375,47.39061138)
\curveto(293.77342373,47.11715676)(294.64451661,46.70309467)(295.45703125,46.14842388)
}
}
{
\newrgbcolor{curcolor}{0 0 0}
\pscustom[linestyle=none,fillstyle=solid,fillcolor=curcolor]
{
\newpath
\moveto(306.6484375,41.10936138)
\curveto(306.40624037,41.24997513)(306.14061564,41.35153753)(305.8515625,41.41404888)
\curveto(305.57030371,41.48434989)(305.25780402,41.51950611)(304.9140625,41.51951763)
\curveto(303.69530559,41.51950611)(302.75780652,41.12106901)(302.1015625,40.32420513)
\curveto(301.45312033,39.53513309)(301.1289019,38.39841548)(301.12890625,36.91404888)
\lineto(301.12890625,29.99998638)
\lineto(298.9609375,29.99998638)
\lineto(298.9609375,43.12498638)
\lineto(301.12890625,43.12498638)
\lineto(301.12890625,41.08592388)
\curveto(301.58202645,41.88278699)(302.17186961,42.47263015)(302.8984375,42.85545513)
\curveto(303.62499316,43.24606688)(304.50780477,43.44137919)(305.546875,43.44139263)
\curveto(305.69530359,43.44137919)(305.85936592,43.42966045)(306.0390625,43.40623638)
\curveto(306.21874056,43.39059799)(306.41795911,43.36325426)(306.63671875,43.32420513)
\lineto(306.6484375,41.10936138)
}
}
{
\newrgbcolor{curcolor}{0 0 0}
\pscustom[linestyle=none,fillstyle=solid,fillcolor=curcolor]
{
\newpath
\moveto(319.64453125,37.10154888)
\lineto(319.64453125,36.04686138)
\lineto(309.73046875,36.04686138)
\curveto(309.82421508,34.56248181)(310.26952714,33.42967045)(311.06640625,32.64842388)
\curveto(311.87108804,31.8749845)(312.98827442,31.48826614)(314.41796875,31.48826763)
\curveto(315.24608466,31.48826614)(316.04686511,31.58982854)(316.8203125,31.79295513)
\curveto(317.60155105,31.99607813)(318.37498778,32.30076533)(319.140625,32.70701763)
\lineto(319.140625,30.66795513)
\curveto(318.36717529,30.33982979)(317.57420733,30.08983004)(316.76171875,29.91795513)
\curveto(315.94920896,29.74608038)(315.12499103,29.66014297)(314.2890625,29.66014263)
\curveto(312.19530646,29.66014297)(310.53515187,30.26951736)(309.30859375,31.48826763)
\curveto(308.08984182,32.70701492)(307.48046743,34.35545077)(307.48046875,36.43358013)
\curveto(307.48046743,38.58200904)(308.05859185,40.28513234)(309.21484375,41.54295513)
\curveto(310.37890203,42.80856732)(311.94530671,43.44137919)(313.9140625,43.44139263)
\curveto(315.67967798,43.44137919)(317.07420783,42.87106726)(318.09765625,41.73045513)
\curveto(319.12889328,40.59763203)(319.64451776,39.05466482)(319.64453125,37.10154888)
\moveto(317.48828125,37.73436138)
\curveto(317.47264493,38.91403996)(317.14061402,39.85544527)(316.4921875,40.55858013)
\curveto(315.8515528,41.26169387)(314.99999116,41.61325601)(313.9375,41.61326763)
\curveto(312.73436842,41.61325601)(311.76952564,41.2734126)(311.04296875,40.59373638)
\curveto(310.32421458,39.91403896)(309.9101525,38.95700867)(309.80078125,37.72264263)
\lineto(317.48828125,37.73436138)
\moveto(315.4140625,49.19529888)
\lineto(317.74609375,49.19529888)
\lineto(313.92578125,44.78904888)
\lineto(312.1328125,44.78904888)
\lineto(315.4140625,49.19529888)
}
}
{
\newrgbcolor{curcolor}{0 0 0}
\pscustom[linestyle=none,fillstyle=solid,fillcolor=curcolor]
{
\newpath
\moveto(334.41015625,37.10154888)
\lineto(334.41015625,36.04686138)
\lineto(324.49609375,36.04686138)
\curveto(324.58984008,34.56248181)(325.03515214,33.42967045)(325.83203125,32.64842388)
\curveto(326.63671304,31.8749845)(327.75389942,31.48826614)(329.18359375,31.48826763)
\curveto(330.01170966,31.48826614)(330.81249011,31.58982854)(331.5859375,31.79295513)
\curveto(332.36717605,31.99607813)(333.14061278,32.30076533)(333.90625,32.70701763)
\lineto(333.90625,30.66795513)
\curveto(333.13280029,30.33982979)(332.33983233,30.08983004)(331.52734375,29.91795513)
\curveto(330.71483396,29.74608038)(329.89061603,29.66014297)(329.0546875,29.66014263)
\curveto(326.96093146,29.66014297)(325.30077687,30.26951736)(324.07421875,31.48826763)
\curveto(322.85546682,32.70701492)(322.24609243,34.35545077)(322.24609375,36.43358013)
\curveto(322.24609243,38.58200904)(322.82421685,40.28513234)(323.98046875,41.54295513)
\curveto(325.14452703,42.80856732)(326.71093171,43.44137919)(328.6796875,43.44139263)
\curveto(330.44530298,43.44137919)(331.83983283,42.87106726)(332.86328125,41.73045513)
\curveto(333.89451828,40.59763203)(334.41014276,39.05466482)(334.41015625,37.10154888)
\moveto(332.25390625,37.73436138)
\curveto(332.23826993,38.91403996)(331.90623902,39.85544527)(331.2578125,40.55858013)
\curveto(330.6171778,41.26169387)(329.76561616,41.61325601)(328.703125,41.61326763)
\curveto(327.49999342,41.61325601)(326.53515064,41.2734126)(325.80859375,40.59373638)
\curveto(325.08983958,39.91403896)(324.6757775,38.95700867)(324.56640625,37.72264263)
\lineto(332.25390625,37.73436138)
}
}
{
\newrgbcolor{curcolor}{0 0 0}
\pscustom[linestyle=none,fillstyle=solid,fillcolor=curcolor]
{
\newpath
\moveto(345.5546875,41.10936138)
\curveto(345.31249037,41.24997513)(345.04686564,41.35153753)(344.7578125,41.41404888)
\curveto(344.47655371,41.48434989)(344.16405402,41.51950611)(343.8203125,41.51951763)
\curveto(342.60155559,41.51950611)(341.66405652,41.12106901)(341.0078125,40.32420513)
\curveto(340.35937033,39.53513309)(340.0351519,38.39841548)(340.03515625,36.91404888)
\lineto(340.03515625,29.99998638)
\lineto(337.8671875,29.99998638)
\lineto(337.8671875,43.12498638)
\lineto(340.03515625,43.12498638)
\lineto(340.03515625,41.08592388)
\curveto(340.48827645,41.88278699)(341.07811961,42.47263015)(341.8046875,42.85545513)
\curveto(342.53124316,43.24606688)(343.41405477,43.44137919)(344.453125,43.44139263)
\curveto(344.60155359,43.44137919)(344.76561592,43.42966045)(344.9453125,43.40623638)
\curveto(345.12499056,43.39059799)(345.32420911,43.36325426)(345.54296875,43.32420513)
\lineto(345.5546875,41.10936138)
}
}
{
\newrgbcolor{curcolor}{0 0 0}
\pscustom[linestyle=none,fillstyle=solid,fillcolor=curcolor]
{
}
}
{
\newrgbcolor{curcolor}{0 0 0}
\pscustom[linestyle=none,fillstyle=solid,fillcolor=curcolor]
{
\newpath
\moveto(357.56640625,31.96873638)
\lineto(357.56640625,25.00779888)
\lineto(355.3984375,25.00779888)
\lineto(355.3984375,43.12498638)
\lineto(357.56640625,43.12498638)
\lineto(357.56640625,41.13279888)
\curveto(358.01952645,41.91403696)(358.58983838,42.49216138)(359.27734375,42.86717388)
\curveto(359.9726495,43.24997313)(360.80077367,43.44137919)(361.76171875,43.44139263)
\curveto(363.35545861,43.44137919)(364.64842607,42.80856732)(365.640625,41.54295513)
\curveto(366.64061158,40.27731985)(367.14061108,38.61325901)(367.140625,36.55076763)
\curveto(367.14061108,34.48826314)(366.64061158,32.8242023)(365.640625,31.55858013)
\curveto(364.64842607,30.29295483)(363.35545861,29.66014297)(361.76171875,29.66014263)
\curveto(360.80077367,29.66014297)(359.9726495,29.84764278)(359.27734375,30.22264263)
\curveto(358.58983838,30.60545452)(358.01952645,31.18748519)(357.56640625,31.96873638)
\moveto(364.90234375,36.55076763)
\curveto(364.90233207,38.13669699)(364.57420739,39.37888325)(363.91796875,40.27733013)
\curveto(363.2695212,41.18356894)(362.37499084,41.63669349)(361.234375,41.63670513)
\curveto(360.09374313,41.63669349)(359.19530652,41.18356894)(358.5390625,40.27733013)
\curveto(357.89062033,39.37888325)(357.5664019,38.13669699)(357.56640625,36.55076763)
\curveto(357.5664019,34.96482516)(357.89062033,33.71873266)(358.5390625,32.81248638)
\curveto(359.19530652,31.91404696)(360.09374313,31.46482866)(361.234375,31.46483013)
\curveto(362.37499084,31.46482866)(363.2695212,31.91404696)(363.91796875,32.81248638)
\curveto(364.57420739,33.71873266)(364.90233207,34.96482516)(364.90234375,36.55076763)
}
}
{
\newrgbcolor{curcolor}{0 0 0}
\pscustom[linestyle=none,fillstyle=solid,fillcolor=curcolor]
{
\newpath
\moveto(376.6796875,36.59764263)
\curveto(374.93749352,36.59763603)(373.73046347,36.39841748)(373.05859375,35.99998638)
\curveto(372.38671482,35.60154328)(372.05077765,34.92185646)(372.05078125,33.96092388)
\curveto(372.05077765,33.19529568)(372.3007774,32.58592129)(372.80078125,32.13279888)
\curveto(373.30858889,31.68748469)(373.99608821,31.46482866)(374.86328125,31.46483013)
\curveto(376.05858614,31.46482866)(377.01561644,31.88670324)(377.734375,32.73045513)
\curveto(378.46092749,33.58201404)(378.82420838,34.71091917)(378.82421875,36.11717388)
\lineto(378.82421875,36.59764263)
\lineto(376.6796875,36.59764263)
\moveto(380.98046875,37.48826763)
\lineto(380.98046875,29.99998638)
\lineto(378.82421875,29.99998638)
\lineto(378.82421875,31.99217388)
\curveto(378.33202137,31.19529768)(377.71874073,30.60545452)(376.984375,30.22264263)
\curveto(376.2499922,29.84764278)(375.3515556,29.66014297)(374.2890625,29.66014263)
\curveto(372.94530801,29.66014297)(371.87499658,30.03514259)(371.078125,30.78514263)
\curveto(370.28906066,31.54295358)(369.89452981,32.55467132)(369.89453125,33.82029888)
\curveto(369.89452981,35.29685608)(370.38671682,36.41013622)(371.37109375,37.16014263)
\curveto(372.36327734,37.91013472)(373.83983836,38.28513434)(375.80078125,38.28514263)
\lineto(378.82421875,38.28514263)
\lineto(378.82421875,38.49608013)
\curveto(378.82420838,39.48825814)(378.49608371,40.25388237)(377.83984375,40.79295513)
\curveto(377.19139751,41.33981879)(376.27733593,41.61325601)(375.09765625,41.61326763)
\curveto(374.34765036,41.61325601)(373.61718234,41.52341235)(372.90625,41.34373638)
\curveto(372.19530876,41.16403771)(371.51171569,40.89450673)(370.85546875,40.53514263)
\lineto(370.85546875,42.52733013)
\curveto(371.64452806,42.83200479)(372.41015229,43.05856707)(373.15234375,43.20701763)
\curveto(373.89452581,43.36325426)(374.61718134,43.44137919)(375.3203125,43.44139263)
\curveto(377.21874123,43.44137919)(378.63670857,42.94919218)(379.57421875,41.96483013)
\curveto(380.51170669,40.98044415)(380.98045622,39.48825814)(380.98046875,37.48826763)
}
}
{
\newrgbcolor{curcolor}{0 0 0}
\pscustom[linestyle=none,fillstyle=solid,fillcolor=curcolor]
{
\newpath
\moveto(393.0390625,41.10936138)
\curveto(392.79686537,41.24997513)(392.53124064,41.35153753)(392.2421875,41.41404888)
\curveto(391.96092871,41.48434989)(391.64842902,41.51950611)(391.3046875,41.51951763)
\curveto(390.08593059,41.51950611)(389.14843152,41.12106901)(388.4921875,40.32420513)
\curveto(387.84374533,39.53513309)(387.5195269,38.39841548)(387.51953125,36.91404888)
\lineto(387.51953125,29.99998638)
\lineto(385.3515625,29.99998638)
\lineto(385.3515625,43.12498638)
\lineto(387.51953125,43.12498638)
\lineto(387.51953125,41.08592388)
\curveto(387.97265145,41.88278699)(388.56249461,42.47263015)(389.2890625,42.85545513)
\curveto(390.01561816,43.24606688)(390.89842977,43.44137919)(391.9375,43.44139263)
\curveto(392.08592859,43.44137919)(392.24999092,43.42966045)(392.4296875,43.40623638)
\curveto(392.60936556,43.39059799)(392.80858411,43.36325426)(393.02734375,43.32420513)
\lineto(393.0390625,41.10936138)
}
}
{
\newrgbcolor{curcolor}{0 0 0}
\pscustom[linestyle=none,fillstyle=solid,fillcolor=curcolor]
{
\newpath
\moveto(397.45703125,46.85154888)
\lineto(397.45703125,43.12498638)
\lineto(401.8984375,43.12498638)
\lineto(401.8984375,41.44920513)
\lineto(397.45703125,41.44920513)
\lineto(397.45703125,34.32420513)
\curveto(397.45702686,33.25388937)(397.60155796,32.56639006)(397.890625,32.26170513)
\curveto(398.18749488,31.95701567)(398.78515053,31.80467207)(399.68359375,31.80467388)
\lineto(401.8984375,31.80467388)
\lineto(401.8984375,29.99998638)
\lineto(399.68359375,29.99998638)
\curveto(398.01952629,29.99998638)(396.87108994,30.30857982)(396.23828125,30.92576763)
\curveto(395.60546621,31.55076608)(395.28906027,32.68357744)(395.2890625,34.32420513)
\lineto(395.2890625,41.44920513)
\lineto(393.70703125,41.44920513)
\lineto(393.70703125,43.12498638)
\lineto(395.2890625,43.12498638)
\lineto(395.2890625,46.85154888)
\lineto(397.45703125,46.85154888)
}
}
{
\newrgbcolor{curcolor}{0 0 0}
\pscustom[linestyle=none,fillstyle=solid,fillcolor=curcolor]
{
\newpath
\moveto(404.74609375,43.12498638)
\lineto(406.90234375,43.12498638)
\lineto(406.90234375,29.99998638)
\lineto(404.74609375,29.99998638)
\lineto(404.74609375,43.12498638)
\moveto(404.74609375,48.23436138)
\lineto(406.90234375,48.23436138)
\lineto(406.90234375,45.50389263)
\lineto(404.74609375,45.50389263)
\lineto(404.74609375,48.23436138)
}
}
{
\newrgbcolor{curcolor}{0 0 0}
\pscustom[linestyle=none,fillstyle=solid,fillcolor=curcolor]
{
\newpath
\moveto(422.62890625,37.10154888)
\lineto(422.62890625,36.04686138)
\lineto(412.71484375,36.04686138)
\curveto(412.80859008,34.56248181)(413.25390214,33.42967045)(414.05078125,32.64842388)
\curveto(414.85546304,31.8749845)(415.97264942,31.48826614)(417.40234375,31.48826763)
\curveto(418.23045966,31.48826614)(419.03124011,31.58982854)(419.8046875,31.79295513)
\curveto(420.58592605,31.99607813)(421.35936278,32.30076533)(422.125,32.70701763)
\lineto(422.125,30.66795513)
\curveto(421.35155029,30.33982979)(420.55858233,30.08983004)(419.74609375,29.91795513)
\curveto(418.93358396,29.74608038)(418.10936603,29.66014297)(417.2734375,29.66014263)
\curveto(415.17968146,29.66014297)(413.51952687,30.26951736)(412.29296875,31.48826763)
\curveto(411.07421682,32.70701492)(410.46484243,34.35545077)(410.46484375,36.43358013)
\curveto(410.46484243,38.58200904)(411.04296685,40.28513234)(412.19921875,41.54295513)
\curveto(413.36327703,42.80856732)(414.92968171,43.44137919)(416.8984375,43.44139263)
\curveto(418.66405298,43.44137919)(420.05858283,42.87106726)(421.08203125,41.73045513)
\curveto(422.11326828,40.59763203)(422.62889276,39.05466482)(422.62890625,37.10154888)
\moveto(420.47265625,37.73436138)
\curveto(420.45701993,38.91403996)(420.12498902,39.85544527)(419.4765625,40.55858013)
\curveto(418.8359278,41.26169387)(417.98436616,41.61325601)(416.921875,41.61326763)
\curveto(415.71874342,41.61325601)(414.75390064,41.2734126)(414.02734375,40.59373638)
\curveto(413.30858958,39.91403896)(412.8945275,38.95700867)(412.78515625,37.72264263)
\lineto(420.47265625,37.73436138)
}
}
{
\newrgbcolor{curcolor}{1 1 1}
\pscustom[linestyle=none,fillstyle=solid,fillcolor=curcolor]
{
\newpath
\moveto(120.1649704,449.97605487)
\lineto(440.11043549,449.97605487)
\lineto(440.11043549,69.95679828)
\lineto(120.1649704,69.95679828)
\closepath
}
}
{
\newrgbcolor{curcolor}{0 0 0}
\pscustom[linewidth=2,linecolor=curcolor]
{
\newpath
\moveto(120.1649704,449.97605487)
\lineto(440.11043549,449.97605487)
\lineto(440.11043549,69.95679828)
\lineto(120.1649704,69.95679828)
\closepath
}
}
{
\newrgbcolor{curcolor}{0 0 0}
\pscustom[linestyle=none,fillstyle=solid,fillcolor=curcolor,opacity=0.11612902]
{
\newpath
\moveto(134.33292294,489.64407131)
\lineto(355.89206791,489.64407131)
\curveto(363.9166436,489.64407131)(370.37686157,483.18385333)(370.37686157,475.15927764)
\lineto(370.37686157,474.18155357)
\curveto(370.37686157,466.15697788)(363.9166436,459.69675991)(355.89206791,459.69675991)
\lineto(134.33292294,459.69675991)
\curveto(126.30834725,459.69675991)(119.84812927,466.15697788)(119.84812927,474.18155357)
\lineto(119.84812927,475.15927764)
\curveto(119.84812927,483.18385333)(126.30834725,489.64407131)(134.33292294,489.64407131)
\closepath
}
}
{
\newrgbcolor{curcolor}{0 0 0}
\pscustom[linewidth=2,linecolor=curcolor]
{
\newpath
\moveto(134.33292294,489.64407131)
\lineto(355.89206791,489.64407131)
\curveto(363.9166436,489.64407131)(370.37686157,483.18385333)(370.37686157,475.15927764)
\lineto(370.37686157,474.18155357)
\curveto(370.37686157,466.15697788)(363.9166436,459.69675991)(355.89206791,459.69675991)
\lineto(134.33292294,459.69675991)
\curveto(126.30834725,459.69675991)(119.84812927,466.15697788)(119.84812927,474.18155357)
\lineto(119.84812927,475.15927764)
\curveto(119.84812927,483.18385333)(126.30834725,489.64407131)(134.33292294,489.64407131)
\closepath
}
}
{
\newrgbcolor{curcolor}{1 1 1}
\pscustom[linestyle=none,fillstyle=solid,fillcolor=curcolor]
{
\newpath
\moveto(135.06185913,49.95851299)
\lineto(214.99641418,49.95851299)
\curveto(223.31771634,49.95851299)(230.01681519,43.25941414)(230.01681519,34.93811199)
\curveto(230.01681519,26.61680983)(223.31771634,19.91771099)(214.99641418,19.91771099)
\lineto(135.06185913,19.91771099)
\curveto(126.74055698,19.91771099)(120.04145813,26.61680983)(120.04145813,34.93811199)
\curveto(120.04145813,43.25941414)(126.74055698,49.95851299)(135.06185913,49.95851299)
\closepath
}
}
{
\newrgbcolor{curcolor}{0 0 0}
\pscustom[linewidth=2,linecolor=curcolor]
{
\newpath
\moveto(135.06185913,49.95851299)
\lineto(214.99641418,49.95851299)
\curveto(223.31771634,49.95851299)(230.01681519,43.25941414)(230.01681519,34.93811199)
\curveto(230.01681519,26.61680983)(223.31771634,19.91771099)(214.99641418,19.91771099)
\lineto(135.06185913,19.91771099)
\curveto(126.74055698,19.91771099)(120.04145813,26.61680983)(120.04145813,34.93811199)
\curveto(120.04145813,43.25941414)(126.74055698,49.95851299)(135.06185913,49.95851299)
\closepath
}
}
{
\newrgbcolor{curcolor}{0 0 0}
\pscustom[linestyle=none,fillstyle=solid,fillcolor=curcolor]
{
\newpath
\moveto(150.65234375,38.20311138)
\curveto(151.16014509,38.03122835)(151.6523321,37.66404121)(152.12890625,37.10154888)
\curveto(152.61326864,36.53904234)(153.09764315,35.76560561)(153.58203125,34.78123638)
\lineto(155.984375,29.99998638)
\lineto(153.44140625,29.99998638)
\lineto(151.203125,34.48826763)
\curveto(150.62498938,35.66013697)(150.06248994,36.43747994)(149.515625,36.82029888)
\curveto(148.97655352,37.20310417)(148.23827301,37.39451023)(147.30078125,37.39451763)
\lineto(144.72265625,37.39451763)
\lineto(144.72265625,29.99998638)
\lineto(142.35546875,29.99998638)
\lineto(142.35546875,47.49608013)
\lineto(147.69921875,47.49608013)
\curveto(149.69920905,47.49606263)(151.19139506,47.0780943)(152.17578125,46.24217388)
\curveto(153.16014309,45.40622097)(153.6523301,44.14450348)(153.65234375,42.45701763)
\curveto(153.6523301,41.35544377)(153.39451786,40.44138219)(152.87890625,39.71483013)
\curveto(152.37108138,38.98825864)(151.62889462,38.48435289)(150.65234375,38.20311138)
\moveto(144.72265625,45.55076763)
\lineto(144.72265625,39.33983013)
\lineto(147.69921875,39.33983013)
\curveto(148.83983491,39.33982079)(149.69920905,39.60153928)(150.27734375,40.12498638)
\curveto(150.86327039,40.65622572)(151.15623884,41.43356869)(151.15625,42.45701763)
\curveto(151.15623884,43.48044165)(150.86327039,44.24997213)(150.27734375,44.76561138)
\curveto(149.69920905,45.28903359)(148.83983491,45.55075208)(147.69921875,45.55076763)
\lineto(144.72265625,45.55076763)
}
}
{
\newrgbcolor{curcolor}{0 0 0}
\pscustom[linestyle=none,fillstyle=solid,fillcolor=curcolor]
{
\newpath
\moveto(169.09765625,37.10154888)
\lineto(169.09765625,36.04686138)
\lineto(159.18359375,36.04686138)
\curveto(159.27734008,34.56248181)(159.72265214,33.42967045)(160.51953125,32.64842388)
\curveto(161.32421304,31.8749845)(162.44139942,31.48826614)(163.87109375,31.48826763)
\curveto(164.69920966,31.48826614)(165.49999011,31.58982854)(166.2734375,31.79295513)
\curveto(167.05467605,31.99607813)(167.82811278,32.30076533)(168.59375,32.70701763)
\lineto(168.59375,30.66795513)
\curveto(167.82030029,30.33982979)(167.02733233,30.08983004)(166.21484375,29.91795513)
\curveto(165.40233396,29.74608038)(164.57811603,29.66014297)(163.7421875,29.66014263)
\curveto(161.64843146,29.66014297)(159.98827687,30.26951736)(158.76171875,31.48826763)
\curveto(157.54296682,32.70701492)(156.93359243,34.35545077)(156.93359375,36.43358013)
\curveto(156.93359243,38.58200904)(157.51171685,40.28513234)(158.66796875,41.54295513)
\curveto(159.83202703,42.80856732)(161.39843171,43.44137919)(163.3671875,43.44139263)
\curveto(165.13280298,43.44137919)(166.52733283,42.87106726)(167.55078125,41.73045513)
\curveto(168.58201828,40.59763203)(169.09764276,39.05466482)(169.09765625,37.10154888)
\moveto(166.94140625,37.73436138)
\curveto(166.92576993,38.91403996)(166.59373902,39.85544527)(165.9453125,40.55858013)
\curveto(165.3046778,41.26169387)(164.45311616,41.61325601)(163.390625,41.61326763)
\curveto(162.18749342,41.61325601)(161.22265064,41.2734126)(160.49609375,40.59373638)
\curveto(159.77733958,39.91403896)(159.3632775,38.95700867)(159.25390625,37.72264263)
\lineto(166.94140625,37.73436138)
}
}
{
\newrgbcolor{curcolor}{0 0 0}
\pscustom[linestyle=none,fillstyle=solid,fillcolor=curcolor]
{
\newpath
\moveto(174.76953125,46.85154888)
\lineto(174.76953125,43.12498638)
\lineto(179.2109375,43.12498638)
\lineto(179.2109375,41.44920513)
\lineto(174.76953125,41.44920513)
\lineto(174.76953125,34.32420513)
\curveto(174.76952686,33.25388937)(174.91405796,32.56639006)(175.203125,32.26170513)
\curveto(175.49999487,31.95701567)(176.09765053,31.80467207)(176.99609375,31.80467388)
\lineto(179.2109375,31.80467388)
\lineto(179.2109375,29.99998638)
\lineto(176.99609375,29.99998638)
\curveto(175.33202629,29.99998638)(174.18358994,30.30857982)(173.55078125,30.92576763)
\curveto(172.91796621,31.55076608)(172.60156027,32.68357744)(172.6015625,34.32420513)
\lineto(172.6015625,41.44920513)
\lineto(171.01953125,41.44920513)
\lineto(171.01953125,43.12498638)
\lineto(172.6015625,43.12498638)
\lineto(172.6015625,46.85154888)
\lineto(174.76953125,46.85154888)
}
}
{
\newrgbcolor{curcolor}{0 0 0}
\pscustom[linestyle=none,fillstyle=solid,fillcolor=curcolor]
{
\newpath
\moveto(187.14453125,41.61326763)
\curveto(185.98827506,41.61325601)(185.07421347,41.16013147)(184.40234375,40.25389263)
\curveto(183.73046482,39.35544577)(183.39452765,38.12107201)(183.39453125,36.55076763)
\curveto(183.39452765,34.98045015)(183.72655857,33.74217013)(184.390625,32.83592388)
\curveto(185.06249473,31.93748444)(185.98046257,31.48826614)(187.14453125,31.48826763)
\curveto(188.29296025,31.48826614)(189.20311559,31.94139069)(189.875,32.84764263)
\curveto(190.54686425,33.75388887)(190.88280141,34.98826264)(190.8828125,36.55076763)
\curveto(190.88280141,38.10544702)(190.54686425,39.33591454)(189.875,40.24217388)
\curveto(189.20311559,41.15622522)(188.29296025,41.61325601)(187.14453125,41.61326763)
\moveto(187.14453125,43.44139263)
\curveto(189.01952203,43.44137919)(190.4921768,42.83200479)(191.5625,41.61326763)
\curveto(192.63279966,40.39450723)(193.16795538,38.70700892)(193.16796875,36.55076763)
\curveto(193.16795538,34.40232572)(192.63279966,32.71482741)(191.5625,31.48826763)
\curveto(190.4921768,30.26951736)(189.01952203,29.66014297)(187.14453125,29.66014263)
\curveto(185.26171329,29.66014297)(183.78515226,30.26951736)(182.71484375,31.48826763)
\curveto(181.65234189,32.71482741)(181.12109243,34.40232572)(181.12109375,36.55076763)
\curveto(181.12109243,38.70700892)(181.65234189,40.39450723)(182.71484375,41.61326763)
\curveto(183.78515226,42.83200479)(185.26171329,43.44137919)(187.14453125,43.44139263)
}
}
{
\newrgbcolor{curcolor}{0 0 0}
\pscustom[linestyle=none,fillstyle=solid,fillcolor=curcolor]
{
\newpath
\moveto(196.5078125,35.17967388)
\lineto(196.5078125,43.12498638)
\lineto(198.6640625,43.12498638)
\lineto(198.6640625,35.26170513)
\curveto(198.6640583,34.01951361)(198.90624556,33.08592079)(199.390625,32.46092388)
\curveto(199.87499459,31.84373453)(200.60155637,31.53514109)(201.5703125,31.53514263)
\curveto(202.73436673,31.53514109)(203.65233457,31.90623447)(204.32421875,32.64842388)
\curveto(205.00389571,33.39060799)(205.34373913,34.40232572)(205.34375,35.68358013)
\lineto(205.34375,43.12498638)
\lineto(207.5,43.12498638)
\lineto(207.5,29.99998638)
\lineto(205.34375,29.99998638)
\lineto(205.34375,32.01561138)
\curveto(204.82030215,31.21873516)(204.21092776,30.62498575)(203.515625,30.23436138)
\curveto(202.82811664,29.85154903)(202.02733619,29.66014297)(201.11328125,29.66014263)
\curveto(199.60546361,29.66014297)(198.46093351,30.1288925)(197.6796875,31.06639263)
\curveto(196.89843507,32.00389062)(196.50781046,33.374983)(196.5078125,35.17967388)
\moveto(201.93359375,43.44139263)
\lineto(201.93359375,43.44139263)
}
}
{
\newrgbcolor{curcolor}{0 0 0}
\pscustom[linestyle=none,fillstyle=solid,fillcolor=curcolor]
{
\newpath
\moveto(219.5703125,41.10936138)
\curveto(219.32811538,41.24997513)(219.06249064,41.35153753)(218.7734375,41.41404888)
\curveto(218.49217871,41.48434989)(218.17967902,41.51950611)(217.8359375,41.51951763)
\curveto(216.61718059,41.51950611)(215.67968152,41.12106901)(215.0234375,40.32420513)
\curveto(214.37499533,39.53513309)(214.0507769,38.39841548)(214.05078125,36.91404888)
\lineto(214.05078125,29.99998638)
\lineto(211.8828125,29.99998638)
\lineto(211.8828125,43.12498638)
\lineto(214.05078125,43.12498638)
\lineto(214.05078125,41.08592388)
\curveto(214.50390145,41.88278699)(215.09374461,42.47263015)(215.8203125,42.85545513)
\curveto(216.54686816,43.24606688)(217.42967977,43.44137919)(218.46875,43.44139263)
\curveto(218.61717859,43.44137919)(218.78124092,43.42966045)(218.9609375,43.40623638)
\curveto(219.14061556,43.39059799)(219.33983411,43.36325426)(219.55859375,43.32420513)
\lineto(219.5703125,41.10936138)
}
}
{
\newrgbcolor{curcolor}{1 1 1}
\pscustom[linestyle=none,fillstyle=solid,fillcolor=curcolor]
{
\newpath
\moveto(405.19685268,490.20115062)
\lineto(425.1650629,490.20115062)
\curveto(433.67793478,490.20115062)(440.53125763,483.34782778)(440.53125763,474.8349559)
\curveto(440.53125763,466.32208402)(433.67793478,459.46876117)(425.1650629,459.46876117)
\lineto(405.19685268,459.46876117)
\curveto(396.68398081,459.46876117)(389.83065796,466.32208402)(389.83065796,474.8349559)
\curveto(389.83065796,483.34782778)(396.68398081,490.20115062)(405.19685268,490.20115062)
\closepath
}
}
{
\newrgbcolor{curcolor}{0 0 0}
\pscustom[linewidth=2,linecolor=curcolor]
{
\newpath
\moveto(405.19685268,490.20115062)
\lineto(425.1650629,490.20115062)
\curveto(433.67793478,490.20115062)(440.53125763,483.34782778)(440.53125763,474.8349559)
\curveto(440.53125763,466.32208402)(433.67793478,459.46876117)(425.1650629,459.46876117)
\lineto(405.19685268,459.46876117)
\curveto(396.68398081,459.46876117)(389.83065796,466.32208402)(389.83065796,474.8349559)
\curveto(389.83065796,483.34782778)(396.68398081,490.20115062)(405.19685268,490.20115062)
\closepath
}
}
{
\newrgbcolor{curcolor}{0 0 0}
\pscustom[linewidth=2,linecolor=curcolor,linestyle=dashed,dash=8 8]
{
\newpath
\moveto(140,540)
\lineto(60,540)
}
}
{
\newrgbcolor{curcolor}{0 0 0}
\pscustom[linestyle=none,fillstyle=solid,fillcolor=curcolor]
{
\newpath
\moveto(129.53769464,544.84048224)
\lineto(142.6487474,540.01921591)
\lineto(129.53769392,535.19795064)
\curveto(131.632292,538.04442372)(131.62022288,541.93889292)(129.53769464,544.84048224)
\lineto(129.53769464,544.84048224)
\lineto(129.53769464,544.84048224)
\closepath
}
}
{
\newrgbcolor{curcolor}{0 0 0}
\pscustom[linewidth=2,linecolor=curcolor,linestyle=dashed,dash=8 8]
{
\newpath
\moveto(150.14286,480)
\lineto(60.142857,480)
}
}
{
\newrgbcolor{curcolor}{0 0 0}
\pscustom[linestyle=none,fillstyle=solid,fillcolor=curcolor]
{
\newpath
\moveto(139.68055464,484.84048224)
\lineto(152.7916074,480.01921591)
\lineto(139.68055392,475.19795064)
\curveto(141.775152,478.04442372)(141.76308288,481.93889292)(139.68055464,484.84048224)
\lineto(139.68055464,484.84048224)
\lineto(139.68055464,484.84048224)
\closepath
}
}
{
\newrgbcolor{curcolor}{0 0 0}
\pscustom[linewidth=2,linecolor=curcolor,linestyle=dashed,dash=8 8]
{
\newpath
\moveto(420,480)
\lineto(500,480)
}
}
{
\newrgbcolor{curcolor}{0 0 0}
\pscustom[linestyle=none,fillstyle=solid,fillcolor=curcolor]
{
\newpath
\moveto(430.46230536,475.15951776)
\lineto(417.3512526,479.98078409)
\lineto(430.46230608,484.80204936)
\curveto(428.367708,481.95557628)(428.37977712,478.06110708)(430.46230536,475.15951776)
\lineto(430.46230536,475.15951776)
\lineto(430.46230536,475.15951776)
\closepath
}
}
{
\newrgbcolor{curcolor}{0 0 0}
\pscustom[linewidth=2,linecolor=curcolor,linestyle=dashed,dash=8 8]
{
\newpath
\moveto(150,320)
\lineto(60,320)
}
}
{
\newrgbcolor{curcolor}{0 0 0}
\pscustom[linestyle=none,fillstyle=solid,fillcolor=curcolor]
{
\newpath
\moveto(139.53769464,324.84048224)
\lineto(152.6487474,320.01921591)
\lineto(139.53769392,315.19795064)
\curveto(141.632292,318.04442372)(141.62022288,321.93889292)(139.53769464,324.84048224)
\lineto(139.53769464,324.84048224)
\lineto(139.53769464,324.84048224)
\closepath
}
}
{
\newrgbcolor{curcolor}{0 0 0}
\pscustom[linewidth=2,linecolor=curcolor,linestyle=dashed,dash=8 8]
{
\newpath
\moveto(420,40)
\lineto(500,40)
}
}
{
\newrgbcolor{curcolor}{0 0 0}
\pscustom[linestyle=none,fillstyle=solid,fillcolor=curcolor]
{
\newpath
\moveto(430.46230536,35.15951776)
\lineto(417.3512526,39.98078409)
\lineto(430.46230608,44.80204936)
\curveto(428.367708,41.95557628)(428.37977712,38.06110708)(430.46230536,35.15951776)
\lineto(430.46230536,35.15951776)
\lineto(430.46230536,35.15951776)
\closepath
}
}
{
\newrgbcolor{curcolor}{0 0 0}
\pscustom[linewidth=2,linecolor=curcolor,linestyle=dashed,dash=8 8]
{
\newpath
\moveto(140,40)
\lineto(60,40)
}
}
{
\newrgbcolor{curcolor}{0 0 0}
\pscustom[linestyle=none,fillstyle=solid,fillcolor=curcolor]
{
\newpath
\moveto(129.53769464,44.84048224)
\lineto(142.6487474,40.01921591)
\lineto(129.53769392,35.19795064)
\curveto(131.632292,38.04442372)(131.62022288,41.93889292)(129.53769464,44.84048224)
\lineto(129.53769464,44.84048224)
\lineto(129.53769464,44.84048224)
\closepath
}
}
{
\newrgbcolor{curcolor}{0 0 0}
\pscustom[linestyle=none,fillstyle=solid,fillcolor=curcolor]
{
\newpath
\moveto(43.96875,532.65625164)
\lineto(49.125,532.65625164)
\lineto(49.125,550.45312664)
\lineto(43.515625,549.32812664)
\lineto(43.515625,552.20312664)
\lineto(49.09375,553.32812664)
\lineto(52.25,553.32812664)
\lineto(52.25,532.65625164)
\lineto(57.40625,532.65625164)
\lineto(57.40625,530.00000164)
\lineto(43.96875,530.00000164)
\lineto(43.96875,532.65625164)
}
}
{
\newrgbcolor{curcolor}{0 0 0}
\pscustom[linestyle=none,fillstyle=solid,fillcolor=curcolor]
{
\newpath
\moveto(46.140625,472.65624973)
\lineto(57.15625,472.65624973)
\lineto(57.15625,469.99999973)
\lineto(42.34375,469.99999973)
\lineto(42.34375,472.65624973)
\curveto(43.54166312,473.89582917)(45.17186983,475.55728584)(47.234375,477.64062473)
\curveto(49.30728236,479.73436499)(50.60936439,481.08332198)(51.140625,481.68749973)
\curveto(52.15102952,482.82290357)(52.85415381,483.78123595)(53.25,484.56249973)
\curveto(53.65623634,485.35415104)(53.85936114,486.13019193)(53.859375,486.89062473)
\curveto(53.85936114,488.13018993)(53.42186158,489.14060559)(52.546875,489.92187473)
\curveto(51.68227998,490.70310403)(50.55207278,491.09372863)(49.15625,491.09374973)
\curveto(48.1666585,491.09372863)(47.11978455,490.92185381)(46.015625,490.57812473)
\curveto(44.92187008,490.23435449)(43.74999625,489.71352168)(42.5,489.01562473)
\lineto(42.5,492.20312473)
\curveto(43.77082956,492.71351868)(44.95832838,493.09893496)(46.0625,493.35937473)
\curveto(47.1666595,493.61976778)(48.17707516,493.74997598)(49.09375,493.74999973)
\curveto(51.51040516,493.74997598)(53.43748656,493.14580992)(54.875,491.93749973)
\curveto(56.31248369,490.72914567)(57.03123297,489.11456395)(57.03125,487.09374973)
\curveto(57.03123297,486.13540026)(56.84894148,485.22394284)(56.484375,484.35937473)
\curveto(56.1301922,483.50519456)(55.47915119,482.4947789)(54.53125,481.32812473)
\curveto(54.27081906,481.02603037)(53.44269489,480.15103124)(52.046875,478.70312473)
\curveto(50.65103102,477.26561746)(48.68228298,475.24999448)(46.140625,472.65624973)
}
}
{
\newrgbcolor{curcolor}{0 0 0}
\pscustom[linestyle=none,fillstyle=solid,fillcolor=curcolor]
{
\newpath
\moveto(512.984375,482.57812473)
\curveto(514.49477717,482.25519581)(515.67185933,481.58332148)(516.515625,480.56249973)
\curveto(517.3697743,479.54165685)(517.7968572,478.28124145)(517.796875,476.78124973)
\curveto(517.7968572,474.47916192)(517.00519133,472.6979137)(515.421875,471.43749973)
\curveto(513.83852783,470.17708288)(511.58853008,469.54687518)(508.671875,469.54687473)
\curveto(507.69270064,469.54687518)(506.68228498,469.64583342)(505.640625,469.84374973)
\curveto(504.60937039,470.0312497)(503.54166313,470.31770774)(502.4375,470.70312473)
\lineto(502.4375,473.74999973)
\curveto(503.31249669,473.23957982)(504.27082906,472.85416354)(505.3125,472.59374973)
\curveto(506.35416031,472.33333073)(507.44270089,472.20312253)(508.578125,472.20312473)
\curveto(510.55728111,472.20312253)(512.06248794,472.59374713)(513.09375,473.37499973)
\curveto(514.13540253,474.15624557)(514.65623534,475.2916611)(514.65625,476.78124973)
\curveto(514.65623534,478.15624157)(514.17186083,479.22915717)(513.203125,479.99999973)
\curveto(512.24477942,480.78123895)(510.90623909,481.17186356)(509.1875,481.17187473)
\lineto(506.46875,481.17187473)
\lineto(506.46875,483.76562473)
\lineto(509.3125,483.76562473)
\curveto(510.86457247,483.76561096)(512.05207128,484.07290232)(512.875,484.68749973)
\curveto(513.69790297,485.31248442)(514.10936089,486.20831685)(514.109375,487.37499973)
\curveto(514.10936089,488.57289782)(513.68227798,489.48956357)(512.828125,490.12499973)
\curveto(511.98436302,490.77081229)(510.77082256,491.09372863)(509.1875,491.09374973)
\curveto(508.32290834,491.09372863)(507.39582594,490.99997873)(506.40625,490.81249973)
\curveto(505.41666125,490.6249791)(504.32812067,490.33331273)(503.140625,489.93749973)
\lineto(503.140625,492.74999973)
\curveto(504.33853733,493.08330998)(505.45832788,493.33330973)(506.5,493.49999973)
\curveto(507.55207578,493.66664273)(508.54165813,493.74997598)(509.46875,493.74999973)
\curveto(511.86457147,493.74997598)(513.76040291,493.20310153)(515.15625,492.10937473)
\curveto(516.55206678,491.02602037)(517.24998275,489.55727184)(517.25,487.70312473)
\curveto(517.24998275,486.41144165)(516.88019145,485.31769274)(516.140625,484.42187473)
\curveto(515.40102627,483.53644453)(514.34894398,482.92186181)(512.984375,482.57812473)
}
}
{
\newrgbcolor{curcolor}{0 0 0}
\pscustom[linestyle=none,fillstyle=solid,fillcolor=curcolor]
{
\newpath
\moveto(52.09375,330.57812664)
\lineto(44.125,318.12500164)
\lineto(52.09375,318.12500164)
\lineto(52.09375,330.57812664)
\moveto(51.265625,333.32812664)
\lineto(55.234375,333.32812664)
\lineto(55.234375,318.12500164)
\lineto(58.5625,318.12500164)
\lineto(58.5625,315.50000164)
\lineto(55.234375,315.50000164)
\lineto(55.234375,310.00000164)
\lineto(52.09375,310.00000164)
\lineto(52.09375,315.50000164)
\lineto(41.5625,315.50000164)
\lineto(41.5625,318.54687664)
\lineto(51.265625,333.32812664)
}
}
{
\newrgbcolor{curcolor}{0 0 0}
\pscustom[linestyle=none,fillstyle=solid,fillcolor=curcolor]
{
\newpath
\moveto(43.453125,53.32811138)
\lineto(55.84375,53.32811138)
\lineto(55.84375,50.67186138)
\lineto(46.34375,50.67186138)
\lineto(46.34375,44.95311138)
\curveto(46.80207653,45.10934627)(47.26040941,45.22392949)(47.71875,45.29686138)
\curveto(48.17707516,45.38017933)(48.63540803,45.42184596)(49.09375,45.42186138)
\curveto(51.69790497,45.42184596)(53.76040291,44.708305)(55.28125,43.28123638)
\curveto(56.80206653,41.85414119)(57.56248244,39.92185146)(57.5625,37.48436138)
\curveto(57.56248244,34.97393974)(56.78123322,33.02081669)(55.21875,31.62498638)
\curveto(53.65623634,30.23956947)(51.45311355,29.54686183)(48.609375,29.54686138)
\curveto(47.6302007,29.54686183)(46.6302017,29.63019508)(45.609375,29.79686138)
\curveto(44.59895373,29.96352808)(43.55207978,30.21352783)(42.46875,30.54686138)
\lineto(42.46875,33.71873638)
\curveto(43.40624659,33.2083165)(44.37499562,32.82810855)(45.375,32.57811138)
\curveto(46.37499363,32.32810905)(47.43228423,32.20310917)(48.546875,32.20311138)
\curveto(50.34894798,32.20310917)(51.77602989,32.67706703)(52.828125,33.62498638)
\curveto(53.88019445,34.57289847)(54.40623559,35.85935552)(54.40625,37.48436138)
\curveto(54.40623559,39.10935227)(53.88019445,40.39580931)(52.828125,41.34373638)
\curveto(51.77602989,42.29164075)(50.34894798,42.76559861)(48.546875,42.76561138)
\curveto(47.7031173,42.76559861)(46.85936814,42.67184871)(46.015625,42.48436138)
\curveto(45.18228648,42.29684908)(44.32812067,42.00518271)(43.453125,41.60936138)
\lineto(43.453125,53.32811138)
}
}
{
\newrgbcolor{curcolor}{0 0 0}
\pscustom[linestyle=none,fillstyle=solid,fillcolor=curcolor]
{
\newpath
\moveto(510.5625,42.92186138)
\curveto(509.14582419,42.92184846)(508.02082531,42.43747394)(507.1875,41.46873638)
\curveto(506.36457697,40.49997588)(505.95311905,39.17185221)(505.953125,37.48436138)
\curveto(505.95311905,35.80727224)(506.36457697,34.47914856)(507.1875,33.49998638)
\curveto(508.02082531,32.53123385)(509.14582419,32.04685933)(510.5625,32.04686138)
\curveto(511.97915469,32.04685933)(513.09894523,32.53123385)(513.921875,33.49998638)
\curveto(514.75519358,34.47914856)(515.17185983,35.80727224)(515.171875,37.48436138)
\curveto(515.17185983,39.17185221)(514.75519358,40.49997588)(513.921875,41.46873638)
\curveto(513.09894523,42.43747394)(511.97915469,42.92184846)(510.5625,42.92186138)
\moveto(516.828125,52.81248638)
\lineto(516.828125,49.93748638)
\curveto(516.0364423,50.31246606)(515.23435977,50.59892411)(514.421875,50.79686138)
\curveto(513.61977805,50.99475705)(512.82290384,51.09371528)(512.03125,51.09373638)
\curveto(509.94790672,51.09371528)(508.35415831,50.39059099)(507.25,48.98436138)
\curveto(506.15624384,47.5780938)(505.53124447,45.45309592)(505.375,42.60936138)
\curveto(505.98957734,43.51559786)(506.76040991,44.2083055)(507.6875,44.68748638)
\curveto(508.61457472,45.17705453)(509.63540703,45.42184596)(510.75,45.42186138)
\curveto(513.09373691,45.42184596)(514.94269339,44.708305)(516.296875,43.28123638)
\curveto(517.66144067,41.86455785)(518.34373166,39.93226811)(518.34375,37.48436138)
\curveto(518.34373166,35.08852296)(517.63539903,33.16664988)(516.21875,31.71873638)
\curveto(514.80206853,30.27081944)(512.91665375,29.54686183)(510.5625,29.54686138)
\curveto(507.86457547,29.54686183)(505.80207753,30.5781108)(504.375,32.64061138)
\curveto(502.94791372,34.71352333)(502.23437277,37.71352033)(502.234375,41.64061138)
\curveto(502.23437277,45.32809605)(503.10937189,48.26559311)(504.859375,50.45311138)
\curveto(506.60936839,52.65100539)(508.95832438,53.74996263)(511.90625,53.74998638)
\curveto(512.69790397,53.74996263)(513.49477817,53.67183771)(514.296875,53.51561138)
\curveto(515.10935989,53.35933802)(515.95310905,53.12496325)(516.828125,52.81248638)
}
}
\end{pspicture}

		\end{center}
		
		\begin{enumerate}
		  \item Fond d'écran
		  \item Champs de texte ``Filtrer''
		  \item Bouton ``Rafraichir''
		  \item Cadre contenant les différentes parties en ligne en cours
		  \item Bouton ``\hyperlink{Page d'accueil}{Retour}''
		  \item Bouton ``\hyperlink{Creer partie multi-joueurs}{Créer partie}''
		\end{enumerate}
		
		\subsubsection{Description des zones}
		
			\begin{tabular}{|c|c|c|c|c|} \hline
				Numéro de zone & Type  & Description & Evènement &	Règle \\\hline
			\end{tabular}
			
		\subsubsection{Description des règles}

			\underline{RG5-01 :}
				\begin{quote}
				
				\end{quote}		
	
\newpage

	\subsection{Créer partie multi-joueurs}
	
		\hypertarget{Creer partie multi-joueurs}{}
		\label{Creer partie multi-joueurs}
	
		\begin{center}
			%LaTeX with PSTricks extensions
%%Creator: inkscape 0.48.0
%%Please note this file requires PSTricks extensions
\psset{xunit=.5pt,yunit=.5pt,runit=.5pt}
\begin{pspicture}(560,600)
{
\newrgbcolor{curcolor}{1 1 1}
\pscustom[linestyle=none,fillstyle=solid,fillcolor=curcolor]
{
\newpath
\moveto(133.12401581,597.52220273)
\lineto(426.87598419,597.52220273)
\curveto(443.85397169,597.52220273)(457.52217102,583.8540034)(457.52217102,566.8760159)
\lineto(457.52217102,33.124017)
\curveto(457.52217102,16.1460295)(443.85397169,2.47783017)(426.87598419,2.47783017)
\lineto(133.12401581,2.47783017)
\curveto(116.14602831,2.47783017)(102.47782898,16.1460295)(102.47782898,33.124017)
\lineto(102.47782898,566.8760159)
\curveto(102.47782898,583.8540034)(116.14602831,597.52220273)(133.12401581,597.52220273)
\closepath
}
}
{
\newrgbcolor{curcolor}{0 0 0}
\pscustom[linewidth=4.95566034,linecolor=curcolor]
{
\newpath
\moveto(133.12401581,597.52220273)
\lineto(426.87598419,597.52220273)
\curveto(443.85397169,597.52220273)(457.52217102,583.8540034)(457.52217102,566.8760159)
\lineto(457.52217102,33.124017)
\curveto(457.52217102,16.1460295)(443.85397169,2.47783017)(426.87598419,2.47783017)
\lineto(133.12401581,2.47783017)
\curveto(116.14602831,2.47783017)(102.47782898,16.1460295)(102.47782898,33.124017)
\lineto(102.47782898,566.8760159)
\curveto(102.47782898,583.8540034)(116.14602831,597.52220273)(133.12401581,597.52220273)
\closepath
}
}
{
\newrgbcolor{curcolor}{1 1 1}
\pscustom[linestyle=none,fillstyle=solid,fillcolor=curcolor]
{
\newpath
\moveto(133.87602234,229.91494751)
\lineto(422.23780823,229.91494751)
\curveto(436.20654425,229.91494751)(447.45213318,218.66935858)(447.45213318,204.70062256)
\lineto(447.45213318,95.27957153)
\curveto(447.45213318,81.31083551)(436.20654425,70.06524658)(422.23780823,70.06524658)
\lineto(133.87602234,70.06524658)
\curveto(119.90728632,70.06524658)(108.66169739,81.31083551)(108.66169739,95.27957153)
\lineto(108.66169739,204.70062256)
\curveto(108.66169739,218.66935858)(119.90728632,229.91494751)(133.87602234,229.91494751)
\closepath
}
}
{
\newrgbcolor{curcolor}{0 0 0}
\pscustom[linewidth=1.87744212,linecolor=curcolor]
{
\newpath
\moveto(133.87602234,229.91494751)
\lineto(422.23780823,229.91494751)
\curveto(436.20654425,229.91494751)(447.45213318,218.66935858)(447.45213318,204.70062256)
\lineto(447.45213318,95.27957153)
\curveto(447.45213318,81.31083551)(436.20654425,70.06524658)(422.23780823,70.06524658)
\lineto(133.87602234,70.06524658)
\curveto(119.90728632,70.06524658)(108.66169739,81.31083551)(108.66169739,95.27957153)
\lineto(108.66169739,204.70062256)
\curveto(108.66169739,218.66935858)(119.90728632,229.91494751)(133.87602234,229.91494751)
\closepath
}
}
{
\newrgbcolor{curcolor}{0 0 0}
\pscustom[linestyle=none,fillstyle=solid,fillcolor=curcolor,opacity=0.11935484]
{
\newpath
\moveto(322.58069324,500)
\lineto(427.16530895,500)
\curveto(434.27572779,500)(440,495.83353307)(440,490.6581459)
\lineto(440,479.6365757)
\curveto(440,474.46118853)(434.27572779,470.2947216)(427.16530895,470.2947216)
\lineto(322.58069324,470.2947216)
\curveto(315.4702744,470.2947216)(309.7460022,474.46118853)(309.7460022,479.6365757)
\lineto(309.7460022,490.6581459)
\curveto(309.7460022,495.83353307)(315.4702744,500)(322.58069324,500)
\closepath
}
}
{
\newrgbcolor{curcolor}{0 0 0}
\pscustom[linewidth=2,linecolor=curcolor]
{
\newpath
\moveto(322.58069324,500)
\lineto(427.16530895,500)
\curveto(434.27572779,500)(440,495.83353307)(440,490.6581459)
\lineto(440,479.6365757)
\curveto(440,474.46118853)(434.27572779,470.2947216)(427.16530895,470.2947216)
\lineto(322.58069324,470.2947216)
\curveto(315.4702744,470.2947216)(309.7460022,474.46118853)(309.7460022,479.6365757)
\lineto(309.7460022,490.6581459)
\curveto(309.7460022,495.83353307)(315.4702744,500)(322.58069324,500)
\closepath
}
}
{
\newrgbcolor{curcolor}{1 1 1}
\pscustom[linestyle=none,fillstyle=solid,fillcolor=curcolor]
{
\newpath
\moveto(188.83351135,280.19555664)
\lineto(370.00863647,280.19555664)
\lineto(370.00863647,249.90542412)
\lineto(188.83351135,249.90542412)
\closepath
}
}
{
\newrgbcolor{curcolor}{0 0 0}
\pscustom[linewidth=2,linecolor=curcolor]
{
\newpath
\moveto(188.83351135,280.19555664)
\lineto(370.00863647,280.19555664)
\lineto(370.00863647,249.90542412)
\lineto(188.83351135,249.90542412)
\closepath
}
}
{
\newrgbcolor{curcolor}{1 1 1}
\pscustom[linestyle=none,fillstyle=solid,fillcolor=curcolor]
{
\newpath
\moveto(218.86587524,409.90138245)
\lineto(338.92390442,409.90138245)
\lineto(338.92390442,289.90190125)
\lineto(218.86587524,289.90190125)
\closepath
}
}
{
\newrgbcolor{curcolor}{0 0 0}
\pscustom[linewidth=2,linecolor=curcolor]
{
\newpath
\moveto(218.86587524,409.90138245)
\lineto(338.92390442,409.90138245)
\lineto(338.92390442,289.90190125)
\lineto(218.86587524,289.90190125)
\closepath
}
}
{
\newrgbcolor{curcolor}{1 1 1}
\pscustom[linestyle=none,fillstyle=solid,fillcolor=curcolor]
{
\newpath
\moveto(358.87203979,390.10528564)
\lineto(438.87251282,390.10528564)
\lineto(438.87251282,310.14836884)
\lineto(358.87203979,310.14836884)
\closepath
}
}
{
\newrgbcolor{curcolor}{0 0 0}
\pscustom[linewidth=2,linecolor=curcolor]
{
\newpath
\moveto(358.87203979,390.10528564)
\lineto(438.87251282,390.10528564)
\lineto(438.87251282,310.14836884)
\lineto(358.87203979,310.14836884)
\closepath
}
}
{
\newrgbcolor{curcolor}{1 1 1}
\pscustom[linestyle=none,fillstyle=solid,fillcolor=curcolor]
{
\newpath
\moveto(118.66197968,390.10528564)
\lineto(198.66246033,390.10528564)
\lineto(198.66246033,310.14836884)
\lineto(118.66197968,310.14836884)
\closepath
}
}
{
\newrgbcolor{curcolor}{0 0 0}
\pscustom[linewidth=2,linecolor=curcolor]
{
\newpath
\moveto(118.66197968,390.10528564)
\lineto(198.66246033,390.10528564)
\lineto(198.66246033,310.14836884)
\lineto(118.66197968,310.14836884)
\closepath
}
}
{
\newrgbcolor{curcolor}{0 0 0}
\pscustom[linestyle=none,fillstyle=solid,fillcolor=curcolor,opacity=0.11935484]
{
\newpath
\moveto(332.67380142,170)
\lineto(418.2635994,170)
\curveto(424.08261341,170)(428.7672348,165.80657851)(428.7672348,160.59770966)
\lineto(428.7672348,149.50483704)
\curveto(428.7672348,144.29596819)(424.08261341,140.10254669)(418.2635994,140.10254669)
\lineto(332.67380142,140.10254669)
\curveto(326.85478741,140.10254669)(322.17016602,144.29596819)(322.17016602,149.50483704)
\lineto(322.17016602,160.59770966)
\curveto(322.17016602,165.80657851)(326.85478741,170)(332.67380142,170)
\closepath
}
}
{
\newrgbcolor{curcolor}{0 0 0}
\pscustom[linewidth=2,linecolor=curcolor]
{
\newpath
\moveto(332.67380142,170)
\lineto(418.2635994,170)
\curveto(424.08261341,170)(428.7672348,165.80657851)(428.7672348,160.59770966)
\lineto(428.7672348,149.50483704)
\curveto(428.7672348,144.29596819)(424.08261341,140.10254669)(418.2635994,140.10254669)
\lineto(332.67380142,140.10254669)
\curveto(326.85478741,140.10254669)(322.17016602,144.29596819)(322.17016602,149.50483704)
\lineto(322.17016602,160.59770966)
\curveto(322.17016602,165.80657851)(326.85478741,170)(332.67380142,170)
\closepath
}
}
{
\newrgbcolor{curcolor}{0 0 0}
\pscustom[linestyle=none,fillstyle=solid,fillcolor=curcolor,opacity=0.11935484]
{
\newpath
\moveto(311.47112656,119.96066284)
\lineto(416.01738358,119.96066284)
\curveto(423.12519402,119.96066284)(428.84736633,115.79414402)(428.84736633,110.6186924)
\lineto(428.84736633,99.59698677)
\curveto(428.84736633,94.42153514)(423.12519402,90.25501633)(416.01738358,90.25501633)
\lineto(311.47112656,90.25501633)
\curveto(304.36331611,90.25501633)(298.6411438,94.42153514)(298.6411438,99.59698677)
\lineto(298.6411438,110.6186924)
\curveto(298.6411438,115.79414402)(304.36331611,119.96066284)(311.47112656,119.96066284)
\closepath
}
}
{
\newrgbcolor{curcolor}{0 0 0}
\pscustom[linewidth=2,linecolor=curcolor]
{
\newpath
\moveto(311.47112656,119.96066284)
\lineto(416.01738358,119.96066284)
\curveto(423.12519402,119.96066284)(428.84736633,115.79414402)(428.84736633,110.6186924)
\lineto(428.84736633,99.59698677)
\curveto(428.84736633,94.42153514)(423.12519402,90.25501633)(416.01738358,90.25501633)
\lineto(311.47112656,90.25501633)
\curveto(304.36331611,90.25501633)(298.6411438,94.42153514)(298.6411438,99.59698677)
\lineto(298.6411438,110.6186924)
\curveto(298.6411438,115.79414402)(304.36331611,119.96066284)(311.47112656,119.96066284)
\closepath
}
}
{
\newrgbcolor{curcolor}{1 1 1}
\pscustom[linestyle=none,fillstyle=solid,fillcolor=curcolor]
{
\newpath
\moveto(329.26973534,49.81427002)
\lineto(420.7846241,49.81427002)
\curveto(425.93425914,49.81427002)(430.0799942,46.91140156)(430.0799942,43.30559635)
\lineto(430.0799942,26.42860699)
\curveto(430.0799942,22.82280177)(425.93425914,19.91993332)(420.7846241,19.91993332)
\lineto(329.26973534,19.91993332)
\curveto(324.1201003,19.91993332)(319.97436523,22.82280177)(319.97436523,26.42860699)
\lineto(319.97436523,43.30559635)
\curveto(319.97436523,46.91140156)(324.1201003,49.81427002)(329.26973534,49.81427002)
\closepath
}
}
{
\newrgbcolor{curcolor}{0 0 0}
\pscustom[linewidth=2,linecolor=curcolor]
{
\newpath
\moveto(329.26973534,49.81427002)
\lineto(420.7846241,49.81427002)
\curveto(425.93425914,49.81427002)(430.0799942,46.91140156)(430.0799942,43.30559635)
\lineto(430.0799942,26.42860699)
\curveto(430.0799942,22.82280177)(425.93425914,19.91993332)(420.7846241,19.91993332)
\lineto(329.26973534,19.91993332)
\curveto(324.1201003,19.91993332)(319.97436523,22.82280177)(319.97436523,26.42860699)
\lineto(319.97436523,43.30559635)
\curveto(319.97436523,46.91140156)(324.1201003,49.81427002)(329.26973534,49.81427002)
\closepath
}
}
{
\newrgbcolor{curcolor}{0 0 0}
\pscustom[linestyle=none,fillstyle=solid,fillcolor=curcolor]
{
\newpath
\moveto(332.35546875,47.49609375)
\lineto(334.72265625,47.49609375)
\lineto(334.72265625,31.9921875)
\lineto(343.2421875,31.9921875)
\lineto(343.2421875,30)
\lineto(332.35546875,30)
\lineto(332.35546875,47.49609375)
}
}
{
\newrgbcolor{curcolor}{0 0 0}
\pscustom[linestyle=none,fillstyle=solid,fillcolor=curcolor]
{
\newpath
\moveto(351.5859375,36.59765625)
\curveto(349.84374352,36.59764965)(348.63671347,36.3984311)(347.96484375,36)
\curveto(347.29296482,35.6015569)(346.95702765,34.92187008)(346.95703125,33.9609375)
\curveto(346.95702765,33.1953093)(347.2070274,32.58593491)(347.70703125,32.1328125)
\curveto(348.21483889,31.68749831)(348.90233821,31.46484229)(349.76953125,31.46484375)
\curveto(350.96483614,31.46484229)(351.92186644,31.88671686)(352.640625,32.73046875)
\curveto(353.36717749,33.58202767)(353.73045838,34.71093279)(353.73046875,36.1171875)
\lineto(353.73046875,36.59765625)
\lineto(351.5859375,36.59765625)
\moveto(355.88671875,37.48828125)
\lineto(355.88671875,30)
\lineto(353.73046875,30)
\lineto(353.73046875,31.9921875)
\curveto(353.23827137,31.1953113)(352.62499073,30.60546814)(351.890625,30.22265625)
\curveto(351.1562422,29.8476564)(350.2578056,29.66015659)(349.1953125,29.66015625)
\curveto(347.85155801,29.66015659)(346.78124658,30.03515621)(345.984375,30.78515625)
\curveto(345.19531066,31.54296721)(344.80077981,32.55468495)(344.80078125,33.8203125)
\curveto(344.80077981,35.2968697)(345.29296682,36.41014984)(346.27734375,37.16015625)
\curveto(347.26952734,37.91014834)(348.74608836,38.28514796)(350.70703125,38.28515625)
\lineto(353.73046875,38.28515625)
\lineto(353.73046875,38.49609375)
\curveto(353.73045838,39.48827176)(353.40233371,40.253896)(352.74609375,40.79296875)
\curveto(352.09764751,41.33983241)(351.18358593,41.61326964)(350.00390625,41.61328125)
\curveto(349.25390036,41.61326964)(348.52343234,41.52342598)(347.8125,41.34375)
\curveto(347.10155876,41.16405134)(346.41796569,40.89452036)(345.76171875,40.53515625)
\lineto(345.76171875,42.52734375)
\curveto(346.55077806,42.83201842)(347.31640229,43.05858069)(348.05859375,43.20703125)
\curveto(348.80077581,43.36326789)(349.52343134,43.44139281)(350.2265625,43.44140625)
\curveto(352.12499123,43.44139281)(353.54295857,42.9492058)(354.48046875,41.96484375)
\curveto(355.41795669,40.98045777)(355.88670622,39.48827176)(355.88671875,37.48828125)
}
}
{
\newrgbcolor{curcolor}{0 0 0}
\pscustom[linestyle=none,fillstyle=solid,fillcolor=curcolor]
{
\newpath
\moveto(371.25,37.921875)
\lineto(371.25,30)
\lineto(369.09375,30)
\lineto(369.09375,37.8515625)
\curveto(369.09373898,39.09374091)(368.85155173,40.02342748)(368.3671875,40.640625)
\curveto(367.8828027,41.25780124)(367.15624092,41.56639468)(366.1875,41.56640625)
\curveto(365.02343055,41.56639468)(364.10546272,41.1953013)(363.43359375,40.453125)
\curveto(362.76171407,39.71092779)(362.4257769,38.69921005)(362.42578125,37.41796875)
\lineto(362.42578125,30)
\lineto(360.2578125,30)
\lineto(360.2578125,43.125)
\lineto(362.42578125,43.125)
\lineto(362.42578125,41.0859375)
\curveto(362.94140139,41.87498812)(363.54686953,42.46483129)(364.2421875,42.85546875)
\curveto(364.94530563,43.2460805)(365.75389857,43.44139281)(366.66796875,43.44140625)
\curveto(368.17577115,43.44139281)(369.31639501,42.97264328)(370.08984375,42.03515625)
\curveto(370.86326846,41.10545764)(371.24998683,39.73436527)(371.25,37.921875)
}
}
{
\newrgbcolor{curcolor}{0 0 0}
\pscustom[linestyle=none,fillstyle=solid,fillcolor=curcolor]
{
\newpath
\moveto(385.01953125,42.62109375)
\lineto(385.01953125,40.60546875)
\curveto(384.41014515,40.94139531)(383.79686452,41.19139506)(383.1796875,41.35546875)
\curveto(382.57030324,41.52733222)(381.95311636,41.61326964)(381.328125,41.61328125)
\curveto(379.92968088,41.61326964)(378.84374447,41.16795758)(378.0703125,40.27734375)
\curveto(377.29687102,39.39452186)(376.91015265,38.1523356)(376.91015625,36.55078125)
\curveto(376.91015265,34.9492138)(377.29687102,33.7031213)(378.0703125,32.8125)
\curveto(378.84374447,31.92968557)(379.92968088,31.48827976)(381.328125,31.48828125)
\curveto(381.95311636,31.48827976)(382.57030324,31.57031093)(383.1796875,31.734375)
\curveto(383.79686452,31.90624809)(384.41014515,32.16015409)(385.01953125,32.49609375)
\lineto(385.01953125,30.50390625)
\curveto(384.41795764,30.22265603)(383.79295827,30.01171874)(383.14453125,29.87109375)
\curveto(382.50389706,29.73046902)(381.82030399,29.66015659)(381.09375,29.66015625)
\curveto(379.1171817,29.66015659)(377.54687077,30.28124972)(376.3828125,31.5234375)
\curveto(375.21874809,32.76562223)(374.63671743,34.44140181)(374.63671875,36.55078125)
\curveto(374.63671743,38.69139756)(375.22265434,40.37498963)(376.39453125,41.6015625)
\curveto(377.57421449,42.82811217)(379.18749413,43.44139281)(381.234375,43.44140625)
\curveto(381.89842891,43.44139281)(382.54686577,43.37108038)(383.1796875,43.23046875)
\curveto(383.8124895,43.09764315)(384.42577014,42.89451836)(385.01953125,42.62109375)
}
}
{
\newrgbcolor{curcolor}{0 0 0}
\pscustom[linestyle=none,fillstyle=solid,fillcolor=curcolor]
{
\newpath
\moveto(400.01953125,37.1015625)
\lineto(400.01953125,36.046875)
\lineto(390.10546875,36.046875)
\curveto(390.19921508,34.56249544)(390.64452714,33.42968407)(391.44140625,32.6484375)
\curveto(392.24608804,31.87499813)(393.36327442,31.48827976)(394.79296875,31.48828125)
\curveto(395.62108466,31.48827976)(396.42186511,31.58984216)(397.1953125,31.79296875)
\curveto(397.97655105,31.99609175)(398.74998778,32.30077895)(399.515625,32.70703125)
\lineto(399.515625,30.66796875)
\curveto(398.74217529,30.33984341)(397.94920733,30.08984366)(397.13671875,29.91796875)
\curveto(396.32420896,29.746094)(395.49999103,29.66015659)(394.6640625,29.66015625)
\curveto(392.57030646,29.66015659)(390.91015187,30.26953098)(389.68359375,31.48828125)
\curveto(388.46484182,32.70702854)(387.85546743,34.35546439)(387.85546875,36.43359375)
\curveto(387.85546743,38.58202267)(388.43359185,40.28514596)(389.58984375,41.54296875)
\curveto(390.75390203,42.80858094)(392.32030671,43.44139281)(394.2890625,43.44140625)
\curveto(396.05467798,43.44139281)(397.44920783,42.87108088)(398.47265625,41.73046875)
\curveto(399.50389328,40.59764565)(400.01951776,39.05467845)(400.01953125,37.1015625)
\moveto(397.86328125,37.734375)
\curveto(397.84764493,38.91405359)(397.51561402,39.85545889)(396.8671875,40.55859375)
\curveto(396.2265528,41.26170749)(395.37499116,41.61326964)(394.3125,41.61328125)
\curveto(393.10936842,41.61326964)(392.14452564,41.27342623)(391.41796875,40.59375)
\curveto(390.69921458,39.91405259)(390.2851525,38.95702229)(390.17578125,37.72265625)
\lineto(397.86328125,37.734375)
}
}
{
\newrgbcolor{curcolor}{0 0 0}
\pscustom[linestyle=none,fillstyle=solid,fillcolor=curcolor]
{
\newpath
\moveto(411.1640625,41.109375)
\curveto(410.92186537,41.24998875)(410.65624064,41.35155115)(410.3671875,41.4140625)
\curveto(410.08592871,41.48436352)(409.77342902,41.51951973)(409.4296875,41.51953125)
\curveto(408.21093059,41.51951973)(407.27343152,41.12108263)(406.6171875,40.32421875)
\curveto(405.96874533,39.53514671)(405.6445269,38.3984291)(405.64453125,36.9140625)
\lineto(405.64453125,30)
\lineto(403.4765625,30)
\lineto(403.4765625,43.125)
\lineto(405.64453125,43.125)
\lineto(405.64453125,41.0859375)
\curveto(406.09765145,41.88280062)(406.68749461,42.47264378)(407.4140625,42.85546875)
\curveto(408.14061816,43.2460805)(409.02342977,43.44139281)(410.0625,43.44140625)
\curveto(410.21092859,43.44139281)(410.37499092,43.42967407)(410.5546875,43.40625)
\curveto(410.73436556,43.39061161)(410.93358411,43.36326789)(411.15234375,43.32421875)
\lineto(411.1640625,41.109375)
}
}
{
\newrgbcolor{curcolor}{1 1 1}
\pscustom[linestyle=none,fillstyle=solid,fillcolor=curcolor]
{
\newpath
\moveto(139.20729923,49.62908936)
\lineto(230.67112637,49.62908936)
\curveto(235.81788884,49.62908936)(239.96131134,46.74393137)(239.96131134,43.16012526)
\lineto(239.96131134,26.38610125)
\curveto(239.96131134,22.80229513)(235.81788884,19.91713715)(230.67112637,19.91713715)
\lineto(139.20729923,19.91713715)
\curveto(134.06053676,19.91713715)(129.91711426,22.80229513)(129.91711426,26.38610125)
\lineto(129.91711426,43.16012526)
\curveto(129.91711426,46.74393137)(134.06053676,49.62908936)(139.20729923,49.62908936)
\closepath
}
}
{
\newrgbcolor{curcolor}{0 0 0}
\pscustom[linewidth=1.72500002,linecolor=curcolor]
{
\newpath
\moveto(139.20729923,49.62908936)
\lineto(230.67112637,49.62908936)
\curveto(235.81788884,49.62908936)(239.96131134,46.74393137)(239.96131134,43.16012526)
\lineto(239.96131134,26.38610125)
\curveto(239.96131134,22.80229513)(235.81788884,19.91713715)(230.67112637,19.91713715)
\lineto(139.20729923,19.91713715)
\curveto(134.06053676,19.91713715)(129.91711426,22.80229513)(129.91711426,26.38610125)
\lineto(129.91711426,43.16012526)
\curveto(129.91711426,46.74393137)(134.06053676,49.62908936)(139.20729923,49.62908936)
\closepath
}
}
{
\newrgbcolor{curcolor}{0 0 0}
\pscustom[linestyle=none,fillstyle=solid,fillcolor=curcolor]
{
\newpath
\moveto(160.65234375,38.203125)
\curveto(161.16014509,38.03124197)(161.6523321,37.66405484)(162.12890625,37.1015625)
\curveto(162.61326864,36.53905596)(163.09764315,35.76561923)(163.58203125,34.78125)
\lineto(165.984375,30)
\lineto(163.44140625,30)
\lineto(161.203125,34.48828125)
\curveto(160.62498938,35.66015059)(160.06248994,36.43749356)(159.515625,36.8203125)
\curveto(158.97655352,37.2031178)(158.23827301,37.39452386)(157.30078125,37.39453125)
\lineto(154.72265625,37.39453125)
\lineto(154.72265625,30)
\lineto(152.35546875,30)
\lineto(152.35546875,47.49609375)
\lineto(157.69921875,47.49609375)
\curveto(159.69920905,47.49607625)(161.19139506,47.07810792)(162.17578125,46.2421875)
\curveto(163.16014309,45.40623459)(163.6523301,44.14451711)(163.65234375,42.45703125)
\curveto(163.6523301,41.35545739)(163.39451786,40.44139581)(162.87890625,39.71484375)
\curveto(162.37108138,38.98827226)(161.62889462,38.48436652)(160.65234375,38.203125)
\moveto(154.72265625,45.55078125)
\lineto(154.72265625,39.33984375)
\lineto(157.69921875,39.33984375)
\curveto(158.83983491,39.33983441)(159.69920905,39.6015529)(160.27734375,40.125)
\curveto(160.86327039,40.65623934)(161.15623884,41.43358232)(161.15625,42.45703125)
\curveto(161.15623884,43.48045527)(160.86327039,44.24998575)(160.27734375,44.765625)
\curveto(159.69920905,45.28904721)(158.83983491,45.5507657)(157.69921875,45.55078125)
\lineto(154.72265625,45.55078125)
}
}
{
\newrgbcolor{curcolor}{0 0 0}
\pscustom[linestyle=none,fillstyle=solid,fillcolor=curcolor]
{
\newpath
\moveto(179.09765625,37.1015625)
\lineto(179.09765625,36.046875)
\lineto(169.18359375,36.046875)
\curveto(169.27734008,34.56249544)(169.72265214,33.42968407)(170.51953125,32.6484375)
\curveto(171.32421304,31.87499813)(172.44139942,31.48827976)(173.87109375,31.48828125)
\curveto(174.69920966,31.48827976)(175.49999011,31.58984216)(176.2734375,31.79296875)
\curveto(177.05467605,31.99609175)(177.82811278,32.30077895)(178.59375,32.70703125)
\lineto(178.59375,30.66796875)
\curveto(177.82030029,30.33984341)(177.02733233,30.08984366)(176.21484375,29.91796875)
\curveto(175.40233396,29.746094)(174.57811603,29.66015659)(173.7421875,29.66015625)
\curveto(171.64843146,29.66015659)(169.98827687,30.26953098)(168.76171875,31.48828125)
\curveto(167.54296682,32.70702854)(166.93359243,34.35546439)(166.93359375,36.43359375)
\curveto(166.93359243,38.58202267)(167.51171685,40.28514596)(168.66796875,41.54296875)
\curveto(169.83202703,42.80858094)(171.39843171,43.44139281)(173.3671875,43.44140625)
\curveto(175.13280298,43.44139281)(176.52733283,42.87108088)(177.55078125,41.73046875)
\curveto(178.58201828,40.59764565)(179.09764276,39.05467845)(179.09765625,37.1015625)
\moveto(176.94140625,37.734375)
\curveto(176.92576993,38.91405359)(176.59373902,39.85545889)(175.9453125,40.55859375)
\curveto(175.3046778,41.26170749)(174.45311616,41.61326964)(173.390625,41.61328125)
\curveto(172.18749342,41.61326964)(171.22265064,41.27342623)(170.49609375,40.59375)
\curveto(169.77733958,39.91405259)(169.3632775,38.95702229)(169.25390625,37.72265625)
\lineto(176.94140625,37.734375)
}
}
{
\newrgbcolor{curcolor}{0 0 0}
\pscustom[linestyle=none,fillstyle=solid,fillcolor=curcolor]
{
\newpath
\moveto(184.76953125,46.8515625)
\lineto(184.76953125,43.125)
\lineto(189.2109375,43.125)
\lineto(189.2109375,41.44921875)
\lineto(184.76953125,41.44921875)
\lineto(184.76953125,34.32421875)
\curveto(184.76952686,33.253903)(184.91405796,32.56640368)(185.203125,32.26171875)
\curveto(185.49999487,31.95702929)(186.09765053,31.8046857)(186.99609375,31.8046875)
\lineto(189.2109375,31.8046875)
\lineto(189.2109375,30)
\lineto(186.99609375,30)
\curveto(185.33202629,30)(184.18358994,30.30859344)(183.55078125,30.92578125)
\curveto(182.91796621,31.5507797)(182.60156027,32.68359107)(182.6015625,34.32421875)
\lineto(182.6015625,41.44921875)
\lineto(181.01953125,41.44921875)
\lineto(181.01953125,43.125)
\lineto(182.6015625,43.125)
\lineto(182.6015625,46.8515625)
\lineto(184.76953125,46.8515625)
}
}
{
\newrgbcolor{curcolor}{0 0 0}
\pscustom[linestyle=none,fillstyle=solid,fillcolor=curcolor]
{
\newpath
\moveto(197.14453125,41.61328125)
\curveto(195.98827506,41.61326964)(195.07421347,41.16014509)(194.40234375,40.25390625)
\curveto(193.73046482,39.35545939)(193.39452765,38.12108563)(193.39453125,36.55078125)
\curveto(193.39452765,34.98046377)(193.72655857,33.74218376)(194.390625,32.8359375)
\curveto(195.06249473,31.93749806)(195.98046257,31.48827976)(197.14453125,31.48828125)
\curveto(198.29296025,31.48827976)(199.20311559,31.94140431)(199.875,32.84765625)
\curveto(200.54686425,33.7539025)(200.88280141,34.98827626)(200.8828125,36.55078125)
\curveto(200.88280141,38.10546064)(200.54686425,39.33592816)(199.875,40.2421875)
\curveto(199.20311559,41.15623884)(198.29296025,41.61326964)(197.14453125,41.61328125)
\moveto(197.14453125,43.44140625)
\curveto(199.01952203,43.44139281)(200.4921768,42.83201842)(201.5625,41.61328125)
\curveto(202.63279966,40.39452086)(203.16795538,38.70702254)(203.16796875,36.55078125)
\curveto(203.16795538,34.40233935)(202.63279966,32.71484104)(201.5625,31.48828125)
\curveto(200.4921768,30.26953098)(199.01952203,29.66015659)(197.14453125,29.66015625)
\curveto(195.26171329,29.66015659)(193.78515226,30.26953098)(192.71484375,31.48828125)
\curveto(191.65234189,32.71484104)(191.12109243,34.40233935)(191.12109375,36.55078125)
\curveto(191.12109243,38.70702254)(191.65234189,40.39452086)(192.71484375,41.61328125)
\curveto(193.78515226,42.83201842)(195.26171329,43.44139281)(197.14453125,43.44140625)
}
}
{
\newrgbcolor{curcolor}{0 0 0}
\pscustom[linestyle=none,fillstyle=solid,fillcolor=curcolor]
{
\newpath
\moveto(206.5078125,35.1796875)
\lineto(206.5078125,43.125)
\lineto(208.6640625,43.125)
\lineto(208.6640625,35.26171875)
\curveto(208.6640583,34.01952723)(208.90624556,33.08593441)(209.390625,32.4609375)
\curveto(209.87499459,31.84374816)(210.60155637,31.53515471)(211.5703125,31.53515625)
\curveto(212.73436673,31.53515471)(213.65233457,31.90624809)(214.32421875,32.6484375)
\curveto(215.00389571,33.39062161)(215.34373913,34.40233935)(215.34375,35.68359375)
\lineto(215.34375,43.125)
\lineto(217.5,43.125)
\lineto(217.5,30)
\lineto(215.34375,30)
\lineto(215.34375,32.015625)
\curveto(214.82030215,31.21874878)(214.21092776,30.62499938)(213.515625,30.234375)
\curveto(212.82811664,29.85156265)(212.02733619,29.66015659)(211.11328125,29.66015625)
\curveto(209.60546361,29.66015659)(208.46093351,30.12890612)(207.6796875,31.06640625)
\curveto(206.89843507,32.00390425)(206.50781046,33.37499662)(206.5078125,35.1796875)
\moveto(211.93359375,43.44140625)
\lineto(211.93359375,43.44140625)
}
}
{
\newrgbcolor{curcolor}{0 0 0}
\pscustom[linestyle=none,fillstyle=solid,fillcolor=curcolor]
{
\newpath
\moveto(229.5703125,41.109375)
\curveto(229.32811538,41.24998875)(229.06249064,41.35155115)(228.7734375,41.4140625)
\curveto(228.49217871,41.48436352)(228.17967902,41.51951973)(227.8359375,41.51953125)
\curveto(226.61718059,41.51951973)(225.67968152,41.12108263)(225.0234375,40.32421875)
\curveto(224.37499533,39.53514671)(224.0507769,38.3984291)(224.05078125,36.9140625)
\lineto(224.05078125,30)
\lineto(221.8828125,30)
\lineto(221.8828125,43.125)
\lineto(224.05078125,43.125)
\lineto(224.05078125,41.0859375)
\curveto(224.50390145,41.88280062)(225.09374461,42.47264378)(225.8203125,42.85546875)
\curveto(226.54686816,43.2460805)(227.42967977,43.44139281)(228.46875,43.44140625)
\curveto(228.61717859,43.44139281)(228.78124092,43.42967407)(228.9609375,43.40625)
\curveto(229.14061556,43.39061161)(229.33983411,43.36326789)(229.55859375,43.32421875)
\lineto(229.5703125,41.109375)
}
}
{
\newrgbcolor{curcolor}{0 0 0}
\pscustom[linewidth=2,linecolor=curcolor,linestyle=dashed,dash=8 8]
{
\newpath
\moveto(140,40)
\lineto(70.11406,39.73606)
}
}
{
\newrgbcolor{curcolor}{0 0 0}
\pscustom[linestyle=none,fillstyle=solid,fillcolor=curcolor]
{
\newpath
\moveto(129.51948821,44.80093475)
\lineto(142.64865594,40.0292193)
\lineto(129.55590443,35.15847191)
\curveto(131.6397373,38.01283536)(131.61296003,41.9072312)(129.51948821,44.80093475)
\lineto(129.51948821,44.80093475)
\closepath
}
}
{
\newrgbcolor{curcolor}{0 0 0}
\pscustom[linewidth=2.06612587,linecolor=curcolor,linestyle=dashed,dash=8.26450373 8.26450373]
{
\newpath
\moveto(218.76726,260)
\lineto(60,260)
}
}
{
\newrgbcolor{curcolor}{0 0 0}
\pscustom[linestyle=none,fillstyle=solid,fillcolor=curcolor]
{
\newpath
\moveto(207.95904012,265.00052279)
\lineto(221.50358276,260.01985125)
\lineto(207.95903938,255.03918079)
\curveto(210.12289102,257.97976663)(210.11042285,262.00299841)(207.95904012,265.00052279)
\lineto(207.95904012,265.00052279)
\closepath
}
}
{
\newrgbcolor{curcolor}{0 0 0}
\pscustom[linewidth=2.21782422,linecolor=curcolor,linestyle=dashed,dash=8.87129678 8.87129678]
{
\newpath
\moveto(62.086063,350)
\lineto(158.76726,350)
}
}
{
\newrgbcolor{curcolor}{0 0 0}
\pscustom[linestyle=none,fillstyle=solid,fillcolor=curcolor]
{
\newpath
\moveto(147.16548288,355.36766938)
\lineto(161.70448807,350.02130875)
\lineto(147.16548209,344.67494931)
\curveto(149.48820726,347.83143778)(149.47482367,352.15006184)(147.16548288,355.36766938)
\closepath
}
}
{
\newrgbcolor{curcolor}{0 0 0}
\pscustom[linewidth=2,linecolor=curcolor,linestyle=dashed,dash=8 8]
{
\newpath
\moveto(498.76725,350)
\lineto(398.47533,350)
}
}
{
\newrgbcolor{curcolor}{0 0 0}
\pscustom[linestyle=none,fillstyle=solid,fillcolor=curcolor]
{
\newpath
\moveto(408.93763536,345.15951776)
\lineto(395.8265826,349.98078409)
\lineto(408.93763608,354.80204936)
\curveto(406.843038,351.95557628)(406.85510712,348.06110708)(408.93763536,345.15951776)
\closepath
}
}
{
\newrgbcolor{curcolor}{0 0 0}
\pscustom[linewidth=2,linecolor=curcolor,linestyle=dashed,dash=8 8]
{
\newpath
\moveto(498.76725,200)
\lineto(398.62788,200)
}
}
{
\newrgbcolor{curcolor}{0 0 0}
\pscustom[linestyle=none,fillstyle=solid,fillcolor=curcolor]
{
\newpath
\moveto(409.09018536,195.15951776)
\lineto(395.9791326,199.98078409)
\lineto(409.09018608,204.80204936)
\curveto(406.995588,201.95557628)(407.00765712,198.06110708)(409.09018536,195.15951776)
\closepath
}
}
{
\newrgbcolor{curcolor}{0 0 0}
\pscustom[linewidth=2.12666917,linecolor=curcolor,linestyle=dashed,dash=8.50667644 8.50667644]
{
\newpath
\moveto(499.2423,149.83946)
\lineto(412.44065,149.83946)
}
}
{
\newrgbcolor{curcolor}{0 0 0}
\pscustom[linestyle=none,fillstyle=solid,fillcolor=curcolor]
{
\newpath
\moveto(423.56558112,144.69240783)
\lineto(409.62414528,149.81902706)
\lineto(423.56558189,154.94564516)
\curveto(421.33832331,151.91889189)(421.35115682,147.7777681)(423.56558112,144.69240783)
\closepath
}
}
{
\newrgbcolor{curcolor}{0 0 0}
\pscustom[linewidth=2.10198331,linecolor=curcolor,linestyle=dashed,dash=8.40793346 8.40793346]
{
\newpath
\moveto(498.39405,101.30557)
\lineto(398.76725,100)
}
}
{
\newrgbcolor{curcolor}{0 0 0}
\pscustom[linestyle=none,fillstyle=solid,fillcolor=curcolor]
{
\newpath
\moveto(409.82876301,95.05721353)
\lineto(395.98394222,99.94332836)
\lineto(409.6959702,105.19056371)
\curveto(407.53395473,102.17035504)(407.60027145,98.07781803)(409.82876301,95.05721353)
\closepath
}
}
{
\newrgbcolor{curcolor}{0 0 0}
\pscustom[linewidth=2,linecolor=curcolor,linestyle=dashed,dash=8 8]
{
\newpath
\moveto(500,40)
\lineto(411.0551,41.27788)
}
}
{
\newrgbcolor{curcolor}{0 0 0}
\pscustom[linestyle=none,fillstyle=solid,fillcolor=curcolor]
{
\newpath
\moveto(421.44678925,36.28759981)
\lineto(408.40634988,41.29671696)
\lineto(421.58531086,45.92913638)
\curveto(419.45003758,43.11304722)(419.40615902,39.21880651)(421.44678925,36.28759981)
\closepath
}
}
{
\newrgbcolor{curcolor}{0 0 0}
\pscustom[linestyle=none,fillstyle=solid,fillcolor=curcolor]
{
\newpath
\moveto(52.984375,452.578125)
\curveto(54.49477717,452.25519608)(55.67185933,451.58332175)(56.515625,450.5625)
\curveto(57.3697743,449.54165713)(57.7968572,448.28124172)(57.796875,446.78125)
\curveto(57.7968572,444.47916219)(57.00519133,442.69791397)(55.421875,441.4375)
\curveto(53.83852783,440.17708316)(51.58853008,439.54687545)(48.671875,439.546875)
\curveto(47.69270064,439.54687545)(46.68228498,439.64583369)(45.640625,439.84375)
\curveto(44.60937039,440.03124997)(43.54166312,440.31770802)(42.4375,440.703125)
\lineto(42.4375,443.75)
\curveto(43.31249669,443.23958009)(44.27082906,442.85416381)(45.3125,442.59375)
\curveto(46.35416031,442.333331)(47.44270089,442.2031228)(48.578125,442.203125)
\curveto(50.55728111,442.2031228)(52.06248794,442.59374741)(53.09375,443.375)
\curveto(54.13540253,444.15624584)(54.65623534,445.29166137)(54.65625,446.78125)
\curveto(54.65623534,448.15624184)(54.17186083,449.22915744)(53.203125,450)
\curveto(52.24477942,450.78123922)(50.90623909,451.17186383)(49.1875,451.171875)
\lineto(46.46875,451.171875)
\lineto(46.46875,453.765625)
\lineto(49.3125,453.765625)
\curveto(50.86457247,453.76561123)(52.05207128,454.07290259)(52.875,454.6875)
\curveto(53.69790297,455.31248469)(54.10936089,456.20831713)(54.109375,457.375)
\curveto(54.10936089,458.57289809)(53.68227798,459.48956384)(52.828125,460.125)
\curveto(51.98436302,460.77081256)(50.77082256,461.09372891)(49.1875,461.09375)
\curveto(48.32290834,461.09372891)(47.39582594,460.999979)(46.40625,460.8125)
\curveto(45.41666125,460.62497938)(44.32812067,460.333313)(43.140625,459.9375)
\lineto(43.140625,462.75)
\curveto(44.33853733,463.08331025)(45.45832788,463.33331)(46.5,463.5)
\curveto(47.55207578,463.666643)(48.54165812,463.74997625)(49.46875,463.75)
\curveto(51.86457147,463.74997625)(53.76040291,463.2031018)(55.15625,462.109375)
\curveto(56.55206678,461.02602064)(57.24998275,459.55727211)(57.25,457.703125)
\curveto(57.24998275,456.41144192)(56.88019145,455.31769302)(56.140625,454.421875)
\curveto(55.40102627,453.5364448)(54.34894398,452.92186208)(52.984375,452.578125)
}
}
{
\newrgbcolor{curcolor}{0 0 0}
\pscustom[linestyle=none,fillstyle=solid,fillcolor=curcolor]
{
\newpath
\moveto(512.09375,420.578125)
\lineto(504.125,408.125)
\lineto(512.09375,408.125)
\lineto(512.09375,420.578125)
\moveto(511.265625,423.328125)
\lineto(515.234375,423.328125)
\lineto(515.234375,408.125)
\lineto(518.5625,408.125)
\lineto(518.5625,405.5)
\lineto(515.234375,405.5)
\lineto(515.234375,400)
\lineto(512.09375,400)
\lineto(512.09375,405.5)
\lineto(501.5625,405.5)
\lineto(501.5625,408.546875)
\lineto(511.265625,423.328125)
}
}
{
\newrgbcolor{curcolor}{0 0 0}
\pscustom[linestyle=none,fillstyle=solid,fillcolor=curcolor]
{
\newpath
\moveto(43.453125,363.328125)
\lineto(55.84375,363.328125)
\lineto(55.84375,360.671875)
\lineto(46.34375,360.671875)
\lineto(46.34375,354.953125)
\curveto(46.80207653,355.10935989)(47.26040941,355.22394311)(47.71875,355.296875)
\curveto(48.17707516,355.38019295)(48.63540803,355.42185958)(49.09375,355.421875)
\curveto(51.69790497,355.42185958)(53.76040291,354.70831863)(55.28125,353.28125)
\curveto(56.80206653,351.85415481)(57.56248244,349.92186508)(57.5625,347.484375)
\curveto(57.56248244,344.97395336)(56.78123322,343.02083031)(55.21875,341.625)
\curveto(53.65623634,340.23958309)(51.45311355,339.54687545)(48.609375,339.546875)
\curveto(47.6302007,339.54687545)(46.6302017,339.6302087)(45.609375,339.796875)
\curveto(44.59895373,339.9635417)(43.55207978,340.21354145)(42.46875,340.546875)
\lineto(42.46875,343.71875)
\curveto(43.40624659,343.20833012)(44.37499562,342.82812217)(45.375,342.578125)
\curveto(46.37499363,342.32812267)(47.43228423,342.2031228)(48.546875,342.203125)
\curveto(50.34894798,342.2031228)(51.77602989,342.67708066)(52.828125,343.625)
\curveto(53.88019445,344.57291209)(54.40623559,345.85936914)(54.40625,347.484375)
\curveto(54.40623559,349.10936589)(53.88019445,350.39582294)(52.828125,351.34375)
\curveto(51.77602989,352.29165438)(50.34894798,352.76561223)(48.546875,352.765625)
\curveto(47.7031173,352.76561223)(46.85936814,352.67186233)(46.015625,352.484375)
\curveto(45.18228648,352.2968627)(44.32812067,352.00519633)(43.453125,351.609375)
\lineto(43.453125,363.328125)
}
}
{
\newrgbcolor{curcolor}{0 0 0}
\pscustom[linestyle=none,fillstyle=solid,fillcolor=curcolor]
{
\newpath
\moveto(510.5625,352.921875)
\curveto(509.14582419,352.92186208)(508.02082531,352.43748756)(507.1875,351.46875)
\curveto(506.36457697,350.4999895)(505.95311905,349.17186583)(505.953125,347.484375)
\curveto(505.95311905,345.80728586)(506.36457697,344.47916219)(507.1875,343.5)
\curveto(508.02082531,342.53124747)(509.14582419,342.04687295)(510.5625,342.046875)
\curveto(511.97915469,342.04687295)(513.09894523,342.53124747)(513.921875,343.5)
\curveto(514.75519358,344.47916219)(515.17185983,345.80728586)(515.171875,347.484375)
\curveto(515.17185983,349.17186583)(514.75519358,350.4999895)(513.921875,351.46875)
\curveto(513.09894523,352.43748756)(511.97915469,352.92186208)(510.5625,352.921875)
\moveto(516.828125,362.8125)
\lineto(516.828125,359.9375)
\curveto(516.0364423,360.31247969)(515.23435977,360.59893773)(514.421875,360.796875)
\curveto(513.61977805,360.99477067)(512.82290384,361.09372891)(512.03125,361.09375)
\curveto(509.94790672,361.09372891)(508.35415831,360.39060461)(507.25,358.984375)
\curveto(506.15624384,357.57810742)(505.53124447,355.45310955)(505.375,352.609375)
\curveto(505.98957734,353.51561148)(506.76040991,354.20831913)(507.6875,354.6875)
\curveto(508.61457472,355.17706816)(509.63540703,355.42185958)(510.75,355.421875)
\curveto(513.09373691,355.42185958)(514.94269339,354.70831863)(516.296875,353.28125)
\curveto(517.66144067,351.86457147)(518.34373166,349.93228173)(518.34375,347.484375)
\curveto(518.34373166,345.08853658)(517.63539903,343.1666635)(516.21875,341.71875)
\curveto(514.80206853,340.27083306)(512.91665375,339.54687545)(510.5625,339.546875)
\curveto(507.86457547,339.54687545)(505.80207753,340.57812442)(504.375,342.640625)
\curveto(502.94791372,344.71353695)(502.23437277,347.71353395)(502.234375,351.640625)
\curveto(502.23437277,355.32810967)(503.10937189,358.26560673)(504.859375,360.453125)
\curveto(506.60936839,362.65101902)(508.95832438,363.74997625)(511.90625,363.75)
\curveto(512.69790397,363.74997625)(513.49477817,363.67185133)(514.296875,363.515625)
\curveto(515.10935989,363.35935164)(515.95310905,363.12497688)(516.828125,362.8125)
}
}
{
\newrgbcolor{curcolor}{0 0 0}
\pscustom[linestyle=none,fillstyle=solid,fillcolor=curcolor]
{
\newpath
\moveto(42.625,273.328125)
\lineto(57.625,273.328125)
\lineto(57.625,271.984375)
\lineto(49.15625,250)
\lineto(45.859375,250)
\lineto(53.828125,270.671875)
\lineto(42.625,270.671875)
\lineto(42.625,273.328125)
}
}
{
\newrgbcolor{curcolor}{0 0 0}
\pscustom[linestyle=none,fillstyle=solid,fillcolor=curcolor]
{
\newpath
\moveto(510.171875,201.078125)
\curveto(508.67186633,201.07811392)(507.48957584,200.67707266)(506.625,199.875)
\curveto(505.77082756,199.07290759)(505.34374466,197.96874203)(505.34375,196.5625)
\curveto(505.34374466,195.15624484)(505.77082756,194.05207928)(506.625,193.25)
\curveto(507.48957584,192.44791422)(508.67186633,192.04687295)(510.171875,192.046875)
\curveto(511.67186333,192.04687295)(512.85415381,192.44791422)(513.71875,193.25)
\curveto(514.58331875,194.06249594)(515.01560998,195.1666615)(515.015625,196.5625)
\curveto(515.01560998,197.96874203)(514.58331875,199.07290759)(513.71875,199.875)
\curveto(512.86457047,200.67707266)(511.68227998,201.07811392)(510.171875,201.078125)
\moveto(507.015625,202.421875)
\curveto(505.66145267,202.75519558)(504.60416206,203.38540328)(503.84375,204.3125)
\curveto(503.09374691,205.23956809)(502.71874728,206.3697753)(502.71875,207.703125)
\curveto(502.71874728,209.56768877)(503.38020495,211.04164563)(504.703125,212.125)
\curveto(506.0364523,213.20831013)(507.85936714,213.74997625)(510.171875,213.75)
\curveto(512.49477917,213.74997625)(514.31769402,213.20831013)(515.640625,212.125)
\curveto(516.9635247,211.04164563)(517.62498237,209.56768877)(517.625,207.703125)
\curveto(517.62498237,206.3697753)(517.24477442,205.23956809)(516.484375,204.3125)
\curveto(515.73435927,203.38540328)(514.68748531,202.75519558)(513.34375,202.421875)
\curveto(514.86456847,202.06769627)(516.04685895,201.37498863)(516.890625,200.34375)
\curveto(517.74477392,199.31249069)(518.17185683,198.05207528)(518.171875,196.5625)
\curveto(518.17185683,194.30207903)(517.47914919,192.56770577)(516.09375,191.359375)
\curveto(514.71873528,190.15104152)(512.74477892,189.54687545)(510.171875,189.546875)
\curveto(507.59895073,189.54687545)(505.61978605,190.15104152)(504.234375,191.359375)
\curveto(502.85937214,192.56770577)(502.17187283,194.30207903)(502.171875,196.5625)
\curveto(502.17187283,198.05207528)(502.59895573,199.31249069)(503.453125,200.34375)
\curveto(504.30728736,201.37498863)(505.49478617,202.06769627)(507.015625,202.421875)
\moveto(505.859375,207.40625)
\curveto(505.85936914,206.19790047)(506.23436877,205.25519308)(506.984375,204.578125)
\curveto(507.74478392,203.90102777)(508.80728286,203.56248644)(510.171875,203.5625)
\curveto(511.52603014,203.56248644)(512.58332075,203.90102777)(513.34375,204.578125)
\curveto(514.11456922,205.25519308)(514.4999855,206.19790047)(514.5,207.40625)
\curveto(514.4999855,208.61456472)(514.11456922,209.55727211)(513.34375,210.234375)
\curveto(512.58332075,210.91143742)(511.52603014,211.24997875)(510.171875,211.25)
\curveto(508.80728286,211.24997875)(507.74478392,210.91143742)(506.984375,210.234375)
\curveto(506.23436877,209.55727211)(505.85936914,208.61456472)(505.859375,207.40625)
}
}
{
\newrgbcolor{curcolor}{0 0 0}
\pscustom[linestyle=none,fillstyle=solid,fillcolor=curcolor]
{
\newpath
\moveto(503.515625,140.484375)
\lineto(503.515625,143.359375)
\curveto(504.30728736,142.98437202)(505.10936989,142.69791397)(505.921875,142.5)
\curveto(506.73436827,142.30208103)(507.53124247,142.2031228)(508.3125,142.203125)
\curveto(510.39582294,142.2031228)(511.98436302,142.90103877)(513.078125,144.296875)
\curveto(514.18227748,145.7031193)(514.81248519,147.8333255)(514.96875,150.6875)
\curveto(514.36456897,149.79165688)(513.59894473,149.10415756)(512.671875,148.625)
\curveto(511.74477992,148.14582519)(510.71873928,147.90624209)(509.59375,147.90625)
\curveto(507.26040941,147.90624209)(505.41145292,148.60936639)(504.046875,150.015625)
\curveto(502.69270564,151.43228023)(502.01562298,153.36456997)(502.015625,155.8125)
\curveto(502.01562298,158.20831513)(502.72395561,160.1301882)(504.140625,161.578125)
\curveto(505.55728611,163.02601864)(507.44270089,163.74997625)(509.796875,163.75)
\curveto(512.49477917,163.74997625)(514.55206878,162.71351895)(515.96875,160.640625)
\curveto(517.39581594,158.57810642)(518.10935689,155.57810942)(518.109375,151.640625)
\curveto(518.10935689,147.9635337)(517.23435777,145.02603664)(515.484375,142.828125)
\curveto(513.74477792,140.64062436)(511.40103027,139.54687545)(508.453125,139.546875)
\curveto(507.66145067,139.54687545)(506.85936814,139.62500038)(506.046875,139.78125)
\curveto(505.23436977,139.93750006)(504.39062061,140.17187483)(503.515625,140.484375)
\moveto(509.796875,150.375)
\curveto(511.21353045,150.37498963)(512.333321,150.85936414)(513.15625,151.828125)
\curveto(513.98956934,152.7968622)(514.40623559,154.12498587)(514.40625,155.8125)
\curveto(514.40623559,157.48956584)(513.98956934,158.81248119)(513.15625,159.78125)
\curveto(512.333321,160.76039591)(511.21353045,161.24997875)(509.796875,161.25)
\curveto(508.38019995,161.24997875)(507.25520108,160.76039591)(506.421875,159.78125)
\curveto(505.59895273,158.81248119)(505.18749481,157.48956584)(505.1875,155.8125)
\curveto(505.18749481,154.12498587)(505.59895273,152.7968622)(506.421875,151.828125)
\curveto(507.25520108,150.85936414)(508.38019995,150.37498963)(509.796875,150.375)
}
}
{
\newrgbcolor{curcolor}{0 0 0}
\pscustom[linewidth=2,linecolor=curcolor,linestyle=dashed,dash=8 8]
{
\newpath
\moveto(59.189408,539.999593)
\lineto(139.51115,539.999593)
}
}
{
\newrgbcolor{curcolor}{0 0 0}
\pscustom[linestyle=none,fillstyle=solid,fillcolor=curcolor]
{
\newpath
\moveto(129.04884464,544.84007524)
\lineto(142.1598974,540.01880891)
\lineto(129.04884392,535.19754364)
\curveto(131.143442,538.04401672)(131.13137288,541.93848592)(129.04884464,544.84007524)
\closepath
}
}
{
\newrgbcolor{curcolor}{0 0 0}
\pscustom[linestyle=none,fillstyle=solid,fillcolor=curcolor]
{
\newpath
\moveto(43.96875,532.65625)
\lineto(49.125,532.65625)
\lineto(49.125,550.453125)
\lineto(43.515625,549.328125)
\lineto(43.515625,552.203125)
\lineto(49.09375,553.328125)
\lineto(52.25,553.328125)
\lineto(52.25,532.65625)
\lineto(57.40625,532.65625)
\lineto(57.40625,530)
\lineto(43.96875,530)
\lineto(43.96875,532.65625)
}
}
{
\newrgbcolor{curcolor}{0 0 0}
\pscustom[linestyle=none,fillstyle=solid,fillcolor=curcolor]
{
\newpath
\moveto(506.140625,482.65625)
\lineto(517.15625,482.65625)
\lineto(517.15625,480)
\lineto(502.34375,480)
\lineto(502.34375,482.65625)
\curveto(503.54166313,483.89582944)(505.17186983,485.55728611)(507.234375,487.640625)
\curveto(509.30728236,489.73436527)(510.60936439,491.08332225)(511.140625,491.6875)
\curveto(512.15102952,492.82290384)(512.85415381,493.78123622)(513.25,494.5625)
\curveto(513.65623634,495.35415131)(513.85936114,496.1301922)(513.859375,496.890625)
\curveto(513.85936114,498.1301902)(513.42186158,499.14060586)(512.546875,499.921875)
\curveto(511.68227998,500.7031043)(510.55207278,501.09372891)(509.15625,501.09375)
\curveto(508.1666585,501.09372891)(507.11978455,500.92185408)(506.015625,500.578125)
\curveto(504.92187008,500.23435477)(503.74999625,499.71352195)(502.5,499.015625)
\lineto(502.5,502.203125)
\curveto(503.77082956,502.71351895)(504.95832838,503.09893523)(506.0625,503.359375)
\curveto(507.1666595,503.61976805)(508.17707516,503.74997625)(509.09375,503.75)
\curveto(511.51040516,503.74997625)(513.43748656,503.14581019)(514.875,501.9375)
\curveto(516.31248369,500.72914594)(517.03123297,499.11456422)(517.03125,497.09375)
\curveto(517.03123297,496.13540053)(516.84894148,495.22394311)(516.484375,494.359375)
\curveto(516.1301922,493.50519483)(515.47915119,492.49477917)(514.53125,491.328125)
\curveto(514.27081906,491.02603064)(513.44269489,490.15103152)(512.046875,488.703125)
\curveto(510.65103102,487.26561773)(508.68228298,485.24999475)(506.140625,482.65625)
}
}
{
\newrgbcolor{curcolor}{0 0 0}
\pscustom[linestyle=none,fillstyle=solid,fillcolor=curcolor]
{
\newpath
\moveto(503.96875,92.65625)
\lineto(509.125,92.65625)
\lineto(509.125,110.453125)
\lineto(503.515625,109.328125)
\lineto(503.515625,112.203125)
\lineto(509.09375,113.328125)
\lineto(512.25,113.328125)
\lineto(512.25,92.65625)
\lineto(517.40625,92.65625)
\lineto(517.40625,90)
\lineto(503.96875,90)
\lineto(503.96875,92.65625)
}
}
{
\newrgbcolor{curcolor}{0 0 0}
\pscustom[linestyle=none,fillstyle=solid,fillcolor=curcolor]
{
\newpath
\moveto(530.546875,111.25)
\curveto(528.92186645,111.24997875)(527.69790934,110.44789622)(526.875,108.84375)
\curveto(526.06249431,107.24998275)(525.65624472,104.84894348)(525.65625,101.640625)
\curveto(525.65624472,98.44269989)(526.06249431,96.04166063)(526.875,94.4375)
\curveto(527.69790934,92.84374716)(528.92186645,92.04687295)(530.546875,92.046875)
\curveto(532.18227986,92.04687295)(533.40623697,92.84374716)(534.21875,94.4375)
\curveto(535.041652,96.04166063)(535.45310992,98.44269989)(535.453125,101.640625)
\curveto(535.45310992,104.84894348)(535.041652,107.24998275)(534.21875,108.84375)
\curveto(533.40623697,110.44789622)(532.18227986,111.24997875)(530.546875,111.25)
\moveto(530.546875,113.75)
\curveto(533.16144555,113.74997625)(535.15623522,112.71351895)(536.53125,110.640625)
\curveto(537.91664913,108.57810642)(538.60935677,105.57810942)(538.609375,101.640625)
\curveto(538.60935677,97.71353395)(537.91664913,94.71353695)(536.53125,92.640625)
\curveto(535.15623522,90.57812442)(533.16144555,89.54687545)(530.546875,89.546875)
\curveto(527.93228411,89.54687545)(525.93228611,90.57812442)(524.546875,92.640625)
\curveto(523.1718722,94.71353695)(522.48437289,97.71353395)(522.484375,101.640625)
\curveto(522.48437289,105.57810942)(523.1718722,108.57810642)(524.546875,110.640625)
\curveto(525.93228611,112.71351895)(527.93228411,113.74997625)(530.546875,113.75)
}
}
{
\newrgbcolor{curcolor}{0 0 0}
\pscustom[linestyle=none,fillstyle=solid,fillcolor=curcolor,opacity=0.11935484]
{
\newpath
\moveto(312.83469105,220)
\lineto(417.41930676,220)
\curveto(424.5297256,220)(430.2539978,215.83353307)(430.2539978,210.6581459)
\lineto(430.2539978,199.6365757)
\curveto(430.2539978,194.46118853)(424.5297256,190.2947216)(417.41930676,190.2947216)
\lineto(312.83469105,190.2947216)
\curveto(305.72427221,190.2947216)(300,194.46118853)(300,199.6365757)
\lineto(300,210.6581459)
\curveto(300,215.83353307)(305.72427221,220)(312.83469105,220)
\closepath
}
}
{
\newrgbcolor{curcolor}{0 0 0}
\pscustom[linewidth=2,linecolor=curcolor]
{
\newpath
\moveto(312.83469105,220)
\lineto(417.41930676,220)
\curveto(424.5297256,220)(430.2539978,215.83353307)(430.2539978,210.6581459)
\lineto(430.2539978,199.6365757)
\curveto(430.2539978,194.46118853)(424.5297256,190.2947216)(417.41930676,190.2947216)
\lineto(312.83469105,190.2947216)
\curveto(305.72427221,190.2947216)(300,194.46118853)(300,199.6365757)
\lineto(300,210.6581459)
\curveto(300,215.83353307)(305.72427221,220)(312.83469105,220)
\closepath
}
}
{
\newrgbcolor{curcolor}{0 0 0}
\pscustom[linewidth=2,linecolor=curcolor,linestyle=dashed,dash=8 8]
{
\newpath
\moveto(420,490)
\lineto(500,490)
}
}
{
\newrgbcolor{curcolor}{0 0 0}
\pscustom[linestyle=none,fillstyle=solid,fillcolor=curcolor]
{
\newpath
\moveto(430.46230536,485.15951776)
\lineto(417.3512526,489.98078409)
\lineto(430.46230608,494.80204936)
\curveto(428.367708,491.95557628)(428.37977712,488.06110708)(430.46230536,485.15951776)
\lineto(430.46230536,485.15951776)
\closepath
}
}
{
\newrgbcolor{curcolor}{0 0 0}
\pscustom[linewidth=2,linecolor=curcolor,linestyle=dashed,dash=8 8]
{
\newpath
\moveto(310,390)
\lineto(500,410)
}
}
{
\newrgbcolor{curcolor}{0 0 0}
\pscustom[linestyle=none,fillstyle=solid,fillcolor=curcolor]
{
\newpath
\moveto(320.91154457,386.2813582)
\lineto(307.36781788,389.70360612)
\lineto(319.90211894,395.87090853)
\curveto(318.11701215,392.82080298)(318.53670663,388.94899558)(320.91154457,386.2813582)
\lineto(320.91154457,386.2813582)
\closepath
}
}
{
\newrgbcolor{curcolor}{0 0 0}
\pscustom[linestyle=none,fillstyle=solid,fillcolor=curcolor]
{
\newpath
\moveto(43.96875,32.65625)
\lineto(49.125,32.65625)
\lineto(49.125,50.453125)
\lineto(43.515625,49.328125)
\lineto(43.515625,52.203125)
\lineto(49.09375,53.328125)
\lineto(52.25,53.328125)
\lineto(52.25,32.65625)
\lineto(57.40625,32.65625)
\lineto(57.40625,30)
\lineto(43.96875,30)
\lineto(43.96875,32.65625)
}
}
{
\newrgbcolor{curcolor}{0 0 0}
\pscustom[linestyle=none,fillstyle=solid,fillcolor=curcolor]
{
\newpath
\moveto(64.34375,32.65625)
\lineto(69.5,32.65625)
\lineto(69.5,50.453125)
\lineto(63.890625,49.328125)
\lineto(63.890625,52.203125)
\lineto(69.46875,53.328125)
\lineto(72.625,53.328125)
\lineto(72.625,32.65625)
\lineto(77.78125,32.65625)
\lineto(77.78125,30)
\lineto(64.34375,30)
\lineto(64.34375,32.65625)
}
}
{
\newrgbcolor{curcolor}{1 1 1}
\pscustom[linestyle=none,fillstyle=solid,fillcolor=curcolor]
{
\newpath
\moveto(160,460)
\lineto(400,460)
\lineto(400,430)
\lineto(160,430)
\closepath
}
}
{
\newrgbcolor{curcolor}{1 0 0}
\pscustom[linewidth=2.04672694,linecolor=curcolor,linestyle=dashed,dash=8.18690742 8.18690742]
{
\newpath
\moveto(160,460)
\lineto(400,460)
\lineto(400,430)
\lineto(160,430)
\closepath
}
}
{
\newrgbcolor{curcolor}{0 0 0}
\pscustom[linewidth=2,linecolor=curcolor,linestyle=dashed,dash=8 8]
{
\newpath
\moveto(190,450)
\lineto(60,450)
}
}
{
\newrgbcolor{curcolor}{0 0 0}
\pscustom[linestyle=none,fillstyle=solid,fillcolor=curcolor]
{
\newpath
\moveto(179.53769464,454.84048224)
\lineto(192.6487474,450.01921591)
\lineto(179.53769392,445.19795064)
\curveto(181.632292,448.04442372)(181.62022288,451.93889292)(179.53769464,454.84048224)
\lineto(179.53769464,454.84048224)
\closepath
}
}
{
\newrgbcolor{curcolor}{0 0 0}
\pscustom[linestyle=none,fillstyle=solid,fillcolor=curcolor]
{
\newpath
\moveto(503.96875,32.65625)
\lineto(509.125,32.65625)
\lineto(509.125,50.453125)
\lineto(503.515625,49.328125)
\lineto(503.515625,52.203125)
\lineto(509.09375,53.328125)
\lineto(512.25,53.328125)
\lineto(512.25,32.65625)
\lineto(517.40625,32.65625)
\lineto(517.40625,30)
\lineto(503.96875,30)
\lineto(503.96875,32.65625)
}
}
{
\newrgbcolor{curcolor}{0 0 0}
\pscustom[linestyle=none,fillstyle=solid,fillcolor=curcolor]
{
\newpath
\moveto(526.515625,32.65625)
\lineto(537.53125,32.65625)
\lineto(537.53125,30)
\lineto(522.71875,30)
\lineto(522.71875,32.65625)
\curveto(523.91666313,33.89582944)(525.54686983,35.55728611)(527.609375,37.640625)
\curveto(529.68228236,39.73436527)(530.98436439,41.08332225)(531.515625,41.6875)
\curveto(532.52602952,42.82290384)(533.22915381,43.78123622)(533.625,44.5625)
\curveto(534.03123634,45.35415131)(534.23436114,46.1301922)(534.234375,46.890625)
\curveto(534.23436114,48.1301902)(533.79686158,49.14060586)(532.921875,49.921875)
\curveto(532.05727998,50.7031043)(530.92707278,51.09372891)(529.53125,51.09375)
\curveto(528.5416585,51.09372891)(527.49478455,50.92185408)(526.390625,50.578125)
\curveto(525.29687008,50.23435477)(524.12499625,49.71352195)(522.875,49.015625)
\lineto(522.875,52.203125)
\curveto(524.14582956,52.71351895)(525.33332837,53.09893523)(526.4375,53.359375)
\curveto(527.5416595,53.61976805)(528.55207516,53.74997625)(529.46875,53.75)
\curveto(531.88540516,53.74997625)(533.81248656,53.14581019)(535.25,51.9375)
\curveto(536.68748369,50.72914594)(537.40623297,49.11456422)(537.40625,47.09375)
\curveto(537.40623297,46.13540053)(537.22394148,45.22394311)(536.859375,44.359375)
\curveto(536.5051922,43.50519483)(535.85415119,42.49477917)(534.90625,41.328125)
\curveto(534.64581906,41.02603064)(533.81769489,40.15103152)(532.421875,38.703125)
\curveto(531.02603102,37.26561773)(529.05728298,35.24999475)(526.515625,32.65625)
}
}
{
\newrgbcolor{curcolor}{0 0 0}
\pscustom[linestyle=none,fillstyle=solid,fillcolor=curcolor]
{
\newpath
\moveto(112.15917969,496.03808594)
\lineto(115.08105469,496.03808594)
\lineto(122.19238281,482.62109375)
\lineto(122.19238281,496.03808594)
\lineto(124.29785156,496.03808594)
\lineto(124.29785156,480)
\lineto(121.37597656,480)
\lineto(114.26464844,493.41699219)
\lineto(114.26464844,480)
\lineto(112.15917969,480)
\lineto(112.15917969,496.03808594)
}
}
{
\newrgbcolor{curcolor}{0 0 0}
\pscustom[linestyle=none,fillstyle=solid,fillcolor=curcolor]
{
\newpath
\moveto(133.19238281,490.64550781)
\curveto(132.1324813,490.64549717)(131.29459152,490.230133)(130.67871094,489.39941406)
\curveto(130.06282192,488.57583778)(129.75487951,487.44432849)(129.75488281,486.00488281)
\curveto(129.75487951,484.56542512)(130.05924119,483.43033511)(130.66796875,482.59960938)
\curveto(131.28384934,481.77603989)(132.12531985,481.36425645)(133.19238281,481.36425781)
\curveto(134.2451094,481.36425645)(135.07941846,481.77962062)(135.6953125,482.61035156)
\curveto(136.31118806,483.44107729)(136.61913046,484.57258657)(136.61914062,486.00488281)
\curveto(136.61913046,487.43000559)(136.31118806,488.55793415)(135.6953125,489.38867188)
\curveto(135.07941846,490.22655227)(134.2451094,490.64549717)(133.19238281,490.64550781)
\moveto(133.19238281,492.32128906)
\curveto(134.91112436,492.32127674)(136.2610579,491.76268355)(137.2421875,490.64550781)
\curveto(138.22329553,489.52831078)(138.71385493,487.98143733)(138.71386719,486.00488281)
\curveto(138.71385493,484.03547774)(138.22329553,482.48860428)(137.2421875,481.36425781)
\curveto(136.2610579,480.24707007)(134.91112436,479.68847687)(133.19238281,479.68847656)
\curveto(131.46646634,479.68847687)(130.11295207,480.24707007)(129.13183594,481.36425781)
\curveto(128.1578759,482.48860428)(127.67089722,484.03547774)(127.67089844,486.00488281)
\curveto(127.67089722,487.98143733)(128.1578759,489.52831078)(129.13183594,490.64550781)
\curveto(130.11295207,491.76268355)(131.46646634,492.32127674)(133.19238281,492.32128906)
}
}
{
\newrgbcolor{curcolor}{0 0 0}
\pscustom[linestyle=none,fillstyle=solid,fillcolor=curcolor]
{
\newpath
\moveto(151.34667969,489.72167969)
\curveto(151.84080838,490.60968991)(152.4316281,491.26496269)(153.11914062,491.6875)
\curveto(153.80662672,492.11001393)(154.61587071,492.32127674)(155.546875,492.32128906)
\curveto(156.80011331,492.32127674)(157.76690922,491.88084749)(158.44726562,491)
\curveto(159.12758495,490.12629196)(159.46775388,488.88019945)(159.46777344,487.26171875)
\lineto(159.46777344,480)
\lineto(157.48046875,480)
\lineto(157.48046875,487.19726562)
\curveto(157.48045118,488.35025207)(157.27634982,489.20604548)(156.86816406,489.76464844)
\curveto(156.45994438,490.32323186)(155.83689813,490.60252846)(154.99902344,490.60253906)
\curveto(153.97492083,490.60252846)(153.16567684,490.26235953)(152.57128906,489.58203125)
\curveto(151.97687595,488.90168381)(151.67967573,487.97427588)(151.6796875,486.79980469)
\lineto(151.6796875,480)
\lineto(149.69238281,480)
\lineto(149.69238281,487.19726562)
\curveto(149.69237303,488.35741352)(149.48827167,489.21320693)(149.08007812,489.76464844)
\curveto(148.67186623,490.32323186)(148.04165853,490.60252846)(147.18945312,490.60253906)
\curveto(146.17968123,490.60252846)(145.3775987,490.2587788)(144.78320312,489.57128906)
\curveto(144.1887978,488.89094163)(143.89159758,487.96711443)(143.89160156,486.79980469)
\lineto(143.89160156,480)
\lineto(141.90429688,480)
\lineto(141.90429688,492.03125)
\lineto(143.89160156,492.03125)
\lineto(143.89160156,490.16210938)
\curveto(144.342769,490.89972868)(144.88345856,491.44399897)(145.51367188,491.79492188)
\curveto(146.14387397,492.14582119)(146.89224562,492.32127674)(147.75878906,492.32128906)
\curveto(148.63247825,492.32127674)(149.37368845,492.09927175)(149.98242188,491.65527344)
\curveto(150.5982966,491.21125181)(151.05304875,490.5667212)(151.34667969,489.72167969)
}
}
{
\newrgbcolor{curcolor}{0 0 0}
\pscustom[linestyle=none,fillstyle=solid,fillcolor=curcolor]
{
}
}
{
\newrgbcolor{curcolor}{0 0 0}
\pscustom[linestyle=none,fillstyle=solid,fillcolor=curcolor]
{
\newpath
\moveto(178.34179688,490.20507812)
\lineto(178.34179688,496.71484375)
\lineto(180.31835938,496.71484375)
\lineto(180.31835938,480)
\lineto(178.34179688,480)
\lineto(178.34179688,481.8046875)
\curveto(177.92642272,481.08854058)(177.40005606,480.55501247)(176.76269531,480.20410156)
\curveto(176.1324792,479.8603517)(175.37336537,479.68847687)(174.48535156,479.68847656)
\curveto(173.03157084,479.68847687)(171.84635067,480.26855442)(170.9296875,481.42871094)
\curveto(170.02018062,482.5888646)(169.56542847,484.1142537)(169.56542969,486.00488281)
\curveto(169.56542847,487.89549992)(170.02018062,489.42088902)(170.9296875,490.58105469)
\curveto(171.84635067,491.7411992)(173.03157084,492.32127674)(174.48535156,492.32128906)
\curveto(175.37336537,492.32127674)(176.1324792,492.14582119)(176.76269531,491.79492188)
\curveto(177.40005606,491.45116042)(177.92642272,490.92121304)(178.34179688,490.20507812)
\moveto(171.60644531,486.00488281)
\curveto(171.60644206,484.55110222)(171.90364228,483.40885076)(172.49804688,482.578125)
\curveto(173.09960463,481.75455554)(173.92317151,481.34277209)(174.96875,481.34277344)
\curveto(176.01431525,481.34277209)(176.83788214,481.75455554)(177.43945312,482.578125)
\curveto(178.04100594,483.40885076)(178.34178688,484.55110222)(178.34179688,486.00488281)
\curveto(178.34178688,487.4586514)(178.04100594,488.59732213)(177.43945312,489.42089844)
\curveto(176.83788214,490.25161735)(176.01431525,490.66698152)(174.96875,490.66699219)
\curveto(173.92317151,490.66698152)(173.09960463,490.25161735)(172.49804688,489.42089844)
\curveto(171.90364228,488.59732213)(171.60644206,487.4586514)(171.60644531,486.00488281)
}
}
{
\newrgbcolor{curcolor}{0 0 0}
\pscustom[linestyle=none,fillstyle=solid,fillcolor=curcolor]
{
\newpath
\moveto(194.68066406,486.50976562)
\lineto(194.68066406,485.54296875)
\lineto(185.59277344,485.54296875)
\curveto(185.67870758,484.18228748)(186.08691029,483.14387706)(186.81738281,482.42773438)
\curveto(187.55500778,481.71874828)(188.5790953,481.36425645)(189.88964844,481.36425781)
\curveto(190.64875469,481.36425645)(191.38280343,481.45735531)(192.09179688,481.64355469)
\curveto(192.80793222,481.82975077)(193.51691588,482.10904737)(194.21875,482.48144531)
\lineto(194.21875,480.61230469)
\curveto(193.50975443,480.31152313)(192.78286714,480.08235669)(192.03808594,479.92480469)
\curveto(191.29328529,479.76725284)(190.5377522,479.68847687)(189.77148438,479.68847656)
\curveto(187.85220801,479.68847687)(186.33039963,480.24707007)(185.20605469,481.36425781)
\curveto(184.08886542,482.48144283)(183.53027222,483.99250903)(183.53027344,485.89746094)
\curveto(183.53027222,487.86685411)(184.06021961,489.42805047)(185.12011719,490.58105469)
\curveto(186.18717061,491.7411992)(187.62304157,492.32127674)(189.42773438,492.32128906)
\curveto(191.04621523,492.32127674)(192.32453426,491.79849081)(193.26269531,490.75292969)
\curveto(194.20799592,489.71450851)(194.6806517,488.30012191)(194.68066406,486.50976562)
\moveto(192.70410156,487.08984375)
\curveto(192.68976827,488.17121579)(192.3854066,489.03417065)(191.79101562,489.67871094)
\curveto(191.20376715,490.32323186)(190.42316898,490.64549717)(189.44921875,490.64550781)
\curveto(188.34634814,490.64549717)(187.46190892,490.33397404)(186.79589844,489.7109375)
\curveto(186.13704045,489.08788154)(185.75748354,488.21060377)(185.65722656,487.07910156)
\lineto(192.70410156,487.08984375)
}
}
{
\newrgbcolor{curcolor}{0 0 0}
\pscustom[linestyle=none,fillstyle=solid,fillcolor=curcolor]
{
}
}
{
\newrgbcolor{curcolor}{0 0 0}
\pscustom[linestyle=none,fillstyle=solid,fillcolor=curcolor]
{
\newpath
\moveto(204.92871094,496.71484375)
\lineto(206.90527344,496.71484375)
\lineto(206.90527344,480)
\lineto(204.92871094,480)
\lineto(204.92871094,496.71484375)
}
}
{
\newrgbcolor{curcolor}{0 0 0}
\pscustom[linestyle=none,fillstyle=solid,fillcolor=curcolor]
{
\newpath
\moveto(216.49804688,486.04785156)
\curveto(214.90103572,486.04784551)(213.79459152,485.86522851)(213.17871094,485.5)
\curveto(212.56282192,485.13476049)(212.25487951,484.51171424)(212.25488281,483.63085938)
\curveto(212.25487951,482.92903353)(212.48404595,482.37044034)(212.94238281,481.95507812)
\curveto(213.40787315,481.54687345)(214.03808086,481.34277209)(214.83300781,481.34277344)
\curveto(215.92870397,481.34277209)(216.80598173,481.72949046)(217.46484375,482.50292969)
\curveto(218.1308502,483.28352536)(218.46385768,484.31835506)(218.46386719,485.60742188)
\lineto(218.46386719,486.04785156)
\lineto(216.49804688,486.04785156)
\moveto(220.44042969,486.86425781)
\lineto(220.44042969,480)
\lineto(218.46386719,480)
\lineto(218.46386719,481.82617188)
\curveto(218.01268626,481.09570203)(217.45051234,480.55501247)(216.77734375,480.20410156)
\curveto(216.10415952,479.8603517)(215.28059263,479.68847687)(214.30664062,479.68847656)
\curveto(213.07486567,479.68847687)(212.09374686,480.03222653)(211.36328125,480.71972656)
\curveto(210.63997228,481.41438661)(210.27831899,482.34179453)(210.27832031,483.50195312)
\curveto(210.27831899,484.85546389)(210.72949042,485.87597069)(211.63183594,486.56347656)
\curveto(212.54133756,487.25096931)(213.89485183,487.59471897)(215.69238281,487.59472656)
\lineto(218.46386719,487.59472656)
\lineto(218.46386719,487.78808594)
\curveto(218.46385768,488.69758245)(218.16307673,489.39940466)(217.56152344,489.89355469)
\curveto(216.96711439,490.39484638)(216.1292246,490.64549717)(215.04785156,490.64550781)
\curveto(214.36034616,490.64549717)(213.69075047,490.56314048)(213.0390625,490.3984375)
\curveto(212.38736636,490.23371372)(211.76073938,489.98664366)(211.15917969,489.65722656)
\lineto(211.15917969,491.48339844)
\curveto(211.88248405,491.76268355)(212.58430627,491.97036563)(213.26464844,492.10644531)
\curveto(213.94498199,492.24966223)(214.60741622,492.32127674)(215.25195312,492.32128906)
\curveto(216.99217946,492.32127674)(218.29198285,491.87010532)(219.15136719,490.96777344)
\curveto(220.01073113,490.06541962)(220.4404182,488.69758245)(220.44042969,486.86425781)
}
}
{
\newrgbcolor{curcolor}{0 0 0}
\pscustom[linestyle=none,fillstyle=solid,fillcolor=curcolor]
{
}
}
{
\newrgbcolor{curcolor}{0 0 0}
\pscustom[linestyle=none,fillstyle=solid,fillcolor=curcolor]
{
\newpath
\moveto(233.43847656,481.8046875)
\lineto(233.43847656,475.42382812)
\lineto(231.45117188,475.42382812)
\lineto(231.45117188,492.03125)
\lineto(233.43847656,492.03125)
\lineto(233.43847656,490.20507812)
\curveto(233.85383675,490.92121304)(234.37662268,491.45116042)(235.00683594,491.79492188)
\curveto(235.64419954,492.14582119)(236.40331336,492.32127674)(237.28417969,492.32128906)
\curveto(238.7451079,492.32127674)(239.93032806,491.7411992)(240.83984375,490.58105469)
\curveto(241.75649811,489.42088902)(242.21483099,487.89549992)(242.21484375,486.00488281)
\curveto(242.21483099,484.1142537)(241.75649811,482.5888646)(240.83984375,481.42871094)
\curveto(239.93032806,480.26855442)(238.7451079,479.68847687)(237.28417969,479.68847656)
\curveto(236.40331336,479.68847687)(235.64419954,479.8603517)(235.00683594,480.20410156)
\curveto(234.37662268,480.55501247)(233.85383675,481.08854058)(233.43847656,481.8046875)
\moveto(240.16308594,486.00488281)
\curveto(240.16307523,487.4586514)(239.86229428,488.59732213)(239.26074219,489.42089844)
\curveto(238.66633193,490.25161735)(237.84634577,490.66698152)(236.80078125,490.66699219)
\curveto(235.75520203,490.66698152)(234.93163515,490.25161735)(234.33007812,489.42089844)
\curveto(233.7356728,488.59732213)(233.43847258,487.4586514)(233.43847656,486.00488281)
\curveto(233.43847258,484.55110222)(233.7356728,483.40885076)(234.33007812,482.578125)
\curveto(234.93163515,481.75455554)(235.75520203,481.34277209)(236.80078125,481.34277344)
\curveto(237.84634577,481.34277209)(238.66633193,481.75455554)(239.26074219,482.578125)
\curveto(239.86229428,483.40885076)(240.16307523,484.55110222)(240.16308594,486.00488281)
}
}
{
\newrgbcolor{curcolor}{0 0 0}
\pscustom[linestyle=none,fillstyle=solid,fillcolor=curcolor]
{
\newpath
\moveto(250.95898438,486.04785156)
\curveto(249.36197322,486.04784551)(248.25552902,485.86522851)(247.63964844,485.5)
\curveto(247.02375942,485.13476049)(246.71581701,484.51171424)(246.71582031,483.63085938)
\curveto(246.71581701,482.92903353)(246.94498345,482.37044034)(247.40332031,481.95507812)
\curveto(247.86881065,481.54687345)(248.49901836,481.34277209)(249.29394531,481.34277344)
\curveto(250.38964147,481.34277209)(251.26691923,481.72949046)(251.92578125,482.50292969)
\curveto(252.5917877,483.28352536)(252.92479518,484.31835506)(252.92480469,485.60742188)
\lineto(252.92480469,486.04785156)
\lineto(250.95898438,486.04785156)
\moveto(254.90136719,486.86425781)
\lineto(254.90136719,480)
\lineto(252.92480469,480)
\lineto(252.92480469,481.82617188)
\curveto(252.47362376,481.09570203)(251.91144984,480.55501247)(251.23828125,480.20410156)
\curveto(250.56509702,479.8603517)(249.74153013,479.68847687)(248.76757812,479.68847656)
\curveto(247.53580317,479.68847687)(246.55468436,480.03222653)(245.82421875,480.71972656)
\curveto(245.10090978,481.41438661)(244.73925649,482.34179453)(244.73925781,483.50195312)
\curveto(244.73925649,484.85546389)(245.19042792,485.87597069)(246.09277344,486.56347656)
\curveto(247.00227506,487.25096931)(248.35578933,487.59471897)(250.15332031,487.59472656)
\lineto(252.92480469,487.59472656)
\lineto(252.92480469,487.78808594)
\curveto(252.92479518,488.69758245)(252.62401423,489.39940466)(252.02246094,489.89355469)
\curveto(251.42805189,490.39484638)(250.5901621,490.64549717)(249.50878906,490.64550781)
\curveto(248.82128366,490.64549717)(248.15168797,490.56314048)(247.5,490.3984375)
\curveto(246.84830386,490.23371372)(246.22167688,489.98664366)(245.62011719,489.65722656)
\lineto(245.62011719,491.48339844)
\curveto(246.34342155,491.76268355)(247.04524377,491.97036563)(247.72558594,492.10644531)
\curveto(248.40591949,492.24966223)(249.06835372,492.32127674)(249.71289062,492.32128906)
\curveto(251.45311696,492.32127674)(252.75292035,491.87010532)(253.61230469,490.96777344)
\curveto(254.47166863,490.06541962)(254.9013557,488.69758245)(254.90136719,486.86425781)
}
}
{
\newrgbcolor{curcolor}{0 0 0}
\pscustom[linestyle=none,fillstyle=solid,fillcolor=curcolor]
{
\newpath
\moveto(265.95507812,490.18359375)
\curveto(265.73306409,490.31248969)(265.48957475,490.40558855)(265.22460938,490.46289062)
\curveto(264.96678882,490.52733322)(264.68033077,490.55955975)(264.36523438,490.55957031)
\curveto(263.24804054,490.55955975)(262.3886664,490.19432574)(261.78710938,489.46386719)
\curveto(261.19270405,488.74055116)(260.89550383,487.69856001)(260.89550781,486.33789062)
\lineto(260.89550781,480)
\lineto(258.90820312,480)
\lineto(258.90820312,492.03125)
\lineto(260.89550781,492.03125)
\lineto(260.89550781,490.16210938)
\curveto(261.310868,490.89256723)(261.85155756,491.4332568)(262.51757812,491.78417969)
\curveto(263.18358748,492.14224046)(263.99283146,492.32127674)(264.9453125,492.32128906)
\curveto(265.08137204,492.32127674)(265.23176251,492.31053456)(265.39648438,492.2890625)
\curveto(265.56118927,492.27472731)(265.74380627,492.24966223)(265.94433594,492.21386719)
\lineto(265.95507812,490.18359375)
}
}
{
\newrgbcolor{curcolor}{0 0 0}
\pscustom[linestyle=none,fillstyle=solid,fillcolor=curcolor]
{
\newpath
\moveto(270.00488281,495.44726562)
\lineto(270.00488281,492.03125)
\lineto(274.07617188,492.03125)
\lineto(274.07617188,490.49511719)
\lineto(270.00488281,490.49511719)
\lineto(270.00488281,483.96386719)
\curveto(270.00487878,482.98274441)(270.13736563,482.35253671)(270.40234375,482.07324219)
\curveto(270.67447447,481.79394352)(271.22232548,481.65429522)(272.04589844,481.65429688)
\lineto(274.07617188,481.65429688)
\lineto(274.07617188,480)
\lineto(272.04589844,480)
\curveto(270.52050327,480)(269.46776995,480.28287732)(268.88769531,480.84863281)
\curveto(268.30761486,481.42154806)(268.01757608,482.45995848)(268.01757812,483.96386719)
\lineto(268.01757812,490.49511719)
\lineto(266.56738281,490.49511719)
\lineto(266.56738281,492.03125)
\lineto(268.01757812,492.03125)
\lineto(268.01757812,495.44726562)
\lineto(270.00488281,495.44726562)
}
}
{
\newrgbcolor{curcolor}{0 0 0}
\pscustom[linestyle=none,fillstyle=solid,fillcolor=curcolor]
{
\newpath
\moveto(276.68652344,492.03125)
\lineto(278.66308594,492.03125)
\lineto(278.66308594,480)
\lineto(276.68652344,480)
\lineto(276.68652344,492.03125)
\moveto(276.68652344,496.71484375)
\lineto(278.66308594,496.71484375)
\lineto(278.66308594,494.21191406)
\lineto(276.68652344,494.21191406)
\lineto(276.68652344,496.71484375)
}
}
{
\newrgbcolor{curcolor}{0 0 0}
\pscustom[linestyle=none,fillstyle=solid,fillcolor=curcolor]
{
\newpath
\moveto(293.07910156,486.50976562)
\lineto(293.07910156,485.54296875)
\lineto(283.99121094,485.54296875)
\curveto(284.07714508,484.18228748)(284.48534779,483.14387706)(285.21582031,482.42773438)
\curveto(285.95344528,481.71874828)(286.9775328,481.36425645)(288.28808594,481.36425781)
\curveto(289.04719219,481.36425645)(289.78124093,481.45735531)(290.49023438,481.64355469)
\curveto(291.20636972,481.82975077)(291.91535338,482.10904737)(292.6171875,482.48144531)
\lineto(292.6171875,480.61230469)
\curveto(291.90819193,480.31152313)(291.18130464,480.08235669)(290.43652344,479.92480469)
\curveto(289.69172279,479.76725284)(288.9361897,479.68847687)(288.16992188,479.68847656)
\curveto(286.25064551,479.68847687)(284.72883713,480.24707007)(283.60449219,481.36425781)
\curveto(282.48730292,482.48144283)(281.92870972,483.99250903)(281.92871094,485.89746094)
\curveto(281.92870972,487.86685411)(282.45865711,489.42805047)(283.51855469,490.58105469)
\curveto(284.58560811,491.7411992)(286.02147907,492.32127674)(287.82617188,492.32128906)
\curveto(289.44465273,492.32127674)(290.72297176,491.79849081)(291.66113281,490.75292969)
\curveto(292.60643342,489.71450851)(293.0790892,488.30012191)(293.07910156,486.50976562)
\moveto(291.10253906,487.08984375)
\curveto(291.08820577,488.17121579)(290.7838441,489.03417065)(290.18945312,489.67871094)
\curveto(289.60220465,490.32323186)(288.82160648,490.64549717)(287.84765625,490.64550781)
\curveto(286.74478564,490.64549717)(285.86034642,490.33397404)(285.19433594,489.7109375)
\curveto(284.53547795,489.08788154)(284.15592104,488.21060377)(284.05566406,487.07910156)
\lineto(291.10253906,487.08984375)
}
}
{
\newrgbcolor{curcolor}{0 0 0}
\pscustom[linestyle=none,fillstyle=solid,fillcolor=curcolor]
{
\newpath
\moveto(129.93554688,216.03808594)
\lineto(143.50292969,216.03808594)
\lineto(143.50292969,214.21191406)
\lineto(137.80957031,214.21191406)
\lineto(137.80957031,200)
\lineto(135.62890625,200)
\lineto(135.62890625,214.21191406)
\lineto(129.93554688,214.21191406)
\lineto(129.93554688,216.03808594)
}
}
{
\newrgbcolor{curcolor}{0 0 0}
\pscustom[linestyle=none,fillstyle=solid,fillcolor=curcolor]
{
\newpath
\moveto(147.09082031,198.8828125)
\curveto(146.53222004,197.45052338)(145.98794975,196.515954)(145.45800781,196.07910156)
\curveto(144.92805498,195.64225696)(144.21907131,195.4238327)(143.33105469,195.42382812)
\lineto(141.75195312,195.42382812)
\lineto(141.75195312,197.078125)
\lineto(142.91210938,197.078125)
\curveto(143.45637676,197.07812792)(143.87890238,197.20703404)(144.1796875,197.46484375)
\curveto(144.48046428,197.72265853)(144.81347176,198.33138188)(145.17871094,199.29101562)
\lineto(145.53320312,200.19335938)
\lineto(140.66699219,212.03125)
\lineto(142.76171875,212.03125)
\lineto(146.52148438,202.62109375)
\lineto(150.28125,212.03125)
\lineto(152.37597656,212.03125)
\lineto(147.09082031,198.8828125)
}
}
{
\newrgbcolor{curcolor}{0 0 0}
\pscustom[linestyle=none,fillstyle=solid,fillcolor=curcolor]
{
\newpath
\moveto(157.01660156,201.8046875)
\lineto(157.01660156,195.42382812)
\lineto(155.02929688,195.42382812)
\lineto(155.02929688,212.03125)
\lineto(157.01660156,212.03125)
\lineto(157.01660156,210.20507812)
\curveto(157.43196175,210.92121304)(157.95474768,211.45116042)(158.58496094,211.79492188)
\curveto(159.22232454,212.14582119)(159.98143836,212.32127674)(160.86230469,212.32128906)
\curveto(162.3232329,212.32127674)(163.50845306,211.7411992)(164.41796875,210.58105469)
\curveto(165.33462311,209.42088902)(165.79295599,207.89549992)(165.79296875,206.00488281)
\curveto(165.79295599,204.1142537)(165.33462311,202.5888646)(164.41796875,201.42871094)
\curveto(163.50845306,200.26855442)(162.3232329,199.68847687)(160.86230469,199.68847656)
\curveto(159.98143836,199.68847687)(159.22232454,199.8603517)(158.58496094,200.20410156)
\curveto(157.95474768,200.55501247)(157.43196175,201.08854058)(157.01660156,201.8046875)
\moveto(163.74121094,206.00488281)
\curveto(163.74120023,207.4586514)(163.44041928,208.59732213)(162.83886719,209.42089844)
\curveto(162.24445693,210.25161735)(161.42447077,210.66698152)(160.37890625,210.66699219)
\curveto(159.33332703,210.66698152)(158.50976015,210.25161735)(157.90820312,209.42089844)
\curveto(157.3137978,208.59732213)(157.01659758,207.4586514)(157.01660156,206.00488281)
\curveto(157.01659758,204.55110222)(157.3137978,203.40885076)(157.90820312,202.578125)
\curveto(158.50976015,201.75455554)(159.33332703,201.34277209)(160.37890625,201.34277344)
\curveto(161.42447077,201.34277209)(162.24445693,201.75455554)(162.83886719,202.578125)
\curveto(163.44041928,203.40885076)(163.74120023,204.55110222)(163.74121094,206.00488281)
}
}
{
\newrgbcolor{curcolor}{0 0 0}
\pscustom[linestyle=none,fillstyle=solid,fillcolor=curcolor]
{
\newpath
\moveto(179.36035156,206.50976562)
\lineto(179.36035156,205.54296875)
\lineto(170.27246094,205.54296875)
\curveto(170.35839508,204.18228748)(170.76659779,203.14387706)(171.49707031,202.42773438)
\curveto(172.23469528,201.71874828)(173.2587828,201.36425645)(174.56933594,201.36425781)
\curveto(175.32844219,201.36425645)(176.06249093,201.45735531)(176.77148438,201.64355469)
\curveto(177.48761972,201.82975077)(178.19660338,202.10904737)(178.8984375,202.48144531)
\lineto(178.8984375,200.61230469)
\curveto(178.18944193,200.31152313)(177.46255464,200.08235669)(176.71777344,199.92480469)
\curveto(175.97297279,199.76725284)(175.2174397,199.68847687)(174.45117188,199.68847656)
\curveto(172.53189551,199.68847687)(171.01008713,200.24707007)(169.88574219,201.36425781)
\curveto(168.76855292,202.48144283)(168.20995972,203.99250903)(168.20996094,205.89746094)
\curveto(168.20995972,207.86685411)(168.73990711,209.42805047)(169.79980469,210.58105469)
\curveto(170.86685811,211.7411992)(172.30272907,212.32127674)(174.10742188,212.32128906)
\curveto(175.72590273,212.32127674)(177.00422176,211.79849081)(177.94238281,210.75292969)
\curveto(178.88768342,209.71450851)(179.3603392,208.30012191)(179.36035156,206.50976562)
\moveto(177.38378906,207.08984375)
\curveto(177.36945577,208.17121579)(177.0650941,209.03417065)(176.47070312,209.67871094)
\curveto(175.88345465,210.32323186)(175.10285648,210.64549717)(174.12890625,210.64550781)
\curveto(173.02603564,210.64549717)(172.14159642,210.33397404)(171.47558594,209.7109375)
\curveto(170.81672795,209.08788154)(170.43717104,208.21060377)(170.33691406,207.07910156)
\lineto(177.38378906,207.08984375)
}
}
{
\newrgbcolor{curcolor}{0 0 0}
\pscustom[linestyle=none,fillstyle=solid,fillcolor=curcolor]
{
}
}
{
\newrgbcolor{curcolor}{0 0 0}
\pscustom[linestyle=none,fillstyle=solid,fillcolor=curcolor]
{
\newpath
\moveto(197.52539062,210.20507812)
\lineto(197.52539062,216.71484375)
\lineto(199.50195312,216.71484375)
\lineto(199.50195312,200)
\lineto(197.52539062,200)
\lineto(197.52539062,201.8046875)
\curveto(197.11001647,201.08854058)(196.58364981,200.55501247)(195.94628906,200.20410156)
\curveto(195.31607295,199.8603517)(194.55695912,199.68847687)(193.66894531,199.68847656)
\curveto(192.21516459,199.68847687)(191.02994442,200.26855442)(190.11328125,201.42871094)
\curveto(189.20377437,202.5888646)(188.74902222,204.1142537)(188.74902344,206.00488281)
\curveto(188.74902222,207.89549992)(189.20377437,209.42088902)(190.11328125,210.58105469)
\curveto(191.02994442,211.7411992)(192.21516459,212.32127674)(193.66894531,212.32128906)
\curveto(194.55695912,212.32127674)(195.31607295,212.14582119)(195.94628906,211.79492188)
\curveto(196.58364981,211.45116042)(197.11001647,210.92121304)(197.52539062,210.20507812)
\moveto(190.79003906,206.00488281)
\curveto(190.79003581,204.55110222)(191.08723603,203.40885076)(191.68164062,202.578125)
\curveto(192.28319838,201.75455554)(193.10676526,201.34277209)(194.15234375,201.34277344)
\curveto(195.197909,201.34277209)(196.02147589,201.75455554)(196.62304688,202.578125)
\curveto(197.22459969,203.40885076)(197.52538063,204.55110222)(197.52539062,206.00488281)
\curveto(197.52538063,207.4586514)(197.22459969,208.59732213)(196.62304688,209.42089844)
\curveto(196.02147589,210.25161735)(195.197909,210.66698152)(194.15234375,210.66699219)
\curveto(193.10676526,210.66698152)(192.28319838,210.25161735)(191.68164062,209.42089844)
\curveto(191.08723603,208.59732213)(190.79003581,207.4586514)(190.79003906,206.00488281)
}
}
{
\newrgbcolor{curcolor}{0 0 0}
\pscustom[linestyle=none,fillstyle=solid,fillcolor=curcolor]
{
\newpath
\moveto(213.86425781,206.50976562)
\lineto(213.86425781,205.54296875)
\lineto(204.77636719,205.54296875)
\curveto(204.86230133,204.18228748)(205.27050404,203.14387706)(206.00097656,202.42773438)
\curveto(206.73860153,201.71874828)(207.76268905,201.36425645)(209.07324219,201.36425781)
\curveto(209.83234844,201.36425645)(210.56639718,201.45735531)(211.27539062,201.64355469)
\curveto(211.99152597,201.82975077)(212.70050963,202.10904737)(213.40234375,202.48144531)
\lineto(213.40234375,200.61230469)
\curveto(212.69334818,200.31152313)(211.96646089,200.08235669)(211.22167969,199.92480469)
\curveto(210.47687904,199.76725284)(209.72134595,199.68847687)(208.95507812,199.68847656)
\curveto(207.03580176,199.68847687)(205.51399338,200.24707007)(204.38964844,201.36425781)
\curveto(203.27245917,202.48144283)(202.71386597,203.99250903)(202.71386719,205.89746094)
\curveto(202.71386597,207.86685411)(203.24381336,209.42805047)(204.30371094,210.58105469)
\curveto(205.37076436,211.7411992)(206.80663532,212.32127674)(208.61132812,212.32128906)
\curveto(210.22980898,212.32127674)(211.50812801,211.79849081)(212.44628906,210.75292969)
\curveto(213.39158967,209.71450851)(213.86424545,208.30012191)(213.86425781,206.50976562)
\moveto(211.88769531,207.08984375)
\curveto(211.87336202,208.17121579)(211.56900035,209.03417065)(210.97460938,209.67871094)
\curveto(210.3873609,210.32323186)(209.60676273,210.64549717)(208.6328125,210.64550781)
\curveto(207.52994189,210.64549717)(206.64550267,210.33397404)(205.97949219,209.7109375)
\curveto(205.3206342,209.08788154)(204.94107729,208.21060377)(204.84082031,207.07910156)
\lineto(211.88769531,207.08984375)
}
}
{
\newrgbcolor{curcolor}{0 0 0}
\pscustom[linestyle=none,fillstyle=solid,fillcolor=curcolor]
{
}
}
{
\newrgbcolor{curcolor}{0 0 0}
\pscustom[linestyle=none,fillstyle=solid,fillcolor=curcolor]
{
\newpath
\moveto(226.02441406,201.8046875)
\lineto(226.02441406,195.42382812)
\lineto(224.03710938,195.42382812)
\lineto(224.03710938,212.03125)
\lineto(226.02441406,212.03125)
\lineto(226.02441406,210.20507812)
\curveto(226.43977425,210.92121304)(226.96256018,211.45116042)(227.59277344,211.79492188)
\curveto(228.23013704,212.14582119)(228.98925086,212.32127674)(229.87011719,212.32128906)
\curveto(231.3310454,212.32127674)(232.51626556,211.7411992)(233.42578125,210.58105469)
\curveto(234.34243561,209.42088902)(234.80076849,207.89549992)(234.80078125,206.00488281)
\curveto(234.80076849,204.1142537)(234.34243561,202.5888646)(233.42578125,201.42871094)
\curveto(232.51626556,200.26855442)(231.3310454,199.68847687)(229.87011719,199.68847656)
\curveto(228.98925086,199.68847687)(228.23013704,199.8603517)(227.59277344,200.20410156)
\curveto(226.96256018,200.55501247)(226.43977425,201.08854058)(226.02441406,201.8046875)
\moveto(232.74902344,206.00488281)
\curveto(232.74901273,207.4586514)(232.44823178,208.59732213)(231.84667969,209.42089844)
\curveto(231.25226943,210.25161735)(230.43228327,210.66698152)(229.38671875,210.66699219)
\curveto(228.34113953,210.66698152)(227.51757265,210.25161735)(226.91601562,209.42089844)
\curveto(226.3216103,208.59732213)(226.02441008,207.4586514)(226.02441406,206.00488281)
\curveto(226.02441008,204.55110222)(226.3216103,203.40885076)(226.91601562,202.578125)
\curveto(227.51757265,201.75455554)(228.34113953,201.34277209)(229.38671875,201.34277344)
\curveto(230.43228327,201.34277209)(231.25226943,201.75455554)(231.84667969,202.578125)
\curveto(232.44823178,203.40885076)(232.74901273,204.55110222)(232.74902344,206.00488281)
}
}
{
\newrgbcolor{curcolor}{0 0 0}
\pscustom[linestyle=none,fillstyle=solid,fillcolor=curcolor]
{
\newpath
\moveto(243.54492188,206.04785156)
\curveto(241.94791072,206.04784551)(240.84146652,205.86522851)(240.22558594,205.5)
\curveto(239.60969692,205.13476049)(239.30175451,204.51171424)(239.30175781,203.63085938)
\curveto(239.30175451,202.92903353)(239.53092095,202.37044034)(239.98925781,201.95507812)
\curveto(240.45474815,201.54687345)(241.08495586,201.34277209)(241.87988281,201.34277344)
\curveto(242.97557897,201.34277209)(243.85285673,201.72949046)(244.51171875,202.50292969)
\curveto(245.1777252,203.28352536)(245.51073268,204.31835506)(245.51074219,205.60742188)
\lineto(245.51074219,206.04785156)
\lineto(243.54492188,206.04785156)
\moveto(247.48730469,206.86425781)
\lineto(247.48730469,200)
\lineto(245.51074219,200)
\lineto(245.51074219,201.82617188)
\curveto(245.05956126,201.09570203)(244.49738734,200.55501247)(243.82421875,200.20410156)
\curveto(243.15103452,199.8603517)(242.32746763,199.68847687)(241.35351562,199.68847656)
\curveto(240.12174067,199.68847687)(239.14062186,200.03222653)(238.41015625,200.71972656)
\curveto(237.68684728,201.41438661)(237.32519399,202.34179453)(237.32519531,203.50195312)
\curveto(237.32519399,204.85546389)(237.77636542,205.87597069)(238.67871094,206.56347656)
\curveto(239.58821256,207.25096931)(240.94172683,207.59471897)(242.73925781,207.59472656)
\lineto(245.51074219,207.59472656)
\lineto(245.51074219,207.78808594)
\curveto(245.51073268,208.69758245)(245.20995173,209.39940466)(244.60839844,209.89355469)
\curveto(244.01398939,210.39484638)(243.1760996,210.64549717)(242.09472656,210.64550781)
\curveto(241.40722116,210.64549717)(240.73762547,210.56314048)(240.0859375,210.3984375)
\curveto(239.43424136,210.23371372)(238.80761438,209.98664366)(238.20605469,209.65722656)
\lineto(238.20605469,211.48339844)
\curveto(238.92935905,211.76268355)(239.63118127,211.97036563)(240.31152344,212.10644531)
\curveto(240.99185699,212.24966223)(241.65429122,212.32127674)(242.29882812,212.32128906)
\curveto(244.03905446,212.32127674)(245.33885785,211.87010532)(246.19824219,210.96777344)
\curveto(247.05760613,210.06541962)(247.4872932,208.69758245)(247.48730469,206.86425781)
}
}
{
\newrgbcolor{curcolor}{0 0 0}
\pscustom[linestyle=none,fillstyle=solid,fillcolor=curcolor]
{
\newpath
\moveto(258.54101562,210.18359375)
\curveto(258.31900159,210.31248969)(258.07551225,210.40558855)(257.81054688,210.46289062)
\curveto(257.55272632,210.52733322)(257.26626827,210.55955975)(256.95117188,210.55957031)
\curveto(255.83397804,210.55955975)(254.9746039,210.19432574)(254.37304688,209.46386719)
\curveto(253.77864155,208.74055116)(253.48144133,207.69856001)(253.48144531,206.33789062)
\lineto(253.48144531,200)
\lineto(251.49414062,200)
\lineto(251.49414062,212.03125)
\lineto(253.48144531,212.03125)
\lineto(253.48144531,210.16210938)
\curveto(253.8968055,210.89256723)(254.43749506,211.4332568)(255.10351562,211.78417969)
\curveto(255.76952498,212.14224046)(256.57876896,212.32127674)(257.53125,212.32128906)
\curveto(257.66730954,212.32127674)(257.81770001,212.31053456)(257.98242188,212.2890625)
\curveto(258.14712677,212.27472731)(258.32974377,212.24966223)(258.53027344,212.21386719)
\lineto(258.54101562,210.18359375)
}
}
{
\newrgbcolor{curcolor}{0 0 0}
\pscustom[linestyle=none,fillstyle=solid,fillcolor=curcolor]
{
\newpath
\moveto(262.59082031,215.44726562)
\lineto(262.59082031,212.03125)
\lineto(266.66210938,212.03125)
\lineto(266.66210938,210.49511719)
\lineto(262.59082031,210.49511719)
\lineto(262.59082031,203.96386719)
\curveto(262.59081628,202.98274441)(262.72330313,202.35253671)(262.98828125,202.07324219)
\curveto(263.26041197,201.79394352)(263.80826298,201.65429522)(264.63183594,201.65429688)
\lineto(266.66210938,201.65429688)
\lineto(266.66210938,200)
\lineto(264.63183594,200)
\curveto(263.10644077,200)(262.05370745,200.28287732)(261.47363281,200.84863281)
\curveto(260.89355236,201.42154806)(260.60351358,202.45995848)(260.60351562,203.96386719)
\lineto(260.60351562,210.49511719)
\lineto(259.15332031,210.49511719)
\lineto(259.15332031,212.03125)
\lineto(260.60351562,212.03125)
\lineto(260.60351562,215.44726562)
\lineto(262.59082031,215.44726562)
}
}
{
\newrgbcolor{curcolor}{0 0 0}
\pscustom[linestyle=none,fillstyle=solid,fillcolor=curcolor]
{
\newpath
\moveto(269.27246094,212.03125)
\lineto(271.24902344,212.03125)
\lineto(271.24902344,200)
\lineto(269.27246094,200)
\lineto(269.27246094,212.03125)
\moveto(269.27246094,216.71484375)
\lineto(271.24902344,216.71484375)
\lineto(271.24902344,214.21191406)
\lineto(269.27246094,214.21191406)
\lineto(269.27246094,216.71484375)
}
}
{
\newrgbcolor{curcolor}{0 0 0}
\pscustom[linestyle=none,fillstyle=solid,fillcolor=curcolor]
{
\newpath
\moveto(285.66503906,206.50976562)
\lineto(285.66503906,205.54296875)
\lineto(276.57714844,205.54296875)
\curveto(276.66308258,204.18228748)(277.07128529,203.14387706)(277.80175781,202.42773438)
\curveto(278.53938278,201.71874828)(279.5634703,201.36425645)(280.87402344,201.36425781)
\curveto(281.63312969,201.36425645)(282.36717843,201.45735531)(283.07617188,201.64355469)
\curveto(283.79230722,201.82975077)(284.50129088,202.10904737)(285.203125,202.48144531)
\lineto(285.203125,200.61230469)
\curveto(284.49412943,200.31152313)(283.76724214,200.08235669)(283.02246094,199.92480469)
\curveto(282.27766029,199.76725284)(281.5221272,199.68847687)(280.75585938,199.68847656)
\curveto(278.83658301,199.68847687)(277.31477463,200.24707007)(276.19042969,201.36425781)
\curveto(275.07324042,202.48144283)(274.51464722,203.99250903)(274.51464844,205.89746094)
\curveto(274.51464722,207.86685411)(275.04459461,209.42805047)(276.10449219,210.58105469)
\curveto(277.17154561,211.7411992)(278.60741657,212.32127674)(280.41210938,212.32128906)
\curveto(282.03059023,212.32127674)(283.30890926,211.79849081)(284.24707031,210.75292969)
\curveto(285.19237092,209.71450851)(285.6650267,208.30012191)(285.66503906,206.50976562)
\moveto(283.68847656,207.08984375)
\curveto(283.67414327,208.17121579)(283.3697816,209.03417065)(282.77539062,209.67871094)
\curveto(282.18814215,210.32323186)(281.40754398,210.64549717)(280.43359375,210.64550781)
\curveto(279.33072314,210.64549717)(278.44628392,210.33397404)(277.78027344,209.7109375)
\curveto(277.12141545,209.08788154)(276.74185854,208.21060377)(276.64160156,207.07910156)
\lineto(283.68847656,207.08984375)
}
}
{
\newrgbcolor{curcolor}{0 0 0}
\pscustom[linestyle=none,fillstyle=solid,fillcolor=curcolor]
{
\newpath
\moveto(112.15917969,166.03808594)
\lineto(115.08105469,166.03808594)
\lineto(122.19238281,152.62109375)
\lineto(122.19238281,166.03808594)
\lineto(124.29785156,166.03808594)
\lineto(124.29785156,150)
\lineto(121.37597656,150)
\lineto(114.26464844,163.41699219)
\lineto(114.26464844,150)
\lineto(112.15917969,150)
\lineto(112.15917969,166.03808594)
}
}
{
\newrgbcolor{curcolor}{0 0 0}
\pscustom[linestyle=none,fillstyle=solid,fillcolor=curcolor]
{
\newpath
\moveto(133.19238281,160.64550781)
\curveto(132.1324813,160.64549717)(131.29459152,160.230133)(130.67871094,159.39941406)
\curveto(130.06282192,158.57583778)(129.75487951,157.44432849)(129.75488281,156.00488281)
\curveto(129.75487951,154.56542512)(130.05924119,153.43033511)(130.66796875,152.59960938)
\curveto(131.28384934,151.77603989)(132.12531985,151.36425645)(133.19238281,151.36425781)
\curveto(134.2451094,151.36425645)(135.07941846,151.77962062)(135.6953125,152.61035156)
\curveto(136.31118806,153.44107729)(136.61913046,154.57258657)(136.61914062,156.00488281)
\curveto(136.61913046,157.43000559)(136.31118806,158.55793415)(135.6953125,159.38867188)
\curveto(135.07941846,160.22655227)(134.2451094,160.64549717)(133.19238281,160.64550781)
\moveto(133.19238281,162.32128906)
\curveto(134.91112436,162.32127674)(136.2610579,161.76268355)(137.2421875,160.64550781)
\curveto(138.22329553,159.52831078)(138.71385493,157.98143733)(138.71386719,156.00488281)
\curveto(138.71385493,154.03547774)(138.22329553,152.48860428)(137.2421875,151.36425781)
\curveto(136.2610579,150.24707007)(134.91112436,149.68847687)(133.19238281,149.68847656)
\curveto(131.46646634,149.68847687)(130.11295207,150.24707007)(129.13183594,151.36425781)
\curveto(128.1578759,152.48860428)(127.67089722,154.03547774)(127.67089844,156.00488281)
\curveto(127.67089722,157.98143733)(128.1578759,159.52831078)(129.13183594,160.64550781)
\curveto(130.11295207,161.76268355)(131.46646634,162.32127674)(133.19238281,162.32128906)
}
}
{
\newrgbcolor{curcolor}{0 0 0}
\pscustom[linestyle=none,fillstyle=solid,fillcolor=curcolor]
{
\newpath
\moveto(151.34667969,159.72167969)
\curveto(151.84080838,160.60968991)(152.4316281,161.26496269)(153.11914062,161.6875)
\curveto(153.80662672,162.11001393)(154.61587071,162.32127674)(155.546875,162.32128906)
\curveto(156.80011331,162.32127674)(157.76690922,161.88084749)(158.44726562,161)
\curveto(159.12758495,160.12629196)(159.46775388,158.88019945)(159.46777344,157.26171875)
\lineto(159.46777344,150)
\lineto(157.48046875,150)
\lineto(157.48046875,157.19726562)
\curveto(157.48045118,158.35025207)(157.27634982,159.20604548)(156.86816406,159.76464844)
\curveto(156.45994438,160.32323186)(155.83689813,160.60252846)(154.99902344,160.60253906)
\curveto(153.97492083,160.60252846)(153.16567684,160.26235953)(152.57128906,159.58203125)
\curveto(151.97687595,158.90168381)(151.67967573,157.97427588)(151.6796875,156.79980469)
\lineto(151.6796875,150)
\lineto(149.69238281,150)
\lineto(149.69238281,157.19726562)
\curveto(149.69237303,158.35741352)(149.48827167,159.21320693)(149.08007812,159.76464844)
\curveto(148.67186623,160.32323186)(148.04165853,160.60252846)(147.18945312,160.60253906)
\curveto(146.17968123,160.60252846)(145.3775987,160.2587788)(144.78320312,159.57128906)
\curveto(144.1887978,158.89094163)(143.89159758,157.96711443)(143.89160156,156.79980469)
\lineto(143.89160156,150)
\lineto(141.90429688,150)
\lineto(141.90429688,162.03125)
\lineto(143.89160156,162.03125)
\lineto(143.89160156,160.16210938)
\curveto(144.342769,160.89972868)(144.88345856,161.44399897)(145.51367188,161.79492188)
\curveto(146.14387397,162.14582119)(146.89224562,162.32127674)(147.75878906,162.32128906)
\curveto(148.63247825,162.32127674)(149.37368845,162.09927175)(149.98242188,161.65527344)
\curveto(150.5982966,161.21125181)(151.05304875,160.5667212)(151.34667969,159.72167969)
}
}
{
\newrgbcolor{curcolor}{0 0 0}
\pscustom[linestyle=none,fillstyle=solid,fillcolor=curcolor]
{
\newpath
\moveto(172.05761719,156.00488281)
\curveto(172.05760648,157.4586514)(171.75682553,158.59732213)(171.15527344,159.42089844)
\curveto(170.56086318,160.25161735)(169.74087702,160.66698152)(168.6953125,160.66699219)
\curveto(167.64973328,160.66698152)(166.8261664,160.25161735)(166.22460938,159.42089844)
\curveto(165.63020405,158.59732213)(165.33300383,157.4586514)(165.33300781,156.00488281)
\curveto(165.33300383,154.55110222)(165.63020405,153.40885076)(166.22460938,152.578125)
\curveto(166.8261664,151.75455554)(167.64973328,151.34277209)(168.6953125,151.34277344)
\curveto(169.74087702,151.34277209)(170.56086318,151.75455554)(171.15527344,152.578125)
\curveto(171.75682553,153.40885076)(172.05760648,154.55110222)(172.05761719,156.00488281)
\moveto(165.33300781,160.20507812)
\curveto(165.748368,160.92121304)(166.27115393,161.45116042)(166.90136719,161.79492188)
\curveto(167.53873079,162.14582119)(168.29784461,162.32127674)(169.17871094,162.32128906)
\curveto(170.63963915,162.32127674)(171.82485931,161.7411992)(172.734375,160.58105469)
\curveto(173.65102936,159.42088902)(174.10936224,157.89549992)(174.109375,156.00488281)
\curveto(174.10936224,154.1142537)(173.65102936,152.5888646)(172.734375,151.42871094)
\curveto(171.82485931,150.26855442)(170.63963915,149.68847687)(169.17871094,149.68847656)
\curveto(168.29784461,149.68847687)(167.53873079,149.8603517)(166.90136719,150.20410156)
\curveto(166.27115393,150.55501247)(165.748368,151.08854058)(165.33300781,151.8046875)
\lineto(165.33300781,150)
\lineto(163.34570312,150)
\lineto(163.34570312,166.71484375)
\lineto(165.33300781,166.71484375)
\lineto(165.33300781,160.20507812)
}
}
{
\newrgbcolor{curcolor}{0 0 0}
\pscustom[linestyle=none,fillstyle=solid,fillcolor=curcolor]
{
\newpath
\moveto(184.35742188,160.18359375)
\curveto(184.13540784,160.31248969)(183.8919185,160.40558855)(183.62695312,160.46289062)
\curveto(183.36913257,160.52733322)(183.08267452,160.55955975)(182.76757812,160.55957031)
\curveto(181.65038429,160.55955975)(180.79101015,160.19432574)(180.18945312,159.46386719)
\curveto(179.5950478,158.74055116)(179.29784758,157.69856001)(179.29785156,156.33789062)
\lineto(179.29785156,150)
\lineto(177.31054688,150)
\lineto(177.31054688,162.03125)
\lineto(179.29785156,162.03125)
\lineto(179.29785156,160.16210938)
\curveto(179.71321175,160.89256723)(180.25390131,161.4332568)(180.91992188,161.78417969)
\curveto(181.58593123,162.14224046)(182.39517521,162.32127674)(183.34765625,162.32128906)
\curveto(183.48371579,162.32127674)(183.63410626,162.31053456)(183.79882812,162.2890625)
\curveto(183.96353302,162.27472731)(184.14615002,162.24966223)(184.34667969,162.21386719)
\lineto(184.35742188,160.18359375)
}
}
{
\newrgbcolor{curcolor}{0 0 0}
\pscustom[linestyle=none,fillstyle=solid,fillcolor=curcolor]
{
\newpath
\moveto(196.27050781,156.50976562)
\lineto(196.27050781,155.54296875)
\lineto(187.18261719,155.54296875)
\curveto(187.26855133,154.18228748)(187.67675404,153.14387706)(188.40722656,152.42773438)
\curveto(189.14485153,151.71874828)(190.16893905,151.36425645)(191.47949219,151.36425781)
\curveto(192.23859844,151.36425645)(192.97264718,151.45735531)(193.68164062,151.64355469)
\curveto(194.39777597,151.82975077)(195.10675963,152.10904737)(195.80859375,152.48144531)
\lineto(195.80859375,150.61230469)
\curveto(195.09959818,150.31152313)(194.37271089,150.08235669)(193.62792969,149.92480469)
\curveto(192.88312904,149.76725284)(192.12759595,149.68847687)(191.36132812,149.68847656)
\curveto(189.44205176,149.68847687)(187.92024338,150.24707007)(186.79589844,151.36425781)
\curveto(185.67870917,152.48144283)(185.12011597,153.99250903)(185.12011719,155.89746094)
\curveto(185.12011597,157.86685411)(185.65006336,159.42805047)(186.70996094,160.58105469)
\curveto(187.77701436,161.7411992)(189.21288532,162.32127674)(191.01757812,162.32128906)
\curveto(192.63605898,162.32127674)(193.91437801,161.79849081)(194.85253906,160.75292969)
\curveto(195.79783967,159.71450851)(196.27049545,158.30012191)(196.27050781,156.50976562)
\moveto(194.29394531,157.08984375)
\curveto(194.27961202,158.17121579)(193.97525035,159.03417065)(193.38085938,159.67871094)
\curveto(192.7936109,160.32323186)(192.01301273,160.64549717)(191.0390625,160.64550781)
\curveto(189.93619189,160.64549717)(189.05175267,160.33397404)(188.38574219,159.7109375)
\curveto(187.7268842,159.08788154)(187.34732729,158.21060377)(187.24707031,157.07910156)
\lineto(194.29394531,157.08984375)
}
}
{
\newrgbcolor{curcolor}{0 0 0}
\pscustom[linestyle=none,fillstyle=solid,fillcolor=curcolor]
{
}
}
{
\newrgbcolor{curcolor}{0 0 0}
\pscustom[linestyle=none,fillstyle=solid,fillcolor=curcolor]
{
\newpath
\moveto(214.43554688,160.20507812)
\lineto(214.43554688,166.71484375)
\lineto(216.41210938,166.71484375)
\lineto(216.41210938,150)
\lineto(214.43554688,150)
\lineto(214.43554688,151.8046875)
\curveto(214.02017272,151.08854058)(213.49380606,150.55501247)(212.85644531,150.20410156)
\curveto(212.2262292,149.8603517)(211.46711537,149.68847687)(210.57910156,149.68847656)
\curveto(209.12532084,149.68847687)(207.94010067,150.26855442)(207.0234375,151.42871094)
\curveto(206.11393062,152.5888646)(205.65917847,154.1142537)(205.65917969,156.00488281)
\curveto(205.65917847,157.89549992)(206.11393062,159.42088902)(207.0234375,160.58105469)
\curveto(207.94010067,161.7411992)(209.12532084,162.32127674)(210.57910156,162.32128906)
\curveto(211.46711537,162.32127674)(212.2262292,162.14582119)(212.85644531,161.79492188)
\curveto(213.49380606,161.45116042)(214.02017272,160.92121304)(214.43554688,160.20507812)
\moveto(207.70019531,156.00488281)
\curveto(207.70019206,154.55110222)(207.99739228,153.40885076)(208.59179688,152.578125)
\curveto(209.19335463,151.75455554)(210.01692151,151.34277209)(211.0625,151.34277344)
\curveto(212.10806525,151.34277209)(212.93163214,151.75455554)(213.53320312,152.578125)
\curveto(214.13475594,153.40885076)(214.43553688,154.55110222)(214.43554688,156.00488281)
\curveto(214.43553688,157.4586514)(214.13475594,158.59732213)(213.53320312,159.42089844)
\curveto(212.93163214,160.25161735)(212.10806525,160.66698152)(211.0625,160.66699219)
\curveto(210.01692151,160.66698152)(209.19335463,160.25161735)(208.59179688,159.42089844)
\curveto(207.99739228,158.59732213)(207.70019206,157.4586514)(207.70019531,156.00488281)
}
}
{
\newrgbcolor{curcolor}{0 0 0}
\pscustom[linestyle=none,fillstyle=solid,fillcolor=curcolor]
{
\newpath
\moveto(222.35253906,166.03808594)
\lineto(222.35253906,160.07617188)
\lineto(220.52636719,160.07617188)
\lineto(220.52636719,166.03808594)
\lineto(222.35253906,166.03808594)
}
}
{
\newrgbcolor{curcolor}{0 0 0}
\pscustom[linestyle=none,fillstyle=solid,fillcolor=curcolor]
{
\newpath
\moveto(236.83300781,156.50976562)
\lineto(236.83300781,155.54296875)
\lineto(227.74511719,155.54296875)
\curveto(227.83105133,154.18228748)(228.23925404,153.14387706)(228.96972656,152.42773438)
\curveto(229.70735153,151.71874828)(230.73143905,151.36425645)(232.04199219,151.36425781)
\curveto(232.80109844,151.36425645)(233.53514718,151.45735531)(234.24414062,151.64355469)
\curveto(234.96027597,151.82975077)(235.66925963,152.10904737)(236.37109375,152.48144531)
\lineto(236.37109375,150.61230469)
\curveto(235.66209818,150.31152313)(234.93521089,150.08235669)(234.19042969,149.92480469)
\curveto(233.44562904,149.76725284)(232.69009595,149.68847687)(231.92382812,149.68847656)
\curveto(230.00455176,149.68847687)(228.48274338,150.24707007)(227.35839844,151.36425781)
\curveto(226.24120917,152.48144283)(225.68261597,153.99250903)(225.68261719,155.89746094)
\curveto(225.68261597,157.86685411)(226.21256336,159.42805047)(227.27246094,160.58105469)
\curveto(228.33951436,161.7411992)(229.77538532,162.32127674)(231.58007812,162.32128906)
\curveto(233.19855898,162.32127674)(234.47687801,161.79849081)(235.41503906,160.75292969)
\curveto(236.36033967,159.71450851)(236.83299545,158.30012191)(236.83300781,156.50976562)
\moveto(234.85644531,157.08984375)
\curveto(234.84211202,158.17121579)(234.53775035,159.03417065)(233.94335938,159.67871094)
\curveto(233.3561109,160.32323186)(232.57551273,160.64549717)(231.6015625,160.64550781)
\curveto(230.49869189,160.64549717)(229.61425267,160.33397404)(228.94824219,159.7109375)
\curveto(228.2893842,159.08788154)(227.90982729,158.21060377)(227.80957031,157.07910156)
\lineto(234.85644531,157.08984375)
}
}
{
\newrgbcolor{curcolor}{0 0 0}
\pscustom[linestyle=none,fillstyle=solid,fillcolor=curcolor]
{
\newpath
\moveto(250.078125,157.26171875)
\lineto(250.078125,150)
\lineto(248.1015625,150)
\lineto(248.1015625,157.19726562)
\curveto(248.1015524,158.33592916)(247.87954742,159.18814185)(247.43554688,159.75390625)
\curveto(246.99152747,160.31965114)(246.32551251,160.60252846)(245.4375,160.60253906)
\curveto(244.37043634,160.60252846)(243.52896583,160.26235953)(242.91308594,159.58203125)
\curveto(242.29719623,158.90168381)(241.98925383,157.97427588)(241.98925781,156.79980469)
\lineto(241.98925781,150)
\lineto(240.00195312,150)
\lineto(240.00195312,162.03125)
\lineto(241.98925781,162.03125)
\lineto(241.98925781,160.16210938)
\curveto(242.4619096,160.88540578)(243.01692207,161.42609534)(243.65429688,161.78417969)
\curveto(244.29882183,162.14224046)(245.04003203,162.32127674)(245.87792969,162.32128906)
\curveto(247.26008189,162.32127674)(248.30565376,161.89158967)(249.01464844,161.03222656)
\curveto(249.72362109,160.18000284)(250.07811293,158.92316816)(250.078125,157.26171875)
}
}
{
\newrgbcolor{curcolor}{0 0 0}
\pscustom[linestyle=none,fillstyle=solid,fillcolor=curcolor]
{
\newpath
\moveto(264.04296875,157.26171875)
\lineto(264.04296875,150)
\lineto(262.06640625,150)
\lineto(262.06640625,157.19726562)
\curveto(262.06639615,158.33592916)(261.84439117,159.18814185)(261.40039062,159.75390625)
\curveto(260.95637122,160.31965114)(260.29035626,160.60252846)(259.40234375,160.60253906)
\curveto(258.33528009,160.60252846)(257.49380958,160.26235953)(256.87792969,159.58203125)
\curveto(256.26203998,158.90168381)(255.95409758,157.97427588)(255.95410156,156.79980469)
\lineto(255.95410156,150)
\lineto(253.96679688,150)
\lineto(253.96679688,162.03125)
\lineto(255.95410156,162.03125)
\lineto(255.95410156,160.16210938)
\curveto(256.42675335,160.88540578)(256.98176582,161.42609534)(257.61914062,161.78417969)
\curveto(258.26366558,162.14224046)(259.00487578,162.32127674)(259.84277344,162.32128906)
\curveto(261.22492564,162.32127674)(262.27049751,161.89158967)(262.97949219,161.03222656)
\curveto(263.68846484,160.18000284)(264.04295668,158.92316816)(264.04296875,157.26171875)
}
}
{
\newrgbcolor{curcolor}{0 0 0}
\pscustom[linestyle=none,fillstyle=solid,fillcolor=curcolor]
{
\newpath
\moveto(278.29785156,156.50976562)
\lineto(278.29785156,155.54296875)
\lineto(269.20996094,155.54296875)
\curveto(269.29589508,154.18228748)(269.70409779,153.14387706)(270.43457031,152.42773438)
\curveto(271.17219528,151.71874828)(272.1962828,151.36425645)(273.50683594,151.36425781)
\curveto(274.26594219,151.36425645)(274.99999093,151.45735531)(275.70898438,151.64355469)
\curveto(276.42511972,151.82975077)(277.13410338,152.10904737)(277.8359375,152.48144531)
\lineto(277.8359375,150.61230469)
\curveto(277.12694193,150.31152313)(276.40005464,150.08235669)(275.65527344,149.92480469)
\curveto(274.91047279,149.76725284)(274.1549397,149.68847687)(273.38867188,149.68847656)
\curveto(271.46939551,149.68847687)(269.94758713,150.24707007)(268.82324219,151.36425781)
\curveto(267.70605292,152.48144283)(267.14745972,153.99250903)(267.14746094,155.89746094)
\curveto(267.14745972,157.86685411)(267.67740711,159.42805047)(268.73730469,160.58105469)
\curveto(269.80435811,161.7411992)(271.24022907,162.32127674)(273.04492188,162.32128906)
\curveto(274.66340273,162.32127674)(275.94172176,161.79849081)(276.87988281,160.75292969)
\curveto(277.82518342,159.71450851)(278.2978392,158.30012191)(278.29785156,156.50976562)
\moveto(276.32128906,157.08984375)
\curveto(276.30695577,158.17121579)(276.0025941,159.03417065)(275.40820312,159.67871094)
\curveto(274.82095465,160.32323186)(274.04035648,160.64549717)(273.06640625,160.64550781)
\curveto(271.96353564,160.64549717)(271.07909642,160.33397404)(270.41308594,159.7109375)
\curveto(269.75422795,159.08788154)(269.37467104,158.21060377)(269.27441406,157.07910156)
\lineto(276.32128906,157.08984375)
}
}
{
\newrgbcolor{curcolor}{0 0 0}
\pscustom[linestyle=none,fillstyle=solid,fillcolor=curcolor]
{
\newpath
\moveto(290.90917969,159.72167969)
\curveto(291.40330838,160.60968991)(291.9941281,161.26496269)(292.68164062,161.6875)
\curveto(293.36912672,162.11001393)(294.17837071,162.32127674)(295.109375,162.32128906)
\curveto(296.36261331,162.32127674)(297.32940922,161.88084749)(298.00976562,161)
\curveto(298.69008495,160.12629196)(299.03025388,158.88019945)(299.03027344,157.26171875)
\lineto(299.03027344,150)
\lineto(297.04296875,150)
\lineto(297.04296875,157.19726562)
\curveto(297.04295118,158.35025207)(296.83884982,159.20604548)(296.43066406,159.76464844)
\curveto(296.02244438,160.32323186)(295.39939813,160.60252846)(294.56152344,160.60253906)
\curveto(293.53742083,160.60252846)(292.72817684,160.26235953)(292.13378906,159.58203125)
\curveto(291.53937595,158.90168381)(291.24217573,157.97427588)(291.2421875,156.79980469)
\lineto(291.2421875,150)
\lineto(289.25488281,150)
\lineto(289.25488281,157.19726562)
\curveto(289.25487303,158.35741352)(289.05077167,159.21320693)(288.64257812,159.76464844)
\curveto(288.23436623,160.32323186)(287.60415853,160.60252846)(286.75195312,160.60253906)
\curveto(285.74218123,160.60252846)(284.9400987,160.2587788)(284.34570312,159.57128906)
\curveto(283.7512978,158.89094163)(283.45409758,157.96711443)(283.45410156,156.79980469)
\lineto(283.45410156,150)
\lineto(281.46679688,150)
\lineto(281.46679688,162.03125)
\lineto(283.45410156,162.03125)
\lineto(283.45410156,160.16210938)
\curveto(283.905269,160.89972868)(284.44595856,161.44399897)(285.07617188,161.79492188)
\curveto(285.70637397,162.14582119)(286.45474562,162.32127674)(287.32128906,162.32128906)
\curveto(288.19497825,162.32127674)(288.93618845,162.09927175)(289.54492188,161.65527344)
\curveto(290.1607966,161.21125181)(290.61554875,160.5667212)(290.90917969,159.72167969)
}
}
{
\newrgbcolor{curcolor}{0 0 0}
\pscustom[linestyle=none,fillstyle=solid,fillcolor=curcolor]
{
\newpath
\moveto(302.98339844,162.03125)
\lineto(304.95996094,162.03125)
\lineto(304.95996094,150)
\lineto(302.98339844,150)
\lineto(302.98339844,162.03125)
\moveto(302.98339844,166.71484375)
\lineto(304.95996094,166.71484375)
\lineto(304.95996094,164.21191406)
\lineto(302.98339844,164.21191406)
\lineto(302.98339844,166.71484375)
}
}
{
\newrgbcolor{curcolor}{0 0 0}
\pscustom[linestyle=none,fillstyle=solid,fillcolor=curcolor]
{
\newpath
\moveto(316.75488281,161.67675781)
\lineto(316.75488281,159.80761719)
\curveto(316.19627988,160.09406543)(315.61620233,160.30890896)(315.01464844,160.45214844)
\curveto(314.41307854,160.59536701)(313.79003228,160.66698152)(313.14550781,160.66699219)
\curveto(312.16438287,160.66698152)(311.4267534,160.51659105)(310.93261719,160.21582031)
\curveto(310.44563459,159.91502915)(310.20214525,159.46385772)(310.20214844,158.86230469)
\curveto(310.20214525,158.40396295)(310.3776008,158.04230967)(310.72851562,157.77734375)
\curveto(311.07942302,157.51952373)(311.78482596,157.27245367)(312.84472656,157.03613281)
\lineto(313.52148438,156.88574219)
\curveto(314.92512229,156.58495435)(315.92056401,156.15884801)(316.5078125,155.60742188)
\curveto(317.10220345,155.06314598)(317.39940367,154.30045143)(317.39941406,153.31933594)
\curveto(317.39940367,152.20214624)(316.9553937,151.31770702)(316.06738281,150.66601562)
\curveto(315.18651526,150.0143229)(313.97264929,149.68847687)(312.42578125,149.68847656)
\curveto(311.78124523,149.68847687)(311.10806882,149.75292993)(310.40625,149.88183594)
\curveto(309.71158584,150.00358073)(308.9775371,150.18977846)(308.20410156,150.44042969)
\lineto(308.20410156,152.48144531)
\curveto(308.93456839,152.10188592)(309.65429423,151.81542787)(310.36328125,151.62207031)
\curveto(311.07226156,151.43587096)(311.77408378,151.34277209)(312.46875,151.34277344)
\curveto(313.3997332,151.34277209)(314.11587831,151.50032402)(314.6171875,151.81542969)
\curveto(315.11848148,152.13769317)(315.36913227,152.5888646)(315.36914062,153.16894531)
\curveto(315.36913227,153.70605098)(315.18651526,154.11783442)(314.82128906,154.40429688)
\curveto(314.46320869,154.69075052)(313.67186834,154.96646639)(312.44726562,155.23144531)
\lineto(311.75976562,155.39257812)
\curveto(310.53515273,155.65038497)(309.65071351,156.04426479)(309.10644531,156.57421875)
\curveto(308.56217293,157.11132101)(308.29003778,157.84536976)(308.29003906,158.77636719)
\curveto(308.29003778,159.9078677)(308.69107905,160.78156474)(309.49316406,161.39746094)
\curveto(310.29524411,162.01333434)(311.43391485,162.32127674)(312.90917969,162.32128906)
\curveto(313.63964181,162.32127674)(314.32714112,162.26756586)(314.97167969,162.16015625)
\curveto(315.61620233,162.05272232)(316.21060278,161.89158967)(316.75488281,161.67675781)
}
}
{
\newrgbcolor{curcolor}{0 0 0}
\pscustom[linestyle=none,fillstyle=solid,fillcolor=curcolor]
{
\newpath
\moveto(165.00097656,124.25488281)
\lineto(165.00097656,111.78320312)
\lineto(167.62207031,111.78320312)
\curveto(169.83495177,111.78320134)(171.45343974,112.28450292)(172.47753906,113.28710938)
\curveto(173.50877623,114.28970925)(174.02440071,115.87238996)(174.02441406,118.03515625)
\curveto(174.02440071,120.18358357)(173.50877623,121.7555221)(172.47753906,122.75097656)
\curveto(171.45343974,123.75356698)(169.83495177,124.25486856)(167.62207031,124.25488281)
\lineto(165.00097656,124.25488281)
\moveto(162.83105469,126.03808594)
\lineto(167.2890625,126.03808594)
\curveto(170.39712569,126.0380699)(172.67804789,125.38995857)(174.13183594,124.09375)
\curveto(175.58559707,122.8046747)(176.31248436,120.78514546)(176.3125,118.03515625)
\curveto(176.31248436,115.27082806)(175.58201634,113.24055666)(174.12109375,111.94433594)
\curveto(172.66014426,110.64811133)(170.38280279,110)(167.2890625,110)
\lineto(162.83105469,110)
\lineto(162.83105469,126.03808594)
}
}
{
\newrgbcolor{curcolor}{0 0 0}
\pscustom[linestyle=none,fillstyle=solid,fillcolor=curcolor]
{
\newpath
\moveto(179.67480469,122.03125)
\lineto(181.65136719,122.03125)
\lineto(181.65136719,110)
\lineto(179.67480469,110)
\lineto(179.67480469,122.03125)
\moveto(179.67480469,126.71484375)
\lineto(181.65136719,126.71484375)
\lineto(181.65136719,124.21191406)
\lineto(179.67480469,124.21191406)
\lineto(179.67480469,126.71484375)
}
}
{
\newrgbcolor{curcolor}{0 0 0}
\pscustom[linestyle=none,fillstyle=solid,fillcolor=curcolor]
{
\newpath
\moveto(191.8671875,126.71484375)
\lineto(191.8671875,125.07128906)
\lineto(189.9765625,125.07128906)
\curveto(189.26757256,125.07127399)(188.77343243,124.92804497)(188.49414062,124.64160156)
\curveto(188.22200069,124.35512887)(188.08593312,123.83950439)(188.0859375,123.09472656)
\lineto(188.0859375,122.03125)
\lineto(193.51074219,122.03125)
\lineto(193.51074219,122.86914062)
\curveto(193.51073238,124.20831913)(193.82225551,125.18227648)(194.4453125,125.79101562)
\curveto(194.64582239,125.99152047)(194.885731,126.15981457)(195.16503906,126.29589844)
\curveto(195.72362079,126.57517874)(196.47199244,126.71482704)(197.41015625,126.71484375)
\lineto(199.27929688,126.71484375)
\lineto(199.27929688,125.07128906)
\lineto(197.38867188,125.07128906)
\curveto(196.67967452,125.07127399)(196.18553439,124.92804497)(195.90625,124.64160156)
\curveto(195.63410265,124.35512887)(195.49803508,123.83950439)(195.49804688,123.09472656)
\lineto(195.49804688,122.03125)
\lineto(202.91015625,122.03125)
\lineto(202.91015625,110)
\lineto(200.92285156,110)
\lineto(200.92285156,120.49511719)
\lineto(195.49804688,120.49511719)
\lineto(195.49804688,110)
\lineto(193.51074219,110)
\lineto(193.51074219,120.49511719)
\lineto(188.0859375,120.49511719)
\lineto(188.0859375,110)
\lineto(186.09863281,110)
\lineto(186.09863281,120.49511719)
\lineto(184.20800781,120.49511719)
\lineto(184.20800781,122.03125)
\lineto(186.09863281,122.03125)
\lineto(186.09863281,122.86914062)
\curveto(186.09863042,124.20831913)(186.41015354,125.18227648)(187.03320312,125.79101562)
\curveto(187.65624605,126.40688463)(188.64452631,126.71482704)(189.99804688,126.71484375)
\lineto(191.8671875,126.71484375)
\moveto(200.92285156,126.69335938)
\lineto(202.91015625,126.69335938)
\lineto(202.91015625,124.19042969)
\lineto(200.92285156,124.19042969)
\lineto(200.92285156,126.69335938)
}
}
{
\newrgbcolor{curcolor}{0 0 0}
\pscustom[linestyle=none,fillstyle=solid,fillcolor=curcolor]
{
\newpath
\moveto(215.70410156,121.56933594)
\lineto(215.70410156,119.72167969)
\curveto(215.14549764,120.02961237)(214.58332372,120.2587788)(214.01757812,120.40917969)
\curveto(213.45897589,120.5667212)(212.89322125,120.64549717)(212.3203125,120.64550781)
\curveto(211.03840539,120.64549717)(210.04296368,120.23729445)(209.33398438,119.42089844)
\curveto(208.62499635,118.61164503)(208.27050451,117.4729743)(208.27050781,116.00488281)
\curveto(208.27050451,114.53677932)(208.62499635,113.39452786)(209.33398438,112.578125)
\curveto(210.04296368,111.76887844)(211.03840539,111.36425645)(212.3203125,111.36425781)
\curveto(212.89322125,111.36425645)(213.45897589,111.43945169)(214.01757812,111.58984375)
\curveto(214.58332372,111.74739409)(215.14549764,111.98014125)(215.70410156,112.28808594)
\lineto(215.70410156,110.46191406)
\curveto(215.15265909,110.20410136)(214.579743,110.01074218)(213.98535156,109.88183594)
\curveto(213.39810355,109.75292993)(212.77147658,109.68847687)(212.10546875,109.68847656)
\curveto(210.29361447,109.68847687)(208.85416279,110.25781224)(207.78710938,111.39648438)
\curveto(206.72005034,112.53515371)(206.18652222,114.07128499)(206.18652344,116.00488281)
\curveto(206.18652222,117.96711443)(206.72363106,119.51040716)(207.79785156,120.63476562)
\curveto(208.87922786,121.75910282)(210.35806753,122.32127674)(212.234375,122.32128906)
\curveto(212.84309109,122.32127674)(213.43749154,122.25682368)(214.01757812,122.12792969)
\curveto(214.59764663,122.00617289)(215.15982054,121.81997516)(215.70410156,121.56933594)
}
}
{
\newrgbcolor{curcolor}{0 0 0}
\pscustom[linestyle=none,fillstyle=solid,fillcolor=curcolor]
{
\newpath
\moveto(218.95898438,114.74804688)
\lineto(218.95898438,122.03125)
\lineto(220.93554688,122.03125)
\lineto(220.93554688,114.82324219)
\curveto(220.93554303,113.68456663)(221.15754802,112.82877321)(221.6015625,112.25585938)
\curveto(222.04556796,111.69010248)(222.71158292,111.40722516)(223.59960938,111.40722656)
\curveto(224.66665909,111.40722516)(225.5081296,111.74739409)(226.12402344,112.42773438)
\curveto(226.74706066,113.10806981)(227.05858378,114.03547774)(227.05859375,115.20996094)
\lineto(227.05859375,122.03125)
\lineto(229.03515625,122.03125)
\lineto(229.03515625,110)
\lineto(227.05859375,110)
\lineto(227.05859375,111.84765625)
\curveto(226.57876655,111.11718638)(226.02017336,110.57291609)(225.3828125,110.21484375)
\curveto(224.7525965,109.86393243)(224.01854776,109.68847687)(223.18066406,109.68847656)
\curveto(221.7984979,109.68847687)(220.7493453,110.11816394)(220.03320312,110.97753906)
\curveto(219.31705506,111.83691223)(218.95898251,113.09374691)(218.95898438,114.74804688)
\moveto(223.93261719,122.32128906)
\lineto(223.93261719,122.32128906)
}
}
{
\newrgbcolor{curcolor}{0 0 0}
\pscustom[linestyle=none,fillstyle=solid,fillcolor=curcolor]
{
\newpath
\moveto(233.12792969,126.71484375)
\lineto(235.10449219,126.71484375)
\lineto(235.10449219,110)
\lineto(233.12792969,110)
\lineto(233.12792969,126.71484375)
}
}
{
\newrgbcolor{curcolor}{0 0 0}
\pscustom[linestyle=none,fillstyle=solid,fillcolor=curcolor]
{
\newpath
\moveto(241.18457031,125.44726562)
\lineto(241.18457031,122.03125)
\lineto(245.25585938,122.03125)
\lineto(245.25585938,120.49511719)
\lineto(241.18457031,120.49511719)
\lineto(241.18457031,113.96386719)
\curveto(241.18456628,112.98274441)(241.31705313,112.35253671)(241.58203125,112.07324219)
\curveto(241.85416197,111.79394352)(242.40201298,111.65429522)(243.22558594,111.65429688)
\lineto(245.25585938,111.65429688)
\lineto(245.25585938,110)
\lineto(243.22558594,110)
\curveto(241.70019077,110)(240.64745745,110.28287732)(240.06738281,110.84863281)
\curveto(239.48730236,111.42154806)(239.19726358,112.45995848)(239.19726562,113.96386719)
\lineto(239.19726562,120.49511719)
\lineto(237.74707031,120.49511719)
\lineto(237.74707031,122.03125)
\lineto(239.19726562,122.03125)
\lineto(239.19726562,125.44726562)
\lineto(241.18457031,125.44726562)
}
}
{
\newrgbcolor{curcolor}{0 0 0}
\pscustom[linestyle=none,fillstyle=solid,fillcolor=curcolor]
{
\newpath
\moveto(258.15722656,116.50976562)
\lineto(258.15722656,115.54296875)
\lineto(249.06933594,115.54296875)
\curveto(249.15527008,114.18228748)(249.56347279,113.14387706)(250.29394531,112.42773438)
\curveto(251.03157028,111.71874828)(252.0556578,111.36425645)(253.36621094,111.36425781)
\curveto(254.12531719,111.36425645)(254.85936593,111.45735531)(255.56835938,111.64355469)
\curveto(256.28449472,111.82975077)(256.99347838,112.10904737)(257.6953125,112.48144531)
\lineto(257.6953125,110.61230469)
\curveto(256.98631693,110.31152313)(256.25942964,110.08235669)(255.51464844,109.92480469)
\curveto(254.76984779,109.76725284)(254.0143147,109.68847687)(253.24804688,109.68847656)
\curveto(251.32877051,109.68847687)(249.80696213,110.24707007)(248.68261719,111.36425781)
\curveto(247.56542792,112.48144283)(247.00683472,113.99250903)(247.00683594,115.89746094)
\curveto(247.00683472,117.86685411)(247.53678211,119.42805047)(248.59667969,120.58105469)
\curveto(249.66373311,121.7411992)(251.09960407,122.32127674)(252.90429688,122.32128906)
\curveto(254.52277773,122.32127674)(255.80109676,121.79849081)(256.73925781,120.75292969)
\curveto(257.68455842,119.71450851)(258.1572142,118.30012191)(258.15722656,116.50976562)
\moveto(256.18066406,117.08984375)
\curveto(256.16633077,118.17121579)(255.8619691,119.03417065)(255.26757812,119.67871094)
\curveto(254.68032965,120.32323186)(253.89973148,120.64549717)(252.92578125,120.64550781)
\curveto(251.82291064,120.64549717)(250.93847142,120.33397404)(250.27246094,119.7109375)
\curveto(249.61360295,119.08788154)(249.23404604,118.21060377)(249.13378906,117.07910156)
\lineto(256.18066406,117.08984375)
\moveto(254.27929688,127.59570312)
\lineto(256.41699219,127.59570312)
\lineto(252.91503906,123.55664062)
\lineto(251.27148438,123.55664062)
\lineto(254.27929688,127.59570312)
}
}
{
\newrgbcolor{curcolor}{0 0 0}
\pscustom[linestyle=none,fillstyle=solid,fillcolor=curcolor]
{
\newpath
\moveto(149.99414062,92.70507812)
\lineto(149.99414062,99.21484375)
\lineto(151.97070312,99.21484375)
\lineto(151.97070312,82.5)
\lineto(149.99414062,82.5)
\lineto(149.99414062,84.3046875)
\curveto(149.57876647,83.58854058)(149.05239981,83.05501247)(148.41503906,82.70410156)
\curveto(147.78482295,82.3603517)(147.02570912,82.18847687)(146.13769531,82.18847656)
\curveto(144.68391459,82.18847687)(143.49869442,82.76855442)(142.58203125,83.92871094)
\curveto(141.67252437,85.0888646)(141.21777222,86.6142537)(141.21777344,88.50488281)
\curveto(141.21777222,90.39549992)(141.67252437,91.92088902)(142.58203125,93.08105469)
\curveto(143.49869442,94.2411992)(144.68391459,94.82127674)(146.13769531,94.82128906)
\curveto(147.02570912,94.82127674)(147.78482295,94.64582119)(148.41503906,94.29492188)
\curveto(149.05239981,93.95116042)(149.57876647,93.42121304)(149.99414062,92.70507812)
\moveto(143.25878906,88.50488281)
\curveto(143.25878581,87.05110222)(143.55598603,85.90885076)(144.15039062,85.078125)
\curveto(144.75194838,84.25455554)(145.57551526,83.84277209)(146.62109375,83.84277344)
\curveto(147.666659,83.84277209)(148.49022589,84.25455554)(149.09179688,85.078125)
\curveto(149.69334969,85.90885076)(149.99413063,87.05110222)(149.99414062,88.50488281)
\curveto(149.99413063,89.9586514)(149.69334969,91.09732213)(149.09179688,91.92089844)
\curveto(148.49022589,92.75161735)(147.666659,93.16698152)(146.62109375,93.16699219)
\curveto(145.57551526,93.16698152)(144.75194838,92.75161735)(144.15039062,91.92089844)
\curveto(143.55598603,91.09732213)(143.25878581,89.9586514)(143.25878906,88.50488281)
}
}
{
\newrgbcolor{curcolor}{0 0 0}
\pscustom[linestyle=none,fillstyle=solid,fillcolor=curcolor]
{
\newpath
\moveto(166.33300781,89.00976562)
\lineto(166.33300781,88.04296875)
\lineto(157.24511719,88.04296875)
\curveto(157.33105133,86.68228748)(157.73925404,85.64387706)(158.46972656,84.92773438)
\curveto(159.20735153,84.21874828)(160.23143905,83.86425645)(161.54199219,83.86425781)
\curveto(162.30109844,83.86425645)(163.03514718,83.95735531)(163.74414062,84.14355469)
\curveto(164.46027597,84.32975077)(165.16925963,84.60904737)(165.87109375,84.98144531)
\lineto(165.87109375,83.11230469)
\curveto(165.16209818,82.81152313)(164.43521089,82.58235669)(163.69042969,82.42480469)
\curveto(162.94562904,82.26725284)(162.19009595,82.18847687)(161.42382812,82.18847656)
\curveto(159.50455176,82.18847687)(157.98274338,82.74707007)(156.85839844,83.86425781)
\curveto(155.74120917,84.98144283)(155.18261597,86.49250903)(155.18261719,88.39746094)
\curveto(155.18261597,90.36685411)(155.71256336,91.92805047)(156.77246094,93.08105469)
\curveto(157.83951436,94.2411992)(159.27538532,94.82127674)(161.08007812,94.82128906)
\curveto(162.69855898,94.82127674)(163.97687801,94.29849081)(164.91503906,93.25292969)
\curveto(165.86033967,92.21450851)(166.33299545,90.80012191)(166.33300781,89.00976562)
\moveto(164.35644531,89.58984375)
\curveto(164.34211202,90.67121579)(164.03775035,91.53417065)(163.44335938,92.17871094)
\curveto(162.8561109,92.82323186)(162.07551273,93.14549717)(161.1015625,93.14550781)
\curveto(159.99869189,93.14549717)(159.11425267,92.83397404)(158.44824219,92.2109375)
\curveto(157.7893842,91.58788154)(157.40982729,90.71060377)(157.30957031,89.57910156)
\lineto(164.35644531,89.58984375)
}
}
{
\newrgbcolor{curcolor}{0 0 0}
\pscustom[linestyle=none,fillstyle=solid,fillcolor=curcolor]
{
\newpath
\moveto(177.24707031,94.17675781)
\lineto(177.24707031,92.30761719)
\curveto(176.68846738,92.59406543)(176.10838983,92.80890896)(175.50683594,92.95214844)
\curveto(174.90526604,93.09536701)(174.28221978,93.16698152)(173.63769531,93.16699219)
\curveto(172.65657037,93.16698152)(171.9189409,93.01659105)(171.42480469,92.71582031)
\curveto(170.93782209,92.41502915)(170.69433275,91.96385772)(170.69433594,91.36230469)
\curveto(170.69433275,90.90396295)(170.8697883,90.54230967)(171.22070312,90.27734375)
\curveto(171.57161052,90.01952373)(172.27701346,89.77245367)(173.33691406,89.53613281)
\lineto(174.01367188,89.38574219)
\curveto(175.41730979,89.08495435)(176.41275151,88.65884801)(177,88.10742188)
\curveto(177.59439095,87.56314598)(177.89159117,86.80045143)(177.89160156,85.81933594)
\curveto(177.89159117,84.70214624)(177.4475812,83.81770702)(176.55957031,83.16601562)
\curveto(175.67870276,82.5143229)(174.46483679,82.18847687)(172.91796875,82.18847656)
\curveto(172.27343273,82.18847687)(171.60025632,82.25292993)(170.8984375,82.38183594)
\curveto(170.20377334,82.50358073)(169.4697246,82.68977846)(168.69628906,82.94042969)
\lineto(168.69628906,84.98144531)
\curveto(169.42675589,84.60188592)(170.14648173,84.31542787)(170.85546875,84.12207031)
\curveto(171.56444906,83.93587096)(172.26627128,83.84277209)(172.9609375,83.84277344)
\curveto(173.8919207,83.84277209)(174.60806581,84.00032402)(175.109375,84.31542969)
\curveto(175.61066898,84.63769317)(175.86131977,85.0888646)(175.86132812,85.66894531)
\curveto(175.86131977,86.20605098)(175.67870276,86.61783442)(175.31347656,86.90429688)
\curveto(174.95539619,87.19075052)(174.16405584,87.46646639)(172.93945312,87.73144531)
\lineto(172.25195312,87.89257812)
\curveto(171.02734023,88.15038497)(170.14290101,88.54426479)(169.59863281,89.07421875)
\curveto(169.05436043,89.61132101)(168.78222528,90.34536976)(168.78222656,91.27636719)
\curveto(168.78222528,92.4078677)(169.18326655,93.28156474)(169.98535156,93.89746094)
\curveto(170.78743161,94.51333434)(171.92610235,94.82127674)(173.40136719,94.82128906)
\curveto(174.13182931,94.82127674)(174.81932862,94.76756586)(175.46386719,94.66015625)
\curveto(176.10838983,94.55272232)(176.70279028,94.39158967)(177.24707031,94.17675781)
}
}
{
\newrgbcolor{curcolor}{0 0 0}
\pscustom[linestyle=none,fillstyle=solid,fillcolor=curcolor]
{
}
}
{
\newrgbcolor{curcolor}{0 0 0}
\pscustom[linestyle=none,fillstyle=solid,fillcolor=curcolor]
{
\newpath
\moveto(198.34472656,89.00976562)
\lineto(198.34472656,88.04296875)
\lineto(189.25683594,88.04296875)
\curveto(189.34277008,86.68228748)(189.75097279,85.64387706)(190.48144531,84.92773438)
\curveto(191.21907028,84.21874828)(192.2431578,83.86425645)(193.55371094,83.86425781)
\curveto(194.31281719,83.86425645)(195.04686593,83.95735531)(195.75585938,84.14355469)
\curveto(196.47199472,84.32975077)(197.18097838,84.60904737)(197.8828125,84.98144531)
\lineto(197.8828125,83.11230469)
\curveto(197.17381693,82.81152313)(196.44692964,82.58235669)(195.70214844,82.42480469)
\curveto(194.95734779,82.26725284)(194.2018147,82.18847687)(193.43554688,82.18847656)
\curveto(191.51627051,82.18847687)(189.99446213,82.74707007)(188.87011719,83.86425781)
\curveto(187.75292792,84.98144283)(187.19433472,86.49250903)(187.19433594,88.39746094)
\curveto(187.19433472,90.36685411)(187.72428211,91.92805047)(188.78417969,93.08105469)
\curveto(189.85123311,94.2411992)(191.28710407,94.82127674)(193.09179688,94.82128906)
\curveto(194.71027773,94.82127674)(195.98859676,94.29849081)(196.92675781,93.25292969)
\curveto(197.87205842,92.21450851)(198.3447142,90.80012191)(198.34472656,89.00976562)
\moveto(196.36816406,89.58984375)
\curveto(196.35383077,90.67121579)(196.0494691,91.53417065)(195.45507812,92.17871094)
\curveto(194.86782965,92.82323186)(194.08723148,93.14549717)(193.11328125,93.14550781)
\curveto(192.01041064,93.14549717)(191.12597142,92.83397404)(190.45996094,92.2109375)
\curveto(189.80110295,91.58788154)(189.42154604,90.71060377)(189.32128906,89.57910156)
\lineto(196.36816406,89.58984375)
}
}
{
\newrgbcolor{curcolor}{0 0 0}
\pscustom[linestyle=none,fillstyle=solid,fillcolor=curcolor]
{
\newpath
\moveto(211.58984375,89.76171875)
\lineto(211.58984375,82.5)
\lineto(209.61328125,82.5)
\lineto(209.61328125,89.69726562)
\curveto(209.61327115,90.83592916)(209.39126617,91.68814185)(208.94726562,92.25390625)
\curveto(208.50324622,92.81965114)(207.83723126,93.10252846)(206.94921875,93.10253906)
\curveto(205.88215509,93.10252846)(205.04068458,92.76235953)(204.42480469,92.08203125)
\curveto(203.80891498,91.40168381)(203.50097258,90.47427588)(203.50097656,89.29980469)
\lineto(203.50097656,82.5)
\lineto(201.51367188,82.5)
\lineto(201.51367188,94.53125)
\lineto(203.50097656,94.53125)
\lineto(203.50097656,92.66210938)
\curveto(203.97362835,93.38540578)(204.52864082,93.92609534)(205.16601562,94.28417969)
\curveto(205.81054058,94.64224046)(206.55175078,94.82127674)(207.38964844,94.82128906)
\curveto(208.77180064,94.82127674)(209.81737251,94.39158967)(210.52636719,93.53222656)
\curveto(211.23533984,92.68000284)(211.58983168,91.42316816)(211.58984375,89.76171875)
}
}
{
\newrgbcolor{curcolor}{0 0 0}
\pscustom[linestyle=none,fillstyle=solid,fillcolor=curcolor]
{
\newpath
\moveto(225.5546875,89.76171875)
\lineto(225.5546875,82.5)
\lineto(223.578125,82.5)
\lineto(223.578125,89.69726562)
\curveto(223.5781149,90.83592916)(223.35610992,91.68814185)(222.91210938,92.25390625)
\curveto(222.46808997,92.81965114)(221.80207501,93.10252846)(220.9140625,93.10253906)
\curveto(219.84699884,93.10252846)(219.00552833,92.76235953)(218.38964844,92.08203125)
\curveto(217.77375873,91.40168381)(217.46581633,90.47427588)(217.46582031,89.29980469)
\lineto(217.46582031,82.5)
\lineto(215.47851562,82.5)
\lineto(215.47851562,94.53125)
\lineto(217.46582031,94.53125)
\lineto(217.46582031,92.66210938)
\curveto(217.9384721,93.38540578)(218.49348457,93.92609534)(219.13085938,94.28417969)
\curveto(219.77538433,94.64224046)(220.51659453,94.82127674)(221.35449219,94.82128906)
\curveto(222.73664439,94.82127674)(223.78221626,94.39158967)(224.49121094,93.53222656)
\curveto(225.20018359,92.68000284)(225.55467543,91.42316816)(225.5546875,89.76171875)
}
}
{
\newrgbcolor{curcolor}{0 0 0}
\pscustom[linestyle=none,fillstyle=solid,fillcolor=curcolor]
{
\newpath
\moveto(239.80957031,89.00976562)
\lineto(239.80957031,88.04296875)
\lineto(230.72167969,88.04296875)
\curveto(230.80761383,86.68228748)(231.21581654,85.64387706)(231.94628906,84.92773438)
\curveto(232.68391403,84.21874828)(233.70800155,83.86425645)(235.01855469,83.86425781)
\curveto(235.77766094,83.86425645)(236.51170968,83.95735531)(237.22070312,84.14355469)
\curveto(237.93683847,84.32975077)(238.64582213,84.60904737)(239.34765625,84.98144531)
\lineto(239.34765625,83.11230469)
\curveto(238.63866068,82.81152313)(237.91177339,82.58235669)(237.16699219,82.42480469)
\curveto(236.42219154,82.26725284)(235.66665845,82.18847687)(234.90039062,82.18847656)
\curveto(232.98111426,82.18847687)(231.45930588,82.74707007)(230.33496094,83.86425781)
\curveto(229.21777167,84.98144283)(228.65917847,86.49250903)(228.65917969,88.39746094)
\curveto(228.65917847,90.36685411)(229.18912586,91.92805047)(230.24902344,93.08105469)
\curveto(231.31607686,94.2411992)(232.75194782,94.82127674)(234.55664062,94.82128906)
\curveto(236.17512148,94.82127674)(237.45344051,94.29849081)(238.39160156,93.25292969)
\curveto(239.33690217,92.21450851)(239.80955795,90.80012191)(239.80957031,89.00976562)
\moveto(237.83300781,89.58984375)
\curveto(237.81867452,90.67121579)(237.51431285,91.53417065)(236.91992188,92.17871094)
\curveto(236.3326734,92.82323186)(235.55207523,93.14549717)(234.578125,93.14550781)
\curveto(233.47525439,93.14549717)(232.59081517,92.83397404)(231.92480469,92.2109375)
\curveto(231.2659467,91.58788154)(230.88638979,90.71060377)(230.78613281,89.57910156)
\lineto(237.83300781,89.58984375)
}
}
{
\newrgbcolor{curcolor}{0 0 0}
\pscustom[linestyle=none,fillstyle=solid,fillcolor=curcolor]
{
\newpath
\moveto(252.42089844,92.22167969)
\curveto(252.91502713,93.10968991)(253.50584685,93.76496269)(254.19335938,94.1875)
\curveto(254.88084547,94.61001393)(255.69008946,94.82127674)(256.62109375,94.82128906)
\curveto(257.87433206,94.82127674)(258.84112797,94.38084749)(259.52148438,93.5)
\curveto(260.2018037,92.62629196)(260.54197263,91.38019945)(260.54199219,89.76171875)
\lineto(260.54199219,82.5)
\lineto(258.5546875,82.5)
\lineto(258.5546875,89.69726562)
\curveto(258.55466993,90.85025207)(258.35056857,91.70604548)(257.94238281,92.26464844)
\curveto(257.53416313,92.82323186)(256.91111688,93.10252846)(256.07324219,93.10253906)
\curveto(255.04913958,93.10252846)(254.23989559,92.76235953)(253.64550781,92.08203125)
\curveto(253.0510947,91.40168381)(252.75389448,90.47427588)(252.75390625,89.29980469)
\lineto(252.75390625,82.5)
\lineto(250.76660156,82.5)
\lineto(250.76660156,89.69726562)
\curveto(250.76659178,90.85741352)(250.56249042,91.71320693)(250.15429688,92.26464844)
\curveto(249.74608498,92.82323186)(249.11587728,93.10252846)(248.26367188,93.10253906)
\curveto(247.25389998,93.10252846)(246.45181745,92.7587788)(245.85742188,92.07128906)
\curveto(245.26301655,91.39094163)(244.96581633,90.46711443)(244.96582031,89.29980469)
\lineto(244.96582031,82.5)
\lineto(242.97851562,82.5)
\lineto(242.97851562,94.53125)
\lineto(244.96582031,94.53125)
\lineto(244.96582031,92.66210938)
\curveto(245.41698775,93.39972868)(245.95767731,93.94399897)(246.58789062,94.29492188)
\curveto(247.21809272,94.64582119)(247.96646437,94.82127674)(248.83300781,94.82128906)
\curveto(249.706697,94.82127674)(250.4479072,94.59927175)(251.05664062,94.15527344)
\curveto(251.67251535,93.71125181)(252.1272675,93.0667212)(252.42089844,92.22167969)
}
}
{
\newrgbcolor{curcolor}{0 0 0}
\pscustom[linestyle=none,fillstyle=solid,fillcolor=curcolor]
{
\newpath
\moveto(264.49511719,94.53125)
\lineto(266.47167969,94.53125)
\lineto(266.47167969,82.5)
\lineto(264.49511719,82.5)
\lineto(264.49511719,94.53125)
\moveto(264.49511719,99.21484375)
\lineto(266.47167969,99.21484375)
\lineto(266.47167969,96.71191406)
\lineto(264.49511719,96.71191406)
\lineto(264.49511719,99.21484375)
}
}
{
\newrgbcolor{curcolor}{0 0 0}
\pscustom[linestyle=none,fillstyle=solid,fillcolor=curcolor]
{
\newpath
\moveto(278.26660156,94.17675781)
\lineto(278.26660156,92.30761719)
\curveto(277.70799863,92.59406543)(277.12792108,92.80890896)(276.52636719,92.95214844)
\curveto(275.92479729,93.09536701)(275.30175103,93.16698152)(274.65722656,93.16699219)
\curveto(273.67610162,93.16698152)(272.93847215,93.01659105)(272.44433594,92.71582031)
\curveto(271.95735334,92.41502915)(271.713864,91.96385772)(271.71386719,91.36230469)
\curveto(271.713864,90.90396295)(271.88931955,90.54230967)(272.24023438,90.27734375)
\curveto(272.59114177,90.01952373)(273.29654471,89.77245367)(274.35644531,89.53613281)
\lineto(275.03320312,89.38574219)
\curveto(276.43684104,89.08495435)(277.43228276,88.65884801)(278.01953125,88.10742188)
\curveto(278.6139222,87.56314598)(278.91112242,86.80045143)(278.91113281,85.81933594)
\curveto(278.91112242,84.70214624)(278.46711245,83.81770702)(277.57910156,83.16601562)
\curveto(276.69823401,82.5143229)(275.48436804,82.18847687)(273.9375,82.18847656)
\curveto(273.29296398,82.18847687)(272.61978757,82.25292993)(271.91796875,82.38183594)
\curveto(271.22330459,82.50358073)(270.48925585,82.68977846)(269.71582031,82.94042969)
\lineto(269.71582031,84.98144531)
\curveto(270.44628714,84.60188592)(271.16601298,84.31542787)(271.875,84.12207031)
\curveto(272.58398031,83.93587096)(273.28580253,83.84277209)(273.98046875,83.84277344)
\curveto(274.91145195,83.84277209)(275.62759706,84.00032402)(276.12890625,84.31542969)
\curveto(276.63020023,84.63769317)(276.88085102,85.0888646)(276.88085938,85.66894531)
\curveto(276.88085102,86.20605098)(276.69823401,86.61783442)(276.33300781,86.90429688)
\curveto(275.97492744,87.19075052)(275.18358709,87.46646639)(273.95898438,87.73144531)
\lineto(273.27148438,87.89257812)
\curveto(272.04687148,88.15038497)(271.16243226,88.54426479)(270.61816406,89.07421875)
\curveto(270.07389168,89.61132101)(269.80175653,90.34536976)(269.80175781,91.27636719)
\curveto(269.80175653,92.4078677)(270.2027978,93.28156474)(271.00488281,93.89746094)
\curveto(271.80696286,94.51333434)(272.9456336,94.82127674)(274.42089844,94.82128906)
\curveto(275.15136056,94.82127674)(275.83885987,94.76756586)(276.48339844,94.66015625)
\curveto(277.12792108,94.55272232)(277.72232153,94.39158967)(278.26660156,94.17675781)
}
}
\end{pspicture}

		\end{center}
		
		\begin{enumerate}
		  \item Fond d'écran
		  \item Champs de texte ``Nom de la partie''
		  \item Carte sélectionnée
		  \item Carte précédente
		  \item Carte suivante
		  \item Nom de la carte sélectionnée
		  \item Liste déroulante ``Type de partie''
		  \item Liste déroulante ``Nombre d'ennemis''
		  \item Liste déroulante ``Difficulté des ennemis''
		  \item Bouton ``\hyperlink{Page d'accueil}{Retour}''
		  \item Bouton ``Lancer'' 
		\end{enumerate}

		\subsubsection{Description des zones}
		
			\begin{tabular}{|c|c|c|c|c|} \hline
				Numéro de zone & Type  & Description & Evènement &	Règle \\\hline
			\end{tabular}
			
		\subsubsection{Description des règles}

			\underline{RG6-01 :}
				\begin{quote}
				
				\end{quote}	
	
\newpage

	\subsection{Options}
	
		\hypertarget{Options}{}
		\label{Options}
	
		%LaTeX with PSTricks extensions
%%Creator: inkscape 0.48.0
%%Please note this file requires PSTricks extensions
\psset{xunit=.5pt,yunit=.5pt,runit=.5pt}
\begin{pspicture}(560,600)
{
\newrgbcolor{curcolor}{1 1 1}
\pscustom[linestyle=none,fillstyle=solid,fillcolor=curcolor]
{
\newpath
\moveto(133.12401581,597.52220317)
\lineto(426.87598419,597.52220317)
\curveto(443.85397169,597.52220317)(457.52217102,583.85400385)(457.52217102,566.87601635)
\lineto(457.52217102,33.12401744)
\curveto(457.52217102,16.14602994)(443.85397169,2.47783062)(426.87598419,2.47783062)
\lineto(133.12401581,2.47783062)
\curveto(116.14602831,2.47783062)(102.47782898,16.14602994)(102.47782898,33.12401744)
\lineto(102.47782898,566.87601635)
\curveto(102.47782898,583.85400385)(116.14602831,597.52220317)(133.12401581,597.52220317)
\closepath
}
}
{
\newrgbcolor{curcolor}{0 0 0}
\pscustom[linewidth=4.95566034,linecolor=curcolor]
{
\newpath
\moveto(133.12401581,597.52220317)
\lineto(426.87598419,597.52220317)
\curveto(443.85397169,597.52220317)(457.52217102,583.85400385)(457.52217102,566.87601635)
\lineto(457.52217102,33.12401744)
\curveto(457.52217102,16.14602994)(443.85397169,2.47783062)(426.87598419,2.47783062)
\lineto(133.12401581,2.47783062)
\curveto(116.14602831,2.47783062)(102.47782898,16.14602994)(102.47782898,33.12401744)
\lineto(102.47782898,566.87601635)
\curveto(102.47782898,583.85400385)(116.14602831,597.52220317)(133.12401581,597.52220317)
\closepath
}
}
{
\newrgbcolor{curcolor}{0 0 0}
\pscustom[linestyle=none,fillstyle=solid,fillcolor=curcolor,opacity=0.11612902]
{
\newpath
\moveto(213.84012413,420.00000164)
\lineto(392.82668495,420.00000164)
\curveto(394.95411372,420.00000164)(396.66680908,418.19472938)(396.66680908,415.952306)
\curveto(396.66680908,413.70988262)(394.95411372,411.90461036)(392.82668495,411.90461036)
\lineto(213.84012413,411.90461036)
\curveto(211.71269536,411.90461036)(210,413.70988262)(210,415.952306)
\curveto(210,418.19472938)(211.71269536,420.00000164)(213.84012413,420.00000164)
\closepath
}
}
{
\newrgbcolor{curcolor}{0 0 0}
\pscustom[linewidth=1.67362297,linecolor=curcolor]
{
\newpath
\moveto(213.84012413,420.00000164)
\lineto(392.82668495,420.00000164)
\curveto(394.95411372,420.00000164)(396.66680908,418.19472938)(396.66680908,415.952306)
\curveto(396.66680908,413.70988262)(394.95411372,411.90461036)(392.82668495,411.90461036)
\lineto(213.84012413,411.90461036)
\curveto(211.71269536,411.90461036)(210,413.70988262)(210,415.952306)
\curveto(210,418.19472938)(211.71269536,420.00000164)(213.84012413,420.00000164)
\closepath
}
}
{
\newrgbcolor{curcolor}{1 1 1}
\pscustom[linestyle=none,fillstyle=solid,fillcolor=curcolor]
{
\newpath
\moveto(201.36239243,319.87677166)
\lineto(368.26269913,319.87677166)
\curveto(374.76516381,319.87677166)(380,314.64193547)(380,308.13947078)
\lineto(380,301.73730632)
\curveto(380,295.23484164)(374.76516381,290.00000545)(368.26269913,290.00000545)
\lineto(201.36239243,290.00000545)
\curveto(194.85992774,290.00000545)(189.62509155,295.23484164)(189.62509155,301.73730632)
\lineto(189.62509155,308.13947078)
\curveto(189.62509155,314.64193547)(194.85992774,319.87677166)(201.36239243,319.87677166)
\closepath
}
}
{
\newrgbcolor{curcolor}{0 0 0}
\pscustom[linewidth=1.86126256,linecolor=curcolor]
{
\newpath
\moveto(201.36239243,319.87677166)
\lineto(368.26269913,319.87677166)
\curveto(374.76516381,319.87677166)(380,314.64193547)(380,308.13947078)
\lineto(380,301.73730632)
\curveto(380,295.23484164)(374.76516381,290.00000545)(368.26269913,290.00000545)
\lineto(201.36239243,290.00000545)
\curveto(194.85992774,290.00000545)(189.62509155,295.23484164)(189.62509155,301.73730632)
\lineto(189.62509155,308.13947078)
\curveto(189.62509155,314.64193547)(194.85992774,319.87677166)(201.36239243,319.87677166)
\closepath
}
}
{
\newrgbcolor{curcolor}{1 1 1}
\pscustom[linestyle=none,fillstyle=solid,fillcolor=curcolor]
{
\newpath
\moveto(131.97757626,49.48121807)
\lineto(218.50890446,49.48121807)
\curveto(224.95737906,49.48121807)(230.14875031,44.28984682)(230.14875031,37.84137222)
\lineto(230.14875031,31.37479087)
\curveto(230.14875031,24.92631627)(224.95737906,19.73494503)(218.50890446,19.73494503)
\lineto(131.97757626,19.73494503)
\curveto(125.52910166,19.73494503)(120.33773041,24.92631627)(120.33773041,31.37479087)
\lineto(120.33773041,37.84137222)
\curveto(120.33773041,44.28984682)(125.52910166,49.48121807)(131.97757626,49.48121807)
\closepath
}
}
{
\newrgbcolor{curcolor}{0 0 0}
\pscustom[linewidth=2.04669738,linecolor=curcolor]
{
\newpath
\moveto(131.97757626,49.48121807)
\lineto(218.50890446,49.48121807)
\curveto(224.95737906,49.48121807)(230.14875031,44.28984682)(230.14875031,37.84137222)
\lineto(230.14875031,31.37479087)
\curveto(230.14875031,24.92631627)(224.95737906,19.73494503)(218.50890446,19.73494503)
\lineto(131.97757626,19.73494503)
\curveto(125.52910166,19.73494503)(120.33773041,24.92631627)(120.33773041,31.37479087)
\lineto(120.33773041,37.84137222)
\curveto(120.33773041,44.28984682)(125.52910166,49.48121807)(131.97757626,49.48121807)
\closepath
}
}
{
\newrgbcolor{curcolor}{0 0 0}
\pscustom[linestyle=none,fillstyle=solid,fillcolor=curcolor]
{
\newpath
\moveto(297.91565156,431.14060947)
\lineto(298.75114799,431.14060947)
\curveto(300.60260608,431.14060947)(302.09313011,429.98939017)(302.09313011,428.55940029)
\lineto(302.09313011,403.34521266)
\curveto(302.09313011,401.91522278)(300.60260608,400.76400348)(298.75114799,400.76400348)
\lineto(297.91565156,400.76400348)
\curveto(296.06419346,400.76400348)(294.57366943,401.91522278)(294.57366943,403.34521266)
\lineto(294.57366943,428.55940029)
\curveto(294.57366943,429.98939017)(296.06419346,431.14060947)(297.91565156,431.14060947)
\closepath
}
}
{
\newrgbcolor{curcolor}{0 0 0}
\pscustom[linewidth=0.62339705,linecolor=curcolor]
{
\newpath
\moveto(297.91565156,431.14060947)
\lineto(298.75114799,431.14060947)
\curveto(300.60260608,431.14060947)(302.09313011,429.98939017)(302.09313011,428.55940029)
\lineto(302.09313011,403.34521266)
\curveto(302.09313011,401.91522278)(300.60260608,400.76400348)(298.75114799,400.76400348)
\lineto(297.91565156,400.76400348)
\curveto(296.06419346,400.76400348)(294.57366943,401.91522278)(294.57366943,403.34521266)
\lineto(294.57366943,428.55940029)
\curveto(294.57366943,429.98939017)(296.06419346,431.14060947)(297.91565156,431.14060947)
\closepath
}
}
{
\newrgbcolor{curcolor}{1 1 1}
\pscustom[linestyle=none,fillstyle=solid,fillcolor=curcolor]
{
\newpath
\moveto(411.40200806,428.59799549)
\lineto(430.00000381,428.59799549)
\lineto(430.00000381,409.99999973)
\lineto(411.40200806,409.99999973)
\closepath
}
}
{
\newrgbcolor{curcolor}{0 0 0}
\pscustom[linewidth=1.40200508,linecolor=curcolor]
{
\newpath
\moveto(411.40200806,428.59799549)
\lineto(430.00000381,428.59799549)
\lineto(430.00000381,409.99999973)
\lineto(411.40200806,409.99999973)
\closepath
}
}
{
\newrgbcolor{curcolor}{0 0 0}
\pscustom[linestyle=none,fillstyle=solid,fillcolor=curcolor]
{
\newpath
\moveto(162.84375,426.92187664)
\lineto(162.84375,424.61328289)
\curveto(161.94530055,425.04295534)(161.09764515,425.36326752)(160.30078125,425.57422039)
\curveto(159.50389675,425.7851421)(158.73436627,425.89061075)(157.9921875,425.89062664)
\curveto(156.7031183,425.89061075)(155.70702554,425.640611)(155.00390625,425.14062664)
\curveto(154.30858944,424.640612)(153.96093354,423.92967521)(153.9609375,423.00781414)
\curveto(153.96093354,422.2343644)(154.19140206,421.64842749)(154.65234375,421.25000164)
\curveto(155.12108863,420.85936578)(156.00390025,420.54295984)(157.30078125,420.30078289)
\lineto(158.73046875,420.00781414)
\curveto(160.49608325,419.67186696)(161.7968632,419.07811756)(162.6328125,418.22656414)
\curveto(163.47654902,417.38280675)(163.8984236,416.24999539)(163.8984375,414.82812664)
\curveto(163.8984236,413.132811)(163.32811167,411.84765604)(162.1875,410.97265789)
\curveto(161.05467645,410.09765779)(159.39061561,409.66015823)(157.1953125,409.66015789)
\curveto(156.36718113,409.66015823)(155.48436952,409.75390813)(154.546875,409.94140789)
\curveto(153.61718388,410.12890776)(152.6523411,410.40625123)(151.65234375,410.77343914)
\lineto(151.65234375,413.21093914)
\curveto(152.61327864,412.67187396)(153.55468395,412.26562437)(154.4765625,411.99218914)
\curveto(155.3984321,411.71874992)(156.3046812,411.5820313)(157.1953125,411.58203289)
\curveto(158.54686645,411.5820313)(159.58983416,411.84765604)(160.32421875,412.37890789)
\curveto(161.05858269,412.91015498)(161.42576982,413.66796672)(161.42578125,414.65234539)
\curveto(161.42576982,415.51171487)(161.16014509,416.1835892)(160.62890625,416.66797039)
\curveto(160.10545864,417.15233823)(159.24217826,417.51561912)(158.0390625,417.75781414)
\lineto(156.59765625,418.03906414)
\curveto(154.83202642,418.39061825)(153.55468395,418.94139894)(152.765625,419.69140789)
\curveto(151.97656052,420.44139744)(151.58202967,421.48436515)(151.58203125,422.82031414)
\curveto(151.58202967,424.36717477)(152.12499787,425.58592355)(153.2109375,426.47656414)
\curveto(154.3046832,427.36717177)(155.80858794,427.81248382)(157.72265625,427.81250164)
\curveto(158.54296021,427.81248382)(159.37889687,427.73826515)(160.23046875,427.58984539)
\curveto(161.08202017,427.44139044)(161.95311305,427.21873442)(162.84375,426.92187664)
}
}
{
\newrgbcolor{curcolor}{0 0 0}
\pscustom[linestyle=none,fillstyle=solid,fillcolor=curcolor]
{
\newpath
\moveto(172.58203125,421.61328289)
\curveto(171.42577506,421.61327127)(170.51171347,421.16014673)(169.83984375,420.25390789)
\curveto(169.16796482,419.35546103)(168.83202765,418.12108726)(168.83203125,416.55078289)
\curveto(168.83202765,414.98046541)(169.16405857,413.74218539)(169.828125,412.83593914)
\curveto(170.49999473,411.9374997)(171.41796257,411.4882814)(172.58203125,411.48828289)
\curveto(173.73046025,411.4882814)(174.64061559,411.94140594)(175.3125,412.84765789)
\curveto(175.98436425,413.75390413)(176.32030141,414.9882779)(176.3203125,416.55078289)
\curveto(176.32030141,418.10546228)(175.98436425,419.3359298)(175.3125,420.24218914)
\curveto(174.64061559,421.15624048)(173.73046025,421.61327127)(172.58203125,421.61328289)
\moveto(172.58203125,423.44140789)
\curveto(174.45702203,423.44139444)(175.9296768,422.83202005)(177,421.61328289)
\curveto(178.07029966,420.39452249)(178.60545538,418.70702418)(178.60546875,416.55078289)
\curveto(178.60545538,414.40234098)(178.07029966,412.71484267)(177,411.48828289)
\curveto(175.9296768,410.26953262)(174.45702203,409.66015823)(172.58203125,409.66015789)
\curveto(170.69921329,409.66015823)(169.22265226,410.26953262)(168.15234375,411.48828289)
\curveto(167.08984189,412.71484267)(166.55859243,414.40234098)(166.55859375,416.55078289)
\curveto(166.55859243,418.70702418)(167.08984189,420.39452249)(168.15234375,421.61328289)
\curveto(169.22265226,422.83202005)(170.69921329,423.44139444)(172.58203125,423.44140789)
}
}
{
\newrgbcolor{curcolor}{0 0 0}
\pscustom[linestyle=none,fillstyle=solid,fillcolor=curcolor]
{
\newpath
\moveto(193.078125,417.92187664)
\lineto(193.078125,410.00000164)
\lineto(190.921875,410.00000164)
\lineto(190.921875,417.85156414)
\curveto(190.92186398,419.09374254)(190.67967673,420.02342911)(190.1953125,420.64062664)
\curveto(189.7109277,421.25780288)(188.98436592,421.56639632)(188.015625,421.56640789)
\curveto(186.85155555,421.56639632)(185.93358772,421.19530294)(185.26171875,420.45312664)
\curveto(184.58983907,419.71092942)(184.2539019,418.69921169)(184.25390625,417.41797039)
\lineto(184.25390625,410.00000164)
\lineto(182.0859375,410.00000164)
\lineto(182.0859375,423.12500164)
\lineto(184.25390625,423.12500164)
\lineto(184.25390625,421.08593914)
\curveto(184.76952639,421.87498976)(185.37499453,422.46483292)(186.0703125,422.85547039)
\curveto(186.77343063,423.24608214)(187.58202357,423.44139444)(188.49609375,423.44140789)
\curveto(190.00389615,423.44139444)(191.14452001,422.97264491)(191.91796875,422.03515789)
\curveto(192.69139346,421.10545928)(193.07811183,419.7343669)(193.078125,417.92187664)
}
}
{
\newrgbcolor{curcolor}{0 0 0}
\pscustom[linestyle=none,fillstyle=solid,fillcolor=curcolor,opacity=0.11612902]
{
\newpath
\moveto(267.26809502,380.00000164)
\lineto(410.85629463,380.00000164)
\curveto(415.92190741,380.00000164)(420,375.92190904)(420,370.85629627)
\lineto(420,357.7938573)
\curveto(420,352.72824453)(415.92190741,348.65015193)(410.85629463,348.65015193)
\lineto(267.26809502,348.65015193)
\curveto(262.20248224,348.65015193)(258.12438965,352.72824453)(258.12438965,357.7938573)
\lineto(258.12438965,370.85629627)
\curveto(258.12438965,375.92190904)(262.20248224,380.00000164)(267.26809502,380.00000164)
\closepath
}
}
{
\newrgbcolor{curcolor}{0 0 0}
\pscustom[linewidth=2.02831459,linecolor=curcolor]
{
\newpath
\moveto(267.26809502,380.00000164)
\lineto(410.85629463,380.00000164)
\curveto(415.92190741,380.00000164)(420,375.92190904)(420,370.85629627)
\lineto(420,357.7938573)
\curveto(420,352.72824453)(415.92190741,348.65015193)(410.85629463,348.65015193)
\lineto(267.26809502,348.65015193)
\curveto(262.20248224,348.65015193)(258.12438965,352.72824453)(258.12438965,357.7938573)
\lineto(258.12438965,370.85629627)
\curveto(258.12438965,375.92190904)(262.20248224,380.00000164)(267.26809502,380.00000164)
\closepath
}
}
{
\newrgbcolor{curcolor}{0 0 0}
\pscustom[linestyle=none,fillstyle=solid,fillcolor=curcolor]
{
\newpath
\moveto(152.35546875,377.49609539)
\lineto(154.72265625,377.49609539)
\lineto(154.72265625,361.99218914)
\lineto(163.2421875,361.99218914)
\lineto(163.2421875,360.00000164)
\lineto(152.35546875,360.00000164)
\lineto(152.35546875,377.49609539)
}
}
{
\newrgbcolor{curcolor}{0 0 0}
\pscustom[linestyle=none,fillstyle=solid,fillcolor=curcolor]
{
\newpath
\moveto(171.5859375,366.59765789)
\curveto(169.84374352,366.59765129)(168.63671347,366.39843274)(167.96484375,366.00000164)
\curveto(167.29296482,365.60155853)(166.95702765,364.92187171)(166.95703125,363.96093914)
\curveto(166.95702765,363.19531094)(167.2070274,362.58593655)(167.70703125,362.13281414)
\curveto(168.21483889,361.68749995)(168.90233821,361.46484392)(169.76953125,361.46484539)
\curveto(170.96483614,361.46484392)(171.92186644,361.8867185)(172.640625,362.73047039)
\curveto(173.36717749,363.5820293)(173.73045838,364.71093442)(173.73046875,366.11718914)
\lineto(173.73046875,366.59765789)
\lineto(171.5859375,366.59765789)
\moveto(175.88671875,367.48828289)
\lineto(175.88671875,360.00000164)
\lineto(173.73046875,360.00000164)
\lineto(173.73046875,361.99218914)
\curveto(173.23827137,361.19531294)(172.62499073,360.60546978)(171.890625,360.22265789)
\curveto(171.1562422,359.84765804)(170.2578056,359.66015823)(169.1953125,359.66015789)
\curveto(167.85155801,359.66015823)(166.78124658,360.03515785)(165.984375,360.78515789)
\curveto(165.19531066,361.54296884)(164.80077981,362.55468658)(164.80078125,363.82031414)
\curveto(164.80077981,365.29687134)(165.29296682,366.41015148)(166.27734375,367.16015789)
\curveto(167.26952734,367.91014998)(168.74608836,368.2851496)(170.70703125,368.28515789)
\lineto(173.73046875,368.28515789)
\lineto(173.73046875,368.49609539)
\curveto(173.73045838,369.4882734)(173.40233371,370.25389763)(172.74609375,370.79297039)
\curveto(172.09764751,371.33983405)(171.18358593,371.61327127)(170.00390625,371.61328289)
\curveto(169.25390036,371.61327127)(168.52343234,371.52342761)(167.8125,371.34375164)
\curveto(167.10155876,371.16405297)(166.41796569,370.89452199)(165.76171875,370.53515789)
\lineto(165.76171875,372.52734539)
\curveto(166.55077806,372.83202005)(167.31640229,373.05858233)(168.05859375,373.20703289)
\curveto(168.80077581,373.36326952)(169.52343134,373.44139444)(170.2265625,373.44140789)
\curveto(172.12499123,373.44139444)(173.54295857,372.94920744)(174.48046875,371.96484539)
\curveto(175.41795669,370.98045941)(175.88670622,369.4882734)(175.88671875,367.48828289)
}
}
{
\newrgbcolor{curcolor}{0 0 0}
\pscustom[linestyle=none,fillstyle=solid,fillcolor=curcolor]
{
\newpath
\moveto(191.25,367.92187664)
\lineto(191.25,360.00000164)
\lineto(189.09375,360.00000164)
\lineto(189.09375,367.85156414)
\curveto(189.09373898,369.09374254)(188.85155173,370.02342911)(188.3671875,370.64062664)
\curveto(187.8828027,371.25780288)(187.15624092,371.56639632)(186.1875,371.56640789)
\curveto(185.02343055,371.56639632)(184.10546272,371.19530294)(183.43359375,370.45312664)
\curveto(182.76171407,369.71092942)(182.4257769,368.69921169)(182.42578125,367.41797039)
\lineto(182.42578125,360.00000164)
\lineto(180.2578125,360.00000164)
\lineto(180.2578125,373.12500164)
\lineto(182.42578125,373.12500164)
\lineto(182.42578125,371.08593914)
\curveto(182.94140139,371.87498976)(183.54686953,372.46483292)(184.2421875,372.85547039)
\curveto(184.94530563,373.24608214)(185.75389857,373.44139444)(186.66796875,373.44140789)
\curveto(188.17577115,373.44139444)(189.31639501,372.97264491)(190.08984375,372.03515789)
\curveto(190.86326846,371.10545928)(191.24998683,369.7343669)(191.25,367.92187664)
}
}
{
\newrgbcolor{curcolor}{0 0 0}
\pscustom[linestyle=none,fillstyle=solid,fillcolor=curcolor]
{
\newpath
\moveto(204.2109375,366.71484539)
\curveto(204.2109266,368.27733711)(203.88670818,369.4882734)(203.23828125,370.34765789)
\curveto(202.59764696,371.20702168)(201.69530412,371.63670875)(200.53125,371.63672039)
\curveto(199.37499394,371.63670875)(198.47265109,371.20702168)(197.82421875,370.34765789)
\curveto(197.18358988,369.4882734)(196.8632777,368.27733711)(196.86328125,366.71484539)
\curveto(196.8632777,365.16015273)(197.18358988,363.95312268)(197.82421875,363.09375164)
\curveto(198.47265109,362.2343744)(199.37499394,361.80468733)(200.53125,361.80468914)
\curveto(201.69530412,361.80468733)(202.59764696,362.2343744)(203.23828125,363.09375164)
\curveto(203.88670818,363.95312268)(204.2109266,365.16015273)(204.2109375,366.71484539)
\moveto(206.3671875,361.62890789)
\curveto(206.36717445,359.39453349)(205.87108119,357.7343789)(204.87890625,356.64843914)
\curveto(203.88670818,355.55469358)(202.36717845,355.00781913)(200.3203125,355.00781414)
\curveto(199.56249375,355.00781913)(198.84765071,355.06641282)(198.17578125,355.18359539)
\curveto(197.50390206,355.29297509)(196.85155896,355.46484992)(196.21875,355.69922039)
\lineto(196.21875,357.79687664)
\curveto(196.85155896,357.45312918)(197.47655834,357.19922319)(198.09375,357.03515789)
\curveto(198.7109321,356.87109851)(199.33983772,356.78906735)(199.98046875,356.78906414)
\curveto(201.39452317,356.78906735)(202.45311586,357.16016073)(203.15625,357.90234539)
\curveto(203.85936445,358.63672175)(204.2109266,359.75000189)(204.2109375,361.24218914)
\lineto(204.2109375,362.30859539)
\curveto(203.76561455,361.53515635)(203.19530262,360.95703193)(202.5,360.57422039)
\curveto(201.80467901,360.19140769)(200.97264859,360.00000164)(200.00390625,360.00000164)
\curveto(198.39452617,360.00000164)(197.09765246,360.61328227)(196.11328125,361.83984539)
\curveto(195.12890443,363.06640482)(194.63671743,364.69140319)(194.63671875,366.71484539)
\curveto(194.63671743,368.74608664)(195.12890443,370.37499126)(196.11328125,371.60156414)
\curveto(197.09765246,372.82811381)(198.39452617,373.44139444)(200.00390625,373.44140789)
\curveto(200.97264859,373.44139444)(201.80467901,373.24998839)(202.5,372.86718914)
\curveto(203.19530262,372.48436415)(203.76561455,371.90623973)(204.2109375,371.13281414)
\lineto(204.2109375,373.12500164)
\lineto(206.3671875,373.12500164)
\lineto(206.3671875,361.62890789)
}
}
{
\newrgbcolor{curcolor}{0 0 0}
\pscustom[linestyle=none,fillstyle=solid,fillcolor=curcolor]
{
\newpath
\moveto(210.5859375,365.17968914)
\lineto(210.5859375,373.12500164)
\lineto(212.7421875,373.12500164)
\lineto(212.7421875,365.26172039)
\curveto(212.7421833,364.01952887)(212.98437056,363.08593605)(213.46875,362.46093914)
\curveto(213.95311959,361.84374979)(214.67968137,361.53515635)(215.6484375,361.53515789)
\curveto(216.81249173,361.53515635)(217.73045957,361.90624973)(218.40234375,362.64843914)
\curveto(219.08202071,363.39062325)(219.42186413,364.40234098)(219.421875,365.68359539)
\lineto(219.421875,373.12500164)
\lineto(221.578125,373.12500164)
\lineto(221.578125,360.00000164)
\lineto(219.421875,360.00000164)
\lineto(219.421875,362.01562664)
\curveto(218.89842715,361.21875042)(218.28905276,360.62500101)(217.59375,360.23437664)
\curveto(216.90624164,359.85156428)(216.10546119,359.66015823)(215.19140625,359.66015789)
\curveto(213.68358861,359.66015823)(212.53905851,360.12890776)(211.7578125,361.06640789)
\curveto(210.97656007,362.00390588)(210.58593546,363.37499826)(210.5859375,365.17968914)
\moveto(216.01171875,373.44140789)
\lineto(216.01171875,373.44140789)
}
}
{
\newrgbcolor{curcolor}{0 0 0}
\pscustom[linestyle=none,fillstyle=solid,fillcolor=curcolor]
{
\newpath
\moveto(237.26953125,367.10156414)
\lineto(237.26953125,366.04687664)
\lineto(227.35546875,366.04687664)
\curveto(227.44921508,364.56249707)(227.89452714,363.42968571)(228.69140625,362.64843914)
\curveto(229.49608804,361.87499976)(230.61327442,361.4882814)(232.04296875,361.48828289)
\curveto(232.87108466,361.4882814)(233.67186511,361.5898438)(234.4453125,361.79297039)
\curveto(235.22655105,361.99609339)(235.99998778,362.30078058)(236.765625,362.70703289)
\lineto(236.765625,360.66797039)
\curveto(235.99217529,360.33984505)(235.19920733,360.0898453)(234.38671875,359.91797039)
\curveto(233.57420896,359.74609564)(232.74999103,359.66015823)(231.9140625,359.66015789)
\curveto(229.82030646,359.66015823)(228.16015187,360.26953262)(226.93359375,361.48828289)
\curveto(225.71484182,362.70703018)(225.10546743,364.35546603)(225.10546875,366.43359539)
\curveto(225.10546743,368.5820243)(225.68359185,370.2851476)(226.83984375,371.54297039)
\curveto(228.00390203,372.80858258)(229.57030671,373.44139444)(231.5390625,373.44140789)
\curveto(233.30467798,373.44139444)(234.69920783,372.87108251)(235.72265625,371.73047039)
\curveto(236.75389328,370.59764729)(237.26951776,369.05468008)(237.26953125,367.10156414)
\moveto(235.11328125,367.73437664)
\curveto(235.09764493,368.91405522)(234.76561402,369.85546053)(234.1171875,370.55859539)
\curveto(233.4765528,371.26170912)(232.62499116,371.61327127)(231.5625,371.61328289)
\curveto(230.35936842,371.61327127)(229.39452564,371.27342786)(228.66796875,370.59375164)
\curveto(227.94921458,369.91405422)(227.5351525,368.95702393)(227.42578125,367.72265789)
\lineto(235.11328125,367.73437664)
}
}
{
\newrgbcolor{curcolor}{0 0 0}
\pscustom[linestyle=none,fillstyle=solid,fillcolor=curcolor]
{
\newpath
\moveto(249.17578125,372.73828289)
\lineto(249.17578125,370.69922039)
\curveto(248.56639623,371.01170937)(247.93358436,371.24608414)(247.27734375,371.40234539)
\curveto(246.62108568,371.55858383)(245.94139886,371.63670875)(245.23828125,371.63672039)
\curveto(244.16796313,371.63670875)(243.36327643,371.47264641)(242.82421875,371.14453289)
\curveto(242.292965,370.81639707)(242.02734027,370.32421006)(242.02734375,369.66797039)
\curveto(242.02734027,369.16796122)(242.21874633,368.77343036)(242.6015625,368.48437664)
\curveto(242.98437056,368.20311843)(243.75390104,367.93358745)(244.91015625,367.67578289)
\lineto(245.6484375,367.51172039)
\curveto(247.17967887,367.1835882)(248.26561528,366.71874492)(248.90625,366.11718914)
\curveto(249.55467649,365.52343361)(249.87889492,364.69140319)(249.87890625,363.62109539)
\curveto(249.87889492,362.40234298)(249.3945204,361.4375002)(248.42578125,360.72656414)
\curveto(247.46483483,360.01562662)(246.14061741,359.66015823)(244.453125,359.66015789)
\curveto(243.7499948,359.66015823)(243.01562053,359.73047066)(242.25,359.87109539)
\curveto(241.49218455,360.00390788)(240.69140411,360.20703268)(239.84765625,360.48047039)
\lineto(239.84765625,362.70703289)
\curveto(240.64452915,362.29296809)(241.42968462,361.98046841)(242.203125,361.76953289)
\curveto(242.97655807,361.56640632)(243.7421823,361.46484392)(244.5,361.46484539)
\curveto(245.51561803,361.46484392)(246.29686725,361.63671875)(246.84375,361.98047039)
\curveto(247.39061616,362.33203055)(247.66405338,362.82421756)(247.6640625,363.45703289)
\curveto(247.66405338,364.04296634)(247.46483483,364.49218464)(247.06640625,364.80468914)
\curveto(246.67577312,365.11718402)(245.81249273,365.41796497)(244.4765625,365.70703289)
\lineto(243.7265625,365.88281414)
\curveto(242.39062116,366.16405797)(241.42577837,366.59374504)(240.83203125,367.17187664)
\curveto(240.23827956,367.75780638)(239.94140486,368.55858683)(239.94140625,369.57422039)
\curveto(239.94140486,370.80858458)(240.37890442,371.76170862)(241.25390625,372.43359539)
\curveto(242.12890267,373.10545728)(243.37108893,373.44139444)(244.98046875,373.44140789)
\curveto(245.77733652,373.44139444)(246.52733577,373.38280075)(247.23046875,373.26562664)
\curveto(247.93358436,373.14842599)(248.58202121,372.97264491)(249.17578125,372.73828289)
}
}
{
\newrgbcolor{curcolor}{0 0 0}
\pscustom[linestyle=none,fillstyle=solid,fillcolor=curcolor]
{
\newpath
\moveto(214.28515625,302.49609539)
\lineto(214.28515625,307.19531414)
\lineto(210.41796875,307.19531414)
\lineto(210.41796875,309.14062664)
\lineto(216.62890625,309.14062664)
\lineto(216.62890625,301.62890789)
\curveto(215.71482804,300.98046941)(214.70701654,300.4882824)(213.60546875,300.15234539)
\curveto(212.50389375,299.82422056)(211.32811367,299.66015823)(210.078125,299.66015789)
\curveto(207.34374266,299.66015823)(205.2031198,300.45703243)(203.65625,302.05078289)
\curveto(202.11718538,303.65234173)(201.3476549,305.87890201)(201.34765625,308.73047039)
\curveto(201.3476549,311.5898338)(202.11718538,313.81639407)(203.65625,315.41015789)
\curveto(205.2031198,317.01170337)(207.34374266,317.81248382)(210.078125,317.81250164)
\curveto(211.21873878,317.81248382)(212.30076895,317.67185896)(213.32421875,317.39062664)
\curveto(214.35545439,317.10935953)(215.3046722,316.69529744)(216.171875,316.14843914)
\lineto(216.171875,313.62890789)
\curveto(215.2968597,314.37108101)(214.36717313,314.92967421)(213.3828125,315.30468914)
\curveto(212.3984251,315.67967346)(211.36326989,315.86717327)(210.27734375,315.86718914)
\curveto(208.13671061,315.86717327)(206.52733722,315.26951762)(205.44921875,314.07422039)
\curveto(204.37890187,312.87889501)(203.84374616,311.09764679)(203.84375,308.73047039)
\curveto(203.84374616,306.37108901)(204.37890187,304.59374704)(205.44921875,303.39843914)
\curveto(206.52733722,302.20312443)(208.13671061,301.60546878)(210.27734375,301.60547039)
\curveto(211.11327014,301.60546878)(211.85936314,301.67578121)(212.515625,301.81640789)
\curveto(213.17186183,301.96484342)(213.76170499,302.19140569)(214.28515625,302.49609539)
}
}
{
\newrgbcolor{curcolor}{0 0 0}
\pscustom[linestyle=none,fillstyle=solid,fillcolor=curcolor]
{
\newpath
\moveto(232.09765625,307.10156414)
\lineto(232.09765625,306.04687664)
\lineto(222.18359375,306.04687664)
\curveto(222.27734008,304.56249707)(222.72265214,303.42968571)(223.51953125,302.64843914)
\curveto(224.32421304,301.87499976)(225.44139942,301.4882814)(226.87109375,301.48828289)
\curveto(227.69920966,301.4882814)(228.49999011,301.5898438)(229.2734375,301.79297039)
\curveto(230.05467605,301.99609339)(230.82811278,302.30078058)(231.59375,302.70703289)
\lineto(231.59375,300.66797039)
\curveto(230.82030029,300.33984505)(230.02733233,300.0898453)(229.21484375,299.91797039)
\curveto(228.40233396,299.74609564)(227.57811603,299.66015823)(226.7421875,299.66015789)
\curveto(224.64843146,299.66015823)(222.98827687,300.26953262)(221.76171875,301.48828289)
\curveto(220.54296682,302.70703018)(219.93359243,304.35546603)(219.93359375,306.43359539)
\curveto(219.93359243,308.5820243)(220.51171685,310.2851476)(221.66796875,311.54297039)
\curveto(222.83202703,312.80858258)(224.39843171,313.44139444)(226.3671875,313.44140789)
\curveto(228.13280298,313.44139444)(229.52733283,312.87108251)(230.55078125,311.73047039)
\curveto(231.58201828,310.59764729)(232.09764276,309.05468008)(232.09765625,307.10156414)
\moveto(229.94140625,307.73437664)
\curveto(229.92576993,308.91405522)(229.59373902,309.85546053)(228.9453125,310.55859539)
\curveto(228.3046778,311.26170912)(227.45311616,311.61327127)(226.390625,311.61328289)
\curveto(225.18749342,311.61327127)(224.22265064,311.27342786)(223.49609375,310.59375164)
\curveto(222.77733958,309.91405422)(222.3632775,308.95702393)(222.25390625,307.72265789)
\lineto(229.94140625,307.73437664)
}
}
{
\newrgbcolor{curcolor}{0 0 0}
\pscustom[linestyle=none,fillstyle=solid,fillcolor=curcolor]
{
\newpath
\moveto(244.00390625,312.73828289)
\lineto(244.00390625,310.69922039)
\curveto(243.39452123,311.01170937)(242.76170936,311.24608414)(242.10546875,311.40234539)
\curveto(241.44921068,311.55858383)(240.76952386,311.63670875)(240.06640625,311.63672039)
\curveto(238.99608813,311.63670875)(238.19140143,311.47264641)(237.65234375,311.14453289)
\curveto(237.12109,310.81639707)(236.85546527,310.32421006)(236.85546875,309.66797039)
\curveto(236.85546527,309.16796122)(237.04687133,308.77343036)(237.4296875,308.48437664)
\curveto(237.81249556,308.20311843)(238.58202604,307.93358745)(239.73828125,307.67578289)
\lineto(240.4765625,307.51172039)
\curveto(242.00780387,307.1835882)(243.09374028,306.71874492)(243.734375,306.11718914)
\curveto(244.38280149,305.52343361)(244.70701992,304.69140319)(244.70703125,303.62109539)
\curveto(244.70701992,302.40234298)(244.2226454,301.4375002)(243.25390625,300.72656414)
\curveto(242.29295983,300.01562662)(240.96874241,299.66015823)(239.28125,299.66015789)
\curveto(238.5781198,299.66015823)(237.84374553,299.73047066)(237.078125,299.87109539)
\curveto(236.32030955,300.00390788)(235.51952911,300.20703268)(234.67578125,300.48047039)
\lineto(234.67578125,302.70703289)
\curveto(235.47265415,302.29296809)(236.25780962,301.98046841)(237.03125,301.76953289)
\curveto(237.80468307,301.56640632)(238.5703073,301.46484392)(239.328125,301.46484539)
\curveto(240.34374303,301.46484392)(241.12499225,301.63671875)(241.671875,301.98047039)
\curveto(242.21874116,302.33203055)(242.49217838,302.82421756)(242.4921875,303.45703289)
\curveto(242.49217838,304.04296634)(242.29295983,304.49218464)(241.89453125,304.80468914)
\curveto(241.50389812,305.11718402)(240.64061773,305.41796497)(239.3046875,305.70703289)
\lineto(238.5546875,305.88281414)
\curveto(237.21874616,306.16405797)(236.25390337,306.59374504)(235.66015625,307.17187664)
\curveto(235.06640456,307.75780638)(234.76952986,308.55858683)(234.76953125,309.57422039)
\curveto(234.76952986,310.80858458)(235.20702942,311.76170862)(236.08203125,312.43359539)
\curveto(236.95702767,313.10545728)(238.19921393,313.44139444)(239.80859375,313.44140789)
\curveto(240.60546152,313.44139444)(241.35546077,313.38280075)(242.05859375,313.26562664)
\curveto(242.76170936,313.14842599)(243.41014621,312.97264491)(244.00390625,312.73828289)
}
}
{
\newrgbcolor{curcolor}{0 0 0}
\pscustom[linestyle=none,fillstyle=solid,fillcolor=curcolor]
{
\newpath
\moveto(250.28515625,316.85156414)
\lineto(250.28515625,313.12500164)
\lineto(254.7265625,313.12500164)
\lineto(254.7265625,311.44922039)
\lineto(250.28515625,311.44922039)
\lineto(250.28515625,304.32422039)
\curveto(250.28515186,303.25390463)(250.42968296,302.56640532)(250.71875,302.26172039)
\curveto(251.01561987,301.95703093)(251.61327553,301.80468733)(252.51171875,301.80468914)
\lineto(254.7265625,301.80468914)
\lineto(254.7265625,300.00000164)
\lineto(252.51171875,300.00000164)
\curveto(250.84765129,300.00000164)(249.69921494,300.30859508)(249.06640625,300.92578289)
\curveto(248.43359121,301.55078133)(248.11718527,302.6835927)(248.1171875,304.32422039)
\lineto(248.1171875,311.44922039)
\lineto(246.53515625,311.44922039)
\lineto(246.53515625,313.12500164)
\lineto(248.1171875,313.12500164)
\lineto(248.1171875,316.85156414)
\lineto(250.28515625,316.85156414)
}
}
{
\newrgbcolor{curcolor}{0 0 0}
\pscustom[linestyle=none,fillstyle=solid,fillcolor=curcolor]
{
\newpath
\moveto(257.57421875,313.12500164)
\lineto(259.73046875,313.12500164)
\lineto(259.73046875,300.00000164)
\lineto(257.57421875,300.00000164)
\lineto(257.57421875,313.12500164)
\moveto(257.57421875,318.23437664)
\lineto(259.73046875,318.23437664)
\lineto(259.73046875,315.50390789)
\lineto(257.57421875,315.50390789)
\lineto(257.57421875,318.23437664)
}
}
{
\newrgbcolor{curcolor}{0 0 0}
\pscustom[linestyle=none,fillstyle=solid,fillcolor=curcolor]
{
\newpath
\moveto(269.31640625,311.61328289)
\curveto(268.16015006,311.61327127)(267.24608847,311.16014673)(266.57421875,310.25390789)
\curveto(265.90233982,309.35546103)(265.56640265,308.12108726)(265.56640625,306.55078289)
\curveto(265.56640265,304.98046541)(265.89843357,303.74218539)(266.5625,302.83593914)
\curveto(267.23436973,301.9374997)(268.15233757,301.4882814)(269.31640625,301.48828289)
\curveto(270.46483525,301.4882814)(271.37499059,301.94140594)(272.046875,302.84765789)
\curveto(272.71873925,303.75390413)(273.05467641,304.9882779)(273.0546875,306.55078289)
\curveto(273.05467641,308.10546228)(272.71873925,309.3359298)(272.046875,310.24218914)
\curveto(271.37499059,311.15624048)(270.46483525,311.61327127)(269.31640625,311.61328289)
\moveto(269.31640625,313.44140789)
\curveto(271.19139703,313.44139444)(272.6640518,312.83202005)(273.734375,311.61328289)
\curveto(274.80467466,310.39452249)(275.33983038,308.70702418)(275.33984375,306.55078289)
\curveto(275.33983038,304.40234098)(274.80467466,302.71484267)(273.734375,301.48828289)
\curveto(272.6640518,300.26953262)(271.19139703,299.66015823)(269.31640625,299.66015789)
\curveto(267.43358829,299.66015823)(265.95702726,300.26953262)(264.88671875,301.48828289)
\curveto(263.82421689,302.71484267)(263.29296743,304.40234098)(263.29296875,306.55078289)
\curveto(263.29296743,308.70702418)(263.82421689,310.39452249)(264.88671875,311.61328289)
\curveto(265.95702726,312.83202005)(267.43358829,313.44139444)(269.31640625,313.44140789)
}
}
{
\newrgbcolor{curcolor}{0 0 0}
\pscustom[linestyle=none,fillstyle=solid,fillcolor=curcolor]
{
\newpath
\moveto(289.8125,307.92187664)
\lineto(289.8125,300.00000164)
\lineto(287.65625,300.00000164)
\lineto(287.65625,307.85156414)
\curveto(287.65623898,309.09374254)(287.41405173,310.02342911)(286.9296875,310.64062664)
\curveto(286.4453027,311.25780288)(285.71874092,311.56639632)(284.75,311.56640789)
\curveto(283.58593055,311.56639632)(282.66796272,311.19530294)(281.99609375,310.45312664)
\curveto(281.32421407,309.71092942)(280.9882769,308.69921169)(280.98828125,307.41797039)
\lineto(280.98828125,300.00000164)
\lineto(278.8203125,300.00000164)
\lineto(278.8203125,313.12500164)
\lineto(280.98828125,313.12500164)
\lineto(280.98828125,311.08593914)
\curveto(281.50390139,311.87498976)(282.10936953,312.46483292)(282.8046875,312.85547039)
\curveto(283.50780563,313.24608214)(284.31639857,313.44139444)(285.23046875,313.44140789)
\curveto(286.73827115,313.44139444)(287.87889501,312.97264491)(288.65234375,312.03515789)
\curveto(289.42576846,311.10545928)(289.81248683,309.7343669)(289.8125,307.92187664)
}
}
{
\newrgbcolor{curcolor}{0 0 0}
\pscustom[linestyle=none,fillstyle=solid,fillcolor=curcolor]
{
}
}
{
\newrgbcolor{curcolor}{0 0 0}
\pscustom[linestyle=none,fillstyle=solid,fillcolor=curcolor]
{
\newpath
\moveto(303.86328125,301.96875164)
\lineto(303.86328125,295.00781414)
\lineto(301.6953125,295.00781414)
\lineto(301.6953125,313.12500164)
\lineto(303.86328125,313.12500164)
\lineto(303.86328125,311.13281414)
\curveto(304.31640145,311.91405222)(304.88671338,312.49217664)(305.57421875,312.86718914)
\curveto(306.2695245,313.24998839)(307.09764867,313.44139444)(308.05859375,313.44140789)
\curveto(309.65233361,313.44139444)(310.94530107,312.80858258)(311.9375,311.54297039)
\curveto(312.93748658,310.27733511)(313.43748608,308.61327427)(313.4375,306.55078289)
\curveto(313.43748608,304.4882784)(312.93748658,302.82421756)(311.9375,301.55859539)
\curveto(310.94530107,300.29297009)(309.65233361,299.66015823)(308.05859375,299.66015789)
\curveto(307.09764867,299.66015823)(306.2695245,299.84765804)(305.57421875,300.22265789)
\curveto(304.88671338,300.60546978)(304.31640145,301.18750045)(303.86328125,301.96875164)
\moveto(311.19921875,306.55078289)
\curveto(311.19920707,308.13671225)(310.87108239,309.37889851)(310.21484375,310.27734539)
\curveto(309.5663962,311.1835842)(308.67186584,311.63670875)(307.53125,311.63672039)
\curveto(306.39061813,311.63670875)(305.49218152,311.1835842)(304.8359375,310.27734539)
\curveto(304.18749533,309.37889851)(303.8632769,308.13671225)(303.86328125,306.55078289)
\curveto(303.8632769,304.96484042)(304.18749533,303.71874792)(304.8359375,302.81250164)
\curveto(305.49218152,301.91406222)(306.39061813,301.46484392)(307.53125,301.46484539)
\curveto(308.67186584,301.46484392)(309.5663962,301.91406222)(310.21484375,302.81250164)
\curveto(310.87108239,303.71874792)(311.19920707,304.96484042)(311.19921875,306.55078289)
}
}
{
\newrgbcolor{curcolor}{0 0 0}
\pscustom[linestyle=none,fillstyle=solid,fillcolor=curcolor]
{
\newpath
\moveto(324.6171875,311.10937664)
\curveto(324.37499037,311.24999039)(324.10936564,311.35155278)(323.8203125,311.41406414)
\curveto(323.53905371,311.48436515)(323.22655402,311.51952137)(322.8828125,311.51953289)
\curveto(321.66405559,311.51952137)(320.72655652,311.12108426)(320.0703125,310.32422039)
\curveto(319.42187033,309.53514835)(319.0976519,308.39843074)(319.09765625,306.91406414)
\lineto(319.09765625,300.00000164)
\lineto(316.9296875,300.00000164)
\lineto(316.9296875,313.12500164)
\lineto(319.09765625,313.12500164)
\lineto(319.09765625,311.08593914)
\curveto(319.55077645,311.88280225)(320.14061961,312.47264541)(320.8671875,312.85547039)
\curveto(321.59374316,313.24608214)(322.47655477,313.44139444)(323.515625,313.44140789)
\curveto(323.66405359,313.44139444)(323.82811592,313.42967571)(324.0078125,313.40625164)
\curveto(324.18749056,313.39061325)(324.38670911,313.36326952)(324.60546875,313.32422039)
\lineto(324.6171875,311.10937664)
}
}
{
\newrgbcolor{curcolor}{0 0 0}
\pscustom[linestyle=none,fillstyle=solid,fillcolor=curcolor]
{
\newpath
\moveto(331.47265625,311.61328289)
\curveto(330.31640006,311.61327127)(329.40233847,311.16014673)(328.73046875,310.25390789)
\curveto(328.05858982,309.35546103)(327.72265265,308.12108726)(327.72265625,306.55078289)
\curveto(327.72265265,304.98046541)(328.05468357,303.74218539)(328.71875,302.83593914)
\curveto(329.39061973,301.9374997)(330.30858757,301.4882814)(331.47265625,301.48828289)
\curveto(332.62108525,301.4882814)(333.53124059,301.94140594)(334.203125,302.84765789)
\curveto(334.87498925,303.75390413)(335.21092641,304.9882779)(335.2109375,306.55078289)
\curveto(335.21092641,308.10546228)(334.87498925,309.3359298)(334.203125,310.24218914)
\curveto(333.53124059,311.15624048)(332.62108525,311.61327127)(331.47265625,311.61328289)
\moveto(331.47265625,313.44140789)
\curveto(333.34764703,313.44139444)(334.8203018,312.83202005)(335.890625,311.61328289)
\curveto(336.96092466,310.39452249)(337.49608038,308.70702418)(337.49609375,306.55078289)
\curveto(337.49608038,304.40234098)(336.96092466,302.71484267)(335.890625,301.48828289)
\curveto(334.8203018,300.26953262)(333.34764703,299.66015823)(331.47265625,299.66015789)
\curveto(329.58983829,299.66015823)(328.11327726,300.26953262)(327.04296875,301.48828289)
\curveto(325.98046689,302.71484267)(325.44921743,304.40234098)(325.44921875,306.55078289)
\curveto(325.44921743,308.70702418)(325.98046689,310.39452249)(327.04296875,311.61328289)
\curveto(328.11327726,312.83202005)(329.58983829,313.44139444)(331.47265625,313.44140789)
}
}
{
\newrgbcolor{curcolor}{0 0 0}
\pscustom[linestyle=none,fillstyle=solid,fillcolor=curcolor]
{
\newpath
\moveto(351.6640625,313.12500164)
\lineto(351.6640625,300.00000164)
\lineto(349.49609375,300.00000164)
\lineto(349.49609375,311.44922039)
\lineto(343.578125,311.44922039)
\lineto(343.578125,300.00000164)
\lineto(341.41015625,300.00000164)
\lineto(341.41015625,311.44922039)
\lineto(339.34765625,311.44922039)
\lineto(339.34765625,313.12500164)
\lineto(341.41015625,313.12500164)
\lineto(341.41015625,314.03906414)
\curveto(341.41015364,315.46873617)(341.7460908,316.52342261)(342.41796875,317.20312664)
\curveto(343.09765195,317.89060875)(344.13671341,318.2343584)(345.53515625,318.23437664)
\lineto(347.703125,318.23437664)
\lineto(347.703125,316.44140789)
\lineto(345.640625,316.44140789)
\curveto(344.86718143,316.44139144)(344.32811947,316.2851416)(344.0234375,315.97265789)
\curveto(343.72655757,315.66014223)(343.57812022,315.09764279)(343.578125,314.28515789)
\lineto(343.578125,313.12500164)
\lineto(351.6640625,313.12500164)
\moveto(349.49609375,318.21093914)
\lineto(351.6640625,318.21093914)
\lineto(351.6640625,315.48047039)
\lineto(349.49609375,315.48047039)
\lineto(349.49609375,318.21093914)
}
}
{
\newrgbcolor{curcolor}{0 0 0}
\pscustom[linestyle=none,fillstyle=solid,fillcolor=curcolor]
{
\newpath
\moveto(356.19921875,318.23437664)
\lineto(358.35546875,318.23437664)
\lineto(358.35546875,300.00000164)
\lineto(356.19921875,300.00000164)
\lineto(356.19921875,318.23437664)
}
}
{
\newrgbcolor{curcolor}{0 0 0}
\pscustom[linestyle=none,fillstyle=solid,fillcolor=curcolor]
{
\newpath
\moveto(150.65234375,38.20311138)
\curveto(151.16014509,38.03122835)(151.6523321,37.66404121)(152.12890625,37.10154888)
\curveto(152.61326864,36.53904234)(153.09764315,35.76560561)(153.58203125,34.78123638)
\lineto(155.984375,29.99998638)
\lineto(153.44140625,29.99998638)
\lineto(151.203125,34.48826763)
\curveto(150.62498938,35.66013697)(150.06248994,36.43747994)(149.515625,36.82029888)
\curveto(148.97655352,37.20310417)(148.23827301,37.39451023)(147.30078125,37.39451763)
\lineto(144.72265625,37.39451763)
\lineto(144.72265625,29.99998638)
\lineto(142.35546875,29.99998638)
\lineto(142.35546875,47.49608013)
\lineto(147.69921875,47.49608013)
\curveto(149.69920905,47.49606263)(151.19139506,47.0780943)(152.17578125,46.24217388)
\curveto(153.16014309,45.40622097)(153.6523301,44.14450348)(153.65234375,42.45701763)
\curveto(153.6523301,41.35544377)(153.39451786,40.44138219)(152.87890625,39.71483013)
\curveto(152.37108138,38.98825864)(151.62889462,38.48435289)(150.65234375,38.20311138)
\moveto(144.72265625,45.55076763)
\lineto(144.72265625,39.33983013)
\lineto(147.69921875,39.33983013)
\curveto(148.83983491,39.33982079)(149.69920905,39.60153928)(150.27734375,40.12498638)
\curveto(150.86327039,40.65622572)(151.15623884,41.43356869)(151.15625,42.45701763)
\curveto(151.15623884,43.48044165)(150.86327039,44.24997213)(150.27734375,44.76561138)
\curveto(149.69920905,45.28903359)(148.83983491,45.55075208)(147.69921875,45.55076763)
\lineto(144.72265625,45.55076763)
}
}
{
\newrgbcolor{curcolor}{0 0 0}
\pscustom[linestyle=none,fillstyle=solid,fillcolor=curcolor]
{
\newpath
\moveto(169.09765625,37.10154888)
\lineto(169.09765625,36.04686138)
\lineto(159.18359375,36.04686138)
\curveto(159.27734008,34.56248181)(159.72265214,33.42967045)(160.51953125,32.64842388)
\curveto(161.32421304,31.8749845)(162.44139942,31.48826614)(163.87109375,31.48826763)
\curveto(164.69920966,31.48826614)(165.49999011,31.58982854)(166.2734375,31.79295513)
\curveto(167.05467605,31.99607813)(167.82811278,32.30076533)(168.59375,32.70701763)
\lineto(168.59375,30.66795513)
\curveto(167.82030029,30.33982979)(167.02733233,30.08983004)(166.21484375,29.91795513)
\curveto(165.40233396,29.74608038)(164.57811603,29.66014297)(163.7421875,29.66014263)
\curveto(161.64843146,29.66014297)(159.98827687,30.26951736)(158.76171875,31.48826763)
\curveto(157.54296682,32.70701492)(156.93359243,34.35545077)(156.93359375,36.43358013)
\curveto(156.93359243,38.58200904)(157.51171685,40.28513234)(158.66796875,41.54295513)
\curveto(159.83202703,42.80856732)(161.39843171,43.44137919)(163.3671875,43.44139263)
\curveto(165.13280298,43.44137919)(166.52733283,42.87106726)(167.55078125,41.73045513)
\curveto(168.58201828,40.59763203)(169.09764276,39.05466482)(169.09765625,37.10154888)
\moveto(166.94140625,37.73436138)
\curveto(166.92576993,38.91403996)(166.59373902,39.85544527)(165.9453125,40.55858013)
\curveto(165.3046778,41.26169387)(164.45311616,41.61325601)(163.390625,41.61326763)
\curveto(162.18749342,41.61325601)(161.22265064,41.2734126)(160.49609375,40.59373638)
\curveto(159.77733958,39.91403896)(159.3632775,38.95700867)(159.25390625,37.72264263)
\lineto(166.94140625,37.73436138)
}
}
{
\newrgbcolor{curcolor}{0 0 0}
\pscustom[linestyle=none,fillstyle=solid,fillcolor=curcolor]
{
\newpath
\moveto(174.76953125,46.85154888)
\lineto(174.76953125,43.12498638)
\lineto(179.2109375,43.12498638)
\lineto(179.2109375,41.44920513)
\lineto(174.76953125,41.44920513)
\lineto(174.76953125,34.32420513)
\curveto(174.76952686,33.25388937)(174.91405796,32.56639006)(175.203125,32.26170513)
\curveto(175.49999487,31.95701567)(176.09765053,31.80467207)(176.99609375,31.80467388)
\lineto(179.2109375,31.80467388)
\lineto(179.2109375,29.99998638)
\lineto(176.99609375,29.99998638)
\curveto(175.33202629,29.99998638)(174.18358994,30.30857982)(173.55078125,30.92576763)
\curveto(172.91796621,31.55076608)(172.60156027,32.68357744)(172.6015625,34.32420513)
\lineto(172.6015625,41.44920513)
\lineto(171.01953125,41.44920513)
\lineto(171.01953125,43.12498638)
\lineto(172.6015625,43.12498638)
\lineto(172.6015625,46.85154888)
\lineto(174.76953125,46.85154888)
}
}
{
\newrgbcolor{curcolor}{0 0 0}
\pscustom[linestyle=none,fillstyle=solid,fillcolor=curcolor]
{
\newpath
\moveto(187.14453125,41.61326763)
\curveto(185.98827506,41.61325601)(185.07421347,41.16013147)(184.40234375,40.25389263)
\curveto(183.73046482,39.35544577)(183.39452765,38.12107201)(183.39453125,36.55076763)
\curveto(183.39452765,34.98045015)(183.72655857,33.74217013)(184.390625,32.83592388)
\curveto(185.06249473,31.93748444)(185.98046257,31.48826614)(187.14453125,31.48826763)
\curveto(188.29296025,31.48826614)(189.20311559,31.94139069)(189.875,32.84764263)
\curveto(190.54686425,33.75388887)(190.88280141,34.98826264)(190.8828125,36.55076763)
\curveto(190.88280141,38.10544702)(190.54686425,39.33591454)(189.875,40.24217388)
\curveto(189.20311559,41.15622522)(188.29296025,41.61325601)(187.14453125,41.61326763)
\moveto(187.14453125,43.44139263)
\curveto(189.01952203,43.44137919)(190.4921768,42.83200479)(191.5625,41.61326763)
\curveto(192.63279966,40.39450723)(193.16795538,38.70700892)(193.16796875,36.55076763)
\curveto(193.16795538,34.40232572)(192.63279966,32.71482741)(191.5625,31.48826763)
\curveto(190.4921768,30.26951736)(189.01952203,29.66014297)(187.14453125,29.66014263)
\curveto(185.26171329,29.66014297)(183.78515226,30.26951736)(182.71484375,31.48826763)
\curveto(181.65234189,32.71482741)(181.12109243,34.40232572)(181.12109375,36.55076763)
\curveto(181.12109243,38.70700892)(181.65234189,40.39450723)(182.71484375,41.61326763)
\curveto(183.78515226,42.83200479)(185.26171329,43.44137919)(187.14453125,43.44139263)
}
}
{
\newrgbcolor{curcolor}{0 0 0}
\pscustom[linestyle=none,fillstyle=solid,fillcolor=curcolor]
{
\newpath
\moveto(196.5078125,35.17967388)
\lineto(196.5078125,43.12498638)
\lineto(198.6640625,43.12498638)
\lineto(198.6640625,35.26170513)
\curveto(198.6640583,34.01951361)(198.90624556,33.08592079)(199.390625,32.46092388)
\curveto(199.87499459,31.84373453)(200.60155637,31.53514109)(201.5703125,31.53514263)
\curveto(202.73436673,31.53514109)(203.65233457,31.90623447)(204.32421875,32.64842388)
\curveto(205.00389571,33.39060799)(205.34373913,34.40232572)(205.34375,35.68358013)
\lineto(205.34375,43.12498638)
\lineto(207.5,43.12498638)
\lineto(207.5,29.99998638)
\lineto(205.34375,29.99998638)
\lineto(205.34375,32.01561138)
\curveto(204.82030215,31.21873516)(204.21092776,30.62498575)(203.515625,30.23436138)
\curveto(202.82811664,29.85154903)(202.02733619,29.66014297)(201.11328125,29.66014263)
\curveto(199.60546361,29.66014297)(198.46093351,30.1288925)(197.6796875,31.06639263)
\curveto(196.89843507,32.00389062)(196.50781046,33.374983)(196.5078125,35.17967388)
\moveto(201.93359375,43.44139263)
\lineto(201.93359375,43.44139263)
}
}
{
\newrgbcolor{curcolor}{0 0 0}
\pscustom[linestyle=none,fillstyle=solid,fillcolor=curcolor]
{
\newpath
\moveto(219.5703125,41.10936138)
\curveto(219.32811538,41.24997513)(219.06249064,41.35153753)(218.7734375,41.41404888)
\curveto(218.49217871,41.48434989)(218.17967902,41.51950611)(217.8359375,41.51951763)
\curveto(216.61718059,41.51950611)(215.67968152,41.12106901)(215.0234375,40.32420513)
\curveto(214.37499533,39.53513309)(214.0507769,38.39841548)(214.05078125,36.91404888)
\lineto(214.05078125,29.99998638)
\lineto(211.8828125,29.99998638)
\lineto(211.8828125,43.12498638)
\lineto(214.05078125,43.12498638)
\lineto(214.05078125,41.08592388)
\curveto(214.50390145,41.88278699)(215.09374461,42.47263015)(215.8203125,42.85545513)
\curveto(216.54686816,43.24606688)(217.42967977,43.44137919)(218.46875,43.44139263)
\curveto(218.61717859,43.44137919)(218.78124092,43.42966045)(218.9609375,43.40623638)
\curveto(219.14061556,43.39059799)(219.33983411,43.36325426)(219.55859375,43.32420513)
\lineto(219.5703125,41.10936138)
}
}
{
\newrgbcolor{curcolor}{0 0 0}
\pscustom[linestyle=none,fillstyle=solid,fillcolor=curcolor]
{
\newpath
\moveto(43.96875,532.65625164)
\lineto(49.125,532.65625164)
\lineto(49.125,550.45312664)
\lineto(43.515625,549.32812664)
\lineto(43.515625,552.20312664)
\lineto(49.09375,553.32812664)
\lineto(52.25,553.32812664)
\lineto(52.25,532.65625164)
\lineto(57.40625,532.65625164)
\lineto(57.40625,530.00000164)
\lineto(43.96875,530.00000164)
\lineto(43.96875,532.65625164)
}
}
{
\newrgbcolor{curcolor}{0 0 0}
\pscustom[linestyle=none,fillstyle=solid,fillcolor=curcolor]
{
\newpath
\moveto(506.140625,502.65625164)
\lineto(517.15625,502.65625164)
\lineto(517.15625,500.00000164)
\lineto(502.34375,500.00000164)
\lineto(502.34375,502.65625164)
\curveto(503.54166313,503.89583107)(505.17186983,505.55728775)(507.234375,507.64062664)
\curveto(509.30728236,509.7343669)(510.60936439,511.08332389)(511.140625,511.68750164)
\curveto(512.15102952,512.82290548)(512.85415381,513.78123785)(513.25,514.56250164)
\curveto(513.65623634,515.35415295)(513.85936114,516.13019384)(513.859375,516.89062664)
\curveto(513.85936114,518.13019184)(513.42186158,519.1406075)(512.546875,519.92187664)
\curveto(511.68227998,520.70310593)(510.55207278,521.09373054)(509.15625,521.09375164)
\curveto(508.1666585,521.09373054)(507.11978455,520.92185571)(506.015625,520.57812664)
\curveto(504.92187008,520.2343564)(503.74999625,519.71352359)(502.5,519.01562664)
\lineto(502.5,522.20312664)
\curveto(503.77082956,522.71352059)(504.95832838,523.09893687)(506.0625,523.35937664)
\curveto(507.1666595,523.61976968)(508.17707516,523.74997789)(509.09375,523.75000164)
\curveto(511.51040516,523.74997789)(513.43748656,523.14581182)(514.875,521.93750164)
\curveto(516.31248369,520.72914757)(517.03123297,519.11456585)(517.03125,517.09375164)
\curveto(517.03123297,516.13540217)(516.84894148,515.22394475)(516.484375,514.35937664)
\curveto(516.1301922,513.50519646)(515.47915119,512.49478081)(514.53125,511.32812664)
\curveto(514.27081906,511.02603228)(513.44269489,510.15103315)(512.046875,508.70312664)
\curveto(510.65103102,507.26561937)(508.68228298,505.24999639)(506.140625,502.65625164)
}
}
{
\newrgbcolor{curcolor}{0 0 0}
\pscustom[linestyle=none,fillstyle=solid,fillcolor=curcolor]
{
\newpath
\moveto(512.984375,422.57812664)
\curveto(514.49477717,422.25519771)(515.67185933,421.58332339)(516.515625,420.56250164)
\curveto(517.3697743,419.54165876)(517.7968572,418.28124335)(517.796875,416.78125164)
\curveto(517.7968572,414.47916382)(517.00519133,412.6979156)(515.421875,411.43750164)
\curveto(513.83852783,410.17708479)(511.58853008,409.54687709)(508.671875,409.54687664)
\curveto(507.69270064,409.54687709)(506.68228498,409.64583532)(505.640625,409.84375164)
\curveto(504.60937039,410.0312516)(503.54166313,410.31770965)(502.4375,410.70312664)
\lineto(502.4375,413.75000164)
\curveto(503.31249669,413.23958173)(504.27082906,412.85416545)(505.3125,412.59375164)
\curveto(506.35416031,412.33333264)(507.44270089,412.20312443)(508.578125,412.20312664)
\curveto(510.55728111,412.20312443)(512.06248794,412.59374904)(513.09375,413.37500164)
\curveto(514.13540253,414.15624748)(514.65623534,415.29166301)(514.65625,416.78125164)
\curveto(514.65623534,418.15624348)(514.17186083,419.22915907)(513.203125,420.00000164)
\curveto(512.24477942,420.78124085)(510.90623909,421.17186546)(509.1875,421.17187664)
\lineto(506.46875,421.17187664)
\lineto(506.46875,423.76562664)
\lineto(509.3125,423.76562664)
\curveto(510.86457247,423.76561287)(512.05207128,424.07290423)(512.875,424.68750164)
\curveto(513.69790297,425.31248632)(514.10936089,426.20831876)(514.109375,427.37500164)
\curveto(514.10936089,428.57289973)(513.68227798,429.48956548)(512.828125,430.12500164)
\curveto(511.98436302,430.7708142)(510.77082256,431.09373054)(509.1875,431.09375164)
\curveto(508.32290834,431.09373054)(507.39582594,430.99998064)(506.40625,430.81250164)
\curveto(505.41666125,430.62498101)(504.32812067,430.33331464)(503.140625,429.93750164)
\lineto(503.140625,432.75000164)
\curveto(504.33853733,433.08331189)(505.45832788,433.33331164)(506.5,433.50000164)
\curveto(507.55207578,433.66664464)(508.54165813,433.74997789)(509.46875,433.75000164)
\curveto(511.86457147,433.74997789)(513.76040291,433.20310343)(515.15625,432.10937664)
\curveto(516.55206678,431.02602228)(517.24998275,429.55727375)(517.25,427.70312664)
\curveto(517.24998275,426.41144356)(516.88019145,425.31769465)(516.140625,424.42187664)
\curveto(515.40102627,423.53644643)(514.34894398,422.92186371)(512.984375,422.57812664)
}
}
{
\newrgbcolor{curcolor}{0 0 0}
\pscustom[linestyle=none,fillstyle=solid,fillcolor=curcolor]
{
\newpath
\moveto(512.09375,370.57812664)
\lineto(504.125,358.12500164)
\lineto(512.09375,358.12500164)
\lineto(512.09375,370.57812664)
\moveto(511.265625,373.32812664)
\lineto(515.234375,373.32812664)
\lineto(515.234375,358.12500164)
\lineto(518.5625,358.12500164)
\lineto(518.5625,355.50000164)
\lineto(515.234375,355.50000164)
\lineto(515.234375,350.00000164)
\lineto(512.09375,350.00000164)
\lineto(512.09375,355.50000164)
\lineto(501.5625,355.50000164)
\lineto(501.5625,358.54687664)
\lineto(511.265625,373.32812664)
}
}
{
\newrgbcolor{curcolor}{0 0 0}
\pscustom[linestyle=none,fillstyle=solid,fillcolor=curcolor]
{
\newpath
\moveto(503.453125,323.32812664)
\lineto(515.84375,323.32812664)
\lineto(515.84375,320.67187664)
\lineto(506.34375,320.67187664)
\lineto(506.34375,314.95312664)
\curveto(506.80207653,315.10936153)(507.26040941,315.22394475)(507.71875,315.29687664)
\curveto(508.17707516,315.38019459)(508.63540803,315.42186121)(509.09375,315.42187664)
\curveto(511.69790497,315.42186121)(513.76040291,314.70832026)(515.28125,313.28125164)
\curveto(516.80206653,311.85415645)(517.56248244,309.92186671)(517.5625,307.48437664)
\curveto(517.56248244,304.973955)(516.78123322,303.02083195)(515.21875,301.62500164)
\curveto(513.65623634,300.23958473)(511.45311355,299.54687709)(508.609375,299.54687664)
\curveto(507.6302007,299.54687709)(506.6302017,299.63021034)(505.609375,299.79687664)
\curveto(504.59895373,299.96354334)(503.55207978,300.21354309)(502.46875,300.54687664)
\lineto(502.46875,303.71875164)
\curveto(503.40624659,303.20833176)(504.37499562,302.82812381)(505.375,302.57812664)
\curveto(506.37499363,302.32812431)(507.43228423,302.20312443)(508.546875,302.20312664)
\curveto(510.34894798,302.20312443)(511.77602989,302.67708229)(512.828125,303.62500164)
\curveto(513.88019445,304.57291373)(514.40623559,305.85937078)(514.40625,307.48437664)
\curveto(514.40623559,309.10936753)(513.88019445,310.39582457)(512.828125,311.34375164)
\curveto(511.77602989,312.29165601)(510.34894798,312.76561387)(508.546875,312.76562664)
\curveto(507.7031173,312.76561387)(506.85936814,312.67186396)(506.015625,312.48437664)
\curveto(505.18228648,312.29686434)(504.32812067,312.00519796)(503.453125,311.60937664)
\lineto(503.453125,323.32812664)
}
}
{
\newrgbcolor{curcolor}{0 0 0}
\pscustom[linestyle=none,fillstyle=solid,fillcolor=curcolor]
{
\newpath
\moveto(60.5625,42.92186138)
\curveto(59.14582419,42.92184846)(58.02082531,42.43747394)(57.1875,41.46873638)
\curveto(56.36457697,40.49997588)(55.95311905,39.17185221)(55.953125,37.48436138)
\curveto(55.95311905,35.80727224)(56.36457697,34.47914856)(57.1875,33.49998638)
\curveto(58.02082531,32.53123385)(59.14582419,32.04685933)(60.5625,32.04686138)
\curveto(61.97915469,32.04685933)(63.09894523,32.53123385)(63.921875,33.49998638)
\curveto(64.75519358,34.47914856)(65.17185983,35.80727224)(65.171875,37.48436138)
\curveto(65.17185983,39.17185221)(64.75519358,40.49997588)(63.921875,41.46873638)
\curveto(63.09894523,42.43747394)(61.97915469,42.92184846)(60.5625,42.92186138)
\moveto(66.828125,52.81248638)
\lineto(66.828125,49.93748638)
\curveto(66.0364423,50.31246606)(65.23435977,50.59892411)(64.421875,50.79686138)
\curveto(63.61977805,50.99475705)(62.82290384,51.09371528)(62.03125,51.09373638)
\curveto(59.94790672,51.09371528)(58.35415831,50.39059099)(57.25,48.98436138)
\curveto(56.15624384,47.5780938)(55.53124447,45.45309592)(55.375,42.60936138)
\curveto(55.98957734,43.51559786)(56.76040991,44.2083055)(57.6875,44.68748638)
\curveto(58.61457472,45.17705453)(59.63540703,45.42184596)(60.75,45.42186138)
\curveto(63.09373691,45.42184596)(64.94269339,44.708305)(66.296875,43.28123638)
\curveto(67.66144067,41.86455785)(68.34373166,39.93226811)(68.34375,37.48436138)
\curveto(68.34373166,35.08852296)(67.63539903,33.16664988)(66.21875,31.71873638)
\curveto(64.80206853,30.27081944)(62.91665375,29.54686183)(60.5625,29.54686138)
\curveto(57.86457547,29.54686183)(55.80207753,30.5781108)(54.375,32.64061138)
\curveto(52.94791372,34.71352333)(52.23437277,37.71352033)(52.234375,41.64061138)
\curveto(52.23437277,45.32809605)(53.10937189,48.26559311)(54.859375,50.45311138)
\curveto(56.60936839,52.65100539)(58.95832438,53.74996263)(61.90625,53.74998638)
\curveto(62.69790397,53.74996263)(63.49477817,53.67183771)(64.296875,53.51561138)
\curveto(65.10935989,53.35933802)(65.95310905,53.12496325)(66.828125,52.81248638)
}
}
{
\newrgbcolor{curcolor}{0 0 0}
\pscustom[linewidth=2,linecolor=curcolor,linestyle=dashed,dash=8 8]
{
\newpath
\moveto(150,540)
\lineto(60,540)
}
}
{
\newrgbcolor{curcolor}{0 0 0}
\pscustom[linestyle=none,fillstyle=solid,fillcolor=curcolor]
{
\newpath
\moveto(139.53769464,544.84048224)
\lineto(152.6487474,540.01921591)
\lineto(139.53769392,535.19795064)
\curveto(141.632292,538.04442372)(141.62022288,541.93889292)(139.53769464,544.84048224)
\lineto(139.53769464,544.84048224)
\lineto(139.53769464,544.84048224)
\closepath
}
}
{
\newrgbcolor{curcolor}{0 0 0}
\pscustom[linewidth=2,linecolor=curcolor,linestyle=dashed,dash=8 8]
{
\newpath
\moveto(420,420)
\lineto(500,420)
}
}
{
\newrgbcolor{curcolor}{0 0 0}
\pscustom[linestyle=none,fillstyle=solid,fillcolor=curcolor]
{
\newpath
\moveto(430.46230536,415.15951776)
\lineto(417.3512526,419.98078409)
\lineto(430.46230608,424.80204936)
\curveto(428.367708,421.95557628)(428.37977712,418.06110708)(430.46230536,415.15951776)
\lineto(430.46230536,415.15951776)
\lineto(430.46230536,415.15951776)
\closepath
}
}
{
\newrgbcolor{curcolor}{0 0 0}
\pscustom[linewidth=2,linecolor=curcolor,linestyle=dashed,dash=8 8]
{
\newpath
\moveto(400,360)
\lineto(500,360)
}
}
{
\newrgbcolor{curcolor}{0 0 0}
\pscustom[linestyle=none,fillstyle=solid,fillcolor=curcolor]
{
\newpath
\moveto(410.46230536,355.15951776)
\lineto(397.3512526,359.98078409)
\lineto(410.46230608,364.80204936)
\curveto(408.367708,361.95557628)(408.37977712,358.06110708)(410.46230536,355.15951776)
\lineto(410.46230536,355.15951776)
\lineto(410.46230536,355.15951776)
\closepath
}
}
{
\newrgbcolor{curcolor}{0 0 0}
\pscustom[linewidth=2,linecolor=curcolor,linestyle=dashed,dash=8 8]
{
\newpath
\moveto(359.96667,310.41667)
\lineto(499.96667,310.41667)
}
}
{
\newrgbcolor{curcolor}{0 0 0}
\pscustom[linestyle=none,fillstyle=solid,fillcolor=curcolor]
{
\newpath
\moveto(370.42897536,305.57618776)
\lineto(357.3179226,310.39745409)
\lineto(370.42897608,315.21871936)
\curveto(368.334378,312.37224628)(368.34644712,308.47777708)(370.42897536,305.57618776)
\lineto(370.42897536,305.57618776)
\lineto(370.42897536,305.57618776)
\closepath
}
}
{
\newrgbcolor{curcolor}{0 0 0}
\pscustom[linewidth=2,linecolor=curcolor,linestyle=dashed,dash=8 8]
{
\newpath
\moveto(140.07143,40)
\lineto(60.071429,40)
}
}
{
\newrgbcolor{curcolor}{0 0 0}
\pscustom[linewidth=2,linecolor=curcolor,linestyle=dashed,dash=8 8]
{
\newpath
\moveto(320,420)
\lineto(500,510)
}
}
{
\newrgbcolor{curcolor}{0 0 0}
\pscustom[linestyle=none,fillstyle=solid,fillcolor=curcolor]
{
\newpath
\moveto(331.52249986,420.34942626)
\lineto(317.63948192,418.79825692)
\lineto(327.21022928,428.97396904)
\curveto(326.60974526,425.49127338)(328.36219978,422.01335171)(331.52249986,420.34942626)
\lineto(331.52249986,420.34942626)
\lineto(331.52249986,420.34942626)
\closepath
}
}
\end{pspicture}

		
		\begin{enumerate}
		  \item Fond d'écran
		  \item Barre de volume
		  \item Check box ``Muet''
		  \item Liste déroulante ``Langues''
		  \item Bouton ``\hyperlink{Page d'accueil}{Gestion du profil}''
		  \item Bouton ``\hyperlink{Page d'accueil}{Retour}''
		\end{enumerate}

		\subsubsection{Description des zones}
		
			\begin{tabular}{|c|c|c|c|c|} \hline
				Numéro de zone & Type  & Description & Evènement &	Règle \\\hline
			\end{tabular}
			
		\subsubsection{Description des règles}

			\underline{RG7-01 :}
				\begin{quote}
				
				\end{quote}
	
\newpage

	\subsection{Gestion du profil}
	
		\hypertarget{Gestion du profil}{}
		\label{Gestion du profil}
		
		%LaTeX with PSTricks extensions
%%Creator: inkscape 0.48.0
%%Please note this file requires PSTricks extensions
\psset{xunit=.5pt,yunit=.5pt,runit=.5pt}
\begin{pspicture}(560,600)
{
\newrgbcolor{curcolor}{1 1 1}
\pscustom[linestyle=none,fillstyle=solid,fillcolor=curcolor]
{
\newpath
\moveto(133.12401581,597.52220317)
\lineto(426.87598419,597.52220317)
\curveto(443.85397169,597.52220317)(457.52217102,583.85400385)(457.52217102,566.87601635)
\lineto(457.52217102,33.12401744)
\curveto(457.52217102,16.14602994)(443.85397169,2.47783062)(426.87598419,2.47783062)
\lineto(133.12401581,2.47783062)
\curveto(116.14602831,2.47783062)(102.47782898,16.14602994)(102.47782898,33.12401744)
\lineto(102.47782898,566.87601635)
\curveto(102.47782898,583.85400385)(116.14602831,597.52220317)(133.12401581,597.52220317)
\closepath
}
}
{
\newrgbcolor{curcolor}{0 0 0}
\pscustom[linewidth=4.95566034,linecolor=curcolor]
{
\newpath
\moveto(133.12401581,597.52220317)
\lineto(426.87598419,597.52220317)
\curveto(443.85397169,597.52220317)(457.52217102,583.85400385)(457.52217102,566.87601635)
\lineto(457.52217102,33.12401744)
\curveto(457.52217102,16.14602994)(443.85397169,2.47783062)(426.87598419,2.47783062)
\lineto(133.12401581,2.47783062)
\curveto(116.14602831,2.47783062)(102.47782898,16.14602994)(102.47782898,33.12401744)
\lineto(102.47782898,566.87601635)
\curveto(102.47782898,583.85400385)(116.14602831,597.52220317)(133.12401581,597.52220317)
\closepath
}
}
{
\newrgbcolor{curcolor}{1 1 1}
\pscustom[linestyle=none,fillstyle=solid,fillcolor=curcolor]
{
\newpath
\moveto(130.63699913,470.02933857)
\lineto(429.34404945,470.02933857)
\curveto(435.2532028,470.02933857)(440.01039124,465.27215013)(440.01039124,459.36299678)
\lineto(440.01039124,380.75620433)
\curveto(440.01039124,374.84705099)(435.2532028,370.08986255)(429.34404945,370.08986255)
\lineto(130.63699913,370.08986255)
\curveto(124.72784578,370.08986255)(119.97065735,374.84705099)(119.97065735,380.75620433)
\lineto(119.97065735,459.36299678)
\curveto(119.97065735,465.27215013)(124.72784578,470.02933857)(130.63699913,470.02933857)
\closepath
}
}
{
\newrgbcolor{curcolor}{0 0 0}
\pscustom[linewidth=2,linecolor=curcolor]
{
\newpath
\moveto(130.63699913,470.02933857)
\lineto(429.34404945,470.02933857)
\curveto(435.2532028,470.02933857)(440.01039124,465.27215013)(440.01039124,459.36299678)
\lineto(440.01039124,380.75620433)
\curveto(440.01039124,374.84705099)(435.2532028,370.08986255)(429.34404945,370.08986255)
\lineto(130.63699913,370.08986255)
\curveto(124.72784578,370.08986255)(119.97065735,374.84705099)(119.97065735,380.75620433)
\lineto(119.97065735,459.36299678)
\curveto(119.97065735,465.27215013)(124.72784578,470.02933857)(130.63699913,470.02933857)
\closepath
}
}
{
\newrgbcolor{curcolor}{0 0 0}
\pscustom[linestyle=none,fillstyle=solid,fillcolor=curcolor]
{
\newpath
\moveto(144.72265625,455.55078098)
\lineto(144.72265625,448.97656223)
\lineto(147.69921875,448.97656223)
\curveto(148.80077245,448.97655325)(149.6523341,449.26170922)(150.25390625,449.83203098)
\curveto(150.85545789,450.40233308)(151.15623884,451.21483226)(151.15625,452.26953098)
\curveto(151.15623884,453.31639266)(150.85545789,454.1249856)(150.25390625,454.69531223)
\curveto(149.6523341,455.26560946)(148.80077245,455.55076543)(147.69921875,455.55078098)
\lineto(144.72265625,455.55078098)
\moveto(142.35546875,457.49609348)
\lineto(147.69921875,457.49609348)
\curveto(149.66014659,457.49607598)(151.14061386,457.05076393)(152.140625,456.16015598)
\curveto(153.14842435,455.2773282)(153.6523301,453.9804545)(153.65234375,452.26953098)
\curveto(153.6523301,450.54295794)(153.14842435,449.23827174)(152.140625,448.35546848)
\curveto(151.14061386,447.47264851)(149.66014659,447.0312427)(147.69921875,447.03124973)
\lineto(144.72265625,447.03124973)
\lineto(144.72265625,439.99999973)
\lineto(142.35546875,439.99999973)
\lineto(142.35546875,457.49609348)
}
}
{
\newrgbcolor{curcolor}{0 0 0}
\pscustom[linestyle=none,fillstyle=solid,fillcolor=curcolor]
{
\newpath
\moveto(164.69140625,452.73828098)
\lineto(164.69140625,450.69921848)
\curveto(164.08202123,451.01170747)(163.44920936,451.24608223)(162.79296875,451.40234348)
\curveto(162.13671068,451.55858192)(161.45702386,451.63670684)(160.75390625,451.63671848)
\curveto(159.68358813,451.63670684)(158.87890143,451.47264451)(158.33984375,451.14453098)
\curveto(157.80859,450.81639516)(157.54296527,450.32420815)(157.54296875,449.66796848)
\curveto(157.54296527,449.16795931)(157.73437133,448.77342845)(158.1171875,448.48437473)
\curveto(158.49999556,448.20311653)(159.26952604,447.93358554)(160.42578125,447.67578098)
\lineto(161.1640625,447.51171848)
\curveto(162.69530387,447.18358629)(163.78124028,446.71874301)(164.421875,446.11718723)
\curveto(165.07030149,445.5234317)(165.39451992,444.69140129)(165.39453125,443.62109348)
\curveto(165.39451992,442.40234108)(164.9101454,441.43749829)(163.94140625,440.72656223)
\curveto(162.98045983,440.01562471)(161.65624241,439.66015632)(159.96875,439.66015598)
\curveto(159.2656198,439.66015632)(158.53124553,439.73046875)(157.765625,439.87109348)
\curveto(157.00780955,440.00390597)(156.20702911,440.20703077)(155.36328125,440.48046848)
\lineto(155.36328125,442.70703098)
\curveto(156.16015415,442.29296619)(156.94530962,441.9804665)(157.71875,441.76953098)
\curveto(158.49218307,441.56640441)(159.2578073,441.46484201)(160.015625,441.46484348)
\curveto(161.03124303,441.46484201)(161.81249225,441.63671684)(162.359375,441.98046848)
\curveto(162.90624116,442.33202865)(163.17967838,442.82421565)(163.1796875,443.45703098)
\curveto(163.17967838,444.04296444)(162.98045983,444.49218274)(162.58203125,444.80468723)
\curveto(162.19139812,445.11718211)(161.32811773,445.41796306)(159.9921875,445.70703098)
\lineto(159.2421875,445.88281223)
\curveto(157.90624616,446.16405606)(156.94140337,446.59374313)(156.34765625,447.17187473)
\curveto(155.75390456,447.75780447)(155.45702986,448.55858492)(155.45703125,449.57421848)
\curveto(155.45702986,450.80858267)(155.89452942,451.76170672)(156.76953125,452.43359348)
\curveto(157.64452767,453.10545537)(158.88671393,453.44139254)(160.49609375,453.44140598)
\curveto(161.29296152,453.44139254)(162.04296077,453.38279885)(162.74609375,453.26562473)
\curveto(163.44920936,453.14842408)(164.09764621,452.97264301)(164.69140625,452.73828098)
}
}
{
\newrgbcolor{curcolor}{0 0 0}
\pscustom[linestyle=none,fillstyle=solid,fillcolor=curcolor]
{
\newpath
\moveto(180.06640625,447.10156223)
\lineto(180.06640625,446.04687473)
\lineto(170.15234375,446.04687473)
\curveto(170.24609008,444.56249517)(170.69140214,443.4296838)(171.48828125,442.64843723)
\curveto(172.29296304,441.87499785)(173.41014942,441.48827949)(174.83984375,441.48828098)
\curveto(175.66795966,441.48827949)(176.46874011,441.58984189)(177.2421875,441.79296848)
\curveto(178.02342605,441.99609148)(178.79686278,442.30077868)(179.5625,442.70703098)
\lineto(179.5625,440.66796848)
\curveto(178.78905029,440.33984314)(177.99608233,440.08984339)(177.18359375,439.91796848)
\curveto(176.37108396,439.74609373)(175.54686603,439.66015632)(174.7109375,439.66015598)
\curveto(172.61718146,439.66015632)(170.95702687,440.26953071)(169.73046875,441.48828098)
\curveto(168.51171682,442.70702827)(167.90234243,444.35546412)(167.90234375,446.43359348)
\curveto(167.90234243,448.5820224)(168.48046685,450.28514569)(169.63671875,451.54296848)
\curveto(170.80077703,452.80858067)(172.36718171,453.44139254)(174.3359375,453.44140598)
\curveto(176.10155298,453.44139254)(177.49608283,452.87108061)(178.51953125,451.73046848)
\curveto(179.55076828,450.59764538)(180.06639276,449.05467817)(180.06640625,447.10156223)
\moveto(177.91015625,447.73437473)
\curveto(177.89451993,448.91405331)(177.56248902,449.85545862)(176.9140625,450.55859348)
\curveto(176.2734278,451.26170722)(175.42186616,451.61326937)(174.359375,451.61328098)
\curveto(173.15624342,451.61326937)(172.19140064,451.27342595)(171.46484375,450.59374973)
\curveto(170.74608958,449.91405231)(170.3320275,448.95702202)(170.22265625,447.72265598)
\lineto(177.91015625,447.73437473)
}
}
{
\newrgbcolor{curcolor}{0 0 0}
\pscustom[linestyle=none,fillstyle=solid,fillcolor=curcolor]
{
\newpath
\moveto(183.3828125,445.17968723)
\lineto(183.3828125,453.12499973)
\lineto(185.5390625,453.12499973)
\lineto(185.5390625,445.26171848)
\curveto(185.5390583,444.01952696)(185.78124556,443.08593414)(186.265625,442.46093723)
\curveto(186.74999459,441.84374788)(187.47655637,441.53515444)(188.4453125,441.53515598)
\curveto(189.60936673,441.53515444)(190.52733457,441.90624782)(191.19921875,442.64843723)
\curveto(191.87889571,443.39062134)(192.21873913,444.40233908)(192.21875,445.68359348)
\lineto(192.21875,453.12499973)
\lineto(194.375,453.12499973)
\lineto(194.375,439.99999973)
\lineto(192.21875,439.99999973)
\lineto(192.21875,442.01562473)
\curveto(191.69530215,441.21874851)(191.08592776,440.6249991)(190.390625,440.23437473)
\curveto(189.70311664,439.85156238)(188.90233619,439.66015632)(187.98828125,439.66015598)
\curveto(186.48046361,439.66015632)(185.33593351,440.12890585)(184.5546875,441.06640598)
\curveto(183.77343507,442.00390397)(183.38281046,443.37499635)(183.3828125,445.17968723)
\moveto(188.80859375,453.44140598)
\lineto(188.80859375,453.44140598)
}
}
{
\newrgbcolor{curcolor}{0 0 0}
\pscustom[linestyle=none,fillstyle=solid,fillcolor=curcolor]
{
\newpath
\moveto(207.4765625,451.13281223)
\lineto(207.4765625,458.23437473)
\lineto(209.6328125,458.23437473)
\lineto(209.6328125,439.99999973)
\lineto(207.4765625,439.99999973)
\lineto(207.4765625,441.96874973)
\curveto(207.02342705,441.18749854)(206.44920888,440.60546787)(205.75390625,440.22265598)
\curveto(205.06639776,439.84765613)(204.23827359,439.66015632)(203.26953125,439.66015598)
\curveto(201.68358864,439.66015632)(200.39062119,440.29296819)(199.390625,441.55859348)
\curveto(198.39843568,442.82421565)(197.90234243,444.48827649)(197.90234375,446.55078098)
\curveto(197.90234243,448.61327237)(198.39843568,450.2773332)(199.390625,451.54296848)
\curveto(200.39062119,452.80858067)(201.68358864,453.44139254)(203.26953125,453.44140598)
\curveto(204.23827359,453.44139254)(205.06639776,453.24998648)(205.75390625,452.86718723)
\curveto(206.44920888,452.49217474)(207.02342705,451.91405031)(207.4765625,451.13281223)
\moveto(200.12890625,446.55078098)
\curveto(200.1289027,444.96483851)(200.45312112,443.71874601)(201.1015625,442.81249973)
\curveto(201.75780732,441.91406031)(202.65624392,441.46484201)(203.796875,441.46484348)
\curveto(204.93749164,441.46484201)(205.83592824,441.91406031)(206.4921875,442.81249973)
\curveto(207.14842693,443.71874601)(207.4765516,444.96483851)(207.4765625,446.55078098)
\curveto(207.4765516,448.13671034)(207.14842693,449.3788966)(206.4921875,450.27734348)
\curveto(205.83592824,451.18358229)(204.93749164,451.63670684)(203.796875,451.63671848)
\curveto(202.65624392,451.63670684)(201.75780732,451.18358229)(201.1015625,450.27734348)
\curveto(200.45312112,449.3788966)(200.1289027,448.13671034)(200.12890625,446.55078098)
}
}
{
\newrgbcolor{curcolor}{0 0 0}
\pscustom[linestyle=none,fillstyle=solid,fillcolor=curcolor]
{
\newpath
\moveto(219.16015625,451.61328098)
\curveto(218.00390006,451.61326937)(217.08983847,451.16014482)(216.41796875,450.25390598)
\curveto(215.74608982,449.35545912)(215.41015265,448.12108536)(215.41015625,446.55078098)
\curveto(215.41015265,444.9804635)(215.74218357,443.74218349)(216.40625,442.83593723)
\curveto(217.07811973,441.93749779)(217.99608757,441.48827949)(219.16015625,441.48828098)
\curveto(220.30858525,441.48827949)(221.21874059,441.94140404)(221.890625,442.84765598)
\curveto(222.56248925,443.75390222)(222.89842641,444.98827599)(222.8984375,446.55078098)
\curveto(222.89842641,448.10546037)(222.56248925,449.33592789)(221.890625,450.24218723)
\curveto(221.21874059,451.15623857)(220.30858525,451.61326937)(219.16015625,451.61328098)
\moveto(219.16015625,453.44140598)
\curveto(221.03514703,453.44139254)(222.5078018,452.83201815)(223.578125,451.61328098)
\curveto(224.64842466,450.39452058)(225.18358038,448.70702227)(225.18359375,446.55078098)
\curveto(225.18358038,444.40233908)(224.64842466,442.71484076)(223.578125,441.48828098)
\curveto(222.5078018,440.26953071)(221.03514703,439.66015632)(219.16015625,439.66015598)
\curveto(217.27733829,439.66015632)(215.80077726,440.26953071)(214.73046875,441.48828098)
\curveto(213.66796689,442.71484076)(213.13671743,444.40233908)(213.13671875,446.55078098)
\curveto(213.13671743,448.70702227)(213.66796689,450.39452058)(214.73046875,451.61328098)
\curveto(215.80077726,452.83201815)(217.27733829,453.44139254)(219.16015625,453.44140598)
}
}
{
\newrgbcolor{curcolor}{1 1 1}
\pscustom[linestyle=none,fillstyle=solid,fillcolor=curcolor]
{
\newpath
\moveto(137.12685585,309.93502971)
\lineto(422.87261009,309.93502971)
\curveto(432.32488956,309.93502971)(439.93448639,302.32543288)(439.93448639,292.87315341)
\lineto(439.93448639,167.13374683)
\curveto(439.93448639,157.68146736)(432.32488956,150.07187053)(422.87261009,150.07187053)
\lineto(137.12685585,150.07187053)
\curveto(127.67457638,150.07187053)(120.06497955,157.68146736)(120.06497955,167.13374683)
\lineto(120.06497955,292.87315341)
\curveto(120.06497955,302.32543288)(127.67457638,309.93502971)(137.12685585,309.93502971)
\closepath
}
}
{
\newrgbcolor{curcolor}{0 0 0}
\pscustom[linewidth=2,linecolor=curcolor]
{
\newpath
\moveto(137.12685585,309.93502971)
\lineto(422.87261009,309.93502971)
\curveto(432.32488956,309.93502971)(439.93448639,302.32543288)(439.93448639,292.87315341)
\lineto(439.93448639,167.13374683)
\curveto(439.93448639,157.68146736)(432.32488956,150.07187053)(422.87261009,150.07187053)
\lineto(137.12685585,150.07187053)
\curveto(127.67457638,150.07187053)(120.06497955,157.68146736)(120.06497955,167.13374683)
\lineto(120.06497955,292.87315341)
\curveto(120.06497955,302.32543288)(127.67457638,309.93502971)(137.12685585,309.93502971)
\closepath
}
}
{
\newrgbcolor{curcolor}{0 0 0}
\pscustom[linestyle=none,fillstyle=solid,fillcolor=curcolor]
{
}
}
{
\newrgbcolor{curcolor}{0 0 0}
\pscustom[linestyle=none,fillstyle=solid,fillcolor=curcolor]
{
}
}
{
\newrgbcolor{curcolor}{0 0 0}
\pscustom[linestyle=none,fillstyle=solid,fillcolor=curcolor]
{
}
}
{
\newrgbcolor{curcolor}{0 0 0}
\pscustom[linestyle=none,fillstyle=solid,fillcolor=curcolor]
{
}
}
{
\newrgbcolor{curcolor}{0 0 0}
\pscustom[linestyle=none,fillstyle=solid,fillcolor=curcolor]
{
}
}
{
\newrgbcolor{curcolor}{0 0 0}
\pscustom[linestyle=none,fillstyle=solid,fillcolor=curcolor]
{
}
}
{
\newrgbcolor{curcolor}{0 0 0}
\pscustom[linestyle=none,fillstyle=solid,fillcolor=curcolor]
{
\newpath
\moveto(178.19921875,297.49609539)
\lineto(181.38671875,297.49609539)
\lineto(189.14453125,282.85937664)
\lineto(189.14453125,297.49609539)
\lineto(191.44140625,297.49609539)
\lineto(191.44140625,280.00000164)
\lineto(188.25390625,280.00000164)
\lineto(180.49609375,294.63672039)
\lineto(180.49609375,280.00000164)
\lineto(178.19921875,280.00000164)
\lineto(178.19921875,297.49609539)
}
}
{
\newrgbcolor{curcolor}{0 0 0}
\pscustom[linestyle=none,fillstyle=solid,fillcolor=curcolor]
{
\newpath
\moveto(201.14453125,291.61328289)
\curveto(199.98827506,291.61327127)(199.07421347,291.16014673)(198.40234375,290.25390789)
\curveto(197.73046482,289.35546103)(197.39452765,288.12108726)(197.39453125,286.55078289)
\curveto(197.39452765,284.98046541)(197.72655857,283.74218539)(198.390625,282.83593914)
\curveto(199.06249473,281.9374997)(199.98046257,281.4882814)(201.14453125,281.48828289)
\curveto(202.29296025,281.4882814)(203.20311559,281.94140594)(203.875,282.84765789)
\curveto(204.54686425,283.75390413)(204.88280141,284.9882779)(204.8828125,286.55078289)
\curveto(204.88280141,288.10546228)(204.54686425,289.3359298)(203.875,290.24218914)
\curveto(203.20311559,291.15624048)(202.29296025,291.61327127)(201.14453125,291.61328289)
\moveto(201.14453125,293.44140789)
\curveto(203.01952203,293.44139444)(204.4921768,292.83202005)(205.5625,291.61328289)
\curveto(206.63279966,290.39452249)(207.16795538,288.70702418)(207.16796875,286.55078289)
\curveto(207.16795538,284.40234098)(206.63279966,282.71484267)(205.5625,281.48828289)
\curveto(204.4921768,280.26953262)(203.01952203,279.66015823)(201.14453125,279.66015789)
\curveto(199.26171329,279.66015823)(197.78515226,280.26953262)(196.71484375,281.48828289)
\curveto(195.65234189,282.71484267)(195.12109243,284.40234098)(195.12109375,286.55078289)
\curveto(195.12109243,288.70702418)(195.65234189,290.39452249)(196.71484375,291.61328289)
\curveto(197.78515226,292.83202005)(199.26171329,293.44139444)(201.14453125,293.44140789)
}
}
{
\newrgbcolor{curcolor}{0 0 0}
\pscustom[linestyle=none,fillstyle=solid,fillcolor=curcolor]
{
\newpath
\moveto(220.94921875,290.60547039)
\curveto(221.48826823,291.57420881)(222.13279884,292.28905185)(222.8828125,292.75000164)
\curveto(223.63279734,293.21092592)(224.51560895,293.44139444)(225.53125,293.44140789)
\curveto(226.89841907,293.44139444)(227.95310552,292.96092617)(228.6953125,292.00000164)
\curveto(229.43747903,291.04686559)(229.80857241,289.68749195)(229.80859375,287.92187664)
\lineto(229.80859375,280.00000164)
\lineto(227.640625,280.00000164)
\lineto(227.640625,287.85156414)
\curveto(227.64060583,289.10936753)(227.4179498,290.04296034)(226.97265625,290.65234539)
\curveto(226.52732569,291.26170912)(225.84763887,291.56639632)(224.93359375,291.56640789)
\curveto(223.8163909,291.56639632)(222.93357929,291.19530294)(222.28515625,290.45312664)
\curveto(221.63670558,289.71092942)(221.31248716,288.69921169)(221.3125,287.41797039)
\lineto(221.3125,280.00000164)
\lineto(219.14453125,280.00000164)
\lineto(219.14453125,287.85156414)
\curveto(219.14452057,289.11718002)(218.92186455,290.05077283)(218.4765625,290.65234539)
\curveto(218.03124044,291.26170912)(217.34374113,291.56639632)(216.4140625,291.56640789)
\curveto(215.31249316,291.56639632)(214.43749403,291.19139669)(213.7890625,290.44140789)
\curveto(213.14062033,289.69921069)(212.8164019,288.69139919)(212.81640625,287.41797039)
\lineto(212.81640625,280.00000164)
\lineto(210.6484375,280.00000164)
\lineto(210.6484375,293.12500164)
\lineto(212.81640625,293.12500164)
\lineto(212.81640625,291.08593914)
\curveto(213.30858891,291.89061475)(213.89843207,292.48436415)(214.5859375,292.86718914)
\curveto(215.2734307,293.24998839)(216.08983613,293.44139444)(217.03515625,293.44140789)
\curveto(217.98827173,293.44139444)(218.79686467,293.19920719)(219.4609375,292.71484539)
\curveto(220.13280084,292.23045816)(220.62889409,291.52733386)(220.94921875,290.60547039)
}
}
{
\newrgbcolor{curcolor}{0 0 0}
\pscustom[linestyle=none,fillstyle=solid,fillcolor=curcolor]
{
}
}
{
\newrgbcolor{curcolor}{0 0 0}
\pscustom[linestyle=none,fillstyle=solid,fillcolor=curcolor]
{
\newpath
\moveto(140.8984375,261.13281414)
\lineto(140.8984375,268.23437664)
\lineto(143.0546875,268.23437664)
\lineto(143.0546875,250.00000164)
\lineto(140.8984375,250.00000164)
\lineto(140.8984375,251.96875164)
\curveto(140.44530205,251.18750045)(139.87108388,250.60546978)(139.17578125,250.22265789)
\curveto(138.48827276,249.84765804)(137.66014859,249.66015823)(136.69140625,249.66015789)
\curveto(135.10546364,249.66015823)(133.81249619,250.29297009)(132.8125,251.55859539)
\curveto(131.82031068,252.82421756)(131.32421743,254.4882784)(131.32421875,256.55078289)
\curveto(131.32421743,258.61327427)(131.82031068,260.27733511)(132.8125,261.54297039)
\curveto(133.81249619,262.80858258)(135.10546364,263.44139444)(136.69140625,263.44140789)
\curveto(137.66014859,263.44139444)(138.48827276,263.24998839)(139.17578125,262.86718914)
\curveto(139.87108388,262.49217664)(140.44530205,261.91405222)(140.8984375,261.13281414)
\moveto(133.55078125,256.55078289)
\curveto(133.5507777,254.96484042)(133.87499612,253.71874792)(134.5234375,252.81250164)
\curveto(135.17968232,251.91406222)(136.07811892,251.46484392)(137.21875,251.46484539)
\curveto(138.35936664,251.46484392)(139.25780324,251.91406222)(139.9140625,252.81250164)
\curveto(140.57030193,253.71874792)(140.8984266,254.96484042)(140.8984375,256.55078289)
\curveto(140.8984266,258.13671225)(140.57030193,259.37889851)(139.9140625,260.27734539)
\curveto(139.25780324,261.1835842)(138.35936664,261.63670875)(137.21875,261.63672039)
\curveto(136.07811892,261.63670875)(135.17968232,261.1835842)(134.5234375,260.27734539)
\curveto(133.87499612,259.37889851)(133.5507777,258.13671225)(133.55078125,256.55078289)
}
}
{
\newrgbcolor{curcolor}{0 0 0}
\pscustom[linestyle=none,fillstyle=solid,fillcolor=curcolor]
{
\newpath
\moveto(149.53515625,267.49609539)
\lineto(149.53515625,260.99218914)
\lineto(147.54296875,260.99218914)
\lineto(147.54296875,267.49609539)
\lineto(149.53515625,267.49609539)
}
}
{
\newrgbcolor{curcolor}{0 0 0}
\pscustom[linestyle=none,fillstyle=solid,fillcolor=curcolor]
{
\newpath
\moveto(153.8828125,255.17968914)
\lineto(153.8828125,263.12500164)
\lineto(156.0390625,263.12500164)
\lineto(156.0390625,255.26172039)
\curveto(156.0390583,254.01952887)(156.28124556,253.08593605)(156.765625,252.46093914)
\curveto(157.24999459,251.84374979)(157.97655637,251.53515635)(158.9453125,251.53515789)
\curveto(160.10936673,251.53515635)(161.02733457,251.90624973)(161.69921875,252.64843914)
\curveto(162.37889571,253.39062325)(162.71873913,254.40234098)(162.71875,255.68359539)
\lineto(162.71875,263.12500164)
\lineto(164.875,263.12500164)
\lineto(164.875,250.00000164)
\lineto(162.71875,250.00000164)
\lineto(162.71875,252.01562664)
\curveto(162.19530215,251.21875042)(161.58592776,250.62500101)(160.890625,250.23437664)
\curveto(160.20311664,249.85156428)(159.40233619,249.66015823)(158.48828125,249.66015789)
\curveto(156.98046361,249.66015823)(155.83593351,250.12890776)(155.0546875,251.06640789)
\curveto(154.27343507,252.00390588)(153.88281046,253.37499826)(153.8828125,255.17968914)
\moveto(159.30859375,263.44140789)
\lineto(159.30859375,263.44140789)
}
}
{
\newrgbcolor{curcolor}{0 0 0}
\pscustom[linestyle=none,fillstyle=solid,fillcolor=curcolor]
{
\newpath
\moveto(171.47265625,266.85156414)
\lineto(171.47265625,263.12500164)
\lineto(175.9140625,263.12500164)
\lineto(175.9140625,261.44922039)
\lineto(171.47265625,261.44922039)
\lineto(171.47265625,254.32422039)
\curveto(171.47265186,253.25390463)(171.61718296,252.56640532)(171.90625,252.26172039)
\curveto(172.20311987,251.95703093)(172.80077553,251.80468733)(173.69921875,251.80468914)
\lineto(175.9140625,251.80468914)
\lineto(175.9140625,250.00000164)
\lineto(173.69921875,250.00000164)
\curveto(172.03515129,250.00000164)(170.88671494,250.30859508)(170.25390625,250.92578289)
\curveto(169.62109121,251.55078133)(169.30468527,252.6835927)(169.3046875,254.32422039)
\lineto(169.3046875,261.44922039)
\lineto(167.72265625,261.44922039)
\lineto(167.72265625,263.12500164)
\lineto(169.3046875,263.12500164)
\lineto(169.3046875,266.85156414)
\lineto(171.47265625,266.85156414)
}
}
{
\newrgbcolor{curcolor}{0 0 0}
\pscustom[linestyle=none,fillstyle=solid,fillcolor=curcolor]
{
\newpath
\moveto(178.76171875,263.12500164)
\lineto(180.91796875,263.12500164)
\lineto(180.91796875,250.00000164)
\lineto(178.76171875,250.00000164)
\lineto(178.76171875,263.12500164)
\moveto(178.76171875,268.23437664)
\lineto(180.91796875,268.23437664)
\lineto(180.91796875,265.50390789)
\lineto(178.76171875,265.50390789)
\lineto(178.76171875,268.23437664)
}
}
{
\newrgbcolor{curcolor}{0 0 0}
\pscustom[linestyle=none,fillstyle=solid,fillcolor=curcolor]
{
\newpath
\moveto(185.41796875,268.23437664)
\lineto(187.57421875,268.23437664)
\lineto(187.57421875,250.00000164)
\lineto(185.41796875,250.00000164)
\lineto(185.41796875,268.23437664)
}
}
{
\newrgbcolor{curcolor}{0 0 0}
\pscustom[linestyle=none,fillstyle=solid,fillcolor=curcolor]
{
\newpath
\moveto(192.07421875,263.12500164)
\lineto(194.23046875,263.12500164)
\lineto(194.23046875,250.00000164)
\lineto(192.07421875,250.00000164)
\lineto(192.07421875,263.12500164)
\moveto(192.07421875,268.23437664)
\lineto(194.23046875,268.23437664)
\lineto(194.23046875,265.50390789)
\lineto(192.07421875,265.50390789)
\lineto(192.07421875,268.23437664)
}
}
{
\newrgbcolor{curcolor}{0 0 0}
\pscustom[linestyle=none,fillstyle=solid,fillcolor=curcolor]
{
\newpath
\moveto(207.09765625,262.73828289)
\lineto(207.09765625,260.69922039)
\curveto(206.48827123,261.01170937)(205.85545936,261.24608414)(205.19921875,261.40234539)
\curveto(204.54296068,261.55858383)(203.86327386,261.63670875)(203.16015625,261.63672039)
\curveto(202.08983813,261.63670875)(201.28515143,261.47264641)(200.74609375,261.14453289)
\curveto(200.21484,260.81639707)(199.94921527,260.32421006)(199.94921875,259.66797039)
\curveto(199.94921527,259.16796122)(200.14062133,258.77343036)(200.5234375,258.48437664)
\curveto(200.90624556,258.20311843)(201.67577604,257.93358745)(202.83203125,257.67578289)
\lineto(203.5703125,257.51172039)
\curveto(205.10155387,257.1835882)(206.18749028,256.71874492)(206.828125,256.11718914)
\curveto(207.47655149,255.52343361)(207.80076992,254.69140319)(207.80078125,253.62109539)
\curveto(207.80076992,252.40234298)(207.3163954,251.4375002)(206.34765625,250.72656414)
\curveto(205.38670983,250.01562662)(204.06249241,249.66015823)(202.375,249.66015789)
\curveto(201.6718698,249.66015823)(200.93749553,249.73047066)(200.171875,249.87109539)
\curveto(199.41405955,250.00390788)(198.61327911,250.20703268)(197.76953125,250.48047039)
\lineto(197.76953125,252.70703289)
\curveto(198.56640415,252.29296809)(199.35155962,251.98046841)(200.125,251.76953289)
\curveto(200.89843307,251.56640632)(201.6640573,251.46484392)(202.421875,251.46484539)
\curveto(203.43749303,251.46484392)(204.21874225,251.63671875)(204.765625,251.98047039)
\curveto(205.31249116,252.33203055)(205.58592838,252.82421756)(205.5859375,253.45703289)
\curveto(205.58592838,254.04296634)(205.38670983,254.49218464)(204.98828125,254.80468914)
\curveto(204.59764812,255.11718402)(203.73436773,255.41796497)(202.3984375,255.70703289)
\lineto(201.6484375,255.88281414)
\curveto(200.31249616,256.16405797)(199.34765337,256.59374504)(198.75390625,257.17187664)
\curveto(198.16015456,257.75780638)(197.86327986,258.55858683)(197.86328125,259.57422039)
\curveto(197.86327986,260.80858458)(198.30077942,261.76170862)(199.17578125,262.43359539)
\curveto(200.05077767,263.10545728)(201.29296393,263.44139444)(202.90234375,263.44140789)
\curveto(203.69921152,263.44139444)(204.44921077,263.38280075)(205.15234375,263.26562664)
\curveto(205.85545936,263.14842599)(206.50389621,262.97264491)(207.09765625,262.73828289)
}
}
{
\newrgbcolor{curcolor}{0 0 0}
\pscustom[linestyle=none,fillstyle=solid,fillcolor=curcolor]
{
\newpath
\moveto(217.2109375,256.59765789)
\curveto(215.46874352,256.59765129)(214.26171347,256.39843274)(213.58984375,256.00000164)
\curveto(212.91796482,255.60155853)(212.58202765,254.92187171)(212.58203125,253.96093914)
\curveto(212.58202765,253.19531094)(212.8320274,252.58593655)(213.33203125,252.13281414)
\curveto(213.83983889,251.68749995)(214.52733821,251.46484392)(215.39453125,251.46484539)
\curveto(216.58983614,251.46484392)(217.54686644,251.8867185)(218.265625,252.73047039)
\curveto(218.99217749,253.5820293)(219.35545838,254.71093442)(219.35546875,256.11718914)
\lineto(219.35546875,256.59765789)
\lineto(217.2109375,256.59765789)
\moveto(221.51171875,257.48828289)
\lineto(221.51171875,250.00000164)
\lineto(219.35546875,250.00000164)
\lineto(219.35546875,251.99218914)
\curveto(218.86327137,251.19531294)(218.24999073,250.60546978)(217.515625,250.22265789)
\curveto(216.7812422,249.84765804)(215.8828056,249.66015823)(214.8203125,249.66015789)
\curveto(213.47655801,249.66015823)(212.40624658,250.03515785)(211.609375,250.78515789)
\curveto(210.82031066,251.54296884)(210.42577981,252.55468658)(210.42578125,253.82031414)
\curveto(210.42577981,255.29687134)(210.91796682,256.41015148)(211.90234375,257.16015789)
\curveto(212.89452734,257.91014998)(214.37108836,258.2851496)(216.33203125,258.28515789)
\lineto(219.35546875,258.28515789)
\lineto(219.35546875,258.49609539)
\curveto(219.35545838,259.4882734)(219.02733371,260.25389763)(218.37109375,260.79297039)
\curveto(217.72264751,261.33983405)(216.80858593,261.61327127)(215.62890625,261.61328289)
\curveto(214.87890036,261.61327127)(214.14843234,261.52342761)(213.4375,261.34375164)
\curveto(212.72655876,261.16405297)(212.04296569,260.89452199)(211.38671875,260.53515789)
\lineto(211.38671875,262.52734539)
\curveto(212.17577806,262.83202005)(212.94140229,263.05858233)(213.68359375,263.20703289)
\curveto(214.42577581,263.36326952)(215.14843134,263.44139444)(215.8515625,263.44140789)
\curveto(217.74999123,263.44139444)(219.16795857,262.94920744)(220.10546875,261.96484539)
\curveto(221.04295669,260.98045941)(221.51170622,259.4882734)(221.51171875,257.48828289)
}
}
{
\newrgbcolor{curcolor}{0 0 0}
\pscustom[linestyle=none,fillstyle=solid,fillcolor=curcolor]
{
\newpath
\moveto(228.09765625,266.85156414)
\lineto(228.09765625,263.12500164)
\lineto(232.5390625,263.12500164)
\lineto(232.5390625,261.44922039)
\lineto(228.09765625,261.44922039)
\lineto(228.09765625,254.32422039)
\curveto(228.09765186,253.25390463)(228.24218296,252.56640532)(228.53125,252.26172039)
\curveto(228.82811987,251.95703093)(229.42577553,251.80468733)(230.32421875,251.80468914)
\lineto(232.5390625,251.80468914)
\lineto(232.5390625,250.00000164)
\lineto(230.32421875,250.00000164)
\curveto(228.66015129,250.00000164)(227.51171494,250.30859508)(226.87890625,250.92578289)
\curveto(226.24609121,251.55078133)(225.92968527,252.6835927)(225.9296875,254.32422039)
\lineto(225.9296875,261.44922039)
\lineto(224.34765625,261.44922039)
\lineto(224.34765625,263.12500164)
\lineto(225.9296875,263.12500164)
\lineto(225.9296875,266.85156414)
\lineto(228.09765625,266.85156414)
}
}
{
\newrgbcolor{curcolor}{0 0 0}
\pscustom[linestyle=none,fillstyle=solid,fillcolor=curcolor]
{
\newpath
\moveto(246.61328125,257.10156414)
\lineto(246.61328125,256.04687664)
\lineto(236.69921875,256.04687664)
\curveto(236.79296508,254.56249707)(237.23827714,253.42968571)(238.03515625,252.64843914)
\curveto(238.83983804,251.87499976)(239.95702442,251.4882814)(241.38671875,251.48828289)
\curveto(242.21483466,251.4882814)(243.01561511,251.5898438)(243.7890625,251.79297039)
\curveto(244.57030105,251.99609339)(245.34373778,252.30078058)(246.109375,252.70703289)
\lineto(246.109375,250.66797039)
\curveto(245.33592529,250.33984505)(244.54295733,250.0898453)(243.73046875,249.91797039)
\curveto(242.91795896,249.74609564)(242.09374103,249.66015823)(241.2578125,249.66015789)
\curveto(239.16405646,249.66015823)(237.50390187,250.26953262)(236.27734375,251.48828289)
\curveto(235.05859182,252.70703018)(234.44921743,254.35546603)(234.44921875,256.43359539)
\curveto(234.44921743,258.5820243)(235.02734185,260.2851476)(236.18359375,261.54297039)
\curveto(237.34765203,262.80858258)(238.91405671,263.44139444)(240.8828125,263.44140789)
\curveto(242.64842798,263.44139444)(244.04295783,262.87108251)(245.06640625,261.73047039)
\curveto(246.09764328,260.59764729)(246.61326776,259.05468008)(246.61328125,257.10156414)
\moveto(244.45703125,257.73437664)
\curveto(244.44139493,258.91405522)(244.10936402,259.85546053)(243.4609375,260.55859539)
\curveto(242.8203028,261.26170912)(241.96874116,261.61327127)(240.90625,261.61328289)
\curveto(239.70311842,261.61327127)(238.73827564,261.27342786)(238.01171875,260.59375164)
\curveto(237.29296458,259.91405422)(236.8789025,258.95702393)(236.76953125,257.72265789)
\lineto(244.45703125,257.73437664)
}
}
{
\newrgbcolor{curcolor}{0 0 0}
\pscustom[linestyle=none,fillstyle=solid,fillcolor=curcolor]
{
\newpath
\moveto(249.9296875,255.17968914)
\lineto(249.9296875,263.12500164)
\lineto(252.0859375,263.12500164)
\lineto(252.0859375,255.26172039)
\curveto(252.0859333,254.01952887)(252.32812056,253.08593605)(252.8125,252.46093914)
\curveto(253.29686959,251.84374979)(254.02343137,251.53515635)(254.9921875,251.53515789)
\curveto(256.15624173,251.53515635)(257.07420957,251.90624973)(257.74609375,252.64843914)
\curveto(258.42577071,253.39062325)(258.76561413,254.40234098)(258.765625,255.68359539)
\lineto(258.765625,263.12500164)
\lineto(260.921875,263.12500164)
\lineto(260.921875,250.00000164)
\lineto(258.765625,250.00000164)
\lineto(258.765625,252.01562664)
\curveto(258.24217715,251.21875042)(257.63280276,250.62500101)(256.9375,250.23437664)
\curveto(256.24999164,249.85156428)(255.44921119,249.66015823)(254.53515625,249.66015789)
\curveto(253.02733861,249.66015823)(251.88280851,250.12890776)(251.1015625,251.06640789)
\curveto(250.32031007,252.00390588)(249.92968546,253.37499826)(249.9296875,255.17968914)
\moveto(255.35546875,263.44140789)
\lineto(255.35546875,263.44140789)
}
}
{
\newrgbcolor{curcolor}{0 0 0}
\pscustom[linestyle=none,fillstyle=solid,fillcolor=curcolor]
{
\newpath
\moveto(272.9921875,261.10937664)
\curveto(272.74999037,261.24999039)(272.48436564,261.35155278)(272.1953125,261.41406414)
\curveto(271.91405371,261.48436515)(271.60155402,261.51952137)(271.2578125,261.51953289)
\curveto(270.03905559,261.51952137)(269.10155652,261.12108426)(268.4453125,260.32422039)
\curveto(267.79687033,259.53514835)(267.4726519,258.39843074)(267.47265625,256.91406414)
\lineto(267.47265625,250.00000164)
\lineto(265.3046875,250.00000164)
\lineto(265.3046875,263.12500164)
\lineto(267.47265625,263.12500164)
\lineto(267.47265625,261.08593914)
\curveto(267.92577645,261.88280225)(268.51561961,262.47264541)(269.2421875,262.85547039)
\curveto(269.96874316,263.24608214)(270.85155477,263.44139444)(271.890625,263.44140789)
\curveto(272.03905359,263.44139444)(272.20311592,263.42967571)(272.3828125,263.40625164)
\curveto(272.56249056,263.39061325)(272.76170911,263.36326952)(272.98046875,263.32422039)
\lineto(272.9921875,261.10937664)
}
}
{
\newrgbcolor{curcolor}{0 0 0}
\pscustom[linestyle=none,fillstyle=solid,fillcolor=curcolor]
{
\newpath
\moveto(132.74804688,330.41211101)
\lineto(136.86328125,330.41211101)
\lineto(142.07226562,316.52148601)
\lineto(147.30859375,330.41211101)
\lineto(151.42382812,330.41211101)
\lineto(151.42382812,310.00000164)
\lineto(148.73046875,310.00000164)
\lineto(148.73046875,327.92382976)
\lineto(143.46679688,313.92382976)
\lineto(140.69140625,313.92382976)
\lineto(135.42773438,327.92382976)
\lineto(135.42773438,310.00000164)
\lineto(132.74804688,310.00000164)
\lineto(132.74804688,330.41211101)
}
}
{
\newrgbcolor{curcolor}{0 0 0}
\pscustom[linestyle=none,fillstyle=solid,fillcolor=curcolor]
{
\newpath
\moveto(156.55078125,316.04297039)
\lineto(156.55078125,325.31250164)
\lineto(159.06640625,325.31250164)
\lineto(159.06640625,316.13867351)
\curveto(159.06640136,314.68945007)(159.34895316,313.60025845)(159.9140625,312.87109539)
\curveto(160.47916036,312.15104115)(161.32681576,311.79101547)(162.45703125,311.79101726)
\curveto(163.81509452,311.79101547)(164.88605699,312.22395775)(165.66992188,313.08984539)
\curveto(166.46287833,313.95572685)(166.85936231,315.13606421)(166.859375,316.63086101)
\lineto(166.859375,325.31250164)
\lineto(169.375,325.31250164)
\lineto(169.375,310.00000164)
\lineto(166.859375,310.00000164)
\lineto(166.859375,312.35156414)
\curveto(166.24868584,311.42187521)(165.53774905,310.72916757)(164.7265625,310.27343914)
\curveto(163.92446941,309.82682473)(162.99022556,309.60351766)(161.92382812,309.60351726)
\curveto(160.16470755,309.60351766)(158.82942243,310.15039211)(157.91796875,311.24414226)
\curveto(157.00650758,312.33788992)(156.55077887,313.9374977)(156.55078125,316.04297039)
\moveto(162.88085938,325.68164226)
\lineto(162.88085938,325.68164226)
}
}
{
\newrgbcolor{curcolor}{0 0 0}
\pscustom[linestyle=none,fillstyle=solid,fillcolor=curcolor]
{
\newpath
\moveto(174.58398438,331.27343914)
\lineto(177.09960938,331.27343914)
\lineto(177.09960938,310.00000164)
\lineto(174.58398438,310.00000164)
\lineto(174.58398438,331.27343914)
}
}
{
\newrgbcolor{curcolor}{0 0 0}
\pscustom[linestyle=none,fillstyle=solid,fillcolor=curcolor]
{
\newpath
\moveto(184.83789062,329.66015789)
\lineto(184.83789062,325.31250164)
\lineto(190.01953125,325.31250164)
\lineto(190.01953125,323.35742351)
\lineto(184.83789062,323.35742351)
\lineto(184.83789062,315.04492351)
\curveto(184.8378855,313.7962218)(185.00650512,312.99413927)(185.34375,312.63867351)
\curveto(185.69009819,312.28320248)(186.38736312,312.10546828)(187.43554688,312.10547039)
\lineto(190.01953125,312.10547039)
\lineto(190.01953125,310.00000164)
\lineto(187.43554688,310.00000164)
\curveto(185.49413484,310.00000164)(184.15429243,310.36002732)(183.41601562,311.08007976)
\curveto(182.67773141,311.80924462)(182.30859115,313.13085788)(182.30859375,315.04492351)
\lineto(182.30859375,323.35742351)
\lineto(180.46289062,323.35742351)
\lineto(180.46289062,325.31250164)
\lineto(182.30859375,325.31250164)
\lineto(182.30859375,329.66015789)
\lineto(184.83789062,329.66015789)
}
}
{
\newrgbcolor{curcolor}{0 0 0}
\pscustom[linestyle=none,fillstyle=solid,fillcolor=curcolor]
{
\newpath
\moveto(193.34179688,325.31250164)
\lineto(195.85742188,325.31250164)
\lineto(195.85742188,310.00000164)
\lineto(193.34179688,310.00000164)
\lineto(193.34179688,325.31250164)
\moveto(193.34179688,331.27343914)
\lineto(195.85742188,331.27343914)
\lineto(195.85742188,328.08789226)
\lineto(193.34179688,328.08789226)
\lineto(193.34179688,331.27343914)
}
}
{
\newrgbcolor{curcolor}{0 0 0}
\pscustom[linestyle=none,fillstyle=solid,fillcolor=curcolor]
{
\newpath
\moveto(201.10742188,325.31250164)
\lineto(203.62304688,325.31250164)
\lineto(203.62304688,309.72656414)
\curveto(203.62304172,307.77604553)(203.24934418,306.36328652)(202.50195312,305.48828289)
\curveto(201.76366858,304.61328827)(200.56965936,304.17578871)(198.91992188,304.17578289)
\lineto(197.96289062,304.17578289)
\lineto(197.96289062,306.30859539)
\lineto(198.6328125,306.30859539)
\curveto(199.58984263,306.30859908)(200.24153469,306.53190615)(200.58789062,306.97851726)
\curveto(200.93424233,307.41601984)(201.10741924,308.33203455)(201.10742188,309.72656414)
\lineto(201.10742188,325.31250164)
\moveto(201.10742188,331.27343914)
\lineto(203.62304688,331.27343914)
\lineto(203.62304688,328.08789226)
\lineto(201.10742188,328.08789226)
\lineto(201.10742188,331.27343914)
}
}
{
\newrgbcolor{curcolor}{0 0 0}
\pscustom[linestyle=none,fillstyle=solid,fillcolor=curcolor]
{
\newpath
\moveto(214.80664062,323.54882976)
\curveto(213.45767507,323.54881621)(212.39126988,323.02017091)(211.60742188,321.96289226)
\curveto(210.82356312,320.91470426)(210.43163643,319.47460154)(210.43164062,317.64257976)
\curveto(210.43163643,315.8105427)(210.81900583,314.36588269)(211.59375,313.30859539)
\curveto(212.37759802,312.26041604)(213.44856049,311.73632802)(214.80664062,311.73632976)
\curveto(216.14647446,311.73632802)(217.20832236,312.26497333)(217.9921875,313.32226726)
\curveto(218.77602913,314.37955455)(219.16795582,315.81965727)(219.16796875,317.64257976)
\curveto(219.16795582,319.45637239)(218.77602913,320.89191783)(217.9921875,321.94922039)
\curveto(217.20832236,323.01561362)(216.14647446,323.54881621)(214.80664062,323.54882976)
\moveto(214.80664062,325.68164226)
\curveto(216.99412987,325.68162658)(218.71222711,324.97068979)(219.9609375,323.54882976)
\curveto(221.20962044,322.12694263)(221.83396878,320.1581946)(221.83398438,317.64257976)
\curveto(221.83396878,315.13606421)(221.20962044,313.16731618)(219.9609375,311.73632976)
\curveto(218.71222711,310.31445445)(216.99412987,309.60351766)(214.80664062,309.60351726)
\curveto(212.61001967,309.60351766)(210.88736514,310.31445445)(209.63867188,311.73632976)
\curveto(208.39908638,313.16731618)(207.77929533,315.13606421)(207.77929688,317.64257976)
\curveto(207.77929533,320.1581946)(208.39908638,322.12694263)(209.63867188,323.54882976)
\curveto(210.88736514,324.97068979)(212.61001967,325.68162658)(214.80664062,325.68164226)
}
}
{
\newrgbcolor{curcolor}{0 0 0}
\pscustom[linestyle=none,fillstyle=solid,fillcolor=curcolor]
{
\newpath
\moveto(225.73046875,316.04297039)
\lineto(225.73046875,325.31250164)
\lineto(228.24609375,325.31250164)
\lineto(228.24609375,316.13867351)
\curveto(228.24608886,314.68945007)(228.52864066,313.60025845)(229.09375,312.87109539)
\curveto(229.65884786,312.15104115)(230.50650326,311.79101547)(231.63671875,311.79101726)
\curveto(232.99478202,311.79101547)(234.06574449,312.22395775)(234.84960938,313.08984539)
\curveto(235.64256583,313.95572685)(236.03904981,315.13606421)(236.0390625,316.63086101)
\lineto(236.0390625,325.31250164)
\lineto(238.5546875,325.31250164)
\lineto(238.5546875,310.00000164)
\lineto(236.0390625,310.00000164)
\lineto(236.0390625,312.35156414)
\curveto(235.42837334,311.42187521)(234.71743655,310.72916757)(233.90625,310.27343914)
\curveto(233.10415691,309.82682473)(232.16991306,309.60351766)(231.10351562,309.60351726)
\curveto(229.34439505,309.60351766)(228.00910993,310.15039211)(227.09765625,311.24414226)
\curveto(226.18619508,312.33788992)(225.73046637,313.9374977)(225.73046875,316.04297039)
\moveto(232.06054688,325.68164226)
\lineto(232.06054688,325.68164226)
}
}
{
\newrgbcolor{curcolor}{0 0 0}
\pscustom[linestyle=none,fillstyle=solid,fillcolor=curcolor]
{
\newpath
\moveto(256.86132812,318.28515789)
\lineto(256.86132812,317.05468914)
\lineto(245.29492188,317.05468914)
\curveto(245.4042926,315.32291298)(245.92382333,314.00129972)(246.85351562,313.08984539)
\curveto(247.79231104,312.18749945)(249.09569515,311.73632802)(250.76367188,311.73632976)
\curveto(251.7298071,311.73632802)(252.66405096,311.85481749)(253.56640625,312.09179851)
\curveto(254.47785123,312.32877535)(255.38019408,312.68424374)(256.2734375,313.15820476)
\lineto(256.2734375,310.77929851)
\curveto(255.3710795,310.39648561)(254.44595022,310.10481924)(253.49804688,309.90429851)
\curveto(252.55011878,309.70377797)(251.5885312,309.60351766)(250.61328125,309.60351726)
\curveto(248.17056587,309.60351766)(246.23371885,310.31445445)(244.80273438,311.73632976)
\curveto(243.38085712,313.1582016)(242.66992033,315.08137676)(242.66992188,317.50586101)
\curveto(242.66992033,320.01236142)(243.34439882,321.99933859)(244.69335938,323.46679851)
\curveto(246.05142737,324.94334607)(247.8788995,325.68162658)(250.17578125,325.68164226)
\curveto(252.23566597,325.68162658)(253.86261747,325.01626266)(255.05664062,323.68554851)
\curveto(256.25975049,322.36392156)(256.86131239,320.56379316)(256.86132812,318.28515789)
\moveto(254.34570312,319.02343914)
\curveto(254.32746076,320.39973082)(253.94009135,321.49803701)(253.18359375,322.31836101)
\curveto(252.43618661,323.13866037)(251.44269802,323.54881621)(250.203125,323.54882976)
\curveto(248.79947149,323.54881621)(247.67382158,323.15233223)(246.82617188,322.35937664)
\curveto(245.98762535,321.56639632)(245.50455291,320.44986098)(245.37695312,319.00976726)
\lineto(254.34570312,319.02343914)
}
}
{
\newrgbcolor{curcolor}{0 0 0}
\pscustom[linestyle=none,fillstyle=solid,fillcolor=curcolor]
{
\newpath
\moveto(260.73046875,316.04297039)
\lineto(260.73046875,325.31250164)
\lineto(263.24609375,325.31250164)
\lineto(263.24609375,316.13867351)
\curveto(263.24608886,314.68945007)(263.52864066,313.60025845)(264.09375,312.87109539)
\curveto(264.65884786,312.15104115)(265.50650326,311.79101547)(266.63671875,311.79101726)
\curveto(267.99478202,311.79101547)(269.06574449,312.22395775)(269.84960938,313.08984539)
\curveto(270.64256583,313.95572685)(271.03904981,315.13606421)(271.0390625,316.63086101)
\lineto(271.0390625,325.31250164)
\lineto(273.5546875,325.31250164)
\lineto(273.5546875,310.00000164)
\lineto(271.0390625,310.00000164)
\lineto(271.0390625,312.35156414)
\curveto(270.42837334,311.42187521)(269.71743655,310.72916757)(268.90625,310.27343914)
\curveto(268.10415691,309.82682473)(267.16991306,309.60351766)(266.10351562,309.60351726)
\curveto(264.34439505,309.60351766)(263.00910993,310.15039211)(262.09765625,311.24414226)
\curveto(261.18619508,312.33788992)(260.73046637,313.9374977)(260.73046875,316.04297039)
\moveto(267.06054688,325.68164226)
\lineto(267.06054688,325.68164226)
}
}
{
\newrgbcolor{curcolor}{0 0 0}
\pscustom[linestyle=none,fillstyle=solid,fillcolor=curcolor]
{
\newpath
\moveto(287.63671875,322.96093914)
\curveto(287.35415544,323.12498851)(287.04425991,323.24347798)(286.70703125,323.31640789)
\curveto(286.378896,323.39842574)(286.01431303,323.43944132)(285.61328125,323.43945476)
\curveto(284.19139818,323.43944132)(283.09764928,322.97459804)(282.33203125,322.04492351)
\curveto(281.57551538,321.12433947)(281.19726055,319.79816892)(281.19726562,318.06640789)
\lineto(281.19726562,310.00000164)
\lineto(278.66796875,310.00000164)
\lineto(278.66796875,325.31250164)
\lineto(281.19726562,325.31250164)
\lineto(281.19726562,322.93359539)
\curveto(281.72590586,323.86326902)(282.41405621,324.55141938)(283.26171875,324.99804851)
\curveto(284.10936702,325.45376222)(285.1393139,325.68162658)(286.3515625,325.68164226)
\curveto(286.52472918,325.68162658)(286.71613524,325.66795472)(286.92578125,325.64062664)
\curveto(287.13540566,325.62238185)(287.3678273,325.59048084)(287.62304688,325.54492351)
\lineto(287.63671875,322.96093914)
}
}
{
\newrgbcolor{curcolor}{0 0 0}
\pscustom[linestyle=none,fillstyle=solid,fillcolor=curcolor,opacity=0.11935484]
{
\newpath
\moveto(297.69348717,289.99965068)
\lineto(414.94477272,289.99965068)
\curveto(423.34524385,289.99965068)(430.108078,283.23681653)(430.108078,274.8363454)
\curveto(430.108078,266.43587427)(423.34524385,259.67304012)(414.94477272,259.67304012)
\lineto(297.69348717,259.67304012)
\curveto(289.29301604,259.67304012)(282.53018188,266.43587427)(282.53018188,274.8363454)
\curveto(282.53018188,283.23681653)(289.29301604,289.99965068)(297.69348717,289.99965068)
\closepath
}
}
{
\newrgbcolor{curcolor}{0 0 0}
\pscustom[linewidth=2,linecolor=curcolor]
{
\newpath
\moveto(297.69348717,289.99965068)
\lineto(414.94477272,289.99965068)
\curveto(423.34524385,289.99965068)(430.108078,283.23681653)(430.108078,274.8363454)
\curveto(430.108078,266.43587427)(423.34524385,259.67304012)(414.94477272,259.67304012)
\lineto(297.69348717,259.67304012)
\curveto(289.29301604,259.67304012)(282.53018188,266.43587427)(282.53018188,274.8363454)
\curveto(282.53018188,283.23681653)(289.29301604,289.99965068)(297.69348717,289.99965068)
\closepath
}
}
{
\newrgbcolor{curcolor}{0 0 0}
\pscustom[linestyle=none,fillstyle=solid,fillcolor=curcolor]
{
\newpath
\moveto(132.35546875,227.49609539)
\lineto(135.8828125,227.49609539)
\lineto(140.34765625,215.58984539)
\lineto(144.8359375,227.49609539)
\lineto(148.36328125,227.49609539)
\lineto(148.36328125,210.00000164)
\lineto(146.0546875,210.00000164)
\lineto(146.0546875,225.36328289)
\lineto(141.54296875,213.36328289)
\lineto(139.1640625,213.36328289)
\lineto(134.65234375,225.36328289)
\lineto(134.65234375,210.00000164)
\lineto(132.35546875,210.00000164)
\lineto(132.35546875,227.49609539)
}
}
{
\newrgbcolor{curcolor}{0 0 0}
\pscustom[linestyle=none,fillstyle=solid,fillcolor=curcolor]
{
\newpath
\moveto(158.06640625,221.61328289)
\curveto(156.91015006,221.61327127)(155.99608847,221.16014673)(155.32421875,220.25390789)
\curveto(154.65233982,219.35546103)(154.31640265,218.12108726)(154.31640625,216.55078289)
\curveto(154.31640265,214.98046541)(154.64843357,213.74218539)(155.3125,212.83593914)
\curveto(155.98436973,211.9374997)(156.90233757,211.4882814)(158.06640625,211.48828289)
\curveto(159.21483525,211.4882814)(160.12499059,211.94140594)(160.796875,212.84765789)
\curveto(161.46873925,213.75390413)(161.80467641,214.9882779)(161.8046875,216.55078289)
\curveto(161.80467641,218.10546228)(161.46873925,219.3359298)(160.796875,220.24218914)
\curveto(160.12499059,221.15624048)(159.21483525,221.61327127)(158.06640625,221.61328289)
\moveto(158.06640625,223.44140789)
\curveto(159.94139703,223.44139444)(161.4140518,222.83202005)(162.484375,221.61328289)
\curveto(163.55467466,220.39452249)(164.08983038,218.70702418)(164.08984375,216.55078289)
\curveto(164.08983038,214.40234098)(163.55467466,212.71484267)(162.484375,211.48828289)
\curveto(161.4140518,210.26953262)(159.94139703,209.66015823)(158.06640625,209.66015789)
\curveto(156.18358829,209.66015823)(154.70702726,210.26953262)(153.63671875,211.48828289)
\curveto(152.57421689,212.71484267)(152.04296743,214.40234098)(152.04296875,216.55078289)
\curveto(152.04296743,218.70702418)(152.57421689,220.39452249)(153.63671875,221.61328289)
\curveto(154.70702726,222.83202005)(156.18358829,223.44139444)(158.06640625,223.44140789)
}
}
{
\newrgbcolor{curcolor}{0 0 0}
\pscustom[linestyle=none,fillstyle=solid,fillcolor=curcolor]
{
\newpath
\moveto(169.78515625,226.85156414)
\lineto(169.78515625,223.12500164)
\lineto(174.2265625,223.12500164)
\lineto(174.2265625,221.44922039)
\lineto(169.78515625,221.44922039)
\lineto(169.78515625,214.32422039)
\curveto(169.78515186,213.25390463)(169.92968296,212.56640532)(170.21875,212.26172039)
\curveto(170.51561987,211.95703093)(171.11327553,211.80468733)(172.01171875,211.80468914)
\lineto(174.2265625,211.80468914)
\lineto(174.2265625,210.00000164)
\lineto(172.01171875,210.00000164)
\curveto(170.34765129,210.00000164)(169.19921494,210.30859508)(168.56640625,210.92578289)
\curveto(167.93359121,211.55078133)(167.61718527,212.6835927)(167.6171875,214.32422039)
\lineto(167.6171875,221.44922039)
\lineto(166.03515625,221.44922039)
\lineto(166.03515625,223.12500164)
\lineto(167.6171875,223.12500164)
\lineto(167.6171875,226.85156414)
\lineto(169.78515625,226.85156414)
}
}
{
\newrgbcolor{curcolor}{0 0 0}
\pscustom[linestyle=none,fillstyle=solid,fillcolor=curcolor]
{
}
}
{
\newrgbcolor{curcolor}{0 0 0}
\pscustom[linestyle=none,fillstyle=solid,fillcolor=curcolor]
{
\newpath
\moveto(193.3515625,221.13281414)
\lineto(193.3515625,228.23437664)
\lineto(195.5078125,228.23437664)
\lineto(195.5078125,210.00000164)
\lineto(193.3515625,210.00000164)
\lineto(193.3515625,211.96875164)
\curveto(192.89842705,211.18750045)(192.32420888,210.60546978)(191.62890625,210.22265789)
\curveto(190.94139776,209.84765804)(190.11327359,209.66015823)(189.14453125,209.66015789)
\curveto(187.55858864,209.66015823)(186.26562119,210.29297009)(185.265625,211.55859539)
\curveto(184.27343568,212.82421756)(183.77734243,214.4882784)(183.77734375,216.55078289)
\curveto(183.77734243,218.61327427)(184.27343568,220.27733511)(185.265625,221.54297039)
\curveto(186.26562119,222.80858258)(187.55858864,223.44139444)(189.14453125,223.44140789)
\curveto(190.11327359,223.44139444)(190.94139776,223.24998839)(191.62890625,222.86718914)
\curveto(192.32420888,222.49217664)(192.89842705,221.91405222)(193.3515625,221.13281414)
\moveto(186.00390625,216.55078289)
\curveto(186.0039027,214.96484042)(186.32812112,213.71874792)(186.9765625,212.81250164)
\curveto(187.63280732,211.91406222)(188.53124392,211.46484392)(189.671875,211.46484539)
\curveto(190.81249164,211.46484392)(191.71092824,211.91406222)(192.3671875,212.81250164)
\curveto(193.02342693,213.71874792)(193.3515516,214.96484042)(193.3515625,216.55078289)
\curveto(193.3515516,218.13671225)(193.02342693,219.37889851)(192.3671875,220.27734539)
\curveto(191.71092824,221.1835842)(190.81249164,221.63670875)(189.671875,221.63672039)
\curveto(188.53124392,221.63670875)(187.63280732,221.1835842)(186.9765625,220.27734539)
\curveto(186.32812112,219.37889851)(186.0039027,218.13671225)(186.00390625,216.55078289)
}
}
{
\newrgbcolor{curcolor}{0 0 0}
\pscustom[linestyle=none,fillstyle=solid,fillcolor=curcolor]
{
\newpath
\moveto(211.17578125,217.10156414)
\lineto(211.17578125,216.04687664)
\lineto(201.26171875,216.04687664)
\curveto(201.35546508,214.56249707)(201.80077714,213.42968571)(202.59765625,212.64843914)
\curveto(203.40233804,211.87499976)(204.51952442,211.4882814)(205.94921875,211.48828289)
\curveto(206.77733466,211.4882814)(207.57811511,211.5898438)(208.3515625,211.79297039)
\curveto(209.13280105,211.99609339)(209.90623778,212.30078058)(210.671875,212.70703289)
\lineto(210.671875,210.66797039)
\curveto(209.89842529,210.33984505)(209.10545733,210.0898453)(208.29296875,209.91797039)
\curveto(207.48045896,209.74609564)(206.65624103,209.66015823)(205.8203125,209.66015789)
\curveto(203.72655646,209.66015823)(202.06640187,210.26953262)(200.83984375,211.48828289)
\curveto(199.62109182,212.70703018)(199.01171743,214.35546603)(199.01171875,216.43359539)
\curveto(199.01171743,218.5820243)(199.58984185,220.2851476)(200.74609375,221.54297039)
\curveto(201.91015203,222.80858258)(203.47655671,223.44139444)(205.4453125,223.44140789)
\curveto(207.21092798,223.44139444)(208.60545783,222.87108251)(209.62890625,221.73047039)
\curveto(210.66014328,220.59764729)(211.17576776,219.05468008)(211.17578125,217.10156414)
\moveto(209.01953125,217.73437664)
\curveto(209.00389493,218.91405522)(208.67186402,219.85546053)(208.0234375,220.55859539)
\curveto(207.3828028,221.26170912)(206.53124116,221.61327127)(205.46875,221.61328289)
\curveto(204.26561842,221.61327127)(203.30077564,221.27342786)(202.57421875,220.59375164)
\curveto(201.85546458,219.91405422)(201.4414025,218.95702393)(201.33203125,217.72265789)
\lineto(209.01953125,217.73437664)
}
}
{
\newrgbcolor{curcolor}{0 0 0}
\pscustom[linestyle=none,fillstyle=solid,fillcolor=curcolor]
{
}
}
{
\newrgbcolor{curcolor}{0 0 0}
\pscustom[linestyle=none,fillstyle=solid,fillcolor=curcolor]
{
\newpath
\moveto(224.44140625,211.96875164)
\lineto(224.44140625,205.00781414)
\lineto(222.2734375,205.00781414)
\lineto(222.2734375,223.12500164)
\lineto(224.44140625,223.12500164)
\lineto(224.44140625,221.13281414)
\curveto(224.89452645,221.91405222)(225.46483838,222.49217664)(226.15234375,222.86718914)
\curveto(226.8476495,223.24998839)(227.67577367,223.44139444)(228.63671875,223.44140789)
\curveto(230.23045861,223.44139444)(231.52342607,222.80858258)(232.515625,221.54297039)
\curveto(233.51561158,220.27733511)(234.01561108,218.61327427)(234.015625,216.55078289)
\curveto(234.01561108,214.4882784)(233.51561158,212.82421756)(232.515625,211.55859539)
\curveto(231.52342607,210.29297009)(230.23045861,209.66015823)(228.63671875,209.66015789)
\curveto(227.67577367,209.66015823)(226.8476495,209.84765804)(226.15234375,210.22265789)
\curveto(225.46483838,210.60546978)(224.89452645,211.18750045)(224.44140625,211.96875164)
\moveto(231.77734375,216.55078289)
\curveto(231.77733207,218.13671225)(231.44920739,219.37889851)(230.79296875,220.27734539)
\curveto(230.1445212,221.1835842)(229.24999084,221.63670875)(228.109375,221.63672039)
\curveto(226.96874313,221.63670875)(226.07030652,221.1835842)(225.4140625,220.27734539)
\curveto(224.76562033,219.37889851)(224.4414019,218.13671225)(224.44140625,216.55078289)
\curveto(224.4414019,214.96484042)(224.76562033,213.71874792)(225.4140625,212.81250164)
\curveto(226.07030652,211.91406222)(226.96874313,211.46484392)(228.109375,211.46484539)
\curveto(229.24999084,211.46484392)(230.1445212,211.91406222)(230.79296875,212.81250164)
\curveto(231.44920739,213.71874792)(231.77733207,214.96484042)(231.77734375,216.55078289)
}
}
{
\newrgbcolor{curcolor}{0 0 0}
\pscustom[linestyle=none,fillstyle=solid,fillcolor=curcolor]
{
\newpath
\moveto(243.5546875,216.59765789)
\curveto(241.81249352,216.59765129)(240.60546347,216.39843274)(239.93359375,216.00000164)
\curveto(239.26171482,215.60155853)(238.92577765,214.92187171)(238.92578125,213.96093914)
\curveto(238.92577765,213.19531094)(239.1757774,212.58593655)(239.67578125,212.13281414)
\curveto(240.18358889,211.68749995)(240.87108821,211.46484392)(241.73828125,211.46484539)
\curveto(242.93358614,211.46484392)(243.89061644,211.8867185)(244.609375,212.73047039)
\curveto(245.33592749,213.5820293)(245.69920838,214.71093442)(245.69921875,216.11718914)
\lineto(245.69921875,216.59765789)
\lineto(243.5546875,216.59765789)
\moveto(247.85546875,217.48828289)
\lineto(247.85546875,210.00000164)
\lineto(245.69921875,210.00000164)
\lineto(245.69921875,211.99218914)
\curveto(245.20702137,211.19531294)(244.59374073,210.60546978)(243.859375,210.22265789)
\curveto(243.1249922,209.84765804)(242.2265556,209.66015823)(241.1640625,209.66015789)
\curveto(239.82030801,209.66015823)(238.74999658,210.03515785)(237.953125,210.78515789)
\curveto(237.16406066,211.54296884)(236.76952981,212.55468658)(236.76953125,213.82031414)
\curveto(236.76952981,215.29687134)(237.26171682,216.41015148)(238.24609375,217.16015789)
\curveto(239.23827734,217.91014998)(240.71483836,218.2851496)(242.67578125,218.28515789)
\lineto(245.69921875,218.28515789)
\lineto(245.69921875,218.49609539)
\curveto(245.69920838,219.4882734)(245.37108371,220.25389763)(244.71484375,220.79297039)
\curveto(244.06639751,221.33983405)(243.15233593,221.61327127)(241.97265625,221.61328289)
\curveto(241.22265036,221.61327127)(240.49218234,221.52342761)(239.78125,221.34375164)
\curveto(239.07030876,221.16405297)(238.38671569,220.89452199)(237.73046875,220.53515789)
\lineto(237.73046875,222.52734539)
\curveto(238.51952806,222.83202005)(239.28515229,223.05858233)(240.02734375,223.20703289)
\curveto(240.76952581,223.36326952)(241.49218134,223.44139444)(242.1953125,223.44140789)
\curveto(244.09374123,223.44139444)(245.51170857,222.94920744)(246.44921875,221.96484539)
\curveto(247.38670669,220.98045941)(247.85545622,219.4882734)(247.85546875,217.48828289)
}
}
{
\newrgbcolor{curcolor}{0 0 0}
\pscustom[linestyle=none,fillstyle=solid,fillcolor=curcolor]
{
\newpath
\moveto(260.67578125,222.73828289)
\lineto(260.67578125,220.69922039)
\curveto(260.06639623,221.01170937)(259.43358436,221.24608414)(258.77734375,221.40234539)
\curveto(258.12108568,221.55858383)(257.44139886,221.63670875)(256.73828125,221.63672039)
\curveto(255.66796313,221.63670875)(254.86327643,221.47264641)(254.32421875,221.14453289)
\curveto(253.792965,220.81639707)(253.52734027,220.32421006)(253.52734375,219.66797039)
\curveto(253.52734027,219.16796122)(253.71874633,218.77343036)(254.1015625,218.48437664)
\curveto(254.48437056,218.20311843)(255.25390104,217.93358745)(256.41015625,217.67578289)
\lineto(257.1484375,217.51172039)
\curveto(258.67967887,217.1835882)(259.76561528,216.71874492)(260.40625,216.11718914)
\curveto(261.05467649,215.52343361)(261.37889492,214.69140319)(261.37890625,213.62109539)
\curveto(261.37889492,212.40234298)(260.8945204,211.4375002)(259.92578125,210.72656414)
\curveto(258.96483483,210.01562662)(257.64061741,209.66015823)(255.953125,209.66015789)
\curveto(255.2499948,209.66015823)(254.51562053,209.73047066)(253.75,209.87109539)
\curveto(252.99218455,210.00390788)(252.19140411,210.20703268)(251.34765625,210.48047039)
\lineto(251.34765625,212.70703289)
\curveto(252.14452915,212.29296809)(252.92968462,211.98046841)(253.703125,211.76953289)
\curveto(254.47655807,211.56640632)(255.2421823,211.46484392)(256,211.46484539)
\curveto(257.01561803,211.46484392)(257.79686725,211.63671875)(258.34375,211.98047039)
\curveto(258.89061616,212.33203055)(259.16405338,212.82421756)(259.1640625,213.45703289)
\curveto(259.16405338,214.04296634)(258.96483483,214.49218464)(258.56640625,214.80468914)
\curveto(258.17577312,215.11718402)(257.31249273,215.41796497)(255.9765625,215.70703289)
\lineto(255.2265625,215.88281414)
\curveto(253.89062116,216.16405797)(252.92577837,216.59374504)(252.33203125,217.17187664)
\curveto(251.73827956,217.75780638)(251.44140486,218.55858683)(251.44140625,219.57422039)
\curveto(251.44140486,220.80858458)(251.87890442,221.76170862)(252.75390625,222.43359539)
\curveto(253.62890267,223.10545728)(254.87108893,223.44139444)(256.48046875,223.44140789)
\curveto(257.27733652,223.44139444)(258.02733577,223.38280075)(258.73046875,223.26562664)
\curveto(259.43358436,223.14842599)(260.08202121,222.97264491)(260.67578125,222.73828289)
}
}
{
\newrgbcolor{curcolor}{0 0 0}
\pscustom[linestyle=none,fillstyle=solid,fillcolor=curcolor]
{
\newpath
\moveto(273.19140625,222.73828289)
\lineto(273.19140625,220.69922039)
\curveto(272.58202123,221.01170937)(271.94920936,221.24608414)(271.29296875,221.40234539)
\curveto(270.63671068,221.55858383)(269.95702386,221.63670875)(269.25390625,221.63672039)
\curveto(268.18358813,221.63670875)(267.37890143,221.47264641)(266.83984375,221.14453289)
\curveto(266.30859,220.81639707)(266.04296527,220.32421006)(266.04296875,219.66797039)
\curveto(266.04296527,219.16796122)(266.23437133,218.77343036)(266.6171875,218.48437664)
\curveto(266.99999556,218.20311843)(267.76952604,217.93358745)(268.92578125,217.67578289)
\lineto(269.6640625,217.51172039)
\curveto(271.19530387,217.1835882)(272.28124028,216.71874492)(272.921875,216.11718914)
\curveto(273.57030149,215.52343361)(273.89451992,214.69140319)(273.89453125,213.62109539)
\curveto(273.89451992,212.40234298)(273.4101454,211.4375002)(272.44140625,210.72656414)
\curveto(271.48045983,210.01562662)(270.15624241,209.66015823)(268.46875,209.66015789)
\curveto(267.7656198,209.66015823)(267.03124553,209.73047066)(266.265625,209.87109539)
\curveto(265.50780955,210.00390788)(264.70702911,210.20703268)(263.86328125,210.48047039)
\lineto(263.86328125,212.70703289)
\curveto(264.66015415,212.29296809)(265.44530962,211.98046841)(266.21875,211.76953289)
\curveto(266.99218307,211.56640632)(267.7578073,211.46484392)(268.515625,211.46484539)
\curveto(269.53124303,211.46484392)(270.31249225,211.63671875)(270.859375,211.98047039)
\curveto(271.40624116,212.33203055)(271.67967838,212.82421756)(271.6796875,213.45703289)
\curveto(271.67967838,214.04296634)(271.48045983,214.49218464)(271.08203125,214.80468914)
\curveto(270.69139812,215.11718402)(269.82811773,215.41796497)(268.4921875,215.70703289)
\lineto(267.7421875,215.88281414)
\curveto(266.40624616,216.16405797)(265.44140337,216.59374504)(264.84765625,217.17187664)
\curveto(264.25390456,217.75780638)(263.95702986,218.55858683)(263.95703125,219.57422039)
\curveto(263.95702986,220.80858458)(264.39452942,221.76170862)(265.26953125,222.43359539)
\curveto(266.14452767,223.10545728)(267.38671393,223.44139444)(268.99609375,223.44140789)
\curveto(269.79296152,223.44139444)(270.54296077,223.38280075)(271.24609375,223.26562664)
\curveto(271.94920936,223.14842599)(272.59764621,222.97264491)(273.19140625,222.73828289)
}
}
{
\newrgbcolor{curcolor}{0 0 0}
\pscustom[linestyle=none,fillstyle=solid,fillcolor=curcolor]
{
\newpath
\moveto(288.56640625,217.10156414)
\lineto(288.56640625,216.04687664)
\lineto(278.65234375,216.04687664)
\curveto(278.74609008,214.56249707)(279.19140214,213.42968571)(279.98828125,212.64843914)
\curveto(280.79296304,211.87499976)(281.91014942,211.4882814)(283.33984375,211.48828289)
\curveto(284.16795966,211.4882814)(284.96874011,211.5898438)(285.7421875,211.79297039)
\curveto(286.52342605,211.99609339)(287.29686278,212.30078058)(288.0625,212.70703289)
\lineto(288.0625,210.66797039)
\curveto(287.28905029,210.33984505)(286.49608233,210.0898453)(285.68359375,209.91797039)
\curveto(284.87108396,209.74609564)(284.04686603,209.66015823)(283.2109375,209.66015789)
\curveto(281.11718146,209.66015823)(279.45702687,210.26953262)(278.23046875,211.48828289)
\curveto(277.01171682,212.70703018)(276.40234243,214.35546603)(276.40234375,216.43359539)
\curveto(276.40234243,218.5820243)(276.98046685,220.2851476)(278.13671875,221.54297039)
\curveto(279.30077703,222.80858258)(280.86718171,223.44139444)(282.8359375,223.44140789)
\curveto(284.60155298,223.44139444)(285.99608283,222.87108251)(287.01953125,221.73047039)
\curveto(288.05076828,220.59764729)(288.56639276,219.05468008)(288.56640625,217.10156414)
\moveto(286.41015625,217.73437664)
\curveto(286.39451993,218.91405522)(286.06248902,219.85546053)(285.4140625,220.55859539)
\curveto(284.7734278,221.26170912)(283.92186616,221.61327127)(282.859375,221.61328289)
\curveto(281.65624342,221.61327127)(280.69140064,221.27342786)(279.96484375,220.59375164)
\curveto(279.24608958,219.91405422)(278.8320275,218.95702393)(278.72265625,217.72265789)
\lineto(286.41015625,217.73437664)
}
}
{
\newrgbcolor{curcolor}{0 0 0}
\pscustom[linestyle=none,fillstyle=solid,fillcolor=curcolor,opacity=0.11935484]
{
\newpath
\moveto(254.75679207,459.86875268)
\lineto(415.0603466,459.86875268)
\curveto(423.38830262,459.86875268)(430.09275818,453.16429712)(430.09275818,444.83634111)
\curveto(430.09275818,436.5083851)(423.38830262,429.80392953)(415.0603466,429.80392953)
\lineto(254.75679207,429.80392953)
\curveto(246.42883606,429.80392953)(239.72438049,436.5083851)(239.72438049,444.83634111)
\curveto(239.72438049,453.16429712)(246.42883606,459.86875268)(254.75679207,459.86875268)
\closepath
}
}
{
\newrgbcolor{curcolor}{0 0 0}
\pscustom[linewidth=2,linecolor=curcolor]
{
\newpath
\moveto(254.75679207,459.86875268)
\lineto(415.0603466,459.86875268)
\curveto(423.38830262,459.86875268)(430.09275818,453.16429712)(430.09275818,444.83634111)
\curveto(430.09275818,436.5083851)(423.38830262,429.80392953)(415.0603466,429.80392953)
\lineto(254.75679207,429.80392953)
\curveto(246.42883606,429.80392953)(239.72438049,436.5083851)(239.72438049,444.83634111)
\curveto(239.72438049,453.16429712)(246.42883606,459.86875268)(254.75679207,459.86875268)
\closepath
}
}
{
\newrgbcolor{curcolor}{1 1 1}
\pscustom[linestyle=none,fillstyle=solid,fillcolor=curcolor]
{
\newpath
\moveto(141.63620281,49.84626934)
\lineto(228.64879322,49.84626934)
\curveto(235.04154532,49.84626934)(240.18805695,44.69975771)(240.18805695,38.30700561)
\lineto(240.18805695,31.46732971)
\curveto(240.18805695,25.0745776)(235.04154532,19.92806598)(228.64879322,19.92806598)
\lineto(141.63620281,19.92806598)
\curveto(135.24345071,19.92806598)(130.09693909,25.0745776)(130.09693909,31.46732971)
\lineto(130.09693909,38.30700561)
\curveto(130.09693909,44.69975771)(135.24345071,49.84626934)(141.63620281,49.84626934)
\closepath
}
}
{
\newrgbcolor{curcolor}{0 0 0}
\pscustom[linewidth=2,linecolor=curcolor]
{
\newpath
\moveto(141.63620281,49.84626934)
\lineto(228.64879322,49.84626934)
\curveto(235.04154532,49.84626934)(240.18805695,44.69975771)(240.18805695,38.30700561)
\lineto(240.18805695,31.46732971)
\curveto(240.18805695,25.0745776)(235.04154532,19.92806598)(228.64879322,19.92806598)
\lineto(141.63620281,19.92806598)
\curveto(135.24345071,19.92806598)(130.09693909,25.0745776)(130.09693909,31.46732971)
\lineto(130.09693909,38.30700561)
\curveto(130.09693909,44.69975771)(135.24345071,49.84626934)(141.63620281,49.84626934)
\closepath
}
}
{
\newrgbcolor{curcolor}{0 0 0}
\pscustom[linestyle=none,fillstyle=solid,fillcolor=curcolor]
{
\newpath
\moveto(160.65234375,38.20311138)
\curveto(161.16014509,38.03122835)(161.6523321,37.66404121)(162.12890625,37.10154888)
\curveto(162.61326864,36.53904234)(163.09764315,35.76560561)(163.58203125,34.78123638)
\lineto(165.984375,29.99998638)
\lineto(163.44140625,29.99998638)
\lineto(161.203125,34.48826763)
\curveto(160.62498938,35.66013697)(160.06248994,36.43747994)(159.515625,36.82029888)
\curveto(158.97655352,37.20310417)(158.23827301,37.39451023)(157.30078125,37.39451763)
\lineto(154.72265625,37.39451763)
\lineto(154.72265625,29.99998638)
\lineto(152.35546875,29.99998638)
\lineto(152.35546875,47.49608013)
\lineto(157.69921875,47.49608013)
\curveto(159.69920905,47.49606263)(161.19139506,47.0780943)(162.17578125,46.24217388)
\curveto(163.16014309,45.40622097)(163.6523301,44.14450348)(163.65234375,42.45701763)
\curveto(163.6523301,41.35544377)(163.39451786,40.44138219)(162.87890625,39.71483013)
\curveto(162.37108138,38.98825864)(161.62889462,38.48435289)(160.65234375,38.20311138)
\moveto(154.72265625,45.55076763)
\lineto(154.72265625,39.33983013)
\lineto(157.69921875,39.33983013)
\curveto(158.83983491,39.33982079)(159.69920905,39.60153928)(160.27734375,40.12498638)
\curveto(160.86327039,40.65622572)(161.15623884,41.43356869)(161.15625,42.45701763)
\curveto(161.15623884,43.48044165)(160.86327039,44.24997213)(160.27734375,44.76561138)
\curveto(159.69920905,45.28903359)(158.83983491,45.55075208)(157.69921875,45.55076763)
\lineto(154.72265625,45.55076763)
}
}
{
\newrgbcolor{curcolor}{0 0 0}
\pscustom[linestyle=none,fillstyle=solid,fillcolor=curcolor]
{
\newpath
\moveto(179.09765625,37.10154888)
\lineto(179.09765625,36.04686138)
\lineto(169.18359375,36.04686138)
\curveto(169.27734008,34.56248181)(169.72265214,33.42967045)(170.51953125,32.64842388)
\curveto(171.32421304,31.8749845)(172.44139942,31.48826614)(173.87109375,31.48826763)
\curveto(174.69920966,31.48826614)(175.49999011,31.58982854)(176.2734375,31.79295513)
\curveto(177.05467605,31.99607813)(177.82811278,32.30076533)(178.59375,32.70701763)
\lineto(178.59375,30.66795513)
\curveto(177.82030029,30.33982979)(177.02733233,30.08983004)(176.21484375,29.91795513)
\curveto(175.40233396,29.74608038)(174.57811603,29.66014297)(173.7421875,29.66014263)
\curveto(171.64843146,29.66014297)(169.98827687,30.26951736)(168.76171875,31.48826763)
\curveto(167.54296682,32.70701492)(166.93359243,34.35545077)(166.93359375,36.43358013)
\curveto(166.93359243,38.58200904)(167.51171685,40.28513234)(168.66796875,41.54295513)
\curveto(169.83202703,42.80856732)(171.39843171,43.44137919)(173.3671875,43.44139263)
\curveto(175.13280298,43.44137919)(176.52733283,42.87106726)(177.55078125,41.73045513)
\curveto(178.58201828,40.59763203)(179.09764276,39.05466482)(179.09765625,37.10154888)
\moveto(176.94140625,37.73436138)
\curveto(176.92576993,38.91403996)(176.59373902,39.85544527)(175.9453125,40.55858013)
\curveto(175.3046778,41.26169387)(174.45311616,41.61325601)(173.390625,41.61326763)
\curveto(172.18749342,41.61325601)(171.22265064,41.2734126)(170.49609375,40.59373638)
\curveto(169.77733958,39.91403896)(169.3632775,38.95700867)(169.25390625,37.72264263)
\lineto(176.94140625,37.73436138)
}
}
{
\newrgbcolor{curcolor}{0 0 0}
\pscustom[linestyle=none,fillstyle=solid,fillcolor=curcolor]
{
\newpath
\moveto(184.76953125,46.85154888)
\lineto(184.76953125,43.12498638)
\lineto(189.2109375,43.12498638)
\lineto(189.2109375,41.44920513)
\lineto(184.76953125,41.44920513)
\lineto(184.76953125,34.32420513)
\curveto(184.76952686,33.25388937)(184.91405796,32.56639006)(185.203125,32.26170513)
\curveto(185.49999487,31.95701567)(186.09765053,31.80467207)(186.99609375,31.80467388)
\lineto(189.2109375,31.80467388)
\lineto(189.2109375,29.99998638)
\lineto(186.99609375,29.99998638)
\curveto(185.33202629,29.99998638)(184.18358994,30.30857982)(183.55078125,30.92576763)
\curveto(182.91796621,31.55076608)(182.60156027,32.68357744)(182.6015625,34.32420513)
\lineto(182.6015625,41.44920513)
\lineto(181.01953125,41.44920513)
\lineto(181.01953125,43.12498638)
\lineto(182.6015625,43.12498638)
\lineto(182.6015625,46.85154888)
\lineto(184.76953125,46.85154888)
}
}
{
\newrgbcolor{curcolor}{0 0 0}
\pscustom[linestyle=none,fillstyle=solid,fillcolor=curcolor]
{
\newpath
\moveto(197.14453125,41.61326763)
\curveto(195.98827506,41.61325601)(195.07421347,41.16013147)(194.40234375,40.25389263)
\curveto(193.73046482,39.35544577)(193.39452765,38.12107201)(193.39453125,36.55076763)
\curveto(193.39452765,34.98045015)(193.72655857,33.74217013)(194.390625,32.83592388)
\curveto(195.06249473,31.93748444)(195.98046257,31.48826614)(197.14453125,31.48826763)
\curveto(198.29296025,31.48826614)(199.20311559,31.94139069)(199.875,32.84764263)
\curveto(200.54686425,33.75388887)(200.88280141,34.98826264)(200.8828125,36.55076763)
\curveto(200.88280141,38.10544702)(200.54686425,39.33591454)(199.875,40.24217388)
\curveto(199.20311559,41.15622522)(198.29296025,41.61325601)(197.14453125,41.61326763)
\moveto(197.14453125,43.44139263)
\curveto(199.01952203,43.44137919)(200.4921768,42.83200479)(201.5625,41.61326763)
\curveto(202.63279966,40.39450723)(203.16795538,38.70700892)(203.16796875,36.55076763)
\curveto(203.16795538,34.40232572)(202.63279966,32.71482741)(201.5625,31.48826763)
\curveto(200.4921768,30.26951736)(199.01952203,29.66014297)(197.14453125,29.66014263)
\curveto(195.26171329,29.66014297)(193.78515226,30.26951736)(192.71484375,31.48826763)
\curveto(191.65234189,32.71482741)(191.12109243,34.40232572)(191.12109375,36.55076763)
\curveto(191.12109243,38.70700892)(191.65234189,40.39450723)(192.71484375,41.61326763)
\curveto(193.78515226,42.83200479)(195.26171329,43.44137919)(197.14453125,43.44139263)
}
}
{
\newrgbcolor{curcolor}{0 0 0}
\pscustom[linestyle=none,fillstyle=solid,fillcolor=curcolor]
{
\newpath
\moveto(206.5078125,35.17967388)
\lineto(206.5078125,43.12498638)
\lineto(208.6640625,43.12498638)
\lineto(208.6640625,35.26170513)
\curveto(208.6640583,34.01951361)(208.90624556,33.08592079)(209.390625,32.46092388)
\curveto(209.87499459,31.84373453)(210.60155637,31.53514109)(211.5703125,31.53514263)
\curveto(212.73436673,31.53514109)(213.65233457,31.90623447)(214.32421875,32.64842388)
\curveto(215.00389571,33.39060799)(215.34373913,34.40232572)(215.34375,35.68358013)
\lineto(215.34375,43.12498638)
\lineto(217.5,43.12498638)
\lineto(217.5,29.99998638)
\lineto(215.34375,29.99998638)
\lineto(215.34375,32.01561138)
\curveto(214.82030215,31.21873516)(214.21092776,30.62498575)(213.515625,30.23436138)
\curveto(212.82811664,29.85154903)(212.02733619,29.66014297)(211.11328125,29.66014263)
\curveto(209.60546361,29.66014297)(208.46093351,30.1288925)(207.6796875,31.06639263)
\curveto(206.89843507,32.00389062)(206.50781046,33.374983)(206.5078125,35.17967388)
\moveto(211.93359375,43.44139263)
\lineto(211.93359375,43.44139263)
}
}
{
\newrgbcolor{curcolor}{0 0 0}
\pscustom[linestyle=none,fillstyle=solid,fillcolor=curcolor]
{
\newpath
\moveto(229.5703125,41.10936138)
\curveto(229.32811538,41.24997513)(229.06249064,41.35153753)(228.7734375,41.41404888)
\curveto(228.49217871,41.48434989)(228.17967902,41.51950611)(227.8359375,41.51951763)
\curveto(226.61718059,41.51950611)(225.67968152,41.12106901)(225.0234375,40.32420513)
\curveto(224.37499533,39.53513309)(224.0507769,38.39841548)(224.05078125,36.91404888)
\lineto(224.05078125,29.99998638)
\lineto(221.8828125,29.99998638)
\lineto(221.8828125,43.12498638)
\lineto(224.05078125,43.12498638)
\lineto(224.05078125,41.08592388)
\curveto(224.50390145,41.88278699)(225.09374461,42.47263015)(225.8203125,42.85545513)
\curveto(226.54686816,43.24606688)(227.42967977,43.44137919)(228.46875,43.44139263)
\curveto(228.61717859,43.44137919)(228.78124092,43.42966045)(228.9609375,43.40623638)
\curveto(229.14061556,43.39059799)(229.33983411,43.36325426)(229.55859375,43.32420513)
\lineto(229.5703125,41.10936138)
}
}
{
\newrgbcolor{curcolor}{1 1 1}
\pscustom[linestyle=none,fillstyle=solid,fillcolor=curcolor]
{
\newpath
\moveto(331.97983837,49.87309429)
\lineto(428.58592701,49.87309429)
\curveto(434.99059414,49.87309429)(440.146698,44.71699043)(440.146698,38.3123233)
\lineto(440.146698,31.48006507)
\curveto(440.146698,25.07539795)(434.99059414,19.91929409)(428.58592701,19.91929409)
\lineto(331.97983837,19.91929409)
\curveto(325.57517124,19.91929409)(320.41906738,25.07539795)(320.41906738,31.48006507)
\lineto(320.41906738,38.3123233)
\curveto(320.41906738,44.71699043)(325.57517124,49.87309429)(331.97983837,49.87309429)
\closepath
}
}
{
\newrgbcolor{curcolor}{0 0 0}
\pscustom[linewidth=2,linecolor=curcolor]
{
\newpath
\moveto(331.97983837,49.87309429)
\lineto(428.58592701,49.87309429)
\curveto(434.99059414,49.87309429)(440.146698,44.71699043)(440.146698,38.3123233)
\lineto(440.146698,31.48006507)
\curveto(440.146698,25.07539795)(434.99059414,19.91929409)(428.58592701,19.91929409)
\lineto(331.97983837,19.91929409)
\curveto(325.57517124,19.91929409)(320.41906738,25.07539795)(320.41906738,31.48006507)
\lineto(320.41906738,38.3123233)
\curveto(320.41906738,44.71699043)(325.57517124,49.87309429)(331.97983837,49.87309429)
\closepath
}
}
{
\newrgbcolor{curcolor}{0 0 0}
\pscustom[linestyle=none,fillstyle=solid,fillcolor=curcolor]
{
\newpath
\moveto(346.8671875,29.99998638)
\lineto(340.1875,47.49608013)
\lineto(342.66015625,47.49608013)
\lineto(348.203125,32.76561138)
\lineto(353.7578125,47.49608013)
\lineto(356.21875,47.49608013)
\lineto(349.55078125,29.99998638)
\lineto(346.8671875,29.99998638)
}
}
{
\newrgbcolor{curcolor}{0 0 0}
\pscustom[linestyle=none,fillstyle=solid,fillcolor=curcolor]
{
\newpath
\moveto(362.7578125,36.59764263)
\curveto(361.01561852,36.59763603)(359.80858847,36.39841748)(359.13671875,35.99998638)
\curveto(358.46483982,35.60154328)(358.12890265,34.92185646)(358.12890625,33.96092388)
\curveto(358.12890265,33.19529568)(358.3789024,32.58592129)(358.87890625,32.13279888)
\curveto(359.38671389,31.68748469)(360.07421321,31.46482866)(360.94140625,31.46483013)
\curveto(362.13671114,31.46482866)(363.09374144,31.88670324)(363.8125,32.73045513)
\curveto(364.53905249,33.58201404)(364.90233338,34.71091917)(364.90234375,36.11717388)
\lineto(364.90234375,36.59764263)
\lineto(362.7578125,36.59764263)
\moveto(367.05859375,37.48826763)
\lineto(367.05859375,29.99998638)
\lineto(364.90234375,29.99998638)
\lineto(364.90234375,31.99217388)
\curveto(364.41014637,31.19529768)(363.79686573,30.60545452)(363.0625,30.22264263)
\curveto(362.3281172,29.84764278)(361.4296806,29.66014297)(360.3671875,29.66014263)
\curveto(359.02343301,29.66014297)(357.95312158,30.03514259)(357.15625,30.78514263)
\curveto(356.36718566,31.54295358)(355.97265481,32.55467132)(355.97265625,33.82029888)
\curveto(355.97265481,35.29685608)(356.46484182,36.41013622)(357.44921875,37.16014263)
\curveto(358.44140234,37.91013472)(359.91796336,38.28513434)(361.87890625,38.28514263)
\lineto(364.90234375,38.28514263)
\lineto(364.90234375,38.49608013)
\curveto(364.90233338,39.48825814)(364.57420871,40.25388237)(363.91796875,40.79295513)
\curveto(363.26952251,41.33981879)(362.35546093,41.61325601)(361.17578125,41.61326763)
\curveto(360.42577536,41.61325601)(359.69530734,41.52341235)(358.984375,41.34373638)
\curveto(358.27343376,41.16403771)(357.58984069,40.89450673)(356.93359375,40.53514263)
\lineto(356.93359375,42.52733013)
\curveto(357.72265306,42.83200479)(358.48827729,43.05856707)(359.23046875,43.20701763)
\curveto(359.97265081,43.36325426)(360.69530634,43.44137919)(361.3984375,43.44139263)
\curveto(363.29686623,43.44137919)(364.71483357,42.94919218)(365.65234375,41.96483013)
\curveto(366.58983169,40.98044415)(367.05858122,39.48825814)(367.05859375,37.48826763)
}
}
{
\newrgbcolor{curcolor}{0 0 0}
\pscustom[linestyle=none,fillstyle=solid,fillcolor=curcolor]
{
\newpath
\moveto(371.51171875,48.23436138)
\lineto(373.66796875,48.23436138)
\lineto(373.66796875,29.99998638)
\lineto(371.51171875,29.99998638)
\lineto(371.51171875,48.23436138)
}
}
{
\newrgbcolor{curcolor}{0 0 0}
\pscustom[linestyle=none,fillstyle=solid,fillcolor=curcolor]
{
\newpath
\moveto(378.16796875,43.12498638)
\lineto(380.32421875,43.12498638)
\lineto(380.32421875,29.99998638)
\lineto(378.16796875,29.99998638)
\lineto(378.16796875,43.12498638)
\moveto(378.16796875,48.23436138)
\lineto(380.32421875,48.23436138)
\lineto(380.32421875,45.50389263)
\lineto(378.16796875,45.50389263)
\lineto(378.16796875,48.23436138)
}
}
{
\newrgbcolor{curcolor}{0 0 0}
\pscustom[linestyle=none,fillstyle=solid,fillcolor=curcolor]
{
\newpath
\moveto(393.4609375,41.13279888)
\lineto(393.4609375,48.23436138)
\lineto(395.6171875,48.23436138)
\lineto(395.6171875,29.99998638)
\lineto(393.4609375,29.99998638)
\lineto(393.4609375,31.96873638)
\curveto(393.00780205,31.18748519)(392.43358388,30.60545452)(391.73828125,30.22264263)
\curveto(391.05077276,29.84764278)(390.22264859,29.66014297)(389.25390625,29.66014263)
\curveto(387.66796364,29.66014297)(386.37499619,30.29295483)(385.375,31.55858013)
\curveto(384.38281068,32.8242023)(383.88671743,34.48826314)(383.88671875,36.55076763)
\curveto(383.88671743,38.61325901)(384.38281068,40.27731985)(385.375,41.54295513)
\curveto(386.37499619,42.80856732)(387.66796364,43.44137919)(389.25390625,43.44139263)
\curveto(390.22264859,43.44137919)(391.05077276,43.24997313)(391.73828125,42.86717388)
\curveto(392.43358388,42.49216138)(393.00780205,41.91403696)(393.4609375,41.13279888)
\moveto(386.11328125,36.55076763)
\curveto(386.1132777,34.96482516)(386.43749612,33.71873266)(387.0859375,32.81248638)
\curveto(387.74218232,31.91404696)(388.64061892,31.46482866)(389.78125,31.46483013)
\curveto(390.92186664,31.46482866)(391.82030324,31.91404696)(392.4765625,32.81248638)
\curveto(393.13280193,33.71873266)(393.4609266,34.96482516)(393.4609375,36.55076763)
\curveto(393.4609266,38.13669699)(393.13280193,39.37888325)(392.4765625,40.27733013)
\curveto(391.82030324,41.18356894)(390.92186664,41.63669349)(389.78125,41.63670513)
\curveto(388.64061892,41.63669349)(387.74218232,41.18356894)(387.0859375,40.27733013)
\curveto(386.43749612,39.37888325)(386.1132777,38.13669699)(386.11328125,36.55076763)
}
}
{
\newrgbcolor{curcolor}{0 0 0}
\pscustom[linestyle=none,fillstyle=solid,fillcolor=curcolor]
{
\newpath
\moveto(411.28515625,37.10154888)
\lineto(411.28515625,36.04686138)
\lineto(401.37109375,36.04686138)
\curveto(401.46484008,34.56248181)(401.91015214,33.42967045)(402.70703125,32.64842388)
\curveto(403.51171304,31.8749845)(404.62889942,31.48826614)(406.05859375,31.48826763)
\curveto(406.88670966,31.48826614)(407.68749011,31.58982854)(408.4609375,31.79295513)
\curveto(409.24217605,31.99607813)(410.01561278,32.30076533)(410.78125,32.70701763)
\lineto(410.78125,30.66795513)
\curveto(410.00780029,30.33982979)(409.21483233,30.08983004)(408.40234375,29.91795513)
\curveto(407.58983396,29.74608038)(406.76561603,29.66014297)(405.9296875,29.66014263)
\curveto(403.83593146,29.66014297)(402.17577687,30.26951736)(400.94921875,31.48826763)
\curveto(399.73046682,32.70701492)(399.12109243,34.35545077)(399.12109375,36.43358013)
\curveto(399.12109243,38.58200904)(399.69921685,40.28513234)(400.85546875,41.54295513)
\curveto(402.01952703,42.80856732)(403.58593171,43.44137919)(405.5546875,43.44139263)
\curveto(407.32030298,43.44137919)(408.71483283,42.87106726)(409.73828125,41.73045513)
\curveto(410.76951828,40.59763203)(411.28514276,39.05466482)(411.28515625,37.10154888)
\moveto(409.12890625,37.73436138)
\curveto(409.11326993,38.91403996)(408.78123902,39.85544527)(408.1328125,40.55858013)
\curveto(407.4921778,41.26169387)(406.64061616,41.61325601)(405.578125,41.61326763)
\curveto(404.37499342,41.61325601)(403.41015064,41.2734126)(402.68359375,40.59373638)
\curveto(401.96483958,39.91403896)(401.5507775,38.95700867)(401.44140625,37.72264263)
\lineto(409.12890625,37.73436138)
}
}
{
\newrgbcolor{curcolor}{0 0 0}
\pscustom[linestyle=none,fillstyle=solid,fillcolor=curcolor]
{
\newpath
\moveto(422.4296875,41.10936138)
\curveto(422.18749037,41.24997513)(421.92186564,41.35153753)(421.6328125,41.41404888)
\curveto(421.35155371,41.48434989)(421.03905402,41.51950611)(420.6953125,41.51951763)
\curveto(419.47655559,41.51950611)(418.53905652,41.12106901)(417.8828125,40.32420513)
\curveto(417.23437033,39.53513309)(416.9101519,38.39841548)(416.91015625,36.91404888)
\lineto(416.91015625,29.99998638)
\lineto(414.7421875,29.99998638)
\lineto(414.7421875,43.12498638)
\lineto(416.91015625,43.12498638)
\lineto(416.91015625,41.08592388)
\curveto(417.36327645,41.88278699)(417.95311961,42.47263015)(418.6796875,42.85545513)
\curveto(419.40624316,43.24606688)(420.28905477,43.44137919)(421.328125,43.44139263)
\curveto(421.47655359,43.44137919)(421.64061592,43.42966045)(421.8203125,43.40623638)
\curveto(421.99999056,43.39059799)(422.19920911,43.36325426)(422.41796875,43.32420513)
\lineto(422.4296875,41.10936138)
}
}
{
\newrgbcolor{curcolor}{0 0 0}
\pscustom[linestyle=none,fillstyle=solid,fillcolor=curcolor]
{
\newpath
\moveto(169.9296875,117.49608013)
\lineto(184.73046875,117.49608013)
\lineto(184.73046875,115.50389263)
\lineto(178.51953125,115.50389263)
\lineto(178.51953125,99.99998638)
\lineto(176.140625,99.99998638)
\lineto(176.140625,115.50389263)
\lineto(169.9296875,115.50389263)
\lineto(169.9296875,117.49608013)
}
}
{
\newrgbcolor{curcolor}{0 0 0}
\pscustom[linestyle=none,fillstyle=solid,fillcolor=curcolor]
{
\newpath
\moveto(188.64453125,98.78123638)
\curveto(188.03514914,97.21873916)(187.44139973,96.19920893)(186.86328125,95.72264263)
\curveto(186.28515089,95.24608488)(185.51171416,95.00780387)(184.54296875,95.00779888)
\lineto(182.8203125,95.00779888)
\lineto(182.8203125,96.81248638)
\lineto(184.0859375,96.81248638)
\curveto(184.67968374,96.81248956)(185.14062078,96.95311442)(185.46875,97.23436138)
\curveto(185.79687012,97.51561386)(186.16015101,98.1796757)(186.55859375,99.22654888)
\lineto(186.9453125,100.21092388)
\lineto(181.63671875,113.12498638)
\lineto(183.921875,113.12498638)
\lineto(188.0234375,102.85936138)
\lineto(192.125,113.12498638)
\lineto(194.41015625,113.12498638)
\lineto(188.64453125,98.78123638)
}
}
{
\newrgbcolor{curcolor}{0 0 0}
\pscustom[linestyle=none,fillstyle=solid,fillcolor=curcolor]
{
\newpath
\moveto(199.47265625,101.96873638)
\lineto(199.47265625,95.00779888)
\lineto(197.3046875,95.00779888)
\lineto(197.3046875,113.12498638)
\lineto(199.47265625,113.12498638)
\lineto(199.47265625,111.13279888)
\curveto(199.92577645,111.91403696)(200.49608838,112.49216138)(201.18359375,112.86717388)
\curveto(201.8788995,113.24997313)(202.70702367,113.44137919)(203.66796875,113.44139263)
\curveto(205.26170861,113.44137919)(206.55467607,112.80856732)(207.546875,111.54295513)
\curveto(208.54686158,110.27731985)(209.04686108,108.61325901)(209.046875,106.55076763)
\curveto(209.04686108,104.48826314)(208.54686158,102.8242023)(207.546875,101.55858013)
\curveto(206.55467607,100.29295483)(205.26170861,99.66014297)(203.66796875,99.66014263)
\curveto(202.70702367,99.66014297)(201.8788995,99.84764278)(201.18359375,100.22264263)
\curveto(200.49608838,100.60545452)(199.92577645,101.18748519)(199.47265625,101.96873638)
\moveto(206.80859375,106.55076763)
\curveto(206.80858207,108.13669699)(206.48045739,109.37888325)(205.82421875,110.27733013)
\curveto(205.1757712,111.18356894)(204.28124084,111.63669349)(203.140625,111.63670513)
\curveto(201.99999313,111.63669349)(201.10155652,111.18356894)(200.4453125,110.27733013)
\curveto(199.79687033,109.37888325)(199.4726519,108.13669699)(199.47265625,106.55076763)
\curveto(199.4726519,104.96482516)(199.79687033,103.71873266)(200.4453125,102.81248638)
\curveto(201.10155652,101.91404696)(201.99999313,101.46482866)(203.140625,101.46483013)
\curveto(204.28124084,101.46482866)(205.1757712,101.91404696)(205.82421875,102.81248638)
\curveto(206.48045739,103.71873266)(206.80858207,104.96482516)(206.80859375,106.55076763)
}
}
{
\newrgbcolor{curcolor}{0 0 0}
\pscustom[linestyle=none,fillstyle=solid,fillcolor=curcolor]
{
\newpath
\moveto(223.84765625,107.10154888)
\lineto(223.84765625,106.04686138)
\lineto(213.93359375,106.04686138)
\curveto(214.02734008,104.56248181)(214.47265214,103.42967045)(215.26953125,102.64842388)
\curveto(216.07421304,101.8749845)(217.19139942,101.48826614)(218.62109375,101.48826763)
\curveto(219.44920966,101.48826614)(220.24999011,101.58982854)(221.0234375,101.79295513)
\curveto(221.80467605,101.99607813)(222.57811278,102.30076533)(223.34375,102.70701763)
\lineto(223.34375,100.66795513)
\curveto(222.57030029,100.33982979)(221.77733233,100.08983004)(220.96484375,99.91795513)
\curveto(220.15233396,99.74608038)(219.32811603,99.66014297)(218.4921875,99.66014263)
\curveto(216.39843146,99.66014297)(214.73827687,100.26951736)(213.51171875,101.48826763)
\curveto(212.29296682,102.70701492)(211.68359243,104.35545077)(211.68359375,106.43358013)
\curveto(211.68359243,108.58200904)(212.26171685,110.28513234)(213.41796875,111.54295513)
\curveto(214.58202703,112.80856732)(216.14843171,113.44137919)(218.1171875,113.44139263)
\curveto(219.88280298,113.44137919)(221.27733283,112.87106726)(222.30078125,111.73045513)
\curveto(223.33201828,110.59763203)(223.84764276,109.05466482)(223.84765625,107.10154888)
\moveto(221.69140625,107.73436138)
\curveto(221.67576993,108.91403996)(221.34373902,109.85544527)(220.6953125,110.55858013)
\curveto(220.0546778,111.26169387)(219.20311616,111.61325601)(218.140625,111.61326763)
\curveto(216.93749342,111.61325601)(215.97265064,111.2734126)(215.24609375,110.59373638)
\curveto(214.52733958,109.91403896)(214.1132775,108.95700867)(214.00390625,107.72264263)
\lineto(221.69140625,107.73436138)
}
}
{
\newrgbcolor{curcolor}{0 0 0}
\pscustom[linestyle=none,fillstyle=solid,fillcolor=curcolor,opacity=0.11935484]
{
\newpath
\moveto(275.16365814,120.32731792)
\lineto(392.30652618,120.32731792)
\curveto(400.70719279,120.32731792)(407.47018433,113.56432639)(407.47018433,105.16365978)
\curveto(407.47018433,96.76299317)(400.70719279,90.00000164)(392.30652618,90.00000164)
\lineto(275.16365814,90.00000164)
\curveto(266.76299153,90.00000164)(260,96.76299317)(260,105.16365978)
\curveto(260,113.56432639)(266.76299153,120.32731792)(275.16365814,120.32731792)
\closepath
}
}
{
\newrgbcolor{curcolor}{0 0 0}
\pscustom[linewidth=2,linecolor=curcolor]
{
\newpath
\moveto(275.16365814,120.32731792)
\lineto(392.30652618,120.32731792)
\curveto(400.70719279,120.32731792)(407.47018433,113.56432639)(407.47018433,105.16365978)
\curveto(407.47018433,96.76299317)(400.70719279,90.00000164)(392.30652618,90.00000164)
\lineto(275.16365814,90.00000164)
\curveto(266.76299153,90.00000164)(260,96.76299317)(260,105.16365978)
\curveto(260,113.56432639)(266.76299153,120.32731792)(275.16365814,120.32731792)
\closepath
}
}
{
\newrgbcolor{curcolor}{0 0 0}
\pscustom[linestyle=none,fillstyle=solid,fillcolor=curcolor]
{
\newpath
\moveto(145.45703125,406.14843914)
\lineto(145.45703125,403.65234539)
\curveto(144.66014159,404.39451849)(143.80857994,404.94920544)(142.90234375,405.31640789)
\curveto(142.00389425,405.6835797)(141.04686395,405.86717327)(140.03125,405.86718914)
\curveto(138.03124197,405.86717327)(136.4999935,405.25389263)(135.4375,404.02734539)
\curveto(134.37499562,402.80858258)(133.84374616,401.04295934)(133.84375,398.73047039)
\curveto(133.84374616,396.42577646)(134.37499562,394.66015323)(135.4375,393.43359539)
\curveto(136.4999935,392.21484317)(138.03124197,391.60546878)(140.03125,391.60547039)
\curveto(141.04686395,391.60546878)(142.00389425,391.78906235)(142.90234375,392.15625164)
\curveto(143.80857994,392.52343661)(144.66014159,393.07812356)(145.45703125,393.82031414)
\lineto(145.45703125,391.34765789)
\curveto(144.62889162,390.7851571)(143.74998625,390.36328252)(142.8203125,390.08203289)
\curveto(141.8984256,389.80078308)(140.92186408,389.66015823)(139.890625,389.66015789)
\curveto(137.24218026,389.66015823)(135.15624484,390.46875117)(133.6328125,392.08593914)
\curveto(132.10937289,393.71093542)(131.3476549,395.92577696)(131.34765625,398.73047039)
\curveto(131.3476549,401.54295884)(132.10937289,403.75780038)(133.6328125,405.37500164)
\curveto(135.15624484,406.99998464)(137.24218026,407.81248382)(139.890625,407.81250164)
\curveto(140.93748906,407.81248382)(141.92186308,407.67185896)(142.84375,407.39062664)
\curveto(143.77342373,407.11717202)(144.64451661,406.70310993)(145.45703125,406.14843914)
}
}
{
\newrgbcolor{curcolor}{0 0 0}
\pscustom[linestyle=none,fillstyle=solid,fillcolor=curcolor]
{
\newpath
\moveto(154.12890625,401.61328289)
\curveto(152.97265006,401.61327127)(152.05858847,401.16014673)(151.38671875,400.25390789)
\curveto(150.71483982,399.35546103)(150.37890265,398.12108726)(150.37890625,396.55078289)
\curveto(150.37890265,394.98046541)(150.71093357,393.74218539)(151.375,392.83593914)
\curveto(152.04686973,391.9374997)(152.96483757,391.4882814)(154.12890625,391.48828289)
\curveto(155.27733525,391.4882814)(156.18749059,391.94140594)(156.859375,392.84765789)
\curveto(157.53123925,393.75390413)(157.86717641,394.9882779)(157.8671875,396.55078289)
\curveto(157.86717641,398.10546228)(157.53123925,399.3359298)(156.859375,400.24218914)
\curveto(156.18749059,401.15624048)(155.27733525,401.61327127)(154.12890625,401.61328289)
\moveto(154.12890625,403.44140789)
\curveto(156.00389703,403.44139444)(157.4765518,402.83202005)(158.546875,401.61328289)
\curveto(159.61717466,400.39452249)(160.15233038,398.70702418)(160.15234375,396.55078289)
\curveto(160.15233038,394.40234098)(159.61717466,392.71484267)(158.546875,391.48828289)
\curveto(157.4765518,390.26953262)(156.00389703,389.66015823)(154.12890625,389.66015789)
\curveto(152.24608829,389.66015823)(150.76952726,390.26953262)(149.69921875,391.48828289)
\curveto(148.63671689,392.71484267)(148.10546743,394.40234098)(148.10546875,396.55078289)
\curveto(148.10546743,398.70702418)(148.63671689,400.39452249)(149.69921875,401.61328289)
\curveto(150.76952726,402.83202005)(152.24608829,403.44139444)(154.12890625,403.44140789)
}
}
{
\newrgbcolor{curcolor}{0 0 0}
\pscustom[linestyle=none,fillstyle=solid,fillcolor=curcolor]
{
\newpath
\moveto(163.4921875,395.17968914)
\lineto(163.4921875,403.12500164)
\lineto(165.6484375,403.12500164)
\lineto(165.6484375,395.26172039)
\curveto(165.6484333,394.01952887)(165.89062056,393.08593605)(166.375,392.46093914)
\curveto(166.85936959,391.84374979)(167.58593137,391.53515635)(168.5546875,391.53515789)
\curveto(169.71874173,391.53515635)(170.63670957,391.90624973)(171.30859375,392.64843914)
\curveto(171.98827071,393.39062325)(172.32811413,394.40234098)(172.328125,395.68359539)
\lineto(172.328125,403.12500164)
\lineto(174.484375,403.12500164)
\lineto(174.484375,390.00000164)
\lineto(172.328125,390.00000164)
\lineto(172.328125,392.01562664)
\curveto(171.80467715,391.21875042)(171.19530276,390.62500101)(170.5,390.23437664)
\curveto(169.81249164,389.85156428)(169.01171119,389.66015823)(168.09765625,389.66015789)
\curveto(166.58983861,389.66015823)(165.44530851,390.12890776)(164.6640625,391.06640789)
\curveto(163.88281007,392.00390588)(163.49218546,393.37499826)(163.4921875,395.17968914)
\moveto(168.91796875,403.44140789)
\lineto(168.91796875,403.44140789)
}
}
{
\newrgbcolor{curcolor}{0 0 0}
\pscustom[linestyle=none,fillstyle=solid,fillcolor=curcolor]
{
\newpath
\moveto(178.94921875,408.23437664)
\lineto(181.10546875,408.23437664)
\lineto(181.10546875,390.00000164)
\lineto(178.94921875,390.00000164)
\lineto(178.94921875,408.23437664)
}
}
{
\newrgbcolor{curcolor}{0 0 0}
\pscustom[linestyle=none,fillstyle=solid,fillcolor=curcolor]
{
\newpath
\moveto(196.83203125,397.10156414)
\lineto(196.83203125,396.04687664)
\lineto(186.91796875,396.04687664)
\curveto(187.01171508,394.56249707)(187.45702714,393.42968571)(188.25390625,392.64843914)
\curveto(189.05858804,391.87499976)(190.17577442,391.4882814)(191.60546875,391.48828289)
\curveto(192.43358466,391.4882814)(193.23436511,391.5898438)(194.0078125,391.79297039)
\curveto(194.78905105,391.99609339)(195.56248778,392.30078058)(196.328125,392.70703289)
\lineto(196.328125,390.66797039)
\curveto(195.55467529,390.33984505)(194.76170733,390.0898453)(193.94921875,389.91797039)
\curveto(193.13670896,389.74609564)(192.31249103,389.66015823)(191.4765625,389.66015789)
\curveto(189.38280646,389.66015823)(187.72265187,390.26953262)(186.49609375,391.48828289)
\curveto(185.27734182,392.70703018)(184.66796743,394.35546603)(184.66796875,396.43359539)
\curveto(184.66796743,398.5820243)(185.24609185,400.2851476)(186.40234375,401.54297039)
\curveto(187.56640203,402.80858258)(189.13280671,403.44139444)(191.1015625,403.44140789)
\curveto(192.86717798,403.44139444)(194.26170783,402.87108251)(195.28515625,401.73047039)
\curveto(196.31639328,400.59764729)(196.83201776,399.05468008)(196.83203125,397.10156414)
\moveto(194.67578125,397.73437664)
\curveto(194.66014493,398.91405522)(194.32811402,399.85546053)(193.6796875,400.55859539)
\curveto(193.0390528,401.26170912)(192.18749116,401.61327127)(191.125,401.61328289)
\curveto(189.92186842,401.61327127)(188.95702564,401.27342786)(188.23046875,400.59375164)
\curveto(187.51171458,399.91405422)(187.0976525,398.95702393)(186.98828125,397.72265789)
\lineto(194.67578125,397.73437664)
}
}
{
\newrgbcolor{curcolor}{0 0 0}
\pscustom[linestyle=none,fillstyle=solid,fillcolor=curcolor]
{
\newpath
\moveto(200.1484375,395.17968914)
\lineto(200.1484375,403.12500164)
\lineto(202.3046875,403.12500164)
\lineto(202.3046875,395.26172039)
\curveto(202.3046833,394.01952887)(202.54687056,393.08593605)(203.03125,392.46093914)
\curveto(203.51561959,391.84374979)(204.24218137,391.53515635)(205.2109375,391.53515789)
\curveto(206.37499173,391.53515635)(207.29295957,391.90624973)(207.96484375,392.64843914)
\curveto(208.64452071,393.39062325)(208.98436413,394.40234098)(208.984375,395.68359539)
\lineto(208.984375,403.12500164)
\lineto(211.140625,403.12500164)
\lineto(211.140625,390.00000164)
\lineto(208.984375,390.00000164)
\lineto(208.984375,392.01562664)
\curveto(208.46092715,391.21875042)(207.85155276,390.62500101)(207.15625,390.23437664)
\curveto(206.46874164,389.85156428)(205.66796119,389.66015823)(204.75390625,389.66015789)
\curveto(203.24608861,389.66015823)(202.10155851,390.12890776)(201.3203125,391.06640789)
\curveto(200.53906007,392.00390588)(200.14843546,393.37499826)(200.1484375,395.17968914)
\moveto(205.57421875,403.44140789)
\lineto(205.57421875,403.44140789)
}
}
{
\newrgbcolor{curcolor}{0 0 0}
\pscustom[linestyle=none,fillstyle=solid,fillcolor=curcolor]
{
\newpath
\moveto(223.2109375,401.10937664)
\curveto(222.96874038,401.24999039)(222.70311564,401.35155278)(222.4140625,401.41406414)
\curveto(222.13280371,401.48436515)(221.82030402,401.51952137)(221.4765625,401.51953289)
\curveto(220.25780559,401.51952137)(219.32030652,401.12108426)(218.6640625,400.32422039)
\curveto(218.01562033,399.53514835)(217.6914019,398.39843074)(217.69140625,396.91406414)
\lineto(217.69140625,390.00000164)
\lineto(215.5234375,390.00000164)
\lineto(215.5234375,403.12500164)
\lineto(217.69140625,403.12500164)
\lineto(217.69140625,401.08593914)
\curveto(218.14452645,401.88280225)(218.73436961,402.47264541)(219.4609375,402.85547039)
\curveto(220.18749316,403.24608214)(221.07030477,403.44139444)(222.109375,403.44140789)
\curveto(222.25780359,403.44139444)(222.42186592,403.42967571)(222.6015625,403.40625164)
\curveto(222.78124056,403.39061325)(222.98045911,403.36326952)(223.19921875,403.32422039)
\lineto(223.2109375,401.10937664)
}
}
{
\newrgbcolor{curcolor}{0 0 0}
\pscustom[linestyle=none,fillstyle=solid,fillcolor=curcolor]
{
}
}
{
\newrgbcolor{curcolor}{0 0 0}
\pscustom[linestyle=none,fillstyle=solid,fillcolor=curcolor]
{
\newpath
\moveto(233.13671875,403.12500164)
\lineto(235.29296875,403.12500164)
\lineto(235.29296875,389.76562664)
\curveto(235.29296433,388.09375354)(234.97265215,386.88281725)(234.33203125,386.13281414)
\curveto(233.69921593,385.38281875)(232.67577945,385.00781913)(231.26171875,385.00781414)
\lineto(230.44140625,385.00781414)
\lineto(230.44140625,386.83593914)
\lineto(231.015625,386.83593914)
\curveto(231.83593654,386.8359423)(232.39452973,387.02734836)(232.69140625,387.41015789)
\curveto(232.98827914,387.7851601)(233.13671649,388.57031557)(233.13671875,389.76562664)
\lineto(233.13671875,403.12500164)
\moveto(233.13671875,408.23437664)
\lineto(235.29296875,408.23437664)
\lineto(235.29296875,405.50390789)
\lineto(233.13671875,405.50390789)
\lineto(233.13671875,408.23437664)
}
}
{
\newrgbcolor{curcolor}{0 0 0}
\pscustom[linestyle=none,fillstyle=solid,fillcolor=curcolor]
{
\newpath
\moveto(244.87890625,401.61328289)
\curveto(243.72265006,401.61327127)(242.80858847,401.16014673)(242.13671875,400.25390789)
\curveto(241.46483982,399.35546103)(241.12890265,398.12108726)(241.12890625,396.55078289)
\curveto(241.12890265,394.98046541)(241.46093357,393.74218539)(242.125,392.83593914)
\curveto(242.79686973,391.9374997)(243.71483757,391.4882814)(244.87890625,391.48828289)
\curveto(246.02733525,391.4882814)(246.93749059,391.94140594)(247.609375,392.84765789)
\curveto(248.28123925,393.75390413)(248.61717641,394.9882779)(248.6171875,396.55078289)
\curveto(248.61717641,398.10546228)(248.28123925,399.3359298)(247.609375,400.24218914)
\curveto(246.93749059,401.15624048)(246.02733525,401.61327127)(244.87890625,401.61328289)
\moveto(244.87890625,403.44140789)
\curveto(246.75389703,403.44139444)(248.2265518,402.83202005)(249.296875,401.61328289)
\curveto(250.36717466,400.39452249)(250.90233038,398.70702418)(250.90234375,396.55078289)
\curveto(250.90233038,394.40234098)(250.36717466,392.71484267)(249.296875,391.48828289)
\curveto(248.2265518,390.26953262)(246.75389703,389.66015823)(244.87890625,389.66015789)
\curveto(242.99608829,389.66015823)(241.51952726,390.26953262)(240.44921875,391.48828289)
\curveto(239.38671689,392.71484267)(238.85546743,394.40234098)(238.85546875,396.55078289)
\curveto(238.85546743,398.70702418)(239.38671689,400.39452249)(240.44921875,401.61328289)
\curveto(241.51952726,402.83202005)(242.99608829,403.44139444)(244.87890625,403.44140789)
}
}
{
\newrgbcolor{curcolor}{0 0 0}
\pscustom[linestyle=none,fillstyle=solid,fillcolor=curcolor]
{
\newpath
\moveto(254.2421875,395.17968914)
\lineto(254.2421875,403.12500164)
\lineto(256.3984375,403.12500164)
\lineto(256.3984375,395.26172039)
\curveto(256.3984333,394.01952887)(256.64062056,393.08593605)(257.125,392.46093914)
\curveto(257.60936959,391.84374979)(258.33593137,391.53515635)(259.3046875,391.53515789)
\curveto(260.46874173,391.53515635)(261.38670957,391.90624973)(262.05859375,392.64843914)
\curveto(262.73827071,393.39062325)(263.07811413,394.40234098)(263.078125,395.68359539)
\lineto(263.078125,403.12500164)
\lineto(265.234375,403.12500164)
\lineto(265.234375,390.00000164)
\lineto(263.078125,390.00000164)
\lineto(263.078125,392.01562664)
\curveto(262.55467715,391.21875042)(261.94530276,390.62500101)(261.25,390.23437664)
\curveto(260.56249164,389.85156428)(259.76171119,389.66015823)(258.84765625,389.66015789)
\curveto(257.33983861,389.66015823)(256.19530851,390.12890776)(255.4140625,391.06640789)
\curveto(254.63281007,392.00390588)(254.24218546,393.37499826)(254.2421875,395.17968914)
\moveto(259.66796875,403.44140789)
\lineto(259.66796875,403.44140789)
}
}
{
\newrgbcolor{curcolor}{0 0 0}
\pscustom[linestyle=none,fillstyle=solid,fillcolor=curcolor]
{
\newpath
\moveto(280.92578125,397.10156414)
\lineto(280.92578125,396.04687664)
\lineto(271.01171875,396.04687664)
\curveto(271.10546508,394.56249707)(271.55077714,393.42968571)(272.34765625,392.64843914)
\curveto(273.15233804,391.87499976)(274.26952442,391.4882814)(275.69921875,391.48828289)
\curveto(276.52733466,391.4882814)(277.32811511,391.5898438)(278.1015625,391.79297039)
\curveto(278.88280105,391.99609339)(279.65623778,392.30078058)(280.421875,392.70703289)
\lineto(280.421875,390.66797039)
\curveto(279.64842529,390.33984505)(278.85545733,390.0898453)(278.04296875,389.91797039)
\curveto(277.23045896,389.74609564)(276.40624103,389.66015823)(275.5703125,389.66015789)
\curveto(273.47655646,389.66015823)(271.81640187,390.26953262)(270.58984375,391.48828289)
\curveto(269.37109182,392.70703018)(268.76171743,394.35546603)(268.76171875,396.43359539)
\curveto(268.76171743,398.5820243)(269.33984185,400.2851476)(270.49609375,401.54297039)
\curveto(271.66015203,402.80858258)(273.22655671,403.44139444)(275.1953125,403.44140789)
\curveto(276.96092798,403.44139444)(278.35545783,402.87108251)(279.37890625,401.73047039)
\curveto(280.41014328,400.59764729)(280.92576776,399.05468008)(280.92578125,397.10156414)
\moveto(278.76953125,397.73437664)
\curveto(278.75389493,398.91405522)(278.42186402,399.85546053)(277.7734375,400.55859539)
\curveto(277.1328028,401.26170912)(276.28124116,401.61327127)(275.21875,401.61328289)
\curveto(274.01561842,401.61327127)(273.05077564,401.27342786)(272.32421875,400.59375164)
\curveto(271.60546458,399.91405422)(271.1914025,398.95702393)(271.08203125,397.72265789)
\lineto(278.76953125,397.73437664)
}
}
{
\newrgbcolor{curcolor}{0 0 0}
\pscustom[linestyle=none,fillstyle=solid,fillcolor=curcolor]
{
\newpath
\moveto(284.2421875,395.17968914)
\lineto(284.2421875,403.12500164)
\lineto(286.3984375,403.12500164)
\lineto(286.3984375,395.26172039)
\curveto(286.3984333,394.01952887)(286.64062056,393.08593605)(287.125,392.46093914)
\curveto(287.60936959,391.84374979)(288.33593137,391.53515635)(289.3046875,391.53515789)
\curveto(290.46874173,391.53515635)(291.38670957,391.90624973)(292.05859375,392.64843914)
\curveto(292.73827071,393.39062325)(293.07811413,394.40234098)(293.078125,395.68359539)
\lineto(293.078125,403.12500164)
\lineto(295.234375,403.12500164)
\lineto(295.234375,390.00000164)
\lineto(293.078125,390.00000164)
\lineto(293.078125,392.01562664)
\curveto(292.55467715,391.21875042)(291.94530276,390.62500101)(291.25,390.23437664)
\curveto(290.56249164,389.85156428)(289.76171119,389.66015823)(288.84765625,389.66015789)
\curveto(287.33983861,389.66015823)(286.19530851,390.12890776)(285.4140625,391.06640789)
\curveto(284.63281007,392.00390588)(284.24218546,393.37499826)(284.2421875,395.17968914)
\moveto(289.66796875,403.44140789)
\lineto(289.66796875,403.44140789)
}
}
{
\newrgbcolor{curcolor}{0 0 0}
\pscustom[linestyle=none,fillstyle=solid,fillcolor=curcolor]
{
\newpath
\moveto(307.3046875,401.10937664)
\curveto(307.06249037,401.24999039)(306.79686564,401.35155278)(306.5078125,401.41406414)
\curveto(306.22655371,401.48436515)(305.91405402,401.51952137)(305.5703125,401.51953289)
\curveto(304.35155559,401.51952137)(303.41405652,401.12108426)(302.7578125,400.32422039)
\curveto(302.10937033,399.53514835)(301.7851519,398.39843074)(301.78515625,396.91406414)
\lineto(301.78515625,390.00000164)
\lineto(299.6171875,390.00000164)
\lineto(299.6171875,403.12500164)
\lineto(301.78515625,403.12500164)
\lineto(301.78515625,401.08593914)
\curveto(302.23827645,401.88280225)(302.82811961,402.47264541)(303.5546875,402.85547039)
\curveto(304.28124316,403.24608214)(305.16405477,403.44139444)(306.203125,403.44140789)
\curveto(306.35155359,403.44139444)(306.51561592,403.42967571)(306.6953125,403.40625164)
\curveto(306.87499056,403.39061325)(307.07420911,403.36326952)(307.29296875,403.32422039)
\lineto(307.3046875,401.10937664)
}
}
{
\newrgbcolor{curcolor}{0 0 0}
\pscustom[linestyle=none,fillstyle=solid,fillcolor=curcolor,opacity=0.11935484]
{
\newpath
\moveto(325.28426552,409.91473552)
\lineto(414.86308098,409.91473552)
\curveto(423.31624407,409.91473552)(430.12149811,403.10948148)(430.12149811,394.65631839)
\curveto(430.12149811,386.2031553)(423.31624407,379.39790126)(414.86308098,379.39790126)
\lineto(325.28426552,379.39790126)
\curveto(316.83110243,379.39790126)(310.02584839,386.2031553)(310.02584839,394.65631839)
\curveto(310.02584839,403.10948148)(316.83110243,409.91473552)(325.28426552,409.91473552)
\closepath
}
}
{
\newrgbcolor{curcolor}{0 0 0}
\pscustom[linewidth=2,linecolor=curcolor]
{
\newpath
\moveto(325.28426552,409.91473552)
\lineto(414.86308098,409.91473552)
\curveto(423.31624407,409.91473552)(430.12149811,403.10948148)(430.12149811,394.65631839)
\curveto(430.12149811,386.2031553)(423.31624407,379.39790126)(414.86308098,379.39790126)
\lineto(325.28426552,379.39790126)
\curveto(316.83110243,379.39790126)(310.02584839,386.2031553)(310.02584839,394.65631839)
\curveto(310.02584839,403.10948148)(316.83110243,409.91473552)(325.28426552,409.91473552)
\closepath
}
}
{
\newrgbcolor{curcolor}{0 0 0}
\pscustom[linestyle=none,fillstyle=solid,fillcolor=curcolor]
{
\newpath
\moveto(144.984375,489.74218723)
\lineto(144.984375,487.04882785)
\curveto(143.93618398,487.55011239)(142.94725268,487.92380993)(142.01757812,488.1699216)
\curveto(141.08787954,488.41599694)(140.19009398,488.53904369)(139.32421875,488.53906223)
\curveto(137.82030468,488.53904369)(136.65819647,488.24737731)(135.83789062,487.66406223)
\curveto(135.02668768,487.08071181)(134.62108913,486.25128556)(134.62109375,485.17578098)
\curveto(134.62108913,484.27342295)(134.88996907,483.58982989)(135.42773438,483.12499973)
\curveto(135.9746034,482.66925789)(137.00455029,482.30011764)(138.51757812,482.01757785)
\lineto(140.18554688,481.67578098)
\curveto(142.24543046,481.28384261)(143.76300707,480.59113497)(144.73828125,479.59765598)
\curveto(145.72264053,478.61327237)(146.21482754,477.2916591)(146.21484375,475.63281223)
\curveto(146.21482754,473.65494399)(145.54946362,472.15559653)(144.21875,471.13476535)
\curveto(142.89712252,470.11393191)(140.95571821,469.60351575)(138.39453125,469.60351535)
\curveto(137.42837799,469.60351575)(136.3984311,469.71289064)(135.3046875,469.93164035)
\curveto(134.22004786,470.1503902)(133.09439795,470.47395759)(131.92773438,470.90234348)
\lineto(131.92773438,473.74609348)
\curveto(133.04882508,473.11718411)(134.14713127,472.64322625)(135.22265625,472.32421848)
\curveto(136.29817079,472.00520606)(137.35546139,471.84570101)(138.39453125,471.84570285)
\curveto(139.9713442,471.84570101)(141.18813985,472.15559653)(142.04492188,472.77539035)
\curveto(142.90167981,473.39517862)(143.33006479,474.27929232)(143.33007812,475.4277341)
\curveto(143.33006479,476.43033184)(143.02016927,477.21418522)(142.40039062,477.7792966)
\curveto(141.78970175,478.34439243)(140.7825413,478.76822013)(139.37890625,479.05078098)
\lineto(137.69726562,479.37890598)
\curveto(135.63736415,479.78905244)(134.14713127,480.43162992)(133.2265625,481.30664035)
\curveto(132.30598728,482.18162817)(131.84570128,483.39842383)(131.84570312,484.95703098)
\curveto(131.84570128,486.76170172)(132.47916419,488.18357529)(133.74609375,489.22265598)
\curveto(135.02213039,490.26169822)(136.77668593,490.78122895)(139.00976562,490.78124973)
\curveto(139.96678691,490.78122895)(140.94204635,490.69464049)(141.93554688,490.5214841)
\curveto(142.92902353,490.34828667)(143.94529855,490.08852131)(144.984375,489.74218723)
}
}
{
\newrgbcolor{curcolor}{0 0 0}
\pscustom[linestyle=none,fillstyle=solid,fillcolor=curcolor]
{
\newpath
\moveto(156.34570312,483.54882785)
\curveto(154.99673757,483.5488143)(153.93033238,483.020169)(153.14648438,481.96289035)
\curveto(152.36262562,480.91470236)(151.97069893,479.47459963)(151.97070312,477.64257785)
\curveto(151.97069893,475.81054079)(152.35806833,474.36588078)(153.1328125,473.30859348)
\curveto(153.91666052,472.26041413)(154.98762299,471.73632612)(156.34570312,471.73632785)
\curveto(157.68553696,471.73632612)(158.74738486,472.26497142)(159.53125,473.32226535)
\curveto(160.31509163,474.37955264)(160.70701832,475.81965537)(160.70703125,477.64257785)
\curveto(160.70701832,479.45637048)(160.31509163,480.89191592)(159.53125,481.94921848)
\curveto(158.74738486,483.01561171)(157.68553696,483.5488143)(156.34570312,483.54882785)
\moveto(156.34570312,485.68164035)
\curveto(158.53319237,485.68162467)(160.25128961,484.97068788)(161.5,483.54882785)
\curveto(162.74868294,482.12694073)(163.37303128,480.1581927)(163.37304688,477.64257785)
\curveto(163.37303128,475.1360623)(162.74868294,473.16731427)(161.5,471.73632785)
\curveto(160.25128961,470.31445254)(158.53319237,469.60351575)(156.34570312,469.60351535)
\curveto(154.14908217,469.60351575)(152.42642764,470.31445254)(151.17773438,471.73632785)
\curveto(149.93814888,473.16731427)(149.31835783,475.1360623)(149.31835938,477.64257785)
\curveto(149.31835783,480.1581927)(149.93814888,482.12694073)(151.17773438,483.54882785)
\curveto(152.42642764,484.97068788)(154.14908217,485.68162467)(156.34570312,485.68164035)
}
}
{
\newrgbcolor{curcolor}{0 0 0}
\pscustom[linestyle=none,fillstyle=solid,fillcolor=curcolor]
{
\newpath
\moveto(167.52929688,491.27343723)
\lineto(170.04492188,491.27343723)
\lineto(170.04492188,469.99999973)
\lineto(167.52929688,469.99999973)
\lineto(167.52929688,491.27343723)
}
}
{
\newrgbcolor{curcolor}{0 0 0}
\pscustom[linestyle=none,fillstyle=solid,fillcolor=curcolor]
{
\newpath
\moveto(181.22851562,483.54882785)
\curveto(179.87955007,483.5488143)(178.81314488,483.020169)(178.02929688,481.96289035)
\curveto(177.24543812,480.91470236)(176.85351143,479.47459963)(176.85351562,477.64257785)
\curveto(176.85351143,475.81054079)(177.24088083,474.36588078)(178.015625,473.30859348)
\curveto(178.79947302,472.26041413)(179.87043549,471.73632612)(181.22851562,471.73632785)
\curveto(182.56834946,471.73632612)(183.63019736,472.26497142)(184.4140625,473.32226535)
\curveto(185.19790413,474.37955264)(185.58983082,475.81965537)(185.58984375,477.64257785)
\curveto(185.58983082,479.45637048)(185.19790413,480.89191592)(184.4140625,481.94921848)
\curveto(183.63019736,483.01561171)(182.56834946,483.5488143)(181.22851562,483.54882785)
\moveto(181.22851562,485.68164035)
\curveto(183.41600487,485.68162467)(185.13410211,484.97068788)(186.3828125,483.54882785)
\curveto(187.63149544,482.12694073)(188.25584378,480.1581927)(188.25585938,477.64257785)
\curveto(188.25584378,475.1360623)(187.63149544,473.16731427)(186.3828125,471.73632785)
\curveto(185.13410211,470.31445254)(183.41600487,469.60351575)(181.22851562,469.60351535)
\curveto(179.03189467,469.60351575)(177.30924014,470.31445254)(176.06054688,471.73632785)
\curveto(174.82096138,473.16731427)(174.20117033,475.1360623)(174.20117188,477.64257785)
\curveto(174.20117033,480.1581927)(174.82096138,482.12694073)(176.06054688,483.54882785)
\curveto(177.30924014,484.97068788)(179.03189467,485.68162467)(181.22851562,485.68164035)
}
}
{
\newrgbcolor{curcolor}{1 1 1}
\pscustom[linestyle=none,fillstyle=solid,fillcolor=curcolor]
{
\newpath
\moveto(304.34478474,228.68957683)
\lineto(405.68919659,228.68957683)
\curveto(413.63620733,228.68957683)(420.03398132,222.29180284)(420.03398132,214.34479209)
\curveto(420.03398132,206.39778135)(413.63620733,200.00000736)(405.68919659,200.00000736)
\lineto(304.34478474,200.00000736)
\curveto(296.39777399,200.00000736)(290,206.39778135)(290,214.34479209)
\curveto(290,222.29180284)(296.39777399,228.68957683)(304.34478474,228.68957683)
\closepath
}
}
{
\newrgbcolor{curcolor}{0 0 0}
\pscustom[linewidth=2,linecolor=curcolor]
{
\newpath
\moveto(304.34478474,228.68957683)
\lineto(405.68919659,228.68957683)
\curveto(413.63620733,228.68957683)(420.03398132,222.29180284)(420.03398132,214.34479209)
\curveto(420.03398132,206.39778135)(413.63620733,200.00000736)(405.68919659,200.00000736)
\lineto(304.34478474,200.00000736)
\curveto(296.39777399,200.00000736)(290,206.39778135)(290,214.34479209)
\curveto(290,222.29180284)(296.39777399,228.68957683)(304.34478474,228.68957683)
\closepath
}
}
{
\newrgbcolor{curcolor}{0 0 0}
\pscustom[linestyle=none,fillstyle=solid,fillcolor=curcolor]
{
\newpath
\moveto(312.33959961,226.20153972)
\lineto(315.86694336,226.20153972)
\lineto(320.33178711,214.29528972)
\lineto(324.82006836,226.20153972)
\lineto(328.34741211,226.20153972)
\lineto(328.34741211,208.70544597)
\lineto(326.03881836,208.70544597)
\lineto(326.03881836,224.06872722)
\lineto(321.52709961,212.06872722)
\lineto(319.14819336,212.06872722)
\lineto(314.63647461,224.06872722)
\lineto(314.63647461,208.70544597)
\lineto(312.33959961,208.70544597)
\lineto(312.33959961,226.20153972)
}
}
{
\newrgbcolor{curcolor}{0 0 0}
\pscustom[linestyle=none,fillstyle=solid,fillcolor=curcolor]
{
\newpath
\moveto(338.05053711,220.31872722)
\curveto(336.89428092,220.31871561)(335.98021933,219.86559106)(335.30834961,218.95935222)
\curveto(334.63647068,218.06090537)(334.30053351,216.8265316)(334.30053711,215.25622722)
\curveto(334.30053351,213.68590974)(334.63256443,212.44762973)(335.29663086,211.54138347)
\curveto(335.96850059,210.64294403)(336.88646843,210.19372573)(338.05053711,210.19372722)
\curveto(339.19896611,210.19372573)(340.10912145,210.64685028)(340.78100586,211.55310222)
\curveto(341.45287011,212.45934847)(341.78880727,213.69372223)(341.78881836,215.25622722)
\curveto(341.78880727,216.81090662)(341.45287011,218.04137414)(340.78100586,218.94763347)
\curveto(340.10912145,219.86168482)(339.19896611,220.31871561)(338.05053711,220.31872722)
\moveto(338.05053711,222.14685222)
\curveto(339.92552789,222.14683878)(341.39818266,221.53746439)(342.46850586,220.31872722)
\curveto(343.53880552,219.09996683)(344.07396124,217.41246851)(344.07397461,215.25622722)
\curveto(344.07396124,213.10778532)(343.53880552,211.42028701)(342.46850586,210.19372722)
\curveto(341.39818266,208.97497695)(339.92552789,208.36560256)(338.05053711,208.36560222)
\curveto(336.16771914,208.36560256)(334.69115812,208.97497695)(333.62084961,210.19372722)
\curveto(332.55834775,211.42028701)(332.02709829,213.10778532)(332.02709961,215.25622722)
\curveto(332.02709829,217.41246851)(332.55834775,219.09996683)(333.62084961,220.31872722)
\curveto(334.69115812,221.53746439)(336.16771914,222.14683878)(338.05053711,222.14685222)
}
}
{
\newrgbcolor{curcolor}{0 0 0}
\pscustom[linestyle=none,fillstyle=solid,fillcolor=curcolor]
{
\newpath
\moveto(356.27319336,219.83825847)
\lineto(356.27319336,226.93982097)
\lineto(358.42944336,226.93982097)
\lineto(358.42944336,208.70544597)
\lineto(356.27319336,208.70544597)
\lineto(356.27319336,210.67419597)
\curveto(355.82005791,209.89294478)(355.24583974,209.31091412)(354.55053711,208.92810222)
\curveto(353.86302862,208.55310237)(353.03490445,208.36560256)(352.06616211,208.36560222)
\curveto(350.4802195,208.36560256)(349.18725205,208.99841443)(348.18725586,210.26403972)
\curveto(347.19506654,211.5296619)(346.69897329,213.19372273)(346.69897461,215.25622722)
\curveto(346.69897329,217.31871861)(347.19506654,218.98277944)(348.18725586,220.24841472)
\curveto(349.18725205,221.51402691)(350.4802195,222.14683878)(352.06616211,222.14685222)
\curveto(353.03490445,222.14683878)(353.86302862,221.95543272)(354.55053711,221.57263347)
\curveto(355.24583974,221.19762098)(355.82005791,220.61949656)(356.27319336,219.83825847)
\moveto(348.92553711,215.25622722)
\curveto(348.92553356,213.67028476)(349.24975198,212.42419225)(349.89819336,211.51794597)
\curveto(350.55443818,210.61950656)(351.45287478,210.17028826)(352.59350586,210.17028972)
\curveto(353.7341225,210.17028826)(354.6325591,210.61950656)(355.28881836,211.51794597)
\curveto(355.94505779,212.42419225)(356.27318246,213.67028476)(356.27319336,215.25622722)
\curveto(356.27318246,216.84215658)(355.94505779,218.08434284)(355.28881836,218.98278972)
\curveto(354.6325591,219.88902854)(353.7341225,220.34215308)(352.59350586,220.34216472)
\curveto(351.45287478,220.34215308)(350.55443818,219.88902854)(349.89819336,218.98278972)
\curveto(349.24975198,218.08434284)(348.92553356,216.84215658)(348.92553711,215.25622722)
}
}
{
\newrgbcolor{curcolor}{0 0 0}
\pscustom[linestyle=none,fillstyle=solid,fillcolor=curcolor]
{
\newpath
\moveto(362.87084961,221.83044597)
\lineto(365.02709961,221.83044597)
\lineto(365.02709961,208.70544597)
\lineto(362.87084961,208.70544597)
\lineto(362.87084961,221.83044597)
\moveto(362.87084961,226.93982097)
\lineto(365.02709961,226.93982097)
\lineto(365.02709961,224.20935222)
\lineto(362.87084961,224.20935222)
\lineto(362.87084961,226.93982097)
}
}
{
\newrgbcolor{curcolor}{0 0 0}
\pscustom[linestyle=none,fillstyle=solid,fillcolor=curcolor]
{
\newpath
\moveto(380.13256836,221.83044597)
\lineto(380.13256836,208.70544597)
\lineto(377.96459961,208.70544597)
\lineto(377.96459961,220.15466472)
\lineto(372.04663086,220.15466472)
\lineto(372.04663086,208.70544597)
\lineto(369.87866211,208.70544597)
\lineto(369.87866211,220.15466472)
\lineto(367.81616211,220.15466472)
\lineto(367.81616211,221.83044597)
\lineto(369.87866211,221.83044597)
\lineto(369.87866211,222.74450847)
\curveto(369.8786595,224.1741805)(370.21459666,225.22886695)(370.88647461,225.90857097)
\curveto(371.56615781,226.59605308)(372.60521927,226.93980274)(374.00366211,226.93982097)
\lineto(376.17163086,226.93982097)
\lineto(376.17163086,225.14685222)
\lineto(374.10913086,225.14685222)
\curveto(373.33568729,225.14683578)(372.79662533,224.99058594)(372.49194336,224.67810222)
\curveto(372.19506343,224.36558656)(372.04662608,223.80308712)(372.04663086,222.99060222)
\lineto(372.04663086,221.83044597)
\lineto(380.13256836,221.83044597)
\moveto(377.96459961,226.91638347)
\lineto(380.13256836,226.91638347)
\lineto(380.13256836,224.18591472)
\lineto(377.96459961,224.18591472)
\lineto(377.96459961,226.91638347)
}
}
{
\newrgbcolor{curcolor}{0 0 0}
\pscustom[linestyle=none,fillstyle=solid,fillcolor=curcolor]
{
\newpath
\moveto(395.89428711,215.80700847)
\lineto(395.89428711,214.75232097)
\lineto(385.98022461,214.75232097)
\curveto(386.07397094,213.26794141)(386.519283,212.13513004)(387.31616211,211.35388347)
\curveto(388.12084389,210.5804441)(389.23803028,210.19372573)(390.66772461,210.19372722)
\curveto(391.49584052,210.19372573)(392.29662097,210.29528813)(393.07006836,210.49841472)
\curveto(393.85130691,210.70153773)(394.62474364,211.00622492)(395.39038086,211.41247722)
\lineto(395.39038086,209.37341472)
\curveto(394.61693115,209.04528938)(393.82396319,208.79528963)(393.01147461,208.62341472)
\curveto(392.19896482,208.45153998)(391.37474689,208.36560256)(390.53881836,208.36560222)
\curveto(388.44506232,208.36560256)(386.78490773,208.97497695)(385.55834961,210.19372722)
\curveto(384.33959768,211.41247451)(383.73022329,213.06091037)(383.73022461,215.13903972)
\curveto(383.73022329,217.28746864)(384.30834771,218.99059194)(385.46459961,220.24841472)
\curveto(386.62865789,221.51402691)(388.19506257,222.14683878)(390.16381836,222.14685222)
\curveto(391.92943384,222.14683878)(393.32396369,221.57652685)(394.34741211,220.43591472)
\curveto(395.37864914,219.30309162)(395.89427362,217.76012442)(395.89428711,215.80700847)
\moveto(393.73803711,216.43982097)
\curveto(393.72240079,217.61949956)(393.39036987,218.56090487)(392.74194336,219.26403972)
\curveto(392.10130866,219.96715346)(391.24974702,220.31871561)(390.18725586,220.31872722)
\curveto(388.98412428,220.31871561)(388.0192815,219.9788722)(387.29272461,219.29919597)
\curveto(386.57397044,218.61949856)(386.15990836,217.66246826)(386.05053711,216.42810222)
\lineto(393.73803711,216.43982097)
}
}
{
\newrgbcolor{curcolor}{0 0 0}
\pscustom[linestyle=none,fillstyle=solid,fillcolor=curcolor]
{
\newpath
\moveto(407.03881836,219.81482097)
\curveto(406.79662123,219.95543472)(406.5309965,220.05699712)(406.24194336,220.11950847)
\curveto(405.96068457,220.18980949)(405.64818488,220.2249657)(405.30444336,220.22497722)
\curveto(404.08568645,220.2249657)(403.14818738,219.8265286)(402.49194336,219.02966472)
\curveto(401.84350119,218.24059269)(401.51928276,217.10387507)(401.51928711,215.61950847)
\lineto(401.51928711,208.70544597)
\lineto(399.35131836,208.70544597)
\lineto(399.35131836,221.83044597)
\lineto(401.51928711,221.83044597)
\lineto(401.51928711,219.79138347)
\curveto(401.97240731,220.58824659)(402.56225047,221.17808975)(403.28881836,221.56091472)
\curveto(404.01537402,221.95152648)(404.89818563,222.14683878)(405.93725586,222.14685222)
\curveto(406.08568445,222.14683878)(406.24974678,222.13512004)(406.42944336,222.11169597)
\curveto(406.60912142,222.09605758)(406.80833997,222.06871386)(407.02709961,222.02966472)
\lineto(407.03881836,219.81482097)
}
}
{
\newrgbcolor{curcolor}{0 0 0}
\pscustom[linestyle=none,fillstyle=solid,fillcolor=curcolor]
{
\newpath
\moveto(43.96875,532.65625164)
\lineto(49.125,532.65625164)
\lineto(49.125,550.45312664)
\lineto(43.515625,549.32812664)
\lineto(43.515625,552.20312664)
\lineto(49.09375,553.32812664)
\lineto(52.25,553.32812664)
\lineto(52.25,532.65625164)
\lineto(57.40625,532.65625164)
\lineto(57.40625,530.00000164)
\lineto(43.96875,530.00000164)
\lineto(43.96875,532.65625164)
}
}
{
\newrgbcolor{curcolor}{0 0 0}
\pscustom[linestyle=none,fillstyle=solid,fillcolor=curcolor]
{
\newpath
\moveto(506.140625,442.65624973)
\lineto(517.15625,442.65624973)
\lineto(517.15625,439.99999973)
\lineto(502.34375,439.99999973)
\lineto(502.34375,442.65624973)
\curveto(503.54166313,443.89582917)(505.17186983,445.55728584)(507.234375,447.64062473)
\curveto(509.30728236,449.73436499)(510.60936439,451.08332198)(511.140625,451.68749973)
\curveto(512.15102952,452.82290357)(512.85415381,453.78123595)(513.25,454.56249973)
\curveto(513.65623634,455.35415104)(513.85936114,456.13019193)(513.859375,456.89062473)
\curveto(513.85936114,458.13018993)(513.42186158,459.14060559)(512.546875,459.92187473)
\curveto(511.68227998,460.70310403)(510.55207278,461.09372863)(509.15625,461.09374973)
\curveto(508.1666585,461.09372863)(507.11978455,460.92185381)(506.015625,460.57812473)
\curveto(504.92187008,460.23435449)(503.74999625,459.71352168)(502.5,459.01562473)
\lineto(502.5,462.20312473)
\curveto(503.77082956,462.71351868)(504.95832838,463.09893496)(506.0625,463.35937473)
\curveto(507.1666595,463.61976778)(508.17707516,463.74997598)(509.09375,463.74999973)
\curveto(511.51040516,463.74997598)(513.43748656,463.14580992)(514.875,461.93749973)
\curveto(516.31248369,460.72914567)(517.03123297,459.11456395)(517.03125,457.09374973)
\curveto(517.03123297,456.13540026)(516.84894148,455.22394284)(516.484375,454.35937473)
\curveto(516.1301922,453.50519456)(515.47915119,452.4947789)(514.53125,451.32812473)
\curveto(514.27081906,451.02603037)(513.44269489,450.15103124)(512.046875,448.70312473)
\curveto(510.65103102,447.26561746)(508.68228298,445.24999448)(506.140625,442.65624973)
}
}
{
\newrgbcolor{curcolor}{0 0 0}
\pscustom[linestyle=none,fillstyle=solid,fillcolor=curcolor]
{
\newpath
\moveto(512.984375,402.57812664)
\curveto(514.49477717,402.25519771)(515.67185933,401.58332339)(516.515625,400.56250164)
\curveto(517.3697743,399.54165876)(517.7968572,398.28124335)(517.796875,396.78125164)
\curveto(517.7968572,394.47916382)(517.00519133,392.6979156)(515.421875,391.43750164)
\curveto(513.83852783,390.17708479)(511.58853008,389.54687709)(508.671875,389.54687664)
\curveto(507.69270064,389.54687709)(506.68228498,389.64583532)(505.640625,389.84375164)
\curveto(504.60937039,390.0312516)(503.54166313,390.31770965)(502.4375,390.70312664)
\lineto(502.4375,393.75000164)
\curveto(503.31249669,393.23958173)(504.27082906,392.85416545)(505.3125,392.59375164)
\curveto(506.35416031,392.33333264)(507.44270089,392.20312443)(508.578125,392.20312664)
\curveto(510.55728111,392.20312443)(512.06248794,392.59374904)(513.09375,393.37500164)
\curveto(514.13540253,394.15624748)(514.65623534,395.29166301)(514.65625,396.78125164)
\curveto(514.65623534,398.15624348)(514.17186083,399.22915907)(513.203125,400.00000164)
\curveto(512.24477942,400.78124085)(510.90623909,401.17186546)(509.1875,401.17187664)
\lineto(506.46875,401.17187664)
\lineto(506.46875,403.76562664)
\lineto(509.3125,403.76562664)
\curveto(510.86457247,403.76561287)(512.05207128,404.07290423)(512.875,404.68750164)
\curveto(513.69790297,405.31248632)(514.10936089,406.20831876)(514.109375,407.37500164)
\curveto(514.10936089,408.57289973)(513.68227798,409.48956548)(512.828125,410.12500164)
\curveto(511.98436302,410.7708142)(510.77082256,411.09373054)(509.1875,411.09375164)
\curveto(508.32290834,411.09373054)(507.39582594,410.99998064)(506.40625,410.81250164)
\curveto(505.41666125,410.62498101)(504.32812067,410.33331464)(503.140625,409.93750164)
\lineto(503.140625,412.75000164)
\curveto(504.33853733,413.08331189)(505.45832788,413.33331164)(506.5,413.50000164)
\curveto(507.55207578,413.66664464)(508.54165813,413.74997789)(509.46875,413.75000164)
\curveto(511.86457147,413.74997789)(513.76040291,413.20310343)(515.15625,412.10937664)
\curveto(516.55206678,411.02602228)(517.24998275,409.55727375)(517.25,407.70312664)
\curveto(517.24998275,406.41144356)(516.88019145,405.31769465)(516.140625,404.42187664)
\curveto(515.40102627,403.53644643)(514.34894398,402.92186371)(512.984375,402.57812664)
}
}
{
\newrgbcolor{curcolor}{0 0 0}
\pscustom[linestyle=none,fillstyle=solid,fillcolor=curcolor]
{
\newpath
\moveto(512.09375,290.57812664)
\lineto(504.125,278.12500164)
\lineto(512.09375,278.12500164)
\lineto(512.09375,290.57812664)
\moveto(511.265625,293.32812664)
\lineto(515.234375,293.32812664)
\lineto(515.234375,278.12500164)
\lineto(518.5625,278.12500164)
\lineto(518.5625,275.50000164)
\lineto(515.234375,275.50000164)
\lineto(515.234375,270.00000164)
\lineto(512.09375,270.00000164)
\lineto(512.09375,275.50000164)
\lineto(501.5625,275.50000164)
\lineto(501.5625,278.54687664)
\lineto(511.265625,293.32812664)
}
}
{
\newrgbcolor{curcolor}{0 0 0}
\pscustom[linestyle=none,fillstyle=solid,fillcolor=curcolor]
{
\newpath
\moveto(503.453125,233.32812664)
\lineto(515.84375,233.32812664)
\lineto(515.84375,230.67187664)
\lineto(506.34375,230.67187664)
\lineto(506.34375,224.95312664)
\curveto(506.80207653,225.10936153)(507.26040941,225.22394475)(507.71875,225.29687664)
\curveto(508.17707516,225.38019459)(508.63540803,225.42186121)(509.09375,225.42187664)
\curveto(511.69790497,225.42186121)(513.76040291,224.70832026)(515.28125,223.28125164)
\curveto(516.80206653,221.85415645)(517.56248244,219.92186671)(517.5625,217.48437664)
\curveto(517.56248244,214.973955)(516.78123322,213.02083195)(515.21875,211.62500164)
\curveto(513.65623634,210.23958473)(511.45311355,209.54687709)(508.609375,209.54687664)
\curveto(507.6302007,209.54687709)(506.6302017,209.63021034)(505.609375,209.79687664)
\curveto(504.59895373,209.96354334)(503.55207978,210.21354309)(502.46875,210.54687664)
\lineto(502.46875,213.71875164)
\curveto(503.40624659,213.20833176)(504.37499562,212.82812381)(505.375,212.57812664)
\curveto(506.37499363,212.32812431)(507.43228423,212.20312443)(508.546875,212.20312664)
\curveto(510.34894798,212.20312443)(511.77602989,212.67708229)(512.828125,213.62500164)
\curveto(513.88019445,214.57291373)(514.40623559,215.85937078)(514.40625,217.48437664)
\curveto(514.40623559,219.10936753)(513.88019445,220.39582457)(512.828125,221.34375164)
\curveto(511.77602989,222.29165601)(510.34894798,222.76561387)(508.546875,222.76562664)
\curveto(507.7031173,222.76561387)(506.85936814,222.67186396)(506.015625,222.48437664)
\curveto(505.18228648,222.29686434)(504.32812067,222.00519796)(503.453125,221.60937664)
\lineto(503.453125,233.32812664)
}
}
{
\newrgbcolor{curcolor}{0 0 0}
\pscustom[linestyle=none,fillstyle=solid,fillcolor=curcolor]
{
\newpath
\moveto(510.5625,172.92186138)
\curveto(509.14582419,172.92184846)(508.02082531,172.43747394)(507.1875,171.46873638)
\curveto(506.36457697,170.49997588)(505.95311905,169.17185221)(505.953125,167.48436138)
\curveto(505.95311905,165.80727224)(506.36457697,164.47914856)(507.1875,163.49998638)
\curveto(508.02082531,162.53123385)(509.14582419,162.04685933)(510.5625,162.04686138)
\curveto(511.97915469,162.04685933)(513.09894523,162.53123385)(513.921875,163.49998638)
\curveto(514.75519358,164.47914856)(515.17185983,165.80727224)(515.171875,167.48436138)
\curveto(515.17185983,169.17185221)(514.75519358,170.49997588)(513.921875,171.46873638)
\curveto(513.09894523,172.43747394)(511.97915469,172.92184846)(510.5625,172.92186138)
\moveto(516.828125,182.81248638)
\lineto(516.828125,179.93748638)
\curveto(516.0364423,180.31246606)(515.23435977,180.59892411)(514.421875,180.79686138)
\curveto(513.61977805,180.99475705)(512.82290384,181.09371528)(512.03125,181.09373638)
\curveto(509.94790672,181.09371528)(508.35415831,180.39059099)(507.25,178.98436138)
\curveto(506.15624384,177.5780938)(505.53124447,175.45309592)(505.375,172.60936138)
\curveto(505.98957734,173.51559786)(506.76040991,174.2083055)(507.6875,174.68748638)
\curveto(508.61457472,175.17705453)(509.63540703,175.42184596)(510.75,175.42186138)
\curveto(513.09373691,175.42184596)(514.94269339,174.708305)(516.296875,173.28123638)
\curveto(517.66144067,171.86455785)(518.34373166,169.93226811)(518.34375,167.48436138)
\curveto(518.34373166,165.08852296)(517.63539903,163.16664988)(516.21875,161.71873638)
\curveto(514.80206853,160.27081944)(512.91665375,159.54686183)(510.5625,159.54686138)
\curveto(507.86457547,159.54686183)(505.80207753,160.5781108)(504.375,162.64061138)
\curveto(502.94791372,164.71352333)(502.23437277,167.71352033)(502.234375,171.64061138)
\curveto(502.23437277,175.32809605)(503.10937189,178.26559311)(504.859375,180.45311138)
\curveto(506.60936839,182.65100539)(508.95832438,183.74996263)(511.90625,183.74998638)
\curveto(512.69790397,183.74996263)(513.49477817,183.67183771)(514.296875,183.51561138)
\curveto(515.10935989,183.35933802)(515.95310905,183.12496325)(516.828125,182.81248638)
}
}
{
\newrgbcolor{curcolor}{0 0 0}
\pscustom[linestyle=none,fillstyle=solid,fillcolor=curcolor]
{
\newpath
\moveto(502.625,123.32811138)
\lineto(517.625,123.32811138)
\lineto(517.625,121.98436138)
\lineto(509.15625,99.99998638)
\lineto(505.859375,99.99998638)
\lineto(513.828125,120.67186138)
\lineto(502.625,120.67186138)
\lineto(502.625,123.32811138)
}
}
{
\newrgbcolor{curcolor}{0 0 0}
\pscustom[linestyle=none,fillstyle=solid,fillcolor=curcolor]
{
\newpath
\moveto(50.171875,41.07811138)
\curveto(48.67186633,41.0781003)(47.48957584,40.67705903)(46.625,39.87498638)
\curveto(45.77082756,39.07289397)(45.34374466,37.96872841)(45.34375,36.56248638)
\curveto(45.34374466,35.15623122)(45.77082756,34.05206566)(46.625,33.24998638)
\curveto(47.48957584,32.4479006)(48.67186633,32.04685933)(50.171875,32.04686138)
\curveto(51.67186333,32.04685933)(52.85415381,32.4479006)(53.71875,33.24998638)
\curveto(54.58331875,34.06248231)(55.01560998,35.16664788)(55.015625,36.56248638)
\curveto(55.01560998,37.96872841)(54.58331875,39.07289397)(53.71875,39.87498638)
\curveto(52.86457047,40.67705903)(51.68227998,41.0781003)(50.171875,41.07811138)
\moveto(47.015625,42.42186138)
\curveto(45.66145267,42.75518196)(44.60416206,43.38538966)(43.84375,44.31248638)
\curveto(43.09374691,45.23955447)(42.71874728,46.36976167)(42.71875,47.70311138)
\curveto(42.71874728,49.56767514)(43.38020495,51.041632)(44.703125,52.12498638)
\curveto(46.0364523,53.2082965)(47.85936714,53.74996263)(50.171875,53.74998638)
\curveto(52.49477917,53.74996263)(54.31769402,53.2082965)(55.640625,52.12498638)
\curveto(56.9635247,51.041632)(57.62498238,49.56767514)(57.625,47.70311138)
\curveto(57.62498238,46.36976167)(57.24477442,45.23955447)(56.484375,44.31248638)
\curveto(55.73435927,43.38538966)(54.68748531,42.75518196)(53.34375,42.42186138)
\curveto(54.86456847,42.06768264)(56.04685895,41.374975)(56.890625,40.34373638)
\curveto(57.74477392,39.31247706)(58.17185683,38.05206166)(58.171875,36.56248638)
\curveto(58.17185683,34.30206541)(57.47914919,32.56769214)(56.09375,31.35936138)
\curveto(54.71873528,30.15102789)(52.74477892,29.54686183)(50.171875,29.54686138)
\curveto(47.59895073,29.54686183)(45.61978605,30.15102789)(44.234375,31.35936138)
\curveto(42.85937214,32.56769214)(42.17187283,34.30206541)(42.171875,36.56248638)
\curveto(42.17187283,38.05206166)(42.59895573,39.31247706)(43.453125,40.34373638)
\curveto(44.30728736,41.374975)(45.49478617,42.06768264)(47.015625,42.42186138)
\moveto(45.859375,47.40623638)
\curveto(45.85936914,46.19788685)(46.23436877,45.25517946)(46.984375,44.57811138)
\curveto(47.74478392,43.90101414)(48.80728286,43.56247281)(50.171875,43.56248638)
\curveto(51.52603014,43.56247281)(52.58332075,43.90101414)(53.34375,44.57811138)
\curveto(54.11456922,45.25517946)(54.4999855,46.19788685)(54.5,47.40623638)
\curveto(54.4999855,48.6145511)(54.11456922,49.55725849)(53.34375,50.23436138)
\curveto(52.58332075,50.9114238)(51.52603014,51.24996513)(50.171875,51.24998638)
\curveto(48.80728286,51.24996513)(47.74478392,50.9114238)(46.984375,50.23436138)
\curveto(46.23436877,49.55725849)(45.85936914,48.6145511)(45.859375,47.40623638)
}
}
{
\newrgbcolor{curcolor}{0 0 0}
\pscustom[linestyle=none,fillstyle=solid,fillcolor=curcolor]
{
\newpath
\moveto(145.45703125,176.14842388)
\lineto(145.45703125,173.65233013)
\curveto(144.66014159,174.39450323)(143.80857994,174.94919018)(142.90234375,175.31639263)
\curveto(142.00389425,175.68356444)(141.04686395,175.86715801)(140.03125,175.86717388)
\curveto(138.03124197,175.86715801)(136.4999935,175.25387737)(135.4375,174.02733013)
\curveto(134.37499562,172.80856732)(133.84374616,171.04294408)(133.84375,168.73045513)
\curveto(133.84374616,166.4257612)(134.37499562,164.66013797)(135.4375,163.43358013)
\curveto(136.4999935,162.21482791)(138.03124197,161.60545352)(140.03125,161.60545513)
\curveto(141.04686395,161.60545352)(142.00389425,161.78904709)(142.90234375,162.15623638)
\curveto(143.80857994,162.52342135)(144.66014159,163.0781083)(145.45703125,163.82029888)
\lineto(145.45703125,161.34764263)
\curveto(144.62889162,160.78514184)(143.74998625,160.36326726)(142.8203125,160.08201763)
\curveto(141.8984256,159.80076783)(140.92186408,159.66014297)(139.890625,159.66014263)
\curveto(137.24218026,159.66014297)(135.15624484,160.46873591)(133.6328125,162.08592388)
\curveto(132.10937289,163.71092017)(131.3476549,165.9257617)(131.34765625,168.73045513)
\curveto(131.3476549,171.54294358)(132.10937289,173.75778512)(133.6328125,175.37498638)
\curveto(135.15624484,176.99996938)(137.24218026,177.81246856)(139.890625,177.81248638)
\curveto(140.93748906,177.81246856)(141.92186308,177.67184371)(142.84375,177.39061138)
\curveto(143.77342373,177.11715676)(144.64451661,176.70309467)(145.45703125,176.14842388)
}
}
{
\newrgbcolor{curcolor}{0 0 0}
\pscustom[linestyle=none,fillstyle=solid,fillcolor=curcolor]
{
\newpath
\moveto(154.12890625,171.61326763)
\curveto(152.97265006,171.61325601)(152.05858847,171.16013147)(151.38671875,170.25389263)
\curveto(150.71483982,169.35544577)(150.37890265,168.12107201)(150.37890625,166.55076763)
\curveto(150.37890265,164.98045015)(150.71093357,163.74217013)(151.375,162.83592388)
\curveto(152.04686973,161.93748444)(152.96483757,161.48826614)(154.12890625,161.48826763)
\curveto(155.27733525,161.48826614)(156.18749059,161.94139069)(156.859375,162.84764263)
\curveto(157.53123925,163.75388887)(157.86717641,164.98826264)(157.8671875,166.55076763)
\curveto(157.86717641,168.10544702)(157.53123925,169.33591454)(156.859375,170.24217388)
\curveto(156.18749059,171.15622522)(155.27733525,171.61325601)(154.12890625,171.61326763)
\moveto(154.12890625,173.44139263)
\curveto(156.00389703,173.44137919)(157.4765518,172.83200479)(158.546875,171.61326763)
\curveto(159.61717466,170.39450723)(160.15233038,168.70700892)(160.15234375,166.55076763)
\curveto(160.15233038,164.40232572)(159.61717466,162.71482741)(158.546875,161.48826763)
\curveto(157.4765518,160.26951736)(156.00389703,159.66014297)(154.12890625,159.66014263)
\curveto(152.24608829,159.66014297)(150.76952726,160.26951736)(149.69921875,161.48826763)
\curveto(148.63671689,162.71482741)(148.10546743,164.40232572)(148.10546875,166.55076763)
\curveto(148.10546743,168.70700892)(148.63671689,170.39450723)(149.69921875,171.61326763)
\curveto(150.76952726,172.83200479)(152.24608829,173.44137919)(154.12890625,173.44139263)
}
}
{
\newrgbcolor{curcolor}{0 0 0}
\pscustom[linestyle=none,fillstyle=solid,fillcolor=curcolor]
{
\newpath
\moveto(174.625,167.92186138)
\lineto(174.625,159.99998638)
\lineto(172.46875,159.99998638)
\lineto(172.46875,167.85154888)
\curveto(172.46873898,169.09372728)(172.22655173,170.02341385)(171.7421875,170.64061138)
\curveto(171.2578027,171.25778762)(170.53124092,171.56638106)(169.5625,171.56639263)
\curveto(168.39843055,171.56638106)(167.48046272,171.19528768)(166.80859375,170.45311138)
\curveto(166.13671407,169.71091417)(165.8007769,168.69919643)(165.80078125,167.41795513)
\lineto(165.80078125,159.99998638)
\lineto(163.6328125,159.99998638)
\lineto(163.6328125,173.12498638)
\lineto(165.80078125,173.12498638)
\lineto(165.80078125,171.08592388)
\curveto(166.31640139,171.8749745)(166.92186953,172.46481766)(167.6171875,172.85545513)
\curveto(168.32030563,173.24606688)(169.12889857,173.44137919)(170.04296875,173.44139263)
\curveto(171.55077115,173.44137919)(172.69139501,172.97262965)(173.46484375,172.03514263)
\curveto(174.23826846,171.10544402)(174.62498683,169.73435164)(174.625,167.92186138)
}
}
{
\newrgbcolor{curcolor}{0 0 0}
\pscustom[linestyle=none,fillstyle=solid,fillcolor=curcolor]
{
\newpath
\moveto(189.859375,167.92186138)
\lineto(189.859375,159.99998638)
\lineto(187.703125,159.99998638)
\lineto(187.703125,167.85154888)
\curveto(187.70311398,169.09372728)(187.46092673,170.02341385)(186.9765625,170.64061138)
\curveto(186.4921777,171.25778762)(185.76561592,171.56638106)(184.796875,171.56639263)
\curveto(183.63280555,171.56638106)(182.71483772,171.19528768)(182.04296875,170.45311138)
\curveto(181.37108907,169.71091417)(181.0351519,168.69919643)(181.03515625,167.41795513)
\lineto(181.03515625,159.99998638)
\lineto(178.8671875,159.99998638)
\lineto(178.8671875,173.12498638)
\lineto(181.03515625,173.12498638)
\lineto(181.03515625,171.08592388)
\curveto(181.55077639,171.8749745)(182.15624453,172.46481766)(182.8515625,172.85545513)
\curveto(183.55468063,173.24606688)(184.36327357,173.44137919)(185.27734375,173.44139263)
\curveto(186.78514615,173.44137919)(187.92577001,172.97262965)(188.69921875,172.03514263)
\curveto(189.47264346,171.10544402)(189.85936183,169.73435164)(189.859375,167.92186138)
}
}
{
\newrgbcolor{curcolor}{0 0 0}
\pscustom[linestyle=none,fillstyle=solid,fillcolor=curcolor]
{
\newpath
\moveto(205.41015625,167.10154888)
\lineto(205.41015625,166.04686138)
\lineto(195.49609375,166.04686138)
\curveto(195.58984008,164.56248181)(196.03515214,163.42967045)(196.83203125,162.64842388)
\curveto(197.63671304,161.8749845)(198.75389942,161.48826614)(200.18359375,161.48826763)
\curveto(201.01170966,161.48826614)(201.81249011,161.58982854)(202.5859375,161.79295513)
\curveto(203.36717605,161.99607813)(204.14061278,162.30076533)(204.90625,162.70701763)
\lineto(204.90625,160.66795513)
\curveto(204.13280029,160.33982979)(203.33983233,160.08983004)(202.52734375,159.91795513)
\curveto(201.71483396,159.74608038)(200.89061603,159.66014297)(200.0546875,159.66014263)
\curveto(197.96093146,159.66014297)(196.30077687,160.26951736)(195.07421875,161.48826763)
\curveto(193.85546682,162.70701492)(193.24609243,164.35545077)(193.24609375,166.43358013)
\curveto(193.24609243,168.58200904)(193.82421685,170.28513234)(194.98046875,171.54295513)
\curveto(196.14452703,172.80856732)(197.71093171,173.44137919)(199.6796875,173.44139263)
\curveto(201.44530298,173.44137919)(202.83983283,172.87106726)(203.86328125,171.73045513)
\curveto(204.89451828,170.59763203)(205.41014276,169.05466482)(205.41015625,167.10154888)
\moveto(203.25390625,167.73436138)
\curveto(203.23826993,168.91403996)(202.90623902,169.85544527)(202.2578125,170.55858013)
\curveto(201.6171778,171.26169387)(200.76561616,171.61325601)(199.703125,171.61326763)
\curveto(198.49999342,171.61325601)(197.53515064,171.2734126)(196.80859375,170.59373638)
\curveto(196.08983958,169.91403896)(195.6757775,168.95700867)(195.56640625,167.72264263)
\lineto(203.25390625,167.73436138)
}
}
{
\newrgbcolor{curcolor}{0 0 0}
\pscustom[linestyle=none,fillstyle=solid,fillcolor=curcolor]
{
\newpath
\moveto(219.4375,173.12498638)
\lineto(214.69140625,166.73826763)
\lineto(219.68359375,159.99998638)
\lineto(217.140625,159.99998638)
\lineto(213.3203125,165.15623638)
\lineto(209.5,159.99998638)
\lineto(206.95703125,159.99998638)
\lineto(212.0546875,166.86717388)
\lineto(207.390625,173.12498638)
\lineto(209.93359375,173.12498638)
\lineto(213.4140625,168.44920513)
\lineto(216.89453125,173.12498638)
\lineto(219.4375,173.12498638)
}
}
{
\newrgbcolor{curcolor}{0 0 0}
\pscustom[linestyle=none,fillstyle=solid,fillcolor=curcolor]
{
\newpath
\moveto(222.73046875,173.12498638)
\lineto(224.88671875,173.12498638)
\lineto(224.88671875,159.99998638)
\lineto(222.73046875,159.99998638)
\lineto(222.73046875,173.12498638)
\moveto(222.73046875,178.23436138)
\lineto(224.88671875,178.23436138)
\lineto(224.88671875,175.50389263)
\lineto(222.73046875,175.50389263)
\lineto(222.73046875,178.23436138)
}
}
{
\newrgbcolor{curcolor}{0 0 0}
\pscustom[linestyle=none,fillstyle=solid,fillcolor=curcolor]
{
\newpath
\moveto(234.47265625,171.61326763)
\curveto(233.31640006,171.61325601)(232.40233847,171.16013147)(231.73046875,170.25389263)
\curveto(231.05858982,169.35544577)(230.72265265,168.12107201)(230.72265625,166.55076763)
\curveto(230.72265265,164.98045015)(231.05468357,163.74217013)(231.71875,162.83592388)
\curveto(232.39061973,161.93748444)(233.30858757,161.48826614)(234.47265625,161.48826763)
\curveto(235.62108525,161.48826614)(236.53124059,161.94139069)(237.203125,162.84764263)
\curveto(237.87498925,163.75388887)(238.21092641,164.98826264)(238.2109375,166.55076763)
\curveto(238.21092641,168.10544702)(237.87498925,169.33591454)(237.203125,170.24217388)
\curveto(236.53124059,171.15622522)(235.62108525,171.61325601)(234.47265625,171.61326763)
\moveto(234.47265625,173.44139263)
\curveto(236.34764703,173.44137919)(237.8203018,172.83200479)(238.890625,171.61326763)
\curveto(239.96092466,170.39450723)(240.49608038,168.70700892)(240.49609375,166.55076763)
\curveto(240.49608038,164.40232572)(239.96092466,162.71482741)(238.890625,161.48826763)
\curveto(237.8203018,160.26951736)(236.34764703,159.66014297)(234.47265625,159.66014263)
\curveto(232.58983829,159.66014297)(231.11327726,160.26951736)(230.04296875,161.48826763)
\curveto(228.98046689,162.71482741)(228.44921743,164.40232572)(228.44921875,166.55076763)
\curveto(228.44921743,168.70700892)(228.98046689,170.39450723)(230.04296875,171.61326763)
\curveto(231.11327726,172.83200479)(232.58983829,173.44137919)(234.47265625,173.44139263)
}
}
{
\newrgbcolor{curcolor}{0 0 0}
\pscustom[linestyle=none,fillstyle=solid,fillcolor=curcolor]
{
\newpath
\moveto(254.96875,167.92186138)
\lineto(254.96875,159.99998638)
\lineto(252.8125,159.99998638)
\lineto(252.8125,167.85154888)
\curveto(252.81248898,169.09372728)(252.57030173,170.02341385)(252.0859375,170.64061138)
\curveto(251.6015527,171.25778762)(250.87499092,171.56638106)(249.90625,171.56639263)
\curveto(248.74218055,171.56638106)(247.82421272,171.19528768)(247.15234375,170.45311138)
\curveto(246.48046407,169.71091417)(246.1445269,168.69919643)(246.14453125,167.41795513)
\lineto(246.14453125,159.99998638)
\lineto(243.9765625,159.99998638)
\lineto(243.9765625,173.12498638)
\lineto(246.14453125,173.12498638)
\lineto(246.14453125,171.08592388)
\curveto(246.66015139,171.8749745)(247.26561953,172.46481766)(247.9609375,172.85545513)
\curveto(248.66405563,173.24606688)(249.47264857,173.44137919)(250.38671875,173.44139263)
\curveto(251.89452115,173.44137919)(253.03514501,172.97262965)(253.80859375,172.03514263)
\curveto(254.58201846,171.10544402)(254.96873683,169.73435164)(254.96875,167.92186138)
}
}
{
\newrgbcolor{curcolor}{0 0 0}
\pscustom[linestyle=none,fillstyle=solid,fillcolor=curcolor]
{
}
}
{
\newrgbcolor{curcolor}{0 0 0}
\pscustom[linestyle=none,fillstyle=solid,fillcolor=curcolor]
{
\newpath
\moveto(272.8984375,166.59764263)
\curveto(271.15624352,166.59763603)(269.94921347,166.39841748)(269.27734375,165.99998638)
\curveto(268.60546482,165.60154328)(268.26952765,164.92185646)(268.26953125,163.96092388)
\curveto(268.26952765,163.19529568)(268.5195274,162.58592129)(269.01953125,162.13279888)
\curveto(269.52733889,161.68748469)(270.21483821,161.46482866)(271.08203125,161.46483013)
\curveto(272.27733614,161.46482866)(273.23436644,161.88670324)(273.953125,162.73045513)
\curveto(274.67967749,163.58201404)(275.04295838,164.71091917)(275.04296875,166.11717388)
\lineto(275.04296875,166.59764263)
\lineto(272.8984375,166.59764263)
\moveto(277.19921875,167.48826763)
\lineto(277.19921875,159.99998638)
\lineto(275.04296875,159.99998638)
\lineto(275.04296875,161.99217388)
\curveto(274.55077137,161.19529768)(273.93749073,160.60545452)(273.203125,160.22264263)
\curveto(272.4687422,159.84764278)(271.5703056,159.66014297)(270.5078125,159.66014263)
\curveto(269.16405801,159.66014297)(268.09374658,160.03514259)(267.296875,160.78514263)
\curveto(266.50781066,161.54295358)(266.11327981,162.55467132)(266.11328125,163.82029888)
\curveto(266.11327981,165.29685608)(266.60546682,166.41013622)(267.58984375,167.16014263)
\curveto(268.58202734,167.91013472)(270.05858836,168.28513434)(272.01953125,168.28514263)
\lineto(275.04296875,168.28514263)
\lineto(275.04296875,168.49608013)
\curveto(275.04295838,169.48825814)(274.71483371,170.25388237)(274.05859375,170.79295513)
\curveto(273.41014751,171.33981879)(272.49608593,171.61325601)(271.31640625,171.61326763)
\curveto(270.56640036,171.61325601)(269.83593234,171.52341235)(269.125,171.34373638)
\curveto(268.41405876,171.16403771)(267.73046569,170.89450673)(267.07421875,170.53514263)
\lineto(267.07421875,172.52733013)
\curveto(267.86327806,172.83200479)(268.62890229,173.05856707)(269.37109375,173.20701763)
\curveto(270.11327581,173.36325426)(270.83593134,173.44137919)(271.5390625,173.44139263)
\curveto(273.43749123,173.44137919)(274.85545857,172.94919218)(275.79296875,171.96483013)
\curveto(276.73045669,170.98044415)(277.19920622,169.48825814)(277.19921875,167.48826763)
}
}
{
\newrgbcolor{curcolor}{0 0 0}
\pscustom[linestyle=none,fillstyle=solid,fillcolor=curcolor]
{
\newpath
\moveto(281.4296875,165.17967388)
\lineto(281.4296875,173.12498638)
\lineto(283.5859375,173.12498638)
\lineto(283.5859375,165.26170513)
\curveto(283.5859333,164.01951361)(283.82812056,163.08592079)(284.3125,162.46092388)
\curveto(284.79686959,161.84373453)(285.52343137,161.53514109)(286.4921875,161.53514263)
\curveto(287.65624173,161.53514109)(288.57420957,161.90623447)(289.24609375,162.64842388)
\curveto(289.92577071,163.39060799)(290.26561413,164.40232572)(290.265625,165.68358013)
\lineto(290.265625,173.12498638)
\lineto(292.421875,173.12498638)
\lineto(292.421875,159.99998638)
\lineto(290.265625,159.99998638)
\lineto(290.265625,162.01561138)
\curveto(289.74217715,161.21873516)(289.13280276,160.62498575)(288.4375,160.23436138)
\curveto(287.74999164,159.85154903)(286.94921119,159.66014297)(286.03515625,159.66014263)
\curveto(284.52733861,159.66014297)(283.38280851,160.1288925)(282.6015625,161.06639263)
\curveto(281.82031007,162.00389062)(281.42968546,163.374983)(281.4296875,165.17967388)
\moveto(286.85546875,173.44139263)
\lineto(286.85546875,173.44139263)
}
}
{
\newrgbcolor{curcolor}{0 0 0}
\pscustom[linestyle=none,fillstyle=solid,fillcolor=curcolor]
{
\newpath
\moveto(299.01953125,176.85154888)
\lineto(299.01953125,173.12498638)
\lineto(303.4609375,173.12498638)
\lineto(303.4609375,171.44920513)
\lineto(299.01953125,171.44920513)
\lineto(299.01953125,164.32420513)
\curveto(299.01952686,163.25388937)(299.16405796,162.56639006)(299.453125,162.26170513)
\curveto(299.74999488,161.95701567)(300.34765053,161.80467207)(301.24609375,161.80467388)
\lineto(303.4609375,161.80467388)
\lineto(303.4609375,159.99998638)
\lineto(301.24609375,159.99998638)
\curveto(299.58202629,159.99998638)(298.43358994,160.30857982)(297.80078125,160.92576763)
\curveto(297.16796621,161.55076608)(296.85156027,162.68357744)(296.8515625,164.32420513)
\lineto(296.8515625,171.44920513)
\lineto(295.26953125,171.44920513)
\lineto(295.26953125,173.12498638)
\lineto(296.8515625,173.12498638)
\lineto(296.8515625,176.85154888)
\lineto(299.01953125,176.85154888)
}
}
{
\newrgbcolor{curcolor}{0 0 0}
\pscustom[linestyle=none,fillstyle=solid,fillcolor=curcolor]
{
\newpath
\moveto(311.39453125,171.61326763)
\curveto(310.23827506,171.61325601)(309.32421347,171.16013147)(308.65234375,170.25389263)
\curveto(307.98046482,169.35544577)(307.64452765,168.12107201)(307.64453125,166.55076763)
\curveto(307.64452765,164.98045015)(307.97655857,163.74217013)(308.640625,162.83592388)
\curveto(309.31249473,161.93748444)(310.23046257,161.48826614)(311.39453125,161.48826763)
\curveto(312.54296025,161.48826614)(313.45311559,161.94139069)(314.125,162.84764263)
\curveto(314.79686425,163.75388887)(315.13280141,164.98826264)(315.1328125,166.55076763)
\curveto(315.13280141,168.10544702)(314.79686425,169.33591454)(314.125,170.24217388)
\curveto(313.45311559,171.15622522)(312.54296025,171.61325601)(311.39453125,171.61326763)
\moveto(311.39453125,173.44139263)
\curveto(313.26952203,173.44137919)(314.7421768,172.83200479)(315.8125,171.61326763)
\curveto(316.88279966,170.39450723)(317.41795538,168.70700892)(317.41796875,166.55076763)
\curveto(317.41795538,164.40232572)(316.88279966,162.71482741)(315.8125,161.48826763)
\curveto(314.7421768,160.26951736)(313.26952203,159.66014297)(311.39453125,159.66014263)
\curveto(309.51171329,159.66014297)(308.03515226,160.26951736)(306.96484375,161.48826763)
\curveto(305.90234189,162.71482741)(305.37109243,164.40232572)(305.37109375,166.55076763)
\curveto(305.37109243,168.70700892)(305.90234189,170.39450723)(306.96484375,171.61326763)
\curveto(308.03515226,172.83200479)(309.51171329,173.44137919)(311.39453125,173.44139263)
}
}
{
\newrgbcolor{curcolor}{1 1 1}
\pscustom[linestyle=none,fillstyle=solid,fillcolor=curcolor]
{
\newpath
\moveto(330,179.99998638)
\lineto(350,179.99998638)
\lineto(350,159.99998638)
\lineto(330,159.99998638)
\closepath
}
}
{
\newrgbcolor{curcolor}{0 0 0}
\pscustom[linewidth=2,linecolor=curcolor]
{
\newpath
\moveto(330,179.99998638)
\lineto(350,179.99998638)
\lineto(350,159.99998638)
\lineto(330,159.99998638)
\closepath
}
}
{
\newrgbcolor{curcolor}{0 0 0}
\pscustom[linewidth=2,linecolor=curcolor,linestyle=dashed,dash=8 8]
{
\newpath
\moveto(150,540)
\lineto(60,540)
}
}
{
\newrgbcolor{curcolor}{0 0 0}
\pscustom[linestyle=none,fillstyle=solid,fillcolor=curcolor]
{
\newpath
\moveto(139.53769464,544.84048224)
\lineto(152.6487474,540.01921591)
\lineto(139.53769392,535.19795064)
\curveto(141.632292,538.04442372)(141.62022288,541.93889292)(139.53769464,544.84048224)
\lineto(139.53769464,544.84048224)
\closepath
}
}
{
\newrgbcolor{curcolor}{0 0 0}
\pscustom[linewidth=2,linecolor=curcolor,linestyle=dashed,dash=8 8]
{
\newpath
\moveto(400,450)
\lineto(500,450)
}
}
{
\newrgbcolor{curcolor}{0 0 0}
\pscustom[linestyle=none,fillstyle=solid,fillcolor=curcolor]
{
\newpath
\moveto(410.46230536,445.15951776)
\lineto(397.3512526,449.98078409)
\lineto(410.46230608,454.80204936)
\curveto(408.367708,451.95557628)(408.37977712,448.06110708)(410.46230536,445.15951776)
\lineto(410.46230536,445.15951776)
\closepath
}
}
{
\newrgbcolor{curcolor}{0 0 0}
\pscustom[linewidth=2,linecolor=curcolor,linestyle=dashed,dash=8 8]
{
\newpath
\moveto(400,400)
\lineto(500,400)
}
}
{
\newrgbcolor{curcolor}{0 0 0}
\pscustom[linestyle=none,fillstyle=solid,fillcolor=curcolor]
{
\newpath
\moveto(410.46230536,395.15951776)
\lineto(397.3512526,399.98078409)
\lineto(410.46230608,404.80204936)
\curveto(408.367708,401.95557628)(408.37977712,398.06110708)(410.46230536,395.15951776)
\lineto(410.46230536,395.15951776)
\closepath
}
}
{
\newrgbcolor{curcolor}{0 0 0}
\pscustom[linewidth=2,linecolor=curcolor,linestyle=dashed,dash=8 8]
{
\newpath
\moveto(400,280)
\lineto(500,280)
}
}
{
\newrgbcolor{curcolor}{0 0 0}
\pscustom[linestyle=none,fillstyle=solid,fillcolor=curcolor]
{
\newpath
\moveto(410.46230536,275.15951776)
\lineto(397.3512526,279.98078409)
\lineto(410.46230608,284.80204936)
\curveto(408.367708,281.95557628)(408.37977712,278.06110708)(410.46230536,275.15951776)
\lineto(410.46230536,275.15951776)
\closepath
}
}
{
\newrgbcolor{curcolor}{0 0 0}
\pscustom[linewidth=2,linecolor=curcolor,linestyle=dashed,dash=8 8]
{
\newpath
\moveto(400,220)
\lineto(500,220)
}
}
{
\newrgbcolor{curcolor}{0 0 0}
\pscustom[linestyle=none,fillstyle=solid,fillcolor=curcolor]
{
\newpath
\moveto(410.46230536,215.15951776)
\lineto(397.3512526,219.98078409)
\lineto(410.46230608,224.80204936)
\curveto(408.367708,221.95557628)(408.37977712,218.06110708)(410.46230536,215.15951776)
\lineto(410.46230536,215.15951776)
\closepath
}
}
{
\newrgbcolor{curcolor}{0 0 0}
\pscustom[linewidth=2,linecolor=curcolor,linestyle=dashed,dash=8 8]
{
\newpath
\moveto(380,110)
\lineto(500,110)
}
}
{
\newrgbcolor{curcolor}{0 0 0}
\pscustom[linestyle=none,fillstyle=solid,fillcolor=curcolor]
{
\newpath
\moveto(390.46230536,105.15951776)
\lineto(377.3512526,109.98078409)
\lineto(390.46230608,114.80204936)
\curveto(388.367708,111.95557628)(388.37977712,108.06110708)(390.46230536,105.15951776)
\lineto(390.46230536,105.15951776)
\closepath
}
}
{
\newrgbcolor{curcolor}{0 0 0}
\pscustom[linewidth=2,linecolor=curcolor,linestyle=dashed,dash=8 8]
{
\newpath
\moveto(420,40)
\lineto(500,40)
}
}
{
\newrgbcolor{curcolor}{0 0 0}
\pscustom[linestyle=none,fillstyle=solid,fillcolor=curcolor]
{
\newpath
\moveto(430.46230536,35.15951776)
\lineto(417.3512526,39.98078409)
\lineto(430.46230608,44.80204936)
\curveto(428.367708,41.95557628)(428.37977712,38.06110708)(430.46230536,35.15951776)
\lineto(430.46230536,35.15951776)
\closepath
}
}
{
\newrgbcolor{curcolor}{0 0 0}
\pscustom[linewidth=2,linecolor=curcolor,linestyle=dashed,dash=8 8]
{
\newpath
\moveto(150,40)
\lineto(60,40)
}
}
{
\newrgbcolor{curcolor}{0 0 0}
\pscustom[linestyle=none,fillstyle=solid,fillcolor=curcolor]
{
\newpath
\moveto(139.53769464,44.84048224)
\lineto(152.6487474,40.01921591)
\lineto(139.53769392,35.19795064)
\curveto(141.632292,38.04442372)(141.62022288,41.93889292)(139.53769464,44.84048224)
\lineto(139.53769464,44.84048224)
\closepath
}
}
{
\newrgbcolor{curcolor}{0 0 0}
\pscustom[linewidth=2,linecolor=curcolor,linestyle=dashed,dash=8 8]
{
\newpath
\moveto(340,170)
\lineto(500,170)
}
}
{
\newrgbcolor{curcolor}{0 0 0}
\pscustom[linestyle=none,fillstyle=solid,fillcolor=curcolor]
{
\newpath
\moveto(350.46230536,165.15951776)
\lineto(337.3512526,169.98078409)
\lineto(350.46230608,174.80204936)
\curveto(348.367708,171.95557628)(348.37977712,168.06110708)(350.46230536,165.15951776)
\lineto(350.46230536,165.15951776)
\closepath
}
}
{
\newrgbcolor{curcolor}{0 0 0}
\pscustom[linestyle=none,fillstyle=solid,fillcolor=curcolor]
{
\newpath
\moveto(503.515625,30.48436138)
\lineto(503.515625,33.35936138)
\curveto(504.30728736,32.98435839)(505.10936989,32.69790035)(505.921875,32.49998638)
\curveto(506.73436827,32.30206741)(507.53124247,32.20310917)(508.3125,32.20311138)
\curveto(510.39582294,32.20310917)(511.98436302,32.90102514)(513.078125,34.29686138)
\curveto(514.18227748,35.70310567)(514.81248519,37.83331188)(514.96875,40.68748638)
\curveto(514.36456897,39.79164325)(513.59894473,39.10414394)(512.671875,38.62498638)
\curveto(511.74477992,38.14581156)(510.71873928,37.90622847)(509.59375,37.90623638)
\curveto(507.26040941,37.90622847)(505.41145292,38.60935277)(504.046875,40.01561138)
\curveto(502.69270564,41.43226661)(502.01562298,43.36455635)(502.015625,45.81248638)
\curveto(502.01562298,48.2083015)(502.72395561,50.13017458)(504.140625,51.57811138)
\curveto(505.55728611,53.02600502)(507.44270089,53.74996263)(509.796875,53.74998638)
\curveto(512.49477917,53.74996263)(514.55206878,52.71350533)(515.96875,50.64061138)
\curveto(517.39581594,48.5780928)(518.10935689,45.5780958)(518.109375,41.64061138)
\curveto(518.10935689,37.96352008)(517.23435777,35.02602302)(515.484375,32.82811138)
\curveto(513.74477792,30.64061074)(511.40103027,29.54686183)(508.453125,29.54686138)
\curveto(507.66145067,29.54686183)(506.85936814,29.62498675)(506.046875,29.78123638)
\curveto(505.23436977,29.93748644)(504.39062061,30.17186121)(503.515625,30.48436138)
\moveto(509.796875,40.37498638)
\curveto(511.21353045,40.374976)(512.333321,40.85935052)(513.15625,41.82811138)
\curveto(513.98956934,42.79684858)(514.40623559,44.12497225)(514.40625,45.81248638)
\curveto(514.40623559,47.48955222)(513.98956934,48.81246756)(513.15625,49.78123638)
\curveto(512.333321,50.76038228)(511.21353045,51.24996513)(509.796875,51.24998638)
\curveto(508.38019995,51.24996513)(507.25520108,50.76038228)(506.421875,49.78123638)
\curveto(505.59895273,48.81246756)(505.18749481,47.48955222)(505.1875,45.81248638)
\curveto(505.18749481,44.12497225)(505.59895273,42.79684858)(506.421875,41.82811138)
\curveto(507.25520108,40.85935052)(508.38019995,40.374976)(509.796875,40.37498638)
}
}
\end{pspicture}

		
		\begin{enumerate}
		  \item Fond d'écran
		  \item Champs de texte ``Pseudo''
		  \item Liste déroulante ``Couleur joueur''
		  \item Champs de texte ``Nom d'utilisateur''
		  \item Bouton ``Modifier''
		  \item Liste déroulante ``Position menu de la partie''
		  \item Bouton ``\hyperlink{Options}{Valider}''
		  \item Bouton ``\hyperlink{Options}{Retour}''
		\end{enumerate}

		\subsubsection{Description des zones}
		
			\begin{tabular}{|c|c|c|c|c|} \hline
				Numéro de zone & Type  & Description & Evènement &	Règle \\\hline
			\end{tabular}
			
		\subsubsection{Description des règles}

			\underline{RG8-01 :}
				\begin{quote}
				
				\end{quote}
	
\newpage

	\subsection{Statistiques}
	
		%LaTeX with PSTricks extensions
%%Creator: inkscape 0.48.0
%%Please note this file requires PSTricks extensions
\psset{xunit=.5pt,yunit=.5pt,runit=.5pt}
\begin{pspicture}(560,600)
{
\newrgbcolor{curcolor}{1 1 1}
\pscustom[linestyle=none,fillstyle=solid,fillcolor=curcolor]
{
\newpath
\moveto(133.12401581,597.52220317)
\lineto(426.87598419,597.52220317)
\curveto(443.85397169,597.52220317)(457.52217102,583.85400385)(457.52217102,566.87601635)
\lineto(457.52217102,33.12401744)
\curveto(457.52217102,16.14602994)(443.85397169,2.47783062)(426.87598419,2.47783062)
\lineto(133.12401581,2.47783062)
\curveto(116.14602831,2.47783062)(102.47782898,16.14602994)(102.47782898,33.12401744)
\lineto(102.47782898,566.87601635)
\curveto(102.47782898,583.85400385)(116.14602831,597.52220317)(133.12401581,597.52220317)
\closepath
}
}
{
\newrgbcolor{curcolor}{0 0 0}
\pscustom[linewidth=4.95566034,linecolor=curcolor]
{
\newpath
\moveto(133.12401581,597.52220317)
\lineto(426.87598419,597.52220317)
\curveto(443.85397169,597.52220317)(457.52217102,583.85400385)(457.52217102,566.87601635)
\lineto(457.52217102,33.12401744)
\curveto(457.52217102,16.14602994)(443.85397169,2.47783062)(426.87598419,2.47783062)
\lineto(133.12401581,2.47783062)
\curveto(116.14602831,2.47783062)(102.47782898,16.14602994)(102.47782898,33.12401744)
\lineto(102.47782898,566.87601635)
\curveto(102.47782898,583.85400385)(116.14602831,597.52220317)(133.12401581,597.52220317)
\closepath
}
}
{
\newrgbcolor{curcolor}{1 1 1}
\pscustom[linestyle=none,fillstyle=solid,fillcolor=curcolor]
{
\newpath
\moveto(129.74061584,489.93649074)
\lineto(350.02220154,489.93649074)
\lineto(350.02220154,459.97780773)
\lineto(129.74061584,459.97780773)
\closepath
}
}
{
\newrgbcolor{curcolor}{0 0 0}
\pscustom[linewidth=1.67714477,linecolor=curcolor]
{
\newpath
\moveto(129.74061584,489.93649074)
\lineto(350.02220154,489.93649074)
\lineto(350.02220154,459.97780773)
\lineto(129.74061584,459.97780773)
\closepath
}
}
{
\newrgbcolor{curcolor}{1 1 1}
\pscustom[linestyle=none,fillstyle=solid,fillcolor=curcolor]
{
\newpath
\moveto(131.83036804,451.01639816)
\lineto(430.0433197,451.01639816)
\lineto(430.0433197,60.03657028)
\lineto(131.83036804,60.03657028)
\closepath
}
}
{
\newrgbcolor{curcolor}{0 0 0}
\pscustom[linewidth=1.94882679,linecolor=curcolor]
{
\newpath
\moveto(131.83036804,451.01639816)
\lineto(430.0433197,451.01639816)
\lineto(430.0433197,60.03657028)
\lineto(131.83036804,60.03657028)
\closepath
}
}
{
\newrgbcolor{curcolor}{0 0 0}
\pscustom[linewidth=1,linecolor=curcolor]
{
\newpath
\moveto(407.7131,319.65631)
\curveto(407.7131,320.1325)(407.7131,320.60869)(407.7131,319.65631)
\closepath
}
}
{
\newrgbcolor{curcolor}{0 0 0}
\pscustom[linestyle=none,fillstyle=solid,fillcolor=curcolor]
{
\newpath
\moveto(43.96875,542.65625164)
\lineto(49.125,542.65625164)
\lineto(49.125,560.45312664)
\lineto(43.515625,559.32812664)
\lineto(43.515625,562.20312664)
\lineto(49.09375,563.32812664)
\lineto(52.25,563.32812664)
\lineto(52.25,542.65625164)
\lineto(57.40625,542.65625164)
\lineto(57.40625,540.00000164)
\lineto(43.96875,540.00000164)
\lineto(43.96875,542.65625164)
}
}
{
\newrgbcolor{curcolor}{0 0 0}
\pscustom[linestyle=none,fillstyle=solid,fillcolor=curcolor]
{
\newpath
\moveto(46.140625,472.65624973)
\lineto(57.15625,472.65624973)
\lineto(57.15625,469.99999973)
\lineto(42.34375,469.99999973)
\lineto(42.34375,472.65624973)
\curveto(43.54166312,473.89582917)(45.17186983,475.55728584)(47.234375,477.64062473)
\curveto(49.30728236,479.73436499)(50.60936439,481.08332198)(51.140625,481.68749973)
\curveto(52.15102952,482.82290357)(52.85415381,483.78123595)(53.25,484.56249973)
\curveto(53.65623634,485.35415104)(53.85936114,486.13019193)(53.859375,486.89062473)
\curveto(53.85936114,488.13018993)(53.42186158,489.14060559)(52.546875,489.92187473)
\curveto(51.68227998,490.70310403)(50.55207278,491.09372863)(49.15625,491.09374973)
\curveto(48.1666585,491.09372863)(47.11978455,490.92185381)(46.015625,490.57812473)
\curveto(44.92187008,490.23435449)(43.74999625,489.71352168)(42.5,489.01562473)
\lineto(42.5,492.20312473)
\curveto(43.77082956,492.71351868)(44.95832838,493.09893496)(46.0625,493.35937473)
\curveto(47.1666595,493.61976778)(48.17707516,493.74997598)(49.09375,493.74999973)
\curveto(51.51040516,493.74997598)(53.43748656,493.14580992)(54.875,491.93749973)
\curveto(56.31248369,490.72914567)(57.03123297,489.11456395)(57.03125,487.09374973)
\curveto(57.03123297,486.13540026)(56.84894148,485.22394284)(56.484375,484.35937473)
\curveto(56.1301922,483.50519456)(55.47915119,482.4947789)(54.53125,481.32812473)
\curveto(54.27081906,481.02603037)(53.44269489,480.15103124)(52.046875,478.70312473)
\curveto(50.65103102,477.26561746)(48.68228298,475.24999448)(46.140625,472.65624973)
}
}
{
\newrgbcolor{curcolor}{0 0 0}
\pscustom[linestyle=none,fillstyle=solid,fillcolor=curcolor]
{
\newpath
\moveto(512.984375,402.57812664)
\curveto(514.49477717,402.25519771)(515.67185933,401.58332339)(516.515625,400.56250164)
\curveto(517.3697743,399.54165876)(517.7968572,398.28124335)(517.796875,396.78125164)
\curveto(517.7968572,394.47916382)(517.00519133,392.6979156)(515.421875,391.43750164)
\curveto(513.83852783,390.17708479)(511.58853008,389.54687709)(508.671875,389.54687664)
\curveto(507.69270064,389.54687709)(506.68228498,389.64583532)(505.640625,389.84375164)
\curveto(504.60937039,390.0312516)(503.54166313,390.31770965)(502.4375,390.70312664)
\lineto(502.4375,393.75000164)
\curveto(503.31249669,393.23958173)(504.27082906,392.85416545)(505.3125,392.59375164)
\curveto(506.35416031,392.33333264)(507.44270089,392.20312443)(508.578125,392.20312664)
\curveto(510.55728111,392.20312443)(512.06248794,392.59374904)(513.09375,393.37500164)
\curveto(514.13540253,394.15624748)(514.65623534,395.29166301)(514.65625,396.78125164)
\curveto(514.65623534,398.15624348)(514.17186083,399.22915907)(513.203125,400.00000164)
\curveto(512.24477942,400.78124085)(510.90623909,401.17186546)(509.1875,401.17187664)
\lineto(506.46875,401.17187664)
\lineto(506.46875,403.76562664)
\lineto(509.3125,403.76562664)
\curveto(510.86457247,403.76561287)(512.05207128,404.07290423)(512.875,404.68750164)
\curveto(513.69790297,405.31248632)(514.10936089,406.20831876)(514.109375,407.37500164)
\curveto(514.10936089,408.57289973)(513.68227798,409.48956548)(512.828125,410.12500164)
\curveto(511.98436302,410.7708142)(510.77082256,411.09373054)(509.1875,411.09375164)
\curveto(508.32290834,411.09373054)(507.39582594,410.99998064)(506.40625,410.81250164)
\curveto(505.41666125,410.62498101)(504.32812067,410.33331464)(503.140625,409.93750164)
\lineto(503.140625,412.75000164)
\curveto(504.33853733,413.08331189)(505.45832788,413.33331164)(506.5,413.50000164)
\curveto(507.55207578,413.66664464)(508.54165813,413.74997789)(509.46875,413.75000164)
\curveto(511.86457147,413.74997789)(513.76040291,413.20310343)(515.15625,412.10937664)
\curveto(516.55206678,411.02602228)(517.24998275,409.55727375)(517.25,407.70312664)
\curveto(517.24998275,406.41144356)(516.88019145,405.31769465)(516.140625,404.42187664)
\curveto(515.40102627,403.53644643)(514.34894398,402.92186371)(512.984375,402.57812664)
}
}
{
\newrgbcolor{curcolor}{1 1 1}
\pscustom[linestyle=none,fillstyle=solid,fillcolor=curcolor]
{
\newpath
\moveto(141.66666698,49.99998638)
\lineto(238.33333302,49.99998638)
\curveto(244.79666652,49.99998638)(250,44.7966529)(250,38.33331939)
\lineto(250,31.66665336)
\curveto(250,25.20331985)(244.79666652,19.99998638)(238.33333302,19.99998638)
\lineto(141.66666698,19.99998638)
\curveto(135.20333348,19.99998638)(130,25.20331985)(130,31.66665336)
\lineto(130,38.33331939)
\curveto(130,44.7966529)(135.20333348,49.99998638)(141.66666698,49.99998638)
\closepath
}
}
{
\newrgbcolor{curcolor}{0 0 0}
\pscustom[linewidth=2,linecolor=curcolor]
{
\newpath
\moveto(141.66666698,49.99998638)
\lineto(238.33333302,49.99998638)
\curveto(244.79666652,49.99998638)(250,44.7966529)(250,38.33331939)
\lineto(250,31.66665336)
\curveto(250,25.20331985)(244.79666652,19.99998638)(238.33333302,19.99998638)
\lineto(141.66666698,19.99998638)
\curveto(135.20333348,19.99998638)(130,25.20331985)(130,31.66665336)
\lineto(130,38.33331939)
\curveto(130,44.7966529)(135.20333348,49.99998638)(141.66666698,49.99998638)
\closepath
}
}
{
\newrgbcolor{curcolor}{0 0 0}
\pscustom[linestyle=none,fillstyle=solid,fillcolor=curcolor]
{
\newpath
\moveto(160.65234375,38.20311138)
\curveto(161.16014509,38.03122835)(161.6523321,37.66404121)(162.12890625,37.10154888)
\curveto(162.61326864,36.53904234)(163.09764315,35.76560561)(163.58203125,34.78123638)
\lineto(165.984375,29.99998638)
\lineto(163.44140625,29.99998638)
\lineto(161.203125,34.48826763)
\curveto(160.62498938,35.66013697)(160.06248994,36.43747994)(159.515625,36.82029888)
\curveto(158.97655352,37.20310417)(158.23827301,37.39451023)(157.30078125,37.39451763)
\lineto(154.72265625,37.39451763)
\lineto(154.72265625,29.99998638)
\lineto(152.35546875,29.99998638)
\lineto(152.35546875,47.49608013)
\lineto(157.69921875,47.49608013)
\curveto(159.69920905,47.49606263)(161.19139506,47.0780943)(162.17578125,46.24217388)
\curveto(163.16014309,45.40622097)(163.6523301,44.14450348)(163.65234375,42.45701763)
\curveto(163.6523301,41.35544377)(163.39451786,40.44138219)(162.87890625,39.71483013)
\curveto(162.37108138,38.98825864)(161.62889462,38.48435289)(160.65234375,38.20311138)
\moveto(154.72265625,45.55076763)
\lineto(154.72265625,39.33983013)
\lineto(157.69921875,39.33983013)
\curveto(158.83983491,39.33982079)(159.69920905,39.60153928)(160.27734375,40.12498638)
\curveto(160.86327039,40.65622572)(161.15623884,41.43356869)(161.15625,42.45701763)
\curveto(161.15623884,43.48044165)(160.86327039,44.24997213)(160.27734375,44.76561138)
\curveto(159.69920905,45.28903359)(158.83983491,45.55075208)(157.69921875,45.55076763)
\lineto(154.72265625,45.55076763)
}
}
{
\newrgbcolor{curcolor}{0 0 0}
\pscustom[linestyle=none,fillstyle=solid,fillcolor=curcolor]
{
\newpath
\moveto(179.09765625,37.10154888)
\lineto(179.09765625,36.04686138)
\lineto(169.18359375,36.04686138)
\curveto(169.27734008,34.56248181)(169.72265214,33.42967045)(170.51953125,32.64842388)
\curveto(171.32421304,31.8749845)(172.44139942,31.48826614)(173.87109375,31.48826763)
\curveto(174.69920966,31.48826614)(175.49999011,31.58982854)(176.2734375,31.79295513)
\curveto(177.05467605,31.99607813)(177.82811278,32.30076533)(178.59375,32.70701763)
\lineto(178.59375,30.66795513)
\curveto(177.82030029,30.33982979)(177.02733233,30.08983004)(176.21484375,29.91795513)
\curveto(175.40233396,29.74608038)(174.57811603,29.66014297)(173.7421875,29.66014263)
\curveto(171.64843146,29.66014297)(169.98827687,30.26951736)(168.76171875,31.48826763)
\curveto(167.54296682,32.70701492)(166.93359243,34.35545077)(166.93359375,36.43358013)
\curveto(166.93359243,38.58200904)(167.51171685,40.28513234)(168.66796875,41.54295513)
\curveto(169.83202703,42.80856732)(171.39843171,43.44137919)(173.3671875,43.44139263)
\curveto(175.13280298,43.44137919)(176.52733283,42.87106726)(177.55078125,41.73045513)
\curveto(178.58201828,40.59763203)(179.09764276,39.05466482)(179.09765625,37.10154888)
\moveto(176.94140625,37.73436138)
\curveto(176.92576993,38.91403996)(176.59373902,39.85544527)(175.9453125,40.55858013)
\curveto(175.3046778,41.26169387)(174.45311616,41.61325601)(173.390625,41.61326763)
\curveto(172.18749342,41.61325601)(171.22265064,41.2734126)(170.49609375,40.59373638)
\curveto(169.77733958,39.91403896)(169.3632775,38.95700867)(169.25390625,37.72264263)
\lineto(176.94140625,37.73436138)
}
}
{
\newrgbcolor{curcolor}{0 0 0}
\pscustom[linestyle=none,fillstyle=solid,fillcolor=curcolor]
{
\newpath
\moveto(184.76953125,46.85154888)
\lineto(184.76953125,43.12498638)
\lineto(189.2109375,43.12498638)
\lineto(189.2109375,41.44920513)
\lineto(184.76953125,41.44920513)
\lineto(184.76953125,34.32420513)
\curveto(184.76952686,33.25388937)(184.91405796,32.56639006)(185.203125,32.26170513)
\curveto(185.49999487,31.95701567)(186.09765053,31.80467207)(186.99609375,31.80467388)
\lineto(189.2109375,31.80467388)
\lineto(189.2109375,29.99998638)
\lineto(186.99609375,29.99998638)
\curveto(185.33202629,29.99998638)(184.18358994,30.30857982)(183.55078125,30.92576763)
\curveto(182.91796621,31.55076608)(182.60156027,32.68357744)(182.6015625,34.32420513)
\lineto(182.6015625,41.44920513)
\lineto(181.01953125,41.44920513)
\lineto(181.01953125,43.12498638)
\lineto(182.6015625,43.12498638)
\lineto(182.6015625,46.85154888)
\lineto(184.76953125,46.85154888)
}
}
{
\newrgbcolor{curcolor}{0 0 0}
\pscustom[linestyle=none,fillstyle=solid,fillcolor=curcolor]
{
\newpath
\moveto(197.14453125,41.61326763)
\curveto(195.98827506,41.61325601)(195.07421347,41.16013147)(194.40234375,40.25389263)
\curveto(193.73046482,39.35544577)(193.39452765,38.12107201)(193.39453125,36.55076763)
\curveto(193.39452765,34.98045015)(193.72655857,33.74217013)(194.390625,32.83592388)
\curveto(195.06249473,31.93748444)(195.98046257,31.48826614)(197.14453125,31.48826763)
\curveto(198.29296025,31.48826614)(199.20311559,31.94139069)(199.875,32.84764263)
\curveto(200.54686425,33.75388887)(200.88280141,34.98826264)(200.8828125,36.55076763)
\curveto(200.88280141,38.10544702)(200.54686425,39.33591454)(199.875,40.24217388)
\curveto(199.20311559,41.15622522)(198.29296025,41.61325601)(197.14453125,41.61326763)
\moveto(197.14453125,43.44139263)
\curveto(199.01952203,43.44137919)(200.4921768,42.83200479)(201.5625,41.61326763)
\curveto(202.63279966,40.39450723)(203.16795538,38.70700892)(203.16796875,36.55076763)
\curveto(203.16795538,34.40232572)(202.63279966,32.71482741)(201.5625,31.48826763)
\curveto(200.4921768,30.26951736)(199.01952203,29.66014297)(197.14453125,29.66014263)
\curveto(195.26171329,29.66014297)(193.78515226,30.26951736)(192.71484375,31.48826763)
\curveto(191.65234189,32.71482741)(191.12109243,34.40232572)(191.12109375,36.55076763)
\curveto(191.12109243,38.70700892)(191.65234189,40.39450723)(192.71484375,41.61326763)
\curveto(193.78515226,42.83200479)(195.26171329,43.44137919)(197.14453125,43.44139263)
}
}
{
\newrgbcolor{curcolor}{0 0 0}
\pscustom[linestyle=none,fillstyle=solid,fillcolor=curcolor]
{
\newpath
\moveto(206.5078125,35.17967388)
\lineto(206.5078125,43.12498638)
\lineto(208.6640625,43.12498638)
\lineto(208.6640625,35.26170513)
\curveto(208.6640583,34.01951361)(208.90624556,33.08592079)(209.390625,32.46092388)
\curveto(209.87499459,31.84373453)(210.60155637,31.53514109)(211.5703125,31.53514263)
\curveto(212.73436673,31.53514109)(213.65233457,31.90623447)(214.32421875,32.64842388)
\curveto(215.00389571,33.39060799)(215.34373913,34.40232572)(215.34375,35.68358013)
\lineto(215.34375,43.12498638)
\lineto(217.5,43.12498638)
\lineto(217.5,29.99998638)
\lineto(215.34375,29.99998638)
\lineto(215.34375,32.01561138)
\curveto(214.82030215,31.21873516)(214.21092776,30.62498575)(213.515625,30.23436138)
\curveto(212.82811664,29.85154903)(212.02733619,29.66014297)(211.11328125,29.66014263)
\curveto(209.60546361,29.66014297)(208.46093351,30.1288925)(207.6796875,31.06639263)
\curveto(206.89843507,32.00389062)(206.50781046,33.374983)(206.5078125,35.17967388)
\moveto(211.93359375,43.44139263)
\lineto(211.93359375,43.44139263)
}
}
{
\newrgbcolor{curcolor}{0 0 0}
\pscustom[linestyle=none,fillstyle=solid,fillcolor=curcolor]
{
\newpath
\moveto(229.5703125,41.10936138)
\curveto(229.32811538,41.24997513)(229.06249064,41.35153753)(228.7734375,41.41404888)
\curveto(228.49217871,41.48434989)(228.17967902,41.51950611)(227.8359375,41.51951763)
\curveto(226.61718059,41.51950611)(225.67968152,41.12106901)(225.0234375,40.32420513)
\curveto(224.37499533,39.53513309)(224.0507769,38.39841548)(224.05078125,36.91404888)
\lineto(224.05078125,29.99998638)
\lineto(221.8828125,29.99998638)
\lineto(221.8828125,43.12498638)
\lineto(224.05078125,43.12498638)
\lineto(224.05078125,41.08592388)
\curveto(224.50390145,41.88278699)(225.09374461,42.47263015)(225.8203125,42.85545513)
\curveto(226.54686816,43.24606688)(227.42967977,43.44137919)(228.46875,43.44139263)
\curveto(228.61717859,43.44137919)(228.78124092,43.42966045)(228.9609375,43.40623638)
\curveto(229.14061556,43.39059799)(229.33983411,43.36325426)(229.55859375,43.32420513)
\lineto(229.5703125,41.10936138)
}
}
{
\newrgbcolor{curcolor}{0 0 0}
\pscustom[linestyle=none,fillstyle=solid,fillcolor=curcolor]
{
\newpath
\moveto(52.09375,50.57811138)
\lineto(44.125,38.12498638)
\lineto(52.09375,38.12498638)
\lineto(52.09375,50.57811138)
\moveto(51.265625,53.32811138)
\lineto(55.234375,53.32811138)
\lineto(55.234375,38.12498638)
\lineto(58.5625,38.12498638)
\lineto(58.5625,35.49998638)
\lineto(55.234375,35.49998638)
\lineto(55.234375,29.99998638)
\lineto(52.09375,29.99998638)
\lineto(52.09375,35.49998638)
\lineto(41.5625,35.49998638)
\lineto(41.5625,38.54686138)
\lineto(51.265625,53.32811138)
}
}
{
\newrgbcolor{curcolor}{0 0 0}
\pscustom[linewidth=2,linecolor=curcolor,linestyle=dashed,dash=8 8]
{
\newpath
\moveto(150,480)
\lineto(60,480)
}
}
{
\newrgbcolor{curcolor}{0 0 0}
\pscustom[linestyle=none,fillstyle=solid,fillcolor=curcolor]
{
\newpath
\moveto(139.53769464,484.84048224)
\lineto(152.6487474,480.01921591)
\lineto(139.53769392,475.19795064)
\curveto(141.632292,478.04442372)(141.62022288,481.93889292)(139.53769464,484.84048224)
\lineto(139.53769464,484.84048224)
\closepath
}
}
{
\newrgbcolor{curcolor}{0 0 0}
\pscustom[linewidth=2,linecolor=curcolor,linestyle=dashed,dash=8 8]
{
\newpath
\moveto(150,550)
\lineto(60,550)
}
}
{
\newrgbcolor{curcolor}{0 0 0}
\pscustom[linestyle=none,fillstyle=solid,fillcolor=curcolor]
{
\newpath
\moveto(139.53769464,554.84048224)
\lineto(152.6487474,550.01921591)
\lineto(139.53769392,545.19795064)
\curveto(141.632292,548.04442372)(141.62022288,551.93889292)(139.53769464,554.84048224)
\lineto(139.53769464,554.84048224)
\closepath
}
}
{
\newrgbcolor{curcolor}{0 0 0}
\pscustom[linewidth=2,linecolor=curcolor,linestyle=dashed,dash=8 8]
{
\newpath
\moveto(400,400)
\lineto(500,400)
}
}
{
\newrgbcolor{curcolor}{0 0 0}
\pscustom[linestyle=none,fillstyle=solid,fillcolor=curcolor]
{
\newpath
\moveto(410.46230536,395.15951776)
\lineto(397.3512526,399.98078409)
\lineto(410.46230608,404.80204936)
\curveto(408.367708,401.95557628)(408.37977712,398.06110708)(410.46230536,395.15951776)
\lineto(410.46230536,395.15951776)
\closepath
}
}
{
\newrgbcolor{curcolor}{0 0 0}
\pscustom[linewidth=2,linecolor=curcolor,linestyle=dashed,dash=8 8]
{
\newpath
\moveto(150,40)
\lineto(60,40)
}
}
{
\newrgbcolor{curcolor}{0 0 0}
\pscustom[linestyle=none,fillstyle=solid,fillcolor=curcolor]
{
\newpath
\moveto(139.53769464,44.84048224)
\lineto(152.6487474,40.01921591)
\lineto(139.53769392,35.19795064)
\curveto(141.632292,38.04442372)(141.62022288,41.93889292)(139.53769464,44.84048224)
\lineto(139.53769464,44.84048224)
\closepath
}
}
\end{pspicture}

		
		\hypertarget{Statistiques}{}
		\label{Statistiques}
		
		\begin{enumerate}
		  \item Fond d'écran
		  \item Nom du compte hors ligne utilisé
		  \item Text Box ``Statistiques''
		  \item Bouton ``\hyperlink{Accueil}{Retour}''
		\end{enumerate}

		\subsubsection{Description des zones}
		
			\begin{tabular}{|c|c|c|c|c|} \hline
				Numéro de zone & Type  & Description & Evènement &	Règle \\\hline
			\end{tabular}
			
		\subsubsection{Description des règles}

			\underline{RG9-01 :}
				\begin{quote}
				
				\end{quote}
	
\newpage
	

\end{document}
