\section{Démarche}
\subsection{Besoin de l'application}

\begin{frame}
\frametitle{Démarche}
\framesubtitle{Besoin de l'application}
Des coordonnées GPS :
\begin{itemize}
	\item Pour situer les bâtiments
	\item Pour placer les entrées des salles
	\item Pour repérer des points d'intérêts
	\item Pour placer des chemins
\end{itemize}
\end{frame}

\subsection{Récolte des données}

\begin{frame}
\frametitle{Démarche}
\framesubtitle{Récolte des données}
\begin{itemize}
	\item Tri de données GPS d'Open Street Map
	\item Comparaison avec Google Map
	\item Application de récupération de coordonnées GPS
\end{itemize}
\end{frame}

\subsection{Positionnement et orientation du téléphone}

\begin{frame}
\frametitle{Démarche}
\framesubtitle{Positionnement et orientation du téléphone}
Le téléphone est géolocalisé grâce aux coordonnées GPS.\\
Il faut savoir où il filme.\\
$\Rightarrow$ azimut au nord\\
$\Rightarrow$ angle vertical par rapport à l'horizontale\\
Utilisation de deux capteurs:
\begin{itemize}
	\item boussole
	\item capteur d'orientation
\end{itemize}
Les informations géolocalisées sont positionnées de façon relative au téléphone.
Réalisé avec deux angles par rapport à l'axe où filme la caméra.
\end{frame}

\subsection{Positionnement des points d'intérêt}

\begin{frame}
\frametitle{Démarche}
\framesubtitle{Positionnement des points d'intérêt}
\begin{center}
\begin{tabular}{l|r}
vue de dessus & vue de profile \\
\includegraphics[height=4.5cm]{img/zazimuth.png} & \includegraphics[height=4.5cm]{img/zangle.png} \\
\end{tabular}
\end{center}
\end{frame}

\begin{frame}
\frametitle{Démarche}
\framesubtitle{Positionnement des points d'intérêt}
\begin{center}
\includegraphics[height=4.5cm]{img/poi.png}
\end{center}
\end{frame}

\begin{frame}
\frametitle{Démarche}
\framesubtitle{Formes géométriques}
\begin{center}\includegraphics[height=3.7cm]{img/Visibilite.png}\end{center}
Utilités:\\
\begin{itemize}
	\item Faire ressortir des objets de la réalité
	\item Mise en valeur de large zone
\end{itemize}
\end{frame}

