Lors du developpement de notre projet nous avons été confronté à une multitude de problèmes. L'un des problèmes majeurs a été le manque de temps du à l'effectif de notre groupe de projet ainsi qu'à la quantité de travail que demandé la conception de ce projet. Ensuite plusieurs problèmes variés en fonction des plate-forme de developpent ont ralenti notre travail. De plus nous avons passé un peu trop de temps sur l'analyse malgres le fait que celle-ci nous ai permis de programmer aisément.

Le developpement mobile nous a aussi imposé de faible resource matérielle qui nous ont obligé à optimiser au maximum l'analyse et le developpement.


\subsection{Android}

\subsection{iOS}

Le premier problème rencontré avec le developpement d'une application sous IOS a été d'apprendre et maitriser le langage Objective-C qui nous été encore inconnu avant le developpement du projet. De plus la documentation et les informations sur ce langage étant seulement disponible en anglais nous a pas facilité la tache.
Un autre problème de l'Objective-C contrairement à JAVA est le manque d'un garbage collector \footnote{Un garbage collector(ramasse-miettes, ou récupérateur de mémoire ) est un sous-système informatique de gestion automatique de la mémoire. Il est responsable du recyclage de la mémoire préalablement allouée puis inutilisée.} Cela nous a obligé a géré la mémoire manuellement, chose que nous n'avions pas encore pris l'habitude de faire.
Ensuite le problème majeur est que pour tester une application IOS, il est obligatoire de posséder un compte \textit{IOS Developper Program} qui est payant et qui permet ensuite de mettre l'application à tester sur le téléphone. Car à cause de cela, il a été impossible de tester le gameplay de l'application IPhone.
