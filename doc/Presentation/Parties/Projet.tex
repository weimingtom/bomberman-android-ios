
 Etant donné que nous voulions principalement créer une application IPhone, nous avons demandé à developper notre propre projet. Un jeu de type bomberman. Mais ce dernier a du subir des modifications pour être approuvé. Nous avons donc du developper le jeu sous IPhone et sous Android. Le but étant de comparé la différence de developpement entre les deux types de telephone et de developper des fonctionnalités en rapport avec les parcours d'enseignement que nous avons choisi ce semestre.\\
	
Un jeu de type bomerman est une jeu d'action dont le but est simple. Le joueur incarne un poseur de bombe et ce dernier doit faire exploser ses ennemis pour pouvoir gagner la partie. A ceci s'ajoute toute une liste de bonus et malus permettant de complexifier le jeu et de le rendre plus amusant.\\
	
Pour pouvoir rapprocher le developpement de cette application à notre parcours d'enseignement. Nous avons choisi de developper un mode solitaire qui permettra de jouer contre une ou plusieur intelligence artificelle qui est en rapport avec le cursus I2A\footnote{Ingéniererie de l'Inteligence Articielle} qui nous est enseigné. Ensuite pour ce qui est du parcours CASAR \footnote{Combinatoire, Algorithmique, Sécurité et Administration Réseau}, nous avons décidé de développer un mode multijoueur qui permettra à plusieurs joueurs connectés en WI-FI\footnote{Wi-Fi est un ensemble de protocoles de communication sans fil qui permet de relier sans fil plusieurs appareils informatiques au sein d'un réseau informatique afin de permettre la transmission de données entre eux.} de jouer en réseau grâce à un serveur qui combinera un serveur d'application et un serveur web. Puis pour ce qui est du parcours DIWEB \footnote{Données, Interaction et Web}, la partie serveur permettra de palier a l'enseignement de ce dernier parcours.
	
	

		
