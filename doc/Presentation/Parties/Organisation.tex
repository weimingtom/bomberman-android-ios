		Afin de garder notre projet coérent et par sécurité, nous avons choisi de le
		stocker sur les serveurs de \emph{\gls{google_code}} qui mettent gratuitement à
		disposition des gestionnaires de versions (\gls{svn})
		distribués sous \gls{licence_apache} et \gls{licence_bsd}.
		
		Les gestionnaires de versions comme leur nom l'indique, permettent d'avoir à
		porté de main toutes les versions qu'il y a eu d'un fichier depuis sa
		création, cela permet donc de pouvoir revenir en arrière si une erreur a été commise.
		De plus grâce à cela, notre projet reste cohérent dans le sens ou pour pouvoir
		être mit à jour, il faut à tout prit avoir modifié un fichier à partir de la
		dernière version de celui-ci.
		
		Dernier avantage d'avoir utilisé un gestionnaire de version et qu'il est
		hébérgé sur le net et donc chaque membre de l'équipe peut y acceder où qu'il
		soit. 
		
		De plus \emph{\gls{google_code}} met aussi disposition des ses utilisateurs des \glspl{wiki}.
		Un \gls{wiki} est un site Web dont les pages sont modifiables par les developpeurs
		afin de permettre l'écriture et l'illustration collaboratives des documents numériques qu'il contient.
		
		Malgré de nombreuses réunions quotidiennes afin d'organiser au mieux le developpement de cette application dont nous ignorons
		tout à nos début. Nous avons utilisé le \gls{wiki} qui nous a été fournis afin de partager au mieux
		les découvertes que nous faisions au fur et à mesure ainci que les articles qui nous semblaient interressants.
		
		En ce qui concerne la répartition du travail, celle-ci s'est faite naturellement car
		seulement la moitiée du groupe possedait de quoi developper sous \gls{ios}, de là nous avons
		donc créé deux équipes une sous \gls{android} et l'autre sous \gls{ios}.
		
		
		DIAGRAMME DE GANT