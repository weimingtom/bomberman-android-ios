\newglossaryentry{os}{
	name={système d'exploitation}, 
	description={Le système d'exploitation est l'ensemble de programmes central d'un appareil informatique qui sert d'interface entre le matériel et les logiciels applicatifs.}, 
	plural={systèmes d'exploitations}}
	
\newglossaryentry{google_code}{
	name={Google Code}, 
	description={Google Code est un site Web destiné aux développeurs intéressés par le développement relatif à Google.}}
	
\newglossaryentry{licence_apache}{
	name={Licence Apache},
	description={La licence Apache est une licence de logiciel libre et open source. Elle est écrite par l'Apache Software Foundation, qui l'applique à tous les logiciels qu'elle publie. Il existe plusieurs versions de cette licence (1.0, 1.1, 2.0). Cette licence n'est pas copyleft.}}

\newglossaryentry{licence_bsd}{
	name={BSD},	
	description={La licence BSD (Berkeley software distribution
		license) est une licence libre utilisée pour la distribution de logiciels.
		Elle permet de réutiliser tout ou une partie du logiciel sans restriction,
		qu'il soit intégré dans un logiciel libre ou propriétaire.}}

\newglossaryentry{wiki}{
	name={wiki},	
	description={Un wiki est un site Web dont les pages sont modifiables par les developpeurs
		afin de permettre l'écriture et l'illustration collaboratives des documents numériques qu'il contient.},
	plural={wikis}}

\newglossaryentry{android}{
	name={Android},	
	description={Android est un système d'exploitation open source fournis par Google et
	developpé par Android.
	Il est majoritairement utilisé sur smartphones, PDA et autres terminaux
	mobiles mais aussi sur tablettes graphiques et même sur certains téléviseurs.}}

\newglossaryentry{ios}{
	name={iOS},	
	description={iOS, connu sous le nom de iPhone OS avant Juin 2010, est le système d'exploitation mobile d'Apple.}}
	
\newglossaryentry{iphone}{
	name={iPhone},	
	description={L'iPhone est une famille de smartphones conçue et commercialisée par Apple.},
	plural={iPhones}}
	
\newglossaryentry{wi-fi}{
	name={Wi-Fi},
	description={Wireless Fidelity (Wi-Fi) est un ensemble de protocoles de communication sans fil qui permet de relier sans fil plusieurs appareils informatiques au sein d'un réseau informatique afin de permettre la transmission de données entre eux.}}

\newglossaryentry{open_source}{
	name={open source},
	description={La désignation open source s'applique aux logiciels dont la licence respecte des critères précisément établis par l'Open Source Initiative, c'est-à-dire la possibilité de libre redistribution, d'accès au code source et de travaux dérivés.}}
	
\newglossaryentry{google}{
	name={Google},
	description={Google Inc. est une société fondée le 4 septembre 1998 dans la Silicon Valley, en Californie, par Larry Page et Sergey Brin, créateurs du moteur de recherche Google.}}
	
\newglossaryentry{linux}{
	name={Linux},
	description={Linux est un logiciel libre créé en 1991 par Linus Torvalds et développé sur Internet par des milliers d’informaticiens bénévoles ou salariés. C'est le noyau de nombreux systèmes d’exploitation. Il est de type UNIX et compatible POSIX.}}

\newglossaryentry{ndk}{
	name={NDK},
	description={NDK ou Native Developpement Kit permet d'utiliser du code dit natif tel que le C ou C++ au sein des application Android.}}
	
\newglossaryentry{jar}{
	name={JAR},
	description={JAR (Java ARchive) est un fichier ZIP utilisé pour distribuer un ensemble de classes Java. Ce format est utilisé pour stocker les définitions des  classes, ainsi que des métadonnées, constituant l'ensemble d'un programme.}}
	

\newglossaryentry{ipod}{
	name={iPod},
	description={L'iPod est un baladeur numérique conçu et commercialisé par Apple}}
	
\newglossaryentry{ipad}{
	name={iPad},
	description={L'iPad est une tablette électronique conçue et commercialisée par Apple}}

\newglossaryentry{macosx}{
	name={Mac OS X},
	description={Mac OS X est une série de systèmes d’exploitation propriétaires développés et commercialisés par Apple, dont la version la plus récente est Mac OS X v10.6, dit Snow Léopard}}
	
\newglossaryentry{ipod_touch}{
	name={iPod touch},
	description={L’iPod touch est un baladeur numérique à écran tactile capacitif multi-touch, conçu et commercialisé par Apple}}

\newglossaryentry{apple_tv}{
	name={Apple TV},
	description={Apple TV est un appareil conçu et commercialisé par Apple permet la communication sans fil entre un ordinateur et un téléviseur}}
	
\newglossaryentry{core_os}{
	name={Core OS},
	description={La couche \og Core OS \fg \, contient les fonctionalitées bas niveau}}

\newglossaryentry{core_services}{
	name={Core Services},
	description={La couche \og Core Services \fg contient les services fondamentaux	du système}}
	
\newglossaryentry{cocoa}{
	name={Cocoa},
	description={\og Cocoa \fg \,est une API native d'Apple pour le développement objet sur son système d'exploitation Mac OS X}}

\newglossaryentry{smalltalk}{
	name={Smalltalk},
	description={Smalltalk est l'un des premiers langages orientés objets créé en 1972}}
	
\newglossaryentry{c_ansi}{
	name={ANSI},
	description={Le C est un langage de programmation impératif conçu pour la programmation système. Inventé au début des années 1970 avec UNIX}}

\newglossaryentry{cplusplus}{
	name={C++},
	description={Le C++ est un langage de programmation permettant la programmation sous de multiples paradigmes comme la programmation procédurale, la programmation orientée objet et la programmation générique}}
	
\newglossaryentry{java}{
	name={Java},
	description={Le langage Java est un langage de programmation informatique orienté objet créé par James Gosling et Patrick Naughton, employés de Sun Microsystems, avec le soutien de Bill Joy (cofondateur de Sun Microsystems en 1982), présenté officiellement le 23 mai 1995 au SunWorld.}}

\newglossaryentry{xml}{
	name={XML},
	description={XML (Extensible Markup Language, « langage de balisage extensible») est un langage informatique de balisage générique qui succède à SGML. Cette syntaxe est dite extensible car elle permet de définir différents espaces de noms, c'est à dire des langages avec chacun leur vocabulaire et leur grammaire, comme XHTML, XSLT, RSS…}}
	
\newglossaryentry{objective-c}{
	name={Objective-C},
	description={L'Objective-C est un langage de programmation orienté objet réflexif. C'est une extension du C ANSI, comme le C++, mais qui se distingue de ce dernier par sa distribution dynamique des messages, son typage faible ou fort, son typage dynamique et son chargement dynamique. Contrairement au C++, il ne permet pas l'héritage multiple mais il existe toutefois des moyens de combiner les avantages de C++ et d'Objective-C.}}

\newglossaryentry{}{
	name={},
	description={}}
	
\newglossaryentry{}{
	name={},
	description={}}

\newglossaryentry{}{
	name={},
	description={}}
	
\newglossaryentry{}{
	name={},
	description={}}

\newglossaryentry{}{
	name={},
	description={}}
	
\newglossaryentry{}{
	name={},
	description={}}

\newglossaryentry{}{
	name={},
	description={}}

\newglossaryentry{}{
	name={},
	description={}}
	
\newglossaryentry{}{
	name={},
	description={}}

\newglossaryentry{}{
	name={},
	description={}}
	
\newglossaryentry{}{
	name={},
	description={}}

\newglossaryentry{}{
	name={},
	description={}}
	
\newglossaryentry{}{
	name={},
	description={}}
	
\newglossaryentry{}{
	name={},
	description={}}

\newglossaryentry{}{
	name={},
	description={}}
	
\newglossaryentry{}{
	name={},
	description={}}

\newglossaryentry{}{
	name={},
	description={}}
	
\newglossaryentry{}{
	name={},
	description={}}

\newglossaryentry{}{
	name={},
	description={}}
	
\newglossaryentry{}{
	name={},
	description={}}
	
	
\newacronym{svn}{SVN}{Subversion}
\newacronym{i2a}{I2A}{Ingéniererie de l'Inteligence Articielle}
\newacronym{casar}{CASAR}{Combinatoire, Algorithmique, Sécurité et Administration Réseau}
\newacronym{diweb}{DIWEB}{Données, Interaction et Web}
\newacronym{jit}{JIT}{just-in-time compilation}
\newacronym{sdk}{SDK}{kit de développement}
\newacronym{opengl_es}{OpenGL ES}{Open Graphics Library for Embedded System}
\newacronym{apk}{APK}{Android Package}
\newacronym{cpu}{CPU}{Central Processing Unit}
