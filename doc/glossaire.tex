\newglossaryentry{os}{
	name={système d'exploitation}, 
	description={Le système d'exploitation est l'ensemble de programmes central d'un appareil informatique qui sert d'interface entre le matériel et les logiciels applicatifs.}, 
	plural={systèmes d'exploitations}}
	
\newglossaryentry{google_code}{
	name={Google Code}, 
	description={Google Code est un site Web destiné aux développeurs intéressés par le développement relatif à Google.}}
	
\newglossaryentry{licence_apache}{
	name={Licence Apache},
	description={La licence Apache est une licence de logiciel libre et open source. Elle est écrite par l'Apache Software Foundation, qui l'applique à tous les logiciels qu'elle publie. Il existe plusieurs versions de cette licence (1.0, 1.1, 2.0). Cette licence n'est pas copyleft.}}

\newglossaryentry{licence_bsd}{
	name={BSD},	
	description={La licence BSD (Berkeley software distribution
		license) est une licence libre utilisée pour la distribution de logiciels.
		Elle permet de réutiliser tout ou une partie du logiciel sans restriction,
		qu'il soit intégré dans un logiciel libre ou propriétaire.}}

\newglossaryentry{wiki}{
	name={wiki},	
	description={Un wiki est un site Web dont les pages sont modifiables par les developpeurs
		afin de permettre l'écriture et l'illustration collaboratives des documents numériques qu'il contient.}
	plurial={wikis}}

\newglossaryentry{android}{
	name={Android},	
	description={Android est un système d'exploitation open source fournis par Google et
	developpé par Android.
	Il est majoritairement utilisé sur smartphones, PDA et autres terminaux
	mobiles mais aussi sur tablettes graphiques et même sur certains téléviseurs.}}

\newglossaryentry{ios}{
	name={iOS},	
	description={iOS, connu sous le nom de iPhone OS avant Juin 2010, est le système d'exploitation mobile d'Apple.}}
	
\newglossaryentry{iphone}{
	name={iPhone},	
	description={L'iPhone est une famille de smartphones conçue et commercialisée par Apple.}
	plurial={iPhones}}
	
\newglossaryentry{wi-fi}{
	name={Wi-Fi},
	description={Wireless Fidelity (Wi-Fi) est un ensemble de protocoles de communication sans fil qui permet de relier sans fil plusieurs appareils informatiques au sein d'un réseau informatique afin de permettre la transmission de données entre eux.}}

\newglossaryentry{open_source}{
	name={open source},
	description={La désignation open source s'applique aux logiciels dont la licence respecte des critères précisément établis par l'Open Source Initiative, c'est-à-dire la possibilité de libre redistribution, d'accès au code source et de travaux dérivés.}}
	
\newglossaryentry{google}{
	name={Google},
	description={Google Inc. est une société fondée le 4 septembre 1998 dans la Silicon Valley, en Californie, par Larry Page et Sergey Brin, créateurs du moteur de recherche Google.}}
	
\newglossaryentry{linux}{
	name={Linux},
	description={Linux est un logiciel libre créé en 1991 par Linus Torvalds et développé sur Internet par des milliers d’informaticiens bénévoles ou salariés. C'est le noyau de nombreux systèmes d’exploitation. Il est de type UNIX et compatible POSIX.}}

\newglossaryentry{}{
	name={},
	description={}}
	
\newglossaryentry{}{
	name={},
	description={}}

\newglossaryentry{}{
	name={},
	description={}}
	
\newglossaryentry{}{
	name={},
	description={}}

\newglossaryentry{}{
	name={},
	description={}}
	
\newglossaryentry{}{
	name={},
	description={}}

\newglossaryentry{}{
	name={},
	description={}}
	
\newglossaryentry{}{
	name={},
	description={}}

\newglossaryentry{}{
	name={},
	description={}}
	
\newglossaryentry{}{
	name={},
	description={}}

\newglossaryentry{}{
	name={},
	description={}}
	
\newglossaryentry{}{
	name={},
	description={}}

\newglossaryentry{}{
	name={},
	description={}}
	
\newglossaryentry{}{
	name={},
	description={}}

\newglossaryentry{}{
	name={},
	description={}}
	
\newglossaryentry{}{
	name={},
	description={}}

\newglossaryentry{}{
	name={},
	description={}}
	
\newglossaryentry{}{
	name={},
	description={}}

\newglossaryentry{}{
	name={},
	description={}}
	
\newglossaryentry{}{
	name={},
	description={}}

\newglossaryentry{}{
	name={},
	description={}}
	
\newglossaryentry{}{
	name={},
	description={}}

\newacronym{svn}{SVN}{Subversion}
\newacronym{i2a}{I2A}{Ingéniererie de l'Inteligence Articielle}
\newacronym{casar}{CASAR}{Combinatoire, Algorithmique, Sécurité et Administration Réseau}
\newacronym{diweb}{DIWEB}{Données, Interaction et Web}