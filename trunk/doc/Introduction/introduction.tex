Nous sommes amenés, dans le cadre de notre première année en master informatique à la faculté des Sciences de Montpellier, à travailler sur l’élaboration complète d’un projet, de son analyse à la conception puis à sa programmation. Tout au long de ce deuxième semestre, le projet nous permet de comprendre quelles sont les phases de développement d’un projet et comment celui-ci doit être conduit. Nous pouvons par ailleurs mesurer notre capacité à réagir face à des problèmes en nous impliquant dans ce projet. Le projet que nous avons choisi constiste à réaliser sur deux \glspl{os} de téléphone différents un jeu de type Bomberman.

Ce projet nous permet de mettre en application l'enseignement que nous avons acquis tout au long de ce semestre. Ce dernier étant spécialisé pour chaque étudiant, il nous a permis de travailler en collaboration avec des personnes aux capacités différentes et d'ainsi mettre nos connaissances en commun. Mais il permet aussi de se faire une idée du travail demandé dans le monde des entreprises et d'ainsi nous préparer à notre stage que nous devrons réaliser l'an prochain.

Afin de comprendre la démarche que nous avons utilisée pour mener ce projet à son terme, notre rapport se compose de cinq grandes parties : 

Tout d'abord, dans une première partie, nous présentons le cadre général de notre projet ainsi que le jeu que nous avons développé. Ensuite, dans une seconde, partie nous verrons comment nous nous sommes orgnanisé pour mener à bien la conduite de notre projet. Puis dans une troisième partie, nous présenterons le travail d'analyse que nous avons effectué pour pouvoir ensuite, dans une quatrième partie, expliquer le developpement du projet. Enfin, nous finirons en expliquant l'implémentation de la réutilisabilité dans le code pour pouvoir ensuite finir sur la conclusion.