\newglossaryentry{os}{
	name={système d'exploitation}, 
	description={Le système d'exploitation est l'ensemble de programmes central d'un appareil informatique qui sert d'interface entre le matériel et les logiciels applicatifs.}, 
	plural={systèmes d'exploitations}}
	
\newglossaryentry{google_code}{
	name={Google Code}, 
	description={Google Code est un site Web destiné aux développeurs intéressés par le développement relatif à Google.}}
	
\newglossaryentry{licence_apache}{
	name={Licence Apache},
	description={La licence Apache est une licence de logiciel libre et open source. Elle est écrite par l'Apache Software Foundation, qui l'applique à tous les logiciels qu'elle publie. Il existe plusieurs versions de cette licence (1.0, 1.1, 2.0). Cette licence n'est pas copyleft.}}

\newglossaryentry{licence_bsd}{
	name={BSD},	
	description={La licence BSD (Berkeley software distribution
		license) est une licence libre utilisée pour la distribution de logiciels.
		Elle permet de réutiliser tout ou une partie du logiciel sans restriction,
		qu'il soit intégré dans un logiciel libre ou propriétaire.}}

\newglossaryentry{wiki}{
	name={wiki},	
	description={Un wiki est un site Web dont les pages sont modifiables par les developpeurs
		afin de permettre l'écriture et l'illustration collaboratives des documents numériques qu'il contient.},
	plural={wikis}}

\newglossaryentry{android}{
	name={Android},	
	description={Android est un système d'exploitation open source fournis par Google et
	developpé par Android.
	Il est majoritairement utilisé sur smartphones, PDA et autres terminaux
	mobiles mais aussi sur tablettes graphiques et même sur certains téléviseurs.}}

\newglossaryentry{ios}{
	name={iOS},	
	description={iOS, connu sous le nom de iPhone OS avant Juin 2010, est le système d'exploitation mobile d'Apple.}}
	
\newglossaryentry{iphone}{
	name={iPhone},	
	description={L'iPhone est une famille de smartphones conçue et commercialisée par Apple.},
	plural={iPhones}}
	
\newglossaryentry{wi-fi}{
	name={Wi-Fi},
	description={Wireless Fidelity (Wi-Fi) est un ensemble de protocoles de communication sans fil qui permet de relier sans fil plusieurs appareils informatiques au sein d'un réseau informatique afin de permettre la transmission de données entre eux.}}

\newglossaryentry{open_source}{
	name={open source},
	description={La désignation open source s'applique aux logiciels dont la licence respecte des critères précisément établis par l'Open Source Initiative, c'est-à-dire la possibilité de libre redistribution, d'accès au code source et de travaux dérivés.}}
	
\newglossaryentry{google}{
	name={Google},
	description={Google Inc. est une société fondée le 4 septembre 1998 dans la Silicon Valley, en Californie, par Larry Page et Sergey Brin, créateurs du moteur de recherche Google.}}
	
\newglossaryentry{linux}{
	name={Linux},
	description={Linux est un logiciel libre créé en 1991 par Linus Torvalds et développé sur Internet par des milliers d’informaticiens bénévoles ou salariés. C'est le noyau de nombreux systèmes d’exploitation. Il est de type UNIX et compatible POSIX.}}

\newglossaryentry{ndk}{
	name={NDK},
	description={NDK ou Native Developpement Kit permet d'utiliser du code dit natif tel que le C ou C++ au sein des application Android}}
	
\newglossaryentry{jar}{
	name={JAR},
	description={JAR (Java ARchive) est un fichier ZIP utilisé pour distribuer un ensemble de classes Java. Ce format est utilisé pour stocker les définitions des  classes, ainsi que des métadonnées, constituant l'ensemble d'un programme.}}

\newglossaryentry{ipod}{
	name={iPod},
	description={L'iPod est un baladeur numérique conçu et commercialisé par Apple}}
	
\newglossaryentry{ipad}{
	name={iPad},
	description={L'iPad est une tablette électronique conçue et commercialisée par Apple}}

\newglossaryentry{macosx}{
	name={Mac OS X},
	description={Mac OS X est une série de systèmes d’exploitation propriétaires développés et commercialisés par Apple, dont la version la plus récente est Mac OS X v10.6, dit Snow Léopard}}
	
\newglossaryentry{ipod_touch}{
	name={iPod touch},
	description={L’iPod touch est un baladeur numérique à écran tactile capacitif multi-touch, conçu et commercialisé par Apple}}

\newglossaryentry{apple_tv}{
	name={Apple TV},
	description={Apple TV est un appareil conçu et commercialisé par Apple permet la communication sans fil entre un ordinateur et un téléviseur}}
	
\newglossaryentry{core_os}{
	name={Core OS},
	description={La couche \og Core OS \fg \, contient les fonctionalitées bas niveau}}

\newglossaryentry{core_services}{
	name={Core Services},
	description={La couche \og Core Services \fg contient les services fondamentaux	du système}}
	
\newglossaryentry{cocoa}{
	name={Cocoa},
	description={\og Cocoa \fg \,est une API native d'Apple pour le développement objet sur son système d'exploitation Mac OS X}}

\newglossaryentry{smalltalk}{
	name={Smalltalk},
	description={Smalltalk est l'un des premiers langages orientés objets créé en 1972}}
	
\newglossaryentry{c_ansi}{
	name={ANSI},
	description={Le C est un langage de programmation impératif conçu pour la programmation système. Inventé au début des années 1970 avec UNIX}}

\newglossaryentry{cplusplus}{
	name={C++},
	description={Le C++ est un langage de programmation permettant la programmation sous de multiples paradigmes comme la programmation procédurale, la programmation orientée objet et la programmation générique}}
	
\newglossaryentry{servlet}{
	name={servlet},
	description={Une servlet est une classe Java qui permet de créer dynamiquement des données au sein d'un serveur HTTP. Ces données sont le plus généralement présentées au format HTML, mais elles peuvent également l'être au format XML ou tout autre format destiné aux navigateurs web. Les servlets utilisent l'API Java Servlet (package javax.servlet).}
	plurial={servlets}}

\newglossaryentry{json}{
	name={JSON},
	description={JSON (JavaScript Object Notation) est un format de données textuel, générique, dérivé de la notation des objets du langage ECMAScript. Il permet de représenter de l’information structurée.}}
	
\newglossaryentry{java}{
	name={Java},
	description={Le langage Java est un langage de programmation informatique orienté objet créé par James Gosling et Patrick Naughton}}
		
\newglossaryentry{jit}{
	name={JIT},
	description={Just-In-Time Compiler (JIT) ou dynamic translation est une technique pour améliorer les performances d'exécution de certains programmes}}

\newglossaryentry{xml}{
	name={XML},
	description={XML ou Extensible Markup Language, « langage de balisage extensible» est un langage informatique de balisage générique}}

\newglossaryentry{objective-c}{
	name={Objective-C},
	description={L'Objective-C est un langage de programmation orienté objet réflexif. C'est une extension du C ANSI, comme le C++, mais qui se distingue de ce dernier par sa distribution dynamique des messages, son typage faible ou fort, son typage dynamique et son chargement dynamique. Contrairement au C++, il ne permet pas l'héritage multiple mais il existe toutefois des moyens de combiner les avantages de C++ et d'Objective-C.}}

\newglossaryentry{algorithme}{
	name={algorithme},
	description={L'algorithmique est l’ensemble des règles et des techniques qui sont impliquées dans la définition et la conception d'algorithmes, c'est-à-dire de processus systématiques de résolution, par le calcul, d'un problème permettant de décrire les étapes vers le résultat. En d'autres termes, un algorithme est une suite finie et non-ambiguë d’opérations permettant de donner la réponse à un problème.},
	plural={algorithme}}
	
\newglossaryentry{recherche_op}{
	name={recherche opérationelle},
	description={La recherche opérationnelle (aussi appelée aide à la décision) peut être définie comme l'ensemble des méthodes et techniques rationnelles d'analyse et de synthèse des phénomènes de management du système d'information utilisables pour élaborer de meilleures décisions.}}

\newglossaryentry{pathfinding}{
	name={pathfinding},
	description={La recherche de chemin, couramment appelée pathfinding, est un problème de l'intelligence artificielle qui se rattache plus généralement au domaine de la planification et de la recherche de solution. Il consiste à trouver comment se déplacer dans un environnement entre un point de départ et un point d'arrivée en prenant en compte différentes contraintes.}}

\newglossaryentry{html}{
	name={HTML},
	description={siginifie HyperText Markup Language et est le format de données conçu pour représenter les pages web.}}
	
\newglossaryentry{bot}{
	name={bot},
	description={Bot (diminutif de robot) désigne un personnage contrôlé par l'ordinateur.},
	plural={bots}}

\newglossaryentry{sdk}{
	name={SDK},
	description={Un kit de développement ou trousse de développement logiciel est un ensemble d'outils permettant aux développeurs de créer des applications de type défini (par exemple pour iPhone ou Android)}}

\newglossaryentry{opengl_es}{
	name={OPENGL-ES},
	description={OpenGL ES (Open Graphics Library for Embedded System, parfois abrégé en OGLES ou GLES) est une spécification du Khronos Group qui définit une API multiplate-forme pour la conception d'applications générant des images 3D dérivée de la spécification OpenGL, sous une forme adaptée aux plateformes mobiles ou embarquées telles que les téléphones mobiles, les assistants personnels (PDA), les consoles de jeux vidéo portables, les lecteurs multimédia de poche ou de salon...}}

\newglossaryentry{apk}{
	name={APK},
	description={APK ou Android Package (ex., "nomfich.apk"), un format de collection de fichiers ("package") pour le système d'exploitation Android}}
	
\newglossaryentry{cpu}{
	name={CPU},
	description={Le processeur, ou CPU (de l'anglais Central Processing Unit, « Unité centrale de traitement »), est le composant de l'ordinateur qui exécute les programmes informatiques}}
	
\newglossaryentry{msx}{
	name={MSX},
	description={MSX est un standard de micro-ordinateurs à vocation domestique (grand public) d’origine principalement japonaise, datant des années 1980. }}
	
\newglossaryentry{lode runner}{
	name={Lode Runner},
	description={Lode Runner est un jeu vidéo de plates-formes et de réflexion publié en 1983 sur ordinateurs personnels 8-bit.}}
	
\newglossaryentry{nes}{
	name={NES},
	description={Le Nintendo Entertainment System, ou NES, est une console de jeux vidéo 8 bits du constructeur japonais Nintendo sortie en 1985}}
	
\newglossaryentry{pattern}{
	name={pattern},
	description={Un pattern est un modèle de conception en informatique}}
	
\newglossaryentry{survivor}{
	name={survivor},
	description={Le mode survivor est un mode où chaque joueur à un nombre de vie limité et dont le gagnant est celui qui sera le dernier a être en vie}}
		
\newglossaryentry{death_match}{
	name={death match},
	description={Le mode death match est un mode où il y a un temps limité et dont le joueur gagnant est celui qui a tué le plus d'ennemis}}	

\newglossaryentry{game_boy}{
	name={Game Boy},
	description={La Game Boy  est une console de jeux vidéo portable à la puissance comparable à celle de la NES}}
	
\newglossaryentry{tile_mapping}{
	name={Tile Mapping},
	description={Le Tile Mapping est une technique de modélisation graphique}}
	
\newglossaryentry{smartphone}{
	name={smartphone},
	description={Traduit littéralement comme \og téléphone intéligent \fg \, en français, c'est un terme utilisé pour désigner les téléphones évolués, qui possèdent des fonctions similaires à celles des assistants personnels. Certains peuvent lire des vidéos, des MP3 et se voir ajouter des programmes spécifiques.},
	plural={smartphones}}
	
\newglossaryentry{gl}{
	name={GL},
	description={Génie Logiciel}}
	
\newglossaryentry{api}{
	name={API},
	description={Une interface de programmation (Application Programming Interface ou API) est une interface fournie par un programme informatique. Elle permet l'interaction des programmes les uns avec les autres, de manière analogue à une interface homme-machine, qui rend possible l'interaction entre un homme et une machine.}}
	
\newglossaryentry{widget}{
	name={widget},
	description={Il peut être considéré comme étant la contraction des termes window (fenêtre) et gadget. Dans notre cas il désigne
	un composant d'interface graphique, un élément visuel d'une interface graphique (bouton, ascenseur, liste déroulante, etc.)}}
	
\newglossaryentry{interface_builder}{
	name={Interface Builder},
	description={Interface Builder est un outil de développement d'interface graphique pour des applications tournant sur Mac OS X.}}
	
\newglossaryentry{xib}{
	name={xib},
	description={xib est l'acronyme de Xml Interface Buillder.}}
	
\newglossaryentry{nib}{
	name={nib},
	description={nib es l'acronyme de NeXT Interface Builder.}}
	
\newglossaryentry{sqlite}{
	name={SQLite},
	description={SQLite est une bibliothèque écrite en C qui propose un moteur de base de données relationnelles accessible par le langage SQL. }}
	
\newglossaryentry{}{
	name={},
	description={}}
	
\newglossaryentry{}{
	name={},
	description={}}
	
\newacronym{svn}{SVN}{Subversion}
\newacronym{i2a}{I2A}{Ingéniererie de l'Inteligence Articielle}
\newacronym{casar}{CASAR}{Combinatoire, Algorithmique, Sécurité et Administration Réseau}
\newacronym{diweb}{DIWEB}{Données, Interaction et Web}
%\newacronym{jit}{JIT}{just-in-time compilation}
%\newacronym{sdk}{SDK}{kit de développement}
%\newacronym{opengl_es}{OpenGL ES}{Open Graphics Library for Embedded System}
%\newacronym{apk}{APK}{Android Package}
%\newacronym{cpu}{CPU}{Central Processing Unit}
\newacronym{mvc}{MVC}{Modèle Vue Contrôleur}
\newacronym{ihm}{IHM}{Interface Homme Machine}
\newacronym{http}{HTTP}{Hypertext Transfer Protocol}


