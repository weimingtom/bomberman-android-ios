Lors du developpement sous en Android et IOS nous avons été confronté à divers problèmes. Notament des problèmes liés aux différencex entre ces deux systèmes d'exploitationx. Dans un premier temps le langage utilisé n'est pas le même, le premier utilise le langage JAVA alors que l'autre utilise l'Objective-C. Ces deux langages étant des langages orientés objets, la difficulté résidé surtout dans la différence de la syntaxe.
Ensuite, lors de l'implémentation de notre diagramme de classe notre premier soucis a été la différence de modélisation des vues et des controlleurs. En effet sous Android les écouteurs d'une vue sont directement situé dans le controlleur associé alors que sous IOS les écouteurs peuvent être implémenté soit dans la vue soit dans le controlleur associé. Pour respecter au maximum le modèle MVC nous avons choisi de placer les écouteurs dans la vue sous IOS. 