		Afin de garder notre projet coérent et par sécurité nous avons choisi de le
		stocker sur les serveurs de \emph{Google Code} qui mettent gratuitement à
		disposition des gestionnaires de versions (Subversion ou SVN en abrégé)
		distribués sous licence Apache  \footnote{La licence Apache est une licence 
		de logiciel libre et open source. Elle est écrite par l'Apache Software
		Foundation,  qui l'applique à tous les logiciels qu'elle publie. Il existe
		plusieurs versions de cette licence (1.0, 1.1, 2.0). Cette licence n'est pas
		copyleft.} et BSD\footnote{La licence BSD (Berkeley software distribution
		license) est une licence libre utilisée pour la distribution de logiciels.
		Elle permet de réutiliser tout ou une partie du logiciel sans restriction,
		qu'il soit intégré dans un logiciel libre ou propriétaire.}.
		
		Les gestionnaires de versions comme leur nom l'indique, permettent d'avoir à
		porté de main toutes les versions qu'il y a eu d'un fichier depuis sa
		création, cela permet donc de pouvoir revenir en arrière si une erreur a été commise.
		De plus grâce à cela notre projet reste cohérent dans le sens ou pour pouvoir
		être mit à jour il faut à tout prit avoir modifié un fichier à partir de la
		dernière version de celui-ci.
		Dernier avantage d'avoir utilisé un gestionnaire de version et qu'il est
		hébérgé sur le net et donc chaque membre de l'équipe peut y acceder où qu'il
		soit.

\begin{itemize}
  
	\item{Subversion}	
	\item{Réunions}
	\item{Répartition du travail}
	\item{Diagramme de Gant}
\end{itemize}