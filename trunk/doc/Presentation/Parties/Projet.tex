
 Etant donné que nous voulions principalement créer une application
 \gls{iphone}, nous avons demandé à développer notre propre projet. Un jeu de
 type bomberman. Mais ce dernier a dû subir des modifications pour être
 approuvé. Nous avons donc dû développer le jeu sous \gls{iphone} et sous
 \gls{android}. Le but étant de comparer la différence de développement entre
 les deux types de téléphones et de développer des fonctionnalités en rapport
 avec les parcours d'enseignement que nous avons choisi ce semestre.\\
	
Un jeu de type Bomberman est un jeu d'action dont le but est simple. Le joueur
incarne un poseur de bombes et ce dernier doit faire exploser ses ennemis pour
pouvoir gagner la partie. A ceci s'ajoute toute une liste de bonus et malus
permettant de complexifier le jeu et de le rendre plus amusant.\\
	
Pour pouvoir rapprocher le développement de cette application à notre parcours
d'enseignement. Nous avons choisi de développer un mode solitaire qui permettra
de jouer contre une ou plusieurs intelligences artificelles qui est en rapport
avec le cursus \gls{i2a} qui nous est enseigné. Ensuite pour ce qui est du
parcours \gls{casar}, nous avons décidé d'implémenter un mode multijoueur qui
permettra à plusieurs joueurs connectés en \gls{wi-fi} de jouer en réseau grâce
à un serveur qui combinera un serveur d'applications et un serveur web. Puis
pour ce qui est du parcours \gls{diweb}, la partie serveur permettra de palier à
l'enseignement de ce dernier parcours.
	
