
 Etant donné que nous voulions principalement créer une application
 \gls{iphone}, nous avons demandé à développer notre propre projet. Un jeu de
 type Bomberman. Mais ce dernier a dû subir des modifications pour être
 approuvé. Nous avons donc dû développer le jeu sous \gls{iphone} et sous
 \gls{android}. Le but étant de comparer la différence de développement entre
 les deux types de téléphones et de développer des fonctionnalités en rapport
 avec les parcours d'enseignement que nous avons choisi ce semestre.\\
	
	
Pour pouvoir rapprocher le développement de cette application à notre parcours
d'enseignement. Nous avons choisi de développer un mode solitaire qui permettra
de jouer contre une ou plusieurs intelligences artificelles qui est en rapport
avec le cursus \gls{i2a} qui nous est enseigné. Ensuite, pour ce qui est du
parcours \gls{casar}, nous avons décidé d'implémenter un mode multijoueur qui
permettra à plusieurs joueurs connectés en \gls{wi-fi} de jouer en réseau grâce
à un serveur qui combinera un serveur d'applications et un serveur web. Pour ce qui est de la conception de toute la structure du programme nous avons mis en application nos connaissances apprises dans le parcours \gls{gl}. Puis pour ce qui est du parcours \gls{diweb}, la partie serveur permettra de palier à
l'enseignement de ce dernier. 
	
