\subsection{Android}

	\gls{android} est un \gls{os} \gls{open_source} fournis par \gls{google} et est
	developpé par la startup Android.
	Il est majoritairement utilisé sur smartphones, PDA et autres terminaux
	mobiles mais aussi sur tablettes graphiques et même sur certains téléviseurs.
	
	Disponible via une \gls{licence_apache} version 2, \gls{android} est fondé  sur
	un noyau \gls{linux}.
	Il comporte une interface spécifique, développée en \gls{java}.
	Les programmes sont exécutés via un interpréteur \gls{jit}, toutefois il est possible
	de passer outre cette interface en programmant ses applications en \gls{c_ansi} ou \gls{cplusplus},
	mais le travail de portabilité en sera plus important.
	
	Il nous est aussi permis d'utiliser du \gls{xml} principalement pour les interfaces
	graphiques ou encore pour la portabilité des données entre chaque type
	d'architecture.
	
	Sa première version date du 5 novembre 2007, actuellement la dernière version
	disponible est la 3.1 Honeycomb (Rayon de miel) mais nous utiliserons ici par
	soucis de compatibilité la version 2.0 dans le but de toucher un maximum de
	périphériques.
	
	Afin de pouvoir développer nos propres applications, \gls{android} met à
	dispotition un \gls{sdk} ainsi qu'un greffon pour Eclipse
	(ADT) afin de simplifier l'utilisation de celui-ci.
	
	Le \gls{sdk} d'\gls{android} comporte les outils de base pour developper une application en
	\gls{java}, le developpement en \gls{c_ansi} et \gls{cplusplus} necessitant d'utiliser le \gls{ndk}.
	Il se compose des librairies principales d'\gls{android}, de divers exemples
	d'applications et surtout d'un émulateur de terminal \gls{android} qui permet de
	pouvoir tester ses applications directement sur \gls{linux}, Mac ou
	Windows sans nécessairement avoir besoin d'un terminal mis à part pour
	\gls{opengl_es} 2.0 qui n'est pas encore pris en charge. Sinon le \gls{sdk}
	permet très bien de tester ses applications en temps réel sur tout
	périphériques tournant sur \gls{android} que ce soit en liaison directe avec un
	ordinateur ou en récupérant l'archive de l'application au format \gls{apk}.
	
	Le format \gls{apk} est une variante du format
	\gls{jar} qui est utilisée pour la distribution et l'installation de composants
	regroupés sur le \gls{os} \gls{android}.


\subsection{iOS}
	Apple est une entreprise multinationale américaine qui a été créée le 1\up{er}
	avril 1976 à Cupertino (Californie) par Steve Jobs et Steve Wozniak. Elle a
	pour but de concevoir et vendre des produits électroniques grand public, des
	ordinateurs personnels et des logiciels informatiques. Parmi les produits les
	plus répendus de l'entreprise se trouvent les ordinateurs Macintosh (1984),
	l'\gls{ipod} (2001), l'\gls{iphone} (2007) et l'\gls{ipad} (2010). Apple a
	développé deux \glspl{os} (\gls{macosx} et \gls{ios})
	permettant de contrôler ses différents produits, notamment les ordinateurs 
	Macintosh, l'\gls{ipod} et l'\gls{ipod_touch}.
			
	\gls{ios}, connu sous le nom de iPhone OS avant Juin 2010, est le \gls{os} 
	mobile d'Apple. Ce dernier a été développé originellement pour l'\gls{iphone}, puis a été étendu pour l'\gls{ipod_touch}, \gls{ipad} et
	\gls{apple_tv} (2007). C'est un dérivé de \gls{macosx}  dont il partage les 
	fondations. Comme celui-ci, \gls{ios} comporte quatre couches d'abstraction : une
	couche \og \gls{core_os} \fg, \og \gls{core_services} \fg, \og Media \fg \, et pour finir \og \gls{cocoa} \fg. Pour
	développer une application Cocoa, il est imposé d'utiliser l'\gls{objective-c} comme
	langage de programmation.
			
	L'\gls{objective-c} est un langage de programmation orienté objet. Il a été inventé
	au début des années 1980 par Brad Cox, créateur de la société Stepstone. Son
	objectif était de combiner la richesse du langage \gls{smalltalk}\, et la
	rapidité du \gls{c_ansi}. 
	C'est  donc une extension du \gls{c_ansi} qui ne permet pas l'héritage multiple
	contrairement au \gls{cplusplus}. Aujourd'hui, il est principalement utilisé
	pour développer des applications sur \gls{macosx} et \gls{iphone}. Une nouvelle version
	\gls{objective-c} 2.0 a été indrotuite avec \gls{macosx} 10.5 (Léopard) en octobre 2007.
			
	Le \gls{sdk} d'\gls{ios} fournit les principaux outils nécessaires pour
	réaliser une application sur \gls{iphone}. Il est composé notamment de \gls{xcode},
	\gls{interface_builder}, \gls{iphone} Simulator et Instruments. \gls{xcode} est l'outil de
	développement Apple, il permet la création de projets \gls{iphone}, l’édition du code
	source \gls{objective-c}, la compilation et le débogage des applications. Interface
	Builder quant à lui permet de construire des interfaces graphiques. L'\gls{iphone}
	Simulator est un logiciel simulant le comportement d'un \gls{iphone}, ce qui 
	permet de pouvoir tester les applications directement sur l'ordinateur. Grâce
	à Instruments, il est possible d'analyser un programme pour surveiller l’état de la
	mémoire, l’utilisation du réseau, du \gls{cpu}, etc. Pour utiliser tous ces outils,
	il est obligatoire de posséder un ordinateur Macintosh sous \gls{macosx} et de
	s'enregistrer sur iPhone Dev Center pour pouvoir télécharger et installer le
	\gls{sdk} d'\gls{ios}.
