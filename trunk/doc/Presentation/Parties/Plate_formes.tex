\subsection{Android}

	Android est un système d'exploitation open source fournis par \emph{Google} et
	developpé par \emph{Android}.
	Il est majoritairement utilisé sur smartphones, PDA et autres terminaux
	mobiles mais aussi sur tablettes graphiques et même sur certains téléviseurs.
	
	Android est fondé sur un noyau Linux, il comporte une interface
	spécifique, développée en Java, les programmes sont exécutés via un
	interpréteur JIT, toutefois il est possible de passer outre cette interface en
	programmant ses applications en C ou C++, mais le travail de portabilité en
	sera plus important.
	
	Il nous est aussi permit d'utiliser du XML principalement pour les interfaces
	graphiques ou encore pour la portabilité des données entre chaque type
	d'architecture.
	
	Sa première version date du 5 novembre 2007, actuellement la dernière version
	disponible est la 3.1 Honeycomb (Rayon de miel) mais nous utiliserons ici par
	soucis de compatibilité la version 2.0 dans le but de toucher un maximums de
	périphériques.
	
	Afin de pouvoir développer nos propres applications, \emph{Android} met à
	dispotition un kit de développement (SDK) ainsi qu'un greffon pour Eclipse
	(ADT) afin de simplifier l'utilisation de celui-ci.
	
	Le kit de développement d'Android comporte les outils de base pour developper une application en
	Java, le developpement en C et C++ necessitant d'utiliser le NDK \footnote{NDK
	ou Native Developpement Kit permet d'utiliser du code dit natif tel que le C
	ou C++ au sein des application Android}.
	Il se compose des librairies principales d'Android, de divers exemples
	d'applications et surtout d'un emulateur de terminal Android qui permet de
	pouvoir tester directement ses applications directement sur Linux, Mac ou
	Windows sans necessairrement avoir besoin d'un terminal mis à part pour
	OPENGL-ES 2 qui n'est pas encore prit en charge par celui-ci, sinon le kit de développement
	permet très bien de tester ses applications en temps réel sur tout
	peripheriques tournant sur Android que ce soit en liaison directe avec un
	ordinateur ou en récupérant l'archive de l'application au format APK.
	
	Le format APK ou Android Pakage est une variante du format
	JAR \footnote{JAR (Java ARchive) est un fichier ZIP utilisé pour distribuer un
	ensemble de classes Java. Ce format est utilisé pour stocker les définitions
	des  classes, ainsi que des métadonnées, constituant l'ensemble d'un
	programme.},  est utilisé pour la distribution et l'installation de composants
	regroupés sur le système d'exploitation Android.


\subsection{iOS}
	Apple est une entreprise multinationale américaine qui a été crée le 1\up{er} avril 1976 à Cupertino (Californie) par Steve Jobs et Steve Wozniak. Elle a pour but de conçoir et vendre des produits électroniques grand public, des ordinateurs personnels et des logiciels informatiques. Parmi les produits les plus répendus de l'entreprise se trouvent les ordinateurs Macintosh (1984), l'iPod \footnote{L'iPod est un baladeur numérique conçu et commercialisé par Apple} (2001), l'iPhone \footnote{L'iPhone est une famille de smartphones conçue et commercialisée par Apple}  (2007) et l'iPad \footnote{L'iPad est une tablette électronique conçue et commercialisée par Apple.} (2010). Apple a développé deux systèmes d'exploitation (Mac OS X \footnote{Mac OS X est une série de systèmes d’exploitation propriétaires développés et commercialisés par Apple, dont la version la plus récente est Mac OS X v10.6, dit Snow Léopard.} et iOS) permettant de contrôler ses différents produits, notamment les ordinateurs Macintosh, l'iPhone et l'iPod touch \footnote{L’iPod touch est un baladeur numérique à écran tactile capacitif multi-touch, conçu et commercialisé par Apple.}.
			
	iOS, connu sous le nom de iPhone OS avant Juin 2010, est le système d'exploitation mobile d'Apple. Ce dernier a été développé originellement pour l'iPhone, puis a été étendu pour l'iPod touch, iPad  et Apple TV \footnote{Apple TV est un appareil conçu et commercialisé par Apple permet la communication sans fil entre un ordinateur et un téléviseur.} (2007). C'est dérivé de Mac OS X  dont il partage les fondations. Comme celui-ci, iOS comporte quatre couches d'abstraction : une couche \og Core OS \fg \footnote{La couche \og Core OS \fg \,  contient les fonctionalitées bas niveau.}, une couche \og Core Services \fg \footnote{La couche \og Core Services \fg contient les services fondamental du système.}, une couche \og Media \fg \, et pour finir une couche \og Cocoa \fg \footnote{\og Cocoa \fg \,est une API native d'Apple pour le développement objet sur son système d'exploitation Mac OS X.}. Pour dévélopper une application Cocoa, il est imposé d'utiliser l'Objective-C comme langage de programmation.
			
	L'Objective-C est le langage de programmation orientés objet. Il a été inventé au début des années 1980 par Brad Cox, créateur de la société Stepstone. Son objectif était de combiner la richesse du langage Smalltalk\,\footnote{Smalltalk est l'un des premiers langages orientés objets créé en 1972.} et la rapidité du C.  C'est une extension du C ANSI \footnote{Le C est un langage de programmation impératif conçu pour la programmation système. Inventé au début des années 1970 avec UNIX.} qui ne permet pas l'héritage multiple contrairement au C++ \footnote{Le C++ est un langage de programmation permettant la programmation sous de multiples paradigmes comme la programmation procédurale, la programmation orientée objet et la programmation générique.}. Aujourd'hui, il est principalement utilisé pour développer des applications sur Mac OS X et iPhone. Une nouvelle version Objective-C 2.0 à été indrotuite avec Mac OS X 10.5 (Léopard) en octobre 2007.
			
	Le kit de développement d'iOS fournit les principaux outils nécessaire pour réaliser une application sur iPhone. Il est composé notamment de Xcode, Interface Builder, iPhone Simulator et Instruments. Xcode est l'outil de développement Apple, il permet la création de projets iPhone, l’édition du code source Objective-C, la compilation et le débogage des applications. Interface Builder quant à lui permet de construire des interfaces graphiques. L'iPhone Simulator est un logiciel simulant le comportement d'un iPhone, ce qui permet de pouvoir tester les applications directement sur l'ordinateur. Grâce à Instruments on peut analyser un programme pour surveiller l’état de la mémoire, l’utilisation du réseau, du CPU, etc. Pour utiliser tous ces outils, il est obligatoire de possèder un ordinateur Macintosh sous Mac OS X et de s'enregistrer sur iPhone Dev Center pour pouvoir télécharger et installer le SDK d'iOS.