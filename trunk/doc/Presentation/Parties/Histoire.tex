
\subsection{Histoire}

Le Bomberman est un jeu culte des années 80. Il a été conçu par l'éditeur japonais \href{http://www.hudsonsoft.net/}{Hudson Soft}. Sa silhouette, reconnaissable parmi toutes, est celle d'un poseur de bombes muni d'un casque avec des membres blancs et un corps bleu. Cependant, à la sortie du premier Bomberman en 1983 sur \gls{msx}, notre héros n'arborait pas cette apparence puisqu'il s'agissait en fait d'un ennemi issu de \gls{lode runner}. Ce n'est qu'en 1985 que l'aspect de Bomberman tel que nous le connaissons aujourd'hui est utilisé lors du portage sur \gls{nes}. C'est cette adaptation qui a réellement permis de faire connaitre le personnage en partie grâce à des mécaniques de jeu très simples.

Au fil du temps et des versions, le poseur de bombes gagne des pouvoirs de plus en plus grands. Ainsi, dans le premier épisode, le personnage ne sait que poser des bombes. Les bonus du jeu se limitent aux bombes, aux flammes et aux bottes permettant respectivement d'augmenter le nombre de bombes que l'on peut porter, la porté des explosions et la vitesse de déplacement. Dans les versions suivantes le personnage peut, après avoir trouvé les bonus correspondants, lancer des bombes ou les pousser, traverser les murs, etc. Certaines versions ont vu apparaitre des montures apportant de nouveaux pouvoirs et modes de déplacement.


\subsection{Principe}
Pour gagner, il faut détruire ses ennemis à l'aide de bombes et obtenir des bonus ou malus qui permettent d'augmenter ou diminuer ses chances de gagner. La maniabilité, le rythme effréné et la possibilité de s’affronter à quatre joueurs ont grandement contribués à sa popularité (dix millions d’exemplaires vendus depuis sa création). Sa simplicité, aussi bien dans le gameplay que dans les graphismes, semble être l’élément moteur de son succès puisque toutes les tentatives de suites, augmentant le nombre de bonus ou passant à la 3D, ont été boudées par le public. L’absence d’évolution du concept n’empêche aucunement à ce jeu d’être l’un des plus addictif du genre, et à son personnage de connaître une immense popularité. 