\section{Android}

\subsection{Le système d'exploitation}

\begin{frame}
\frametitle{Android}
\framesubtitle{Le système d'exploitation}
Android est un système basé sur un noyau Linux, développé par Google.
\begin{itemize}
	\item collaboration avec les membres de l'Open Handset Alliance (OHA)
\end{itemize}
\setbeamertemplate{blocks}[rounded,shadow=true]
\begin{block}{OHA}
consortium de compagnie rassemblant leurs compétences pour l'innovation dans le domaine de la mobilité.
\begin{itemize}
	\item Initié par Google en 2007
	\item Opérateur téléphonique, fabricants électroniques, sociétés commerciales
\end{itemize}
\end{block}
\end{frame}

\begin{frame}
\frametitle{Android}
\framesubtitle{Le système d'exploitation}
Structure:
\begin{itemize}
	\item Noyaux linux pour exploiter le matériel
	\item Librairies connues et open source (OpenGL ES, SQLite,...)
	\item Machine virtuelle Java (Dalvik virtual machine)
	\item API Java riche (package de Java SE, open source ou spécifique au système)
\end{itemize}
\end{frame}

\subsection{Développement}

\begin{frame}
\frametitle{Android}
\framesubtitle{Développement}
\begin{itemize}
	\item Langage principale Java, développement en C/C++ possible
	\item Kit de développement multiplateforme
	\item Développement sur téléphone ou sur émulateur
	\item Déploiement des applications peu coûteux
\end{itemize}
\end{frame}

\subsection{Les téléphones mobiles}

\begin{frame}
\frametitle{Android}
\framesubtitle{Les téléphones mobiles}
Les téléphones ont tous l'équipement nécessaire pour la réalité augmentée depuis le premier modèle.\\
\begin{block}{Inconvénient}
\begin{itemize}
	\item Fragmentation des téléphones et des versions du système.
\end{itemize}
\end{block}
%Support des angles de vue de la caméra depuis Android 2.2.\\
%$\Rightarrow$ 70\% des téléphones sous cette version
\end{frame}

\begin{frame}
\frametitle{Android}
\framesubtitle{Les téléphones mobiles}
\begin{block}{Fragmentation}
\begin{itemize}
	\item 7 versions du système en circulation
	\item Différentes dimensions et densités d'écran
	\item Smartphone, tablette (et télévision)
\end{itemize}
\end{block}
\textbf{Besoin:} applications compatibles sur tout les appareils.\\
\textbf{Solution:} compatibilité descendante des applications, système de ressource évolué.
\end{frame}

%\subsection{Structure des applications}

%\begin{frame}
%\frametitle{Android}
%\framesubtitle{Structure des applications}
%\begin{itemize}
%	\item AndroidManifest.xml: déclaration des composant de l'application, des permissions, de la compatibilité, ...
%	\item Ressources: images, descriptions d'interface, texte (multilingue), ...
%	\item Code exécutable: classe Java (package)
%\end{itemize}
%\end{frame}


