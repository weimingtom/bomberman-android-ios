\section{Application}
\subsection{Les différentes fenêtres}

\begin{frame}
\frametitle{Application}
\framesubtitle{Les différentes fenêtres}

\begin{tabular}{cc}
	\begin{minipage}{6cm}
		Lancement de l'application
		\begin{itemize}
			\item Fenêtre de lancement
			\item Fenêtre de réalité augmentée
			\item Fenêtre de la carte 2D
			\item Fenêtre de préférences
		\end{itemize}
	\end{minipage} &
	\begin{minipage}{6cm}
		\includegraphics[width=3cm]{ScreenShot/v3-hd.png} 
	\end{minipage}\\
\end{tabular}
\end{frame}

\begin{frame}
\frametitle{Application}
\framesubtitle{Réalité augmentée et Carte}
Les fenêtres de RA et de la carte 
\begin{center}
	\begin{tabular}{c}
		\includegraphics[width=5cm]{ScreenShot/RA_Activity.png} \\
		\includegraphics[width=5cm]{ScreenShot/Map_Activity.png}
	\end{tabular}
\end{center}
\end{frame}

%\begin{frame}
%\frametitle{Application}
%\framesubtitle{Activité de lancement}
%Vue de lancement \\
%\begin{center}\includegraphics[width=8cm]{ScreenShot/Start_Activity.png}\end{center}
%\end{frame}

%\begin{frame}
%\frametitle{Application}
%\framesubtitle{Préférences}
%Ecran des préférences \\
%\begin{center}\includegraphics[height=6cm]{ScreenShot/Preference_Activity.png}\end{center}
%\end{frame}

%\begin{frame}
%\frametitle{Application}
%\framesubtitle{Carte 2D}
%Ecran de la carte\\
%
%\begin{center}
%	\includegraphics[width=10cm]{ScreenShot/Map_retouche.png}
%\end{center}
%\end{frame}

%
%\begin{frame}
%\frametitle{Application}
%\framesubtitle{Réalité augmentée}
%Ecran de la réalité augmentée\\
%\begin{center}
%   	\includegraphics[width=8cm]{ScreenShot/RA_Activity.png}
%\end{center}
%\end{frame}



\subsection{Informations}


\begin{frame}
\frametitle{Application}
\framesubtitle{Menu choix des informations}
Ecran choix des informations\\

	\begin{tabular}{lll}
	\begin{minipage}{4.5cm}
		\begin{itemize}
			\item Les bâtiments 
			\item Les salles de cours, de TP
			\item Événements
			\item Lieux clés
		\end{itemize}
	\end{minipage} &
		\begin{minipage}{3cm}
	   		\includegraphics[width=3cm]{ScreenShot/Provider_Activity.png}
   		\end{minipage} &
   		\begin{minipage}{3cm}	
			\includegraphics[width=3cm]{ScreenShot/Provider_cle.jpeg}
		\end{minipage}	\\
	\end{tabular}

\end{frame}


\begin{frame}
\frametitle{Application}
\framesubtitle{Choix des informations}

Vous avez choisi ? Vous pouvez voir :

\begin{itemize}
	\item L'information qui s'y trouve
	\item Les contours des bâtiments
	\item Le chemin pour s'y rendre
\end{itemize}

 \textit{Valable pour la carte et la RA via le menu Préférences}
\end{frame}

\subsection{Accès aux informations}

\begin{frame}
\frametitle{Application}
\framesubtitle{Base de données SQLite}

\begin{center}
\begin{tabular}{cc}
	\begin{minipage}{6cm}
		Base de données SQLite
		\begin{itemize}
			\item Toutes informations relatives aux positions
			\item Données de Open Street Map
			\item Données récupérées à la main
		\end{itemize}
	\end{minipage}  &		
	\begin{minipage}{6cm}
		\includegraphics[width=3cm]{img/sqlite.jpeg} 
	\end{minipage}\\
\end{tabular}
\end{center}
\end{frame}

\begin{frame}
\frametitle{Application}
\framesubtitle{Service Web}

\begin{center}
\begin{tabular}{ll}
	\begin{minipage}{5cm}

		Service Web
		\begin{itemize}
			\item Utilise une petite base de données: mySQL
			\item Permet l'ajout de points d'intérêts
		\end{itemize}
	\end{minipage}  &		
	\begin{minipage}{7cm}
		\includegraphics[width=6.5cm]{img/WebService.png} 
	\end{minipage}\\
\end{tabular}
\end{center}

\end{frame}

\subsection{Affichage des informations}

\begin{frame}
\frametitle{Application}
\framesubtitle{Graphismes}

\begin{center}
\begin{tabular}{cc}
	\begin{minipage}{6cm}

		Points d'intérêts
		\begin{itemize}
			\item Indicateur 
		\end{itemize}
		Dessins correspondants
		\begin{itemize}
			\item Dessins des contours
			\item Dessins des chemins 
		\end{itemize}
	\end{minipage}  &		
	\begin{minipage}{4cm}
		\includegraphics[width=2cm]{img/indicateur.png} 
		\includegraphics[width=3cm]{img/drawable.png} 
	\end{minipage}\\
\end{tabular}
\end{center}
\begin{center}
\begin{tabular}{cc}
	\begin{minipage}{6cm}

		Open GL ES
		\begin{itemize}
			\item Conception d'objet 3D
			\item Flèche directionnelle
		\end{itemize}
	\end{minipage}  &		
	\begin{minipage}{4cm}
		\includegraphics[width=2.5cm]{img/opengles.png} \\
		\includegraphics[width=1.2cm]{img/fleche_lumiere.png}
	\end{minipage}\\
\end{tabular}
\end{center}
\end{frame}

\begin{frame}
\frametitle{Application}
\framesubtitle{Les capteurs}

Les capteurs
\begin{itemize}
	\item Rappel : caméra, boussole, accéléromètre, GPS
	\item Pour le calcul des angles de vue
	\item Faire bouger les éléments graphiques
	\item Se positionner et positionner les éléments
\end{itemize}

\end{frame}

\begin{frame}
\frametitle{Application}
\framesubtitle{Calcul du plus court chemin}

\begin{center}
\begin{tabular}{cc}
	\begin{minipage}{6cm}
		Calcul du plus court chemin
		\begin{itemize}
			\item Djikstra
			\item Noeuds stockés dans la base de données SQLite
		\end{itemize}
	\end{minipage}  &		
	\begin{minipage}{6cm}
		\includegraphics[width=4cm]{img/mapPath.png} 
	\end{minipage}\\
\end{tabular}
\end{center}
\end{frame}

\subsection{Evolution}

\begin{frame}
\frametitle{Application}
\framesubtitle{Evolution - Réutisabilité}

Evolution:
\begin{itemize}
	\item Affichage des emplois du temps
	\item Affichage de l'emploi du temps personnel d'un étudiant 
	\item Jeux
\end{itemize}

Réutisabilité:
\begin{itemize}
	\item Au sein d'une autre université
	\item N'importe quelle zone géographique
\end{itemize}
En ajoutant ou remplaçant les implémentations.
\end{frame}



\begin{frame}
\frametitle{Application}
\framesubtitle{Evolution - Réutisabilité}

\includegraphics[width=11cm]{img/UML.png} 
\end{frame}







