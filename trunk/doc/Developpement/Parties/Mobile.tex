\subsection{Menus}
	\subsubsection{API et widget}
	\paragraph{Android\\}
		Afin de créer les différents menus de notre application, Android met à la
		disposition des developpeurs une API\footnote{API : Application Programming
		Interface, c'est un ensemble de classes mis à disposition par une
		bibliothèque logicielle.} très bien fournie. Parmis celle-ci le package Widget
		nous a été très utile. Grâce à ce dernier de nombreux objets ont été utilisé
		afin de mettre en oeuvre et rendre pleinement fonctionnel nos menus.
		Parmi les plus utilisés il y a eu bien sûr Button, TextView, CheckBox et
		EditText pour les plus explicites. Les objets comme Spinner, SeekBar et
		Gallery étant respectivement utilisés pour les menus déroulants, les barres de
		progression et les galeries d'images pour la sélection de cartes de jeu.
		
		Concernant leur positionnement, un système de layout est utilisé. Une syntaxe
		de balises xml est là aussi à disposition afin de mettre en forme tout objet
		présent sous forme de vue. Ils sont accessibles via des id uniques par
		fichiers xml, qui permettent de les identifiers. Les listeners présents dans
		les classes java permettent à leur tour d'écouter les évènements utilisateurs
		ou system et réagir en fonction comme l'accès au menu suivant ou le lancement
		d'un partie.
	\paragraph{iOS\\}
		
	\subsubsection{BDD}

		Après avoir effectué divers recherches, il s'est avéré que les mobiles
		utilisent un moteur de base de données relationnelles, accessible par le
		langage SQL. Dans notre cas il s'agit de SQLite 3. Sa particularité est de ne
		pas reproduire le schéma habituel client-serveur mais d'être directement intégrée au programme.
		L'accès à la base de données SQLite se fait par l'ouverture du fichier
		correspondant à celle-ci : chaque base de données est enregistrée dans un fichier qui lui est propre,
		 avec ses déclarations, ses tables, ses index mais aussi ses données.
			
		\paragraph{Android\\}
			
			Pour manipuler aisément les bases de données depuis l'application,
			nous avons crée une classe héritant de \textit{SQLiteOpenHelper}. Cette
			dernière fournit des outils de manipulations. Un attribut y est
			instancié, il s'agit de la base de données elle même, de type
			\textit{SQLiteDatabase}.
			
			Nous y avons crée 3 tables, \textit{PlayerAccount} sauvegardant toutes les
			informations sur les utilisateurs locaux, \textit{System} concervant les
			propriétés du système, et enfin \textit{Map} décrivant les informations
			relatives au cartes de jeu crées par l'utilisateur.
			
			Ainsi de nombreuses	fonctions ont été implémenté dans le but de simplifier les interactions
			avec cette base de données depuis l'application. Il est par exemple possible de créer un nouvel 
			utilisateur local, modifier ses préférences, gérer les configurations systèmes comme la langue ou le volume
			du son, ajoûter de nouvelles maps ou même récupérer toutes les informations
			concernant un utilisateur.\\
			
			Voici un exemple d'insertion d'un nouveau compte local dans la base de
			donnée. Rappelons que les tests d'existance du compte ont été fait depuis
			l'application même. Dans cet exemple vous verrez ainsi que nous commençons
			par récupérer les droits en écritures sur la base de données locale, puis
			nous créons un container qui servira à l'insertion de valeur dans la base. Et
			enfin l'insertion est faite. Nous terminons tout de même en fermant l'accès à
			cette base.
			
			Il s'agit la d'un schéma classique de fonction d'interaction avec notre
			base.
						
			\begin{verbatim}
			/** ajout compte hors ligne **/
				public long newAccount(String nomCompte){
					base = getWritableDatabase();
			
					ContentValues entree = new ContentValues();
					
					entree.put("pseudo", nomCompte);
					long var = base.insert("PlayerAccount", null, entree);
					
					base.close();
					return var;
				}
			\end{verbatim}

			
		\paragraph{iOS}
				
	\subsubsection{Première utilisation}
	\subsubsection{Création utilisateur}
	\subsubsection{Gestion utilisateur}
	\subsubsection{Gestion des préférences système}
	\subsubsection{Création de carte (charger)}
	\subsubsection{Création partie solo (tout)}
	\subsubsection{Création partie multi (officielle)}
			

\subsection{Editeur de carte}
	\subsubsection{Rendu}
		\subsubsection{Interface utilisateur}
		\subsubsection{Sauvegarde}


\subsection{Jeu}
	\subsubsection{Moteurs}
		\paragraph{Rendu}
			\subparagraph{Structure utilisée}
				\begin{itemize}
					\item{Pourquoi}
					\item{Avantages}
				\end{itemize}
		\paragraph{Physique}
			\subparagraph{Structure utilisée}
			\subparagraph{Mouvements (collisions)}
			\subparagraph{Gestion des bombes}
				\begin{itemize}
					\item{Threads}
				\end{itemize}
	
	\subsubsection{IA}
		\paragraph{Pathfinding}
			\subparagraph{A*}
			\subparagraph{Aléatoire}
			\paragraph{Prise de décision}
	
	\subsubsection{Interface utilisaeur}
		\paragraph{Android}
		\paragraph{iOS}