\subsection{Menus}
	\subsubsection{API et widget}

			TODO ludo
		
	\subsubsection{BDD}

		Après avoir effectué divers recherches, il s'est avéré que le gestionnaire
		de bases de données utilisé sur les deux plateformes est SQLite 3. 
			
		\paragraph{Android\\}
			
			Pour manipuler aisément les bases de données depuis l'application,
			nous avons crée une classe héritant de \textit{SQLiteOpenHelper}. Cette
			dernière fournit des outils de manipulations. Un attribut y est
			instancié, il s'agit de la base de données elle même, de type
			\textit{SQLiteDatabase}.
			
			Nous y avons crée 3 Tables, \textit{PlayerAccount} sauvegardant toutes les
			informations sur les utilisateurs locaux, \textit{System} concervant les
			propriétés du système, et enfin \textit{Map} décrivant les informations
			relatives au cartes de jeu crées par l'utilisateur.
			
			Ainsi de nombreuses	fonctions ont été implémenté dans le but de simplifier les interactions
			avec cette base de données depuis l'application. Il est par exemple possible de créer un nouvel 
			utilisateur local, modifier ses préférences, gérer les configurations systèmes comme la langue ou le volume
			du son, ajoûter de nouvelles maps ou même récupérer toutes les informations
			concernant un utilisateur.\\
			
			Voici un exemple d'insertion d'un nouveau compte local dans la base de
			donnée. Rappelons que les tests d'existance du compte ont été fait depuis
			l'application même. Dans cet exemple vous verrez ainsi que nous commençons
			par récupérer les droits en écritures sur la base de données locale, puis
			nous créons un container qui servira à l'insertion de valeur dans la base. Et
			enfin l'insertion est faite. Nous terminons tout de même en fermant l'accès à
			cette base.
			
			Il s'agit la d'un schéma classique de fonction d'interaction avec notre
			base.
						
			\begin{verbatim}
			/** ajout compte hors ligne **/
				public long newAccount(String nomCompte){
					base = getWritableDatabase();
			
					ContentValues entree = new ContentValues();
					
					entree.put("pseudo", nomCompte);
					long var = base.insert("PlayerAccount", null, entree);
					
					base.close();
					return var;
				}
			\end{verbatim}

			
		\paragraph{iOS}
				
	\subsubsection{Première utilisation}
	\subsubsection{Création utilisateur}
	\subsubsection{Gestion utilisateur}
	\subsubsection{Gestion des préférences système}
	\subsubsection{Création de carte (charger)}
	\subsubsection{Création partie solo (tout)}
	\subsubsection{Création partie multi (officielle)}
			

\subsection{Editeur de carte}
	\subsubsection{Rendu}
		\subsubsection{Interface utilisateur}
		\subsubsection{Sauvegarde}


\subsection{Jeu}
	\subsubsection{Moteurs}
		\paragraph{Rendu}
			\subparagraph{Structure utilisée}
				\begin{itemize}
					\item{Pourquoi}
					\item{Avantages}
				\end{itemize}
		\paragraph{Physique}
			\subparagraph{Structure utilisée}
			\subparagraph{Mouvements (collisions)}
			\subparagraph{Gestion des bombes}
				\begin{itemize}
					\item{Threads}
				\end{itemize}
	
	\subsubsection{IA}
		\paragraph{Pathfinding}
			\subparagraph{A*}
			\subparagraph{Aléatoire}
			\paragraph{Prise de décision}
	
	\subsubsection{Interface utilisaeur}
		\paragraph{Android}
		\paragraph{iOS}