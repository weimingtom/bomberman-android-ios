\documentclass{report}

\usepackage[utf8]{inputenc}
\usepackage[french]{babel}
\usepackage{a4wide}

\usepackage{pstricks}

\usepackage{hyperref}

\begin{document}

\chapter{Annexes}

	\section{Scénarios}
	
		Description de la partie \ldots
		
		\newpage

\newpage
	
	\subsection{Chargement de l'application}
	
		\hypertarget{Chargement de l'application}{}
		\label{Chargement de l'application}

		%LaTeX with PSTricks extensions
%%Creator: inkscape 0.48.0
%%Please note this file requires PSTricks extensions
\psset{xunit=.5pt,yunit=.5pt,runit=.5pt}
\begin{pspicture}(560,600)
{
\newrgbcolor{curcolor}{1 1 1}
\pscustom[linestyle=none,fillstyle=solid,fillcolor=curcolor]
{
\newpath
\moveto(133.12401716,597.5221417)
\lineto(426.87598554,597.5221417)
\curveto(443.85397304,597.5221417)(457.52217237,583.85394237)(457.52217237,566.87595487)
\lineto(457.52217237,33.124017)
\curveto(457.52217237,16.1460295)(443.85397304,2.47783017)(426.87598554,2.47783017)
\lineto(133.12401716,2.47783017)
\curveto(116.14602965,2.47783017)(102.47783033,16.1460295)(102.47783033,33.124017)
\lineto(102.47783033,566.87595487)
\curveto(102.47783033,583.85394237)(116.14602965,597.5221417)(133.12401716,597.5221417)
\closepath
}
}
{
\newrgbcolor{curcolor}{0 0 0}
\pscustom[linewidth=4.95566034,linecolor=curcolor]
{
\newpath
\moveto(133.12401716,597.5221417)
\lineto(426.87598554,597.5221417)
\curveto(443.85397304,597.5221417)(457.52217237,583.85394237)(457.52217237,566.87595487)
\lineto(457.52217237,33.124017)
\curveto(457.52217237,16.1460295)(443.85397304,2.47783017)(426.87598554,2.47783017)
\lineto(133.12401716,2.47783017)
\curveto(116.14602965,2.47783017)(102.47783033,16.1460295)(102.47783033,33.124017)
\lineto(102.47783033,566.87595487)
\curveto(102.47783033,583.85394237)(116.14602965,597.5221417)(133.12401716,597.5221417)
\closepath
}
}
{
\newrgbcolor{curcolor}{1 1 1}
\pscustom[linestyle=none,fillstyle=solid,fillcolor=curcolor]
{
\newpath
\moveto(260.00365964,158.92373777)
\lineto(309.96551649,158.92373777)
\lineto(309.96551649,110.05766797)
\lineto(260.00365964,110.05766797)
\closepath
}
}
{
\newrgbcolor{curcolor}{0 0 0}
\pscustom[linewidth=2,linecolor=curcolor]
{
\newpath
\moveto(260.00365964,158.92373777)
\lineto(309.96551649,158.92373777)
\lineto(309.96551649,110.05766797)
\lineto(260.00365964,110.05766797)
\closepath
}
}
{
\newrgbcolor{curcolor}{0 0 0}
\pscustom[linestyle=none,fillstyle=solid,fillcolor=curcolor]
{
\newpath
\moveto(43.96874944,502.65625119)
\lineto(49.12499944,502.65625119)
\lineto(49.12499944,520.45312619)
\lineto(43.51562444,519.32812619)
\lineto(43.51562444,522.20312619)
\lineto(49.09374944,523.32812619)
\lineto(52.24999944,523.32812619)
\lineto(52.24999944,502.65625119)
\lineto(57.40624944,502.65625119)
\lineto(57.40624944,500.00000119)
\lineto(43.96874944,500.00000119)
\lineto(43.96874944,502.65625119)
}
}
{
\newrgbcolor{curcolor}{0 0 0}
\pscustom[linestyle=none,fillstyle=solid,fillcolor=curcolor]
{
\newpath
\moveto(506.14062254,122.65622068)
\lineto(517.15624754,122.65622068)
\lineto(517.15624754,119.99997068)
\lineto(502.34374754,119.99997068)
\lineto(502.34374754,122.65622068)
\curveto(503.54166066,123.89580011)(505.17186736,125.55725679)(507.23437254,127.64059568)
\curveto(509.30727989,129.73433594)(510.60936193,131.08329293)(511.14062254,131.68747068)
\curveto(512.15102705,132.82287452)(512.85415135,133.78120689)(513.24999754,134.56247068)
\curveto(513.65623388,135.35412199)(513.85935868,136.13016288)(513.85937254,136.89059568)
\curveto(513.85935868,138.13016088)(513.42185911,139.14057654)(512.54687254,139.92184568)
\curveto(511.68227752,140.70307497)(510.55207032,141.09369958)(509.15624754,141.09372068)
\curveto(508.16665604,141.09369958)(507.11978208,140.92182475)(506.01562254,140.57809568)
\curveto(504.92186761,140.23432544)(503.74999379,139.71349263)(502.49999754,139.01559568)
\lineto(502.49999754,142.20309568)
\curveto(503.7708271,142.71348963)(504.95832591,143.09890591)(506.06249754,143.35934568)
\curveto(507.16665704,143.61973872)(508.17707269,143.74994693)(509.09374754,143.74997068)
\curveto(511.51040269,143.74994693)(513.4374841,143.14578086)(514.87499754,141.93747068)
\curveto(516.31248122,140.72911661)(517.0312305,139.11453489)(517.03124754,137.09372068)
\curveto(517.0312305,136.13537121)(516.84893902,135.22391379)(516.48437254,134.35934568)
\curveto(516.13018974,133.5051655)(515.47914872,132.49474985)(514.53124754,131.32809568)
\curveto(514.2708166,131.02600132)(513.44269243,130.15100219)(512.04687254,128.70309568)
\curveto(510.65102855,127.26558841)(508.68228052,125.24996543)(506.14062254,122.65622068)
}
}
{
\newrgbcolor{curcolor}{0 0 0}
\pscustom[linewidth=2,linecolor=curcolor,linestyle=dashed,dash=8 8]
{
\newpath
\moveto(150,510)
\lineto(60,510)
}
}
{
\newrgbcolor{curcolor}{0 0 0}
\pscustom[linestyle=none,fillstyle=solid,fillcolor=curcolor]
{
\newpath
\moveto(139.53769464,514.84048224)
\lineto(152.6487474,510.01921591)
\lineto(139.53769392,505.19795064)
\curveto(141.632292,508.04442372)(141.62022288,511.93889292)(139.53769464,514.84048224)
\lineto(139.53769464,514.84048224)
\closepath
}
}
{
\newrgbcolor{curcolor}{0 0 0}
\pscustom[linewidth=2,linecolor=curcolor,linestyle=dashed,dash=8 8]
{
\newpath
\moveto(290,130)
\lineto(500,130)
}
}
{
\newrgbcolor{curcolor}{0 0 0}
\pscustom[linestyle=none,fillstyle=solid,fillcolor=curcolor]
{
\newpath
\moveto(300.46230536,125.15951776)
\lineto(287.3512526,129.98078409)
\lineto(300.46230608,134.80204936)
\curveto(298.367708,131.95557628)(298.37977712,128.06110708)(300.46230536,125.15951776)
\lineto(300.46230536,125.15951776)
\closepath
}
}
\end{pspicture}

		
		\subsubsection{Déscription des zones}

			\begin{tabular}{|c|c|c|c|c|} \hline
				Numéro de zone & Type  & Description & Evenement &	Règle \\\hline 
				1 & Image & Fond d'écran présent dans l'archive, &Lancement de l'application &RG001\\\ 
				 &  & stocké sur le téléphone &	&\\\hline
				2 & Image & Spinner présent dansl'archive, & Lancement de l'application &	RG001\\
				 & & stocké sur le téléphone & &\\\hline
			\end{tabular}
		
		\subsubsection{Description des règles}
		
		\underline{RG001 :}
		\begin{quote}
			Déclenchée au démarrage de l'application.\\
			Affichage du fond d'écran recadré en fonction de la taille de l'écran.\\
			Affichage du spinner montrant le chargement de l'application.\\
			\underline{Chargement de l'application :}
				\begin{quote}
					Chargement de la page d'accueil%
					\footnote[1]{
						\hyperlink{Page d'accueil}{Page d'accueil}
						\og voir section \ref{accueil}, page \pageref{accueil}.\fg
					}.\\
					Vérification d'un compte local existant (stocké sur le téléphone).\\
					Si aucun compte n'est trouvé alors la page de création du profil%
					\footnote[2]{
						\hyperlink{Création du profil}{Page de création du profil}
						\og voir section \ref{profil}, page \pageref{profil}.\fg
					}
					est chargée puis affichée.\\
					Sinon la page d'accueil%
					\footnotemark[1]
					est affichée directement. \\
					La page représentant le chargement de l'application est détruite une fois le chargement de l'application terminé.\\
				\end{quote}
		\end{quote}

	
\newpage

	\subsection{Création du profil}
		\hypertarget{Création du profil}{}
		\label{profil}
	
		%LaTeX with PSTricks extensions
%%Creator: inkscape 0.48.0
%%Please note this file requires PSTricks extensions
\psset{xunit=.4pt,yunit=.4pt,runit=.4pt}
\begin{pspicture}(560,600)
{
\newrgbcolor{curcolor}{1 1 1}
\pscustom[linestyle=none,fillstyle=solid,fillcolor=curcolor]
{
\newpath
\moveto(133.17790741,598.5302021)
\lineto(426.82210296,598.5302021)
\curveto(443.82879388,598.5302021)(457.52010101,584.83889497)(457.52010101,567.83220405)
\lineto(457.52010101,33.17790718)
\curveto(457.52010101,16.17121626)(443.82879388,2.47990913)(426.82210296,2.47990913)
\lineto(133.17790741,2.47990913)
\curveto(116.17121649,2.47990913)(102.47990936,16.17121626)(102.47990936,33.17790718)
\lineto(102.47990936,567.83220405)
\curveto(102.47990936,584.83889497)(116.17121649,598.5302021)(133.17790741,598.5302021)
\closepath
}
}
{
\newrgbcolor{curcolor}{0 0 0}
\pscustom[linewidth=4.95981836,linecolor=curcolor]
{
\newpath
\moveto(133.17790741,598.5302021)
\lineto(426.82210296,598.5302021)
\curveto(443.82879388,598.5302021)(457.52010101,584.83889497)(457.52010101,567.83220405)
\lineto(457.52010101,33.17790718)
\curveto(457.52010101,16.17121626)(443.82879388,2.47990913)(426.82210296,2.47990913)
\lineto(133.17790741,2.47990913)
\curveto(116.17121649,2.47990913)(102.47990936,16.17121626)(102.47990936,33.17790718)
\lineto(102.47990936,567.83220405)
\curveto(102.47990936,584.83889497)(116.17121649,598.5302021)(133.17790741,598.5302021)
\closepath
}
}
{
\newrgbcolor{curcolor}{1 1 1}
\pscustom[linestyle=none,fillstyle=solid,fillcolor=curcolor]
{
\newpath
\moveto(164.38434976,370.09907455)
\lineto(396.60310358,370.09907455)
\curveto(409.61615135,370.09907455)(420.09235948,359.62286642)(420.09235948,346.60981865)
\lineto(420.09235948,243.48925514)
\curveto(420.09235948,230.47620737)(409.61615135,219.99999924)(396.60310358,219.99999924)
\lineto(164.38434976,219.99999924)
\curveto(151.37130199,219.99999924)(140.89509386,230.47620737)(140.89509386,243.48925514)
\lineto(140.89509386,346.60981865)
\curveto(140.89509386,359.62286642)(151.37130199,370.09907455)(164.38434976,370.09907455)
\closepath
}
}
{
\newrgbcolor{curcolor}{0 0 0}
\pscustom[linewidth=1.75071895,linecolor=curcolor]
{
\newpath
\moveto(164.38434976,370.09907455)
\lineto(396.60310358,370.09907455)
\curveto(409.61615135,370.09907455)(420.09235948,359.62286642)(420.09235948,346.60981865)
\lineto(420.09235948,243.48925514)
\curveto(420.09235948,230.47620737)(409.61615135,219.99999924)(396.60310358,219.99999924)
\lineto(164.38434976,219.99999924)
\curveto(151.37130199,219.99999924)(140.89509386,230.47620737)(140.89509386,243.48925514)
\lineto(140.89509386,346.60981865)
\curveto(140.89509386,359.62286642)(151.37130199,370.09907455)(164.38434976,370.09907455)
\closepath
}
}
{
\newrgbcolor{curcolor}{0 0 0}
\pscustom[linestyle=none,fillstyle=solid,fillcolor=curcolor]
{
\newpath
\moveto(164.7226581,345.55079574)
\lineto(164.7226581,338.97657699)
\lineto(167.6992206,338.97657699)
\curveto(168.8007743,338.97656802)(169.65233595,339.26172398)(170.2539081,339.83204574)
\curveto(170.85545974,340.40234784)(171.15624069,341.21484703)(171.15625185,342.26954574)
\curveto(171.15624069,343.31640743)(170.85545974,344.12500037)(170.2539081,344.69532699)
\curveto(169.65233595,345.26562423)(168.8007743,345.55078019)(167.6992206,345.55079574)
\lineto(164.7226581,345.55079574)
\moveto(162.3554706,347.49610824)
\lineto(167.6992206,347.49610824)
\curveto(169.66014844,347.49609075)(171.14061571,347.05077869)(172.14062685,346.16017074)
\curveto(173.1484262,345.27734297)(173.65233195,343.98046926)(173.6523456,342.26954574)
\curveto(173.65233195,340.5429727)(173.1484262,339.23828651)(172.14062685,338.35548324)
\curveto(171.14061571,337.47266327)(169.66014844,337.03125746)(167.6992206,337.03126449)
\lineto(164.7226581,337.03126449)
\lineto(164.7226581,330.00001449)
\lineto(162.3554706,330.00001449)
\lineto(162.3554706,347.49610824)
}
}
{
\newrgbcolor{curcolor}{0 0 0}
\pscustom[linestyle=none,fillstyle=solid,fillcolor=curcolor]
{
\newpath
\moveto(184.6914081,342.73829574)
\lineto(184.6914081,340.69923324)
\curveto(184.08202308,341.01172223)(183.44921121,341.246097)(182.7929706,341.40235824)
\curveto(182.13671252,341.55859669)(181.4570257,341.63672161)(180.7539081,341.63673324)
\curveto(179.68358998,341.63672161)(178.87890328,341.47265927)(178.3398456,341.14454574)
\curveto(177.80859185,340.81640993)(177.54296712,340.32422292)(177.5429706,339.66798324)
\curveto(177.54296712,339.16797408)(177.73437318,338.77344322)(178.11718935,338.48438949)
\curveto(178.49999741,338.20313129)(179.26952789,337.93360031)(180.4257831,337.67579574)
\lineto(181.16406435,337.51173324)
\curveto(182.69530571,337.18360106)(183.78124213,336.71875778)(184.42187685,336.11720199)
\curveto(185.07030334,335.52344647)(185.39452177,334.69141605)(185.3945331,333.62110824)
\curveto(185.39452177,332.40235584)(184.91014725,331.43751306)(183.9414081,330.72657699)
\curveto(182.98046168,330.01563948)(181.65624425,329.66017108)(179.96875185,329.66017074)
\curveto(179.26562164,329.66017108)(178.53124738,329.73048351)(177.76562685,329.87110824)
\curveto(177.0078114,330.00392074)(176.20703095,330.20704554)(175.3632831,330.48048324)
\lineto(175.3632831,332.70704574)
\curveto(176.160156,332.29298095)(176.94531146,331.98048126)(177.71875185,331.76954574)
\curveto(178.49218492,331.56641918)(179.25780915,331.46485678)(180.01562685,331.46485824)
\curveto(181.03124488,331.46485678)(181.8124941,331.63673161)(182.35937685,331.98048324)
\curveto(182.906243,332.33204341)(183.17968023,332.82423042)(183.17968935,333.45704574)
\curveto(183.17968023,334.0429792)(182.98046168,334.4921975)(182.5820331,334.80470199)
\curveto(182.19139997,335.11719688)(181.32811958,335.41797783)(179.99218935,335.70704574)
\lineto(179.24218935,335.88282699)
\curveto(177.906248,336.16407083)(176.94140522,336.5937579)(176.3476581,337.17188949)
\curveto(175.75390641,337.75781924)(175.4570317,338.55859969)(175.4570331,339.57423324)
\curveto(175.4570317,340.80859744)(175.89453127,341.76172148)(176.7695331,342.43360824)
\curveto(177.64452952,343.10547014)(178.88671577,343.4414073)(180.4960956,343.44142074)
\curveto(181.29296337,343.4414073)(182.04296262,343.38281361)(182.7460956,343.26563949)
\curveto(183.44921121,343.14843885)(184.09764806,342.97265777)(184.6914081,342.73829574)
}
}
{
\newrgbcolor{curcolor}{0 0 0}
\pscustom[linestyle=none,fillstyle=solid,fillcolor=curcolor]
{
\newpath
\moveto(200.0664081,337.10157699)
\lineto(200.0664081,336.04688949)
\lineto(190.1523456,336.04688949)
\curveto(190.24609193,334.56250993)(190.69140398,333.42969856)(191.4882831,332.64845199)
\curveto(192.29296488,331.87501262)(193.41015127,331.48829426)(194.8398456,331.48829574)
\curveto(195.66796151,331.48829426)(196.46874196,331.58985665)(197.24218935,331.79298324)
\curveto(198.0234279,331.99610625)(198.79686463,332.30079344)(199.56250185,332.70704574)
\lineto(199.56250185,330.66798324)
\curveto(198.78905214,330.3398579)(197.99608418,330.08985815)(197.1835956,329.91798324)
\curveto(196.3710858,329.7461085)(195.54686788,329.66017108)(194.71093935,329.66017074)
\curveto(192.61718331,329.66017108)(190.95702872,330.26954547)(189.7304706,331.48829574)
\curveto(188.51171866,332.70704304)(187.90234427,334.35547889)(187.9023456,336.43360824)
\curveto(187.90234427,338.58203716)(188.4804687,340.28516046)(189.6367206,341.54298324)
\curveto(190.80077887,342.80859544)(192.36718356,343.4414073)(194.33593935,343.44142074)
\curveto(196.10155482,343.4414073)(197.49608468,342.87109537)(198.5195331,341.73048324)
\curveto(199.55077012,340.59766015)(200.06639461,339.05469294)(200.0664081,337.10157699)
\moveto(197.9101581,337.73438949)
\curveto(197.89452178,338.91406808)(197.56249086,339.85547339)(196.91406435,340.55860824)
\curveto(196.27342965,341.26172198)(195.421868,341.61328413)(194.35937685,341.61329574)
\curveto(193.15624527,341.61328413)(192.19140248,341.27344072)(191.4648456,340.59376449)
\curveto(190.74609143,339.91406708)(190.33202934,338.95703679)(190.2226581,337.72267074)
\lineto(197.9101581,337.73438949)
}
}
{
\newrgbcolor{curcolor}{0 0 0}
\pscustom[linestyle=none,fillstyle=solid,fillcolor=curcolor]
{
\newpath
\moveto(203.38281435,335.17970199)
\lineto(203.38281435,343.12501449)
\lineto(205.53906435,343.12501449)
\lineto(205.53906435,335.26173324)
\curveto(205.53906015,334.01954172)(205.78124741,333.08594891)(206.26562685,332.46095199)
\curveto(206.74999644,331.84376265)(207.47655821,331.53516921)(208.44531435,331.53517074)
\curveto(209.60936858,331.53516921)(210.52733641,331.90626259)(211.1992206,332.64845199)
\curveto(211.87889756,333.3906361)(212.21874097,334.40235384)(212.21875185,335.68360824)
\lineto(212.21875185,343.12501449)
\lineto(214.37500185,343.12501449)
\lineto(214.37500185,330.00001449)
\lineto(212.21875185,330.00001449)
\lineto(212.21875185,332.01563949)
\curveto(211.695304,331.21876328)(211.08592961,330.62501387)(210.39062685,330.23438949)
\curveto(209.70311849,329.85157714)(208.90233804,329.66017108)(207.9882831,329.66017074)
\curveto(206.48046546,329.66017108)(205.33593536,330.12892062)(204.55468935,331.06642074)
\curveto(203.77343692,332.00391874)(203.38281231,333.37501112)(203.38281435,335.17970199)
\moveto(208.8085956,343.44142074)
\lineto(208.8085956,343.44142074)
}
}
{
\newrgbcolor{curcolor}{0 0 0}
\pscustom[linestyle=none,fillstyle=solid,fillcolor=curcolor]
{
\newpath
\moveto(227.47656435,341.13282699)
\lineto(227.47656435,348.23438949)
\lineto(229.63281435,348.23438949)
\lineto(229.63281435,330.00001449)
\lineto(227.47656435,330.00001449)
\lineto(227.47656435,331.96876449)
\curveto(227.0234289,331.18751331)(226.44921073,330.60548264)(225.7539081,330.22267074)
\curveto(225.06639961,329.8476709)(224.23827544,329.66017108)(223.2695331,329.66017074)
\curveto(221.68359049,329.66017108)(220.39062304,330.29298295)(219.39062685,331.55860824)
\curveto(218.39843753,332.82423042)(217.90234427,334.48829126)(217.9023456,336.55079574)
\curveto(217.90234427,338.61328713)(218.39843753,340.27734797)(219.39062685,341.54298324)
\curveto(220.39062304,342.80859544)(221.68359049,343.4414073)(223.2695331,343.44142074)
\curveto(224.23827544,343.4414073)(225.06639961,343.25000124)(225.7539081,342.86720199)
\curveto(226.44921073,342.4921895)(227.0234289,341.91406508)(227.47656435,341.13282699)
\moveto(220.1289081,336.55079574)
\curveto(220.12890455,334.96485328)(220.45312297,333.71876078)(221.10156435,332.81251449)
\curveto(221.75780917,331.91407508)(222.65624577,331.46485678)(223.79687685,331.46485824)
\curveto(224.93749349,331.46485678)(225.83593009,331.91407508)(226.49218935,332.81251449)
\curveto(227.14842878,333.71876078)(227.47655345,334.96485328)(227.47656435,336.55079574)
\curveto(227.47655345,338.13672511)(227.14842878,339.37891137)(226.49218935,340.27735824)
\curveto(225.83593009,341.18359706)(224.93749349,341.63672161)(223.79687685,341.63673324)
\curveto(222.65624577,341.63672161)(221.75780917,341.18359706)(221.10156435,340.27735824)
\curveto(220.45312297,339.37891137)(220.12890455,338.13672511)(220.1289081,336.55079574)
}
}
{
\newrgbcolor{curcolor}{0 0 0}
\pscustom[linestyle=none,fillstyle=solid,fillcolor=curcolor]
{
\newpath
\moveto(239.1601581,341.61329574)
\curveto(238.00390191,341.61328413)(237.08984032,341.16015958)(236.4179706,340.25392074)
\curveto(235.74609166,339.35547389)(235.4101545,338.12110012)(235.4101581,336.55079574)
\curveto(235.4101545,334.98047826)(235.74218542,333.74219825)(236.40625185,332.83595199)
\curveto(237.07812158,331.93751256)(237.99608941,331.48829426)(239.1601581,331.48829574)
\curveto(240.3085871,331.48829426)(241.21874244,331.9414188)(241.89062685,332.84767074)
\curveto(242.5624911,333.75391699)(242.89842826,334.98829076)(242.89843935,336.55079574)
\curveto(242.89842826,338.10547514)(242.5624911,339.33594266)(241.89062685,340.24220199)
\curveto(241.21874244,341.15625334)(240.3085871,341.61328413)(239.1601581,341.61329574)
\moveto(239.1601581,343.44142074)
\curveto(241.03514888,343.4414073)(242.50780365,342.83203291)(243.57812685,341.61329574)
\curveto(244.64842651,340.39453535)(245.18358223,338.70703704)(245.1835956,336.55079574)
\curveto(245.18358223,334.40235384)(244.64842651,332.71485553)(243.57812685,331.48829574)
\curveto(242.50780365,330.26954547)(241.03514888,329.66017108)(239.1601581,329.66017074)
\curveto(237.27734013,329.66017108)(235.80077911,330.26954547)(234.7304706,331.48829574)
\curveto(233.66796874,332.71485553)(233.13671927,334.40235384)(233.1367206,336.55079574)
\curveto(233.13671927,338.70703704)(233.66796874,340.39453535)(234.7304706,341.61329574)
\curveto(235.80077911,342.83203291)(237.27734013,343.4414073)(239.1601581,343.44142074)
}
}
{
\newrgbcolor{curcolor}{0 0 0}
\pscustom[linestyle=none,fillstyle=solid,fillcolor=curcolor,opacity=0.15686275]
{
\newpath
\moveto(265.28100675,349.97837753)
\lineto(384.74003881,349.97837753)
\curveto(393.21111723,349.97837753)(400.03079408,346.49264783)(400.03079408,342.16284008)
\lineto(400.03079408,327.93718166)
\curveto(400.03079408,323.60737391)(393.21111723,320.12164421)(384.74003881,320.12164421)
\lineto(265.28100675,320.12164421)
\curveto(256.80992833,320.12164421)(249.99025148,323.60737391)(249.99025148,327.93718166)
\lineto(249.99025148,342.16284008)
\curveto(249.99025148,346.49264783)(256.80992833,349.97837753)(265.28100675,349.97837753)
\closepath
}
}
{
\newrgbcolor{curcolor}{0 0 0}
\pscustom[linewidth=2,linecolor=curcolor]
{
\newpath
\moveto(265.28100675,349.97837753)
\lineto(384.74003881,349.97837753)
\curveto(393.21111723,349.97837753)(400.03079408,346.49264783)(400.03079408,342.16284008)
\lineto(400.03079408,327.93718166)
\curveto(400.03079408,323.60737391)(393.21111723,320.12164421)(384.74003881,320.12164421)
\lineto(265.28100675,320.12164421)
\curveto(256.80992833,320.12164421)(249.99025148,323.60737391)(249.99025148,327.93718166)
\lineto(249.99025148,342.16284008)
\curveto(249.99025148,346.49264783)(256.80992833,349.97837753)(265.28100675,349.97837753)
\closepath
}
}
{
\newrgbcolor{curcolor}{1 1 1}
\pscustom[linestyle=none,fillstyle=solid,fillcolor=curcolor]
{
\newpath
\moveto(234.64676946,269.59635849)
\lineto(325.16059202,269.59635849)
\curveto(333.3281575,269.59635849)(339.90349001,263.02102599)(339.90349001,254.8534605)
\curveto(339.90349001,246.68589502)(333.3281575,240.11056251)(325.16059202,240.11056251)
\lineto(234.64676946,240.11056251)
\curveto(226.47920398,240.11056251)(219.90387148,246.68589502)(219.90387148,254.8534605)
\curveto(219.90387148,263.02102599)(226.47920398,269.59635849)(234.64676946,269.59635849)
\closepath
}
}
{
\newrgbcolor{curcolor}{0 0 0}
\pscustom[linewidth=2,linecolor=curcolor]
{
\newpath
\moveto(234.64676946,269.59635849)
\lineto(325.16059202,269.59635849)
\curveto(333.3281575,269.59635849)(339.90349001,263.02102599)(339.90349001,254.8534605)
\curveto(339.90349001,246.68589502)(333.3281575,240.11056251)(325.16059202,240.11056251)
\lineto(234.64676946,240.11056251)
\curveto(226.47920398,240.11056251)(219.90387148,246.68589502)(219.90387148,254.8534605)
\curveto(219.90387148,263.02102599)(226.47920398,269.59635849)(234.64676946,269.59635849)
\closepath
}
}
{
\newrgbcolor{curcolor}{0 0 0}
\pscustom[linestyle=none,fillstyle=solid,fillcolor=curcolor]
{
\newpath
\moveto(246.86718935,250.00001449)
\lineto(240.18750185,267.49610824)
\lineto(242.6601581,267.49610824)
\lineto(248.20312685,252.76563949)
\lineto(253.75781435,267.49610824)
\lineto(256.21875185,267.49610824)
\lineto(249.5507831,250.00001449)
\lineto(246.86718935,250.00001449)
}
}
{
\newrgbcolor{curcolor}{0 0 0}
\pscustom[linestyle=none,fillstyle=solid,fillcolor=curcolor]
{
\newpath
\moveto(262.75781435,256.59767074)
\curveto(261.01562036,256.59766415)(259.80859032,256.3984456)(259.1367206,256.00001449)
\curveto(258.46484166,255.60157139)(258.1289045,254.92188457)(258.1289081,253.96095199)
\curveto(258.1289045,253.1953238)(258.37890425,252.58594941)(258.8789081,252.13282699)
\curveto(259.38671574,251.68751281)(260.07421505,251.46485678)(260.9414081,251.46485824)
\curveto(262.13671299,251.46485678)(263.09374329,251.88673136)(263.81250185,252.73048324)
\curveto(264.53905434,253.58204216)(264.90233523,254.71094728)(264.9023456,256.11720199)
\lineto(264.9023456,256.59767074)
\lineto(262.75781435,256.59767074)
\moveto(267.0585956,257.48829574)
\lineto(267.0585956,250.00001449)
\lineto(264.9023456,250.00001449)
\lineto(264.9023456,251.99220199)
\curveto(264.41014822,251.1953258)(263.79686758,250.60548264)(263.06250185,250.22267074)
\curveto(262.32811905,249.8476709)(261.42968245,249.66017108)(260.36718935,249.66017074)
\curveto(259.02343486,249.66017108)(257.95312343,250.03517071)(257.15625185,250.78517074)
\curveto(256.36718751,251.5429817)(255.97265666,252.55469944)(255.9726581,253.82032699)
\curveto(255.97265666,255.2968842)(256.46484366,256.41016433)(257.4492206,257.16017074)
\curveto(258.44140419,257.91016283)(259.91796521,258.28516246)(261.8789081,258.28517074)
\lineto(264.9023456,258.28517074)
\lineto(264.9023456,258.49610824)
\curveto(264.90233523,259.48828626)(264.57421055,260.25391049)(263.9179706,260.79298324)
\curveto(263.26952436,261.3398469)(262.35546277,261.61328413)(261.1757831,261.61329574)
\curveto(260.4257772,261.61328413)(259.69530918,261.52344047)(258.98437685,261.34376449)
\curveto(258.27343561,261.16406583)(257.58984254,260.89453485)(256.9335956,260.53517074)
\lineto(256.9335956,262.52735824)
\curveto(257.72265491,262.83203291)(258.48827914,263.05859519)(259.2304706,263.20704574)
\curveto(259.97265266,263.36328238)(260.69530818,263.4414073)(261.39843935,263.44142074)
\curveto(263.29686808,263.4414073)(264.71483541,262.94922029)(265.6523456,261.96485824)
\curveto(266.58983354,260.98047226)(267.05858307,259.48828626)(267.0585956,257.48829574)
}
}
{
\newrgbcolor{curcolor}{0 0 0}
\pscustom[linestyle=none,fillstyle=solid,fillcolor=curcolor]
{
\newpath
\moveto(271.5117206,268.23438949)
\lineto(273.6679706,268.23438949)
\lineto(273.6679706,250.00001449)
\lineto(271.5117206,250.00001449)
\lineto(271.5117206,268.23438949)
}
}
{
\newrgbcolor{curcolor}{0 0 0}
\pscustom[linestyle=none,fillstyle=solid,fillcolor=curcolor]
{
\newpath
\moveto(278.1679706,263.12501449)
\lineto(280.3242206,263.12501449)
\lineto(280.3242206,250.00001449)
\lineto(278.1679706,250.00001449)
\lineto(278.1679706,263.12501449)
\moveto(278.1679706,268.23438949)
\lineto(280.3242206,268.23438949)
\lineto(280.3242206,265.50392074)
\lineto(278.1679706,265.50392074)
\lineto(278.1679706,268.23438949)
}
}
{
\newrgbcolor{curcolor}{0 0 0}
\pscustom[linestyle=none,fillstyle=solid,fillcolor=curcolor]
{
\newpath
\moveto(293.46093935,261.13282699)
\lineto(293.46093935,268.23438949)
\lineto(295.61718935,268.23438949)
\lineto(295.61718935,250.00001449)
\lineto(293.46093935,250.00001449)
\lineto(293.46093935,251.96876449)
\curveto(293.0078039,251.18751331)(292.43358573,250.60548264)(291.7382831,250.22267074)
\curveto(291.05077461,249.8476709)(290.22265044,249.66017108)(289.2539081,249.66017074)
\curveto(287.66796549,249.66017108)(286.37499804,250.29298295)(285.37500185,251.55860824)
\curveto(284.38281253,252.82423042)(283.88671927,254.48829126)(283.8867206,256.55079574)
\curveto(283.88671927,258.61328713)(284.38281253,260.27734797)(285.37500185,261.54298324)
\curveto(286.37499804,262.80859544)(287.66796549,263.4414073)(289.2539081,263.44142074)
\curveto(290.22265044,263.4414073)(291.05077461,263.25000124)(291.7382831,262.86720199)
\curveto(292.43358573,262.4921895)(293.0078039,261.91406508)(293.46093935,261.13282699)
\moveto(286.1132831,256.55079574)
\curveto(286.11327955,254.96485328)(286.43749797,253.71876078)(287.08593935,252.81251449)
\curveto(287.74218417,251.91407508)(288.64062077,251.46485678)(289.78125185,251.46485824)
\curveto(290.92186849,251.46485678)(291.82030509,251.91407508)(292.47656435,252.81251449)
\curveto(293.13280378,253.71876078)(293.46092845,254.96485328)(293.46093935,256.55079574)
\curveto(293.46092845,258.13672511)(293.13280378,259.37891137)(292.47656435,260.27735824)
\curveto(291.82030509,261.18359706)(290.92186849,261.63672161)(289.78125185,261.63673324)
\curveto(288.64062077,261.63672161)(287.74218417,261.18359706)(287.08593935,260.27735824)
\curveto(286.43749797,259.37891137)(286.11327955,258.13672511)(286.1132831,256.55079574)
}
}
{
\newrgbcolor{curcolor}{0 0 0}
\pscustom[linestyle=none,fillstyle=solid,fillcolor=curcolor]
{
\newpath
\moveto(311.2851581,257.10157699)
\lineto(311.2851581,256.04688949)
\lineto(301.3710956,256.04688949)
\curveto(301.46484193,254.56250993)(301.91015398,253.42969856)(302.7070331,252.64845199)
\curveto(303.51171488,251.87501262)(304.62890127,251.48829426)(306.0585956,251.48829574)
\curveto(306.88671151,251.48829426)(307.68749196,251.58985665)(308.46093935,251.79298324)
\curveto(309.2421779,251.99610625)(310.01561463,252.30079344)(310.78125185,252.70704574)
\lineto(310.78125185,250.66798324)
\curveto(310.00780214,250.3398579)(309.21483418,250.08985815)(308.4023456,249.91798324)
\curveto(307.5898358,249.7461085)(306.76561788,249.66017108)(305.92968935,249.66017074)
\curveto(303.83593331,249.66017108)(302.17577872,250.26954547)(300.9492206,251.48829574)
\curveto(299.73046866,252.70704304)(299.12109427,254.35547889)(299.1210956,256.43360824)
\curveto(299.12109427,258.58203716)(299.6992187,260.28516046)(300.8554706,261.54298324)
\curveto(302.01952887,262.80859544)(303.58593356,263.4414073)(305.55468935,263.44142074)
\curveto(307.32030482,263.4414073)(308.71483468,262.87109537)(309.7382831,261.73048324)
\curveto(310.76952012,260.59766015)(311.28514461,259.05469294)(311.2851581,257.10157699)
\moveto(309.1289081,257.73438949)
\curveto(309.11327178,258.91406808)(308.78124086,259.85547339)(308.13281435,260.55860824)
\curveto(307.49217965,261.26172198)(306.640618,261.61328413)(305.57812685,261.61329574)
\curveto(304.37499527,261.61328413)(303.41015248,261.27344072)(302.6835956,260.59376449)
\curveto(301.96484143,259.91406708)(301.55077934,258.95703679)(301.4414081,257.72267074)
\lineto(309.1289081,257.73438949)
}
}
{
\newrgbcolor{curcolor}{0 0 0}
\pscustom[linestyle=none,fillstyle=solid,fillcolor=curcolor]
{
\newpath
\moveto(322.42968935,261.10938949)
\curveto(322.18749222,261.25000324)(321.92186749,261.35156564)(321.63281435,261.41407699)
\curveto(321.35155556,261.48437801)(321.03905587,261.51953422)(320.69531435,261.51954574)
\curveto(319.47655743,261.51953422)(318.53905837,261.12109712)(317.88281435,260.32423324)
\curveto(317.23437218,259.53516121)(316.91015375,258.3984436)(316.9101581,256.91407699)
\lineto(316.9101581,250.00001449)
\lineto(314.74218935,250.00001449)
\lineto(314.74218935,263.12501449)
\lineto(316.9101581,263.12501449)
\lineto(316.9101581,261.08595199)
\curveto(317.3632783,261.88281511)(317.95312146,262.47265827)(318.67968935,262.85548324)
\curveto(319.406245,263.246095)(320.28905662,263.4414073)(321.32812685,263.44142074)
\curveto(321.47655543,263.4414073)(321.64061777,263.42968856)(321.82031435,263.40626449)
\curveto(321.99999241,263.3906261)(322.19921096,263.36328238)(322.4179706,263.32423324)
\lineto(322.42968935,261.10938949)
}
}
{
\newrgbcolor{curcolor}{0 0 0}
\pscustom[linestyle=none,fillstyle=solid,fillcolor=curcolor]
{
\newpath
\moveto(506.14062685,322.65626449)
\lineto(517.15625185,322.65626449)
\lineto(517.15625185,320.00001449)
\lineto(502.34375185,320.00001449)
\lineto(502.34375185,322.65626449)
\curveto(503.54166497,323.89584393)(505.17187168,325.5573006)(507.23437685,327.64063949)
\curveto(509.30728421,329.73437976)(510.60936624,331.08333674)(511.14062685,331.68751449)
\curveto(512.15103136,332.82291834)(512.85415566,333.78125071)(513.25000185,334.56251449)
\curveto(513.65623819,335.35416581)(513.85936299,336.1302067)(513.85937685,336.89063949)
\curveto(513.85936299,338.1302047)(513.42186343,339.14062035)(512.54687685,339.92188949)
\curveto(511.68228183,340.70311879)(510.55207463,341.0937434)(509.15625185,341.09376449)
\curveto(508.16666035,341.0937434)(507.11978639,340.92186857)(506.01562685,340.57813949)
\curveto(504.92187193,340.23436926)(503.7499981,339.71353645)(502.50000185,339.01563949)
\lineto(502.50000185,342.20313949)
\curveto(503.77083141,342.71353345)(504.95833022,343.09894973)(506.06250185,343.35938949)
\curveto(507.16666135,343.61978254)(508.177077,343.74999074)(509.09375185,343.75001449)
\curveto(511.510407,343.74999074)(513.43748841,343.14582468)(514.87500185,341.93751449)
\curveto(516.31248554,340.72916043)(517.03123482,339.11457871)(517.03125185,337.09376449)
\curveto(517.03123482,336.13541503)(516.84894333,335.2239576)(516.48437685,334.35938949)
\curveto(516.13019405,333.50520932)(515.47915304,332.49479367)(514.53125185,331.32813949)
\curveto(514.27082091,331.02604513)(513.44269674,330.15104601)(512.04687685,328.70313949)
\curveto(510.65103286,327.26563223)(508.68228483,325.25000924)(506.14062685,322.65626449)
}
}
{
\newrgbcolor{curcolor}{0 0 0}
\pscustom[linewidth=2,linecolor=curcolor,linestyle=dashed,dash=8 8]
{
\newpath
\moveto(150,530)
\lineto(60,530)
}
}
{
\newrgbcolor{curcolor}{0 0 0}
\pscustom[linestyle=none,fillstyle=solid,fillcolor=curcolor]
{
\newpath
\moveto(139.53769464,534.84048224)
\lineto(152.6487474,530.01921591)
\lineto(139.53769392,525.19795064)
\curveto(141.632292,528.04442372)(141.62022288,531.93889292)(139.53769464,534.84048224)
\lineto(139.53769464,534.84048224)
\closepath
}
}
{
\newrgbcolor{curcolor}{0 0 0}
\pscustom[linewidth=2,linecolor=curcolor,linestyle=dashed,dash=8 8]
{
\newpath
\moveto(370,330)
\lineto(500,330)
}
}
{
\newrgbcolor{curcolor}{0 0 0}
\pscustom[linestyle=none,fillstyle=solid,fillcolor=curcolor]
{
\newpath
\moveto(380.46230536,325.15951776)
\lineto(367.3512526,329.98078409)
\lineto(380.46230608,334.80204936)
\curveto(378.367708,331.95557628)(378.37977712,328.06110708)(380.46230536,325.15951776)
\lineto(380.46230536,325.15951776)
\closepath
}
}
{
\newrgbcolor{curcolor}{0 0 0}
\pscustom[linewidth=2,linecolor=curcolor,linestyle=dashed,dash=8 8]
{
\newpath
\moveto(230,260)
\lineto(60,260)
}
}
{
\newrgbcolor{curcolor}{0 0 0}
\pscustom[linestyle=none,fillstyle=solid,fillcolor=curcolor]
{
\newpath
\moveto(219.53769464,264.84048224)
\lineto(232.6487474,260.01921591)
\lineto(219.53769392,255.19795064)
\curveto(221.632292,258.04442372)(221.62022288,261.93889292)(219.53769464,264.84048224)
\lineto(219.53769464,264.84048224)
\closepath
}
}
{
\newrgbcolor{curcolor}{0 0 0}
\pscustom[linestyle=none,fillstyle=solid,fillcolor=curcolor]
{
\newpath
\moveto(43.96875185,522.65624924)
\lineto(49.12500185,522.65624924)
\lineto(49.12500185,540.45312424)
\lineto(43.51562685,539.32812424)
\lineto(43.51562685,542.20312424)
\lineto(49.09375185,543.32812424)
\lineto(52.25000185,543.32812424)
\lineto(52.25000185,522.65624924)
\lineto(57.40625185,522.65624924)
\lineto(57.40625185,519.99999924)
\lineto(43.96875185,519.99999924)
\lineto(43.96875185,522.65624924)
}
}
{
\newrgbcolor{curcolor}{0 0 0}
\pscustom[linestyle=none,fillstyle=solid,fillcolor=curcolor]
{
\newpath
\moveto(512.98437685,292.57813949)
\curveto(514.49477902,292.25521057)(515.67186118,291.58333624)(516.51562685,290.56251449)
\curveto(517.36977614,289.54167162)(517.79685905,288.28125621)(517.79687685,286.78126449)
\curveto(517.79685905,284.47917668)(517.00519318,282.69792846)(515.42187685,281.43751449)
\curveto(513.83852968,280.17709765)(511.58853193,279.54688995)(508.67187685,279.54688949)
\curveto(507.69270249,279.54688995)(506.68228683,279.64584818)(505.64062685,279.84376449)
\curveto(504.60937224,280.03126446)(503.54166497,280.31772251)(502.43750185,280.70313949)
\lineto(502.43750185,283.75001449)
\curveto(503.31249854,283.23959459)(504.27083091,282.85417831)(505.31250185,282.59376449)
\curveto(506.35416216,282.33334549)(507.44270274,282.20313729)(508.57812685,282.20313949)
\curveto(510.55728296,282.20313729)(512.06248979,282.5937619)(513.09375185,283.37501449)
\curveto(514.13540438,284.15626034)(514.65623719,285.29167587)(514.65625185,286.78126449)
\curveto(514.65623719,288.15625634)(514.17186268,289.22917193)(513.20312685,290.00001449)
\curveto(512.24478127,290.78125371)(510.90624094,291.17187832)(509.18750185,291.17188949)
\lineto(506.46875185,291.17188949)
\lineto(506.46875185,293.76563949)
\lineto(509.31250185,293.76563949)
\curveto(510.86457432,293.76562573)(512.05207313,294.07291709)(512.87500185,294.68751449)
\curveto(513.69790482,295.31249918)(514.10936274,296.20833162)(514.10937685,297.37501449)
\curveto(514.10936274,298.57291259)(513.68227983,299.48957834)(512.82812685,300.12501449)
\curveto(511.98436486,300.77082706)(510.77082441,301.0937434)(509.18750185,301.09376449)
\curveto(508.32291019,301.0937434)(507.39582779,300.99999349)(506.40625185,300.81251449)
\curveto(505.4166631,300.62499387)(504.32812252,300.33332749)(503.14062685,299.93751449)
\lineto(503.14062685,302.75001449)
\curveto(504.33853918,303.08332474)(505.45832972,303.33332449)(506.50000185,303.50001449)
\curveto(507.55207763,303.66665749)(508.54165997,303.74999074)(509.46875185,303.75001449)
\curveto(511.86457332,303.74999074)(513.76040475,303.20311629)(515.15625185,302.10938949)
\curveto(516.55206863,301.02603513)(517.2499846,299.5572866)(517.25000185,297.70313949)
\curveto(517.2499846,296.41145642)(516.8801933,295.31770751)(516.14062685,294.42188949)
\curveto(515.40102811,293.53645929)(514.34894583,292.92187657)(512.98437685,292.57813949)
}
}
{
\newrgbcolor{curcolor}{1 1 1}
\pscustom[linestyle=none,fillstyle=solid,fillcolor=curcolor]
{
\newpath
\moveto(160.00001634,310.00001449)
\lineto(399.99998735,310.00001449)
\lineto(400.00000185,310)
\lineto(400.00000185,280.00001945)
\lineto(399.99998735,280.00000496)
\lineto(160.00001634,280.00000496)
\lineto(160.00000185,280.00001945)
\lineto(160.00000185,310)
\lineto(160.00001634,310.00001449)
\closepath
}
}
{
\newrgbcolor{curcolor}{1 0 0}
\pscustom[linewidth=2,linecolor=curcolor,linestyle=dashed,dash=8 8]
{
\newpath
\moveto(160.00001634,310.00001449)
\lineto(399.99998735,310.00001449)
\lineto(400.00000185,310)
\lineto(400.00000185,280.00001945)
\lineto(399.99998735,280.00000496)
\lineto(160.00001634,280.00000496)
\lineto(160.00000185,280.00001945)
\lineto(160.00000185,310)
\lineto(160.00001634,310.00001449)
\closepath
}
}
{
\newrgbcolor{curcolor}{0 0 0}
\pscustom[linewidth=2,linecolor=curcolor,linestyle=dashed,dash=8 8]
{
\newpath
\moveto(380,290)
\lineto(500,290)
}
}
{
\newrgbcolor{curcolor}{0 0 0}
\pscustom[linestyle=none,fillstyle=solid,fillcolor=curcolor]
{
\newpath
\moveto(390.46230536,285.15951776)
\lineto(377.3512526,289.98078409)
\lineto(390.46230608,294.80204936)
\curveto(388.367708,291.95557628)(388.37977712,288.06110708)(390.46230536,285.15951776)
\lineto(390.46230536,285.15951776)
\closepath
}
}
{
\newrgbcolor{curcolor}{0 0 0}
\pscustom[linestyle=none,fillstyle=solid,fillcolor=curcolor]
{
\newpath
\moveto(52.09375185,270.57813949)
\lineto(44.12500185,258.12501449)
\lineto(52.09375185,258.12501449)
\lineto(52.09375185,270.57813949)
\moveto(51.26562685,273.32813949)
\lineto(55.23437685,273.32813949)
\lineto(55.23437685,258.12501449)
\lineto(58.56250185,258.12501449)
\lineto(58.56250185,255.50001449)
\lineto(55.23437685,255.50001449)
\lineto(55.23437685,250.00001449)
\lineto(52.09375185,250.00001449)
\lineto(52.09375185,255.50001449)
\lineto(41.56250185,255.50001449)
\lineto(41.56250185,258.54688949)
\lineto(51.26562685,273.32813949)
}
}
\end{pspicture}

		
		\begin{enumerate}
		  \item Fond d'écran
		  \item Champs de texte
		  \item Bouton \hyperlink{Page d'accueil}{Valider}
		\end{enumerate}
			
		$\,$
		
		Lors de l'appui sur le bouton de validation une vérification a lieu afin de
		voir si un compte portant le même n'existe pas déjà ou que les caractères
		entrés sont valides (pas d'espace, etc \ldots). \\		
		
		$\,$
	
\newpage

	\subsection{Page d'accueil}
		\hypertarget{Page d'accueil}{}
		\label{accueil}
	
		%LaTeX with PSTricks extensions
%%Creator: inkscape 0.48.0
%%Please note this file requires PSTricks extensions
\psset{xunit=.5pt,yunit=.5pt,runit=.5pt}
\begin{pspicture}(560,600)
{
\newrgbcolor{curcolor}{1 1 1}
\pscustom[linestyle=none,fillstyle=solid,fillcolor=curcolor]
{
\newpath
\moveto(133.12418633,597.52237246)
\lineto(426.87615472,597.52237246)
\curveto(443.85414222,597.52237246)(457.52234155,583.85417314)(457.52234155,566.87618563)
\lineto(457.52234155,33.12418673)
\curveto(457.52234155,16.14619923)(443.85414222,2.4779999)(426.87615472,2.4779999)
\lineto(133.12418633,2.4779999)
\curveto(116.14619883,2.4779999)(102.4779995,16.14619923)(102.4779995,33.12418673)
\lineto(102.4779995,566.87618563)
\curveto(102.4779995,583.85417314)(116.14619883,597.52237246)(133.12418633,597.52237246)
\closepath
}
}
{
\newrgbcolor{curcolor}{0 0 0}
\pscustom[linewidth=4.95599985,linecolor=curcolor]
{
\newpath
\moveto(133.12418633,597.52237246)
\lineto(426.87615472,597.52237246)
\curveto(443.85414222,597.52237246)(457.52234155,583.85417314)(457.52234155,566.87618563)
\lineto(457.52234155,33.12418673)
\curveto(457.52234155,16.14619923)(443.85414222,2.4779999)(426.87615472,2.4779999)
\lineto(133.12418633,2.4779999)
\curveto(116.14619883,2.4779999)(102.4779995,16.14619923)(102.4779995,33.12418673)
\lineto(102.4779995,566.87618563)
\curveto(102.4779995,583.85417314)(116.14619883,597.52237246)(133.12418633,597.52237246)
\closepath
}
}
{
\newrgbcolor{curcolor}{1 1 1}
\pscustom[linestyle=none,fillstyle=solid,fillcolor=curcolor]
{
\newpath
\moveto(201.70414239,399.9999964)
\lineto(358.44696313,399.9999964)
\curveto(364.84734492,399.9999964)(369.99999886,394.84734246)(369.99999886,388.44696067)
\lineto(369.99999886,378.53398154)
\curveto(369.99999886,372.13359974)(364.84734492,366.9809458)(358.44696313,366.9809458)
\lineto(201.70414239,366.9809458)
\curveto(195.30376059,366.9809458)(190.15110665,372.13359974)(190.15110665,378.53398154)
\lineto(190.15110665,388.44696067)
\curveto(190.15110665,394.84734246)(195.30376059,399.9999964)(201.70414239,399.9999964)
\closepath
}
}
{
\newrgbcolor{curcolor}{0 0 0}
\pscustom[linewidth=2,linecolor=curcolor]
{
\newpath
\moveto(201.70414239,399.9999964)
\lineto(358.44696313,399.9999964)
\curveto(364.84734492,399.9999964)(369.99999886,394.84734246)(369.99999886,388.44696067)
\lineto(369.99999886,378.53398154)
\curveto(369.99999886,372.13359974)(364.84734492,366.9809458)(358.44696313,366.9809458)
\lineto(201.70414239,366.9809458)
\curveto(195.30376059,366.9809458)(190.15110665,372.13359974)(190.15110665,378.53398154)
\lineto(190.15110665,388.44696067)
\curveto(190.15110665,394.84734246)(195.30376059,399.9999964)(201.70414239,399.9999964)
\closepath
}
}
{
\newrgbcolor{curcolor}{0 0 0}
\pscustom[linestyle=none,fillstyle=solid,fillcolor=curcolor]
{
\newpath
\moveto(212.58688241,394.55735419)
\lineto(214.95406991,394.55735419)
\lineto(214.95406991,378.28001044)
\curveto(214.95406519,376.17063633)(214.55172184,374.63938786)(213.74703866,373.68626044)
\curveto(212.95016094,372.73313977)(211.66500597,372.25657775)(209.89156991,372.25657294)
\lineto(208.98922616,372.25657294)
\lineto(208.98922616,374.24876044)
\lineto(209.72750741,374.24876044)
\curveto(210.77438187,374.24876325)(211.51266238,374.54173171)(211.94235116,375.12766669)
\curveto(212.37203652,375.71360554)(212.58688005,376.76438574)(212.58688241,378.28001044)
\lineto(212.58688241,394.55735419)
}
}
{
\newrgbcolor{curcolor}{0 0 0}
\pscustom[linestyle=none,fillstyle=solid,fillcolor=curcolor]
{
\newpath
\moveto(224.65719491,388.67454169)
\curveto(223.50093872,388.67453008)(222.58687713,388.22140553)(221.91500741,387.31516669)
\curveto(221.24312847,386.41671983)(220.90719131,385.18234607)(220.90719491,383.61204169)
\curveto(220.90719131,382.04172421)(221.23922223,380.8034442)(221.90328866,379.89719794)
\curveto(222.57515839,378.9987585)(223.49312622,378.5495402)(224.65719491,378.54954169)
\curveto(225.80562391,378.5495402)(226.71577925,379.00266475)(227.38766366,379.90891669)
\curveto(228.05952791,380.81516294)(228.39546507,382.0495367)(228.39547616,383.61204169)
\curveto(228.39546507,385.16672108)(228.05952791,386.3971886)(227.38766366,387.30344794)
\curveto(226.71577925,388.21749928)(225.80562391,388.67453008)(224.65719491,388.67454169)
\moveto(224.65719491,390.50266669)
\curveto(226.53218569,390.50265325)(228.00484046,389.89327886)(229.07516366,388.67454169)
\curveto(230.14546332,387.4557813)(230.68061904,385.76828298)(230.68063241,383.61204169)
\curveto(230.68061904,381.46359979)(230.14546332,379.77610148)(229.07516366,378.54954169)
\curveto(228.00484046,377.33079142)(226.53218569,376.72141703)(224.65719491,376.72141669)
\curveto(222.77437694,376.72141703)(221.29781592,377.33079142)(220.22750741,378.54954169)
\curveto(219.16500555,379.77610148)(218.63375608,381.46359979)(218.63375741,383.61204169)
\curveto(218.63375608,385.76828298)(219.16500555,387.4557813)(220.22750741,388.67454169)
\curveto(221.29781592,389.89327886)(222.77437694,390.50265325)(224.65719491,390.50266669)
}
}
{
\newrgbcolor{curcolor}{0 0 0}
\pscustom[linestyle=none,fillstyle=solid,fillcolor=curcolor]
{
\newpath
\moveto(234.02047616,382.24094794)
\lineto(234.02047616,390.18626044)
\lineto(236.17672616,390.18626044)
\lineto(236.17672616,382.32297919)
\curveto(236.17672196,381.08078767)(236.41890922,380.14719485)(236.90328866,379.52219794)
\curveto(237.38765825,378.9050086)(238.11422003,378.59641516)(239.08297616,378.59641669)
\curveto(240.24703039,378.59641516)(241.16499822,378.96750853)(241.83688241,379.70969794)
\curveto(242.51655937,380.45188205)(242.85640278,381.46359979)(242.85641366,382.74485419)
\lineto(242.85641366,390.18626044)
\lineto(245.01266366,390.18626044)
\lineto(245.01266366,377.06126044)
\lineto(242.85641366,377.06126044)
\lineto(242.85641366,379.07688544)
\curveto(242.33296581,378.28000922)(241.72359142,377.68625982)(241.02828866,377.29563544)
\curveto(240.3407803,376.91282309)(239.53999985,376.72141703)(238.62594491,376.72141669)
\curveto(237.11812727,376.72141703)(235.97359717,377.19016656)(235.19235116,378.12766669)
\curveto(234.41109873,379.06516469)(234.02047412,380.43625707)(234.02047616,382.24094794)
\moveto(239.44625741,390.50266669)
\lineto(239.44625741,390.50266669)
}
}
{
\newrgbcolor{curcolor}{0 0 0}
\pscustom[linestyle=none,fillstyle=solid,fillcolor=curcolor]
{
\newpath
\moveto(260.70406991,384.16282294)
\lineto(260.70406991,383.10813544)
\lineto(250.79000741,383.10813544)
\curveto(250.88375374,381.62375588)(251.32906579,380.49094451)(252.12594491,379.70969794)
\curveto(252.93062669,378.93625857)(254.04781308,378.5495402)(255.47750741,378.54954169)
\curveto(256.30562332,378.5495402)(257.10640377,378.6511026)(257.87985116,378.85422919)
\curveto(258.66108971,379.05735219)(259.43452644,379.36203939)(260.20016366,379.76829169)
\lineto(260.20016366,377.72922919)
\curveto(259.42671395,377.40110385)(258.63374599,377.1511041)(257.82125741,376.97922919)
\curveto(257.00874762,376.80735444)(256.18452969,376.72141703)(255.34860116,376.72141669)
\curveto(253.25484512,376.72141703)(251.59469053,377.33079142)(250.36813241,378.54954169)
\curveto(249.14938047,379.76828898)(248.54000608,381.41672483)(248.54000741,383.49485419)
\curveto(248.54000608,385.64328311)(249.11813051,387.34640641)(250.27438241,388.60422919)
\curveto(251.43844069,389.86984138)(253.00484537,390.50265325)(254.97360116,390.50266669)
\curveto(256.73921663,390.50265325)(258.13374649,389.93234132)(259.15719491,388.79172919)
\curveto(260.18843194,387.65890609)(260.70405642,386.11593889)(260.70406991,384.16282294)
\moveto(258.54781991,384.79563544)
\curveto(258.53218359,385.97531403)(258.20015267,386.91671933)(257.55172616,387.61985419)
\curveto(256.91109146,388.32296793)(256.05952981,388.67453008)(254.99703866,388.67454169)
\curveto(253.79390708,388.67453008)(252.82906429,388.33468667)(252.10250741,387.65501044)
\curveto(251.38375324,386.97531303)(250.96969115,386.01828273)(250.86031991,384.78391669)
\lineto(258.54781991,384.79563544)
}
}
{
\newrgbcolor{curcolor}{0 0 0}
\pscustom[linestyle=none,fillstyle=solid,fillcolor=curcolor]
{
\newpath
\moveto(271.84860116,388.17063544)
\curveto(271.60640403,388.31124919)(271.3407793,388.41281159)(271.05172616,388.47532294)
\curveto(270.77046737,388.54562396)(270.45796768,388.58078017)(270.11422616,388.58079169)
\curveto(268.89546924,388.58078017)(267.95797018,388.18234307)(267.30172616,387.38547919)
\curveto(266.65328399,386.59640716)(266.32906556,385.45968954)(266.32906991,383.97532294)
\lineto(266.32906991,377.06126044)
\lineto(264.16110116,377.06126044)
\lineto(264.16110116,390.18626044)
\lineto(266.32906991,390.18626044)
\lineto(266.32906991,388.14719794)
\curveto(266.78219011,388.94406106)(267.37203327,389.53390422)(268.09860116,389.91672919)
\curveto(268.82515681,390.30734094)(269.70796843,390.50265325)(270.74703866,390.50266669)
\curveto(270.89546724,390.50265325)(271.05952958,390.49093451)(271.23922616,390.46751044)
\curveto(271.41890422,390.45187205)(271.61812277,390.42452833)(271.83688241,390.38547919)
\lineto(271.84860116,388.17063544)
}
}
{
\newrgbcolor{curcolor}{0 0 0}
\pscustom[linestyle=none,fillstyle=solid,fillcolor=curcolor]
{
}
}
{
\newrgbcolor{curcolor}{0 0 0}
\pscustom[linestyle=none,fillstyle=solid,fillcolor=curcolor]
{
\newpath
\moveto(293.00094491,384.16282294)
\lineto(293.00094491,383.10813544)
\lineto(283.08688241,383.10813544)
\curveto(283.18062874,381.62375588)(283.62594079,380.49094451)(284.42281991,379.70969794)
\curveto(285.22750169,378.93625857)(286.34468808,378.5495402)(287.77438241,378.54954169)
\curveto(288.60249832,378.5495402)(289.40327877,378.6511026)(290.17672616,378.85422919)
\curveto(290.95796471,379.05735219)(291.73140144,379.36203939)(292.49703866,379.76829169)
\lineto(292.49703866,377.72922919)
\curveto(291.72358895,377.40110385)(290.93062099,377.1511041)(290.11813241,376.97922919)
\curveto(289.30562262,376.80735444)(288.48140469,376.72141703)(287.64547616,376.72141669)
\curveto(285.55172012,376.72141703)(283.89156553,377.33079142)(282.66500741,378.54954169)
\curveto(281.44625547,379.76828898)(280.83688108,381.41672483)(280.83688241,383.49485419)
\curveto(280.83688108,385.64328311)(281.41500551,387.34640641)(282.57125741,388.60422919)
\curveto(283.73531569,389.86984138)(285.30172037,390.50265325)(287.27047616,390.50266669)
\curveto(289.03609163,390.50265325)(290.43062149,389.93234132)(291.45406991,388.79172919)
\curveto(292.48530694,387.65890609)(293.00093142,386.11593889)(293.00094491,384.16282294)
\moveto(290.84469491,384.79563544)
\curveto(290.82905859,385.97531403)(290.49702767,386.91671933)(289.84860116,387.61985419)
\curveto(289.20796646,388.32296793)(288.35640481,388.67453008)(287.29391366,388.67454169)
\curveto(286.09078208,388.67453008)(285.12593929,388.33468667)(284.39938241,387.65501044)
\curveto(283.68062824,386.97531303)(283.26656615,386.01828273)(283.15719491,384.78391669)
\lineto(290.84469491,384.79563544)
}
}
{
\newrgbcolor{curcolor}{0 0 0}
\pscustom[linestyle=none,fillstyle=solid,fillcolor=curcolor]
{
\newpath
\moveto(307.45016366,384.98313544)
\lineto(307.45016366,377.06126044)
\lineto(305.29391366,377.06126044)
\lineto(305.29391366,384.91282294)
\curveto(305.29390264,386.15500135)(305.05171538,387.08468792)(304.56735116,387.70188544)
\curveto(304.08296635,388.31906168)(303.35640458,388.62765512)(302.38766366,388.62766669)
\curveto(301.22359421,388.62765512)(300.30562638,388.25656175)(299.63375741,387.51438544)
\curveto(298.96187772,386.77218823)(298.62594056,385.76047049)(298.62594491,384.47922919)
\lineto(298.62594491,377.06126044)
\lineto(296.45797616,377.06126044)
\lineto(296.45797616,390.18626044)
\lineto(298.62594491,390.18626044)
\lineto(298.62594491,388.14719794)
\curveto(299.14156504,388.93624857)(299.74703319,389.52609173)(300.44235116,389.91672919)
\curveto(301.14546929,390.30734094)(301.95406223,390.50265325)(302.86813241,390.50266669)
\curveto(304.37593481,390.50265325)(305.51655867,390.03390372)(306.29000741,389.09641669)
\curveto(307.06343212,388.16671808)(307.45015049,386.79562571)(307.45016366,384.98313544)
}
}
{
\newrgbcolor{curcolor}{0 0 0}
\pscustom[linestyle=none,fillstyle=solid,fillcolor=curcolor]
{
}
}
{
\newrgbcolor{curcolor}{0 0 0}
\pscustom[linestyle=none,fillstyle=solid,fillcolor=curcolor]
{
\newpath
\moveto(327.78219491,389.79954169)
\lineto(327.78219491,387.76047919)
\curveto(327.17280989,388.07296818)(326.53999802,388.30734294)(325.88375741,388.46360419)
\curveto(325.22749933,388.61984263)(324.54781251,388.69796755)(323.84469491,388.69797919)
\curveto(322.77437679,388.69796755)(321.96969009,388.53390522)(321.43063241,388.20579169)
\curveto(320.89937866,387.87765587)(320.63375393,387.38546887)(320.63375741,386.72922919)
\curveto(320.63375393,386.22922002)(320.82515999,385.83468917)(321.20797616,385.54563544)
\curveto(321.59078422,385.26437724)(322.3603147,384.99484626)(323.51656991,384.73704169)
\lineto(324.25485116,384.57297919)
\curveto(325.78609253,384.24484701)(326.87202894,383.78000372)(327.51266366,383.17844794)
\curveto(328.16109015,382.58469242)(328.48530858,381.752662)(328.48531991,380.68235419)
\curveto(328.48530858,379.46360179)(328.00093406,378.498759)(327.03219491,377.78782294)
\curveto(326.07124849,377.07688542)(324.74703106,376.72141703)(323.05953866,376.72141669)
\curveto(322.35640846,376.72141703)(321.62203419,376.79172946)(320.85641366,376.93235419)
\curveto(320.09859821,377.06516669)(319.29781776,377.26829148)(318.45406991,377.54172919)
\lineto(318.45406991,379.76829169)
\curveto(319.25094281,379.3542269)(320.03609828,379.04172721)(320.80953866,378.83079169)
\curveto(321.58297173,378.62766512)(322.34859596,378.52610273)(323.10641366,378.52610419)
\curveto(324.12203169,378.52610273)(324.90328091,378.69797755)(325.45016366,379.04172919)
\curveto(325.99702981,379.39328936)(326.27046704,379.88547637)(326.27047616,380.51829169)
\curveto(326.27046704,381.10422515)(326.07124849,381.55344345)(325.67281991,381.86594794)
\curveto(325.28218678,382.17844282)(324.41890639,382.47922377)(323.08297616,382.76829169)
\lineto(322.33297616,382.94407294)
\curveto(320.99703481,383.22531678)(320.03219203,383.65500385)(319.43844491,384.23313544)
\curveto(318.84469322,384.81906518)(318.54781851,385.61984563)(318.54781991,386.63547919)
\curveto(318.54781851,387.86984338)(318.98531808,388.82296743)(319.86031991,389.49485419)
\curveto(320.73531633,390.16671608)(321.97750258,390.50265325)(323.58688241,390.50266669)
\curveto(324.38375018,390.50265325)(325.13374943,390.44405956)(325.83688241,390.32688544)
\curveto(326.53999802,390.20968479)(327.18843487,390.03390372)(327.78219491,389.79954169)
}
}
{
\newrgbcolor{curcolor}{0 0 0}
\pscustom[linestyle=none,fillstyle=solid,fillcolor=curcolor]
{
\newpath
\moveto(337.01656991,388.67454169)
\curveto(335.86031372,388.67453008)(334.94625213,388.22140553)(334.27438241,387.31516669)
\curveto(333.60250347,386.41671983)(333.26656631,385.18234607)(333.26656991,383.61204169)
\curveto(333.26656631,382.04172421)(333.59859723,380.8034442)(334.26266366,379.89719794)
\curveto(334.93453339,378.9987585)(335.85250122,378.5495402)(337.01656991,378.54954169)
\curveto(338.16499891,378.5495402)(339.07515425,379.00266475)(339.74703866,379.90891669)
\curveto(340.41890291,380.81516294)(340.75484007,382.0495367)(340.75485116,383.61204169)
\curveto(340.75484007,385.16672108)(340.41890291,386.3971886)(339.74703866,387.30344794)
\curveto(339.07515425,388.21749928)(338.16499891,388.67453008)(337.01656991,388.67454169)
\moveto(337.01656991,390.50266669)
\curveto(338.89156069,390.50265325)(340.36421546,389.89327886)(341.43453866,388.67454169)
\curveto(342.50483832,387.4557813)(343.03999404,385.76828298)(343.04000741,383.61204169)
\curveto(343.03999404,381.46359979)(342.50483832,379.77610148)(341.43453866,378.54954169)
\curveto(340.36421546,377.33079142)(338.89156069,376.72141703)(337.01656991,376.72141669)
\curveto(335.13375194,376.72141703)(333.65719092,377.33079142)(332.58688241,378.54954169)
\curveto(331.52438055,379.77610148)(330.99313108,381.46359979)(330.99313241,383.61204169)
\curveto(330.99313108,385.76828298)(331.52438055,387.4557813)(332.58688241,388.67454169)
\curveto(333.65719092,389.89327886)(335.13375194,390.50265325)(337.01656991,390.50266669)
}
}
{
\newrgbcolor{curcolor}{0 0 0}
\pscustom[linestyle=none,fillstyle=solid,fillcolor=curcolor]
{
\newpath
\moveto(346.60250741,395.29563544)
\lineto(348.75875741,395.29563544)
\lineto(348.75875741,377.06126044)
\lineto(346.60250741,377.06126044)
\lineto(346.60250741,395.29563544)
}
}
{
\newrgbcolor{curcolor}{0 0 0}
\pscustom[linestyle=none,fillstyle=solid,fillcolor=curcolor]
{
\newpath
\moveto(358.34469491,388.67454169)
\curveto(357.18843872,388.67453008)(356.27437713,388.22140553)(355.60250741,387.31516669)
\curveto(354.93062847,386.41671983)(354.59469131,385.18234607)(354.59469491,383.61204169)
\curveto(354.59469131,382.04172421)(354.92672223,380.8034442)(355.59078866,379.89719794)
\curveto(356.26265839,378.9987585)(357.18062622,378.5495402)(358.34469491,378.54954169)
\curveto(359.49312391,378.5495402)(360.40327925,379.00266475)(361.07516366,379.90891669)
\curveto(361.74702791,380.81516294)(362.08296507,382.0495367)(362.08297616,383.61204169)
\curveto(362.08296507,385.16672108)(361.74702791,386.3971886)(361.07516366,387.30344794)
\curveto(360.40327925,388.21749928)(359.49312391,388.67453008)(358.34469491,388.67454169)
\moveto(358.34469491,390.50266669)
\curveto(360.21968569,390.50265325)(361.69234046,389.89327886)(362.76266366,388.67454169)
\curveto(363.83296332,387.4557813)(364.36811904,385.76828298)(364.36813241,383.61204169)
\curveto(364.36811904,381.46359979)(363.83296332,379.77610148)(362.76266366,378.54954169)
\curveto(361.69234046,377.33079142)(360.21968569,376.72141703)(358.34469491,376.72141669)
\curveto(356.46187694,376.72141703)(354.98531592,377.33079142)(353.91500741,378.54954169)
\curveto(352.85250555,379.77610148)(352.32125608,381.46359979)(352.32125741,383.61204169)
\curveto(352.32125608,385.76828298)(352.85250555,387.4557813)(353.91500741,388.67454169)
\curveto(354.98531592,389.89327886)(356.46187694,390.50265325)(358.34469491,390.50266669)
}
}
{
\newrgbcolor{curcolor}{1 1 1}
\pscustom[linestyle=none,fillstyle=solid,fillcolor=curcolor]
{
\newpath
\moveto(151.8098362,349.9999964)
\lineto(401.00449258,349.9999964)
\curveto(406.79787809,349.9999964)(411.46186715,345.33600734)(411.46186715,339.54262183)
\lineto(411.46186715,330.68223021)
\curveto(411.46186715,324.8888447)(406.79787809,320.22485564)(401.00449258,320.22485564)
\lineto(151.8098362,320.22485564)
\curveto(146.01645069,320.22485564)(141.35246163,324.8888447)(141.35246163,330.68223021)
\lineto(141.35246163,339.54262183)
\curveto(141.35246163,345.33600734)(146.01645069,349.9999964)(151.8098362,349.9999964)
\closepath
}
}
{
\newrgbcolor{curcolor}{0 0 0}
\pscustom[linewidth=2,linecolor=curcolor]
{
\newpath
\moveto(151.8098362,349.9999964)
\lineto(401.00449258,349.9999964)
\curveto(406.79787809,349.9999964)(411.46186715,345.33600734)(411.46186715,339.54262183)
\lineto(411.46186715,330.68223021)
\curveto(411.46186715,324.8888447)(406.79787809,320.22485564)(401.00449258,320.22485564)
\lineto(151.8098362,320.22485564)
\curveto(146.01645069,320.22485564)(141.35246163,324.8888447)(141.35246163,330.68223021)
\lineto(141.35246163,339.54262183)
\curveto(141.35246163,345.33600734)(146.01645069,349.9999964)(151.8098362,349.9999964)
\closepath
}
}
{
\newrgbcolor{curcolor}{0 0 0}
\pscustom[linestyle=none,fillstyle=solid,fillcolor=curcolor]
{
\newpath
\moveto(152.35546761,347.49609015)
\lineto(154.72265511,347.49609015)
\lineto(154.72265511,331.2187464)
\curveto(154.72265039,329.10937229)(154.32030704,327.57812382)(153.51562386,326.6249964)
\curveto(152.71874614,325.67187573)(151.43359118,325.19531371)(149.66015511,325.1953089)
\lineto(148.75781136,325.1953089)
\lineto(148.75781136,327.1874964)
\lineto(149.49609261,327.1874964)
\curveto(150.54296707,327.18749921)(151.28124758,327.48046767)(151.71093636,328.06640265)
\curveto(152.14062172,328.6523415)(152.35546526,329.7031217)(152.35546761,331.2187464)
\lineto(152.35546761,347.49609015)
}
}
{
\newrgbcolor{curcolor}{0 0 0}
\pscustom[linestyle=none,fillstyle=solid,fillcolor=curcolor]
{
\newpath
\moveto(164.42578011,341.61327765)
\curveto(163.26952392,341.61326604)(162.35546234,341.16014149)(161.68359261,340.25390265)
\curveto(161.01171368,339.3554558)(160.67577652,338.12108203)(160.67578011,336.55077765)
\curveto(160.67577652,334.98046017)(161.00780743,333.74218016)(161.67187386,332.8359339)
\curveto(162.3437436,331.93749446)(163.26171143,331.48827616)(164.42578011,331.48827765)
\curveto(165.57420912,331.48827616)(166.48436446,331.94140071)(167.15624886,332.84765265)
\curveto(167.82811311,333.7538989)(168.16405028,334.98827266)(168.16406136,336.55077765)
\curveto(168.16405028,338.10545705)(167.82811311,339.33592457)(167.15624886,340.2421839)
\curveto(166.48436446,341.15623525)(165.57420912,341.61326604)(164.42578011,341.61327765)
\moveto(164.42578011,343.44140265)
\curveto(166.30077089,343.44138921)(167.77342567,342.83201482)(168.84374886,341.61327765)
\curveto(169.91404853,340.39451726)(170.44920424,338.70701895)(170.44921761,336.55077765)
\curveto(170.44920424,334.40233575)(169.91404853,332.71483744)(168.84374886,331.48827765)
\curveto(167.77342567,330.26952738)(166.30077089,329.66015299)(164.42578011,329.66015265)
\curveto(162.54296215,329.66015299)(161.06640112,330.26952738)(159.99609261,331.48827765)
\curveto(158.93359076,332.71483744)(158.40234129,334.40233575)(158.40234261,336.55077765)
\curveto(158.40234129,338.70701895)(158.93359076,340.39451726)(159.99609261,341.61327765)
\curveto(161.06640112,342.83201482)(162.54296215,343.44138921)(164.42578011,343.44140265)
}
}
{
\newrgbcolor{curcolor}{0 0 0}
\pscustom[linestyle=none,fillstyle=solid,fillcolor=curcolor]
{
\newpath
\moveto(173.78906136,335.1796839)
\lineto(173.78906136,343.1249964)
\lineto(175.94531136,343.1249964)
\lineto(175.94531136,335.26171515)
\curveto(175.94530717,334.01952363)(176.18749443,333.08593082)(176.67187386,332.4609339)
\curveto(177.15624346,331.84374456)(177.88280523,331.53515112)(178.85156136,331.53515265)
\curveto(180.0156156,331.53515112)(180.93358343,331.9062445)(181.60546761,332.6484339)
\curveto(182.28514458,333.39061801)(182.62498799,334.40233575)(182.62499886,335.68359015)
\lineto(182.62499886,343.1249964)
\lineto(184.78124886,343.1249964)
\lineto(184.78124886,329.9999964)
\lineto(182.62499886,329.9999964)
\lineto(182.62499886,332.0156214)
\curveto(182.10155101,331.21874518)(181.49217662,330.62499578)(180.79687386,330.2343714)
\curveto(180.1093655,329.85155905)(179.30858505,329.66015299)(178.39453011,329.66015265)
\curveto(176.88671248,329.66015299)(175.74218237,330.12890252)(174.96093636,331.06640265)
\curveto(174.17968393,332.00390065)(173.78905932,333.37499303)(173.78906136,335.1796839)
\moveto(179.21484261,343.44140265)
\lineto(179.21484261,343.44140265)
}
}
{
\newrgbcolor{curcolor}{0 0 0}
\pscustom[linestyle=none,fillstyle=solid,fillcolor=curcolor]
{
\newpath
\moveto(200.47265511,337.1015589)
\lineto(200.47265511,336.0468714)
\lineto(190.55859261,336.0468714)
\curveto(190.65233895,334.56249184)(191.097651,333.42968047)(191.89453011,332.6484339)
\curveto(192.6992119,331.87499453)(193.81639828,331.48827616)(195.24609261,331.48827765)
\curveto(196.07420852,331.48827616)(196.87498897,331.58983856)(197.64843636,331.79296515)
\curveto(198.42967492,331.99608816)(199.20311164,332.30077535)(199.96874886,332.70702765)
\lineto(199.96874886,330.66796515)
\curveto(199.19529915,330.33983981)(198.4023312,330.08984006)(197.58984261,329.91796515)
\curveto(196.77733282,329.74609041)(195.95311489,329.66015299)(195.11718636,329.66015265)
\curveto(193.02343032,329.66015299)(191.36327573,330.26952738)(190.13671761,331.48827765)
\curveto(188.91796568,332.70702495)(188.30859129,334.3554608)(188.30859261,336.43359015)
\curveto(188.30859129,338.58201907)(188.88671571,340.28514237)(190.04296761,341.54296515)
\curveto(191.20702589,342.80857734)(192.77343057,343.44138921)(194.74218636,343.44140265)
\curveto(196.50780184,343.44138921)(197.9023317,342.87107728)(198.92578011,341.73046515)
\curveto(199.95701714,340.59764205)(200.47264162,339.05467485)(200.47265511,337.1015589)
\moveto(198.31640511,337.7343714)
\curveto(198.3007688,338.91404999)(197.96873788,339.8554553)(197.32031136,340.55859015)
\curveto(196.67967667,341.26170389)(195.82811502,341.61326604)(194.76562386,341.61327765)
\curveto(193.56249229,341.61326604)(192.5976495,341.27342263)(191.87109261,340.5937464)
\curveto(191.15233845,339.91404899)(190.73827636,338.9570187)(190.62890511,337.72265265)
\lineto(198.31640511,337.7343714)
}
}
{
\newrgbcolor{curcolor}{0 0 0}
\pscustom[linestyle=none,fillstyle=solid,fillcolor=curcolor]
{
\newpath
\moveto(211.61718636,341.1093714)
\curveto(211.37498924,341.24998515)(211.1093645,341.35154755)(210.82031136,341.4140589)
\curveto(210.53905257,341.48435992)(210.22655289,341.51951613)(209.88281136,341.51952765)
\curveto(208.66405445,341.51951613)(207.72655539,341.12107903)(207.07031136,340.32421515)
\curveto(206.42186919,339.53514312)(206.09765077,338.3984255)(206.09765511,336.9140589)
\lineto(206.09765511,329.9999964)
\lineto(203.92968636,329.9999964)
\lineto(203.92968636,343.1249964)
\lineto(206.09765511,343.1249964)
\lineto(206.09765511,341.0859339)
\curveto(206.55077531,341.88279702)(207.14061847,342.47264018)(207.86718636,342.85546515)
\curveto(208.59374202,343.24607691)(209.47655364,343.44138921)(210.51562386,343.44140265)
\curveto(210.66405245,343.44138921)(210.82811479,343.42967047)(211.00781136,343.4062464)
\curveto(211.18748943,343.39060801)(211.38670798,343.36326429)(211.60546761,343.32421515)
\lineto(211.61718636,341.1093714)
}
}
{
\newrgbcolor{curcolor}{0 0 0}
\pscustom[linestyle=none,fillstyle=solid,fillcolor=curcolor]
{
}
}
{
\newrgbcolor{curcolor}{0 0 0}
\pscustom[linestyle=none,fillstyle=solid,fillcolor=curcolor]
{
\newpath
\moveto(232.76953011,337.1015589)
\lineto(232.76953011,336.0468714)
\lineto(222.85546761,336.0468714)
\curveto(222.94921395,334.56249184)(223.394526,333.42968047)(224.19140511,332.6484339)
\curveto(224.9960869,331.87499453)(226.11327328,331.48827616)(227.54296761,331.48827765)
\curveto(228.37108352,331.48827616)(229.17186397,331.58983856)(229.94531136,331.79296515)
\curveto(230.72654992,331.99608816)(231.49998664,332.30077535)(232.26562386,332.70702765)
\lineto(232.26562386,330.66796515)
\curveto(231.49217415,330.33983981)(230.6992062,330.08984006)(229.88671761,329.91796515)
\curveto(229.07420782,329.74609041)(228.24998989,329.66015299)(227.41406136,329.66015265)
\curveto(225.32030532,329.66015299)(223.66015073,330.26952738)(222.43359261,331.48827765)
\curveto(221.21484068,332.70702495)(220.60546629,334.3554608)(220.60546761,336.43359015)
\curveto(220.60546629,338.58201907)(221.18359071,340.28514237)(222.33984261,341.54296515)
\curveto(223.50390089,342.80857734)(225.07030557,343.44138921)(227.03906136,343.44140265)
\curveto(228.80467684,343.44138921)(230.1992067,342.87107728)(231.22265511,341.73046515)
\curveto(232.25389214,340.59764205)(232.76951662,339.05467485)(232.76953011,337.1015589)
\moveto(230.61328011,337.7343714)
\curveto(230.5976438,338.91404999)(230.26561288,339.8554553)(229.61718636,340.55859015)
\curveto(228.97655167,341.26170389)(228.12499002,341.61326604)(227.06249886,341.61327765)
\curveto(225.85936729,341.61326604)(224.8945245,341.27342263)(224.16796761,340.5937464)
\curveto(223.44921345,339.91404899)(223.03515136,338.9570187)(222.92578011,337.72265265)
\lineto(230.61328011,337.7343714)
}
}
{
\newrgbcolor{curcolor}{0 0 0}
\pscustom[linestyle=none,fillstyle=solid,fillcolor=curcolor]
{
\newpath
\moveto(247.21874886,337.9218714)
\lineto(247.21874886,329.9999964)
\lineto(245.06249886,329.9999964)
\lineto(245.06249886,337.8515589)
\curveto(245.06248785,339.09373731)(244.82030059,340.02342388)(244.33593636,340.6406214)
\curveto(243.85155156,341.25779764)(243.12498979,341.56639109)(242.15624886,341.56640265)
\curveto(240.99217942,341.56639109)(240.07421159,341.19529771)(239.40234261,340.4531214)
\curveto(238.73046293,339.71092419)(238.39452577,338.69920645)(238.39453011,337.41796515)
\lineto(238.39453011,329.9999964)
\lineto(236.22656136,329.9999964)
\lineto(236.22656136,343.1249964)
\lineto(238.39453011,343.1249964)
\lineto(238.39453011,341.0859339)
\curveto(238.91015025,341.87498453)(239.51561839,342.46482769)(240.21093636,342.85546515)
\curveto(240.9140545,343.24607691)(241.72264744,343.44138921)(242.63671761,343.44140265)
\curveto(244.14452002,343.44138921)(245.28514388,342.97263968)(246.05859261,342.03515265)
\curveto(246.83201733,341.10545405)(247.21873569,339.73436167)(247.21874886,337.9218714)
}
}
{
\newrgbcolor{curcolor}{0 0 0}
\pscustom[linestyle=none,fillstyle=solid,fillcolor=curcolor]
{
}
}
{
\newrgbcolor{curcolor}{0 0 0}
\pscustom[linestyle=none,fillstyle=solid,fillcolor=curcolor]
{
\newpath
\moveto(269.40234261,340.60546515)
\curveto(269.94139209,341.57420358)(270.5859227,342.28904661)(271.33593636,342.7499964)
\curveto(272.0859212,343.21092069)(272.96873282,343.44138921)(273.98437386,343.44140265)
\curveto(275.35154293,343.44138921)(276.40622938,342.96092094)(277.14843636,341.9999964)
\curveto(277.89060289,341.04686036)(278.26169627,339.68748671)(278.26171761,337.9218714)
\lineto(278.26171761,329.9999964)
\lineto(276.09374886,329.9999964)
\lineto(276.09374886,337.8515589)
\curveto(276.09372969,339.10936229)(275.87107366,340.04295511)(275.42578011,340.65234015)
\curveto(274.98044955,341.26170389)(274.30076273,341.56639109)(273.38671761,341.56640265)
\curveto(272.26951477,341.56639109)(271.38670315,341.19529771)(270.73828011,340.4531214)
\curveto(270.08982945,339.71092419)(269.76561102,338.69920645)(269.76562386,337.41796515)
\lineto(269.76562386,329.9999964)
\lineto(267.59765511,329.9999964)
\lineto(267.59765511,337.8515589)
\curveto(267.59764444,339.11717479)(267.37498841,340.0507676)(266.92968636,340.65234015)
\curveto(266.4843643,341.26170389)(265.79686499,341.56639109)(264.86718636,341.56640265)
\curveto(263.76561702,341.56639109)(262.89061789,341.19139146)(262.24218636,340.44140265)
\curveto(261.59374419,339.69920545)(261.26952577,338.69139396)(261.26953011,337.41796515)
\lineto(261.26953011,329.9999964)
\lineto(259.10156136,329.9999964)
\lineto(259.10156136,343.1249964)
\lineto(261.26953011,343.1249964)
\lineto(261.26953011,341.0859339)
\curveto(261.76171277,341.89060951)(262.35155593,342.48435892)(263.03906136,342.8671839)
\curveto(263.72655456,343.24998315)(264.54295999,343.44138921)(265.48828011,343.44140265)
\curveto(266.44139559,343.44138921)(267.24998854,343.19920195)(267.91406136,342.71484015)
\curveto(268.5859247,342.23045292)(269.08201795,341.52732862)(269.40234261,340.60546515)
}
}
{
\newrgbcolor{curcolor}{0 0 0}
\pscustom[linestyle=none,fillstyle=solid,fillcolor=curcolor]
{
\newpath
\moveto(282.35156136,335.1796839)
\lineto(282.35156136,343.1249964)
\lineto(284.50781136,343.1249964)
\lineto(284.50781136,335.26171515)
\curveto(284.50780717,334.01952363)(284.74999443,333.08593082)(285.23437386,332.4609339)
\curveto(285.71874346,331.84374456)(286.44530523,331.53515112)(287.41406136,331.53515265)
\curveto(288.5781156,331.53515112)(289.49608343,331.9062445)(290.16796761,332.6484339)
\curveto(290.84764458,333.39061801)(291.18748799,334.40233575)(291.18749886,335.68359015)
\lineto(291.18749886,343.1249964)
\lineto(293.34374886,343.1249964)
\lineto(293.34374886,329.9999964)
\lineto(291.18749886,329.9999964)
\lineto(291.18749886,332.0156214)
\curveto(290.66405101,331.21874518)(290.05467662,330.62499578)(289.35937386,330.2343714)
\curveto(288.6718655,329.85155905)(287.87108505,329.66015299)(286.95703011,329.66015265)
\curveto(285.44921248,329.66015299)(284.30468237,330.12890252)(283.52343636,331.06640265)
\curveto(282.74218393,332.00390065)(282.35155932,333.37499303)(282.35156136,335.1796839)
\moveto(287.77734261,343.44140265)
\lineto(287.77734261,343.44140265)
}
}
{
\newrgbcolor{curcolor}{0 0 0}
\pscustom[linestyle=none,fillstyle=solid,fillcolor=curcolor]
{
\newpath
\moveto(297.80859261,348.2343714)
\lineto(299.96484261,348.2343714)
\lineto(299.96484261,329.9999964)
\lineto(297.80859261,329.9999964)
\lineto(297.80859261,348.2343714)
}
}
{
\newrgbcolor{curcolor}{0 0 0}
\pscustom[linestyle=none,fillstyle=solid,fillcolor=curcolor]
{
\newpath
\moveto(306.59765511,346.8515589)
\lineto(306.59765511,343.1249964)
\lineto(311.03906136,343.1249964)
\lineto(311.03906136,341.44921515)
\lineto(306.59765511,341.44921515)
\lineto(306.59765511,334.32421515)
\curveto(306.59765072,333.2538994)(306.74218182,332.56640009)(307.03124886,332.26171515)
\curveto(307.32811874,331.9570257)(307.92577439,331.8046821)(308.82421761,331.8046839)
\lineto(311.03906136,331.8046839)
\lineto(311.03906136,329.9999964)
\lineto(308.82421761,329.9999964)
\curveto(307.16015016,329.9999964)(306.0117138,330.30858984)(305.37890511,330.92577765)
\curveto(304.74609007,331.5507761)(304.42968414,332.68358747)(304.42968636,334.32421515)
\lineto(304.42968636,341.44921515)
\lineto(302.84765511,341.44921515)
\lineto(302.84765511,343.1249964)
\lineto(304.42968636,343.1249964)
\lineto(304.42968636,346.8515589)
\lineto(306.59765511,346.8515589)
}
}
{
\newrgbcolor{curcolor}{0 0 0}
\pscustom[linestyle=none,fillstyle=solid,fillcolor=curcolor]
{
\newpath
\moveto(313.88671761,343.1249964)
\lineto(316.04296761,343.1249964)
\lineto(316.04296761,329.9999964)
\lineto(313.88671761,329.9999964)
\lineto(313.88671761,343.1249964)
\moveto(313.88671761,348.2343714)
\lineto(316.04296761,348.2343714)
\lineto(316.04296761,345.50390265)
\lineto(313.88671761,345.50390265)
\lineto(313.88671761,348.2343714)
}
}
{
\newrgbcolor{curcolor}{0 0 0}
\pscustom[linestyle=none,fillstyle=solid,fillcolor=curcolor]
{
\newpath
\moveto(320.54296761,343.1249964)
\lineto(322.69921761,343.1249964)
\lineto(322.69921761,329.7656214)
\curveto(322.6992132,328.09374831)(322.37890102,326.88281202)(321.73828011,326.1328089)
\curveto(321.10546479,325.38281352)(320.08202831,325.00781389)(318.66796761,325.0078089)
\lineto(317.84765511,325.0078089)
\lineto(317.84765511,326.8359339)
\lineto(318.42187386,326.8359339)
\curveto(319.2421854,326.83593707)(319.80077859,327.02734313)(320.09765511,327.41015265)
\curveto(320.394528,327.78515487)(320.54296535,328.57031033)(320.54296761,329.7656214)
\lineto(320.54296761,343.1249964)
\moveto(320.54296761,348.2343714)
\lineto(322.69921761,348.2343714)
\lineto(322.69921761,345.50390265)
\lineto(320.54296761,345.50390265)
\lineto(320.54296761,348.2343714)
}
}
{
\newrgbcolor{curcolor}{0 0 0}
\pscustom[linestyle=none,fillstyle=solid,fillcolor=curcolor]
{
\newpath
\moveto(332.28515511,341.61327765)
\curveto(331.12889892,341.61326604)(330.21483734,341.16014149)(329.54296761,340.25390265)
\curveto(328.87108868,339.3554558)(328.53515152,338.12108203)(328.53515511,336.55077765)
\curveto(328.53515152,334.98046017)(328.86718243,333.74218016)(329.53124886,332.8359339)
\curveto(330.2031186,331.93749446)(331.12108643,331.48827616)(332.28515511,331.48827765)
\curveto(333.43358412,331.48827616)(334.34373946,331.94140071)(335.01562386,332.84765265)
\curveto(335.68748811,333.7538989)(336.02342528,334.98827266)(336.02343636,336.55077765)
\curveto(336.02342528,338.10545705)(335.68748811,339.33592457)(335.01562386,340.2421839)
\curveto(334.34373946,341.15623525)(333.43358412,341.61326604)(332.28515511,341.61327765)
\moveto(332.28515511,343.44140265)
\curveto(334.16014589,343.44138921)(335.63280067,342.83201482)(336.70312386,341.61327765)
\curveto(337.77342353,340.39451726)(338.30857924,338.70701895)(338.30859261,336.55077765)
\curveto(338.30857924,334.40233575)(337.77342353,332.71483744)(336.70312386,331.48827765)
\curveto(335.63280067,330.26952738)(334.16014589,329.66015299)(332.28515511,329.66015265)
\curveto(330.40233715,329.66015299)(328.92577612,330.26952738)(327.85546761,331.48827765)
\curveto(326.79296576,332.71483744)(326.26171629,334.40233575)(326.26171761,336.55077765)
\curveto(326.26171629,338.70701895)(326.79296576,340.39451726)(327.85546761,341.61327765)
\curveto(328.92577612,342.83201482)(330.40233715,343.44138921)(332.28515511,343.44140265)
}
}
{
\newrgbcolor{curcolor}{0 0 0}
\pscustom[linestyle=none,fillstyle=solid,fillcolor=curcolor]
{
\newpath
\moveto(341.64843636,335.1796839)
\lineto(341.64843636,343.1249964)
\lineto(343.80468636,343.1249964)
\lineto(343.80468636,335.26171515)
\curveto(343.80468217,334.01952363)(344.04686943,333.08593082)(344.53124886,332.4609339)
\curveto(345.01561846,331.84374456)(345.74218023,331.53515112)(346.71093636,331.53515265)
\curveto(347.8749906,331.53515112)(348.79295843,331.9062445)(349.46484261,332.6484339)
\curveto(350.14451958,333.39061801)(350.48436299,334.40233575)(350.48437386,335.68359015)
\lineto(350.48437386,343.1249964)
\lineto(352.64062386,343.1249964)
\lineto(352.64062386,329.9999964)
\lineto(350.48437386,329.9999964)
\lineto(350.48437386,332.0156214)
\curveto(349.96092601,331.21874518)(349.35155162,330.62499578)(348.65624886,330.2343714)
\curveto(347.9687405,329.85155905)(347.16796005,329.66015299)(346.25390511,329.66015265)
\curveto(344.74608748,329.66015299)(343.60155737,330.12890252)(342.82031136,331.06640265)
\curveto(342.03905893,332.00390065)(341.64843432,333.37499303)(341.64843636,335.1796839)
\moveto(347.07421761,343.44140265)
\lineto(347.07421761,343.44140265)
}
}
{
\newrgbcolor{curcolor}{0 0 0}
\pscustom[linestyle=none,fillstyle=solid,fillcolor=curcolor]
{
\newpath
\moveto(368.33203011,337.1015589)
\lineto(368.33203011,336.0468714)
\lineto(358.41796761,336.0468714)
\curveto(358.51171395,334.56249184)(358.957026,333.42968047)(359.75390511,332.6484339)
\curveto(360.5585869,331.87499453)(361.67577328,331.48827616)(363.10546761,331.48827765)
\curveto(363.93358352,331.48827616)(364.73436397,331.58983856)(365.50781136,331.79296515)
\curveto(366.28904992,331.99608816)(367.06248664,332.30077535)(367.82812386,332.70702765)
\lineto(367.82812386,330.66796515)
\curveto(367.05467415,330.33983981)(366.2617062,330.08984006)(365.44921761,329.91796515)
\curveto(364.63670782,329.74609041)(363.81248989,329.66015299)(362.97656136,329.66015265)
\curveto(360.88280532,329.66015299)(359.22265073,330.26952738)(357.99609261,331.48827765)
\curveto(356.77734068,332.70702495)(356.16796629,334.3554608)(356.16796761,336.43359015)
\curveto(356.16796629,338.58201907)(356.74609071,340.28514237)(357.90234261,341.54296515)
\curveto(359.06640089,342.80857734)(360.63280557,343.44138921)(362.60156136,343.44140265)
\curveto(364.36717684,343.44138921)(365.7617067,342.87107728)(366.78515511,341.73046515)
\curveto(367.81639214,340.59764205)(368.33201662,339.05467485)(368.33203011,337.1015589)
\moveto(366.17578011,337.7343714)
\curveto(366.1601438,338.91404999)(365.82811288,339.8554553)(365.17968636,340.55859015)
\curveto(364.53905167,341.26170389)(363.68749002,341.61326604)(362.62499886,341.61327765)
\curveto(361.42186729,341.61326604)(360.4570245,341.27342263)(359.73046761,340.5937464)
\curveto(359.01171345,339.91404899)(358.59765136,338.9570187)(358.48828011,337.72265265)
\lineto(366.17578011,337.7343714)
}
}
{
\newrgbcolor{curcolor}{0 0 0}
\pscustom[linestyle=none,fillstyle=solid,fillcolor=curcolor]
{
\newpath
\moveto(371.64843636,335.1796839)
\lineto(371.64843636,343.1249964)
\lineto(373.80468636,343.1249964)
\lineto(373.80468636,335.26171515)
\curveto(373.80468217,334.01952363)(374.04686943,333.08593082)(374.53124886,332.4609339)
\curveto(375.01561846,331.84374456)(375.74218023,331.53515112)(376.71093636,331.53515265)
\curveto(377.8749906,331.53515112)(378.79295843,331.9062445)(379.46484261,332.6484339)
\curveto(380.14451958,333.39061801)(380.48436299,334.40233575)(380.48437386,335.68359015)
\lineto(380.48437386,343.1249964)
\lineto(382.64062386,343.1249964)
\lineto(382.64062386,329.9999964)
\lineto(380.48437386,329.9999964)
\lineto(380.48437386,332.0156214)
\curveto(379.96092601,331.21874518)(379.35155162,330.62499578)(378.65624886,330.2343714)
\curveto(377.9687405,329.85155905)(377.16796005,329.66015299)(376.25390511,329.66015265)
\curveto(374.74608748,329.66015299)(373.60155737,330.12890252)(372.82031136,331.06640265)
\curveto(372.03905893,332.00390065)(371.64843432,333.37499303)(371.64843636,335.1796839)
\moveto(377.07421761,343.44140265)
\lineto(377.07421761,343.44140265)
}
}
{
\newrgbcolor{curcolor}{0 0 0}
\pscustom[linestyle=none,fillstyle=solid,fillcolor=curcolor]
{
\newpath
\moveto(394.71093636,341.1093714)
\curveto(394.46873924,341.24998515)(394.2031145,341.35154755)(393.91406136,341.4140589)
\curveto(393.63280257,341.48435992)(393.32030289,341.51951613)(392.97656136,341.51952765)
\curveto(391.75780445,341.51951613)(390.82030539,341.12107903)(390.16406136,340.32421515)
\curveto(389.51561919,339.53514312)(389.19140077,338.3984255)(389.19140511,336.9140589)
\lineto(389.19140511,329.9999964)
\lineto(387.02343636,329.9999964)
\lineto(387.02343636,343.1249964)
\lineto(389.19140511,343.1249964)
\lineto(389.19140511,341.0859339)
\curveto(389.64452531,341.88279702)(390.23436847,342.47264018)(390.96093636,342.85546515)
\curveto(391.68749202,343.24607691)(392.57030364,343.44138921)(393.60937386,343.44140265)
\curveto(393.75780245,343.44138921)(393.92186479,343.42967047)(394.10156136,343.4062464)
\curveto(394.28123943,343.39060801)(394.48045798,343.36326429)(394.69921761,343.32421515)
\lineto(394.71093636,341.1093714)
}
}
{
\newrgbcolor{curcolor}{0 0 0}
\pscustom[linestyle=none,fillstyle=solid,fillcolor=curcolor]
{
\newpath
\moveto(405.36328011,342.73827765)
\lineto(405.36328011,340.69921515)
\curveto(404.75389509,341.01170414)(404.12108323,341.24607891)(403.46484261,341.40234015)
\curveto(402.80858454,341.55857859)(402.12889772,341.63670352)(401.42578011,341.63671515)
\curveto(400.35546199,341.63670352)(399.5507753,341.47264118)(399.01171761,341.14452765)
\curveto(398.48046387,340.81639184)(398.21483913,340.32420483)(398.21484261,339.66796515)
\curveto(398.21483913,339.16795598)(398.40624519,338.77342513)(398.78906136,338.4843714)
\curveto(399.17186943,338.2031132)(399.94139991,337.93358222)(401.09765511,337.67577765)
\lineto(401.83593636,337.51171515)
\curveto(403.36717773,337.18358297)(404.45311414,336.71873968)(405.09374886,336.1171839)
\curveto(405.74217536,335.52342838)(406.06639378,334.69139796)(406.06640511,333.62109015)
\curveto(406.06639378,332.40233775)(405.58201927,331.43749496)(404.61328011,330.7265589)
\curveto(403.6523337,330.01562139)(402.32811627,329.66015299)(400.64062386,329.66015265)
\curveto(399.93749366,329.66015299)(399.20311939,329.73046542)(398.43749886,329.87109015)
\curveto(397.67968342,330.00390265)(396.87890297,330.20702745)(396.03515511,330.48046515)
\lineto(396.03515511,332.70702765)
\curveto(396.83202802,332.29296286)(397.61718348,331.98046317)(398.39062386,331.76952765)
\curveto(399.16405693,331.56640109)(399.92968117,331.46483869)(400.68749886,331.46484015)
\curveto(401.70311689,331.46483869)(402.48436611,331.63671352)(403.03124886,331.98046515)
\curveto(403.57811502,332.33202532)(403.85155225,332.82421233)(403.85156136,333.45702765)
\curveto(403.85155225,334.04296111)(403.6523337,334.49217941)(403.25390511,334.8046839)
\curveto(402.86327198,335.11717879)(401.9999916,335.41795973)(400.66406136,335.70702765)
\lineto(399.91406136,335.8828089)
\curveto(398.57812002,336.16405274)(397.61327723,336.59373981)(397.01953011,337.1718714)
\curveto(396.42577842,337.75780114)(396.12890372,338.55858159)(396.12890511,339.57421515)
\curveto(396.12890372,340.80857934)(396.56640328,341.76170339)(397.44140511,342.43359015)
\curveto(398.31640153,343.10545205)(399.55858779,343.44138921)(401.16796761,343.44140265)
\curveto(401.96483538,343.44138921)(402.71483463,343.38279552)(403.41796761,343.2656214)
\curveto(404.12108323,343.14842075)(404.76952008,342.97263968)(405.36328011,342.73827765)
}
}
{
\newrgbcolor{curcolor}{1 1 1}
\pscustom[linestyle=none,fillstyle=solid,fillcolor=curcolor]
{
\newpath
\moveto(230.75260812,299.9999964)
\lineto(330.59985238,299.9999964)
\curveto(335.80753353,299.9999964)(339.99999886,295.80753107)(339.99999886,290.59984992)
\lineto(339.99999886,279.61122916)
\curveto(339.99999886,274.40354801)(335.80753353,270.21108268)(330.59985238,270.21108268)
\lineto(230.75260812,270.21108268)
\curveto(225.54492696,270.21108268)(221.35246163,274.40354801)(221.35246163,279.61122916)
\lineto(221.35246163,290.59984992)
\curveto(221.35246163,295.80753107)(225.54492696,299.9999964)(230.75260812,299.9999964)
\closepath
}
}
{
\newrgbcolor{curcolor}{0 0 0}
\pscustom[linewidth=2,linecolor=curcolor]
{
\newpath
\moveto(230.75260812,299.9999964)
\lineto(330.59985238,299.9999964)
\curveto(335.80753353,299.9999964)(339.99999886,295.80753107)(339.99999886,290.59984992)
\lineto(339.99999886,279.61122916)
\curveto(339.99999886,274.40354801)(335.80753353,270.21108268)(330.59985238,270.21108268)
\lineto(230.75260812,270.21108268)
\curveto(225.54492696,270.21108268)(221.35246163,274.40354801)(221.35246163,279.61122916)
\lineto(221.35246163,290.59984992)
\curveto(221.35246163,295.80753107)(225.54492696,299.9999964)(230.75260812,299.9999964)
\closepath
}
}
{
\newrgbcolor{curcolor}{0 0 0}
\pscustom[linestyle=none,fillstyle=solid,fillcolor=curcolor]
{
\newpath
\moveto(245.21453744,293.55099128)
\curveto(243.4957797,293.55097538)(242.12859357,292.91035103)(241.11297494,291.62911628)
\curveto(240.10515809,290.34785359)(239.60125234,288.60176158)(239.60125619,286.39083503)
\curveto(239.60125234,284.1877035)(240.10515809,282.44551774)(241.11297494,281.16427253)
\curveto(242.12859357,279.8830203)(243.4957797,279.24239594)(245.21453744,279.24239753)
\curveto(246.93327626,279.24239594)(248.2926499,279.8830203)(249.29266244,281.16427253)
\curveto(250.30046039,282.44551774)(250.80436614,284.1877035)(250.80438119,286.39083503)
\curveto(250.80436614,288.60176158)(250.30046039,290.34785359)(249.29266244,291.62911628)
\curveto(248.2926499,292.91035103)(246.93327626,293.55097538)(245.21453744,293.55099128)
\moveto(245.21453744,295.47286628)
\curveto(247.66765053,295.47284846)(249.62858607,294.64863054)(251.09734994,293.00021003)
\curveto(252.56608313,291.35957133)(253.30045739,289.15644853)(253.30047494,286.39083503)
\curveto(253.30045739,283.63301655)(252.56608313,281.42989376)(251.09734994,279.78146003)
\curveto(249.62858607,278.14083454)(247.66765053,277.32052287)(245.21453744,277.32052253)
\curveto(242.75359294,277.32052287)(240.78484491,278.14083454)(239.30828744,279.78146003)
\curveto(237.83953536,281.42208126)(237.10516109,283.62520406)(237.10516244,286.39083503)
\curveto(237.10516109,289.15644853)(237.83953536,291.35957133)(239.30828744,293.00021003)
\curveto(240.78484491,294.64863054)(242.75359294,295.47284846)(245.21453744,295.47286628)
}
}
{
\newrgbcolor{curcolor}{0 0 0}
\pscustom[linestyle=none,fillstyle=solid,fillcolor=curcolor]
{
\newpath
\moveto(258.99578744,279.62911628)
\lineto(258.99578744,272.66817878)
\lineto(256.82781869,272.66817878)
\lineto(256.82781869,290.78536628)
\lineto(258.99578744,290.78536628)
\lineto(258.99578744,288.79317878)
\curveto(259.44890764,289.57441686)(260.01921957,290.15254128)(260.70672494,290.52755378)
\curveto(261.40203068,290.91035303)(262.23015486,291.10175908)(263.19109994,291.10177253)
\curveto(264.7848398,291.10175908)(266.07780726,290.46894722)(267.07000619,289.20333503)
\curveto(268.06999277,287.93769975)(268.56999227,286.27363891)(268.57000619,284.21114753)
\curveto(268.56999227,282.14864304)(268.06999277,280.4845822)(267.07000619,279.21896003)
\curveto(266.07780726,277.95333473)(264.7848398,277.32052287)(263.19109994,277.32052253)
\curveto(262.23015486,277.32052287)(261.40203068,277.50802268)(260.70672494,277.88302253)
\curveto(260.01921957,278.26583442)(259.44890764,278.84786509)(258.99578744,279.62911628)
\moveto(266.33172494,284.21114753)
\curveto(266.33171325,285.79707689)(266.00358858,287.03926315)(265.34734994,287.93771003)
\curveto(264.69890239,288.84394884)(263.80437203,289.29707339)(262.66375619,289.29708503)
\curveto(261.52312431,289.29707339)(260.62468771,288.84394884)(259.96844369,287.93771003)
\curveto(259.32000152,287.03926315)(258.99578309,285.79707689)(258.99578744,284.21114753)
\curveto(258.99578309,282.62520506)(259.32000152,281.37911256)(259.96844369,280.47286628)
\curveto(260.62468771,279.57442686)(261.52312431,279.12520856)(262.66375619,279.12521003)
\curveto(263.80437203,279.12520856)(264.69890239,279.57442686)(265.34734994,280.47286628)
\curveto(266.00358858,281.37911256)(266.33171325,282.62520506)(266.33172494,284.21114753)
}
}
{
\newrgbcolor{curcolor}{0 0 0}
\pscustom[linestyle=none,fillstyle=solid,fillcolor=curcolor]
{
\newpath
\moveto(274.27703744,294.51192878)
\lineto(274.27703744,290.78536628)
\lineto(278.71844369,290.78536628)
\lineto(278.71844369,289.10958503)
\lineto(274.27703744,289.10958503)
\lineto(274.27703744,281.98458503)
\curveto(274.27703304,280.91426927)(274.42156415,280.22676996)(274.71063119,279.92208503)
\curveto(275.00750106,279.61739557)(275.60515671,279.46505197)(276.50359994,279.46505378)
\lineto(278.71844369,279.46505378)
\lineto(278.71844369,277.66036628)
\lineto(276.50359994,277.66036628)
\curveto(274.83953248,277.66036628)(273.69109613,277.96895972)(273.05828744,278.58614753)
\curveto(272.42547239,279.21114597)(272.10906646,280.34395734)(272.10906869,281.98458503)
\lineto(272.10906869,289.10958503)
\lineto(270.52703744,289.10958503)
\lineto(270.52703744,290.78536628)
\lineto(272.10906869,290.78536628)
\lineto(272.10906869,294.51192878)
\lineto(274.27703744,294.51192878)
}
}
{
\newrgbcolor{curcolor}{0 0 0}
\pscustom[linestyle=none,fillstyle=solid,fillcolor=curcolor]
{
\newpath
\moveto(281.56609994,290.78536628)
\lineto(283.72234994,290.78536628)
\lineto(283.72234994,277.66036628)
\lineto(281.56609994,277.66036628)
\lineto(281.56609994,290.78536628)
\moveto(281.56609994,295.89474128)
\lineto(283.72234994,295.89474128)
\lineto(283.72234994,293.16427253)
\lineto(281.56609994,293.16427253)
\lineto(281.56609994,295.89474128)
}
}
{
\newrgbcolor{curcolor}{0 0 0}
\pscustom[linestyle=none,fillstyle=solid,fillcolor=curcolor]
{
\newpath
\moveto(293.30828744,289.27364753)
\curveto(292.15203125,289.27363591)(291.23796966,288.82051137)(290.56609994,287.91427253)
\curveto(289.894221,287.01582567)(289.55828384,285.7814519)(289.55828744,284.21114753)
\curveto(289.55828384,282.64083004)(289.89031476,281.40255003)(290.55438119,280.49630378)
\curveto(291.22625092,279.59786434)(292.14421875,279.14864604)(293.30828744,279.14864753)
\curveto(294.45671644,279.14864604)(295.36687178,279.60177058)(296.03875619,280.50802253)
\curveto(296.71062044,281.41426877)(297.0465576,282.64864254)(297.04656869,284.21114753)
\curveto(297.0465576,285.76582692)(296.71062044,286.99629444)(296.03875619,287.90255378)
\curveto(295.36687178,288.81660512)(294.45671644,289.27363591)(293.30828744,289.27364753)
\moveto(293.30828744,291.10177253)
\curveto(295.18327821,291.10175908)(296.65593299,290.49238469)(297.72625619,289.27364753)
\curveto(298.79655585,288.05488713)(299.33171157,286.36738882)(299.33172494,284.21114753)
\curveto(299.33171157,282.06270562)(298.79655585,280.37520731)(297.72625619,279.14864753)
\curveto(296.65593299,277.92989726)(295.18327821,277.32052287)(293.30828744,277.32052253)
\curveto(291.42546947,277.32052287)(289.94890845,277.92989726)(288.87859994,279.14864753)
\curveto(287.81609808,280.37520731)(287.28484861,282.06270562)(287.28484994,284.21114753)
\curveto(287.28484861,286.36738882)(287.81609808,288.05488713)(288.87859994,289.27364753)
\curveto(289.94890845,290.49238469)(291.42546947,291.10175908)(293.30828744,291.10177253)
}
}
{
\newrgbcolor{curcolor}{0 0 0}
\pscustom[linestyle=none,fillstyle=solid,fillcolor=curcolor]
{
\newpath
\moveto(313.80438119,285.58224128)
\lineto(313.80438119,277.66036628)
\lineto(311.64813119,277.66036628)
\lineto(311.64813119,285.51192878)
\curveto(311.64812017,286.75410718)(311.40593291,287.68379375)(310.92156869,288.30099128)
\curveto(310.43718388,288.91816752)(309.71062211,289.22676096)(308.74188119,289.22677253)
\curveto(307.57781174,289.22676096)(306.65984391,288.85566758)(305.98797494,288.11349128)
\curveto(305.31609525,287.37129406)(304.98015809,286.35957633)(304.98016244,285.07833503)
\lineto(304.98016244,277.66036628)
\lineto(302.81219369,277.66036628)
\lineto(302.81219369,290.78536628)
\lineto(304.98016244,290.78536628)
\lineto(304.98016244,288.74630378)
\curveto(305.49578257,289.5353544)(306.10125072,290.12519756)(306.79656869,290.51583503)
\curveto(307.49968682,290.90644678)(308.30827976,291.10175908)(309.22234994,291.10177253)
\curveto(310.73015234,291.10175908)(311.8707762,290.63300955)(312.64422494,289.69552253)
\curveto(313.41764965,288.76582392)(313.80436802,287.39473154)(313.80438119,285.58224128)
}
}
{
\newrgbcolor{curcolor}{0 0 0}
\pscustom[linestyle=none,fillstyle=solid,fillcolor=curcolor]
{
\newpath
\moveto(326.49578744,290.39864753)
\lineto(326.49578744,288.35958503)
\curveto(325.88640242,288.67207401)(325.25359055,288.90644878)(324.59734994,289.06271003)
\curveto(323.94109186,289.21894847)(323.26140504,289.29707339)(322.55828744,289.29708503)
\curveto(321.48796932,289.29707339)(320.68328262,289.13301105)(320.14422494,288.80489753)
\curveto(319.61297119,288.47676171)(319.34734646,287.9845747)(319.34734994,287.32833503)
\curveto(319.34734646,286.82832586)(319.53875252,286.433795)(319.92156869,286.14474128)
\curveto(320.30437675,285.86348307)(321.07390723,285.59395209)(322.23016244,285.33614753)
\lineto(322.96844369,285.17208503)
\curveto(324.49968505,284.84395284)(325.58562147,284.37910956)(326.22625619,283.77755378)
\curveto(326.87468268,283.18379825)(327.19890111,282.35176783)(327.19891244,281.28146003)
\curveto(327.19890111,280.06270762)(326.71452659,279.09786484)(325.74578744,278.38692878)
\curveto(324.78484102,277.67599126)(323.46062359,277.32052287)(321.77313119,277.32052253)
\curveto(321.07000098,277.32052287)(320.33562672,277.39083529)(319.57000619,277.53146003)
\curveto(318.81219074,277.66427252)(318.01141029,277.86739732)(317.16766244,278.14083503)
\lineto(317.16766244,280.36739753)
\curveto(317.96453534,279.95333273)(318.7496908,279.64083304)(319.52313119,279.42989753)
\curveto(320.29656426,279.22677096)(321.06218849,279.12520856)(321.82000619,279.12521003)
\curveto(322.83562422,279.12520856)(323.61687344,279.29708339)(324.16375619,279.64083503)
\curveto(324.71062234,279.99239519)(324.98405957,280.4845822)(324.98406869,281.11739753)
\curveto(324.98405957,281.70333098)(324.78484102,282.15254928)(324.38641244,282.46505378)
\curveto(323.99577931,282.77754866)(323.13249892,283.07832961)(321.79656869,283.36739753)
\lineto(321.04656869,283.54317878)
\curveto(319.71062734,283.82442261)(318.74578456,284.25410968)(318.15203744,284.83224128)
\curveto(317.55828575,285.41817102)(317.26141104,286.21895147)(317.26141244,287.23458503)
\curveto(317.26141104,288.46894922)(317.69891061,289.42207326)(318.57391244,290.09396003)
\curveto(319.44890886,290.76582192)(320.69109511,291.10175908)(322.30047494,291.10177253)
\curveto(323.09734271,291.10175908)(323.84734196,291.04316539)(324.55047494,290.92599128)
\curveto(325.25359055,290.80879063)(325.9020274,290.63300955)(326.49578744,290.39864753)
}
}
{
\newrgbcolor{curcolor}{1 1 1}
\pscustom[linestyle=none,fillstyle=solid,fillcolor=curcolor]
{
\newpath
\moveto(182.63554459,249.84527228)
\lineto(378.59052735,249.84527228)
\curveto(384.94255905,249.84527228)(390.05628854,244.73154279)(390.05628854,238.3795111)
\lineto(390.05628854,231.39312384)
\curveto(390.05628854,225.04109215)(384.94255905,219.92736266)(378.59052735,219.92736266)
\lineto(182.63554459,219.92736266)
\curveto(176.2835129,219.92736266)(171.16978341,225.04109215)(171.16978341,231.39312384)
\lineto(171.16978341,238.3795111)
\curveto(171.16978341,244.73154279)(176.2835129,249.84527228)(182.63554459,249.84527228)
\closepath
}
}
{
\newrgbcolor{curcolor}{0 0 0}
\pscustom[linewidth=1.5483532,linecolor=curcolor]
{
\newpath
\moveto(182.63554459,249.84527228)
\lineto(378.59052735,249.84527228)
\curveto(384.94255905,249.84527228)(390.05628854,244.73154279)(390.05628854,238.3795111)
\lineto(390.05628854,231.39312384)
\curveto(390.05628854,225.04109215)(384.94255905,219.92736266)(378.59052735,219.92736266)
\lineto(182.63554459,219.92736266)
\curveto(176.2835129,219.92736266)(171.16978341,225.04109215)(171.16978341,231.39312384)
\lineto(171.16978341,238.3795111)
\curveto(171.16978341,244.73154279)(176.2835129,249.84527228)(182.63554459,249.84527228)
\closepath
}
}
{
\newrgbcolor{curcolor}{0 0 0}
\pscustom[linestyle=none,fillstyle=solid,fillcolor=curcolor]
{
\newpath
\moveto(192.35546761,247.49609015)
\lineto(195.88281136,247.49609015)
\lineto(200.34765511,235.58984015)
\lineto(204.83593636,247.49609015)
\lineto(208.36328011,247.49609015)
\lineto(208.36328011,229.9999964)
\lineto(206.05468636,229.9999964)
\lineto(206.05468636,245.36327765)
\lineto(201.54296761,233.36327765)
\lineto(199.16406136,233.36327765)
\lineto(194.65234261,245.36327765)
\lineto(194.65234261,229.9999964)
\lineto(192.35546761,229.9999964)
\lineto(192.35546761,247.49609015)
}
}
{
\newrgbcolor{curcolor}{0 0 0}
\pscustom[linestyle=none,fillstyle=solid,fillcolor=curcolor]
{
\newpath
\moveto(224.20703011,237.1015589)
\lineto(224.20703011,236.0468714)
\lineto(214.29296761,236.0468714)
\curveto(214.38671395,234.56249184)(214.832026,233.42968047)(215.62890511,232.6484339)
\curveto(216.4335869,231.87499453)(217.55077328,231.48827616)(218.98046761,231.48827765)
\curveto(219.80858352,231.48827616)(220.60936397,231.58983856)(221.38281136,231.79296515)
\curveto(222.16404992,231.99608816)(222.93748664,232.30077535)(223.70312386,232.70702765)
\lineto(223.70312386,230.66796515)
\curveto(222.92967415,230.33983981)(222.1367062,230.08984006)(221.32421761,229.91796515)
\curveto(220.51170782,229.74609041)(219.68748989,229.66015299)(218.85156136,229.66015265)
\curveto(216.75780532,229.66015299)(215.09765073,230.26952738)(213.87109261,231.48827765)
\curveto(212.65234068,232.70702495)(212.04296629,234.3554608)(212.04296761,236.43359015)
\curveto(212.04296629,238.58201907)(212.62109071,240.28514237)(213.77734261,241.54296515)
\curveto(214.94140089,242.80857734)(216.50780557,243.44138921)(218.47656136,243.44140265)
\curveto(220.24217684,243.44138921)(221.6367067,242.87107728)(222.66015511,241.73046515)
\curveto(223.69139214,240.59764205)(224.20701662,239.05467485)(224.20703011,237.1015589)
\moveto(222.05078011,237.7343714)
\curveto(222.0351438,238.91404999)(221.70311288,239.8554553)(221.05468636,240.55859015)
\curveto(220.41405167,241.26170389)(219.56249002,241.61326604)(218.49999886,241.61327765)
\curveto(217.29686729,241.61326604)(216.3320245,241.27342263)(215.60546761,240.5937464)
\curveto(214.88671345,239.91404899)(214.47265136,238.9570187)(214.36328011,237.72265265)
\lineto(222.05078011,237.7343714)
}
}
{
\newrgbcolor{curcolor}{0 0 0}
\pscustom[linestyle=none,fillstyle=solid,fillcolor=curcolor]
{
\newpath
\moveto(236.11328011,242.73827765)
\lineto(236.11328011,240.69921515)
\curveto(235.50389509,241.01170414)(234.87108323,241.24607891)(234.21484261,241.40234015)
\curveto(233.55858454,241.55857859)(232.87889772,241.63670352)(232.17578011,241.63671515)
\curveto(231.10546199,241.63670352)(230.3007753,241.47264118)(229.76171761,241.14452765)
\curveto(229.23046387,240.81639184)(228.96483913,240.32420483)(228.96484261,239.66796515)
\curveto(228.96483913,239.16795598)(229.15624519,238.77342513)(229.53906136,238.4843714)
\curveto(229.92186943,238.2031132)(230.69139991,237.93358222)(231.84765511,237.67577765)
\lineto(232.58593636,237.51171515)
\curveto(234.11717773,237.18358297)(235.20311414,236.71873968)(235.84374886,236.1171839)
\curveto(236.49217536,235.52342838)(236.81639378,234.69139796)(236.81640511,233.62109015)
\curveto(236.81639378,232.40233775)(236.33201927,231.43749496)(235.36328011,230.7265589)
\curveto(234.4023337,230.01562139)(233.07811627,229.66015299)(231.39062386,229.66015265)
\curveto(230.68749366,229.66015299)(229.95311939,229.73046542)(229.18749886,229.87109015)
\curveto(228.42968342,230.00390265)(227.62890297,230.20702745)(226.78515511,230.48046515)
\lineto(226.78515511,232.70702765)
\curveto(227.58202802,232.29296286)(228.36718348,231.98046317)(229.14062386,231.76952765)
\curveto(229.91405693,231.56640109)(230.67968117,231.46483869)(231.43749886,231.46484015)
\curveto(232.45311689,231.46483869)(233.23436611,231.63671352)(233.78124886,231.98046515)
\curveto(234.32811502,232.33202532)(234.60155225,232.82421233)(234.60156136,233.45702765)
\curveto(234.60155225,234.04296111)(234.4023337,234.49217941)(234.00390511,234.8046839)
\curveto(233.61327198,235.11717879)(232.7499916,235.41795973)(231.41406136,235.70702765)
\lineto(230.66406136,235.8828089)
\curveto(229.32812002,236.16405274)(228.36327723,236.59373981)(227.76953011,237.1718714)
\curveto(227.17577842,237.75780114)(226.87890372,238.55858159)(226.87890511,239.57421515)
\curveto(226.87890372,240.80857934)(227.31640328,241.76170339)(228.19140511,242.43359015)
\curveto(229.06640153,243.10545205)(230.30858779,243.44138921)(231.91796761,243.44140265)
\curveto(232.71483538,243.44138921)(233.46483463,243.38279552)(234.16796761,243.2656214)
\curveto(234.87108323,243.14842075)(235.51952008,242.97263968)(236.11328011,242.73827765)
}
}
{
\newrgbcolor{curcolor}{0 0 0}
\pscustom[linestyle=none,fillstyle=solid,fillcolor=curcolor]
{
}
}
{
\newrgbcolor{curcolor}{0 0 0}
\pscustom[linestyle=none,fillstyle=solid,fillcolor=curcolor]
{
\newpath
\moveto(256.26953011,242.73827765)
\lineto(256.26953011,240.69921515)
\curveto(255.66014509,241.01170414)(255.02733323,241.24607891)(254.37109261,241.40234015)
\curveto(253.71483454,241.55857859)(253.03514772,241.63670352)(252.33203011,241.63671515)
\curveto(251.26171199,241.63670352)(250.4570253,241.47264118)(249.91796761,241.14452765)
\curveto(249.38671387,240.81639184)(249.12108913,240.32420483)(249.12109261,239.66796515)
\curveto(249.12108913,239.16795598)(249.31249519,238.77342513)(249.69531136,238.4843714)
\curveto(250.07811943,238.2031132)(250.84764991,237.93358222)(252.00390511,237.67577765)
\lineto(252.74218636,237.51171515)
\curveto(254.27342773,237.18358297)(255.35936414,236.71873968)(255.99999886,236.1171839)
\curveto(256.64842536,235.52342838)(256.97264378,234.69139796)(256.97265511,233.62109015)
\curveto(256.97264378,232.40233775)(256.48826927,231.43749496)(255.51953011,230.7265589)
\curveto(254.5585837,230.01562139)(253.23436627,229.66015299)(251.54687386,229.66015265)
\curveto(250.84374366,229.66015299)(250.10936939,229.73046542)(249.34374886,229.87109015)
\curveto(248.58593342,230.00390265)(247.78515297,230.20702745)(246.94140511,230.48046515)
\lineto(246.94140511,232.70702765)
\curveto(247.73827802,232.29296286)(248.52343348,231.98046317)(249.29687386,231.76952765)
\curveto(250.07030693,231.56640109)(250.83593117,231.46483869)(251.59374886,231.46484015)
\curveto(252.60936689,231.46483869)(253.39061611,231.63671352)(253.93749886,231.98046515)
\curveto(254.48436502,232.33202532)(254.75780225,232.82421233)(254.75781136,233.45702765)
\curveto(254.75780225,234.04296111)(254.5585837,234.49217941)(254.16015511,234.8046839)
\curveto(253.76952198,235.11717879)(252.9062416,235.41795973)(251.57031136,235.70702765)
\lineto(250.82031136,235.8828089)
\curveto(249.48437002,236.16405274)(248.51952723,236.59373981)(247.92578011,237.1718714)
\curveto(247.33202842,237.75780114)(247.03515372,238.55858159)(247.03515511,239.57421515)
\curveto(247.03515372,240.80857934)(247.47265328,241.76170339)(248.34765511,242.43359015)
\curveto(249.22265153,243.10545205)(250.46483779,243.44138921)(252.07421761,243.44140265)
\curveto(252.87108538,243.44138921)(253.62108463,243.38279552)(254.32421761,243.2656214)
\curveto(255.02733323,243.14842075)(255.67577008,242.97263968)(256.26953011,242.73827765)
}
}
{
\newrgbcolor{curcolor}{0 0 0}
\pscustom[linestyle=none,fillstyle=solid,fillcolor=curcolor]
{
\newpath
\moveto(262.55078011,246.8515589)
\lineto(262.55078011,243.1249964)
\lineto(266.99218636,243.1249964)
\lineto(266.99218636,241.44921515)
\lineto(262.55078011,241.44921515)
\lineto(262.55078011,234.32421515)
\curveto(262.55077572,233.2538994)(262.69530682,232.56640009)(262.98437386,232.26171515)
\curveto(263.28124374,231.9570257)(263.87889939,231.8046821)(264.77734261,231.8046839)
\lineto(266.99218636,231.8046839)
\lineto(266.99218636,229.9999964)
\lineto(264.77734261,229.9999964)
\curveto(263.11327516,229.9999964)(261.9648388,230.30858984)(261.33203011,230.92577765)
\curveto(260.69921507,231.5507761)(260.38280914,232.68358747)(260.38281136,234.32421515)
\lineto(260.38281136,241.44921515)
\lineto(258.80078011,241.44921515)
\lineto(258.80078011,243.1249964)
\lineto(260.38281136,243.1249964)
\lineto(260.38281136,246.8515589)
\lineto(262.55078011,246.8515589)
}
}
{
\newrgbcolor{curcolor}{0 0 0}
\pscustom[linestyle=none,fillstyle=solid,fillcolor=curcolor]
{
\newpath
\moveto(275.80468636,236.59765265)
\curveto(274.06249238,236.59764605)(272.85546234,236.3984275)(272.18359261,235.9999964)
\curveto(271.51171368,235.6015533)(271.17577652,234.92186648)(271.17578011,233.9609339)
\curveto(271.17577652,233.19530571)(271.42577627,232.58593132)(271.92578011,232.1328089)
\curveto(272.43358776,231.68749471)(273.12108707,231.46483869)(273.98828011,231.46484015)
\curveto(275.18358501,231.46483869)(276.1406153,231.88671327)(276.85937386,232.73046515)
\curveto(277.58592636,233.58202407)(277.94920724,234.71092919)(277.94921761,236.1171839)
\lineto(277.94921761,236.59765265)
\lineto(275.80468636,236.59765265)
\moveto(280.10546761,237.48827765)
\lineto(280.10546761,229.9999964)
\lineto(277.94921761,229.9999964)
\lineto(277.94921761,231.9921839)
\curveto(277.45702023,231.19530771)(276.8437396,230.60546455)(276.10937386,230.22265265)
\curveto(275.37499107,229.8476528)(274.47655446,229.66015299)(273.41406136,229.66015265)
\curveto(272.07030687,229.66015299)(270.99999544,230.03515262)(270.20312386,230.78515265)
\curveto(269.41405953,231.54296361)(269.01952867,232.55468135)(269.01953011,233.8203089)
\curveto(269.01952867,235.29686611)(269.51171568,236.41014624)(270.49609261,237.16015265)
\curveto(271.4882762,237.91014474)(272.96483723,238.28514437)(274.92578011,238.28515265)
\lineto(277.94921761,238.28515265)
\lineto(277.94921761,238.49609015)
\curveto(277.94920724,239.48826816)(277.62108257,240.2538924)(276.96484261,240.79296515)
\curveto(276.31639638,241.33982881)(275.40233479,241.61326604)(274.22265511,241.61327765)
\curveto(273.47264922,241.61326604)(272.7421812,241.52342238)(272.03124886,241.3437464)
\curveto(271.32030762,241.16404774)(270.63671455,240.89451676)(269.98046761,240.53515265)
\lineto(269.98046761,242.52734015)
\curveto(270.76952692,242.83201482)(271.53515116,243.05857709)(272.27734261,243.20702765)
\curveto(273.01952467,243.36326429)(273.7421802,243.44138921)(274.44531136,243.44140265)
\curveto(276.3437401,243.44138921)(277.76170743,242.9492022)(278.69921761,241.96484015)
\curveto(279.63670555,240.98045417)(280.10545509,239.48826816)(280.10546761,237.48827765)
}
}
{
\newrgbcolor{curcolor}{0 0 0}
\pscustom[linestyle=none,fillstyle=solid,fillcolor=curcolor]
{
\newpath
\moveto(286.69140511,246.8515589)
\lineto(286.69140511,243.1249964)
\lineto(291.13281136,243.1249964)
\lineto(291.13281136,241.44921515)
\lineto(286.69140511,241.44921515)
\lineto(286.69140511,234.32421515)
\curveto(286.69140072,233.2538994)(286.83593182,232.56640009)(287.12499886,232.26171515)
\curveto(287.42186874,231.9570257)(288.01952439,231.8046821)(288.91796761,231.8046839)
\lineto(291.13281136,231.8046839)
\lineto(291.13281136,229.9999964)
\lineto(288.91796761,229.9999964)
\curveto(287.25390016,229.9999964)(286.1054638,230.30858984)(285.47265511,230.92577765)
\curveto(284.83984007,231.5507761)(284.52343414,232.68358747)(284.52343636,234.32421515)
\lineto(284.52343636,241.44921515)
\lineto(282.94140511,241.44921515)
\lineto(282.94140511,243.1249964)
\lineto(284.52343636,243.1249964)
\lineto(284.52343636,246.8515589)
\lineto(286.69140511,246.8515589)
}
}
{
\newrgbcolor{curcolor}{0 0 0}
\pscustom[linestyle=none,fillstyle=solid,fillcolor=curcolor]
{
\newpath
\moveto(293.98046761,243.1249964)
\lineto(296.13671761,243.1249964)
\lineto(296.13671761,229.9999964)
\lineto(293.98046761,229.9999964)
\lineto(293.98046761,243.1249964)
\moveto(293.98046761,248.2343714)
\lineto(296.13671761,248.2343714)
\lineto(296.13671761,245.50390265)
\lineto(293.98046761,245.50390265)
\lineto(293.98046761,248.2343714)
}
}
{
\newrgbcolor{curcolor}{0 0 0}
\pscustom[linestyle=none,fillstyle=solid,fillcolor=curcolor]
{
\newpath
\moveto(309.00390511,242.73827765)
\lineto(309.00390511,240.69921515)
\curveto(308.39452009,241.01170414)(307.76170823,241.24607891)(307.10546761,241.40234015)
\curveto(306.44920954,241.55857859)(305.76952272,241.63670352)(305.06640511,241.63671515)
\curveto(303.99608699,241.63670352)(303.1914003,241.47264118)(302.65234261,241.14452765)
\curveto(302.12108887,240.81639184)(301.85546413,240.32420483)(301.85546761,239.66796515)
\curveto(301.85546413,239.16795598)(302.04687019,238.77342513)(302.42968636,238.4843714)
\curveto(302.81249443,238.2031132)(303.58202491,237.93358222)(304.73828011,237.67577765)
\lineto(305.47656136,237.51171515)
\curveto(307.00780273,237.18358297)(308.09373914,236.71873968)(308.73437386,236.1171839)
\curveto(309.38280036,235.52342838)(309.70701878,234.69139796)(309.70703011,233.62109015)
\curveto(309.70701878,232.40233775)(309.22264427,231.43749496)(308.25390511,230.7265589)
\curveto(307.2929587,230.01562139)(305.96874127,229.66015299)(304.28124886,229.66015265)
\curveto(303.57811866,229.66015299)(302.84374439,229.73046542)(302.07812386,229.87109015)
\curveto(301.32030842,230.00390265)(300.51952797,230.20702745)(299.67578011,230.48046515)
\lineto(299.67578011,232.70702765)
\curveto(300.47265302,232.29296286)(301.25780848,231.98046317)(302.03124886,231.76952765)
\curveto(302.80468193,231.56640109)(303.57030617,231.46483869)(304.32812386,231.46484015)
\curveto(305.34374189,231.46483869)(306.12499111,231.63671352)(306.67187386,231.98046515)
\curveto(307.21874002,232.33202532)(307.49217725,232.82421233)(307.49218636,233.45702765)
\curveto(307.49217725,234.04296111)(307.2929587,234.49217941)(306.89453011,234.8046839)
\curveto(306.50389698,235.11717879)(305.6406166,235.41795973)(304.30468636,235.70702765)
\lineto(303.55468636,235.8828089)
\curveto(302.21874502,236.16405274)(301.25390223,236.59373981)(300.66015511,237.1718714)
\curveto(300.06640342,237.75780114)(299.76952872,238.55858159)(299.76953011,239.57421515)
\curveto(299.76952872,240.80857934)(300.20702828,241.76170339)(301.08203011,242.43359015)
\curveto(301.95702653,243.10545205)(303.19921279,243.44138921)(304.80859261,243.44140265)
\curveto(305.60546038,243.44138921)(306.35545963,243.38279552)(307.05859261,243.2656214)
\curveto(307.76170823,243.14842075)(308.41014508,242.97263968)(309.00390511,242.73827765)
}
}
{
\newrgbcolor{curcolor}{0 0 0}
\pscustom[linestyle=none,fillstyle=solid,fillcolor=curcolor]
{
\newpath
\moveto(315.28515511,246.8515589)
\lineto(315.28515511,243.1249964)
\lineto(319.72656136,243.1249964)
\lineto(319.72656136,241.44921515)
\lineto(315.28515511,241.44921515)
\lineto(315.28515511,234.32421515)
\curveto(315.28515072,233.2538994)(315.42968182,232.56640009)(315.71874886,232.26171515)
\curveto(316.01561874,231.9570257)(316.61327439,231.8046821)(317.51171761,231.8046839)
\lineto(319.72656136,231.8046839)
\lineto(319.72656136,229.9999964)
\lineto(317.51171761,229.9999964)
\curveto(315.84765016,229.9999964)(314.6992138,230.30858984)(314.06640511,230.92577765)
\curveto(313.43359007,231.5507761)(313.11718414,232.68358747)(313.11718636,234.32421515)
\lineto(313.11718636,241.44921515)
\lineto(311.53515511,241.44921515)
\lineto(311.53515511,243.1249964)
\lineto(313.11718636,243.1249964)
\lineto(313.11718636,246.8515589)
\lineto(315.28515511,246.8515589)
}
}
{
\newrgbcolor{curcolor}{0 0 0}
\pscustom[linestyle=none,fillstyle=solid,fillcolor=curcolor]
{
\newpath
\moveto(322.57421761,243.1249964)
\lineto(324.73046761,243.1249964)
\lineto(324.73046761,229.9999964)
\lineto(322.57421761,229.9999964)
\lineto(322.57421761,243.1249964)
\moveto(322.57421761,248.2343714)
\lineto(324.73046761,248.2343714)
\lineto(324.73046761,245.50390265)
\lineto(322.57421761,245.50390265)
\lineto(322.57421761,248.2343714)
}
}
{
\newrgbcolor{curcolor}{0 0 0}
\pscustom[linestyle=none,fillstyle=solid,fillcolor=curcolor]
{
\newpath
\moveto(330.51953011,236.55077765)
\curveto(330.51952656,234.96483519)(330.84374499,233.71874268)(331.49218636,232.8124964)
\curveto(332.14843118,231.91405699)(333.04686779,231.46483869)(334.18749886,231.46484015)
\curveto(335.3281155,231.46483869)(336.22655211,231.91405699)(336.88281136,232.8124964)
\curveto(337.53905079,233.71874268)(337.86717546,234.96483519)(337.86718636,236.55077765)
\curveto(337.86717546,238.13670702)(337.53905079,239.37889327)(336.88281136,240.27734015)
\curveto(336.22655211,241.18357897)(335.3281155,241.63670352)(334.18749886,241.63671515)
\curveto(333.04686779,241.63670352)(332.14843118,241.18357897)(331.49218636,240.27734015)
\curveto(330.84374499,239.37889327)(330.51952656,238.13670702)(330.51953011,236.55077765)
\moveto(337.86718636,231.9687464)
\curveto(337.41405092,231.18749521)(336.83983274,230.60546455)(336.14453011,230.22265265)
\curveto(335.45702162,229.8476528)(334.62889745,229.66015299)(333.66015511,229.66015265)
\curveto(332.07421251,229.66015299)(330.78124505,230.29296486)(329.78124886,231.55859015)
\curveto(328.78905954,232.82421233)(328.29296629,234.48827316)(328.29296761,236.55077765)
\curveto(328.29296629,238.61326904)(328.78905954,240.27732988)(329.78124886,241.54296515)
\curveto(330.78124505,242.80857734)(332.07421251,243.44138921)(333.66015511,243.44140265)
\curveto(334.62889745,243.44138921)(335.45702162,243.24998315)(336.14453011,242.8671839)
\curveto(336.83983274,242.49217141)(337.41405092,241.91404699)(337.86718636,241.1328089)
\lineto(337.86718636,243.1249964)
\lineto(340.02343636,243.1249964)
\lineto(340.02343636,225.0078089)
\lineto(337.86718636,225.0078089)
\lineto(337.86718636,231.9687464)
}
}
{
\newrgbcolor{curcolor}{0 0 0}
\pscustom[linestyle=none,fillstyle=solid,fillcolor=curcolor]
{
\newpath
\moveto(344.24218636,235.1796839)
\lineto(344.24218636,243.1249964)
\lineto(346.39843636,243.1249964)
\lineto(346.39843636,235.26171515)
\curveto(346.39843217,234.01952363)(346.64061943,233.08593082)(347.12499886,232.4609339)
\curveto(347.60936846,231.84374456)(348.33593023,231.53515112)(349.30468636,231.53515265)
\curveto(350.4687406,231.53515112)(351.38670843,231.9062445)(352.05859261,232.6484339)
\curveto(352.73826958,233.39061801)(353.07811299,234.40233575)(353.07812386,235.68359015)
\lineto(353.07812386,243.1249964)
\lineto(355.23437386,243.1249964)
\lineto(355.23437386,229.9999964)
\lineto(353.07812386,229.9999964)
\lineto(353.07812386,232.0156214)
\curveto(352.55467601,231.21874518)(351.94530162,230.62499578)(351.24999886,230.2343714)
\curveto(350.5624905,229.85155905)(349.76171005,229.66015299)(348.84765511,229.66015265)
\curveto(347.33983748,229.66015299)(346.19530737,230.12890252)(345.41406136,231.06640265)
\curveto(344.63280893,232.00390065)(344.24218432,233.37499303)(344.24218636,235.1796839)
\moveto(349.66796761,243.44140265)
\lineto(349.66796761,243.44140265)
}
}
{
\newrgbcolor{curcolor}{0 0 0}
\pscustom[linestyle=none,fillstyle=solid,fillcolor=curcolor]
{
\newpath
\moveto(370.92578011,237.1015589)
\lineto(370.92578011,236.0468714)
\lineto(361.01171761,236.0468714)
\curveto(361.10546395,234.56249184)(361.550776,233.42968047)(362.34765511,232.6484339)
\curveto(363.1523369,231.87499453)(364.26952328,231.48827616)(365.69921761,231.48827765)
\curveto(366.52733352,231.48827616)(367.32811397,231.58983856)(368.10156136,231.79296515)
\curveto(368.88279992,231.99608816)(369.65623664,232.30077535)(370.42187386,232.70702765)
\lineto(370.42187386,230.66796515)
\curveto(369.64842415,230.33983981)(368.8554562,230.08984006)(368.04296761,229.91796515)
\curveto(367.23045782,229.74609041)(366.40623989,229.66015299)(365.57031136,229.66015265)
\curveto(363.47655532,229.66015299)(361.81640073,230.26952738)(360.58984261,231.48827765)
\curveto(359.37109068,232.70702495)(358.76171629,234.3554608)(358.76171761,236.43359015)
\curveto(358.76171629,238.58201907)(359.33984071,240.28514237)(360.49609261,241.54296515)
\curveto(361.66015089,242.80857734)(363.22655557,243.44138921)(365.19531136,243.44140265)
\curveto(366.96092684,243.44138921)(368.3554567,242.87107728)(369.37890511,241.73046515)
\curveto(370.41014214,240.59764205)(370.92576662,239.05467485)(370.92578011,237.1015589)
\moveto(368.76953011,237.7343714)
\curveto(368.7538938,238.91404999)(368.42186288,239.8554553)(367.77343636,240.55859015)
\curveto(367.13280167,241.26170389)(366.28124002,241.61326604)(365.21874886,241.61327765)
\curveto(364.01561729,241.61326604)(363.0507745,241.27342263)(362.32421761,240.5937464)
\curveto(361.60546345,239.91404899)(361.19140136,238.9570187)(361.08203011,237.72265265)
\lineto(368.76953011,237.7343714)
}
}
{
\newrgbcolor{curcolor}{0 0 0}
\pscustom[linestyle=none,fillstyle=solid,fillcolor=curcolor]
{
\newpath
\moveto(382.83203011,242.73827765)
\lineto(382.83203011,240.69921515)
\curveto(382.22264509,241.01170414)(381.58983323,241.24607891)(380.93359261,241.40234015)
\curveto(380.27733454,241.55857859)(379.59764772,241.63670352)(378.89453011,241.63671515)
\curveto(377.82421199,241.63670352)(377.0195253,241.47264118)(376.48046761,241.14452765)
\curveto(375.94921387,240.81639184)(375.68358913,240.32420483)(375.68359261,239.66796515)
\curveto(375.68358913,239.16795598)(375.87499519,238.77342513)(376.25781136,238.4843714)
\curveto(376.64061943,238.2031132)(377.41014991,237.93358222)(378.56640511,237.67577765)
\lineto(379.30468636,237.51171515)
\curveto(380.83592773,237.18358297)(381.92186414,236.71873968)(382.56249886,236.1171839)
\curveto(383.21092536,235.52342838)(383.53514378,234.69139796)(383.53515511,233.62109015)
\curveto(383.53514378,232.40233775)(383.05076927,231.43749496)(382.08203011,230.7265589)
\curveto(381.1210837,230.01562139)(379.79686627,229.66015299)(378.10937386,229.66015265)
\curveto(377.40624366,229.66015299)(376.67186939,229.73046542)(375.90624886,229.87109015)
\curveto(375.14843342,230.00390265)(374.34765297,230.20702745)(373.50390511,230.48046515)
\lineto(373.50390511,232.70702765)
\curveto(374.30077802,232.29296286)(375.08593348,231.98046317)(375.85937386,231.76952765)
\curveto(376.63280693,231.56640109)(377.39843117,231.46483869)(378.15624886,231.46484015)
\curveto(379.17186689,231.46483869)(379.95311611,231.63671352)(380.49999886,231.98046515)
\curveto(381.04686502,232.33202532)(381.32030225,232.82421233)(381.32031136,233.45702765)
\curveto(381.32030225,234.04296111)(381.1210837,234.49217941)(380.72265511,234.8046839)
\curveto(380.33202198,235.11717879)(379.4687416,235.41795973)(378.13281136,235.70702765)
\lineto(377.38281136,235.8828089)
\curveto(376.04687002,236.16405274)(375.08202723,236.59373981)(374.48828011,237.1718714)
\curveto(373.89452842,237.75780114)(373.59765372,238.55858159)(373.59765511,239.57421515)
\curveto(373.59765372,240.80857934)(374.03515328,241.76170339)(374.91015511,242.43359015)
\curveto(375.78515153,243.10545205)(377.02733779,243.44138921)(378.63671761,243.44140265)
\curveto(379.43358538,243.44138921)(380.18358463,243.38279552)(380.88671761,243.2656214)
\curveto(381.58983323,243.14842075)(382.23827008,242.97263968)(382.83203011,242.73827765)
}
}
{
\newrgbcolor{curcolor}{1 1 1}
\pscustom[linestyle=none,fillstyle=solid,fillcolor=curcolor]
{
\newpath
\moveto(203.99147492,199.9999964)
\lineto(357.5801323,199.9999964)
\curveto(364.46073838,199.9999964)(369.99999886,194.46073592)(369.99999886,187.58012984)
\lineto(369.99999886,182.59711669)
\curveto(369.99999886,175.71651061)(364.46073838,170.17725013)(357.5801323,170.17725013)
\lineto(203.99147492,170.17725013)
\curveto(197.11086885,170.17725013)(191.57160836,175.71651061)(191.57160836,182.59711669)
\lineto(191.57160836,187.58012984)
\curveto(191.57160836,194.46073592)(197.11086885,199.9999964)(203.99147492,199.9999964)
\closepath
}
}
{
\newrgbcolor{curcolor}{0 0 0}
\pscustom[linewidth=2,linecolor=curcolor]
{
\newpath
\moveto(203.99147492,199.9999964)
\lineto(357.5801323,199.9999964)
\curveto(364.46073838,199.9999964)(369.99999886,194.46073592)(369.99999886,187.58012984)
\lineto(369.99999886,182.59711669)
\curveto(369.99999886,175.71651061)(364.46073838,170.17725013)(357.5801323,170.17725013)
\lineto(203.99147492,170.17725013)
\curveto(197.11086885,170.17725013)(191.57160836,175.71651061)(191.57160836,182.59711669)
\lineto(191.57160836,187.58012984)
\curveto(191.57160836,194.46073592)(197.11086885,199.9999964)(203.99147492,199.9999964)
\closepath
}
}
{
\newrgbcolor{curcolor}{0 0 0}
\pscustom[linestyle=none,fillstyle=solid,fillcolor=curcolor]
{
\newpath
\moveto(225.76302987,196.24282477)
\lineto(225.76302987,193.74673102)
\curveto(224.96614021,194.48890413)(224.11457856,195.04359107)(223.20834237,195.41079352)
\curveto(222.30989287,195.77796534)(221.35286257,195.9615589)(220.33724862,195.96157477)
\curveto(218.33724059,195.9615589)(216.80599212,195.34827827)(215.74349862,194.12173102)
\curveto(214.68099424,192.90296821)(214.14974478,191.13734498)(214.14974862,188.82485602)
\curveto(214.14974478,186.5201621)(214.68099424,184.75453886)(215.74349862,183.52798102)
\curveto(216.80599212,182.30922881)(218.33724059,181.69985442)(220.33724862,181.69985602)
\curveto(221.35286257,181.69985442)(222.30989287,181.88344798)(223.20834237,182.25063727)
\curveto(224.11457856,182.61782225)(224.96614021,183.17250919)(225.76302987,183.91469977)
\lineto(225.76302987,181.44204352)
\curveto(224.93489024,180.87954274)(224.05598487,180.45766816)(223.12631112,180.17641852)
\curveto(222.20442422,179.89516872)(221.2278627,179.75454386)(220.19662362,179.75454352)
\curveto(217.54817888,179.75454386)(215.46224346,180.5631368)(213.93881112,182.18032477)
\curveto(212.41537151,183.80532106)(211.65365352,186.0201626)(211.65365487,188.82485602)
\curveto(211.65365352,191.63734448)(212.41537151,193.85218601)(213.93881112,195.46938727)
\curveto(215.46224346,197.09437027)(217.54817888,197.90686946)(220.19662362,197.90688727)
\curveto(221.24348768,197.90686946)(222.2278617,197.7662446)(223.14974862,197.48501227)
\curveto(224.07942235,197.21155765)(224.95051522,196.79749557)(225.76302987,196.24282477)
}
}
{
\newrgbcolor{curcolor}{0 0 0}
\pscustom[linestyle=none,fillstyle=solid,fillcolor=curcolor]
{
\newpath
\moveto(236.95443612,191.20376227)
\curveto(236.71223899,191.34437602)(236.44661426,191.44593842)(236.15756112,191.50844977)
\curveto(235.87630233,191.57875079)(235.56380264,191.613907)(235.22006112,191.61391852)
\curveto(234.00130421,191.613907)(233.06380514,191.2154699)(232.40756112,190.41860602)
\curveto(231.75911895,189.62953399)(231.43490052,188.49281637)(231.43490487,187.00844977)
\lineto(231.43490487,180.09438727)
\lineto(229.26693612,180.09438727)
\lineto(229.26693612,193.21938727)
\lineto(231.43490487,193.21938727)
\lineto(231.43490487,191.18032477)
\curveto(231.88802507,191.97718789)(232.47786823,192.56703105)(233.20443612,192.94985602)
\curveto(233.93099178,193.34046778)(234.81380339,193.53578008)(235.85287362,193.53579352)
\curveto(236.00130221,193.53578008)(236.16536454,193.52406134)(236.34506112,193.50063727)
\curveto(236.52473918,193.48499888)(236.72395773,193.45765516)(236.94271737,193.41860602)
\lineto(236.95443612,191.20376227)
}
}
{
\newrgbcolor{curcolor}{0 0 0}
\pscustom[linestyle=none,fillstyle=solid,fillcolor=curcolor]
{
\newpath
\moveto(249.95052987,187.19594977)
\lineto(249.95052987,186.14126227)
\lineto(240.03646737,186.14126227)
\curveto(240.1302137,184.65688271)(240.57552576,183.52407134)(241.37240487,182.74282477)
\curveto(242.17708665,181.9693854)(243.29427304,181.58266703)(244.72396737,181.58266852)
\curveto(245.55208328,181.58266703)(246.35286373,181.68422943)(247.12631112,181.88735602)
\curveto(247.90754967,182.09047903)(248.6809864,182.39516622)(249.44662362,182.80141852)
\lineto(249.44662362,180.76235602)
\curveto(248.67317391,180.43423068)(247.88020595,180.18423093)(247.06771737,180.01235602)
\curveto(246.25520758,179.84048128)(245.43098965,179.75454386)(244.59506112,179.75454352)
\curveto(242.50130508,179.75454386)(240.84115049,180.36391825)(239.61459237,181.58266852)
\curveto(238.39584044,182.80141581)(237.78646604,184.44985167)(237.78646737,186.52798102)
\curveto(237.78646604,188.67640994)(238.36459047,190.37953324)(239.52084237,191.63735602)
\curveto(240.68490065,192.90296821)(242.25130533,193.53578008)(244.22006112,193.53579352)
\curveto(245.9856766,193.53578008)(247.38020645,192.96546815)(248.40365487,191.82485602)
\curveto(249.4348919,190.69203292)(249.95051638,189.14906572)(249.95052987,187.19594977)
\moveto(247.79427987,187.82876227)
\curveto(247.77864355,189.00844086)(247.44661263,189.94984617)(246.79818612,190.65298102)
\curveto(246.15755142,191.35609476)(245.30598978,191.70765691)(244.24349862,191.70766852)
\curveto(243.04036704,191.70765691)(242.07552426,191.3678135)(241.34896737,190.68813727)
\curveto(240.6302132,190.00843986)(240.21615112,189.05140956)(240.10677987,187.81704352)
\lineto(247.79427987,187.82876227)
\moveto(245.72006112,199.28969977)
\lineto(248.05209237,199.28969977)
\lineto(244.23177987,194.88344977)
\lineto(242.43881112,194.88344977)
\lineto(245.72006112,199.28969977)
}
}
{
\newrgbcolor{curcolor}{0 0 0}
\pscustom[linestyle=none,fillstyle=solid,fillcolor=curcolor]
{
\newpath
\moveto(264.71615487,187.19594977)
\lineto(264.71615487,186.14126227)
\lineto(254.80209237,186.14126227)
\curveto(254.8958387,184.65688271)(255.34115076,183.52407134)(256.13802987,182.74282477)
\curveto(256.94271165,181.9693854)(258.05989804,181.58266703)(259.48959237,181.58266852)
\curveto(260.31770828,181.58266703)(261.11848873,181.68422943)(261.89193612,181.88735602)
\curveto(262.67317467,182.09047903)(263.4466114,182.39516622)(264.21224862,182.80141852)
\lineto(264.21224862,180.76235602)
\curveto(263.43879891,180.43423068)(262.64583095,180.18423093)(261.83334237,180.01235602)
\curveto(261.02083258,179.84048128)(260.19661465,179.75454386)(259.36068612,179.75454352)
\curveto(257.26693008,179.75454386)(255.60677549,180.36391825)(254.38021737,181.58266852)
\curveto(253.16146544,182.80141581)(252.55209104,184.44985167)(252.55209237,186.52798102)
\curveto(252.55209104,188.67640994)(253.13021547,190.37953324)(254.28646737,191.63735602)
\curveto(255.45052565,192.90296821)(257.01693033,193.53578008)(258.98568612,193.53579352)
\curveto(260.7513016,193.53578008)(262.14583145,192.96546815)(263.16927987,191.82485602)
\curveto(264.2005169,190.69203292)(264.71614138,189.14906572)(264.71615487,187.19594977)
\moveto(262.55990487,187.82876227)
\curveto(262.54426855,189.00844086)(262.21223763,189.94984617)(261.56381112,190.65298102)
\curveto(260.92317642,191.35609476)(260.07161478,191.70765691)(259.00912362,191.70766852)
\curveto(257.80599204,191.70765691)(256.84114926,191.3678135)(256.11459237,190.68813727)
\curveto(255.3958382,190.00843986)(254.98177612,189.05140956)(254.87240487,187.81704352)
\lineto(262.55990487,187.82876227)
}
}
{
\newrgbcolor{curcolor}{0 0 0}
\pscustom[linestyle=none,fillstyle=solid,fillcolor=curcolor]
{
\newpath
\moveto(275.86068612,191.20376227)
\curveto(275.61848899,191.34437602)(275.35286426,191.44593842)(275.06381112,191.50844977)
\curveto(274.78255233,191.57875079)(274.47005264,191.613907)(274.12631112,191.61391852)
\curveto(272.90755421,191.613907)(271.97005514,191.2154699)(271.31381112,190.41860602)
\curveto(270.66536895,189.62953399)(270.34115052,188.49281637)(270.34115487,187.00844977)
\lineto(270.34115487,180.09438727)
\lineto(268.17318612,180.09438727)
\lineto(268.17318612,193.21938727)
\lineto(270.34115487,193.21938727)
\lineto(270.34115487,191.18032477)
\curveto(270.79427507,191.97718789)(271.38411823,192.56703105)(272.11068612,192.94985602)
\curveto(272.83724178,193.34046778)(273.72005339,193.53578008)(274.75912362,193.53579352)
\curveto(274.90755221,193.53578008)(275.07161454,193.52406134)(275.25131112,193.50063727)
\curveto(275.43098918,193.48499888)(275.63020773,193.45765516)(275.84896737,193.41860602)
\lineto(275.86068612,191.20376227)
}
}
{
\newrgbcolor{curcolor}{0 0 0}
\pscustom[linestyle=none,fillstyle=solid,fillcolor=curcolor]
{
}
}
{
\newrgbcolor{curcolor}{0 0 0}
\pscustom[linestyle=none,fillstyle=solid,fillcolor=curcolor]
{
\newpath
\moveto(296.69662362,188.01626227)
\lineto(296.69662362,180.09438727)
\lineto(294.54037362,180.09438727)
\lineto(294.54037362,187.94594977)
\curveto(294.5403626,189.18812818)(294.29817535,190.11781475)(293.81381112,190.73501227)
\curveto(293.32942631,191.35218851)(292.60286454,191.66078196)(291.63412362,191.66079352)
\curveto(290.47005417,191.66078196)(289.55208634,191.28968858)(288.88021737,190.54751227)
\curveto(288.20833769,189.80531506)(287.87240052,188.79359732)(287.87240487,187.51235602)
\lineto(287.87240487,180.09438727)
\lineto(285.70443612,180.09438727)
\lineto(285.70443612,193.21938727)
\lineto(287.87240487,193.21938727)
\lineto(287.87240487,191.18032477)
\curveto(288.38802501,191.9693754)(288.99349315,192.55921856)(289.68881112,192.94985602)
\curveto(290.39192925,193.34046778)(291.20052219,193.53578008)(292.11459237,193.53579352)
\curveto(293.62239477,193.53578008)(294.76301863,193.06703055)(295.53646737,192.12954352)
\curveto(296.30989208,191.19984492)(296.69661045,189.82875254)(296.69662362,188.01626227)
}
}
{
\newrgbcolor{curcolor}{0 0 0}
\pscustom[linestyle=none,fillstyle=solid,fillcolor=curcolor]
{
\newpath
\moveto(301.02084237,193.21938727)
\lineto(303.17709237,193.21938727)
\lineto(303.17709237,180.09438727)
\lineto(301.02084237,180.09438727)
\lineto(301.02084237,193.21938727)
\moveto(301.02084237,198.32876227)
\lineto(303.17709237,198.32876227)
\lineto(303.17709237,195.59829352)
\lineto(301.02084237,195.59829352)
\lineto(301.02084237,198.32876227)
}
}
{
\newrgbcolor{curcolor}{0 0 0}
\pscustom[linestyle=none,fillstyle=solid,fillcolor=curcolor]
{
\newpath
\moveto(306.13021737,193.21938727)
\lineto(308.41537362,193.21938727)
\lineto(312.51693612,182.20376227)
\lineto(316.61849862,193.21938727)
\lineto(318.90365487,193.21938727)
\lineto(313.98177987,180.09438727)
\lineto(311.05209237,180.09438727)
\lineto(306.13021737,193.21938727)
}
}
{
\newrgbcolor{curcolor}{0 0 0}
\pscustom[linestyle=none,fillstyle=solid,fillcolor=curcolor]
{
\newpath
\moveto(333.10677987,187.19594977)
\lineto(333.10677987,186.14126227)
\lineto(323.19271737,186.14126227)
\curveto(323.2864637,184.65688271)(323.73177576,183.52407134)(324.52865487,182.74282477)
\curveto(325.33333665,181.9693854)(326.45052304,181.58266703)(327.88021737,181.58266852)
\curveto(328.70833328,181.58266703)(329.50911373,181.68422943)(330.28256112,181.88735602)
\curveto(331.06379967,182.09047903)(331.8372364,182.39516622)(332.60287362,182.80141852)
\lineto(332.60287362,180.76235602)
\curveto(331.82942391,180.43423068)(331.03645595,180.18423093)(330.22396737,180.01235602)
\curveto(329.41145758,179.84048128)(328.58723965,179.75454386)(327.75131112,179.75454352)
\curveto(325.65755508,179.75454386)(323.99740049,180.36391825)(322.77084237,181.58266852)
\curveto(321.55209044,182.80141581)(320.94271604,184.44985167)(320.94271737,186.52798102)
\curveto(320.94271604,188.67640994)(321.52084047,190.37953324)(322.67709237,191.63735602)
\curveto(323.84115065,192.90296821)(325.40755533,193.53578008)(327.37631112,193.53579352)
\curveto(329.1419266,193.53578008)(330.53645645,192.96546815)(331.55990487,191.82485602)
\curveto(332.5911419,190.69203292)(333.10676638,189.14906572)(333.10677987,187.19594977)
\moveto(330.95052987,187.82876227)
\curveto(330.93489355,189.00844086)(330.60286263,189.94984617)(329.95443612,190.65298102)
\curveto(329.31380142,191.35609476)(328.46223978,191.70765691)(327.39974862,191.70766852)
\curveto(326.19661704,191.70765691)(325.23177426,191.3678135)(324.50521737,190.68813727)
\curveto(323.7864632,190.00843986)(323.37240112,189.05140956)(323.26302987,187.81704352)
\lineto(330.95052987,187.82876227)
}
}
{
\newrgbcolor{curcolor}{0 0 0}
\pscustom[linestyle=none,fillstyle=solid,fillcolor=curcolor]
{
\newpath
\moveto(342.61068612,186.69204352)
\curveto(340.86849213,186.69203692)(339.66146209,186.49281837)(338.98959237,186.09438727)
\curveto(338.31771344,185.69594417)(337.98177627,185.01625735)(337.98177987,184.05532477)
\curveto(337.98177627,183.28969658)(338.23177602,182.68032219)(338.73177987,182.22719977)
\curveto(339.23958751,181.78188558)(339.92708683,181.55922956)(340.79427987,181.55923102)
\curveto(341.98958476,181.55922956)(342.94661506,181.98110413)(343.66537362,182.82485602)
\curveto(344.39192611,183.67641494)(344.755207,184.80532006)(344.75521737,186.21157477)
\lineto(344.75521737,186.69204352)
\lineto(342.61068612,186.69204352)
\moveto(346.91146737,187.58266852)
\lineto(346.91146737,180.09438727)
\lineto(344.75521737,180.09438727)
\lineto(344.75521737,182.08657477)
\curveto(344.26301999,181.28969858)(343.64973935,180.69985542)(342.91537362,180.31704352)
\curveto(342.18099082,179.94204367)(341.28255422,179.75454386)(340.22006112,179.75454352)
\curveto(338.87630663,179.75454386)(337.8059952,180.12954349)(337.00912362,180.87954352)
\curveto(336.22005928,181.63735448)(335.82552843,182.64907222)(335.82552987,183.91469977)
\curveto(335.82552843,185.39125697)(336.31771544,186.50453711)(337.30209237,187.25454352)
\curveto(338.29427596,188.00453561)(339.77083698,188.37953524)(341.73177987,188.37954352)
\lineto(344.75521737,188.37954352)
\lineto(344.75521737,188.59048102)
\curveto(344.755207,189.58265903)(344.42708233,190.34828327)(343.77084237,190.88735602)
\curveto(343.12239613,191.43421968)(342.20833454,191.70765691)(341.02865487,191.70766852)
\curveto(340.27864897,191.70765691)(339.54818096,191.61781325)(338.83724862,191.43813727)
\curveto(338.12630738,191.25843861)(337.44271431,190.98890763)(336.78646737,190.62954352)
\lineto(336.78646737,192.62173102)
\curveto(337.57552668,192.92640569)(338.34115091,193.15296796)(339.08334237,193.30141852)
\curveto(339.82552443,193.45765516)(340.54817996,193.53578008)(341.25131112,193.53579352)
\curveto(343.14973985,193.53578008)(344.56770719,193.04359307)(345.50521737,192.05923102)
\curveto(346.44270531,191.07484504)(346.91145484,189.58265903)(346.91146737,187.58266852)
}
}
{
\newrgbcolor{curcolor}{0 0 0}
\pscustom[linestyle=none,fillstyle=solid,fillcolor=curcolor]
{
\newpath
\moveto(351.14193612,185.27407477)
\lineto(351.14193612,193.21938727)
\lineto(353.29818612,193.21938727)
\lineto(353.29818612,185.35610602)
\curveto(353.29818192,184.1139145)(353.54036918,183.18032169)(354.02474862,182.55532477)
\curveto(354.50911821,181.93813543)(355.23567999,181.62954199)(356.20443612,181.62954352)
\curveto(357.36849035,181.62954199)(358.28645819,182.00063537)(358.95834237,182.74282477)
\curveto(359.63801933,183.48500888)(359.97786274,184.49672662)(359.97787362,185.77798102)
\lineto(359.97787362,193.21938727)
\lineto(362.13412362,193.21938727)
\lineto(362.13412362,180.09438727)
\lineto(359.97787362,180.09438727)
\lineto(359.97787362,182.11001227)
\curveto(359.45442577,181.31313605)(358.84505138,180.71938665)(358.14974862,180.32876227)
\curveto(357.46224026,179.94594992)(356.66145981,179.75454386)(355.74740487,179.75454352)
\curveto(354.23958723,179.75454386)(353.09505713,180.22329339)(352.31381112,181.16079352)
\curveto(351.53255869,182.09829152)(351.14193408,183.4693839)(351.14193612,185.27407477)
\moveto(356.56771737,193.53579352)
\lineto(356.56771737,193.53579352)
}
}
{
\newrgbcolor{curcolor}{1 1 1}
\pscustom[linestyle=none,fillstyle=solid,fillcolor=curcolor]
{
\newpath
\moveto(254.69320374,150.5329859)
\lineto(315.21460801,150.5329859)
\curveto(323.33513314,150.5329859)(329.8725956,143.99552344)(329.8725956,135.87499831)
\lineto(329.8725956,134.83607695)
\curveto(329.8725956,126.71555183)(323.33513314,120.17808936)(315.21460801,120.17808936)
\lineto(254.69320374,120.17808936)
\curveto(246.57267862,120.17808936)(240.03521615,126.71555183)(240.03521615,134.83607695)
\lineto(240.03521615,135.87499831)
\curveto(240.03521615,143.99552344)(246.57267862,150.5329859)(254.69320374,150.5329859)
\closepath
}
}
{
\newrgbcolor{curcolor}{0 0 0}
\pscustom[linewidth=2,linecolor=curcolor]
{
\newpath
\moveto(254.69320374,150.5329859)
\lineto(315.21460801,150.5329859)
\curveto(323.33513314,150.5329859)(329.8725956,143.99552344)(329.8725956,135.87499831)
\lineto(329.8725956,134.83607695)
\curveto(329.8725956,126.71555183)(323.33513314,120.17808936)(315.21460801,120.17808936)
\lineto(254.69320374,120.17808936)
\curveto(246.57267862,120.17808936)(240.03521615,126.71555183)(240.03521615,134.83607695)
\lineto(240.03521615,135.87499831)
\curveto(240.03521615,143.99552344)(246.57267862,150.5329859)(254.69320374,150.5329859)
\closepath
}
}
{
\newrgbcolor{curcolor}{0 0 0}
\pscustom[linestyle=none,fillstyle=solid,fillcolor=curcolor]
{
\newpath
\moveto(268.20312386,145.1640589)
\lineto(264.99218636,136.45702765)
\lineto(271.42578011,136.45702765)
\lineto(268.20312386,145.1640589)
\moveto(266.86718636,147.49609015)
\lineto(269.55078011,147.49609015)
\lineto(276.21874886,129.9999964)
\lineto(273.75781136,129.9999964)
\lineto(272.16406136,134.48827765)
\lineto(264.27734261,134.48827765)
\lineto(262.68359261,129.9999964)
\lineto(260.18749886,129.9999964)
\lineto(266.86718636,147.49609015)
}
}
{
\newrgbcolor{curcolor}{0 0 0}
\pscustom[linestyle=none,fillstyle=solid,fillcolor=curcolor]
{
\newpath
\moveto(278.66796761,143.1249964)
\lineto(280.82421761,143.1249964)
\lineto(280.82421761,129.9999964)
\lineto(278.66796761,129.9999964)
\lineto(278.66796761,143.1249964)
\moveto(278.66796761,148.2343714)
\lineto(280.82421761,148.2343714)
\lineto(280.82421761,145.50390265)
\lineto(278.66796761,145.50390265)
\lineto(278.66796761,148.2343714)
}
}
{
\newrgbcolor{curcolor}{0 0 0}
\pscustom[linestyle=none,fillstyle=solid,fillcolor=curcolor]
{
\newpath
\moveto(293.96093636,141.1328089)
\lineto(293.96093636,148.2343714)
\lineto(296.11718636,148.2343714)
\lineto(296.11718636,129.9999964)
\lineto(293.96093636,129.9999964)
\lineto(293.96093636,131.9687464)
\curveto(293.50780092,131.18749521)(292.93358274,130.60546455)(292.23828011,130.22265265)
\curveto(291.55077162,129.8476528)(290.72264745,129.66015299)(289.75390511,129.66015265)
\curveto(288.16796251,129.66015299)(286.87499505,130.29296486)(285.87499886,131.55859015)
\curveto(284.88280954,132.82421233)(284.38671629,134.48827316)(284.38671761,136.55077765)
\curveto(284.38671629,138.61326904)(284.88280954,140.27732988)(285.87499886,141.54296515)
\curveto(286.87499505,142.80857734)(288.16796251,143.44138921)(289.75390511,143.44140265)
\curveto(290.72264745,143.44138921)(291.55077162,143.24998315)(292.23828011,142.8671839)
\curveto(292.93358274,142.49217141)(293.50780092,141.91404699)(293.96093636,141.1328089)
\moveto(286.61328011,136.55077765)
\curveto(286.61327656,134.96483519)(286.93749499,133.71874268)(287.58593636,132.8124964)
\curveto(288.24218118,131.91405699)(289.14061779,131.46483869)(290.28124886,131.46484015)
\curveto(291.4218655,131.46483869)(292.32030211,131.91405699)(292.97656136,132.8124964)
\curveto(293.63280079,133.71874268)(293.96092546,134.96483519)(293.96093636,136.55077765)
\curveto(293.96092546,138.13670702)(293.63280079,139.37889327)(292.97656136,140.27734015)
\curveto(292.32030211,141.18357897)(291.4218655,141.63670352)(290.28124886,141.63671515)
\curveto(289.14061779,141.63670352)(288.24218118,141.18357897)(287.58593636,140.27734015)
\curveto(286.93749499,139.37889327)(286.61327656,138.13670702)(286.61328011,136.55077765)
}
}
{
\newrgbcolor{curcolor}{0 0 0}
\pscustom[linestyle=none,fillstyle=solid,fillcolor=curcolor]
{
\newpath
\moveto(311.78515511,137.1015589)
\lineto(311.78515511,136.0468714)
\lineto(301.87109261,136.0468714)
\curveto(301.96483895,134.56249184)(302.410151,133.42968047)(303.20703011,132.6484339)
\curveto(304.0117119,131.87499453)(305.12889828,131.48827616)(306.55859261,131.48827765)
\curveto(307.38670852,131.48827616)(308.18748897,131.58983856)(308.96093636,131.79296515)
\curveto(309.74217492,131.99608816)(310.51561164,132.30077535)(311.28124886,132.70702765)
\lineto(311.28124886,130.66796515)
\curveto(310.50779915,130.33983981)(309.7148312,130.08984006)(308.90234261,129.91796515)
\curveto(308.08983282,129.74609041)(307.26561489,129.66015299)(306.42968636,129.66015265)
\curveto(304.33593032,129.66015299)(302.67577573,130.26952738)(301.44921761,131.48827765)
\curveto(300.23046568,132.70702495)(299.62109129,134.3554608)(299.62109261,136.43359015)
\curveto(299.62109129,138.58201907)(300.19921571,140.28514237)(301.35546761,141.54296515)
\curveto(302.51952589,142.80857734)(304.08593057,143.44138921)(306.05468636,143.44140265)
\curveto(307.82030184,143.44138921)(309.2148317,142.87107728)(310.23828011,141.73046515)
\curveto(311.26951714,140.59764205)(311.78514162,139.05467485)(311.78515511,137.1015589)
\moveto(309.62890511,137.7343714)
\curveto(309.6132688,138.91404999)(309.28123788,139.8554553)(308.63281136,140.55859015)
\curveto(307.99217667,141.26170389)(307.14061502,141.61326604)(306.07812386,141.61327765)
\curveto(304.87499229,141.61326604)(303.9101495,141.27342263)(303.18359261,140.5937464)
\curveto(302.46483845,139.91404899)(302.05077636,138.9570187)(301.94140511,137.72265265)
\lineto(309.62890511,137.7343714)
}
}
{
\newrgbcolor{curcolor}{0 0 0}
\pscustom[linestyle=none,fillstyle=solid,fillcolor=curcolor,opacity=0.11935484]
{
\newpath
\moveto(178.69516545,59.9999964)
\lineto(377.49469739,59.9999964)
\curveto(382.31181968,59.9999964)(386.18986398,56.1219521)(386.18986398,51.30482981)
\lineto(386.18986398,38.88316462)
\curveto(386.18986398,34.06604233)(382.31181968,30.18799804)(377.49469739,30.18799804)
\lineto(178.69516545,30.18799804)
\curveto(173.87804316,30.18799804)(169.99999886,34.06604233)(169.99999886,38.88316462)
\lineto(169.99999886,51.30482981)
\curveto(169.99999886,56.1219521)(173.87804316,59.9999964)(178.69516545,59.9999964)
\closepath
}
}
{
\newrgbcolor{curcolor}{0 0 0}
\pscustom[linewidth=2,linecolor=curcolor]
{
\newpath
\moveto(178.69516545,59.9999964)
\lineto(377.49469739,59.9999964)
\curveto(382.31181968,59.9999964)(386.18986398,56.1219521)(386.18986398,51.30482981)
\lineto(386.18986398,38.88316462)
\curveto(386.18986398,34.06604233)(382.31181968,30.18799804)(377.49469739,30.18799804)
\lineto(178.69516545,30.18799804)
\curveto(173.87804316,30.18799804)(169.99999886,34.06604233)(169.99999886,38.88316462)
\lineto(169.99999886,51.30482981)
\curveto(169.99999886,56.1219521)(173.87804316,59.9999964)(178.69516545,59.9999964)
\closepath
}
}
{
\newrgbcolor{curcolor}{0 0 0}
\pscustom[linestyle=none,fillstyle=solid,fillcolor=curcolor]
{
\newpath
\moveto(43.96874886,512.6562464)
\lineto(49.12499886,512.6562464)
\lineto(49.12499886,530.4531214)
\lineto(43.51562386,529.3281214)
\lineto(43.51562386,532.2031214)
\lineto(49.09374886,533.3281214)
\lineto(52.24999886,533.3281214)
\lineto(52.24999886,512.6562464)
\lineto(57.40624886,512.6562464)
\lineto(57.40624886,509.9999964)
\lineto(43.96874886,509.9999964)
\lineto(43.96874886,512.6562464)
}
}
{
\newrgbcolor{curcolor}{0 0 0}
\pscustom[linestyle=none,fillstyle=solid,fillcolor=curcolor]
{
\newpath
\moveto(46.14062386,382.6562464)
\lineto(57.15624886,382.6562464)
\lineto(57.15624886,379.9999964)
\lineto(42.34374886,379.9999964)
\lineto(42.34374886,382.6562464)
\curveto(43.54166199,383.89582584)(45.17186869,385.55728251)(47.23437386,387.6406214)
\curveto(49.30728122,389.73436167)(50.60936325,391.08331865)(51.14062386,391.6874964)
\curveto(52.15102838,392.82290025)(52.85415268,393.78123262)(53.24999886,394.5624964)
\curveto(53.65623521,395.35414771)(53.85936,396.13018861)(53.85937386,396.8906214)
\curveto(53.85936,398.13018661)(53.42186044,399.14060226)(52.54687386,399.9218714)
\curveto(51.68227885,400.7031007)(50.55207164,401.09372531)(49.15624886,401.0937464)
\curveto(48.16665736,401.09372531)(47.11978341,400.92185048)(46.01562386,400.5781214)
\curveto(44.92186894,400.23435117)(43.74999511,399.71351836)(42.49999886,399.0156214)
\lineto(42.49999886,402.2031214)
\curveto(43.77082843,402.71351536)(44.95832724,403.09893164)(46.06249886,403.3593714)
\curveto(47.16665836,403.61976445)(48.17707402,403.74997265)(49.09374886,403.7499964)
\curveto(51.51040402,403.74997265)(53.43748543,403.14580659)(54.87499886,401.9374964)
\curveto(56.31248255,400.72914234)(57.03123183,399.11456062)(57.03124886,397.0937464)
\curveto(57.03123183,396.13539693)(56.84894035,395.22393951)(56.48437386,394.3593714)
\curveto(56.13019107,393.50519123)(55.47915005,392.49477557)(54.53124886,391.3281214)
\curveto(54.27081793,391.02602704)(53.44269375,390.15102792)(52.04687386,388.7031214)
\curveto(50.65102988,387.26561414)(48.68228185,385.24999115)(46.14062386,382.6562464)
}
}
{
\newrgbcolor{curcolor}{0 0 0}
\pscustom[linestyle=none,fillstyle=solid,fillcolor=curcolor]
{
\newpath
\moveto(512.98437386,342.5781214)
\curveto(514.49477604,342.25519248)(515.67185819,341.58331815)(516.51562386,340.5624964)
\curveto(517.36977316,339.54165353)(517.79685607,338.28123812)(517.79687386,336.7812464)
\curveto(517.79685607,334.47915859)(517.00519019,332.69791037)(515.42187386,331.4374964)
\curveto(513.83852669,330.17707956)(511.58852894,329.54687186)(508.67187386,329.5468714)
\curveto(507.6926995,329.54687186)(506.68228385,329.64583009)(505.64062386,329.8437464)
\curveto(504.60936925,330.03124637)(503.54166199,330.31770442)(502.43749886,330.7031214)
\lineto(502.43749886,333.7499964)
\curveto(503.31249555,333.2395765)(504.27082793,332.85416021)(505.31249886,332.5937464)
\curveto(506.35415918,332.3333274)(507.44269975,332.2031192)(508.57812386,332.2031214)
\curveto(510.55727997,332.2031192)(512.0624868,332.59374381)(513.09374886,333.3749964)
\curveto(514.13540139,334.15624225)(514.65623421,335.29165778)(514.65624886,336.7812464)
\curveto(514.65623421,338.15623825)(514.17185969,339.22915384)(513.20312386,339.9999964)
\curveto(512.24477829,340.78123562)(510.90623796,341.17186023)(509.18749886,341.1718714)
\lineto(506.46874886,341.1718714)
\lineto(506.46874886,343.7656214)
\lineto(509.31249886,343.7656214)
\curveto(510.86457133,343.76560764)(512.05207014,344.072899)(512.87499886,344.6874964)
\curveto(513.69790183,345.31248109)(514.10935975,346.20831353)(514.10937386,347.3749964)
\curveto(514.10935975,348.5728945)(513.68227685,349.48956025)(512.82812386,350.1249964)
\curveto(511.98436188,350.77080896)(510.77082143,351.09372531)(509.18749886,351.0937464)
\curveto(508.32290721,351.09372531)(507.3958248,350.9999754)(506.40624886,350.8124964)
\curveto(505.41666011,350.62497578)(504.32811954,350.3333094)(503.14062386,349.9374964)
\lineto(503.14062386,352.7499964)
\curveto(504.33853619,353.08330665)(505.45832674,353.3333064)(506.49999886,353.4999964)
\curveto(507.55207464,353.6666394)(508.54165699,353.74997265)(509.46874886,353.7499964)
\curveto(511.86457033,353.74997265)(513.76040177,353.2030982)(515.15624886,352.1093714)
\curveto(516.55206564,351.02601704)(517.24998161,349.55726851)(517.24999886,347.7031214)
\curveto(517.24998161,346.41143832)(516.88019032,345.31768942)(516.14062386,344.4218714)
\curveto(515.40102513,343.5364412)(514.34894285,342.92185848)(512.98437386,342.5781214)
}
}
{
\newrgbcolor{curcolor}{0 0 0}
\pscustom[linestyle=none,fillstyle=solid,fillcolor=curcolor]
{
\newpath
\moveto(512.09374886,300.5781214)
\lineto(504.12499886,288.1249964)
\lineto(512.09374886,288.1249964)
\lineto(512.09374886,300.5781214)
\moveto(511.26562386,303.3281214)
\lineto(515.23437386,303.3281214)
\lineto(515.23437386,288.1249964)
\lineto(518.56249886,288.1249964)
\lineto(518.56249886,285.4999964)
\lineto(515.23437386,285.4999964)
\lineto(515.23437386,279.9999964)
\lineto(512.09374886,279.9999964)
\lineto(512.09374886,285.4999964)
\lineto(501.56249886,285.4999964)
\lineto(501.56249886,288.5468714)
\lineto(511.26562386,303.3281214)
}
}
{
\newrgbcolor{curcolor}{0 0 0}
\pscustom[linestyle=none,fillstyle=solid,fillcolor=curcolor]
{
\newpath
\moveto(503.45312386,253.3281214)
\lineto(515.84374886,253.3281214)
\lineto(515.84374886,250.6718714)
\lineto(506.34374886,250.6718714)
\lineto(506.34374886,244.9531214)
\curveto(506.80207539,245.10935629)(507.26040827,245.22393951)(507.71874886,245.2968714)
\curveto(508.17707402,245.38018936)(508.63540689,245.42185598)(509.09374886,245.4218714)
\curveto(511.69790383,245.42185598)(513.76040177,244.70831503)(515.28124886,243.2812464)
\curveto(516.80206539,241.85415121)(517.5624813,239.92186148)(517.56249886,237.4843714)
\curveto(517.5624813,234.97394976)(516.78123208,233.02082671)(515.21874886,231.6249964)
\curveto(513.65623521,230.2395795)(511.45311241,229.54687186)(508.60937386,229.5468714)
\curveto(507.63019957,229.54687186)(506.63020057,229.63020511)(505.60937386,229.7968714)
\curveto(504.5989526,229.96353811)(503.55207864,230.21353786)(502.46874886,230.5468714)
\lineto(502.46874886,233.7187464)
\curveto(503.40624546,233.20832653)(504.37499449,232.82811857)(505.37499886,232.5781214)
\curveto(506.37499249,232.32811907)(507.4322831,232.2031192)(508.54687386,232.2031214)
\curveto(510.34894685,232.2031192)(511.77602875,232.67707706)(512.82812386,233.6249964)
\curveto(513.88019332,234.5729085)(514.40623446,235.85936554)(514.40624886,237.4843714)
\curveto(514.40623446,239.10936229)(513.88019332,240.39581934)(512.82812386,241.3437464)
\curveto(511.77602875,242.29165078)(510.34894685,242.76560864)(508.54687386,242.7656214)
\curveto(507.70311616,242.76560864)(506.859367,242.67185873)(506.01562386,242.4843714)
\curveto(505.18228535,242.29685911)(504.32811954,242.00519273)(503.45312386,241.6093714)
\lineto(503.45312386,253.3281214)
}
}
{
\newrgbcolor{curcolor}{0 0 0}
\pscustom[linestyle=none,fillstyle=solid,fillcolor=curcolor]
{
\newpath
\moveto(510.56249886,192.9218714)
\curveto(509.14582305,192.92185848)(508.02082418,192.43748396)(507.18749886,191.4687464)
\curveto(506.36457583,190.4999859)(505.95311791,189.17186223)(505.95312386,187.4843714)
\curveto(505.95311791,185.80728226)(506.36457583,184.47915859)(507.18749886,183.4999964)
\curveto(508.02082418,182.53124387)(509.14582305,182.04686936)(510.56249886,182.0468714)
\curveto(511.97915355,182.04686936)(513.0989441,182.53124387)(513.92187386,183.4999964)
\curveto(514.75519244,184.47915859)(515.17185869,185.80728226)(515.17187386,187.4843714)
\curveto(515.17185869,189.17186223)(514.75519244,190.4999859)(513.92187386,191.4687464)
\curveto(513.0989441,192.43748396)(511.97915355,192.92185848)(510.56249886,192.9218714)
\moveto(516.82812386,202.8124964)
\lineto(516.82812386,199.9374964)
\curveto(516.03644116,200.31247609)(515.23435863,200.59893414)(514.42187386,200.7968714)
\curveto(513.61977691,200.99476707)(512.82290271,201.09372531)(512.03124886,201.0937464)
\curveto(509.94790558,201.09372531)(508.35415718,200.39060101)(507.24999886,198.9843714)
\curveto(506.15624271,197.57810382)(505.53124333,195.45310595)(505.37499886,192.6093714)
\curveto(505.98957621,193.51560789)(506.76040877,194.20831553)(507.68749886,194.6874964)
\curveto(508.61457358,195.17706456)(509.63540589,195.42185598)(510.74999886,195.4218714)
\curveto(513.09373577,195.42185598)(514.94269225,194.70831503)(516.29687386,193.2812464)
\curveto(517.66143954,191.86456787)(518.34373052,189.93227814)(518.34374886,187.4843714)
\curveto(518.34373052,185.08853298)(517.63539789,183.1666599)(516.21874886,181.7187464)
\curveto(514.80206739,180.27082946)(512.91665261,179.54687186)(510.56249886,179.5468714)
\curveto(507.86457433,179.54687186)(505.80207639,180.57812082)(504.37499886,182.6406214)
\curveto(502.94791258,184.71353336)(502.23437163,187.71353036)(502.23437386,191.6406214)
\curveto(502.23437163,195.32810607)(503.10937075,198.26560314)(504.85937386,200.4531214)
\curveto(506.60936725,202.65101542)(508.95832324,203.74997265)(511.90624886,203.7499964)
\curveto(512.69790283,203.74997265)(513.49477704,203.67184773)(514.29687386,203.5156214)
\curveto(515.10935875,203.35934804)(515.95310791,203.12497328)(516.82812386,202.8124964)
}
}
{
\newrgbcolor{curcolor}{0 0 0}
\pscustom[linestyle=none,fillstyle=solid,fillcolor=curcolor]
{
\newpath
\moveto(42.62499886,153.3281214)
\lineto(57.62499886,153.3281214)
\lineto(57.62499886,151.9843714)
\lineto(49.15624886,129.9999964)
\lineto(45.85937386,129.9999964)
\lineto(53.82812386,150.6718714)
\lineto(42.62499886,150.6718714)
\lineto(42.62499886,153.3281214)
}
}
{
\newrgbcolor{curcolor}{0 0 0}
\pscustom[linestyle=none,fillstyle=solid,fillcolor=curcolor]
{
\newpath
\moveto(50.17187386,51.0781214)
\curveto(48.67186519,51.07811032)(47.48957471,50.67706906)(46.62499886,49.8749964)
\curveto(45.77082643,49.072904)(45.34374352,47.96873843)(45.34374886,46.5624964)
\curveto(45.34374352,45.15624125)(45.77082643,44.05207568)(46.62499886,43.2499964)
\curveto(47.48957471,42.44791062)(48.67186519,42.04686936)(50.17187386,42.0468714)
\curveto(51.67186219,42.04686936)(52.85415268,42.44791062)(53.71874886,43.2499964)
\curveto(54.58331761,44.06249234)(55.01560885,45.1666579)(55.01562386,46.5624964)
\curveto(55.01560885,47.96873843)(54.58331761,49.072904)(53.71874886,49.8749964)
\curveto(52.86456933,50.67706906)(51.68227885,51.07811032)(50.17187386,51.0781214)
\moveto(47.01562386,52.4218714)
\curveto(45.66145154,52.75519198)(44.60416093,53.38539968)(43.84374886,54.3124964)
\curveto(43.09374577,55.2395645)(42.71874614,56.3697717)(42.71874886,57.7031214)
\curveto(42.71874614,59.56768517)(43.38020382,61.04164203)(44.70312386,62.1249964)
\curveto(46.03645116,63.20830653)(47.859366,63.74997265)(50.17187386,63.7499964)
\curveto(52.49477804,63.74997265)(54.31769288,63.20830653)(55.64062386,62.1249964)
\curveto(56.96352357,61.04164203)(57.62498124,59.56768517)(57.62499886,57.7031214)
\curveto(57.62498124,56.3697717)(57.24477329,55.2395645)(56.48437386,54.3124964)
\curveto(55.73435813,53.38539968)(54.68748418,52.75519198)(53.34374886,52.4218714)
\curveto(54.86456733,52.06769267)(56.04685782,51.37498503)(56.89062386,50.3437464)
\curveto(57.74477279,49.31248709)(58.17185569,48.05207168)(58.17187386,46.5624964)
\curveto(58.17185569,44.30207543)(57.47914805,42.56770217)(56.09374886,41.3593714)
\curveto(54.71873414,40.15103792)(52.74477779,39.54687186)(50.17187386,39.5468714)
\curveto(47.5989496,39.54687186)(45.61978491,40.15103792)(44.23437386,41.3593714)
\curveto(42.859371,42.56770217)(42.17187169,44.30207543)(42.17187386,46.5624964)
\curveto(42.17187169,48.05207168)(42.5989546,49.31248709)(43.45312386,50.3437464)
\curveto(44.30728622,51.37498503)(45.49478504,52.06769267)(47.01562386,52.4218714)
\moveto(45.85937386,57.4062464)
\curveto(45.859368,56.19789687)(46.23436763,55.25518948)(46.98437386,54.5781214)
\curveto(47.74478279,53.90102417)(48.80728172,53.56248284)(50.17187386,53.5624964)
\curveto(51.526029,53.56248284)(52.58331961,53.90102417)(53.34374886,54.5781214)
\curveto(54.11456808,55.25518948)(54.49998436,56.19789687)(54.49999886,57.4062464)
\curveto(54.49998436,58.61456112)(54.11456808,59.55726851)(53.34374886,60.2343714)
\curveto(52.58331961,60.91143382)(51.526029,61.24997515)(50.17187386,61.2499964)
\curveto(48.80728172,61.24997515)(47.74478279,60.91143382)(46.98437386,60.2343714)
\curveto(46.23436763,59.55726851)(45.859368,58.61456112)(45.85937386,57.4062464)
}
}
{
\newrgbcolor{curcolor}{0 0 0}
\pscustom[linewidth=1.80982327,linecolor=curcolor,linestyle=dashed,dash=7.23929296 7.23929296]
{
\newpath
\moveto(409.999999,339.999999)
\lineto(499.999999,339.999999)
}
}
{
\newrgbcolor{curcolor}{0 0 0}
\pscustom[linestyle=none,fillstyle=solid,fillcolor=curcolor]
{
\newpath
\moveto(419.46746087,335.61979029)
\lineto(407.60311665,339.9826103)
\lineto(419.46746152,344.34542935)
\curveto(417.57203535,341.76962273)(417.58295683,338.24547223)(419.46746087,335.61979029)
\lineto(419.46746087,335.61979029)
\closepath
}
}
{
\newrgbcolor{curcolor}{0 0 0}
\pscustom[linewidth=1.89703047,linecolor=curcolor,linestyle=dashed,dash=7.58812199 7.58812199]
{
\newpath
\moveto(339.999999,289.999819)
\lineto(500.262979,289.999819)
}
}
{
\newrgbcolor{curcolor}{0 0 0}
\pscustom[linestyle=none,fillstyle=solid,fillcolor=curcolor]
{
\newpath
\moveto(349.92365504,285.40854784)
\lineto(337.48762173,289.98159241)
\lineto(349.92365572,294.55463598)
\curveto(347.93689753,291.8547129)(347.94834528,288.16074952)(349.92365504,285.40854784)
\lineto(349.92365504,285.40854784)
\closepath
}
}
{
\newrgbcolor{curcolor}{0 0 0}
\pscustom[linewidth=1.86349273,linecolor=curcolor,linestyle=dashed,dash=7.4539707 7.4539707]
{
\newpath
\moveto(369.999999,189.999789)
\lineto(500.322879,189.999789)
}
}
{
\newrgbcolor{curcolor}{0 0 0}
\pscustom[linestyle=none,fillstyle=solid,fillcolor=curcolor]
{
\newpath
\moveto(379.74821397,185.48968727)
\lineto(367.53203824,189.98188464)
\lineto(379.74821465,194.47408103)
\curveto(377.7965805,191.82189009)(377.80782586,188.19323257)(379.74821397,185.48968727)
\lineto(379.74821397,185.48968727)
\closepath
}
}
{
\newrgbcolor{curcolor}{0 0 0}
\pscustom[linewidth=1.84759188,linecolor=curcolor,linestyle=dashed,dash=7.39036736 7.39036736]
{
\newpath
\moveto(389.999999,239.998149)
\lineto(501.401259,239.833959)
}
}
{
\newrgbcolor{curcolor}{0 0 0}
\pscustom[linestyle=none,fillstyle=solid,fillcolor=curcolor]
{
\newpath
\moveto(399.65843316,235.51229112)
\lineto(387.5530734,239.98400383)
\lineto(399.67156258,244.42001297)
\curveto(397.73270788,241.79330745)(397.73855477,238.19560009)(399.65843316,235.51229112)
\lineto(399.65843316,235.51229112)
\closepath
}
}
{
\newrgbcolor{curcolor}{0 0 0}
\pscustom[linewidth=1.85803509,linecolor=curcolor,linestyle=dashed,dash=7.43214029 7.43214029]
{
\newpath
\moveto(189.999999,389.999999)
\lineto(60,389.999999)
}
}
{
\newrgbcolor{curcolor}{0 0 0}
\pscustom[linestyle=none,fillstyle=solid,fillcolor=curcolor]
{
\newpath
\moveto(180.28033377,394.49689192)
\lineto(192.4607318,390.01785092)
\lineto(180.2803331,385.5388109)
\curveto(182.22625147,388.18323433)(182.21503904,391.80126454)(180.28033377,394.49689192)
\lineto(180.28033377,394.49689192)
\closepath
}
}
{
\newrgbcolor{curcolor}{0 0 0}
\pscustom[linewidth=1.89736664,linecolor=curcolor,linestyle=dashed,dash=7.58946638 7.58946638]
{
\newpath
\moveto(239.999999,139.999999)
\lineto(60,139.999999)
}
}
{
\newrgbcolor{curcolor}{0 0 0}
\pscustom[linestyle=none,fillstyle=solid,fillcolor=curcolor]
{
\newpath
\moveto(230.0745844,144.59208377)
\lineto(242.51282148,140.01822882)
\lineto(230.07458372,135.44437486)
\curveto(232.06169398,138.1447764)(232.05024421,141.83939438)(230.0745844,144.59208377)
\lineto(230.0745844,144.59208377)
\closepath
}
}
{
\newrgbcolor{curcolor}{0 0 0}
\pscustom[linewidth=2,linecolor=curcolor,linestyle=dashed,dash=8 8]
{
\newpath
\moveto(190,50)
\lineto(60,50)
}
}
{
\newrgbcolor{curcolor}{0 0 0}
\pscustom[linestyle=none,fillstyle=solid,fillcolor=curcolor]
{
\newpath
\moveto(179.53769464,54.84048224)
\lineto(192.6487474,50.01921591)
\lineto(179.53769392,45.19795064)
\curveto(181.632292,48.04442372)(181.62022288,51.93889292)(179.53769464,54.84048224)
\lineto(179.53769464,54.84048224)
\closepath
}
}
{
\newrgbcolor{curcolor}{0 0 0}
\pscustom[linewidth=2,linecolor=curcolor,linestyle=dashed,dash=8 8]
{
\newpath
\moveto(60,520)
\lineto(160,520)
}
}
{
\newrgbcolor{curcolor}{0 0 0}
\pscustom[linestyle=none,fillstyle=solid,fillcolor=curcolor]
{
\newpath
\moveto(149.53769464,524.84048224)
\lineto(162.6487474,520.01921591)
\lineto(149.53769392,515.19795064)
\curveto(151.632292,518.04442372)(151.62022288,521.93889292)(149.53769464,524.84048224)
\closepath
}
}
\end{pspicture}

		
		\begin{enumerate}
		  \item Fond d'écran
		  \item Bouton ``\hyperlink{Creer partie solo}{Jouer en solo}''
		  \item Bouton ``\hyperlink{Connexion multi-joueurs}{Jouer en multi-joueurs}''
		  \item Bouton ``\hyperlink{Options}{Options}''
		  \item Bouton ``\hyperlink{Statistiques}{Mes statistiques}''
		  \item Bouton ``''
		  \item Bouton ``\hyperlink{Aide}{Aide}''
		  \item Liste déroulante ``Comptes hors lignes''
		  \item Bouton ``\hyperlink{Création du profil}{+}'' 
		\end{enumerate}
		
		$\,$
		
		Décrire les autres boutons ? \\		
		La liste déroulante contient la totalité des comptes hors lignes crées stockés
		sur le téléphone. \\
		Le bouton \hyperlink{Création du profil}{+} permet de créer un nouveau compte
		hors ligne, pour cela il redirige l'utilisateur vers la page de création du profil%
		\footnote[1]{
			\hyperlink{Création du profil}{Page de création du profil}
			\og voir section \ref{profil}, page \pageref{profil}.\fg
		}
		
		$\,$
	
\newpage

	\subsection{Créer partie solo}
	
		\hypertarget{Creer partie solo}{}
		\label{Creer partie solo}
	
		%LaTeX with PSTricks extensions
%%Creator: inkscape 0.48.0
%%Please note this file requires PSTricks extensions
\psset{xunit=.4pt,yunit=.4pt,runit=.4pt}
\begin{pspicture}(560,600)
{
\newrgbcolor{curcolor}{1 1 1}
\pscustom[linestyle=none,fillstyle=solid,fillcolor=curcolor]
{
\newpath
\moveto(133.12401581,597.52220273)
\lineto(426.87598419,597.52220273)
\curveto(443.85397169,597.52220273)(457.52217102,583.8540034)(457.52217102,566.8760159)
\lineto(457.52217102,33.124017)
\curveto(457.52217102,16.1460295)(443.85397169,2.47783017)(426.87598419,2.47783017)
\lineto(133.12401581,2.47783017)
\curveto(116.14602831,2.47783017)(102.47782898,16.1460295)(102.47782898,33.124017)
\lineto(102.47782898,566.8760159)
\curveto(102.47782898,583.8540034)(116.14602831,597.52220273)(133.12401581,597.52220273)
\closepath
}
}
{
\newrgbcolor{curcolor}{0 0 0}
\pscustom[linewidth=4.95566034,linecolor=curcolor]
{
\newpath
\moveto(133.12401581,597.52220273)
\lineto(426.87598419,597.52220273)
\curveto(443.85397169,597.52220273)(457.52217102,583.8540034)(457.52217102,566.8760159)
\lineto(457.52217102,33.124017)
\curveto(457.52217102,16.1460295)(443.85397169,2.47783017)(426.87598419,2.47783017)
\lineto(133.12401581,2.47783017)
\curveto(116.14602831,2.47783017)(102.47782898,16.1460295)(102.47782898,33.124017)
\lineto(102.47782898,566.8760159)
\curveto(102.47782898,583.8540034)(116.14602831,597.52220273)(133.12401581,597.52220273)
\closepath
}
}
{
\newrgbcolor{curcolor}{1 1 1}
\pscustom[linestyle=none,fillstyle=solid,fillcolor=curcolor]
{
\newpath
\moveto(135.10877228,299.91494751)
\lineto(423.47055817,299.91494751)
\curveto(437.43929419,299.91494751)(448.68488312,288.66935858)(448.68488312,274.70062256)
\lineto(448.68488312,165.27957153)
\curveto(448.68488312,151.31083551)(437.43929419,140.06524658)(423.47055817,140.06524658)
\lineto(135.10877228,140.06524658)
\curveto(121.14003625,140.06524658)(109.89444733,151.31083551)(109.89444733,165.27957153)
\lineto(109.89444733,274.70062256)
\curveto(109.89444733,288.66935858)(121.14003625,299.91494751)(135.10877228,299.91494751)
\closepath
}
}
{
\newrgbcolor{curcolor}{0 0 0}
\pscustom[linewidth=2,linecolor=curcolor]
{
\newpath
\moveto(135.10877228,299.91494751)
\lineto(423.47055817,299.91494751)
\curveto(437.43929419,299.91494751)(448.68488312,288.66935858)(448.68488312,274.70062256)
\lineto(448.68488312,165.27957153)
\curveto(448.68488312,151.31083551)(437.43929419,140.06524658)(423.47055817,140.06524658)
\lineto(135.10877228,140.06524658)
\curveto(121.14003625,140.06524658)(109.89444733,151.31083551)(109.89444733,165.27957153)
\lineto(109.89444733,274.70062256)
\curveto(109.89444733,288.66935858)(121.14003625,299.91494751)(135.10877228,299.91494751)
\closepath
}
}
{
\newrgbcolor{curcolor}{0 0 0}
\pscustom[linestyle=none,fillstyle=solid,fillcolor=curcolor,opacity=0.11935484]
{
\newpath
\moveto(312.75479603,289.96047974)
\lineto(417.33941174,289.96047974)
\curveto(424.44983058,289.96047974)(430.17410278,285.79401281)(430.17410278,280.61862564)
\lineto(430.17410278,269.59705544)
\curveto(430.17410278,264.42166827)(424.44983058,260.25520134)(417.33941174,260.25520134)
\lineto(312.75479603,260.25520134)
\curveto(305.64437719,260.25520134)(299.92010498,264.42166827)(299.92010498,269.59705544)
\lineto(299.92010498,280.61862564)
\curveto(299.92010498,285.79401281)(305.64437719,289.96047974)(312.75479603,289.96047974)
\closepath
}
}
{
\newrgbcolor{curcolor}{0 0 0}
\pscustom[linewidth=2,linecolor=curcolor]
{
\newpath
\moveto(312.75479603,289.96047974)
\lineto(417.33941174,289.96047974)
\curveto(424.44983058,289.96047974)(430.17410278,285.79401281)(430.17410278,280.61862564)
\lineto(430.17410278,269.59705544)
\curveto(430.17410278,264.42166827)(424.44983058,260.25520134)(417.33941174,260.25520134)
\lineto(312.75479603,260.25520134)
\curveto(305.64437719,260.25520134)(299.92010498,264.42166827)(299.92010498,269.59705544)
\lineto(299.92010498,280.61862564)
\curveto(299.92010498,285.79401281)(305.64437719,289.96047974)(312.75479603,289.96047974)
\closepath
}
}
{
\newrgbcolor{curcolor}{1 1 1}
\pscustom[linestyle=none,fillstyle=solid,fillcolor=curcolor]
{
\newpath
\moveto(190.0634613,350.19837952)
\lineto(369.92247009,350.19837952)
\lineto(369.92247009,319.90265656)
\lineto(190.0634613,319.90265656)
\closepath
}
}
{
\newrgbcolor{curcolor}{0 0 0}
\pscustom[linewidth=1.57047963,linecolor=curcolor]
{
\newpath
\moveto(190.0634613,350.19837952)
\lineto(369.92247009,350.19837952)
\lineto(369.92247009,319.90265656)
\lineto(190.0634613,319.90265656)
\closepath
}
}
{
\newrgbcolor{curcolor}{1 1 1}
\pscustom[linestyle=none,fillstyle=solid,fillcolor=curcolor]
{
\newpath
\moveto(220.09861755,479.90138245)
\lineto(340.15664673,479.90138245)
\lineto(340.15664673,359.90190125)
\lineto(220.09861755,359.90190125)
\closepath
}
}
{
\newrgbcolor{curcolor}{0 0 0}
\pscustom[linewidth=2,linecolor=curcolor]
{
\newpath
\moveto(220.09861755,479.90138245)
\lineto(340.15664673,479.90138245)
\lineto(340.15664673,359.90190125)
\lineto(220.09861755,359.90190125)
\closepath
}
}
{
\newrgbcolor{curcolor}{1 1 1}
\pscustom[linestyle=none,fillstyle=solid,fillcolor=curcolor]
{
\newpath
\moveto(360.10479736,460.10528564)
\lineto(440.10527039,460.10528564)
\lineto(440.10527039,380.14836884)
\lineto(360.10479736,380.14836884)
\closepath
}
}
{
\newrgbcolor{curcolor}{0 0 0}
\pscustom[linewidth=2,linecolor=curcolor]
{
\newpath
\moveto(360.10479736,460.10528564)
\lineto(440.10527039,460.10528564)
\lineto(440.10527039,380.14836884)
\lineto(360.10479736,380.14836884)
\closepath
}
}
{
\newrgbcolor{curcolor}{1 1 1}
\pscustom[linestyle=none,fillstyle=solid,fillcolor=curcolor]
{
\newpath
\moveto(119.89472198,460.10528564)
\lineto(199.89520264,460.10528564)
\lineto(199.89520264,380.14836884)
\lineto(119.89472198,380.14836884)
\closepath
}
}
{
\newrgbcolor{curcolor}{0 0 0}
\pscustom[linewidth=2,linecolor=curcolor]
{
\newpath
\moveto(119.89472198,460.10528564)
\lineto(199.89520264,460.10528564)
\lineto(199.89520264,380.14836884)
\lineto(119.89472198,380.14836884)
\closepath
}
}
{
\newrgbcolor{curcolor}{0 0 0}
\pscustom[linestyle=none,fillstyle=solid,fillcolor=curcolor]
{
\newpath
\moveto(119.93554688,286.03808594)
\lineto(133.50292969,286.03808594)
\lineto(133.50292969,284.21191406)
\lineto(127.80957031,284.21191406)
\lineto(127.80957031,270)
\lineto(125.62890625,270)
\lineto(125.62890625,284.21191406)
\lineto(119.93554688,284.21191406)
\lineto(119.93554688,286.03808594)
}
}
{
\newrgbcolor{curcolor}{0 0 0}
\pscustom[linestyle=none,fillstyle=solid,fillcolor=curcolor]
{
\newpath
\moveto(137.09082031,268.8828125)
\curveto(136.53222004,267.45052338)(135.98794975,266.515954)(135.45800781,266.07910156)
\curveto(134.92805498,265.64225696)(134.21907131,265.4238327)(133.33105469,265.42382812)
\lineto(131.75195312,265.42382812)
\lineto(131.75195312,267.078125)
\lineto(132.91210938,267.078125)
\curveto(133.45637676,267.07812792)(133.87890238,267.20703404)(134.1796875,267.46484375)
\curveto(134.48046428,267.72265853)(134.81347176,268.33138188)(135.17871094,269.29101562)
\lineto(135.53320312,270.19335938)
\lineto(130.66699219,282.03125)
\lineto(132.76171875,282.03125)
\lineto(136.52148438,272.62109375)
\lineto(140.28125,282.03125)
\lineto(142.37597656,282.03125)
\lineto(137.09082031,268.8828125)
}
}
{
\newrgbcolor{curcolor}{0 0 0}
\pscustom[linestyle=none,fillstyle=solid,fillcolor=curcolor]
{
\newpath
\moveto(147.01660156,271.8046875)
\lineto(147.01660156,265.42382812)
\lineto(145.02929688,265.42382812)
\lineto(145.02929688,282.03125)
\lineto(147.01660156,282.03125)
\lineto(147.01660156,280.20507812)
\curveto(147.43196175,280.92121304)(147.95474768,281.45116042)(148.58496094,281.79492188)
\curveto(149.22232454,282.14582119)(149.98143836,282.32127674)(150.86230469,282.32128906)
\curveto(152.3232329,282.32127674)(153.50845306,281.7411992)(154.41796875,280.58105469)
\curveto(155.33462311,279.42088902)(155.79295599,277.89549992)(155.79296875,276.00488281)
\curveto(155.79295599,274.1142537)(155.33462311,272.5888646)(154.41796875,271.42871094)
\curveto(153.50845306,270.26855442)(152.3232329,269.68847687)(150.86230469,269.68847656)
\curveto(149.98143836,269.68847687)(149.22232454,269.8603517)(148.58496094,270.20410156)
\curveto(147.95474768,270.55501247)(147.43196175,271.08854058)(147.01660156,271.8046875)
\moveto(153.74121094,276.00488281)
\curveto(153.74120023,277.4586514)(153.44041928,278.59732213)(152.83886719,279.42089844)
\curveto(152.24445693,280.25161735)(151.42447077,280.66698152)(150.37890625,280.66699219)
\curveto(149.33332703,280.66698152)(148.50976015,280.25161735)(147.90820312,279.42089844)
\curveto(147.3137978,278.59732213)(147.01659758,277.4586514)(147.01660156,276.00488281)
\curveto(147.01659758,274.55110222)(147.3137978,273.40885076)(147.90820312,272.578125)
\curveto(148.50976015,271.75455554)(149.33332703,271.34277209)(150.37890625,271.34277344)
\curveto(151.42447077,271.34277209)(152.24445693,271.75455554)(152.83886719,272.578125)
\curveto(153.44041928,273.40885076)(153.74120023,274.55110222)(153.74121094,276.00488281)
}
}
{
\newrgbcolor{curcolor}{0 0 0}
\pscustom[linestyle=none,fillstyle=solid,fillcolor=curcolor]
{
\newpath
\moveto(169.36035156,276.50976562)
\lineto(169.36035156,275.54296875)
\lineto(160.27246094,275.54296875)
\curveto(160.35839508,274.18228748)(160.76659779,273.14387706)(161.49707031,272.42773438)
\curveto(162.23469528,271.71874828)(163.2587828,271.36425645)(164.56933594,271.36425781)
\curveto(165.32844219,271.36425645)(166.06249093,271.45735531)(166.77148438,271.64355469)
\curveto(167.48761972,271.82975077)(168.19660338,272.10904737)(168.8984375,272.48144531)
\lineto(168.8984375,270.61230469)
\curveto(168.18944193,270.31152313)(167.46255464,270.08235669)(166.71777344,269.92480469)
\curveto(165.97297279,269.76725284)(165.2174397,269.68847687)(164.45117188,269.68847656)
\curveto(162.53189551,269.68847687)(161.01008713,270.24707007)(159.88574219,271.36425781)
\curveto(158.76855292,272.48144283)(158.20995972,273.99250903)(158.20996094,275.89746094)
\curveto(158.20995972,277.86685411)(158.73990711,279.42805047)(159.79980469,280.58105469)
\curveto(160.86685811,281.7411992)(162.30272907,282.32127674)(164.10742188,282.32128906)
\curveto(165.72590273,282.32127674)(167.00422176,281.79849081)(167.94238281,280.75292969)
\curveto(168.88768342,279.71450851)(169.3603392,278.30012191)(169.36035156,276.50976562)
\moveto(167.38378906,277.08984375)
\curveto(167.36945577,278.17121579)(167.0650941,279.03417065)(166.47070312,279.67871094)
\curveto(165.88345465,280.32323186)(165.10285648,280.64549717)(164.12890625,280.64550781)
\curveto(163.02603564,280.64549717)(162.14159642,280.33397404)(161.47558594,279.7109375)
\curveto(160.81672795,279.08788154)(160.43717104,278.21060377)(160.33691406,277.07910156)
\lineto(167.38378906,277.08984375)
}
}
{
\newrgbcolor{curcolor}{0 0 0}
\pscustom[linestyle=none,fillstyle=solid,fillcolor=curcolor]
{
}
}
{
\newrgbcolor{curcolor}{0 0 0}
\pscustom[linestyle=none,fillstyle=solid,fillcolor=curcolor]
{
\newpath
\moveto(187.52539062,280.20507812)
\lineto(187.52539062,286.71484375)
\lineto(189.50195312,286.71484375)
\lineto(189.50195312,270)
\lineto(187.52539062,270)
\lineto(187.52539062,271.8046875)
\curveto(187.11001647,271.08854058)(186.58364981,270.55501247)(185.94628906,270.20410156)
\curveto(185.31607295,269.8603517)(184.55695912,269.68847687)(183.66894531,269.68847656)
\curveto(182.21516459,269.68847687)(181.02994442,270.26855442)(180.11328125,271.42871094)
\curveto(179.20377437,272.5888646)(178.74902222,274.1142537)(178.74902344,276.00488281)
\curveto(178.74902222,277.89549992)(179.20377437,279.42088902)(180.11328125,280.58105469)
\curveto(181.02994442,281.7411992)(182.21516459,282.32127674)(183.66894531,282.32128906)
\curveto(184.55695912,282.32127674)(185.31607295,282.14582119)(185.94628906,281.79492188)
\curveto(186.58364981,281.45116042)(187.11001647,280.92121304)(187.52539062,280.20507812)
\moveto(180.79003906,276.00488281)
\curveto(180.79003581,274.55110222)(181.08723603,273.40885076)(181.68164062,272.578125)
\curveto(182.28319838,271.75455554)(183.10676526,271.34277209)(184.15234375,271.34277344)
\curveto(185.197909,271.34277209)(186.02147589,271.75455554)(186.62304688,272.578125)
\curveto(187.22459969,273.40885076)(187.52538063,274.55110222)(187.52539062,276.00488281)
\curveto(187.52538063,277.4586514)(187.22459969,278.59732213)(186.62304688,279.42089844)
\curveto(186.02147589,280.25161735)(185.197909,280.66698152)(184.15234375,280.66699219)
\curveto(183.10676526,280.66698152)(182.28319838,280.25161735)(181.68164062,279.42089844)
\curveto(181.08723603,278.59732213)(180.79003581,277.4586514)(180.79003906,276.00488281)
}
}
{
\newrgbcolor{curcolor}{0 0 0}
\pscustom[linestyle=none,fillstyle=solid,fillcolor=curcolor]
{
\newpath
\moveto(203.86425781,276.50976562)
\lineto(203.86425781,275.54296875)
\lineto(194.77636719,275.54296875)
\curveto(194.86230133,274.18228748)(195.27050404,273.14387706)(196.00097656,272.42773438)
\curveto(196.73860153,271.71874828)(197.76268905,271.36425645)(199.07324219,271.36425781)
\curveto(199.83234844,271.36425645)(200.56639718,271.45735531)(201.27539062,271.64355469)
\curveto(201.99152597,271.82975077)(202.70050963,272.10904737)(203.40234375,272.48144531)
\lineto(203.40234375,270.61230469)
\curveto(202.69334818,270.31152313)(201.96646089,270.08235669)(201.22167969,269.92480469)
\curveto(200.47687904,269.76725284)(199.72134595,269.68847687)(198.95507812,269.68847656)
\curveto(197.03580176,269.68847687)(195.51399338,270.24707007)(194.38964844,271.36425781)
\curveto(193.27245917,272.48144283)(192.71386597,273.99250903)(192.71386719,275.89746094)
\curveto(192.71386597,277.86685411)(193.24381336,279.42805047)(194.30371094,280.58105469)
\curveto(195.37076436,281.7411992)(196.80663532,282.32127674)(198.61132812,282.32128906)
\curveto(200.22980898,282.32127674)(201.50812801,281.79849081)(202.44628906,280.75292969)
\curveto(203.39158967,279.71450851)(203.86424545,278.30012191)(203.86425781,276.50976562)
\moveto(201.88769531,277.08984375)
\curveto(201.87336202,278.17121579)(201.56900035,279.03417065)(200.97460938,279.67871094)
\curveto(200.3873609,280.32323186)(199.60676273,280.64549717)(198.6328125,280.64550781)
\curveto(197.52994189,280.64549717)(196.64550267,280.33397404)(195.97949219,279.7109375)
\curveto(195.3206342,279.08788154)(194.94107729,278.21060377)(194.84082031,277.07910156)
\lineto(201.88769531,277.08984375)
}
}
{
\newrgbcolor{curcolor}{0 0 0}
\pscustom[linestyle=none,fillstyle=solid,fillcolor=curcolor]
{
}
}
{
\newrgbcolor{curcolor}{0 0 0}
\pscustom[linestyle=none,fillstyle=solid,fillcolor=curcolor]
{
\newpath
\moveto(216.02441406,271.8046875)
\lineto(216.02441406,265.42382812)
\lineto(214.03710938,265.42382812)
\lineto(214.03710938,282.03125)
\lineto(216.02441406,282.03125)
\lineto(216.02441406,280.20507812)
\curveto(216.43977425,280.92121304)(216.96256018,281.45116042)(217.59277344,281.79492188)
\curveto(218.23013704,282.14582119)(218.98925086,282.32127674)(219.87011719,282.32128906)
\curveto(221.3310454,282.32127674)(222.51626556,281.7411992)(223.42578125,280.58105469)
\curveto(224.34243561,279.42088902)(224.80076849,277.89549992)(224.80078125,276.00488281)
\curveto(224.80076849,274.1142537)(224.34243561,272.5888646)(223.42578125,271.42871094)
\curveto(222.51626556,270.26855442)(221.3310454,269.68847687)(219.87011719,269.68847656)
\curveto(218.98925086,269.68847687)(218.23013704,269.8603517)(217.59277344,270.20410156)
\curveto(216.96256018,270.55501247)(216.43977425,271.08854058)(216.02441406,271.8046875)
\moveto(222.74902344,276.00488281)
\curveto(222.74901273,277.4586514)(222.44823178,278.59732213)(221.84667969,279.42089844)
\curveto(221.25226943,280.25161735)(220.43228327,280.66698152)(219.38671875,280.66699219)
\curveto(218.34113953,280.66698152)(217.51757265,280.25161735)(216.91601562,279.42089844)
\curveto(216.3216103,278.59732213)(216.02441008,277.4586514)(216.02441406,276.00488281)
\curveto(216.02441008,274.55110222)(216.3216103,273.40885076)(216.91601562,272.578125)
\curveto(217.51757265,271.75455554)(218.34113953,271.34277209)(219.38671875,271.34277344)
\curveto(220.43228327,271.34277209)(221.25226943,271.75455554)(221.84667969,272.578125)
\curveto(222.44823178,273.40885076)(222.74901273,274.55110222)(222.74902344,276.00488281)
}
}
{
\newrgbcolor{curcolor}{0 0 0}
\pscustom[linestyle=none,fillstyle=solid,fillcolor=curcolor]
{
\newpath
\moveto(233.54492188,276.04785156)
\curveto(231.94791072,276.04784551)(230.84146652,275.86522851)(230.22558594,275.5)
\curveto(229.60969692,275.13476049)(229.30175451,274.51171424)(229.30175781,273.63085938)
\curveto(229.30175451,272.92903353)(229.53092095,272.37044034)(229.98925781,271.95507812)
\curveto(230.45474815,271.54687345)(231.08495586,271.34277209)(231.87988281,271.34277344)
\curveto(232.97557897,271.34277209)(233.85285673,271.72949046)(234.51171875,272.50292969)
\curveto(235.1777252,273.28352536)(235.51073268,274.31835506)(235.51074219,275.60742188)
\lineto(235.51074219,276.04785156)
\lineto(233.54492188,276.04785156)
\moveto(237.48730469,276.86425781)
\lineto(237.48730469,270)
\lineto(235.51074219,270)
\lineto(235.51074219,271.82617188)
\curveto(235.05956126,271.09570203)(234.49738734,270.55501247)(233.82421875,270.20410156)
\curveto(233.15103452,269.8603517)(232.32746763,269.68847687)(231.35351562,269.68847656)
\curveto(230.12174067,269.68847687)(229.14062186,270.03222653)(228.41015625,270.71972656)
\curveto(227.68684728,271.41438661)(227.32519399,272.34179453)(227.32519531,273.50195312)
\curveto(227.32519399,274.85546389)(227.77636542,275.87597069)(228.67871094,276.56347656)
\curveto(229.58821256,277.25096931)(230.94172683,277.59471897)(232.73925781,277.59472656)
\lineto(235.51074219,277.59472656)
\lineto(235.51074219,277.78808594)
\curveto(235.51073268,278.69758245)(235.20995173,279.39940466)(234.60839844,279.89355469)
\curveto(234.01398939,280.39484638)(233.1760996,280.64549717)(232.09472656,280.64550781)
\curveto(231.40722116,280.64549717)(230.73762547,280.56314048)(230.0859375,280.3984375)
\curveto(229.43424136,280.23371372)(228.80761438,279.98664366)(228.20605469,279.65722656)
\lineto(228.20605469,281.48339844)
\curveto(228.92935905,281.76268355)(229.63118127,281.97036563)(230.31152344,282.10644531)
\curveto(230.99185699,282.24966223)(231.65429122,282.32127674)(232.29882812,282.32128906)
\curveto(234.03905446,282.32127674)(235.33885785,281.87010532)(236.19824219,280.96777344)
\curveto(237.05760613,280.06541962)(237.4872932,278.69758245)(237.48730469,276.86425781)
}
}
{
\newrgbcolor{curcolor}{0 0 0}
\pscustom[linestyle=none,fillstyle=solid,fillcolor=curcolor]
{
\newpath
\moveto(248.54101562,280.18359375)
\curveto(248.31900159,280.31248969)(248.07551225,280.40558855)(247.81054688,280.46289062)
\curveto(247.55272632,280.52733322)(247.26626827,280.55955975)(246.95117188,280.55957031)
\curveto(245.83397804,280.55955975)(244.9746039,280.19432574)(244.37304688,279.46386719)
\curveto(243.77864155,278.74055116)(243.48144133,277.69856001)(243.48144531,276.33789062)
\lineto(243.48144531,270)
\lineto(241.49414062,270)
\lineto(241.49414062,282.03125)
\lineto(243.48144531,282.03125)
\lineto(243.48144531,280.16210938)
\curveto(243.8968055,280.89256723)(244.43749506,281.4332568)(245.10351562,281.78417969)
\curveto(245.76952498,282.14224046)(246.57876896,282.32127674)(247.53125,282.32128906)
\curveto(247.66730954,282.32127674)(247.81770001,282.31053456)(247.98242188,282.2890625)
\curveto(248.14712677,282.27472731)(248.32974377,282.24966223)(248.53027344,282.21386719)
\lineto(248.54101562,280.18359375)
}
}
{
\newrgbcolor{curcolor}{0 0 0}
\pscustom[linestyle=none,fillstyle=solid,fillcolor=curcolor]
{
\newpath
\moveto(252.59082031,285.44726562)
\lineto(252.59082031,282.03125)
\lineto(256.66210938,282.03125)
\lineto(256.66210938,280.49511719)
\lineto(252.59082031,280.49511719)
\lineto(252.59082031,273.96386719)
\curveto(252.59081628,272.98274441)(252.72330313,272.35253671)(252.98828125,272.07324219)
\curveto(253.26041197,271.79394352)(253.80826298,271.65429522)(254.63183594,271.65429688)
\lineto(256.66210938,271.65429688)
\lineto(256.66210938,270)
\lineto(254.63183594,270)
\curveto(253.10644077,270)(252.05370745,270.28287732)(251.47363281,270.84863281)
\curveto(250.89355236,271.42154806)(250.60351358,272.45995848)(250.60351562,273.96386719)
\lineto(250.60351562,280.49511719)
\lineto(249.15332031,280.49511719)
\lineto(249.15332031,282.03125)
\lineto(250.60351562,282.03125)
\lineto(250.60351562,285.44726562)
\lineto(252.59082031,285.44726562)
}
}
{
\newrgbcolor{curcolor}{0 0 0}
\pscustom[linestyle=none,fillstyle=solid,fillcolor=curcolor]
{
\newpath
\moveto(259.27246094,282.03125)
\lineto(261.24902344,282.03125)
\lineto(261.24902344,270)
\lineto(259.27246094,270)
\lineto(259.27246094,282.03125)
\moveto(259.27246094,286.71484375)
\lineto(261.24902344,286.71484375)
\lineto(261.24902344,284.21191406)
\lineto(259.27246094,284.21191406)
\lineto(259.27246094,286.71484375)
}
}
{
\newrgbcolor{curcolor}{0 0 0}
\pscustom[linestyle=none,fillstyle=solid,fillcolor=curcolor]
{
\newpath
\moveto(275.66503906,276.50976562)
\lineto(275.66503906,275.54296875)
\lineto(266.57714844,275.54296875)
\curveto(266.66308258,274.18228748)(267.07128529,273.14387706)(267.80175781,272.42773438)
\curveto(268.53938278,271.71874828)(269.5634703,271.36425645)(270.87402344,271.36425781)
\curveto(271.63312969,271.36425645)(272.36717843,271.45735531)(273.07617188,271.64355469)
\curveto(273.79230722,271.82975077)(274.50129088,272.10904737)(275.203125,272.48144531)
\lineto(275.203125,270.61230469)
\curveto(274.49412943,270.31152313)(273.76724214,270.08235669)(273.02246094,269.92480469)
\curveto(272.27766029,269.76725284)(271.5221272,269.68847687)(270.75585938,269.68847656)
\curveto(268.83658301,269.68847687)(267.31477463,270.24707007)(266.19042969,271.36425781)
\curveto(265.07324042,272.48144283)(264.51464722,273.99250903)(264.51464844,275.89746094)
\curveto(264.51464722,277.86685411)(265.04459461,279.42805047)(266.10449219,280.58105469)
\curveto(267.17154561,281.7411992)(268.60741657,282.32127674)(270.41210938,282.32128906)
\curveto(272.03059023,282.32127674)(273.30890926,281.79849081)(274.24707031,280.75292969)
\curveto(275.19237092,279.71450851)(275.6650267,278.30012191)(275.66503906,276.50976562)
\moveto(273.68847656,277.08984375)
\curveto(273.67414327,278.17121579)(273.3697816,279.03417065)(272.77539062,279.67871094)
\curveto(272.18814215,280.32323186)(271.40754398,280.64549717)(270.43359375,280.64550781)
\curveto(269.33072314,280.64549717)(268.44628392,280.33397404)(267.78027344,279.7109375)
\curveto(267.12141545,279.08788154)(266.74185854,278.21060377)(266.64160156,277.07910156)
\lineto(273.68847656,277.08984375)
}
}
{
\newrgbcolor{curcolor}{0 0 0}
\pscustom[linestyle=none,fillstyle=solid,fillcolor=curcolor]
{
\newpath
\moveto(112.111327,235.68264286)
\lineto(114.96844613,235.68264286)
\lineto(121.92216989,222.56300392)
\lineto(121.92216989,235.68264286)
\lineto(123.98097632,235.68264286)
\lineto(123.98097632,220)
\lineto(121.12385719,220)
\lineto(114.17013343,233.11963894)
\lineto(114.17013343,220)
\lineto(112.111327,220)
\lineto(112.111327,235.68264286)
}
}
{
\newrgbcolor{curcolor}{0 0 0}
\pscustom[linestyle=none,fillstyle=solid,fillcolor=curcolor]
{
\newpath
\moveto(132.67838398,230.40957741)
\curveto(131.64197248,230.409567)(130.82265237,230.00340832)(130.22042121,229.19110014)
\curveto(129.61818178,228.38577631)(129.31706414,227.27934403)(129.31706736,225.87179997)
\curveto(129.31706414,224.46424417)(129.61468042,223.35431052)(130.20991709,222.54199569)
\curveto(130.81214827,221.73667852)(131.63496974,221.3340212)(132.67838398,221.33402253)
\curveto(133.70777958,221.3340212)(134.52359832,221.74017989)(135.12584265,222.55249981)
\curveto(135.72806891,223.36481463)(136.02918655,224.47124691)(136.02919649,225.87179997)
\curveto(136.02918655,227.26533856)(135.72806891,228.36826947)(135.12584265,229.18059602)
\curveto(134.52359832,229.99990695)(133.70777958,230.409567)(132.67838398,230.40957741)
\moveto(132.67838398,232.04821926)
\curveto(134.35903403,232.04820722)(135.67904976,231.50199381)(136.63843513,230.40957741)
\curveto(137.59779941,229.31714019)(138.07748682,227.80454923)(138.07749881,225.87179997)
\curveto(138.07748682,223.94604171)(137.59779941,222.43345075)(136.63843513,221.33402253)
\curveto(135.67904976,220.24159439)(134.35903403,219.69538099)(132.67838398,219.69538068)
\curveto(130.99071803,219.69538099)(129.66720094,220.24159439)(128.70782873,221.33402253)
\curveto(127.75545402,222.43345075)(127.27926797,223.94604171)(127.27926916,225.87179997)
\curveto(127.27926797,227.80454923)(127.75545402,229.31714019)(128.70782873,230.40957741)
\curveto(129.66720094,231.50199381)(130.99071803,232.04820722)(132.67838398,232.04821926)
}
}
{
\newrgbcolor{curcolor}{0 0 0}
\pscustom[linestyle=none,fillstyle=solid,fillcolor=curcolor]
{
\newpath
\moveto(150.43033777,229.50622357)
\curveto(150.91351536,230.37455332)(151.49124108,231.01530366)(152.16351665,231.42847651)
\curveto(152.83576638,231.84162651)(153.62707555,232.04820722)(154.53744651,232.04821926)
\curveto(155.76291001,232.04820722)(156.70827937,231.61753895)(157.37355741,230.75621319)
\curveto(158.0387992,229.90186865)(158.37142916,228.68339259)(158.37144828,227.10078136)
\lineto(158.37144828,220)
\lineto(156.42818711,220)
\lineto(156.42818711,227.03775668)
\curveto(156.42816993,228.16519013)(156.22859195,229.00201707)(155.82945259,229.54824003)
\curveto(155.43028005,230.09444388)(154.82104203,230.36755059)(154.00173668,230.36756095)
\curveto(153.00033068,230.36755059)(152.20902151,230.03492063)(151.62780681,229.36967008)
\curveto(151.04656734,228.7044008)(150.7559538,227.79754649)(150.75596531,226.64910444)
\lineto(150.75596531,220)
\lineto(148.81270414,220)
\lineto(148.81270414,227.03775668)
\curveto(148.81269457,228.17219286)(148.6131166,229.00901981)(148.21396962,229.54824003)
\curveto(147.8148047,230.09444388)(147.19856394,230.36755059)(146.36524548,230.36756095)
\curveto(145.37785259,230.36755059)(144.59354616,230.03141926)(144.01232384,229.35916597)
\curveto(143.43109199,228.69389669)(143.14047845,227.79054375)(143.14048234,226.64910444)
\lineto(143.14048234,220)
\lineto(141.19722117,220)
\lineto(141.19722117,231.76460817)
\lineto(143.14048234,231.76460817)
\lineto(143.14048234,229.93689226)
\curveto(143.58165081,230.65816413)(144.11035738,231.19037206)(144.72660363,231.53351766)
\curveto(145.34283891,231.87664019)(146.07462481,232.04820722)(146.92196354,232.04821926)
\curveto(147.77628965,232.04820722)(148.50107282,231.8311224)(149.09631523,231.39696417)
\curveto(149.69854067,230.96278314)(150.14321441,230.33253691)(150.43033777,229.50622357)
}
}
{
\newrgbcolor{curcolor}{0 0 0}
\pscustom[linestyle=none,fillstyle=solid,fillcolor=curcolor]
{
\newpath
\moveto(170.68227312,225.87179997)
\curveto(170.68226265,227.2933495)(170.38814774,228.40678452)(169.79992751,229.21210837)
\curveto(169.21869083,230.02441652)(168.41687756,230.43057521)(167.3944853,230.43058564)
\curveto(166.37207866,230.43057521)(165.56676403,230.02441652)(164.97853898,229.21210837)
\curveto(164.39730712,228.40678452)(164.10669358,227.2933495)(164.10669748,225.87179997)
\curveto(164.10669358,224.4502387)(164.39730712,223.33330232)(164.97853898,222.52098747)
\curveto(165.56676403,221.71567031)(166.37207866,221.31301299)(167.3944853,221.31301431)
\curveto(168.41687756,221.31301299)(169.21869083,221.71567031)(169.79992751,222.52098747)
\curveto(170.38814774,223.33330232)(170.68226265,224.4502387)(170.68227312,225.87179997)
\moveto(164.10669748,229.97890872)
\curveto(164.51285227,230.67917234)(165.02405199,231.1973748)(165.64029819,231.53351766)
\curveto(166.26353626,231.87664019)(167.00582627,232.04820722)(167.86717045,232.04821926)
\curveto(169.29572093,232.04820722)(170.45467373,231.4809856)(171.34403233,230.34655272)
\curveto(172.2403714,229.21209915)(172.6885465,227.72051639)(172.68855898,225.87179997)
\curveto(172.6885465,224.02307181)(172.2403714,222.53148905)(171.34403233,221.39704722)
\curveto(170.45467373,220.2626026)(169.29572093,219.69538099)(167.86717045,219.69538068)
\curveto(167.00582627,219.69538099)(166.26353626,219.86344665)(165.64029819,220.19957817)
\curveto(165.02405199,220.54271204)(164.51285227,221.06441587)(164.10669748,221.76469123)
\lineto(164.10669748,220)
\lineto(162.16343631,220)
\lineto(162.16343631,236.34440207)
\lineto(164.10669748,236.34440207)
\lineto(164.10669748,229.97890872)
}
}
{
\newrgbcolor{curcolor}{0 0 0}
\pscustom[linestyle=none,fillstyle=solid,fillcolor=curcolor]
{
\newpath
\moveto(182.7094813,229.95790049)
\curveto(182.49238764,230.08393978)(182.25429462,230.17497535)(181.99520152,230.23100747)
\curveto(181.74309489,230.29402186)(181.46298545,230.32553417)(181.15487236,230.3255445)
\curveto(180.06243826,230.32553417)(179.22210995,229.96839464)(178.63388489,229.25412482)
\curveto(178.05265304,228.54683924)(177.7620395,227.52794116)(177.7620434,226.19742752)
\lineto(177.7620434,220)
\lineto(175.81878222,220)
\lineto(175.81878222,231.76460817)
\lineto(177.7620434,231.76460817)
\lineto(177.7620434,229.93689226)
\curveto(178.16819818,230.65116139)(178.69690475,231.17986796)(179.34816468,231.52301354)
\curveto(179.99941364,231.87313882)(180.7907228,232.04820722)(181.72209454,232.04821926)
\curveto(181.85513867,232.04820722)(182.00219612,232.03770311)(182.16326735,232.01670692)
\curveto(182.32432197,232.00268943)(182.50289174,231.97817986)(182.69897718,231.94317812)
\lineto(182.7094813,229.95790049)
}
}
{
\newrgbcolor{curcolor}{0 0 0}
\pscustom[linestyle=none,fillstyle=solid,fillcolor=curcolor]
{
\newpath
\moveto(194.35854627,226.36549335)
\lineto(194.35854627,225.42012305)
\lineto(185.47206545,225.42012305)
\curveto(185.55609508,224.0895978)(185.95525103,223.07420109)(186.6695345,222.37392986)
\curveto(187.3908119,221.68065663)(188.39220314,221.3340212)(189.67371123,221.33402253)
\curveto(190.41599383,221.3340212)(191.13377427,221.42505677)(191.82705469,221.60712951)
\curveto(192.52731873,221.78919904)(193.22058959,222.06230574)(193.90686935,222.42645044)
\lineto(193.90686935,220.59873452)
\curveto(193.21358685,220.30461901)(192.50280915,220.08053146)(191.77453412,219.9264712)
\curveto(191.04624007,219.77241108)(190.30745143,219.69538099)(189.55816597,219.69538068)
\curveto(187.68142544,219.69538099)(186.19334405,220.24159439)(185.09391733,221.33402253)
\curveto(184.0014877,222.42644801)(183.45527429,223.9040253)(183.45527548,225.76675883)
\curveto(183.45527429,227.69250545)(183.97347675,229.21910189)(185.00988442,230.34655272)
\curveto(186.05328933,231.4809856)(187.45733789,232.04820722)(189.22203431,232.04821926)
\curveto(190.80464568,232.04820722)(192.05463405,231.53700749)(192.97200316,230.51461856)
\curveto(193.89635361,229.49921133)(194.35853418,228.11617098)(194.35854627,226.36549335)
\moveto(192.42578921,226.93271553)
\curveto(192.41177358,227.99012173)(192.1141573,228.83395141)(191.53293948,229.46420711)
\curveto(190.95870587,230.09444388)(190.19540765,230.409567)(189.24304254,230.40957741)
\curveto(188.16461422,230.409567)(187.29977633,230.10494799)(186.64852627,229.49571946)
\curveto(186.00427018,228.88647193)(185.63312517,228.02863678)(185.53509014,226.92221142)
\lineto(192.42578921,226.93271553)
}
}
{
\newrgbcolor{curcolor}{0 0 0}
\pscustom[linestyle=none,fillstyle=solid,fillcolor=curcolor]
{
}
}
{
\newrgbcolor{curcolor}{0 0 0}
\pscustom[linestyle=none,fillstyle=solid,fillcolor=curcolor]
{
\newpath
\moveto(212.12099928,229.97890872)
\lineto(212.12099928,236.34440207)
\lineto(214.05375634,236.34440207)
\lineto(214.05375634,220)
\lineto(212.12099928,220)
\lineto(212.12099928,221.76469123)
\curveto(211.71483083,221.06441587)(211.20012973,220.54271204)(210.57689446,220.19957817)
\curveto(209.96064547,219.86344665)(209.21835546,219.69538099)(208.3500222,219.69538068)
\curveto(206.9284608,219.69538099)(205.769508,220.2626026)(204.87316032,221.39704722)
\curveto(203.98381033,222.53148905)(203.5391366,224.02307181)(203.53913778,225.87179997)
\curveto(203.5391366,227.72051639)(203.98381033,229.21209915)(204.87316032,230.34655272)
\curveto(205.769508,231.4809856)(206.9284608,232.04820722)(208.3500222,232.04821926)
\curveto(209.21835546,232.04820722)(209.96064547,231.87664019)(210.57689446,231.53351766)
\curveto(211.20012973,231.1973748)(211.71483083,230.67917234)(212.12099928,229.97890872)
\moveto(205.53491953,225.87179997)
\curveto(205.53491634,224.4502387)(205.82552989,223.33330232)(206.40676103,222.52098747)
\curveto(206.99498679,221.71567031)(207.80030143,221.31301299)(208.82270735,221.31301431)
\curveto(209.84510033,221.31301299)(210.65041496,221.71567031)(211.23865367,222.52098747)
\curveto(211.8268746,223.33330232)(212.12098951,224.4502387)(212.12099928,225.87179997)
\curveto(212.12098951,227.2933495)(211.8268746,228.40678452)(211.23865367,229.21210837)
\curveto(210.65041496,230.02441652)(209.84510033,230.43057521)(208.82270735,230.43058564)
\curveto(207.80030143,230.43057521)(206.99498679,230.02441652)(206.40676103,229.21210837)
\curveto(205.82552989,228.40678452)(205.53491634,227.2933495)(205.53491953,225.87179997)
}
}
{
\newrgbcolor{curcolor}{0 0 0}
\pscustom[linestyle=none,fillstyle=solid,fillcolor=curcolor]
{
\newpath
\moveto(219.86253258,235.68264286)
\lineto(219.86253258,229.85285935)
\lineto(218.07683313,229.85285935)
\lineto(218.07683313,235.68264286)
\lineto(219.86253258,235.68264286)
}
}
{
\newrgbcolor{curcolor}{0 0 0}
\pscustom[linestyle=none,fillstyle=solid,fillcolor=curcolor]
{
\newpath
\moveto(234.02208234,226.36549335)
\lineto(234.02208234,225.42012305)
\lineto(225.13560152,225.42012305)
\curveto(225.21963115,224.0895978)(225.6187871,223.07420109)(226.33307057,222.37392986)
\curveto(227.05434797,221.68065663)(228.05573921,221.3340212)(229.3372473,221.33402253)
\curveto(230.07952991,221.3340212)(230.79731034,221.42505677)(231.49059076,221.60712951)
\curveto(232.1908548,221.78919904)(232.88412566,222.06230574)(233.57040542,222.42645044)
\lineto(233.57040542,220.59873452)
\curveto(232.87712292,220.30461901)(232.16634522,220.08053146)(231.43807019,219.9264712)
\curveto(230.70977614,219.77241108)(229.9709875,219.69538099)(229.22170204,219.69538068)
\curveto(227.34496151,219.69538099)(225.85688012,220.24159439)(224.7574534,221.33402253)
\curveto(223.66502377,222.42644801)(223.11881036,223.9040253)(223.11881155,225.76675883)
\curveto(223.11881036,227.69250545)(223.63701282,229.21910189)(224.67342049,230.34655272)
\curveto(225.7168254,231.4809856)(227.12087396,232.04820722)(228.88557038,232.04821926)
\curveto(230.46818175,232.04820722)(231.71817012,231.53700749)(232.63553923,230.51461856)
\curveto(233.55988968,229.49921133)(234.02207025,228.11617098)(234.02208234,226.36549335)
\moveto(232.08932528,226.93271553)
\curveto(232.07530965,227.99012173)(231.77769338,228.83395141)(231.19647556,229.46420711)
\curveto(230.62224194,230.09444388)(229.85894372,230.409567)(228.90657861,230.40957741)
\curveto(227.8281503,230.409567)(226.9633124,230.10494799)(226.31206234,229.49571946)
\curveto(225.66780625,228.88647193)(225.29666125,228.02863678)(225.19862621,226.92221142)
\lineto(232.08932528,226.93271553)
}
}
{
\newrgbcolor{curcolor}{0 0 0}
\pscustom[linestyle=none,fillstyle=solid,fillcolor=curcolor]
{
\newpath
\moveto(246.97365497,227.10078136)
\lineto(246.97365497,220)
\lineto(245.04089791,220)
\lineto(245.04089791,227.03775668)
\curveto(245.04088804,228.15118466)(244.82380322,228.98451023)(244.38964281,229.53773591)
\curveto(243.95546396,230.09094252)(243.30420952,230.36755059)(242.43587753,230.36756095)
\curveto(241.3924626,230.36755059)(240.56964113,230.03492063)(239.96741063,229.36967008)
\curveto(239.36517054,228.7044008)(239.0640529,227.79754649)(239.06405679,226.64910444)
\lineto(239.06405679,220)
\lineto(237.12079562,220)
\lineto(237.12079562,231.76460817)
\lineto(239.06405679,231.76460817)
\lineto(239.06405679,229.93689226)
\curveto(239.52623347,230.64415866)(240.06894551,231.17286522)(240.69219453,231.52301354)
\curveto(241.32243524,231.87313882)(242.04721841,232.04820722)(242.86654622,232.04821926)
\curveto(244.21806656,232.04820722)(245.24046601,231.62804306)(245.93374764,230.78772553)
\curveto(246.62700773,229.95438916)(246.97364316,228.725409)(246.97365497,227.10078136)
}
}
{
\newrgbcolor{curcolor}{0 0 0}
\pscustom[linestyle=none,fillstyle=solid,fillcolor=curcolor]
{
\newpath
\moveto(260.62899707,227.10078136)
\lineto(260.62899707,220)
\lineto(258.69624001,220)
\lineto(258.69624001,227.03775668)
\curveto(258.69623014,228.15118466)(258.47914532,228.98451023)(258.04498492,229.53773591)
\curveto(257.61080606,230.09094252)(256.95955162,230.36755059)(256.09121963,230.36756095)
\curveto(255.0478047,230.36755059)(254.22498323,230.03492063)(253.62275274,229.36967008)
\curveto(253.02051264,228.7044008)(252.719395,227.79754649)(252.71939889,226.64910444)
\lineto(252.71939889,220)
\lineto(250.77613772,220)
\lineto(250.77613772,231.76460817)
\lineto(252.71939889,231.76460817)
\lineto(252.71939889,229.93689226)
\curveto(253.18157557,230.64415866)(253.72428761,231.17286522)(254.34753663,231.52301354)
\curveto(254.97777734,231.87313882)(255.70256052,232.04820722)(256.52188832,232.04821926)
\curveto(257.87340866,232.04820722)(258.89580811,231.62804306)(259.58908974,230.78772553)
\curveto(260.28234983,229.95438916)(260.62898526,228.725409)(260.62899707,227.10078136)
}
}
{
\newrgbcolor{curcolor}{0 0 0}
\pscustom[linestyle=none,fillstyle=solid,fillcolor=curcolor]
{
\newpath
\moveto(274.56795789,226.36549335)
\lineto(274.56795789,225.42012305)
\lineto(265.68147707,225.42012305)
\curveto(265.7655067,224.0895978)(266.16466265,223.07420109)(266.87894612,222.37392986)
\curveto(267.60022352,221.68065663)(268.60161476,221.3340212)(269.88312285,221.33402253)
\curveto(270.62540546,221.3340212)(271.34318589,221.42505677)(272.03646631,221.60712951)
\curveto(272.73673035,221.78919904)(273.43000121,222.06230574)(274.11628097,222.42645044)
\lineto(274.11628097,220.59873452)
\curveto(273.42299847,220.30461901)(272.71222077,220.08053146)(271.98394574,219.9264712)
\curveto(271.25565169,219.77241108)(270.51686305,219.69538099)(269.76757759,219.69538068)
\curveto(267.89083706,219.69538099)(266.40275567,220.24159439)(265.30332895,221.33402253)
\curveto(264.21089932,222.42644801)(263.66468591,223.9040253)(263.6646871,225.76675883)
\curveto(263.66468591,227.69250545)(264.18288837,229.21910189)(265.21929604,230.34655272)
\curveto(266.26270095,231.4809856)(267.66674951,232.04820722)(269.43144593,232.04821926)
\curveto(271.0140573,232.04820722)(272.26404567,231.53700749)(273.18141478,230.51461856)
\curveto(274.10576523,229.49921133)(274.5679458,228.11617098)(274.56795789,226.36549335)
\moveto(272.63520083,226.93271553)
\curveto(272.6211852,227.99012173)(272.32356893,228.83395141)(271.74235111,229.46420711)
\curveto(271.16811749,230.09444388)(270.40481927,230.409567)(269.45245416,230.40957741)
\curveto(268.37402584,230.409567)(267.50918795,230.10494799)(266.85793789,229.49571946)
\curveto(266.2136818,228.88647193)(265.8425368,228.02863678)(265.74450176,226.92221142)
\lineto(272.63520083,226.93271553)
}
}
{
\newrgbcolor{curcolor}{0 0 0}
\pscustom[linestyle=none,fillstyle=solid,fillcolor=curcolor]
{
\newpath
\moveto(286.89978776,229.50622357)
\curveto(287.38296536,230.37455332)(287.96069107,231.01530366)(288.63296665,231.42847651)
\curveto(289.30521638,231.84162651)(290.09652554,232.04820722)(291.00689651,232.04821926)
\curveto(292.23236001,232.04820722)(293.17772936,231.61753895)(293.84300741,230.75621319)
\curveto(294.5082492,229.90186865)(294.84087915,228.68339259)(294.84089828,227.10078136)
\lineto(294.84089828,220)
\lineto(292.89763711,220)
\lineto(292.89763711,227.03775668)
\curveto(292.89761993,228.16519013)(292.69804195,229.00201707)(292.29890259,229.54824003)
\curveto(291.89973005,230.09444388)(291.29049202,230.36755059)(290.47118667,230.36756095)
\curveto(289.46978067,230.36755059)(288.67847151,230.03492063)(288.09725681,229.36967008)
\curveto(287.51601734,228.7044008)(287.2254038,227.79754649)(287.22541531,226.64910444)
\lineto(287.22541531,220)
\lineto(285.28215414,220)
\lineto(285.28215414,227.03775668)
\curveto(285.28214457,228.17219286)(285.0825666,229.00901981)(284.68341962,229.54824003)
\curveto(284.2842547,230.09444388)(283.66801393,230.36755059)(282.83469548,230.36756095)
\curveto(281.84730258,230.36755059)(281.06299616,230.03141926)(280.48177384,229.35916597)
\curveto(279.90054199,228.69389669)(279.60992844,227.79054375)(279.60993234,226.64910444)
\lineto(279.60993234,220)
\lineto(277.66667117,220)
\lineto(277.66667117,231.76460817)
\lineto(279.60993234,231.76460817)
\lineto(279.60993234,229.93689226)
\curveto(280.05110081,230.65816413)(280.57980737,231.19037206)(281.19605362,231.53351766)
\curveto(281.8122889,231.87664019)(282.54407481,232.04820722)(283.39141354,232.04821926)
\curveto(284.24573965,232.04820722)(284.97052282,231.8311224)(285.56576523,231.39696417)
\curveto(286.16799067,230.96278314)(286.6126644,230.33253691)(286.89978776,229.50622357)
}
}
{
\newrgbcolor{curcolor}{0 0 0}
\pscustom[linestyle=none,fillstyle=solid,fillcolor=curcolor]
{
\newpath
\moveto(298.70641701,231.76460817)
\lineto(300.63917407,231.76460817)
\lineto(300.63917407,220)
\lineto(298.70641701,220)
\lineto(298.70641701,231.76460817)
\moveto(298.70641701,236.34440207)
\lineto(300.63917407,236.34440207)
\lineto(300.63917407,233.89694341)
\lineto(298.70641701,233.89694341)
\lineto(298.70641701,236.34440207)
}
}
{
\newrgbcolor{curcolor}{0 0 0}
\pscustom[linestyle=none,fillstyle=solid,fillcolor=curcolor]
{
\newpath
\moveto(312.17269383,231.4179724)
\lineto(312.17269383,229.59025649)
\curveto(311.6264709,229.87035633)(311.05924929,230.08043841)(310.4710273,230.22050335)
\curveto(309.88278965,230.36054785)(309.27355162,230.43057521)(308.64331138,230.43058564)
\curveto(307.68393056,230.43057521)(306.96264875,230.28351775)(306.47946381,229.98941283)
\curveto(306.00327393,229.69528793)(305.76518091,229.25411557)(305.76518403,228.66589441)
\curveto(305.76518091,228.21771065)(305.93674794,227.86407248)(306.27988563,227.60497886)
\curveto(306.62301606,227.35287276)(307.31278555,227.11127837)(308.34919618,226.88019496)
\lineto(309.01095539,226.73313736)
\curveto(310.38348527,226.43901571)(311.35686557,226.02235292)(311.9310992,225.48314774)
\curveto(312.512317,224.95093432)(312.80293054,224.20514294)(312.8029407,223.24577136)
\curveto(312.80293054,222.15334131)(312.36876091,221.28850342)(311.50043051,220.6512551)
\curveto(310.63908513,220.01400547)(309.45212139,219.69538099)(307.93953571,219.69538068)
\curveto(307.30928418,219.69538099)(306.651027,219.75840561)(305.9647622,219.88445474)
\curveto(305.28549349,220.00350137)(304.56771306,220.1855725)(303.81141874,220.43066869)
\lineto(303.81141874,222.42645044)
\curveto(304.52569664,222.055303)(305.2294716,221.77519357)(305.92274574,221.58612128)
\curveto(306.61601332,221.40404856)(307.30228145,221.31301299)(307.98155217,221.31301431)
\curveto(308.89190251,221.31301299)(309.59217611,221.46707318)(310.08237506,221.77519534)
\curveto(310.57255914,222.09031668)(310.8176549,222.53148905)(310.81766307,223.09871376)
\curveto(310.8176549,223.62391586)(310.63908513,224.02657318)(310.28195324,224.30668692)
\curveto(309.9318088,224.58679205)(309.15800648,224.85639739)(307.96054394,225.11550373)
\lineto(307.28828062,225.27306545)
\curveto(306.09080813,225.52515867)(305.22597024,225.91030915)(304.69376435,226.42851804)
\curveto(304.16155437,226.95371681)(303.8954504,227.67149724)(303.89545165,228.5818615)
\curveto(303.8954504,229.6882852)(304.28760362,230.54261899)(305.07191247,231.14486542)
\curveto(305.85621647,231.74708957)(306.96965149,232.04820722)(308.41222086,232.04821926)
\curveto(309.12649417,232.04820722)(309.79875682,231.9956867)(310.42901084,231.89065755)
\curveto(311.05924929,231.78560462)(311.64047637,231.62804306)(312.17269383,231.4179724)
}
}
{
\newrgbcolor{curcolor}{0 0 0}
\pscustom[linestyle=none,fillstyle=solid,fillcolor=curcolor,opacity=0.11935484]
{
\newpath
\moveto(333.90655899,240)
\lineto(419.49635696,240)
\curveto(425.31537098,240)(429.99999237,235.80657851)(429.99999237,230.59770966)
\lineto(429.99999237,219.50483704)
\curveto(429.99999237,214.29596819)(425.31537098,210.10254669)(419.49635696,210.10254669)
\lineto(333.90655899,210.10254669)
\curveto(328.08754498,210.10254669)(323.40292358,214.29596819)(323.40292358,219.50483704)
\lineto(323.40292358,230.59770966)
\curveto(323.40292358,235.80657851)(328.08754498,240)(333.90655899,240)
\closepath
}
}
{
\newrgbcolor{curcolor}{0 0 0}
\pscustom[linewidth=2,linecolor=curcolor]
{
\newpath
\moveto(333.90655899,240)
\lineto(419.49635696,240)
\curveto(425.31537098,240)(429.99999237,235.80657851)(429.99999237,230.59770966)
\lineto(429.99999237,219.50483704)
\curveto(429.99999237,214.29596819)(425.31537098,210.10254669)(419.49635696,210.10254669)
\lineto(333.90655899,210.10254669)
\curveto(328.08754498,210.10254669)(323.40292358,214.29596819)(323.40292358,219.50483704)
\lineto(323.40292358,230.59770966)
\curveto(323.40292358,235.80657851)(328.08754498,240)(333.90655899,240)
\closepath
}
}
{
\newrgbcolor{curcolor}{0 0 0}
\pscustom[linestyle=none,fillstyle=solid,fillcolor=curcolor,opacity=0.11935484]
{
\newpath
\moveto(312.70388412,189.96066284)
\lineto(417.25014114,189.96066284)
\curveto(424.35795159,189.96066284)(430.0801239,185.79414402)(430.0801239,180.6186924)
\lineto(430.0801239,169.59698677)
\curveto(430.0801239,164.42153514)(424.35795159,160.25501633)(417.25014114,160.25501633)
\lineto(312.70388412,160.25501633)
\curveto(305.59607368,160.25501633)(299.87390137,164.42153514)(299.87390137,169.59698677)
\lineto(299.87390137,180.6186924)
\curveto(299.87390137,185.79414402)(305.59607368,189.96066284)(312.70388412,189.96066284)
\closepath
}
}
{
\newrgbcolor{curcolor}{0 0 0}
\pscustom[linewidth=2,linecolor=curcolor]
{
\newpath
\moveto(312.70388412,189.96066284)
\lineto(417.25014114,189.96066284)
\curveto(424.35795159,189.96066284)(430.0801239,185.79414402)(430.0801239,180.6186924)
\lineto(430.0801239,169.59698677)
\curveto(430.0801239,164.42153514)(424.35795159,160.25501633)(417.25014114,160.25501633)
\lineto(312.70388412,160.25501633)
\curveto(305.59607368,160.25501633)(299.87390137,164.42153514)(299.87390137,169.59698677)
\lineto(299.87390137,180.6186924)
\curveto(299.87390137,185.79414402)(305.59607368,189.96066284)(312.70388412,189.96066284)
\closepath
}
}
{
\newrgbcolor{curcolor}{1 1 1}
\pscustom[linestyle=none,fillstyle=solid,fillcolor=curcolor]
{
\newpath
\moveto(329.26973534,49.81427002)
\lineto(420.7846241,49.81427002)
\curveto(425.93425914,49.81427002)(430.0799942,46.91140156)(430.0799942,43.30559635)
\lineto(430.0799942,26.42860699)
\curveto(430.0799942,22.82280177)(425.93425914,19.91993332)(420.7846241,19.91993332)
\lineto(329.26973534,19.91993332)
\curveto(324.1201003,19.91993332)(319.97436523,22.82280177)(319.97436523,26.42860699)
\lineto(319.97436523,43.30559635)
\curveto(319.97436523,46.91140156)(324.1201003,49.81427002)(329.26973534,49.81427002)
\closepath
}
}
{
\newrgbcolor{curcolor}{0 0 0}
\pscustom[linewidth=2,linecolor=curcolor]
{
\newpath
\moveto(329.26973534,49.81427002)
\lineto(420.7846241,49.81427002)
\curveto(425.93425914,49.81427002)(430.0799942,46.91140156)(430.0799942,43.30559635)
\lineto(430.0799942,26.42860699)
\curveto(430.0799942,22.82280177)(425.93425914,19.91993332)(420.7846241,19.91993332)
\lineto(329.26973534,19.91993332)
\curveto(324.1201003,19.91993332)(319.97436523,22.82280177)(319.97436523,26.42860699)
\lineto(319.97436523,43.30559635)
\curveto(319.97436523,46.91140156)(324.1201003,49.81427002)(329.26973534,49.81427002)
\closepath
}
}
{
\newrgbcolor{curcolor}{0 0 0}
\pscustom[linestyle=none,fillstyle=solid,fillcolor=curcolor]
{
\newpath
\moveto(342.35546875,47.49609375)
\lineto(344.72265625,47.49609375)
\lineto(344.72265625,31.9921875)
\lineto(353.2421875,31.9921875)
\lineto(353.2421875,30)
\lineto(342.35546875,30)
\lineto(342.35546875,47.49609375)
}
}
{
\newrgbcolor{curcolor}{0 0 0}
\pscustom[linestyle=none,fillstyle=solid,fillcolor=curcolor]
{
\newpath
\moveto(361.5859375,36.59765625)
\curveto(359.84374352,36.59764965)(358.63671347,36.3984311)(357.96484375,36)
\curveto(357.29296482,35.6015569)(356.95702765,34.92187008)(356.95703125,33.9609375)
\curveto(356.95702765,33.1953093)(357.2070274,32.58593491)(357.70703125,32.1328125)
\curveto(358.21483889,31.68749831)(358.90233821,31.46484229)(359.76953125,31.46484375)
\curveto(360.96483614,31.46484229)(361.92186644,31.88671686)(362.640625,32.73046875)
\curveto(363.36717749,33.58202767)(363.73045838,34.71093279)(363.73046875,36.1171875)
\lineto(363.73046875,36.59765625)
\lineto(361.5859375,36.59765625)
\moveto(365.88671875,37.48828125)
\lineto(365.88671875,30)
\lineto(363.73046875,30)
\lineto(363.73046875,31.9921875)
\curveto(363.23827137,31.1953113)(362.62499073,30.60546814)(361.890625,30.22265625)
\curveto(361.1562422,29.8476564)(360.2578056,29.66015659)(359.1953125,29.66015625)
\curveto(357.85155801,29.66015659)(356.78124658,30.03515621)(355.984375,30.78515625)
\curveto(355.19531066,31.54296721)(354.80077981,32.55468495)(354.80078125,33.8203125)
\curveto(354.80077981,35.2968697)(355.29296682,36.41014984)(356.27734375,37.16015625)
\curveto(357.26952734,37.91014834)(358.74608836,38.28514796)(360.70703125,38.28515625)
\lineto(363.73046875,38.28515625)
\lineto(363.73046875,38.49609375)
\curveto(363.73045838,39.48827176)(363.40233371,40.253896)(362.74609375,40.79296875)
\curveto(362.09764751,41.33983241)(361.18358593,41.61326964)(360.00390625,41.61328125)
\curveto(359.25390036,41.61326964)(358.52343234,41.52342598)(357.8125,41.34375)
\curveto(357.10155876,41.16405134)(356.41796569,40.89452036)(355.76171875,40.53515625)
\lineto(355.76171875,42.52734375)
\curveto(356.55077806,42.83201842)(357.31640229,43.05858069)(358.05859375,43.20703125)
\curveto(358.80077581,43.36326789)(359.52343134,43.44139281)(360.2265625,43.44140625)
\curveto(362.12499123,43.44139281)(363.54295857,42.9492058)(364.48046875,41.96484375)
\curveto(365.41795669,40.98045777)(365.88670622,39.48827176)(365.88671875,37.48828125)
}
}
{
\newrgbcolor{curcolor}{0 0 0}
\pscustom[linestyle=none,fillstyle=solid,fillcolor=curcolor]
{
\newpath
\moveto(381.25,37.921875)
\lineto(381.25,30)
\lineto(379.09375,30)
\lineto(379.09375,37.8515625)
\curveto(379.09373898,39.09374091)(378.85155173,40.02342748)(378.3671875,40.640625)
\curveto(377.8828027,41.25780124)(377.15624092,41.56639468)(376.1875,41.56640625)
\curveto(375.02343055,41.56639468)(374.10546272,41.1953013)(373.43359375,40.453125)
\curveto(372.76171407,39.71092779)(372.4257769,38.69921005)(372.42578125,37.41796875)
\lineto(372.42578125,30)
\lineto(370.2578125,30)
\lineto(370.2578125,43.125)
\lineto(372.42578125,43.125)
\lineto(372.42578125,41.0859375)
\curveto(372.94140139,41.87498812)(373.54686953,42.46483129)(374.2421875,42.85546875)
\curveto(374.94530563,43.2460805)(375.75389857,43.44139281)(376.66796875,43.44140625)
\curveto(378.17577115,43.44139281)(379.31639501,42.97264328)(380.08984375,42.03515625)
\curveto(380.86326846,41.10545764)(381.24998683,39.73436527)(381.25,37.921875)
}
}
{
\newrgbcolor{curcolor}{0 0 0}
\pscustom[linestyle=none,fillstyle=solid,fillcolor=curcolor]
{
\newpath
\moveto(395.01953125,42.62109375)
\lineto(395.01953125,40.60546875)
\curveto(394.41014515,40.94139531)(393.79686452,41.19139506)(393.1796875,41.35546875)
\curveto(392.57030324,41.52733222)(391.95311636,41.61326964)(391.328125,41.61328125)
\curveto(389.92968088,41.61326964)(388.84374447,41.16795758)(388.0703125,40.27734375)
\curveto(387.29687102,39.39452186)(386.91015265,38.1523356)(386.91015625,36.55078125)
\curveto(386.91015265,34.9492138)(387.29687102,33.7031213)(388.0703125,32.8125)
\curveto(388.84374447,31.92968557)(389.92968088,31.48827976)(391.328125,31.48828125)
\curveto(391.95311636,31.48827976)(392.57030324,31.57031093)(393.1796875,31.734375)
\curveto(393.79686452,31.90624809)(394.41014515,32.16015409)(395.01953125,32.49609375)
\lineto(395.01953125,30.50390625)
\curveto(394.41795764,30.22265603)(393.79295827,30.01171874)(393.14453125,29.87109375)
\curveto(392.50389706,29.73046902)(391.82030399,29.66015659)(391.09375,29.66015625)
\curveto(389.1171817,29.66015659)(387.54687077,30.28124972)(386.3828125,31.5234375)
\curveto(385.21874809,32.76562223)(384.63671743,34.44140181)(384.63671875,36.55078125)
\curveto(384.63671743,38.69139756)(385.22265434,40.37498963)(386.39453125,41.6015625)
\curveto(387.57421449,42.82811217)(389.18749413,43.44139281)(391.234375,43.44140625)
\curveto(391.89842891,43.44139281)(392.54686577,43.37108038)(393.1796875,43.23046875)
\curveto(393.8124895,43.09764315)(394.42577014,42.89451836)(395.01953125,42.62109375)
}
}
{
\newrgbcolor{curcolor}{0 0 0}
\pscustom[linestyle=none,fillstyle=solid,fillcolor=curcolor]
{
\newpath
\moveto(410.01953125,37.1015625)
\lineto(410.01953125,36.046875)
\lineto(400.10546875,36.046875)
\curveto(400.19921508,34.56249544)(400.64452714,33.42968407)(401.44140625,32.6484375)
\curveto(402.24608804,31.87499813)(403.36327442,31.48827976)(404.79296875,31.48828125)
\curveto(405.62108466,31.48827976)(406.42186511,31.58984216)(407.1953125,31.79296875)
\curveto(407.97655105,31.99609175)(408.74998778,32.30077895)(409.515625,32.70703125)
\lineto(409.515625,30.66796875)
\curveto(408.74217529,30.33984341)(407.94920733,30.08984366)(407.13671875,29.91796875)
\curveto(406.32420896,29.746094)(405.49999103,29.66015659)(404.6640625,29.66015625)
\curveto(402.57030646,29.66015659)(400.91015187,30.26953098)(399.68359375,31.48828125)
\curveto(398.46484182,32.70702854)(397.85546743,34.35546439)(397.85546875,36.43359375)
\curveto(397.85546743,38.58202267)(398.43359185,40.28514596)(399.58984375,41.54296875)
\curveto(400.75390203,42.80858094)(402.32030671,43.44139281)(404.2890625,43.44140625)
\curveto(406.05467798,43.44139281)(407.44920783,42.87108088)(408.47265625,41.73046875)
\curveto(409.50389328,40.59764565)(410.01951776,39.05467845)(410.01953125,37.1015625)
\moveto(407.86328125,37.734375)
\curveto(407.84764493,38.91405359)(407.51561402,39.85545889)(406.8671875,40.55859375)
\curveto(406.2265528,41.26170749)(405.37499116,41.61326964)(404.3125,41.61328125)
\curveto(403.10936842,41.61326964)(402.14452564,41.27342623)(401.41796875,40.59375)
\curveto(400.69921458,39.91405259)(400.2851525,38.95702229)(400.17578125,37.72265625)
\lineto(407.86328125,37.734375)
}
}
{
\newrgbcolor{curcolor}{0 0 0}
\pscustom[linestyle=none,fillstyle=solid,fillcolor=curcolor]
{
\newpath
\moveto(421.1640625,41.109375)
\curveto(420.92186537,41.24998875)(420.65624064,41.35155115)(420.3671875,41.4140625)
\curveto(420.08592871,41.48436352)(419.77342902,41.51951973)(419.4296875,41.51953125)
\curveto(418.21093059,41.51951973)(417.27343152,41.12108263)(416.6171875,40.32421875)
\curveto(415.96874533,39.53514671)(415.6445269,38.3984291)(415.64453125,36.9140625)
\lineto(415.64453125,30)
\lineto(413.4765625,30)
\lineto(413.4765625,43.125)
\lineto(415.64453125,43.125)
\lineto(415.64453125,41.0859375)
\curveto(416.09765145,41.88280062)(416.68749461,42.47264378)(417.4140625,42.85546875)
\curveto(418.14061816,43.2460805)(419.02342977,43.44139281)(420.0625,43.44140625)
\curveto(420.21092859,43.44139281)(420.37499092,43.42967407)(420.5546875,43.40625)
\curveto(420.73436556,43.39061161)(420.93358411,43.36326789)(421.15234375,43.32421875)
\lineto(421.1640625,41.109375)
}
}
{
\newrgbcolor{curcolor}{1 1 1}
\pscustom[linestyle=none,fillstyle=solid,fillcolor=curcolor]
{
\newpath
\moveto(139.20729923,49.62908936)
\lineto(230.67112637,49.62908936)
\curveto(235.81788884,49.62908936)(239.96131134,46.74393137)(239.96131134,43.16012526)
\lineto(239.96131134,26.38610125)
\curveto(239.96131134,22.80229513)(235.81788884,19.91713715)(230.67112637,19.91713715)
\lineto(139.20729923,19.91713715)
\curveto(134.06053676,19.91713715)(129.91711426,22.80229513)(129.91711426,26.38610125)
\lineto(129.91711426,43.16012526)
\curveto(129.91711426,46.74393137)(134.06053676,49.62908936)(139.20729923,49.62908936)
\closepath
}
}
{
\newrgbcolor{curcolor}{0 0 0}
\pscustom[linewidth=1.72500002,linecolor=curcolor]
{
\newpath
\moveto(139.20729923,49.62908936)
\lineto(230.67112637,49.62908936)
\curveto(235.81788884,49.62908936)(239.96131134,46.74393137)(239.96131134,43.16012526)
\lineto(239.96131134,26.38610125)
\curveto(239.96131134,22.80229513)(235.81788884,19.91713715)(230.67112637,19.91713715)
\lineto(139.20729923,19.91713715)
\curveto(134.06053676,19.91713715)(129.91711426,22.80229513)(129.91711426,26.38610125)
\lineto(129.91711426,43.16012526)
\curveto(129.91711426,46.74393137)(134.06053676,49.62908936)(139.20729923,49.62908936)
\closepath
}
}
{
\newrgbcolor{curcolor}{0 0 0}
\pscustom[linestyle=none,fillstyle=solid,fillcolor=curcolor]
{
\newpath
\moveto(160.65234375,38.203125)
\curveto(161.16014509,38.03124197)(161.6523321,37.66405484)(162.12890625,37.1015625)
\curveto(162.61326864,36.53905596)(163.09764315,35.76561923)(163.58203125,34.78125)
\lineto(165.984375,30)
\lineto(163.44140625,30)
\lineto(161.203125,34.48828125)
\curveto(160.62498938,35.66015059)(160.06248994,36.43749356)(159.515625,36.8203125)
\curveto(158.97655352,37.2031178)(158.23827301,37.39452386)(157.30078125,37.39453125)
\lineto(154.72265625,37.39453125)
\lineto(154.72265625,30)
\lineto(152.35546875,30)
\lineto(152.35546875,47.49609375)
\lineto(157.69921875,47.49609375)
\curveto(159.69920905,47.49607625)(161.19139506,47.07810792)(162.17578125,46.2421875)
\curveto(163.16014309,45.40623459)(163.6523301,44.14451711)(163.65234375,42.45703125)
\curveto(163.6523301,41.35545739)(163.39451786,40.44139581)(162.87890625,39.71484375)
\curveto(162.37108138,38.98827226)(161.62889462,38.48436652)(160.65234375,38.203125)
\moveto(154.72265625,45.55078125)
\lineto(154.72265625,39.33984375)
\lineto(157.69921875,39.33984375)
\curveto(158.83983491,39.33983441)(159.69920905,39.6015529)(160.27734375,40.125)
\curveto(160.86327039,40.65623934)(161.15623884,41.43358232)(161.15625,42.45703125)
\curveto(161.15623884,43.48045527)(160.86327039,44.24998575)(160.27734375,44.765625)
\curveto(159.69920905,45.28904721)(158.83983491,45.5507657)(157.69921875,45.55078125)
\lineto(154.72265625,45.55078125)
}
}
{
\newrgbcolor{curcolor}{0 0 0}
\pscustom[linestyle=none,fillstyle=solid,fillcolor=curcolor]
{
\newpath
\moveto(179.09765625,37.1015625)
\lineto(179.09765625,36.046875)
\lineto(169.18359375,36.046875)
\curveto(169.27734008,34.56249544)(169.72265214,33.42968407)(170.51953125,32.6484375)
\curveto(171.32421304,31.87499813)(172.44139942,31.48827976)(173.87109375,31.48828125)
\curveto(174.69920966,31.48827976)(175.49999011,31.58984216)(176.2734375,31.79296875)
\curveto(177.05467605,31.99609175)(177.82811278,32.30077895)(178.59375,32.70703125)
\lineto(178.59375,30.66796875)
\curveto(177.82030029,30.33984341)(177.02733233,30.08984366)(176.21484375,29.91796875)
\curveto(175.40233396,29.746094)(174.57811603,29.66015659)(173.7421875,29.66015625)
\curveto(171.64843146,29.66015659)(169.98827687,30.26953098)(168.76171875,31.48828125)
\curveto(167.54296682,32.70702854)(166.93359243,34.35546439)(166.93359375,36.43359375)
\curveto(166.93359243,38.58202267)(167.51171685,40.28514596)(168.66796875,41.54296875)
\curveto(169.83202703,42.80858094)(171.39843171,43.44139281)(173.3671875,43.44140625)
\curveto(175.13280298,43.44139281)(176.52733283,42.87108088)(177.55078125,41.73046875)
\curveto(178.58201828,40.59764565)(179.09764276,39.05467845)(179.09765625,37.1015625)
\moveto(176.94140625,37.734375)
\curveto(176.92576993,38.91405359)(176.59373902,39.85545889)(175.9453125,40.55859375)
\curveto(175.3046778,41.26170749)(174.45311616,41.61326964)(173.390625,41.61328125)
\curveto(172.18749342,41.61326964)(171.22265064,41.27342623)(170.49609375,40.59375)
\curveto(169.77733958,39.91405259)(169.3632775,38.95702229)(169.25390625,37.72265625)
\lineto(176.94140625,37.734375)
}
}
{
\newrgbcolor{curcolor}{0 0 0}
\pscustom[linestyle=none,fillstyle=solid,fillcolor=curcolor]
{
\newpath
\moveto(184.76953125,46.8515625)
\lineto(184.76953125,43.125)
\lineto(189.2109375,43.125)
\lineto(189.2109375,41.44921875)
\lineto(184.76953125,41.44921875)
\lineto(184.76953125,34.32421875)
\curveto(184.76952686,33.253903)(184.91405796,32.56640368)(185.203125,32.26171875)
\curveto(185.49999487,31.95702929)(186.09765053,31.8046857)(186.99609375,31.8046875)
\lineto(189.2109375,31.8046875)
\lineto(189.2109375,30)
\lineto(186.99609375,30)
\curveto(185.33202629,30)(184.18358994,30.30859344)(183.55078125,30.92578125)
\curveto(182.91796621,31.5507797)(182.60156027,32.68359107)(182.6015625,34.32421875)
\lineto(182.6015625,41.44921875)
\lineto(181.01953125,41.44921875)
\lineto(181.01953125,43.125)
\lineto(182.6015625,43.125)
\lineto(182.6015625,46.8515625)
\lineto(184.76953125,46.8515625)
}
}
{
\newrgbcolor{curcolor}{0 0 0}
\pscustom[linestyle=none,fillstyle=solid,fillcolor=curcolor]
{
\newpath
\moveto(197.14453125,41.61328125)
\curveto(195.98827506,41.61326964)(195.07421347,41.16014509)(194.40234375,40.25390625)
\curveto(193.73046482,39.35545939)(193.39452765,38.12108563)(193.39453125,36.55078125)
\curveto(193.39452765,34.98046377)(193.72655857,33.74218376)(194.390625,32.8359375)
\curveto(195.06249473,31.93749806)(195.98046257,31.48827976)(197.14453125,31.48828125)
\curveto(198.29296025,31.48827976)(199.20311559,31.94140431)(199.875,32.84765625)
\curveto(200.54686425,33.7539025)(200.88280141,34.98827626)(200.8828125,36.55078125)
\curveto(200.88280141,38.10546064)(200.54686425,39.33592816)(199.875,40.2421875)
\curveto(199.20311559,41.15623884)(198.29296025,41.61326964)(197.14453125,41.61328125)
\moveto(197.14453125,43.44140625)
\curveto(199.01952203,43.44139281)(200.4921768,42.83201842)(201.5625,41.61328125)
\curveto(202.63279966,40.39452086)(203.16795538,38.70702254)(203.16796875,36.55078125)
\curveto(203.16795538,34.40233935)(202.63279966,32.71484104)(201.5625,31.48828125)
\curveto(200.4921768,30.26953098)(199.01952203,29.66015659)(197.14453125,29.66015625)
\curveto(195.26171329,29.66015659)(193.78515226,30.26953098)(192.71484375,31.48828125)
\curveto(191.65234189,32.71484104)(191.12109243,34.40233935)(191.12109375,36.55078125)
\curveto(191.12109243,38.70702254)(191.65234189,40.39452086)(192.71484375,41.61328125)
\curveto(193.78515226,42.83201842)(195.26171329,43.44139281)(197.14453125,43.44140625)
}
}
{
\newrgbcolor{curcolor}{0 0 0}
\pscustom[linestyle=none,fillstyle=solid,fillcolor=curcolor]
{
\newpath
\moveto(206.5078125,35.1796875)
\lineto(206.5078125,43.125)
\lineto(208.6640625,43.125)
\lineto(208.6640625,35.26171875)
\curveto(208.6640583,34.01952723)(208.90624556,33.08593441)(209.390625,32.4609375)
\curveto(209.87499459,31.84374816)(210.60155637,31.53515471)(211.5703125,31.53515625)
\curveto(212.73436673,31.53515471)(213.65233457,31.90624809)(214.32421875,32.6484375)
\curveto(215.00389571,33.39062161)(215.34373913,34.40233935)(215.34375,35.68359375)
\lineto(215.34375,43.125)
\lineto(217.5,43.125)
\lineto(217.5,30)
\lineto(215.34375,30)
\lineto(215.34375,32.015625)
\curveto(214.82030215,31.21874878)(214.21092776,30.62499938)(213.515625,30.234375)
\curveto(212.82811664,29.85156265)(212.02733619,29.66015659)(211.11328125,29.66015625)
\curveto(209.60546361,29.66015659)(208.46093351,30.12890612)(207.6796875,31.06640625)
\curveto(206.89843507,32.00390425)(206.50781046,33.37499662)(206.5078125,35.1796875)
\moveto(211.93359375,43.44140625)
\lineto(211.93359375,43.44140625)
}
}
{
\newrgbcolor{curcolor}{0 0 0}
\pscustom[linestyle=none,fillstyle=solid,fillcolor=curcolor]
{
\newpath
\moveto(229.5703125,41.109375)
\curveto(229.32811538,41.24998875)(229.06249064,41.35155115)(228.7734375,41.4140625)
\curveto(228.49217871,41.48436352)(228.17967902,41.51951973)(227.8359375,41.51953125)
\curveto(226.61718059,41.51951973)(225.67968152,41.12108263)(225.0234375,40.32421875)
\curveto(224.37499533,39.53514671)(224.0507769,38.3984291)(224.05078125,36.9140625)
\lineto(224.05078125,30)
\lineto(221.8828125,30)
\lineto(221.8828125,43.125)
\lineto(224.05078125,43.125)
\lineto(224.05078125,41.0859375)
\curveto(224.50390145,41.88280062)(225.09374461,42.47264378)(225.8203125,42.85546875)
\curveto(226.54686816,43.2460805)(227.42967977,43.44139281)(228.46875,43.44140625)
\curveto(228.61717859,43.44139281)(228.78124092,43.42967407)(228.9609375,43.40625)
\curveto(229.14061556,43.39061161)(229.33983411,43.36326789)(229.55859375,43.32421875)
\lineto(229.5703125,41.109375)
}
}
{
\newrgbcolor{curcolor}{0 0 0}
\pscustom[linewidth=2,linecolor=curcolor,linestyle=dashed,dash=8 8]
{
\newpath
\moveto(299.8133,460.48668)
\lineto(500.32522,499.8184)
}
}
{
\newrgbcolor{curcolor}{0 0 0}
\pscustom[linestyle=none,fillstyle=solid,fillcolor=curcolor]
{
\newpath
\moveto(311.01168865,457.75058911)
\lineto(297.21778492,459.95797069)
\lineto(309.15561533,467.21279855)
\curveto(307.64810037,464.01637106)(308.40958335,460.1970544)(311.01168865,457.75058911)
\lineto(311.01168865,457.75058911)
\closepath
}
}
{
\newrgbcolor{curcolor}{0 0 0}
\pscustom[linewidth=2,linecolor=curcolor,linestyle=dashed,dash=8 8]
{
\newpath
\moveto(140,40)
\lineto(70.11406,39.73606)
}
}
{
\newrgbcolor{curcolor}{0 0 0}
\pscustom[linestyle=none,fillstyle=solid,fillcolor=curcolor]
{
\newpath
\moveto(129.51948821,44.80093475)
\lineto(142.64865594,40.0292193)
\lineto(129.55590443,35.15847191)
\curveto(131.6397373,38.01283536)(131.61296003,41.9072312)(129.51948821,44.80093475)
\lineto(129.51948821,44.80093475)
\closepath
}
}
{
\newrgbcolor{curcolor}{0 0 0}
\pscustom[linewidth=2,linecolor=curcolor,linestyle=dashed,dash=8 8]
{
\newpath
\moveto(220,330)
\lineto(71.232745,330)
}
}
{
\newrgbcolor{curcolor}{0 0 0}
\pscustom[linestyle=none,fillstyle=solid,fillcolor=curcolor]
{
\newpath
\moveto(209.53769464,334.84048224)
\lineto(222.6487474,330.01921591)
\lineto(209.53769392,325.19795064)
\curveto(211.632292,328.04442372)(211.62022288,331.93889292)(209.53769464,334.84048224)
\lineto(209.53769464,334.84048224)
\closepath
}
}
{
\newrgbcolor{curcolor}{0 0 0}
\pscustom[linewidth=2.21782422,linecolor=curcolor,linestyle=dashed,dash=8.87129678 8.87129678]
{
\newpath
\moveto(63.318808,420)
\lineto(160,420)
}
}
{
\newrgbcolor{curcolor}{0 0 0}
\pscustom[linestyle=none,fillstyle=solid,fillcolor=curcolor]
{
\newpath
\moveto(148.39822288,425.36766938)
\lineto(162.93722807,420.02130876)
\lineto(148.39822209,414.67494931)
\curveto(150.72094726,417.83143778)(150.70756367,422.15006184)(148.39822288,425.36766938)
\closepath
}
}
{
\newrgbcolor{curcolor}{0 0 0}
\pscustom[linewidth=2,linecolor=curcolor,linestyle=dashed,dash=8 8]
{
\newpath
\moveto(500,420)
\lineto(399.70808,420)
}
}
{
\newrgbcolor{curcolor}{0 0 0}
\pscustom[linestyle=none,fillstyle=solid,fillcolor=curcolor]
{
\newpath
\moveto(410.17038536,415.15951776)
\lineto(397.0593326,419.98078409)
\lineto(410.17038608,424.80204936)
\curveto(408.075788,421.95557628)(408.08785712,418.06110708)(410.17038536,415.15951776)
\closepath
}
}
{
\newrgbcolor{curcolor}{0 0 0}
\pscustom[linewidth=2,linecolor=curcolor,linestyle=dashed,dash=8 8]
{
\newpath
\moveto(500,270)
\lineto(399.86062,270)
}
}
{
\newrgbcolor{curcolor}{0 0 0}
\pscustom[linestyle=none,fillstyle=solid,fillcolor=curcolor]
{
\newpath
\moveto(410.32292536,265.15951776)
\lineto(397.2118726,269.98078409)
\lineto(410.32292608,274.80204936)
\curveto(408.228328,271.95557628)(408.24039712,268.06110708)(410.32292536,265.15951776)
\closepath
}
}
{
\newrgbcolor{curcolor}{0 0 0}
\pscustom[linewidth=2.12666917,linecolor=curcolor,linestyle=dashed,dash=8.50667644 8.50667644]
{
\newpath
\moveto(500.47505,219.83946)
\lineto(413.6734,219.83946)
}
}
{
\newrgbcolor{curcolor}{0 0 0}
\pscustom[linestyle=none,fillstyle=solid,fillcolor=curcolor]
{
\newpath
\moveto(424.79833112,214.69240783)
\lineto(410.85689528,219.81902706)
\lineto(424.79833189,224.94564516)
\curveto(422.57107331,221.91889189)(422.58390682,217.7777681)(424.79833112,214.69240783)
\closepath
}
}
{
\newrgbcolor{curcolor}{0 0 0}
\pscustom[linewidth=2.10198331,linecolor=curcolor,linestyle=dashed,dash=8.40793346 8.40793346]
{
\newpath
\moveto(499.6268,171.30557)
\lineto(400,170)
}
}
{
\newrgbcolor{curcolor}{0 0 0}
\pscustom[linestyle=none,fillstyle=solid,fillcolor=curcolor]
{
\newpath
\moveto(411.06151301,165.05721353)
\lineto(397.21669222,169.94332835)
\lineto(410.9287202,175.19056371)
\curveto(408.76670473,172.17035504)(408.83302145,168.07781803)(411.06151301,165.05721353)
\closepath
}
}
{
\newrgbcolor{curcolor}{0 0 0}
\pscustom[linewidth=2,linecolor=curcolor,linestyle=dashed,dash=8 8]
{
\newpath
\moveto(490,40)
\lineto(411.0551,41.27788)
}
}
{
\newrgbcolor{curcolor}{0 0 0}
\pscustom[linestyle=none,fillstyle=solid,fillcolor=curcolor]
{
\newpath
\moveto(421.43769241,36.26870077)
\lineto(408.40638854,41.30153623)
\lineto(421.59375622,45.90996935)
\curveto(419.45336271,43.0977699)(419.40239879,39.20361547)(421.43769241,36.26870077)
\closepath
}
}
{
\newrgbcolor{curcolor}{0 0 0}
\pscustom[linestyle=none,fillstyle=solid,fillcolor=curcolor]
{
\newpath
\moveto(52.984375,422.578125)
\curveto(54.49477717,422.25519608)(55.67185933,421.58332175)(56.515625,420.5625)
\curveto(57.3697743,419.54165713)(57.7968572,418.28124172)(57.796875,416.78125)
\curveto(57.7968572,414.47916219)(57.00519133,412.69791397)(55.421875,411.4375)
\curveto(53.83852783,410.17708316)(51.58853008,409.54687545)(48.671875,409.546875)
\curveto(47.69270064,409.54687545)(46.68228498,409.64583369)(45.640625,409.84375)
\curveto(44.60937039,410.03124997)(43.54166312,410.31770802)(42.4375,410.703125)
\lineto(42.4375,413.75)
\curveto(43.31249669,413.23958009)(44.27082906,412.85416381)(45.3125,412.59375)
\curveto(46.35416031,412.333331)(47.44270089,412.2031228)(48.578125,412.203125)
\curveto(50.55728111,412.2031228)(52.06248794,412.59374741)(53.09375,413.375)
\curveto(54.13540253,414.15624584)(54.65623534,415.29166137)(54.65625,416.78125)
\curveto(54.65623534,418.15624184)(54.17186083,419.22915744)(53.203125,420)
\curveto(52.24477942,420.78123922)(50.90623909,421.17186383)(49.1875,421.171875)
\lineto(46.46875,421.171875)
\lineto(46.46875,423.765625)
\lineto(49.3125,423.765625)
\curveto(50.86457247,423.76561123)(52.05207128,424.07290259)(52.875,424.6875)
\curveto(53.69790297,425.31248469)(54.10936089,426.20831713)(54.109375,427.375)
\curveto(54.10936089,428.57289809)(53.68227798,429.48956384)(52.828125,430.125)
\curveto(51.98436302,430.77081256)(50.77082256,431.09372891)(49.1875,431.09375)
\curveto(48.32290834,431.09372891)(47.39582594,430.999979)(46.40625,430.8125)
\curveto(45.41666125,430.62497938)(44.32812067,430.333313)(43.140625,429.9375)
\lineto(43.140625,432.75)
\curveto(44.33853733,433.08331025)(45.45832788,433.33331)(46.5,433.5)
\curveto(47.55207578,433.666643)(48.54165812,433.74997625)(49.46875,433.75)
\curveto(51.86457147,433.74997625)(53.76040291,433.2031018)(55.15625,432.109375)
\curveto(56.55206678,431.02602064)(57.24998275,429.55727211)(57.25,427.703125)
\curveto(57.24998275,426.41144192)(56.88019145,425.31769302)(56.140625,424.421875)
\curveto(55.40102627,423.5364448)(54.34894398,422.92186208)(52.984375,422.578125)
}
}
{
\newrgbcolor{curcolor}{0 0 0}
\pscustom[linestyle=none,fillstyle=solid,fillcolor=curcolor]
{
\newpath
\moveto(512.09375,430.578125)
\lineto(504.125,418.125)
\lineto(512.09375,418.125)
\lineto(512.09375,430.578125)
\moveto(511.265625,433.328125)
\lineto(515.234375,433.328125)
\lineto(515.234375,418.125)
\lineto(518.5625,418.125)
\lineto(518.5625,415.5)
\lineto(515.234375,415.5)
\lineto(515.234375,410)
\lineto(512.09375,410)
\lineto(512.09375,415.5)
\lineto(501.5625,415.5)
\lineto(501.5625,418.546875)
\lineto(511.265625,433.328125)
}
}
{
\newrgbcolor{curcolor}{0 0 0}
\pscustom[linestyle=none,fillstyle=solid,fillcolor=curcolor]
{
\newpath
\moveto(53.453125,343.328125)
\lineto(65.84375,343.328125)
\lineto(65.84375,340.671875)
\lineto(56.34375,340.671875)
\lineto(56.34375,334.953125)
\curveto(56.80207653,335.10935989)(57.26040941,335.22394311)(57.71875,335.296875)
\curveto(58.17707516,335.38019295)(58.63540803,335.42185958)(59.09375,335.421875)
\curveto(61.69790497,335.42185958)(63.76040291,334.70831863)(65.28125,333.28125)
\curveto(66.80206653,331.85415481)(67.56248244,329.92186508)(67.5625,327.484375)
\curveto(67.56248244,324.97395336)(66.78123322,323.02083031)(65.21875,321.625)
\curveto(63.65623634,320.23958309)(61.45311355,319.54687545)(58.609375,319.546875)
\curveto(57.6302007,319.54687545)(56.6302017,319.6302087)(55.609375,319.796875)
\curveto(54.59895373,319.9635417)(53.55207978,320.21354145)(52.46875,320.546875)
\lineto(52.46875,323.71875)
\curveto(53.40624659,323.20833012)(54.37499562,322.82812217)(55.375,322.578125)
\curveto(56.37499363,322.32812267)(57.43228423,322.2031228)(58.546875,322.203125)
\curveto(60.34894798,322.2031228)(61.77602989,322.67708066)(62.828125,323.625)
\curveto(63.88019445,324.57291209)(64.40623559,325.85936914)(64.40625,327.484375)
\curveto(64.40623559,329.10936589)(63.88019445,330.39582294)(62.828125,331.34375)
\curveto(61.77602989,332.29165438)(60.34894798,332.76561223)(58.546875,332.765625)
\curveto(57.7031173,332.76561223)(56.85936814,332.67186233)(56.015625,332.484375)
\curveto(55.18228648,332.2968627)(54.32812067,332.00519633)(53.453125,331.609375)
\lineto(53.453125,343.328125)
}
}
{
\newrgbcolor{curcolor}{0 0 0}
\pscustom[linestyle=none,fillstyle=solid,fillcolor=curcolor]
{
\newpath
\moveto(510.5625,272.921875)
\curveto(509.14582419,272.92186208)(508.02082531,272.43748756)(507.1875,271.46875)
\curveto(506.36457697,270.4999895)(505.95311905,269.17186583)(505.953125,267.484375)
\curveto(505.95311905,265.80728586)(506.36457697,264.47916219)(507.1875,263.5)
\curveto(508.02082531,262.53124747)(509.14582419,262.04687295)(510.5625,262.046875)
\curveto(511.97915469,262.04687295)(513.09894523,262.53124747)(513.921875,263.5)
\curveto(514.75519358,264.47916219)(515.17185983,265.80728586)(515.171875,267.484375)
\curveto(515.17185983,269.17186583)(514.75519358,270.4999895)(513.921875,271.46875)
\curveto(513.09894523,272.43748756)(511.97915469,272.92186208)(510.5625,272.921875)
\moveto(516.828125,282.8125)
\lineto(516.828125,279.9375)
\curveto(516.0364423,280.31247969)(515.23435977,280.59893773)(514.421875,280.796875)
\curveto(513.61977805,280.99477067)(512.82290384,281.09372891)(512.03125,281.09375)
\curveto(509.94790672,281.09372891)(508.35415831,280.39060461)(507.25,278.984375)
\curveto(506.15624384,277.57810742)(505.53124447,275.45310955)(505.375,272.609375)
\curveto(505.98957734,273.51561148)(506.76040991,274.20831913)(507.6875,274.6875)
\curveto(508.61457472,275.17706816)(509.63540703,275.42185958)(510.75,275.421875)
\curveto(513.09373691,275.42185958)(514.94269339,274.70831863)(516.296875,273.28125)
\curveto(517.66144067,271.86457147)(518.34373166,269.93228173)(518.34375,267.484375)
\curveto(518.34373166,265.08853658)(517.63539903,263.1666635)(516.21875,261.71875)
\curveto(514.80206853,260.27083306)(512.91665375,259.54687545)(510.5625,259.546875)
\curveto(507.86457547,259.54687545)(505.80207753,260.57812442)(504.375,262.640625)
\curveto(502.94791372,264.71353695)(502.23437277,267.71353395)(502.234375,271.640625)
\curveto(502.23437277,275.32810967)(503.10937189,278.26560673)(504.859375,280.453125)
\curveto(506.60936839,282.65101902)(508.95832438,283.74997625)(511.90625,283.75)
\curveto(512.69790397,283.74997625)(513.49477817,283.67185133)(514.296875,283.515625)
\curveto(515.10935989,283.35935164)(515.95310905,283.12497688)(516.828125,282.8125)
}
}
{
\newrgbcolor{curcolor}{0 0 0}
\pscustom[linestyle=none,fillstyle=solid,fillcolor=curcolor]
{
\newpath
\moveto(502.625,233.328125)
\lineto(517.625,233.328125)
\lineto(517.625,231.984375)
\lineto(509.15625,210)
\lineto(505.859375,210)
\lineto(513.828125,230.671875)
\lineto(502.625,230.671875)
\lineto(502.625,233.328125)
}
}
{
\newrgbcolor{curcolor}{0 0 0}
\pscustom[linestyle=none,fillstyle=solid,fillcolor=curcolor]
{
\newpath
\moveto(510.171875,171.078125)
\curveto(508.67186633,171.07811392)(507.48957584,170.67707266)(506.625,169.875)
\curveto(505.77082756,169.07290759)(505.34374466,167.96874203)(505.34375,166.5625)
\curveto(505.34374466,165.15624484)(505.77082756,164.05207928)(506.625,163.25)
\curveto(507.48957584,162.44791422)(508.67186633,162.04687295)(510.171875,162.046875)
\curveto(511.67186333,162.04687295)(512.85415381,162.44791422)(513.71875,163.25)
\curveto(514.58331875,164.06249594)(515.01560998,165.1666615)(515.015625,166.5625)
\curveto(515.01560998,167.96874203)(514.58331875,169.07290759)(513.71875,169.875)
\curveto(512.86457047,170.67707266)(511.68227998,171.07811392)(510.171875,171.078125)
\moveto(507.015625,172.421875)
\curveto(505.66145267,172.75519558)(504.60416206,173.38540328)(503.84375,174.3125)
\curveto(503.09374691,175.23956809)(502.71874728,176.3697753)(502.71875,177.703125)
\curveto(502.71874728,179.56768877)(503.38020495,181.04164563)(504.703125,182.125)
\curveto(506.0364523,183.20831013)(507.85936714,183.74997625)(510.171875,183.75)
\curveto(512.49477917,183.74997625)(514.31769402,183.20831013)(515.640625,182.125)
\curveto(516.9635247,181.04164563)(517.62498237,179.56768877)(517.625,177.703125)
\curveto(517.62498237,176.3697753)(517.24477442,175.23956809)(516.484375,174.3125)
\curveto(515.73435927,173.38540328)(514.68748531,172.75519558)(513.34375,172.421875)
\curveto(514.86456847,172.06769627)(516.04685895,171.37498863)(516.890625,170.34375)
\curveto(517.74477392,169.31249069)(518.17185683,168.05207528)(518.171875,166.5625)
\curveto(518.17185683,164.30207903)(517.47914919,162.56770577)(516.09375,161.359375)
\curveto(514.71873528,160.15104152)(512.74477892,159.54687545)(510.171875,159.546875)
\curveto(507.59895073,159.54687545)(505.61978605,160.15104152)(504.234375,161.359375)
\curveto(502.85937214,162.56770577)(502.17187283,164.30207903)(502.171875,166.5625)
\curveto(502.17187283,168.05207528)(502.59895573,169.31249069)(503.453125,170.34375)
\curveto(504.30728736,171.37498863)(505.49478617,172.06769627)(507.015625,172.421875)
\moveto(505.859375,177.40625)
\curveto(505.85936914,176.19790047)(506.23436877,175.25519308)(506.984375,174.578125)
\curveto(507.74478392,173.90102777)(508.80728286,173.56248644)(510.171875,173.5625)
\curveto(511.52603014,173.56248644)(512.58332075,173.90102777)(513.34375,174.578125)
\curveto(514.11456922,175.25519308)(514.4999855,176.19790047)(514.5,177.40625)
\curveto(514.4999855,178.61456472)(514.11456922,179.55727211)(513.34375,180.234375)
\curveto(512.58332075,180.91143742)(511.52603014,181.24997875)(510.171875,181.25)
\curveto(508.80728286,181.24997875)(507.74478392,180.91143742)(506.984375,180.234375)
\curveto(506.23436877,179.55727211)(505.85936914,178.61456472)(505.859375,177.40625)
}
}
{
\newrgbcolor{curcolor}{0 0 0}
\pscustom[linestyle=none,fillstyle=solid,fillcolor=curcolor]
{
\newpath
\moveto(53.515625,30.484375)
\lineto(53.515625,33.359375)
\curveto(54.30728736,32.98437202)(55.10936989,32.69791397)(55.921875,32.5)
\curveto(56.73436827,32.30208103)(57.53124247,32.2031228)(58.3125,32.203125)
\curveto(60.39582294,32.2031228)(61.98436302,32.90103877)(63.078125,34.296875)
\curveto(64.18227748,35.7031193)(64.81248519,37.8333255)(64.96875,40.6875)
\curveto(64.36456897,39.79165687)(63.59894473,39.10415756)(62.671875,38.625)
\curveto(61.74477992,38.14582519)(60.71873928,37.90624209)(59.59375,37.90625)
\curveto(57.26040941,37.90624209)(55.41145292,38.60936639)(54.046875,40.015625)
\curveto(52.69270564,41.43228023)(52.01562298,43.36456997)(52.015625,45.8125)
\curveto(52.01562298,48.20831513)(52.72395561,50.1301882)(54.140625,51.578125)
\curveto(55.55728611,53.02601864)(57.44270089,53.74997625)(59.796875,53.75)
\curveto(62.49477917,53.74997625)(64.55206878,52.71351895)(65.96875,50.640625)
\curveto(67.39581594,48.57810642)(68.10935689,45.57810942)(68.109375,41.640625)
\curveto(68.10935689,37.9635337)(67.23435777,35.02603664)(65.484375,32.828125)
\curveto(63.74477792,30.64062436)(61.40103027,29.54687545)(58.453125,29.546875)
\curveto(57.66145067,29.54687545)(56.85936814,29.62500038)(56.046875,29.78125)
\curveto(55.23436977,29.93750006)(54.39062061,30.17187483)(53.515625,30.484375)
\moveto(59.796875,40.375)
\curveto(61.21353045,40.37498963)(62.333321,40.85936414)(63.15625,41.828125)
\curveto(63.98956934,42.7968622)(64.40623559,44.12498587)(64.40625,45.8125)
\curveto(64.40623559,47.48956584)(63.98956934,48.81248119)(63.15625,49.78125)
\curveto(62.333321,50.76039591)(61.21353045,51.24997875)(59.796875,51.25)
\curveto(58.38019995,51.24997875)(57.25520108,50.76039591)(56.421875,49.78125)
\curveto(55.59895273,48.81248119)(55.18749481,47.48956584)(55.1875,45.8125)
\curveto(55.18749481,44.12498587)(55.59895273,42.7968622)(56.421875,41.828125)
\curveto(57.25520108,40.85936414)(58.38019995,40.37498963)(59.796875,40.375)
}
}
{
\newrgbcolor{curcolor}{0 0 0}
\pscustom[linewidth=2,linecolor=curcolor,linestyle=dashed,dash=8 8]
{
\newpath
\moveto(59.189408,539.999593)
\lineto(139.51115,539.999593)
}
}
{
\newrgbcolor{curcolor}{0 0 0}
\pscustom[linestyle=none,fillstyle=solid,fillcolor=curcolor]
{
\newpath
\moveto(129.04884464,544.84007524)
\lineto(142.1598974,540.01880891)
\lineto(129.04884392,535.19754364)
\curveto(131.143442,538.04401672)(131.13137288,541.93848592)(129.04884464,544.84007524)
\closepath
}
}
{
\newrgbcolor{curcolor}{0 0 0}
\pscustom[linestyle=none,fillstyle=solid,fillcolor=curcolor]
{
\newpath
\moveto(43.96875,532.65625)
\lineto(49.125,532.65625)
\lineto(49.125,550.453125)
\lineto(43.515625,549.328125)
\lineto(43.515625,552.203125)
\lineto(49.09375,553.328125)
\lineto(52.25,553.328125)
\lineto(52.25,532.65625)
\lineto(57.40625,532.65625)
\lineto(57.40625,530)
\lineto(43.96875,530)
\lineto(43.96875,532.65625)
}
}
{
\newrgbcolor{curcolor}{0 0 0}
\pscustom[linestyle=none,fillstyle=solid,fillcolor=curcolor]
{
\newpath
\moveto(506.140625,492.65625)
\lineto(517.15625,492.65625)
\lineto(517.15625,490)
\lineto(502.34375,490)
\lineto(502.34375,492.65625)
\curveto(503.54166313,493.89582944)(505.17186983,495.55728611)(507.234375,497.640625)
\curveto(509.30728236,499.73436527)(510.60936439,501.08332225)(511.140625,501.6875)
\curveto(512.15102952,502.82290384)(512.85415381,503.78123622)(513.25,504.5625)
\curveto(513.65623634,505.35415131)(513.85936114,506.1301922)(513.859375,506.890625)
\curveto(513.85936114,508.1301902)(513.42186158,509.14060586)(512.546875,509.921875)
\curveto(511.68227998,510.7031043)(510.55207278,511.09372891)(509.15625,511.09375)
\curveto(508.1666585,511.09372891)(507.11978455,510.92185408)(506.015625,510.578125)
\curveto(504.92187008,510.23435477)(503.74999625,509.71352195)(502.5,509.015625)
\lineto(502.5,512.203125)
\curveto(503.77082956,512.71351895)(504.95832838,513.09893523)(506.0625,513.359375)
\curveto(507.1666595,513.61976805)(508.17707516,513.74997625)(509.09375,513.75)
\curveto(511.51040516,513.74997625)(513.43748656,513.14581019)(514.875,511.9375)
\curveto(516.31248369,510.72914594)(517.03123297,509.11456422)(517.03125,507.09375)
\curveto(517.03123297,506.13540053)(516.84894148,505.22394311)(516.484375,504.359375)
\curveto(516.1301922,503.50519483)(515.47915119,502.49477917)(514.53125,501.328125)
\curveto(514.27081906,501.02603064)(513.44269489,500.15103152)(512.046875,498.703125)
\curveto(510.65103102,497.26561773)(508.68228298,495.24999475)(506.140625,492.65625)
}
}
{
\newrgbcolor{curcolor}{0 0 0}
\pscustom[linestyle=none,fillstyle=solid,fillcolor=curcolor]
{
\newpath
\moveto(493.96875,32.65625)
\lineto(499.125,32.65625)
\lineto(499.125,50.453125)
\lineto(493.515625,49.328125)
\lineto(493.515625,52.203125)
\lineto(499.09375,53.328125)
\lineto(502.25,53.328125)
\lineto(502.25,32.65625)
\lineto(507.40625,32.65625)
\lineto(507.40625,30)
\lineto(493.96875,30)
\lineto(493.96875,32.65625)
}
}
{
\newrgbcolor{curcolor}{0 0 0}
\pscustom[linestyle=none,fillstyle=solid,fillcolor=curcolor]
{
\newpath
\moveto(520.546875,51.25)
\curveto(518.92186645,51.24997875)(517.69790934,50.44789622)(516.875,48.84375)
\curveto(516.06249431,47.24998275)(515.65624472,44.84894348)(515.65625,41.640625)
\curveto(515.65624472,38.44269989)(516.06249431,36.04166062)(516.875,34.4375)
\curveto(517.69790934,32.84374716)(518.92186645,32.04687295)(520.546875,32.046875)
\curveto(522.18227986,32.04687295)(523.40623697,32.84374716)(524.21875,34.4375)
\curveto(525.041652,36.04166062)(525.45310992,38.44269989)(525.453125,41.640625)
\curveto(525.45310992,44.84894348)(525.041652,47.24998275)(524.21875,48.84375)
\curveto(523.40623697,50.44789622)(522.18227986,51.24997875)(520.546875,51.25)
\moveto(520.546875,53.75)
\curveto(523.16144555,53.74997625)(525.15623522,52.71351895)(526.53125,50.640625)
\curveto(527.91664913,48.57810642)(528.60935677,45.57810942)(528.609375,41.640625)
\curveto(528.60935677,37.71353395)(527.91664913,34.71353695)(526.53125,32.640625)
\curveto(525.15623522,30.57812442)(523.16144555,29.54687545)(520.546875,29.546875)
\curveto(517.93228411,29.54687545)(515.93228611,30.57812442)(514.546875,32.640625)
\curveto(513.1718722,34.71353695)(512.48437289,37.71353395)(512.484375,41.640625)
\curveto(512.48437289,45.57810942)(513.1718722,48.57810642)(514.546875,50.640625)
\curveto(515.93228611,52.71351895)(517.93228411,53.74997625)(520.546875,53.75)
}
}
{
\newrgbcolor{curcolor}{0 0 0}
\pscustom[linestyle=none,fillstyle=solid,fillcolor=curcolor]
{
\newpath
\moveto(142.01269531,194.25488281)
\lineto(142.01269531,181.78320312)
\lineto(144.63378906,181.78320312)
\curveto(146.84667052,181.78320134)(148.46515849,182.28450292)(149.48925781,183.28710938)
\curveto(150.52049498,184.28970925)(151.03611946,185.87238996)(151.03613281,188.03515625)
\curveto(151.03611946,190.18358357)(150.52049498,191.7555221)(149.48925781,192.75097656)
\curveto(148.46515849,193.75356698)(146.84667052,194.25486856)(144.63378906,194.25488281)
\lineto(142.01269531,194.25488281)
\moveto(139.84277344,196.03808594)
\lineto(144.30078125,196.03808594)
\curveto(147.40884444,196.0380699)(149.68976664,195.38995857)(151.14355469,194.09375)
\curveto(152.59731582,192.8046747)(153.32420311,190.78514546)(153.32421875,188.03515625)
\curveto(153.32420311,185.27082806)(152.59373509,183.24055666)(151.1328125,181.94433594)
\curveto(149.67186301,180.64811133)(147.39452154,180)(144.30078125,180)
\lineto(139.84277344,180)
\lineto(139.84277344,196.03808594)
}
}
{
\newrgbcolor{curcolor}{0 0 0}
\pscustom[linestyle=none,fillstyle=solid,fillcolor=curcolor]
{
\newpath
\moveto(156.68652344,192.03125)
\lineto(158.66308594,192.03125)
\lineto(158.66308594,180)
\lineto(156.68652344,180)
\lineto(156.68652344,192.03125)
\moveto(156.68652344,196.71484375)
\lineto(158.66308594,196.71484375)
\lineto(158.66308594,194.21191406)
\lineto(156.68652344,194.21191406)
\lineto(156.68652344,196.71484375)
}
}
{
\newrgbcolor{curcolor}{0 0 0}
\pscustom[linestyle=none,fillstyle=solid,fillcolor=curcolor]
{
\newpath
\moveto(168.87890625,196.71484375)
\lineto(168.87890625,195.07128906)
\lineto(166.98828125,195.07128906)
\curveto(166.27929131,195.07127399)(165.78515118,194.92804497)(165.50585938,194.64160156)
\curveto(165.23371944,194.35512887)(165.09765187,193.83950439)(165.09765625,193.09472656)
\lineto(165.09765625,192.03125)
\lineto(170.52246094,192.03125)
\lineto(170.52246094,192.86914062)
\curveto(170.52245113,194.20831913)(170.83397426,195.18227648)(171.45703125,195.79101562)
\curveto(171.65754114,195.99152047)(171.89744975,196.15981457)(172.17675781,196.29589844)
\curveto(172.73533954,196.57517874)(173.48371119,196.71482704)(174.421875,196.71484375)
\lineto(176.29101562,196.71484375)
\lineto(176.29101562,195.07128906)
\lineto(174.40039062,195.07128906)
\curveto(173.69139327,195.07127399)(173.19725314,194.92804497)(172.91796875,194.64160156)
\curveto(172.6458214,194.35512887)(172.50975383,193.83950439)(172.50976562,193.09472656)
\lineto(172.50976562,192.03125)
\lineto(179.921875,192.03125)
\lineto(179.921875,180)
\lineto(177.93457031,180)
\lineto(177.93457031,190.49511719)
\lineto(172.50976562,190.49511719)
\lineto(172.50976562,180)
\lineto(170.52246094,180)
\lineto(170.52246094,190.49511719)
\lineto(165.09765625,190.49511719)
\lineto(165.09765625,180)
\lineto(163.11035156,180)
\lineto(163.11035156,190.49511719)
\lineto(161.21972656,190.49511719)
\lineto(161.21972656,192.03125)
\lineto(163.11035156,192.03125)
\lineto(163.11035156,192.86914062)
\curveto(163.11034917,194.20831913)(163.42187229,195.18227648)(164.04492188,195.79101562)
\curveto(164.6679648,196.40688463)(165.65624506,196.71482704)(167.00976562,196.71484375)
\lineto(168.87890625,196.71484375)
\moveto(177.93457031,196.69335938)
\lineto(179.921875,196.69335938)
\lineto(179.921875,194.19042969)
\lineto(177.93457031,194.19042969)
\lineto(177.93457031,196.69335938)
}
}
{
\newrgbcolor{curcolor}{0 0 0}
\pscustom[linestyle=none,fillstyle=solid,fillcolor=curcolor]
{
\newpath
\moveto(192.71582031,191.56933594)
\lineto(192.71582031,189.72167969)
\curveto(192.15721639,190.02961237)(191.59504247,190.2587788)(191.02929688,190.40917969)
\curveto(190.47069464,190.5667212)(189.90494,190.64549717)(189.33203125,190.64550781)
\curveto(188.05012414,190.64549717)(187.05468243,190.23729445)(186.34570312,189.42089844)
\curveto(185.6367151,188.61164503)(185.28222326,187.4729743)(185.28222656,186.00488281)
\curveto(185.28222326,184.53677932)(185.6367151,183.39452786)(186.34570312,182.578125)
\curveto(187.05468243,181.76887844)(188.05012414,181.36425645)(189.33203125,181.36425781)
\curveto(189.90494,181.36425645)(190.47069464,181.43945169)(191.02929688,181.58984375)
\curveto(191.59504247,181.74739409)(192.15721639,181.98014125)(192.71582031,182.28808594)
\lineto(192.71582031,180.46191406)
\curveto(192.16437784,180.20410136)(191.59146175,180.01074218)(190.99707031,179.88183594)
\curveto(190.4098223,179.75292993)(189.78319533,179.68847687)(189.1171875,179.68847656)
\curveto(187.30533322,179.68847687)(185.86588154,180.25781224)(184.79882812,181.39648438)
\curveto(183.73176909,182.53515371)(183.19824097,184.07128499)(183.19824219,186.00488281)
\curveto(183.19824097,187.96711443)(183.73534981,189.51040716)(184.80957031,190.63476562)
\curveto(185.89094661,191.75910282)(187.36978628,192.32127674)(189.24609375,192.32128906)
\curveto(189.85480984,192.32127674)(190.44921029,192.25682368)(191.02929688,192.12792969)
\curveto(191.60936538,192.00617289)(192.17153929,191.81997516)(192.71582031,191.56933594)
}
}
{
\newrgbcolor{curcolor}{0 0 0}
\pscustom[linestyle=none,fillstyle=solid,fillcolor=curcolor]
{
\newpath
\moveto(195.97070312,184.74804688)
\lineto(195.97070312,192.03125)
\lineto(197.94726562,192.03125)
\lineto(197.94726562,184.82324219)
\curveto(197.94726178,183.68456663)(198.16926677,182.82877321)(198.61328125,182.25585938)
\curveto(199.05728671,181.69010248)(199.72330167,181.40722516)(200.61132812,181.40722656)
\curveto(201.67837784,181.40722516)(202.51984835,181.74739409)(203.13574219,182.42773438)
\curveto(203.75877941,183.10806981)(204.07030253,184.03547774)(204.0703125,185.20996094)
\lineto(204.0703125,192.03125)
\lineto(206.046875,192.03125)
\lineto(206.046875,180)
\lineto(204.0703125,180)
\lineto(204.0703125,181.84765625)
\curveto(203.5904853,181.11718638)(203.03189211,180.57291609)(202.39453125,180.21484375)
\curveto(201.76431525,179.86393243)(201.03026651,179.68847687)(200.19238281,179.68847656)
\curveto(198.81021665,179.68847687)(197.76106405,180.11816394)(197.04492188,180.97753906)
\curveto(196.32877381,181.83691223)(195.97070126,183.09374691)(195.97070312,184.74804688)
\moveto(200.94433594,192.32128906)
\lineto(200.94433594,192.32128906)
}
}
{
\newrgbcolor{curcolor}{0 0 0}
\pscustom[linestyle=none,fillstyle=solid,fillcolor=curcolor]
{
\newpath
\moveto(210.13964844,196.71484375)
\lineto(212.11621094,196.71484375)
\lineto(212.11621094,180)
\lineto(210.13964844,180)
\lineto(210.13964844,196.71484375)
}
}
{
\newrgbcolor{curcolor}{0 0 0}
\pscustom[linestyle=none,fillstyle=solid,fillcolor=curcolor]
{
\newpath
\moveto(218.19628906,195.44726562)
\lineto(218.19628906,192.03125)
\lineto(222.26757812,192.03125)
\lineto(222.26757812,190.49511719)
\lineto(218.19628906,190.49511719)
\lineto(218.19628906,183.96386719)
\curveto(218.19628503,182.98274441)(218.32877188,182.35253671)(218.59375,182.07324219)
\curveto(218.86588072,181.79394352)(219.41373173,181.65429522)(220.23730469,181.65429688)
\lineto(222.26757812,181.65429688)
\lineto(222.26757812,180)
\lineto(220.23730469,180)
\curveto(218.71190952,180)(217.6591762,180.28287732)(217.07910156,180.84863281)
\curveto(216.49902111,181.42154806)(216.20898233,182.45995848)(216.20898438,183.96386719)
\lineto(216.20898438,190.49511719)
\lineto(214.75878906,190.49511719)
\lineto(214.75878906,192.03125)
\lineto(216.20898438,192.03125)
\lineto(216.20898438,195.44726562)
\lineto(218.19628906,195.44726562)
}
}
{
\newrgbcolor{curcolor}{0 0 0}
\pscustom[linestyle=none,fillstyle=solid,fillcolor=curcolor]
{
\newpath
\moveto(235.16894531,186.50976562)
\lineto(235.16894531,185.54296875)
\lineto(226.08105469,185.54296875)
\curveto(226.16698883,184.18228748)(226.57519154,183.14387706)(227.30566406,182.42773438)
\curveto(228.04328903,181.71874828)(229.06737655,181.36425645)(230.37792969,181.36425781)
\curveto(231.13703594,181.36425645)(231.87108468,181.45735531)(232.58007812,181.64355469)
\curveto(233.29621347,181.82975077)(234.00519713,182.10904737)(234.70703125,182.48144531)
\lineto(234.70703125,180.61230469)
\curveto(233.99803568,180.31152313)(233.27114839,180.08235669)(232.52636719,179.92480469)
\curveto(231.78156654,179.76725284)(231.02603345,179.68847687)(230.25976562,179.68847656)
\curveto(228.34048926,179.68847687)(226.81868088,180.24707007)(225.69433594,181.36425781)
\curveto(224.57714667,182.48144283)(224.01855347,183.99250903)(224.01855469,185.89746094)
\curveto(224.01855347,187.86685411)(224.54850086,189.42805047)(225.60839844,190.58105469)
\curveto(226.67545186,191.7411992)(228.11132282,192.32127674)(229.91601562,192.32128906)
\curveto(231.53449648,192.32127674)(232.81281551,191.79849081)(233.75097656,190.75292969)
\curveto(234.69627717,189.71450851)(235.16893295,188.30012191)(235.16894531,186.50976562)
\moveto(233.19238281,187.08984375)
\curveto(233.17804952,188.17121579)(232.87368785,189.03417065)(232.27929688,189.67871094)
\curveto(231.6920484,190.32323186)(230.91145023,190.64549717)(229.9375,190.64550781)
\curveto(228.83462939,190.64549717)(227.95019017,190.33397404)(227.28417969,189.7109375)
\curveto(226.6253217,189.08788154)(226.24576479,188.21060377)(226.14550781,187.07910156)
\lineto(233.19238281,187.08984375)
\moveto(231.29101562,197.59570312)
\lineto(233.42871094,197.59570312)
\lineto(229.92675781,193.55664062)
\lineto(228.28320312,193.55664062)
\lineto(231.29101562,197.59570312)
}
}
{
\newrgbcolor{curcolor}{0 0 0}
\pscustom[linestyle=none,fillstyle=solid,fillcolor=curcolor]
{
}
}
{
\newrgbcolor{curcolor}{0 0 0}
\pscustom[linestyle=none,fillstyle=solid,fillcolor=curcolor]
{
\newpath
\moveto(253.33398438,190.20507812)
\lineto(253.33398438,196.71484375)
\lineto(255.31054688,196.71484375)
\lineto(255.31054688,180)
\lineto(253.33398438,180)
\lineto(253.33398438,181.8046875)
\curveto(252.91861022,181.08854058)(252.39224356,180.55501247)(251.75488281,180.20410156)
\curveto(251.1246667,179.8603517)(250.36555287,179.68847687)(249.47753906,179.68847656)
\curveto(248.02375834,179.68847687)(246.83853817,180.26855442)(245.921875,181.42871094)
\curveto(245.01236812,182.5888646)(244.55761597,184.1142537)(244.55761719,186.00488281)
\curveto(244.55761597,187.89549992)(245.01236812,189.42088902)(245.921875,190.58105469)
\curveto(246.83853817,191.7411992)(248.02375834,192.32127674)(249.47753906,192.32128906)
\curveto(250.36555287,192.32127674)(251.1246667,192.14582119)(251.75488281,191.79492188)
\curveto(252.39224356,191.45116042)(252.91861022,190.92121304)(253.33398438,190.20507812)
\moveto(246.59863281,186.00488281)
\curveto(246.59862956,184.55110222)(246.89582978,183.40885076)(247.49023438,182.578125)
\curveto(248.09179213,181.75455554)(248.91535901,181.34277209)(249.9609375,181.34277344)
\curveto(251.00650275,181.34277209)(251.83006964,181.75455554)(252.43164062,182.578125)
\curveto(253.03319344,183.40885076)(253.33397438,184.55110222)(253.33398438,186.00488281)
\curveto(253.33397438,187.4586514)(253.03319344,188.59732213)(252.43164062,189.42089844)
\curveto(251.83006964,190.25161735)(251.00650275,190.66698152)(249.9609375,190.66699219)
\curveto(248.91535901,190.66698152)(248.09179213,190.25161735)(247.49023438,189.42089844)
\curveto(246.89582978,188.59732213)(246.59862956,187.4586514)(246.59863281,186.00488281)
}
}
{
\newrgbcolor{curcolor}{0 0 0}
\pscustom[linestyle=none,fillstyle=solid,fillcolor=curcolor]
{
\newpath
\moveto(269.67285156,186.50976562)
\lineto(269.67285156,185.54296875)
\lineto(260.58496094,185.54296875)
\curveto(260.67089508,184.18228748)(261.07909779,183.14387706)(261.80957031,182.42773438)
\curveto(262.54719528,181.71874828)(263.5712828,181.36425645)(264.88183594,181.36425781)
\curveto(265.64094219,181.36425645)(266.37499093,181.45735531)(267.08398438,181.64355469)
\curveto(267.80011972,181.82975077)(268.50910338,182.10904737)(269.2109375,182.48144531)
\lineto(269.2109375,180.61230469)
\curveto(268.50194193,180.31152313)(267.77505464,180.08235669)(267.03027344,179.92480469)
\curveto(266.28547279,179.76725284)(265.5299397,179.68847687)(264.76367188,179.68847656)
\curveto(262.84439551,179.68847687)(261.32258713,180.24707007)(260.19824219,181.36425781)
\curveto(259.08105292,182.48144283)(258.52245972,183.99250903)(258.52246094,185.89746094)
\curveto(258.52245972,187.86685411)(259.05240711,189.42805047)(260.11230469,190.58105469)
\curveto(261.17935811,191.7411992)(262.61522907,192.32127674)(264.41992188,192.32128906)
\curveto(266.03840273,192.32127674)(267.31672176,191.79849081)(268.25488281,190.75292969)
\curveto(269.20018342,189.71450851)(269.6728392,188.30012191)(269.67285156,186.50976562)
\moveto(267.69628906,187.08984375)
\curveto(267.68195577,188.17121579)(267.3775941,189.03417065)(266.78320312,189.67871094)
\curveto(266.19595465,190.32323186)(265.41535648,190.64549717)(264.44140625,190.64550781)
\curveto(263.33853564,190.64549717)(262.45409642,190.33397404)(261.78808594,189.7109375)
\curveto(261.12922795,189.08788154)(260.74967104,188.21060377)(260.64941406,187.07910156)
\lineto(267.69628906,187.08984375)
}
}
{
\newrgbcolor{curcolor}{0 0 0}
\pscustom[linestyle=none,fillstyle=solid,fillcolor=curcolor]
{
\newpath
\moveto(280.58691406,191.67675781)
\lineto(280.58691406,189.80761719)
\curveto(280.02831113,190.09406543)(279.44823358,190.30890896)(278.84667969,190.45214844)
\curveto(278.24510979,190.59536701)(277.62206353,190.66698152)(276.97753906,190.66699219)
\curveto(275.99641412,190.66698152)(275.25878465,190.51659105)(274.76464844,190.21582031)
\curveto(274.27766584,189.91502915)(274.0341765,189.46385772)(274.03417969,188.86230469)
\curveto(274.0341765,188.40396295)(274.20963205,188.04230967)(274.56054688,187.77734375)
\curveto(274.91145427,187.51952373)(275.61685721,187.27245367)(276.67675781,187.03613281)
\lineto(277.35351562,186.88574219)
\curveto(278.75715354,186.58495435)(279.75259526,186.15884801)(280.33984375,185.60742188)
\curveto(280.9342347,185.06314598)(281.23143492,184.30045143)(281.23144531,183.31933594)
\curveto(281.23143492,182.20214624)(280.78742495,181.31770702)(279.89941406,180.66601562)
\curveto(279.01854651,180.0143229)(277.80468054,179.68847687)(276.2578125,179.68847656)
\curveto(275.61327648,179.68847687)(274.94010007,179.75292993)(274.23828125,179.88183594)
\curveto(273.54361709,180.00358073)(272.80956835,180.18977846)(272.03613281,180.44042969)
\lineto(272.03613281,182.48144531)
\curveto(272.76659964,182.10188592)(273.48632548,181.81542787)(274.1953125,181.62207031)
\curveto(274.90429281,181.43587096)(275.60611503,181.34277209)(276.30078125,181.34277344)
\curveto(277.23176445,181.34277209)(277.94790956,181.50032402)(278.44921875,181.81542969)
\curveto(278.95051273,182.13769317)(279.20116352,182.5888646)(279.20117188,183.16894531)
\curveto(279.20116352,183.70605098)(279.01854651,184.11783442)(278.65332031,184.40429688)
\curveto(278.29523994,184.69075052)(277.50389959,184.96646639)(276.27929688,185.23144531)
\lineto(275.59179688,185.39257812)
\curveto(274.36718398,185.65038497)(273.48274476,186.04426479)(272.93847656,186.57421875)
\curveto(272.39420418,187.11132101)(272.12206903,187.84536976)(272.12207031,188.77636719)
\curveto(272.12206903,189.9078677)(272.5231103,190.78156474)(273.32519531,191.39746094)
\curveto(274.12727536,192.01333434)(275.2659461,192.32127674)(276.74121094,192.32128906)
\curveto(277.47167306,192.32127674)(278.15917237,192.26756586)(278.80371094,192.16015625)
\curveto(279.44823358,192.05272232)(280.04263403,191.89158967)(280.58691406,191.67675781)
}
}
{
\newrgbcolor{curcolor}{0 0 0}
\pscustom[linestyle=none,fillstyle=solid,fillcolor=curcolor]
{
\newpath
\moveto(175.35644531,159.00976562)
\lineto(175.35644531,158.04296875)
\lineto(166.26855469,158.04296875)
\curveto(166.35448883,156.68228748)(166.76269154,155.64387706)(167.49316406,154.92773438)
\curveto(168.23078903,154.21874828)(169.25487655,153.86425645)(170.56542969,153.86425781)
\curveto(171.32453594,153.86425645)(172.05858468,153.95735531)(172.76757812,154.14355469)
\curveto(173.48371347,154.32975077)(174.19269713,154.60904737)(174.89453125,154.98144531)
\lineto(174.89453125,153.11230469)
\curveto(174.18553568,152.81152313)(173.45864839,152.58235669)(172.71386719,152.42480469)
\curveto(171.96906654,152.26725284)(171.21353345,152.18847687)(170.44726562,152.18847656)
\curveto(168.52798926,152.18847687)(167.00618088,152.74707007)(165.88183594,153.86425781)
\curveto(164.76464667,154.98144283)(164.20605347,156.49250903)(164.20605469,158.39746094)
\curveto(164.20605347,160.36685411)(164.73600086,161.92805047)(165.79589844,163.08105469)
\curveto(166.86295186,164.2411992)(168.29882282,164.82127674)(170.10351562,164.82128906)
\curveto(171.72199648,164.82127674)(173.00031551,164.29849081)(173.93847656,163.25292969)
\curveto(174.88377717,162.21450851)(175.35643295,160.80012191)(175.35644531,159.00976562)
\moveto(173.37988281,159.58984375)
\curveto(173.36554952,160.67121579)(173.06118785,161.53417065)(172.46679688,162.17871094)
\curveto(171.8795484,162.82323186)(171.09895023,163.14549717)(170.125,163.14550781)
\curveto(169.02212939,163.14549717)(168.13769017,162.83397404)(167.47167969,162.2109375)
\curveto(166.8128217,161.58788154)(166.43326479,160.71060377)(166.33300781,159.57910156)
\lineto(173.37988281,159.58984375)
}
}
{
\newrgbcolor{curcolor}{0 0 0}
\pscustom[linestyle=none,fillstyle=solid,fillcolor=curcolor]
{
\newpath
\moveto(188.6015625,159.76171875)
\lineto(188.6015625,152.5)
\lineto(186.625,152.5)
\lineto(186.625,159.69726562)
\curveto(186.6249899,160.83592916)(186.40298492,161.68814185)(185.95898438,162.25390625)
\curveto(185.51496497,162.81965114)(184.84895001,163.10252846)(183.9609375,163.10253906)
\curveto(182.89387384,163.10252846)(182.05240333,162.76235953)(181.43652344,162.08203125)
\curveto(180.82063373,161.40168381)(180.51269133,160.47427588)(180.51269531,159.29980469)
\lineto(180.51269531,152.5)
\lineto(178.52539062,152.5)
\lineto(178.52539062,164.53125)
\lineto(180.51269531,164.53125)
\lineto(180.51269531,162.66210938)
\curveto(180.9853471,163.38540578)(181.54035957,163.92609534)(182.17773438,164.28417969)
\curveto(182.82225933,164.64224046)(183.56346953,164.82127674)(184.40136719,164.82128906)
\curveto(185.78351939,164.82127674)(186.82909126,164.39158967)(187.53808594,163.53222656)
\curveto(188.24705859,162.68000284)(188.60155043,161.42316816)(188.6015625,159.76171875)
}
}
{
\newrgbcolor{curcolor}{0 0 0}
\pscustom[linestyle=none,fillstyle=solid,fillcolor=curcolor]
{
\newpath
\moveto(202.56640625,159.76171875)
\lineto(202.56640625,152.5)
\lineto(200.58984375,152.5)
\lineto(200.58984375,159.69726562)
\curveto(200.58983365,160.83592916)(200.36782867,161.68814185)(199.92382812,162.25390625)
\curveto(199.47980872,162.81965114)(198.81379376,163.10252846)(197.92578125,163.10253906)
\curveto(196.85871759,163.10252846)(196.01724708,162.76235953)(195.40136719,162.08203125)
\curveto(194.78547748,161.40168381)(194.47753508,160.47427588)(194.47753906,159.29980469)
\lineto(194.47753906,152.5)
\lineto(192.49023438,152.5)
\lineto(192.49023438,164.53125)
\lineto(194.47753906,164.53125)
\lineto(194.47753906,162.66210938)
\curveto(194.95019085,163.38540578)(195.50520332,163.92609534)(196.14257812,164.28417969)
\curveto(196.78710308,164.64224046)(197.52831328,164.82127674)(198.36621094,164.82128906)
\curveto(199.74836314,164.82127674)(200.79393501,164.39158967)(201.50292969,163.53222656)
\curveto(202.21190234,162.68000284)(202.56639418,161.42316816)(202.56640625,159.76171875)
}
}
{
\newrgbcolor{curcolor}{0 0 0}
\pscustom[linestyle=none,fillstyle=solid,fillcolor=curcolor]
{
\newpath
\moveto(216.82128906,159.00976562)
\lineto(216.82128906,158.04296875)
\lineto(207.73339844,158.04296875)
\curveto(207.81933258,156.68228748)(208.22753529,155.64387706)(208.95800781,154.92773438)
\curveto(209.69563278,154.21874828)(210.7197203,153.86425645)(212.03027344,153.86425781)
\curveto(212.78937969,153.86425645)(213.52342843,153.95735531)(214.23242188,154.14355469)
\curveto(214.94855722,154.32975077)(215.65754088,154.60904737)(216.359375,154.98144531)
\lineto(216.359375,153.11230469)
\curveto(215.65037943,152.81152313)(214.92349214,152.58235669)(214.17871094,152.42480469)
\curveto(213.43391029,152.26725284)(212.6783772,152.18847687)(211.91210938,152.18847656)
\curveto(209.99283301,152.18847687)(208.47102463,152.74707007)(207.34667969,153.86425781)
\curveto(206.22949042,154.98144283)(205.67089722,156.49250903)(205.67089844,158.39746094)
\curveto(205.67089722,160.36685411)(206.20084461,161.92805047)(207.26074219,163.08105469)
\curveto(208.32779561,164.2411992)(209.76366657,164.82127674)(211.56835938,164.82128906)
\curveto(213.18684023,164.82127674)(214.46515926,164.29849081)(215.40332031,163.25292969)
\curveto(216.34862092,162.21450851)(216.8212767,160.80012191)(216.82128906,159.00976562)
\moveto(214.84472656,159.58984375)
\curveto(214.83039327,160.67121579)(214.5260316,161.53417065)(213.93164062,162.17871094)
\curveto(213.34439215,162.82323186)(212.56379398,163.14549717)(211.58984375,163.14550781)
\curveto(210.48697314,163.14549717)(209.60253392,162.83397404)(208.93652344,162.2109375)
\curveto(208.27766545,161.58788154)(207.89810854,160.71060377)(207.79785156,159.57910156)
\lineto(214.84472656,159.58984375)
}
}
{
\newrgbcolor{curcolor}{0 0 0}
\pscustom[linestyle=none,fillstyle=solid,fillcolor=curcolor]
{
\newpath
\moveto(229.43261719,162.22167969)
\curveto(229.92674588,163.10968991)(230.5175656,163.76496269)(231.20507812,164.1875)
\curveto(231.89256422,164.61001393)(232.70180821,164.82127674)(233.6328125,164.82128906)
\curveto(234.88605081,164.82127674)(235.85284672,164.38084749)(236.53320312,163.5)
\curveto(237.21352245,162.62629196)(237.55369138,161.38019945)(237.55371094,159.76171875)
\lineto(237.55371094,152.5)
\lineto(235.56640625,152.5)
\lineto(235.56640625,159.69726562)
\curveto(235.56638868,160.85025207)(235.36228732,161.70604548)(234.95410156,162.26464844)
\curveto(234.54588188,162.82323186)(233.92283563,163.10252846)(233.08496094,163.10253906)
\curveto(232.06085833,163.10252846)(231.25161434,162.76235953)(230.65722656,162.08203125)
\curveto(230.06281345,161.40168381)(229.76561323,160.47427588)(229.765625,159.29980469)
\lineto(229.765625,152.5)
\lineto(227.77832031,152.5)
\lineto(227.77832031,159.69726562)
\curveto(227.77831053,160.85741352)(227.57420917,161.71320693)(227.16601562,162.26464844)
\curveto(226.75780373,162.82323186)(226.12759603,163.10252846)(225.27539062,163.10253906)
\curveto(224.26561873,163.10252846)(223.4635362,162.7587788)(222.86914062,162.07128906)
\curveto(222.2747353,161.39094163)(221.97753508,160.46711443)(221.97753906,159.29980469)
\lineto(221.97753906,152.5)
\lineto(219.99023438,152.5)
\lineto(219.99023438,164.53125)
\lineto(221.97753906,164.53125)
\lineto(221.97753906,162.66210938)
\curveto(222.4287065,163.39972868)(222.96939606,163.94399897)(223.59960938,164.29492188)
\curveto(224.22981147,164.64582119)(224.97818312,164.82127674)(225.84472656,164.82128906)
\curveto(226.71841575,164.82127674)(227.45962595,164.59927175)(228.06835938,164.15527344)
\curveto(228.6842341,163.71125181)(229.13898625,163.0667212)(229.43261719,162.22167969)
}
}
{
\newrgbcolor{curcolor}{0 0 0}
\pscustom[linestyle=none,fillstyle=solid,fillcolor=curcolor]
{
\newpath
\moveto(241.50683594,164.53125)
\lineto(243.48339844,164.53125)
\lineto(243.48339844,152.5)
\lineto(241.50683594,152.5)
\lineto(241.50683594,164.53125)
\moveto(241.50683594,169.21484375)
\lineto(243.48339844,169.21484375)
\lineto(243.48339844,166.71191406)
\lineto(241.50683594,166.71191406)
\lineto(241.50683594,169.21484375)
}
}
{
\newrgbcolor{curcolor}{0 0 0}
\pscustom[linestyle=none,fillstyle=solid,fillcolor=curcolor]
{
\newpath
\moveto(255.27832031,164.17675781)
\lineto(255.27832031,162.30761719)
\curveto(254.71971738,162.59406543)(254.13963983,162.80890896)(253.53808594,162.95214844)
\curveto(252.93651604,163.09536701)(252.31346978,163.16698152)(251.66894531,163.16699219)
\curveto(250.68782037,163.16698152)(249.9501909,163.01659105)(249.45605469,162.71582031)
\curveto(248.96907209,162.41502915)(248.72558275,161.96385772)(248.72558594,161.36230469)
\curveto(248.72558275,160.90396295)(248.9010383,160.54230967)(249.25195312,160.27734375)
\curveto(249.60286052,160.01952373)(250.30826346,159.77245367)(251.36816406,159.53613281)
\lineto(252.04492188,159.38574219)
\curveto(253.44855979,159.08495435)(254.44400151,158.65884801)(255.03125,158.10742188)
\curveto(255.62564095,157.56314598)(255.92284117,156.80045143)(255.92285156,155.81933594)
\curveto(255.92284117,154.70214624)(255.4788312,153.81770702)(254.59082031,153.16601562)
\curveto(253.70995276,152.5143229)(252.49608679,152.18847687)(250.94921875,152.18847656)
\curveto(250.30468273,152.18847687)(249.63150632,152.25292993)(248.9296875,152.38183594)
\curveto(248.23502334,152.50358073)(247.5009746,152.68977846)(246.72753906,152.94042969)
\lineto(246.72753906,154.98144531)
\curveto(247.45800589,154.60188592)(248.17773173,154.31542787)(248.88671875,154.12207031)
\curveto(249.59569906,153.93587096)(250.29752128,153.84277209)(250.9921875,153.84277344)
\curveto(251.9231707,153.84277209)(252.63931581,154.00032402)(253.140625,154.31542969)
\curveto(253.64191898,154.63769317)(253.89256977,155.0888646)(253.89257812,155.66894531)
\curveto(253.89256977,156.20605098)(253.70995276,156.61783442)(253.34472656,156.90429688)
\curveto(252.98664619,157.19075052)(252.19530584,157.46646639)(250.97070312,157.73144531)
\lineto(250.28320312,157.89257812)
\curveto(249.05859023,158.15038497)(248.17415101,158.54426479)(247.62988281,159.07421875)
\curveto(247.08561043,159.61132101)(246.81347528,160.34536976)(246.81347656,161.27636719)
\curveto(246.81347528,162.4078677)(247.21451655,163.28156474)(248.01660156,163.89746094)
\curveto(248.81868161,164.51333434)(249.95735235,164.82127674)(251.43261719,164.82128906)
\curveto(252.16307931,164.82127674)(252.85057862,164.76756586)(253.49511719,164.66015625)
\curveto(254.13963983,164.55272232)(254.73404028,164.39158967)(255.27832031,164.17675781)
}
}
\end{pspicture}

		
		\begin{enumerate}
		  \item Fond d'écran
		  \item Carte sélectionnée
		  \item Carte précédente
		  \item Carte suivante
		  \item Nom de la carte sélectionnée
		  \item Liste déroulante ``Type de partie''
		  \item Liste déroulante ``Nombre d'ennemis''
		  \item Liste déroulante ``Difficulté des ennemis''
		  \item Bouton ``\hyperlink{Page d'accueil}{Retour}''
		  \item Bouton ``Lancer'' 
		\end{enumerate}
		
		$\,$
		
		(2) permet à l'utilisateur de choisir une carte parmi celles présentes sur le
		téléphone en ayant un aperçu de la précédente (3) et de la suivante (4), tout
		en ayant le nom de celle sélectionnée (5). Une fois sa carte choisie il peut 
		régler différents paramètres comme le type de la partie (6) qui liste les
		différents types de parties stockées sur le téléphone, le nombre d'ennemis (7)
		compris entre 1 et 3 ainsi que la difficulté des ennemis (8).
		Le bouton ``\hyperlink{Page d'accueil}{Retour}'' (9) permet de revenir à la page d'accueil%
		\footnote[1]{
			\hyperlink{Page d'accueil}{Page d'accueil}
			\og voir section \ref{accueil}, page \pageref{accueil}.\fg
		}
		(les paramètres choisis seront oubliés).\\
		Le bouton ``Lancer" (10) permet, une fois les paramètres sélectionnés de
		lancer une partie locale.
		
		$\,$
	
\newpage

	\subsection{Création d'un compte multi-joueurs}
	
		\hypertarget{Creation compte multi-joueurs}{}
		\label{Creation compte multi-joueurs}
	
		%LaTeX with PSTricks extensions
%%Creator: inkscape 0.47
%%Please note this file requires PSTricks extensions
\psset{xunit=.5pt,yunit=.5pt,runit=.5pt}
\begin{pspicture}(560,600)
{
\newrgbcolor{curcolor}{1 1 1}
\pscustom[linestyle=none,fillstyle=solid,fillcolor=curcolor]
{
\newpath
\moveto(133.12401581,597.52220274)
\lineto(426.87598419,597.52220274)
\curveto(443.85397169,597.52220274)(457.52217102,583.85400341)(457.52217102,566.87601591)
\lineto(457.52217102,33.12401701)
\curveto(457.52217102,16.14602951)(443.85397169,2.47783018)(426.87598419,2.47783018)
\lineto(133.12401581,2.47783018)
\curveto(116.14602831,2.47783018)(102.47782898,16.14602951)(102.47782898,33.12401701)
\lineto(102.47782898,566.87601591)
\curveto(102.47782898,583.85400341)(116.14602831,597.52220274)(133.12401581,597.52220274)
\closepath
}
}
{
\newrgbcolor{curcolor}{0 0 0}
\pscustom[linewidth=4.95566034,linecolor=curcolor]
{
\newpath
\moveto(133.12401581,597.52220274)
\lineto(426.87598419,597.52220274)
\curveto(443.85397169,597.52220274)(457.52217102,583.85400341)(457.52217102,566.87601591)
\lineto(457.52217102,33.12401701)
\curveto(457.52217102,16.14602951)(443.85397169,2.47783018)(426.87598419,2.47783018)
\lineto(133.12401581,2.47783018)
\curveto(116.14602831,2.47783018)(102.47782898,16.14602951)(102.47782898,33.12401701)
\lineto(102.47782898,566.87601591)
\curveto(102.47782898,583.85400341)(116.14602831,597.52220274)(133.12401581,597.52220274)
\closepath
}
}
{
\newrgbcolor{curcolor}{1 1 1}
\pscustom[linestyle=none,fillstyle=solid,fillcolor=curcolor]
{
\newpath
\moveto(165.03649902,459.92251707)
\lineto(394.86647034,459.92251707)
\curveto(419.81245988,459.92251707)(439.89533234,439.83964462)(439.89533234,414.89365507)
\lineto(439.89533234,125.0288632)
\curveto(439.89533234,100.08287365)(419.81245988,80.0000012)(394.86647034,80.0000012)
\lineto(165.03649902,80.0000012)
\curveto(140.09050948,80.0000012)(120.00763702,100.08287365)(120.00763702,125.0288632)
\lineto(120.00763702,414.89365507)
\curveto(120.00763702,439.83964462)(140.09050948,459.92251707)(165.03649902,459.92251707)
\closepath
}
}
{
\newrgbcolor{curcolor}{0 0 0}
\pscustom[linewidth=2.08383346,linecolor=curcolor]
{
\newpath
\moveto(165.03649902,459.92251707)
\lineto(394.86647034,459.92251707)
\curveto(419.81245988,459.92251707)(439.89533234,439.83964462)(439.89533234,414.89365507)
\lineto(439.89533234,125.0288632)
\curveto(439.89533234,100.08287365)(419.81245988,80.0000012)(394.86647034,80.0000012)
\lineto(165.03649902,80.0000012)
\curveto(140.09050948,80.0000012)(120.00763702,100.08287365)(120.00763702,125.0288632)
\lineto(120.00763702,414.89365507)
\curveto(120.00763702,439.83964462)(140.09050948,459.92251707)(165.03649902,459.92251707)
\closepath
}
}
{
\newrgbcolor{curcolor}{0 0 0}
\pscustom[linestyle=none,fillstyle=solid,fillcolor=curcolor]
{
\newpath
\moveto(329.568,126.60797068)
\lineto(329.568,125.91197068)
\curveto(330.91199866,125.86397073)(331.20000048,125.76796941)(331.68,124.49597068)
\curveto(332.44799923,122.4799727)(333.33600084,120.07996828)(334.176,117.67997068)
\lineto(336.888,109.99997068)
\lineto(338.592,109.99997068)
\lineto(341.664,118.27997068)
\curveto(342.64799902,120.94396802)(343.60800058,123.39197196)(344.184,124.66397068)
\curveto(344.6879995,125.79196956)(344.9760013,125.83997076)(346.272,125.91197068)
\lineto(346.272,126.60797068)
\lineto(340.968,126.60797068)
\lineto(340.968,125.91197068)
\lineto(342.288,125.79197068)
\curveto(342.91199938,125.74397073)(342.93599988,125.50397023)(342.816,125.04797068)
\curveto(342.55200026,123.99197174)(341.80799914,121.87996828)(340.944,119.47997068)
\lineto(338.448,112.59197068)
\lineto(338.4,112.63997068)
\lineto(336.168,118.80797068)
\curveto(335.42400074,120.8959686)(334.67999938,123.0079726)(334.056,124.92797068)
\curveto(333.86400019,125.52797008)(333.93600058,125.74397073)(334.512,125.79197068)
\lineto(335.76,125.91197068)
\lineto(335.76,126.60797068)
\lineto(329.568,126.60797068)
}
}
{
\newrgbcolor{curcolor}{0 0 0}
\pscustom[linestyle=none,fillstyle=solid,fillcolor=curcolor]
{
\newpath
\moveto(353.466375,113.43197068)
\curveto(353.466375,111.48797263)(351.97837406,110.98397068)(351.042375,110.98397068)
\curveto(349.55437649,110.98397068)(348.834375,112.0399721)(348.834375,113.45597068)
\curveto(348.834375,114.58396956)(349.36237634,115.15997119)(350.706375,115.66397068)
\curveto(351.66637404,116.02397032)(352.93837553,116.47997102)(353.466375,116.81597068)
\lineto(353.466375,113.43197068)
\moveto(355.626375,118.85597068)
\curveto(355.626375,120.24796929)(355.31437138,122.04797068)(351.690375,122.04797068)
\curveto(348.97837771,122.04797068)(346.842375,120.63196936)(346.842375,119.31197068)
\curveto(346.842375,118.54397145)(347.73037546,118.18397068)(348.186375,118.18397068)
\curveto(348.6903745,118.18397068)(348.83437512,118.44797109)(348.954375,118.85597068)
\curveto(349.48237447,120.63196891)(350.46637596,121.20797068)(351.426375,121.20797068)
\curveto(352.36237406,121.20797068)(353.466375,120.72796876)(353.466375,118.80797068)
\lineto(353.466375,117.79997068)
\curveto(352.8663756,117.17597131)(350.5383731,116.57597008)(348.642375,115.97597068)
\curveto(346.91437673,115.44797121)(346.410375,114.24796956)(346.410375,113.11997068)
\curveto(346.410375,111.31997248)(347.61037726,109.71197068)(349.866375,109.71197068)
\curveto(351.35437351,109.75997064)(352.67437577,110.64797119)(353.442375,111.15197068)
\curveto(353.77837466,110.26397157)(354.16237584,109.71197068)(355.002375,109.71197068)
\curveto(355.89037411,109.71197068)(356.92237591,109.97597114)(357.834375,110.43197068)
\lineto(357.690375,111.00797068)
\curveto(357.35437534,110.93597076)(356.82637464,110.88797078)(356.466375,110.98397068)
\curveto(356.03437543,111.07997059)(355.626375,111.5359722)(355.626375,113.04797068)
\lineto(355.626375,118.85597068)
}
}
{
\newrgbcolor{curcolor}{0 0 0}
\pscustom[linestyle=none,fillstyle=solid,fillcolor=curcolor]
{
\newpath
\moveto(362.332875,127.66397068)
\lineto(362.164875,127.80797068)
\lineto(358.396875,127.20797068)
\lineto(358.396875,126.60797068)
\lineto(359.308875,126.51197068)
\curveto(359.93287438,126.43997076)(360.076875,126.31996977)(360.076875,125.40797068)
\lineto(360.076875,112.08797068)
\curveto(360.076875,110.86397191)(360.00487327,110.81597056)(358.276875,110.69597068)
\lineto(358.276875,109.99997068)
\lineto(364.132875,109.99997068)
\lineto(364.132875,110.69597068)
\curveto(362.4288767,110.81597056)(362.332875,110.86397191)(362.332875,112.08797068)
\lineto(362.332875,127.66397068)
}
}
{
\newrgbcolor{curcolor}{0 0 0}
\pscustom[linestyle=none,fillstyle=solid,fillcolor=curcolor]
{
\newpath
\moveto(365.388,121.44797068)
\lineto(365.388,120.84797068)
\lineto(366.3,120.72797068)
\curveto(366.92399938,120.63197078)(367.068,120.51196982)(367.068,119.64797068)
\lineto(367.068,112.08797068)
\curveto(367.068,110.86397191)(366.99599827,110.81597056)(365.268,110.69597068)
\lineto(365.268,109.99997068)
\lineto(371.124,109.99997068)
\lineto(371.124,110.69597068)
\curveto(369.4200017,110.81597056)(369.324,110.86397191)(369.324,112.08797068)
\lineto(369.324,121.90397068)
\lineto(369.156,122.04797068)
\lineto(365.388,121.44797068)
\moveto(368.172,127.15997068)
\curveto(367.30800086,127.15997068)(366.708,126.53596982)(366.708,125.67197068)
\curveto(366.708,124.83197152)(367.30800086,124.23197068)(368.172,124.23197068)
\curveto(369.05999911,124.23197068)(369.61200002,124.83197152)(369.636,125.67197068)
\curveto(369.636,126.53596982)(369.05999911,127.15997068)(368.172,127.15997068)
}
}
{
\newrgbcolor{curcolor}{0 0 0}
\pscustom[linestyle=none,fillstyle=solid,fillcolor=curcolor]
{
\newpath
\moveto(385.476375,109.99997068)
\lineto(385.476375,110.69597068)
\curveto(383.7723767,110.81597056)(383.676375,110.86397191)(383.676375,112.08797068)
\lineto(383.676375,127.66397068)
\lineto(383.508375,127.80797068)
\lineto(379.740375,127.20797068)
\lineto(379.740375,126.60797068)
\lineto(380.652375,126.51197068)
\curveto(381.27637438,126.43997076)(381.420375,126.31996977)(381.420375,125.40797068)
\lineto(381.420375,121.47197068)
\curveto(380.74837567,121.83197032)(379.8603739,122.04797068)(378.756375,122.04797068)
\curveto(376.78837697,122.04797068)(375.32437399,121.42396965)(374.316375,120.39197068)
\curveto(373.28437603,119.31197176)(372.660375,117.7279686)(372.660375,115.63997068)
\curveto(372.660375,112.08797424)(374.50837817,109.71197068)(377.676375,109.71197068)
\curveto(378.9723737,109.71197068)(380.1003763,110.28797174)(381.396375,111.34397068)
\lineto(381.396375,110.16797068)
\lineto(381.588375,109.99997068)
\lineto(385.476375,109.99997068)
\moveto(381.420375,114.07997068)
\curveto(381.420375,113.55197121)(381.39637486,113.14397028)(381.252375,112.73597068)
\curveto(380.77237548,111.48797193)(379.71637375,111.00797068)(378.468375,111.00797068)
\curveto(376.26037721,111.00797068)(375.156375,113.16797344)(375.156375,115.92797068)
\curveto(375.156375,119.04796756)(376.28437745,121.20797068)(378.732375,121.20797068)
\curveto(379.86037387,121.20797068)(380.77237543,120.6799697)(381.204375,119.69597068)
\curveto(381.39637481,119.28797109)(381.420375,118.95197004)(381.420375,118.30397068)
\lineto(381.420375,114.07997068)
}
}
{
\newrgbcolor{curcolor}{0 0 0}
\pscustom[linestyle=none,fillstyle=solid,fillcolor=curcolor]
{
\newpath
\moveto(396.412125,116.31197068)
\curveto(396.9161245,116.31197068)(397.372125,116.43197148)(397.372125,117.22397068)
\curveto(397.372125,118.63996927)(396.91612061,122.04797068)(392.524125,122.04797068)
\curveto(388.78012874,122.04797068)(387.004125,119.38396708)(387.004125,115.78397068)
\curveto(387.004125,111.99197448)(388.63612891,109.66397073)(392.548125,109.71197068)
\curveto(395.21212234,109.73597066)(396.58012565,111.19997258)(397.228125,113.09597068)
\lineto(396.508125,113.47997068)
\curveto(395.83612567,112.08797208)(394.9241231,110.93597068)(393.028125,110.93597068)
\curveto(390.05212798,110.93597068)(389.45212505,113.81597318)(389.500125,116.31197068)
\lineto(396.412125,116.31197068)
\moveto(389.524125,117.22397068)
\curveto(389.524125,118.2079697)(389.8841275,121.20797068)(392.380125,121.20797068)
\curveto(394.61212277,121.20797068)(394.876125,118.92796975)(394.876125,117.99197068)
\curveto(394.876125,117.53597114)(394.73212433,117.22397068)(394.060125,117.22397068)
\lineto(389.524125,117.22397068)
}
}
{
\newrgbcolor{curcolor}{0 0 0}
\pscustom[linestyle=none,fillstyle=solid,fillcolor=curcolor]
{
\newpath
\moveto(402.69225,122.04797068)
\lineto(399.04425,121.44797068)
\lineto(399.04425,120.84797068)
\lineto(399.95625,120.72797068)
\curveto(400.58024938,120.63197078)(400.72425,120.51196982)(400.72425,119.64797068)
\lineto(400.72425,112.08797068)
\curveto(400.72425,110.86397191)(400.60424832,110.81597056)(398.92425,110.69597068)
\lineto(398.92425,109.99997068)
\lineto(405.26025,109.99997068)
\lineto(405.26025,110.69597068)
\curveto(403.12425214,110.81597056)(402.98025,110.86397191)(402.98025,112.08797068)
\lineto(402.98025,117.70397068)
\curveto(402.98025,119.55196884)(403.79625055,120.17597068)(404.34825,120.17597068)
\curveto(404.73224962,120.17597068)(405.1402507,120.0319703)(405.83625,119.64797068)
\curveto(406.00424983,119.55197078)(406.19625012,119.52797068)(406.31625,119.52797068)
\curveto(406.89224942,119.52797068)(407.42025,120.12797143)(407.42025,120.87197068)
\curveto(407.42025,121.39997016)(407.08424906,122.04797068)(406.14825,122.04797068)
\curveto(405.28425086,122.04797068)(404.56424842,121.51996929)(402.98025,120.12797068)
\lineto(402.69225,122.04797068)
}
}
{
\newrgbcolor{curcolor}{0 0 0}
\pscustom[linestyle=none,fillstyle=solid,fillcolor=curcolor]
{
\newpath
\moveto(165.68358765,109.88037229)
\lineto(165.68358765,110.57637229)
\curveto(164.29158904,110.69637217)(164.02758721,110.76837354)(163.59558765,112.01637229)
\lineto(158.50758765,126.48837229)
\lineto(156.73158765,126.48837229)
\lineto(154.21158765,119.38437229)
\curveto(153.46758839,117.29637438)(152.45958681,114.41637011)(151.61958765,112.23237229)
\curveto(151.11558815,110.93637359)(150.89958613,110.64837222)(149.38758765,110.57637229)
\lineto(149.38758765,109.88037229)
\lineto(154.66758765,109.88037229)
\lineto(154.66758765,110.57637229)
\lineto(153.41958765,110.69637229)
\curveto(152.69958837,110.76837222)(152.62758784,111.00837294)(152.81958765,111.65637229)
\curveto(153.22758724,113.09637085)(153.7315882,114.584374)(154.28358765,116.28837229)
\lineto(159.65958765,116.28837229)
\lineto(161.26758765,111.70437229)
\curveto(161.50758741,111.00837299)(161.38758688,110.74437222)(160.61958765,110.67237229)
\lineto(159.53958765,110.57637229)
\lineto(159.53958765,109.88037229)
\lineto(165.68358765,109.88037229)
\moveto(159.37158765,117.20037229)
\lineto(154.59558765,117.20037229)
\curveto(155.36358688,119.60036989)(156.20358837,121.95237441)(156.92358765,124.06437229)
\lineto(156.99558765,124.06437229)
\lineto(159.37158765,117.20037229)
}
}
{
\newrgbcolor{curcolor}{0 0 0}
\pscustom[linestyle=none,fillstyle=solid,fillcolor=curcolor]
{
\newpath
\moveto(178.01396265,118.11237229)
\curveto(178.01396265,120.53636987)(176.69396044,121.92837229)(174.48596265,121.92837229)
\curveto(172.70996442,121.92837229)(171.60596116,121.06437133)(170.11796265,120.10437229)
\lineto(169.78196265,121.92837229)
\lineto(166.20596265,121.32837229)
\lineto(166.20596265,120.72837229)
\lineto(167.11796265,120.60837229)
\curveto(167.74196202,120.51237239)(167.88596265,120.39237143)(167.88596265,119.52837229)
\lineto(167.88596265,111.96837229)
\curveto(167.88596265,110.74437352)(167.81396092,110.69637217)(166.08596265,110.57637229)
\lineto(166.08596265,109.88037229)
\lineto(171.94196265,109.88037229)
\lineto(171.94196265,110.57637229)
\curveto(170.23796435,110.69637217)(170.14196265,110.74437352)(170.14196265,111.96837229)
\lineto(170.14196265,117.58437229)
\curveto(170.14196265,118.18437169)(170.18996284,118.52037268)(170.38196265,118.90437229)
\curveto(170.88596214,119.84037136)(171.96596392,120.63237229)(173.23796265,120.63237229)
\curveto(174.86996101,120.63237229)(175.75796265,119.72037021)(175.75796265,117.63237229)
\lineto(175.75796265,111.96837229)
\curveto(175.75796265,110.74437352)(175.68596092,110.69637217)(173.95796265,110.57637229)
\lineto(173.95796265,109.88037229)
\lineto(179.81396265,109.88037229)
\lineto(179.81396265,110.57637229)
\curveto(178.10996435,110.69637217)(178.01396265,110.74437352)(178.01396265,111.96837229)
\lineto(178.01396265,118.11237229)
}
}
{
\newrgbcolor{curcolor}{0 0 0}
\pscustom[linestyle=none,fillstyle=solid,fillcolor=curcolor]
{
\newpath
\moveto(192.87333765,118.11237229)
\curveto(192.87333765,120.53636987)(191.55333544,121.92837229)(189.34533765,121.92837229)
\curveto(187.56933942,121.92837229)(186.46533616,121.06437133)(184.97733765,120.10437229)
\lineto(184.64133765,121.92837229)
\lineto(181.06533765,121.32837229)
\lineto(181.06533765,120.72837229)
\lineto(181.97733765,120.60837229)
\curveto(182.60133702,120.51237239)(182.74533765,120.39237143)(182.74533765,119.52837229)
\lineto(182.74533765,111.96837229)
\curveto(182.74533765,110.74437352)(182.67333592,110.69637217)(180.94533765,110.57637229)
\lineto(180.94533765,109.88037229)
\lineto(186.80133765,109.88037229)
\lineto(186.80133765,110.57637229)
\curveto(185.09733935,110.69637217)(185.00133765,110.74437352)(185.00133765,111.96837229)
\lineto(185.00133765,117.58437229)
\curveto(185.00133765,118.18437169)(185.04933784,118.52037268)(185.24133765,118.90437229)
\curveto(185.74533714,119.84037136)(186.82533892,120.63237229)(188.09733765,120.63237229)
\curveto(189.72933601,120.63237229)(190.61733765,119.72037021)(190.61733765,117.63237229)
\lineto(190.61733765,111.96837229)
\curveto(190.61733765,110.74437352)(190.54533592,110.69637217)(188.81733765,110.57637229)
\lineto(188.81733765,109.88037229)
\lineto(194.67333765,109.88037229)
\lineto(194.67333765,110.57637229)
\curveto(192.96933935,110.69637217)(192.87333765,110.74437352)(192.87333765,111.96837229)
\lineto(192.87333765,118.11237229)
}
}
{
\newrgbcolor{curcolor}{0 0 0}
\pscustom[linestyle=none,fillstyle=solid,fillcolor=curcolor]
{
\newpath
\moveto(209.22071265,109.88037229)
\lineto(209.22071265,110.57637229)
\curveto(207.5647143,110.69637217)(207.42071265,110.74437352)(207.42071265,111.96837229)
\lineto(207.42071265,121.78437229)
\lineto(207.25271265,121.92837229)
\lineto(203.48471265,121.32837229)
\lineto(203.48471265,120.72837229)
\lineto(204.39671265,120.60837229)
\curveto(205.02071202,120.51237239)(205.16471265,120.39237143)(205.16471265,119.52837229)
\lineto(205.16471265,114.12837229)
\curveto(205.16471265,113.45637297)(205.09271253,112.92837205)(204.97271265,112.68837229)
\curveto(204.4207132,111.5843734)(203.2687114,110.88837229)(202.02071265,110.88837229)
\curveto(200.62871404,110.88837229)(199.59671265,111.75237417)(199.59671265,113.62437229)
\lineto(199.59671265,121.78437229)
\lineto(199.42871265,121.92837229)
\lineto(195.66071265,121.32837229)
\lineto(195.66071265,120.72837229)
\lineto(196.57271265,120.60837229)
\curveto(197.19671202,120.51237239)(197.34071265,120.39237143)(197.34071265,119.52837229)
\lineto(197.34071265,113.19237229)
\curveto(197.34071265,110.43237505)(198.97271442,109.59237229)(200.74871265,109.59237229)
\curveto(202.78871061,109.59237229)(204.44471334,110.98437263)(205.14071265,111.32037229)
\lineto(205.38071265,109.88037229)
\lineto(209.22071265,109.88037229)
}
}
{
\newrgbcolor{curcolor}{0 0 0}
\pscustom[linestyle=none,fillstyle=solid,fillcolor=curcolor]
{
\newpath
\moveto(214.10396265,127.54437229)
\lineto(213.93596265,127.68837229)
\lineto(210.16796265,127.08837229)
\lineto(210.16796265,126.48837229)
\lineto(211.07996265,126.39237229)
\curveto(211.70396202,126.32037237)(211.84796265,126.20037138)(211.84796265,125.28837229)
\lineto(211.84796265,111.96837229)
\curveto(211.84796265,110.74437352)(211.77596092,110.69637217)(210.04796265,110.57637229)
\lineto(210.04796265,109.88037229)
\lineto(215.90396265,109.88037229)
\lineto(215.90396265,110.57637229)
\curveto(214.19996435,110.69637217)(214.10396265,110.74437352)(214.10396265,111.96837229)
\lineto(214.10396265,127.54437229)
}
}
{
\newrgbcolor{curcolor}{0 0 0}
\pscustom[linestyle=none,fillstyle=solid,fillcolor=curcolor]
{
\newpath
\moveto(226.85508765,116.19237229)
\curveto(227.35908714,116.19237229)(227.81508765,116.31237309)(227.81508765,117.10437229)
\curveto(227.81508765,118.52037088)(227.35908325,121.92837229)(222.96708765,121.92837229)
\curveto(219.22309139,121.92837229)(217.44708765,119.26436869)(217.44708765,115.66437229)
\curveto(217.44708765,111.87237609)(219.07909156,109.54437234)(222.99108765,109.59237229)
\curveto(225.65508498,109.61637227)(227.02308829,111.08037419)(227.67108765,112.97637229)
\lineto(226.95108765,113.36037229)
\curveto(226.27908832,111.96837369)(225.36708575,110.81637229)(223.47108765,110.81637229)
\curveto(220.49509062,110.81637229)(219.89508769,113.69637479)(219.94308765,116.19237229)
\lineto(226.85508765,116.19237229)
\moveto(219.96708765,117.10437229)
\curveto(219.96708765,118.08837131)(220.32709014,121.08837229)(222.82308765,121.08837229)
\curveto(225.05508541,121.08837229)(225.31908765,118.80837136)(225.31908765,117.87237229)
\curveto(225.31908765,117.41637275)(225.17508697,117.10437229)(224.50308765,117.10437229)
\lineto(219.96708765,117.10437229)
}
}
{
\newrgbcolor{curcolor}{0 0 0}
\pscustom[linestyle=none,fillstyle=solid,fillcolor=curcolor]
{
\newpath
\moveto(233.13521265,121.92837229)
\lineto(229.48721265,121.32837229)
\lineto(229.48721265,120.72837229)
\lineto(230.39921265,120.60837229)
\curveto(231.02321202,120.51237239)(231.16721265,120.39237143)(231.16721265,119.52837229)
\lineto(231.16721265,111.96837229)
\curveto(231.16721265,110.74437352)(231.04721097,110.69637217)(229.36721265,110.57637229)
\lineto(229.36721265,109.88037229)
\lineto(235.70321265,109.88037229)
\lineto(235.70321265,110.57637229)
\curveto(233.56721478,110.69637217)(233.42321265,110.74437352)(233.42321265,111.96837229)
\lineto(233.42321265,117.58437229)
\curveto(233.42321265,119.43237045)(234.2392132,120.05637229)(234.79121265,120.05637229)
\curveto(235.17521226,120.05637229)(235.58321334,119.91237191)(236.27921265,119.52837229)
\curveto(236.44721248,119.43237239)(236.63921277,119.40837229)(236.75921265,119.40837229)
\curveto(237.33521207,119.40837229)(237.86321265,120.00837304)(237.86321265,120.75237229)
\curveto(237.86321265,121.28037177)(237.52721171,121.92837229)(236.59121265,121.92837229)
\curveto(235.72721351,121.92837229)(235.00721106,121.4003709)(233.42321265,120.00837229)
\lineto(233.13521265,121.92837229)
}
}
{
\newrgbcolor{curcolor}{1 1 1}
\pscustom[linestyle=none,fillstyle=solid,fillcolor=curcolor]
{
\newpath
\moveto(149.98284912,480.0171578)
\lineto(329.94728088,480.0171578)
\lineto(329.94728088,450.20041967)
\lineto(149.98284912,450.20041967)
\lineto(149.98284912,480.0171578)
\closepath
}
}
{
\newrgbcolor{curcolor}{0 0 0}
\pscustom[linewidth=2,linecolor=curcolor]
{
\newpath
\moveto(149.98284912,480.0171578)
\lineto(329.94728088,480.0171578)
\lineto(329.94728088,450.20041967)
\lineto(149.98284912,450.20041967)
\lineto(149.98284912,480.0171578)
\closepath
}
}
{
\newrgbcolor{curcolor}{0 0 0}
\pscustom[linestyle=none,fillstyle=solid,fillcolor=curcolor]
{
}
}
{
\newrgbcolor{curcolor}{0 0 0}
\pscustom[linestyle=none,fillstyle=solid,fillcolor=curcolor]
{
}
}
{
\newrgbcolor{curcolor}{0 0 0}
\pscustom[linestyle=none,fillstyle=solid,fillcolor=curcolor]
{
}
}
{
\newrgbcolor{curcolor}{0 0 0}
\pscustom[linestyle=none,fillstyle=solid,fillcolor=curcolor]
{
}
}
{
\newrgbcolor{curcolor}{0 0 0}
\pscustom[linestyle=none,fillstyle=solid,fillcolor=curcolor]
{
}
}
{
\newrgbcolor{curcolor}{0 0 0}
\pscustom[linestyle=none,fillstyle=solid,fillcolor=curcolor]
{
}
}
{
\newrgbcolor{curcolor}{0 0 0}
\pscustom[linestyle=none,fillstyle=solid,fillcolor=curcolor]
{
}
}
{
\newrgbcolor{curcolor}{0 0 0}
\pscustom[linestyle=none,fillstyle=solid,fillcolor=curcolor]
{
\newpath
\moveto(168.550375,376.6080012)
\lineto(168.550375,375.9120012)
\curveto(170.66237289,375.76800135)(170.710375,375.71999986)(170.710375,374.3760012)
\lineto(170.710375,364.8720012)
\curveto(170.710375,363.60000247)(170.66237488,362.68800041)(170.542375,361.8960012)
\curveto(170.39837514,360.88800221)(169.82237375,360.79200111)(168.574375,360.6960012)
\lineto(168.574375,360.0000012)
\lineto(174.262375,360.0000012)
\lineto(174.262375,360.6960012)
\curveto(172.70237656,360.81600108)(172.12637486,360.88800221)(171.982375,361.8960012)
\curveto(171.86237512,362.68800041)(171.814375,363.60000247)(171.814375,364.8720012)
\lineto(171.814375,374.4480012)
\lineto(171.862375,374.4480012)
\curveto(175.29437157,369.62400603)(178.70237838,364.82399638)(182.086375,360.0000012)
\lineto(183.670375,360.0000012)
\lineto(183.670375,371.7360012)
\curveto(183.670375,373.00799993)(183.71837512,373.92000199)(183.838375,374.7120012)
\curveto(183.98237486,375.72000019)(184.55837625,375.8160013)(185.806375,375.9120012)
\lineto(185.806375,376.6080012)
\lineto(180.118375,376.6080012)
\lineto(180.118375,375.9120012)
\curveto(181.67837344,375.79200132)(182.25437514,375.72000019)(182.398375,374.7120012)
\curveto(182.51837488,373.92000199)(182.566375,373.00799993)(182.566375,371.7360012)
\lineto(182.566375,363.3120012)
\lineto(182.518375,363.3600012)
\curveto(179.37437814,367.77599679)(176.2543719,372.19200562)(173.158375,376.6080012)
\lineto(168.550375,376.6080012)
}
}
{
\newrgbcolor{curcolor}{0 0 0}
\pscustom[linestyle=none,fillstyle=solid,fillcolor=curcolor]
{
\newpath
\moveto(193.3495,372.0480012)
\curveto(189.50950384,372.0480012)(187.6135,369.76799731)(187.6135,365.8800012)
\curveto(187.6135,361.99200509)(189.50950384,359.7120012)(193.3495,359.7120012)
\curveto(197.23749611,359.7120012)(199.1095,361.99200509)(199.1095,365.8800012)
\curveto(199.1095,369.76799731)(197.23749611,372.0480012)(193.3495,372.0480012)
\moveto(190.1095,365.8800012)
\curveto(190.1095,369.19199789)(191.16550218,371.2080012)(193.3495,371.2080012)
\curveto(195.58149777,371.2080012)(196.6135,369.19199789)(196.6135,365.8800012)
\curveto(196.6135,362.56800451)(195.58149777,360.5520012)(193.3495,360.5520012)
\curveto(191.16550218,360.5520012)(190.1095,362.56800451)(190.1095,365.8800012)
}
}
{
\newrgbcolor{curcolor}{0 0 0}
\pscustom[linestyle=none,fillstyle=solid,fillcolor=curcolor]
{
\newpath
\moveto(201.153625,371.4480012)
\lineto(201.153625,370.8480012)
\lineto(202.065625,370.7280012)
\curveto(202.68962438,370.6320013)(202.833625,370.51200034)(202.833625,369.6480012)
\lineto(202.833625,362.0880012)
\curveto(202.833625,360.86400243)(202.76162327,360.81600108)(201.033625,360.6960012)
\lineto(201.033625,360.0000012)
\lineto(206.889625,360.0000012)
\lineto(206.889625,360.6960012)
\curveto(205.1856267,360.81600108)(205.089625,360.86400243)(205.089625,362.0880012)
\lineto(205.089625,367.7040012)
\curveto(205.089625,368.3040006)(205.11362517,368.64000154)(205.281625,368.9760012)
\curveto(205.73762454,369.93600024)(206.79362632,370.7520012)(208.113625,370.7520012)
\curveto(210.17762294,370.7520012)(210.417625,369.23999971)(210.417625,367.7520012)
\lineto(210.417625,362.0880012)
\curveto(210.417625,360.86400243)(210.34562327,360.81600108)(208.617625,360.6960012)
\lineto(208.617625,360.0000012)
\lineto(214.473625,360.0000012)
\lineto(214.473625,360.6960012)
\curveto(212.7696267,360.81600108)(212.673625,360.86400243)(212.673625,362.0880012)
\lineto(212.673625,367.7040012)
\curveto(212.673625,368.3040006)(212.69762514,368.64000159)(212.841625,369.0240012)
\curveto(213.20162464,369.98400024)(214.23362634,370.7520012)(215.577625,370.7520012)
\curveto(217.08962349,370.80000115)(218.001625,369.83999911)(218.001625,367.7520012)
\lineto(218.001625,362.0880012)
\curveto(218.001625,360.86400243)(217.92962327,360.81600108)(216.201625,360.6960012)
\lineto(216.201625,360.0000012)
\lineto(222.057625,360.0000012)
\lineto(222.057625,360.6960012)
\curveto(220.3536267,360.81600108)(220.257625,360.86400243)(220.257625,362.0880012)
\lineto(220.257625,368.2320012)
\curveto(220.257625,370.65599878)(218.96162272,372.0480012)(216.681625,372.0480012)
\curveto(214.71362697,372.0480012)(213.29762416,370.72800067)(212.457625,370.2000012)
\curveto(211.88162558,371.32800007)(210.96962332,372.0480012)(209.289625,372.0480012)
\curveto(207.44162685,372.0480012)(205.92962416,370.84800058)(205.089625,370.2240012)
\lineto(204.873625,372.0480012)
\lineto(201.153625,371.4480012)
}
}
{
\newrgbcolor{curcolor}{0 0 0}
\pscustom[linestyle=none,fillstyle=solid,fillcolor=curcolor]
{
}
}
{
\newrgbcolor{curcolor}{0 0 0}
\pscustom[linestyle=none,fillstyle=solid,fillcolor=curcolor]
{
\newpath
\moveto(143.992,330.0000012)
\lineto(143.992,330.6960012)
\curveto(142.2880017,330.81600108)(142.192,330.86400243)(142.192,332.0880012)
\lineto(142.192,347.6640012)
\lineto(142.024,347.8080012)
\lineto(138.256,347.2080012)
\lineto(138.256,346.6080012)
\lineto(139.168,346.5120012)
\curveto(139.79199938,346.44000127)(139.936,346.32000029)(139.936,345.4080012)
\lineto(139.936,341.4720012)
\curveto(139.26400067,341.83200084)(138.3759989,342.0480012)(137.272,342.0480012)
\curveto(135.30400197,342.0480012)(133.83999899,341.42400017)(132.832,340.3920012)
\curveto(131.80000103,339.31200228)(131.176,337.72799911)(131.176,335.6400012)
\curveto(131.176,332.08800475)(133.02400317,329.7120012)(136.192,329.7120012)
\curveto(137.4879987,329.7120012)(138.6160013,330.28800226)(139.912,331.3440012)
\lineto(139.912,330.1680012)
\lineto(140.104,330.0000012)
\lineto(143.992,330.0000012)
\moveto(139.936,334.0800012)
\curveto(139.936,333.55200173)(139.91199986,333.14400079)(139.768,332.7360012)
\curveto(139.28800048,331.48800245)(138.23199875,331.0080012)(136.984,331.0080012)
\curveto(134.77600221,331.0080012)(133.672,333.16800396)(133.672,335.9280012)
\curveto(133.672,339.04799808)(134.80000245,341.2080012)(137.248,341.2080012)
\curveto(138.37599887,341.2080012)(139.28800043,340.68000022)(139.72,339.6960012)
\curveto(139.91199981,339.28800161)(139.936,338.95200055)(139.936,338.3040012)
\lineto(139.936,334.0800012)
}
}
{
\newrgbcolor{curcolor}{0 0 0}
\pscustom[linestyle=none,fillstyle=solid,fillcolor=curcolor]
{
\newpath
\moveto(147.91975,341.1360012)
\lineto(148.75975,346.5360012)
\curveto(148.87974988,347.40000034)(148.56774911,347.8080012)(147.67975,347.8080012)
\curveto(146.79175089,347.8080012)(146.47975012,347.40000034)(146.59975,346.5360012)
\lineto(147.43975,341.1360012)
\lineto(147.91975,341.1360012)
}
}
{
\newrgbcolor{curcolor}{0 0 0}
\pscustom[linestyle=none,fillstyle=solid,fillcolor=curcolor]
{
\newpath
\moveto(165.184,330.0000012)
\lineto(165.184,330.6960012)
\curveto(163.52800166,330.81600108)(163.384,330.86400243)(163.384,332.0880012)
\lineto(163.384,341.9040012)
\lineto(163.216,342.0480012)
\lineto(159.448,341.4480012)
\lineto(159.448,340.8480012)
\lineto(160.36,340.7280012)
\curveto(160.98399938,340.6320013)(161.128,340.51200034)(161.128,339.6480012)
\lineto(161.128,334.2480012)
\curveto(161.128,333.57600187)(161.05599988,333.04800096)(160.936,332.8080012)
\curveto(160.38400055,331.70400231)(159.23199875,331.0080012)(157.984,331.0080012)
\curveto(156.59200139,331.0080012)(155.56,331.87200307)(155.56,333.7440012)
\lineto(155.56,341.9040012)
\lineto(155.392,342.0480012)
\lineto(151.624,341.4480012)
\lineto(151.624,340.8480012)
\lineto(152.536,340.7280012)
\curveto(153.15999938,340.6320013)(153.304,340.51200034)(153.304,339.6480012)
\lineto(153.304,333.3120012)
\curveto(153.304,330.55200396)(154.93600178,329.7120012)(156.712,329.7120012)
\curveto(158.75199796,329.7120012)(160.4080007,331.10400154)(161.104,331.4400012)
\lineto(161.344,330.0000012)
\lineto(165.184,330.0000012)
}
}
{
\newrgbcolor{curcolor}{0 0 0}
\pscustom[linestyle=none,fillstyle=solid,fillcolor=curcolor]
{
\newpath
\moveto(173.35525,340.8000012)
\lineto(173.35525,341.7600012)
\lineto(170.25925,341.7600012)
\lineto(170.25925,344.7840012)
\lineto(169.56325,344.7840012)
\lineto(168.09925,341.7600012)
\lineto(166.27525,341.7600012)
\lineto(166.27525,340.8000012)
\lineto(168.00325,340.8000012)
\lineto(168.00325,332.4480012)
\curveto(168.00325,330.00000365)(169.82725084,329.7120012)(170.66725,329.7120012)
\curveto(171.89124878,329.7120012)(173.0432507,330.33600163)(173.73925,330.7680012)
\lineto(173.54725,331.2720012)
\curveto(172.97125058,331.03200144)(172.4432494,330.9600012)(171.84325,330.9600012)
\curveto(171.02725082,330.9600012)(170.25925,331.53600303)(170.25925,333.3600012)
\lineto(170.25925,340.8000012)
\lineto(173.35525,340.8000012)
}
}
{
\newrgbcolor{curcolor}{0 0 0}
\pscustom[linestyle=none,fillstyle=solid,fillcolor=curcolor]
{
\newpath
\moveto(174.66925,341.4480012)
\lineto(174.66925,340.8480012)
\lineto(175.58125,340.7280012)
\curveto(176.20524938,340.6320013)(176.34925,340.51200034)(176.34925,339.6480012)
\lineto(176.34925,332.0880012)
\curveto(176.34925,330.86400243)(176.27724827,330.81600108)(174.54925,330.6960012)
\lineto(174.54925,330.0000012)
\lineto(180.40525,330.0000012)
\lineto(180.40525,330.6960012)
\curveto(178.7012517,330.81600108)(178.60525,330.86400243)(178.60525,332.0880012)
\lineto(178.60525,341.9040012)
\lineto(178.43725,342.0480012)
\lineto(174.66925,341.4480012)
\moveto(177.45325,347.1600012)
\curveto(176.58925086,347.1600012)(175.98925,346.53600034)(175.98925,345.6720012)
\curveto(175.98925,344.83200204)(176.58925086,344.2320012)(177.45325,344.2320012)
\curveto(178.34124911,344.2320012)(178.89325002,344.83200204)(178.91725,345.6720012)
\curveto(178.91725,346.53600034)(178.34124911,347.1600012)(177.45325,347.1600012)
}
}
{
\newrgbcolor{curcolor}{0 0 0}
\pscustom[linestyle=none,fillstyle=solid,fillcolor=curcolor]
{
\newpath
\moveto(185.301625,347.6640012)
\lineto(185.133625,347.8080012)
\lineto(181.365625,347.2080012)
\lineto(181.365625,346.6080012)
\lineto(182.277625,346.5120012)
\curveto(182.90162438,346.44000127)(183.045625,346.32000029)(183.045625,345.4080012)
\lineto(183.045625,332.0880012)
\curveto(183.045625,330.86400243)(182.97362327,330.81600108)(181.245625,330.6960012)
\lineto(181.245625,330.0000012)
\lineto(187.101625,330.0000012)
\lineto(187.101625,330.6960012)
\curveto(185.3976267,330.81600108)(185.301625,330.86400243)(185.301625,332.0880012)
\lineto(185.301625,347.6640012)
}
}
{
\newrgbcolor{curcolor}{0 0 0}
\pscustom[linestyle=none,fillstyle=solid,fillcolor=curcolor]
{
\newpath
\moveto(188.35675,341.4480012)
\lineto(188.35675,340.8480012)
\lineto(189.26875,340.7280012)
\curveto(189.89274938,340.6320013)(190.03675,340.51200034)(190.03675,339.6480012)
\lineto(190.03675,332.0880012)
\curveto(190.03675,330.86400243)(189.96474827,330.81600108)(188.23675,330.6960012)
\lineto(188.23675,330.0000012)
\lineto(194.09275,330.0000012)
\lineto(194.09275,330.6960012)
\curveto(192.3887517,330.81600108)(192.29275,330.86400243)(192.29275,332.0880012)
\lineto(192.29275,341.9040012)
\lineto(192.12475,342.0480012)
\lineto(188.35675,341.4480012)
\moveto(191.14075,347.1600012)
\curveto(190.27675086,347.1600012)(189.67675,346.53600034)(189.67675,345.6720012)
\curveto(189.67675,344.83200204)(190.27675086,344.2320012)(191.14075,344.2320012)
\curveto(192.02874911,344.2320012)(192.58075002,344.83200204)(192.60475,345.6720012)
\curveto(192.60475,346.53600034)(192.02874911,347.1600012)(191.14075,347.1600012)
}
}
{
\newrgbcolor{curcolor}{0 0 0}
\pscustom[linestyle=none,fillstyle=solid,fillcolor=curcolor]
{
\newpath
\moveto(203.261125,338.4240012)
\lineto(203.261125,341.2320012)
\curveto(202.46912579,341.80800063)(201.24512392,342.0480012)(200.165125,342.0480012)
\curveto(197.57312759,342.0480012)(195.82112498,340.82399892)(195.797125,338.5440012)
\curveto(195.82112498,336.55200319)(197.42912699,335.6640006)(199.421125,335.0640012)
\curveto(200.50112392,334.72800154)(202.013125,334.17599957)(202.013125,332.5440012)
\curveto(202.013125,331.32000243)(201.05312373,330.5520012)(199.781125,330.5520012)
\curveto(197.83712694,330.5520012)(196.75712452,331.96800324)(196.277125,334.0080012)
\lineto(195.581125,334.0080012)
\lineto(195.821125,330.8640012)
\curveto(196.68512414,330.09600197)(198.14912642,329.7120012)(199.565125,329.7120012)
\curveto(202.42112214,329.7120012)(204.053125,331.22400324)(204.053125,333.2640012)
\curveto(204.053125,335.37599909)(202.75712253,336.33600197)(200.285125,337.1040012)
\curveto(199.30112598,337.41600089)(197.717125,337.92000255)(197.717125,339.2640012)
\curveto(197.74112498,340.51199995)(198.6771262,341.2080012)(199.877125,341.2080012)
\curveto(201.5811233,341.2080012)(202.39712517,339.88799974)(202.565125,338.4240012)
\lineto(203.261125,338.4240012)
}
}
{
\newrgbcolor{curcolor}{0 0 0}
\pscustom[linestyle=none,fillstyle=solid,fillcolor=curcolor]
{
\newpath
\moveto(213.13825,333.4320012)
\curveto(213.13825,331.48800315)(211.65024906,330.9840012)(210.71425,330.9840012)
\curveto(209.22625149,330.9840012)(208.50625,332.04000262)(208.50625,333.4560012)
\curveto(208.50625,334.58400007)(209.03425134,335.16000171)(210.37825,335.6640012)
\curveto(211.33824904,336.02400084)(212.61025053,336.48000154)(213.13825,336.8160012)
\lineto(213.13825,333.4320012)
\moveto(215.29825,338.8560012)
\curveto(215.29825,340.24799981)(214.98624638,342.0480012)(211.36225,342.0480012)
\curveto(208.65025271,342.0480012)(206.51425,340.63199988)(206.51425,339.3120012)
\curveto(206.51425,338.54400197)(207.40225046,338.1840012)(207.85825,338.1840012)
\curveto(208.3622495,338.1840012)(208.50625012,338.44800161)(208.62625,338.8560012)
\curveto(209.15424947,340.63199943)(210.13825096,341.2080012)(211.09825,341.2080012)
\curveto(212.03424906,341.2080012)(213.13825,340.72799928)(213.13825,338.8080012)
\lineto(213.13825,337.8000012)
\curveto(212.5382506,337.17600183)(210.2102481,336.5760006)(208.31425,335.9760012)
\curveto(206.58625173,335.44800173)(206.08225,334.24800007)(206.08225,333.1200012)
\curveto(206.08225,331.320003)(207.28225226,329.7120012)(209.53825,329.7120012)
\curveto(211.02624851,329.76000115)(212.34625077,330.64800171)(213.11425,331.1520012)
\curveto(213.45024966,330.26400209)(213.83425084,329.7120012)(214.67425,329.7120012)
\curveto(215.56224911,329.7120012)(216.59425091,329.97600166)(217.50625,330.4320012)
\lineto(217.36225,331.0080012)
\curveto(217.02625034,330.93600127)(216.49824964,330.8880013)(216.13825,330.9840012)
\curveto(215.70625043,331.08000111)(215.29825,331.53600271)(215.29825,333.0480012)
\lineto(215.29825,338.8560012)
}
}
{
\newrgbcolor{curcolor}{0 0 0}
\pscustom[linestyle=none,fillstyle=solid,fillcolor=curcolor]
{
\newpath
\moveto(225.29275,340.8000012)
\lineto(225.29275,341.7600012)
\lineto(222.19675,341.7600012)
\lineto(222.19675,344.7840012)
\lineto(221.50075,344.7840012)
\lineto(220.03675,341.7600012)
\lineto(218.21275,341.7600012)
\lineto(218.21275,340.8000012)
\lineto(219.94075,340.8000012)
\lineto(219.94075,332.4480012)
\curveto(219.94075,330.00000365)(221.76475084,329.7120012)(222.60475,329.7120012)
\curveto(223.82874878,329.7120012)(224.9807507,330.33600163)(225.67675,330.7680012)
\lineto(225.48475,331.2720012)
\curveto(224.90875058,331.03200144)(224.3807494,330.9600012)(223.78075,330.9600012)
\curveto(222.96475082,330.9600012)(222.19675,331.53600303)(222.19675,333.3600012)
\lineto(222.19675,340.8000012)
\lineto(225.29275,340.8000012)
}
}
{
\newrgbcolor{curcolor}{0 0 0}
\pscustom[linestyle=none,fillstyle=solid,fillcolor=curcolor]
{
\newpath
\moveto(236.30275,336.3120012)
\curveto(236.8067495,336.3120012)(237.26275,336.43200199)(237.26275,337.2240012)
\curveto(237.26275,338.63999979)(236.80674561,342.0480012)(232.41475,342.0480012)
\curveto(228.67075374,342.0480012)(226.89475,339.3839976)(226.89475,335.7840012)
\curveto(226.89475,331.99200499)(228.52675391,329.66400125)(232.43875,329.7120012)
\curveto(235.10274734,329.73600118)(236.47075065,331.2000031)(237.11875,333.0960012)
\lineto(236.39875,333.4800012)
\curveto(235.72675067,332.08800259)(234.8147481,330.9360012)(232.91875,330.9360012)
\curveto(229.94275298,330.9360012)(229.34275005,333.8160037)(229.39075,336.3120012)
\lineto(236.30275,336.3120012)
\moveto(229.41475,337.2240012)
\curveto(229.41475,338.20800022)(229.7747525,341.2080012)(232.27075,341.2080012)
\curveto(234.50274777,341.2080012)(234.76675,338.92800027)(234.76675,337.9920012)
\curveto(234.76675,337.53600166)(234.62274933,337.2240012)(233.95075,337.2240012)
\lineto(229.41475,337.2240012)
}
}
{
\newrgbcolor{curcolor}{0 0 0}
\pscustom[linestyle=none,fillstyle=solid,fillcolor=curcolor]
{
\newpath
\moveto(252.230875,330.0000012)
\lineto(252.230875,330.6960012)
\curveto(250.57487666,330.81600108)(250.430875,330.86400243)(250.430875,332.0880012)
\lineto(250.430875,341.9040012)
\lineto(250.262875,342.0480012)
\lineto(246.494875,341.4480012)
\lineto(246.494875,340.8480012)
\lineto(247.406875,340.7280012)
\curveto(248.03087438,340.6320013)(248.174875,340.51200034)(248.174875,339.6480012)
\lineto(248.174875,334.2480012)
\curveto(248.174875,333.57600187)(248.10287488,333.04800096)(247.982875,332.8080012)
\curveto(247.43087555,331.70400231)(246.27887375,331.0080012)(245.030875,331.0080012)
\curveto(243.63887639,331.0080012)(242.606875,331.87200307)(242.606875,333.7440012)
\lineto(242.606875,341.9040012)
\lineto(242.438875,342.0480012)
\lineto(238.670875,341.4480012)
\lineto(238.670875,340.8480012)
\lineto(239.582875,340.7280012)
\curveto(240.20687438,340.6320013)(240.350875,340.51200034)(240.350875,339.6480012)
\lineto(240.350875,333.3120012)
\curveto(240.350875,330.55200396)(241.98287678,329.7120012)(243.758875,329.7120012)
\curveto(245.79887296,329.7120012)(247.4548757,331.10400154)(248.150875,331.4400012)
\lineto(248.390875,330.0000012)
\lineto(252.230875,330.0000012)
}
}
{
\newrgbcolor{curcolor}{0 0 0}
\pscustom[linestyle=none,fillstyle=solid,fillcolor=curcolor]
{
\newpath
\moveto(257.114125,342.0480012)
\lineto(253.466125,341.4480012)
\lineto(253.466125,340.8480012)
\lineto(254.378125,340.7280012)
\curveto(255.00212438,340.6320013)(255.146125,340.51200034)(255.146125,339.6480012)
\lineto(255.146125,332.0880012)
\curveto(255.146125,330.86400243)(255.02612332,330.81600108)(253.346125,330.6960012)
\lineto(253.346125,330.0000012)
\lineto(259.682125,330.0000012)
\lineto(259.682125,330.6960012)
\curveto(257.54612714,330.81600108)(257.402125,330.86400243)(257.402125,332.0880012)
\lineto(257.402125,337.7040012)
\curveto(257.402125,339.55199935)(258.21812555,340.1760012)(258.770125,340.1760012)
\curveto(259.15412462,340.1760012)(259.5621257,340.03200082)(260.258125,339.6480012)
\curveto(260.42612483,339.5520013)(260.61812512,339.5280012)(260.738125,339.5280012)
\curveto(261.31412442,339.5280012)(261.842125,340.12800195)(261.842125,340.8720012)
\curveto(261.842125,341.40000067)(261.50612406,342.0480012)(260.570125,342.0480012)
\curveto(259.70612586,342.0480012)(258.98612342,341.51999981)(257.402125,340.1280012)
\lineto(257.114125,342.0480012)
}
}
{
\newrgbcolor{curcolor}{0 0 0}
\pscustom[linestyle=none,fillstyle=solid,fillcolor=curcolor]
{
\newpath
\moveto(136.504,280.0000012)
\lineto(136.504,280.6960012)
\curveto(134.94400156,280.81600108)(134.36799986,280.88800221)(134.224,281.8960012)
\curveto(134.10400012,282.68800041)(134.056,283.60000247)(134.056,284.8720012)
\lineto(134.056,294.4000012)
\lineto(134.104,294.4000012)
\lineto(140.296,280.2880012)
\lineto(140.944,280.2880012)
\lineto(147.208,294.4000012)
\lineto(147.28,294.4000012)
\lineto(147.28,282.2320012)
\curveto(147.28,280.88800255)(147.23199789,280.84000106)(145.12,280.6960012)
\lineto(145.12,280.0000012)
\lineto(151.816,280.0000012)
\lineto(151.816,280.6960012)
\curveto(149.72800209,280.84000106)(149.656,280.88800255)(149.656,282.2320012)
\lineto(149.656,294.3760012)
\curveto(149.656,295.71999986)(149.72800209,295.76800135)(151.816,295.9120012)
\lineto(151.816,296.6080012)
\lineto(146.992,296.6080012)
\curveto(146.17600082,294.47200334)(145.19199904,292.35999909)(144.232,290.2480012)
\lineto(141.352,283.8160012)
\lineto(141.304,283.8160012)
\lineto(138.472,290.2240012)
\curveto(137.53600094,292.35999907)(136.52799914,294.47200334)(135.664,296.6080012)
\lineto(130.792,296.6080012)
\lineto(130.792,295.9120012)
\curveto(132.90399789,295.76800135)(132.952,295.71999986)(132.952,294.3760012)
\lineto(132.952,284.8720012)
\curveto(132.952,283.60000247)(132.90399988,282.68800041)(132.784,281.8960012)
\curveto(132.64000014,280.88800221)(132.06399875,280.79200111)(130.816,280.6960012)
\lineto(130.816,280.0000012)
\lineto(136.504,280.0000012)
}
}
{
\newrgbcolor{curcolor}{0 0 0}
\pscustom[linestyle=none,fillstyle=solid,fillcolor=curcolor]
{
\newpath
\moveto(159.552625,292.0480012)
\curveto(155.71262884,292.0480012)(153.816625,289.76799731)(153.816625,285.8800012)
\curveto(153.816625,281.99200509)(155.71262884,279.7120012)(159.552625,279.7120012)
\curveto(163.44062111,279.7120012)(165.312625,281.99200509)(165.312625,285.8800012)
\curveto(165.312625,289.76799731)(163.44062111,292.0480012)(159.552625,292.0480012)
\moveto(156.312625,285.8800012)
\curveto(156.312625,289.19199789)(157.36862718,291.2080012)(159.552625,291.2080012)
\curveto(161.78462277,291.2080012)(162.816625,289.19199789)(162.816625,285.8800012)
\curveto(162.816625,282.56800451)(161.78462277,280.5520012)(159.552625,280.5520012)
\curveto(157.36862718,280.5520012)(156.312625,282.56800451)(156.312625,285.8800012)
}
}
{
\newrgbcolor{curcolor}{0 0 0}
\pscustom[linestyle=none,fillstyle=solid,fillcolor=curcolor]
{
\newpath
\moveto(174.29275,290.8000012)
\lineto(174.29275,291.7600012)
\lineto(171.19675,291.7600012)
\lineto(171.19675,294.7840012)
\lineto(170.50075,294.7840012)
\lineto(169.03675,291.7600012)
\lineto(167.21275,291.7600012)
\lineto(167.21275,290.8000012)
\lineto(168.94075,290.8000012)
\lineto(168.94075,282.4480012)
\curveto(168.94075,280.00000365)(170.76475084,279.7120012)(171.60475,279.7120012)
\curveto(172.82874878,279.7120012)(173.9807507,280.33600163)(174.67675,280.7680012)
\lineto(174.48475,281.2720012)
\curveto(173.90875058,281.03200144)(173.3807494,280.9600012)(172.78075,280.9600012)
\curveto(171.96475082,280.9600012)(171.19675,281.53600303)(171.19675,283.3600012)
\lineto(171.19675,290.8000012)
\lineto(174.29275,290.8000012)
}
}
{
\newrgbcolor{curcolor}{0 0 0}
\pscustom[linestyle=none,fillstyle=solid,fillcolor=curcolor]
{
}
}
{
\newrgbcolor{curcolor}{0 0 0}
\pscustom[linestyle=none,fillstyle=solid,fillcolor=curcolor]
{
\newpath
\moveto(194.101375,280.0000012)
\lineto(194.101375,280.6960012)
\curveto(192.3973767,280.81600108)(192.301375,280.86400243)(192.301375,282.0880012)
\lineto(192.301375,297.6640012)
\lineto(192.133375,297.8080012)
\lineto(188.365375,297.2080012)
\lineto(188.365375,296.6080012)
\lineto(189.277375,296.5120012)
\curveto(189.90137438,296.44000127)(190.045375,296.32000029)(190.045375,295.4080012)
\lineto(190.045375,291.4720012)
\curveto(189.37337567,291.83200084)(188.4853739,292.0480012)(187.381375,292.0480012)
\curveto(185.41337697,292.0480012)(183.94937399,291.42400017)(182.941375,290.3920012)
\curveto(181.90937603,289.31200228)(181.285375,287.72799911)(181.285375,285.6400012)
\curveto(181.285375,282.08800475)(183.13337817,279.7120012)(186.301375,279.7120012)
\curveto(187.5973737,279.7120012)(188.7253763,280.28800226)(190.021375,281.3440012)
\lineto(190.021375,280.1680012)
\lineto(190.213375,280.0000012)
\lineto(194.101375,280.0000012)
\moveto(190.045375,284.0800012)
\curveto(190.045375,283.55200173)(190.02137486,283.14400079)(189.877375,282.7360012)
\curveto(189.39737548,281.48800245)(188.34137375,281.0080012)(187.093375,281.0080012)
\curveto(184.88537721,281.0080012)(183.781375,283.16800396)(183.781375,285.9280012)
\curveto(183.781375,289.04799808)(184.90937745,291.2080012)(187.357375,291.2080012)
\curveto(188.48537387,291.2080012)(189.39737543,290.68000022)(189.829375,289.6960012)
\curveto(190.02137481,289.28800161)(190.045375,288.95200055)(190.045375,288.3040012)
\lineto(190.045375,284.0800012)
}
}
{
\newrgbcolor{curcolor}{0 0 0}
\pscustom[linestyle=none,fillstyle=solid,fillcolor=curcolor]
{
\newpath
\moveto(205.037125,286.3120012)
\curveto(205.5411245,286.3120012)(205.997125,286.43200199)(205.997125,287.2240012)
\curveto(205.997125,288.63999979)(205.54112061,292.0480012)(201.149125,292.0480012)
\curveto(197.40512874,292.0480012)(195.629125,289.3839976)(195.629125,285.7840012)
\curveto(195.629125,281.99200499)(197.26112891,279.66400125)(201.173125,279.7120012)
\curveto(203.83712234,279.73600118)(205.20512565,281.2000031)(205.853125,283.0960012)
\lineto(205.133125,283.4800012)
\curveto(204.46112567,282.08800259)(203.5491231,280.9360012)(201.653125,280.9360012)
\curveto(198.67712798,280.9360012)(198.07712505,283.8160037)(198.125125,286.3120012)
\lineto(205.037125,286.3120012)
\moveto(198.149125,287.2240012)
\curveto(198.149125,288.20800022)(198.5091275,291.2080012)(201.005125,291.2080012)
\curveto(203.23712277,291.2080012)(203.501125,288.92800027)(203.501125,287.9920012)
\curveto(203.501125,287.53600166)(203.35712433,287.2240012)(202.685125,287.2240012)
\lineto(198.149125,287.2240012)
}
}
{
\newrgbcolor{curcolor}{0 0 0}
\pscustom[linestyle=none,fillstyle=solid,fillcolor=curcolor]
{
}
}
{
\newrgbcolor{curcolor}{0 0 0}
\pscustom[linestyle=none,fillstyle=solid,fillcolor=curcolor]
{
\newpath
\moveto(216.827875,287.7040012)
\curveto(216.827875,288.3040006)(216.92387522,288.64000163)(217.139875,289.0720012)
\curveto(217.69187445,290.1760001)(218.72387627,290.7520012)(219.995875,290.7520012)
\curveto(220.95587404,290.7520012)(223.091875,290.12799707)(223.091875,286.0000012)
\curveto(223.091875,282.47200473)(221.9878726,280.5520012)(219.587875,280.5520012)
\curveto(218.33987625,280.5520012)(217.37987462,281.20000231)(216.995875,282.3040012)
\curveto(216.85187514,282.73600077)(216.827875,283.21600175)(216.827875,283.7680012)
\lineto(216.827875,287.7040012)
\moveto(212.891875,291.4480012)
\lineto(212.891875,290.8480012)
\lineto(213.803875,290.7280012)
\curveto(214.42787438,290.6320013)(214.571875,290.51200034)(214.571875,289.6480012)
\lineto(214.571875,276.5680012)
\curveto(214.571875,275.34400243)(214.45187332,275.29600108)(212.771875,275.1760012)
\lineto(212.771875,274.4800012)
\lineto(218.987875,274.4800012)
\lineto(218.987875,275.1760012)
\curveto(216.97187702,275.29600108)(216.827875,275.34400243)(216.827875,276.5680012)
\lineto(216.827875,280.2640012)
\curveto(217.3318745,279.97600149)(218.4598761,279.7120012)(219.563875,279.7120012)
\curveto(222.80387176,279.7120012)(225.587875,281.20000615)(225.587875,286.1440012)
\curveto(225.587875,287.8479995)(225.27587063,292.0480012)(220.907875,292.0480012)
\curveto(219.15587675,292.0480012)(217.76387406,290.8720006)(216.827875,290.2720012)
\lineto(216.515875,292.0480012)
\lineto(212.891875,291.4480012)
}
}
{
\newrgbcolor{curcolor}{0 0 0}
\pscustom[linestyle=none,fillstyle=solid,fillcolor=curcolor]
{
\newpath
\moveto(234.982,283.4320012)
\curveto(234.982,281.48800315)(233.49399906,280.9840012)(232.558,280.9840012)
\curveto(231.07000149,280.9840012)(230.35,282.04000262)(230.35,283.4560012)
\curveto(230.35,284.58400007)(230.87800134,285.16000171)(232.222,285.6640012)
\curveto(233.18199904,286.02400084)(234.45400053,286.48000154)(234.982,286.8160012)
\lineto(234.982,283.4320012)
\moveto(237.142,288.8560012)
\curveto(237.142,290.24799981)(236.82999638,292.0480012)(233.206,292.0480012)
\curveto(230.49400271,292.0480012)(228.358,290.63199988)(228.358,289.3120012)
\curveto(228.358,288.54400197)(229.24600046,288.1840012)(229.702,288.1840012)
\curveto(230.2059995,288.1840012)(230.35000012,288.44800161)(230.47,288.8560012)
\curveto(230.99799947,290.63199943)(231.98200096,291.2080012)(232.942,291.2080012)
\curveto(233.87799906,291.2080012)(234.982,290.72799928)(234.982,288.8080012)
\lineto(234.982,287.8000012)
\curveto(234.3820006,287.17600183)(232.0539981,286.5760006)(230.158,285.9760012)
\curveto(228.43000173,285.44800173)(227.926,284.24800007)(227.926,283.1200012)
\curveto(227.926,281.320003)(229.12600226,279.7120012)(231.382,279.7120012)
\curveto(232.86999851,279.76000115)(234.19000077,280.64800171)(234.958,281.1520012)
\curveto(235.29399966,280.26400209)(235.67800084,279.7120012)(236.518,279.7120012)
\curveto(237.40599911,279.7120012)(238.43800091,279.97600166)(239.35,280.4320012)
\lineto(239.206,281.0080012)
\curveto(238.87000034,280.93600127)(238.34199964,280.8880013)(237.982,280.9840012)
\curveto(237.55000043,281.08000111)(237.142,281.53600271)(237.142,283.0480012)
\lineto(237.142,288.8560012)
}
}
{
\newrgbcolor{curcolor}{0 0 0}
\pscustom[linestyle=none,fillstyle=solid,fillcolor=curcolor]
{
\newpath
\moveto(248.1205,288.4240012)
\lineto(248.1205,291.2320012)
\curveto(247.32850079,291.80800063)(246.10449892,292.0480012)(245.0245,292.0480012)
\curveto(242.43250259,292.0480012)(240.68049998,290.82399892)(240.6565,288.5440012)
\curveto(240.68049998,286.55200319)(242.28850199,285.6640006)(244.2805,285.0640012)
\curveto(245.36049892,284.72800154)(246.8725,284.17599957)(246.8725,282.5440012)
\curveto(246.8725,281.32000243)(245.91249873,280.5520012)(244.6405,280.5520012)
\curveto(242.69650194,280.5520012)(241.61649952,281.96800324)(241.1365,284.0080012)
\lineto(240.4405,284.0080012)
\lineto(240.6805,280.8640012)
\curveto(241.54449914,280.09600197)(243.00850142,279.7120012)(244.4245,279.7120012)
\curveto(247.28049714,279.7120012)(248.9125,281.22400324)(248.9125,283.2640012)
\curveto(248.9125,285.37599909)(247.61649753,286.33600197)(245.1445,287.1040012)
\curveto(244.16050098,287.41600089)(242.5765,287.92000255)(242.5765,289.2640012)
\curveto(242.60049998,290.51199995)(243.5365012,291.2080012)(244.7365,291.2080012)
\curveto(246.4404983,291.2080012)(247.25650017,289.88799974)(247.4245,288.4240012)
\lineto(248.1205,288.4240012)
}
}
{
\newrgbcolor{curcolor}{0 0 0}
\pscustom[linestyle=none,fillstyle=solid,fillcolor=curcolor]
{
\newpath
\moveto(258.573625,288.4240012)
\lineto(258.573625,291.2320012)
\curveto(257.78162579,291.80800063)(256.55762392,292.0480012)(255.477625,292.0480012)
\curveto(252.88562759,292.0480012)(251.13362498,290.82399892)(251.109625,288.5440012)
\curveto(251.13362498,286.55200319)(252.74162699,285.6640006)(254.733625,285.0640012)
\curveto(255.81362392,284.72800154)(257.325625,284.17599957)(257.325625,282.5440012)
\curveto(257.325625,281.32000243)(256.36562373,280.5520012)(255.093625,280.5520012)
\curveto(253.14962694,280.5520012)(252.06962452,281.96800324)(251.589625,284.0080012)
\lineto(250.893625,284.0080012)
\lineto(251.133625,280.8640012)
\curveto(251.99762414,280.09600197)(253.46162642,279.7120012)(254.877625,279.7120012)
\curveto(257.73362214,279.7120012)(259.365625,281.22400324)(259.365625,283.2640012)
\curveto(259.365625,285.37599909)(258.06962253,286.33600197)(255.597625,287.1040012)
\curveto(254.61362598,287.41600089)(253.029625,287.92000255)(253.029625,289.2640012)
\curveto(253.05362498,290.51199995)(253.9896262,291.2080012)(255.189625,291.2080012)
\curveto(256.8936233,291.2080012)(257.70962517,289.88799974)(257.877625,288.4240012)
\lineto(258.573625,288.4240012)
}
}
{
\newrgbcolor{curcolor}{0 0 0}
\pscustom[linestyle=none,fillstyle=solid,fillcolor=curcolor]
{
\newpath
\moveto(270.80275,286.3120012)
\curveto(271.3067495,286.3120012)(271.76275,286.43200199)(271.76275,287.2240012)
\curveto(271.76275,288.63999979)(271.30674561,292.0480012)(266.91475,292.0480012)
\curveto(263.17075374,292.0480012)(261.39475,289.3839976)(261.39475,285.7840012)
\curveto(261.39475,281.99200499)(263.02675391,279.66400125)(266.93875,279.7120012)
\curveto(269.60274734,279.73600118)(270.97075065,281.2000031)(271.61875,283.0960012)
\lineto(270.89875,283.4800012)
\curveto(270.22675067,282.08800259)(269.3147481,280.9360012)(267.41875,280.9360012)
\curveto(264.44275298,280.9360012)(263.84275005,283.8160037)(263.89075,286.3120012)
\lineto(270.80275,286.3120012)
\moveto(263.91475,287.2240012)
\curveto(263.91475,288.20800022)(264.2747525,291.2080012)(266.77075,291.2080012)
\curveto(269.00274777,291.2080012)(269.26675,288.92800027)(269.26675,287.9920012)
\curveto(269.26675,287.53600166)(269.12274933,287.2240012)(268.45075,287.2240012)
\lineto(263.91475,287.2240012)
}
}
{
\newrgbcolor{curcolor}{0 0 0}
\pscustom[linestyle=none,fillstyle=solid,fillcolor=curcolor]
{
\newpath
\moveto(139.568,256.6080012)
\lineto(139.568,255.9120012)
\curveto(140.91199866,255.86400125)(141.20000048,255.76799993)(141.68,254.4960012)
\curveto(142.44799923,252.48000322)(143.33600084,250.0799988)(144.176,247.6800012)
\lineto(146.888,240.0000012)
\lineto(148.592,240.0000012)
\lineto(151.664,248.2800012)
\curveto(152.64799902,250.94399854)(153.60800058,253.39200247)(154.184,254.6640012)
\curveto(154.6879995,255.79200007)(154.9760013,255.84000127)(156.272,255.9120012)
\lineto(156.272,256.6080012)
\lineto(150.968,256.6080012)
\lineto(150.968,255.9120012)
\lineto(152.288,255.7920012)
\curveto(152.91199938,255.74400125)(152.93599988,255.50400075)(152.816,255.0480012)
\curveto(152.55200026,253.99200226)(151.80799914,251.8799988)(150.944,249.4800012)
\lineto(148.448,242.5920012)
\lineto(148.4,242.6400012)
\lineto(146.168,248.8080012)
\curveto(145.42400074,250.89599911)(144.67999938,253.00800312)(144.056,254.9280012)
\curveto(143.86400019,255.5280006)(143.93600058,255.74400125)(144.512,255.7920012)
\lineto(145.76,255.9120012)
\lineto(145.76,256.6080012)
\lineto(139.568,256.6080012)
}
}
{
\newrgbcolor{curcolor}{0 0 0}
\pscustom[linestyle=none,fillstyle=solid,fillcolor=curcolor]
{
\newpath
\moveto(165.818375,246.3120012)
\curveto(166.3223745,246.3120012)(166.778375,246.43200199)(166.778375,247.2240012)
\curveto(166.778375,248.63999979)(166.32237061,252.0480012)(161.930375,252.0480012)
\curveto(158.18637874,252.0480012)(156.410375,249.3839976)(156.410375,245.7840012)
\curveto(156.410375,241.99200499)(158.04237891,239.66400125)(161.954375,239.7120012)
\curveto(164.61837234,239.73600118)(165.98637565,241.2000031)(166.634375,243.0960012)
\lineto(165.914375,243.4800012)
\curveto(165.24237567,242.08800259)(164.3303731,240.9360012)(162.434375,240.9360012)
\curveto(159.45837798,240.9360012)(158.85837505,243.8160037)(158.906375,246.3120012)
\lineto(165.818375,246.3120012)
\moveto(158.930375,247.2240012)
\curveto(158.930375,248.20800022)(159.2903775,251.2080012)(161.786375,251.2080012)
\curveto(164.01837277,251.2080012)(164.282375,248.92800027)(164.282375,247.9920012)
\curveto(164.282375,247.53600166)(164.13837433,247.2240012)(163.466375,247.2240012)
\lineto(158.930375,247.2240012)
\moveto(160.418375,253.0080012)
\lineto(164.042375,255.0240012)
\curveto(164.76237428,255.40800082)(165.386375,255.81600171)(165.386375,256.3200012)
\curveto(165.386375,256.80000072)(164.81037464,257.3520012)(164.450375,257.3520012)
\curveto(164.04237541,257.3520012)(163.63437428,257.11200046)(162.914375,256.3680012)
\lineto(160.058375,253.4880012)
\lineto(160.418375,253.0080012)
}
}
{
\newrgbcolor{curcolor}{0 0 0}
\pscustom[linestyle=none,fillstyle=solid,fillcolor=curcolor]
{
\newpath
\moveto(172.0985,252.0480012)
\lineto(168.4505,251.4480012)
\lineto(168.4505,250.8480012)
\lineto(169.3625,250.7280012)
\curveto(169.98649938,250.6320013)(170.1305,250.51200034)(170.1305,249.6480012)
\lineto(170.1305,242.0880012)
\curveto(170.1305,240.86400243)(170.01049832,240.81600108)(168.3305,240.6960012)
\lineto(168.3305,240.0000012)
\lineto(174.6665,240.0000012)
\lineto(174.6665,240.6960012)
\curveto(172.53050214,240.81600108)(172.3865,240.86400243)(172.3865,242.0880012)
\lineto(172.3865,247.7040012)
\curveto(172.3865,249.55199935)(173.20250055,250.1760012)(173.7545,250.1760012)
\curveto(174.13849962,250.1760012)(174.5465007,250.03200082)(175.2425,249.6480012)
\curveto(175.41049983,249.5520013)(175.60250012,249.5280012)(175.7225,249.5280012)
\curveto(176.29849942,249.5280012)(176.8265,250.12800195)(176.8265,250.8720012)
\curveto(176.8265,251.40000067)(176.49049906,252.0480012)(175.5545,252.0480012)
\curveto(174.69050086,252.0480012)(173.97049842,251.51999981)(172.3865,250.1280012)
\lineto(172.0985,252.0480012)
}
}
{
\newrgbcolor{curcolor}{0 0 0}
\pscustom[linestyle=none,fillstyle=solid,fillcolor=curcolor]
{
\newpath
\moveto(177.778625,251.4480012)
\lineto(177.778625,250.8480012)
\lineto(178.690625,250.7280012)
\curveto(179.31462438,250.6320013)(179.458625,250.51200034)(179.458625,249.6480012)
\lineto(179.458625,242.0880012)
\curveto(179.458625,240.86400243)(179.38662327,240.81600108)(177.658625,240.6960012)
\lineto(177.658625,240.0000012)
\lineto(183.514625,240.0000012)
\lineto(183.514625,240.6960012)
\curveto(181.8106267,240.81600108)(181.714625,240.86400243)(181.714625,242.0880012)
\lineto(181.714625,251.9040012)
\lineto(181.546625,252.0480012)
\lineto(177.778625,251.4480012)
\moveto(180.562625,257.1600012)
\curveto(179.69862586,257.1600012)(179.098625,256.53600034)(179.098625,255.6720012)
\curveto(179.098625,254.83200204)(179.69862586,254.2320012)(180.562625,254.2320012)
\curveto(181.45062411,254.2320012)(182.00262502,254.83200204)(182.026625,255.6720012)
\curveto(182.026625,256.53600034)(181.45062411,257.1600012)(180.562625,257.1600012)
}
}
{
\newrgbcolor{curcolor}{0 0 0}
\pscustom[linestyle=none,fillstyle=solid,fillcolor=curcolor]
{
\newpath
\moveto(188.699,250.8000012)
\lineto(191.627,250.8000012)
\lineto(191.627,251.7600012)
\lineto(188.699,251.7600012)
\lineto(188.603,254.4480012)
\curveto(188.53100007,256.72799892)(189.37100038,256.9680012)(189.755,256.9680012)
\curveto(190.2589995,256.9680012)(190.7870006,256.72799981)(191.387,255.3360012)
\curveto(191.50699988,255.09600144)(191.67500036,254.9280012)(192.035,254.9280012)
\curveto(192.51499952,254.9280012)(193.211,255.33600192)(193.211,256.0560012)
\curveto(193.211,256.84800041)(192.20299808,257.8080012)(190.283,257.8080012)
\curveto(188.96300132,257.8080012)(187.8349994,257.20800029)(187.235,256.2960012)
\curveto(186.68300055,255.45600204)(186.443,254.18399962)(186.443,252.6000012)
\lineto(186.443,251.7600012)
\lineto(184.595,251.7600012)
\lineto(184.595,250.8000012)
\lineto(186.443,250.8000012)
\lineto(186.443,242.0880012)
\curveto(186.443,240.86400243)(186.37099827,240.81600108)(184.643,240.6960012)
\lineto(184.643,240.0000012)
\lineto(190.979,240.0000012)
\lineto(190.979,240.6960012)
\curveto(188.79500218,240.79200111)(188.699,240.84000243)(188.699,242.0640012)
\lineto(188.699,250.8000012)
}
}
{
\newrgbcolor{curcolor}{0 0 0}
\pscustom[linestyle=none,fillstyle=solid,fillcolor=curcolor]
{
\newpath
\moveto(192.403625,251.4480012)
\lineto(192.403625,250.8480012)
\lineto(193.315625,250.7280012)
\curveto(193.93962438,250.6320013)(194.083625,250.51200034)(194.083625,249.6480012)
\lineto(194.083625,242.0880012)
\curveto(194.083625,240.86400243)(194.01162327,240.81600108)(192.283625,240.6960012)
\lineto(192.283625,240.0000012)
\lineto(198.139625,240.0000012)
\lineto(198.139625,240.6960012)
\curveto(196.4356267,240.81600108)(196.339625,240.86400243)(196.339625,242.0880012)
\lineto(196.339625,251.9040012)
\lineto(196.171625,252.0480012)
\lineto(192.403625,251.4480012)
\moveto(195.187625,257.1600012)
\curveto(194.32362586,257.1600012)(193.723625,256.53600034)(193.723625,255.6720012)
\curveto(193.723625,254.83200204)(194.32362586,254.2320012)(195.187625,254.2320012)
\curveto(196.07562411,254.2320012)(196.62762502,254.83200204)(196.651625,255.6720012)
\curveto(196.651625,256.53600034)(196.07562411,257.1600012)(195.187625,257.1600012)
}
}
{
\newrgbcolor{curcolor}{0 0 0}
\pscustom[linestyle=none,fillstyle=solid,fillcolor=curcolor]
{
\newpath
\moveto(205.7,240.9600012)
\curveto(202.84400286,240.9600012)(202.172,243.93600346)(202.172,246.1920012)
\curveto(202.172,249.76799763)(203.58800166,251.2080012)(205.244,251.2080012)
\curveto(206.3479989,251.2080012)(207.04400048,250.41599991)(207.524,249.1200012)
\curveto(207.66799986,248.73600159)(207.81200046,248.4960012)(208.268,248.4960012)
\curveto(208.74799952,248.4960012)(209.684,248.80800211)(209.684,249.7200012)
\curveto(209.684,250.8240001)(208.09999741,252.0480012)(205.508,252.0480012)
\curveto(201.14000437,252.0480012)(199.676,248.95199801)(199.676,245.7600012)
\curveto(199.676,241.75200521)(201.6440036,239.7120012)(205.244,239.7120012)
\curveto(206.92399832,239.7120012)(209.18000067,240.60000377)(209.852,243.1680012)
\lineto(209.156,243.5040012)
\curveto(208.41200074,241.82400288)(207.45199825,240.9600012)(205.7,240.9600012)
}
}
{
\newrgbcolor{curcolor}{0 0 0}
\pscustom[linestyle=none,fillstyle=solid,fillcolor=curcolor]
{
\newpath
\moveto(218.63825,243.4320012)
\curveto(218.63825,241.48800315)(217.15024906,240.9840012)(216.21425,240.9840012)
\curveto(214.72625149,240.9840012)(214.00625,242.04000262)(214.00625,243.4560012)
\curveto(214.00625,244.58400007)(214.53425134,245.16000171)(215.87825,245.6640012)
\curveto(216.83824904,246.02400084)(218.11025053,246.48000154)(218.63825,246.8160012)
\lineto(218.63825,243.4320012)
\moveto(220.79825,248.8560012)
\curveto(220.79825,250.24799981)(220.48624638,252.0480012)(216.86225,252.0480012)
\curveto(214.15025271,252.0480012)(212.01425,250.63199988)(212.01425,249.3120012)
\curveto(212.01425,248.54400197)(212.90225046,248.1840012)(213.35825,248.1840012)
\curveto(213.8622495,248.1840012)(214.00625012,248.44800161)(214.12625,248.8560012)
\curveto(214.65424947,250.63199943)(215.63825096,251.2080012)(216.59825,251.2080012)
\curveto(217.53424906,251.2080012)(218.63825,250.72799928)(218.63825,248.8080012)
\lineto(218.63825,247.8000012)
\curveto(218.0382506,247.17600183)(215.7102481,246.5760006)(213.81425,245.9760012)
\curveto(212.08625173,245.44800173)(211.58225,244.24800007)(211.58225,243.1200012)
\curveto(211.58225,241.320003)(212.78225226,239.7120012)(215.03825,239.7120012)
\curveto(216.52624851,239.76000115)(217.84625077,240.64800171)(218.61425,241.1520012)
\curveto(218.95024966,240.26400209)(219.33425084,239.7120012)(220.17425,239.7120012)
\curveto(221.06224911,239.7120012)(222.09425091,239.97600166)(223.00625,240.4320012)
\lineto(222.86225,241.0080012)
\curveto(222.52625034,240.93600127)(221.99824964,240.8880013)(221.63825,240.9840012)
\curveto(221.20625043,241.08000111)(220.79825,241.53600271)(220.79825,243.0480012)
\lineto(220.79825,248.8560012)
}
}
{
\newrgbcolor{curcolor}{0 0 0}
\pscustom[linestyle=none,fillstyle=solid,fillcolor=curcolor]
{
\newpath
\moveto(230.79275,250.8000012)
\lineto(230.79275,251.7600012)
\lineto(227.69675,251.7600012)
\lineto(227.69675,254.7840012)
\lineto(227.00075,254.7840012)
\lineto(225.53675,251.7600012)
\lineto(223.71275,251.7600012)
\lineto(223.71275,250.8000012)
\lineto(225.44075,250.8000012)
\lineto(225.44075,242.4480012)
\curveto(225.44075,240.00000365)(227.26475084,239.7120012)(228.10475,239.7120012)
\curveto(229.32874878,239.7120012)(230.4807507,240.33600163)(231.17675,240.7680012)
\lineto(230.98475,241.2720012)
\curveto(230.40875058,241.03200144)(229.8807494,240.9600012)(229.28075,240.9600012)
\curveto(228.46475082,240.9600012)(227.69675,241.53600303)(227.69675,243.3600012)
\lineto(227.69675,250.8000012)
\lineto(230.79275,250.8000012)
}
}
{
\newrgbcolor{curcolor}{0 0 0}
\pscustom[linestyle=none,fillstyle=solid,fillcolor=curcolor]
{
\newpath
\moveto(232.10675,251.4480012)
\lineto(232.10675,250.8480012)
\lineto(233.01875,250.7280012)
\curveto(233.64274938,250.6320013)(233.78675,250.51200034)(233.78675,249.6480012)
\lineto(233.78675,242.0880012)
\curveto(233.78675,240.86400243)(233.71474827,240.81600108)(231.98675,240.6960012)
\lineto(231.98675,240.0000012)
\lineto(237.84275,240.0000012)
\lineto(237.84275,240.6960012)
\curveto(236.1387517,240.81600108)(236.04275,240.86400243)(236.04275,242.0880012)
\lineto(236.04275,251.9040012)
\lineto(235.87475,252.0480012)
\lineto(232.10675,251.4480012)
\moveto(234.89075,257.1600012)
\curveto(234.02675086,257.1600012)(233.42675,256.53600034)(233.42675,255.6720012)
\curveto(233.42675,254.83200204)(234.02675086,254.2320012)(234.89075,254.2320012)
\curveto(235.77874911,254.2320012)(236.33075002,254.83200204)(236.35475,255.6720012)
\curveto(236.35475,256.53600034)(235.77874911,257.1600012)(234.89075,257.1600012)
}
}
{
\newrgbcolor{curcolor}{0 0 0}
\pscustom[linestyle=none,fillstyle=solid,fillcolor=curcolor]
{
\newpath
\moveto(245.115125,252.0480012)
\curveto(241.27512884,252.0480012)(239.379125,249.76799731)(239.379125,245.8800012)
\curveto(239.379125,241.99200509)(241.27512884,239.7120012)(245.115125,239.7120012)
\curveto(249.00312111,239.7120012)(250.875125,241.99200509)(250.875125,245.8800012)
\curveto(250.875125,249.76799731)(249.00312111,252.0480012)(245.115125,252.0480012)
\moveto(241.875125,245.8800012)
\curveto(241.875125,249.19199789)(242.93112718,251.2080012)(245.115125,251.2080012)
\curveto(247.34712277,251.2080012)(248.379125,249.19199789)(248.379125,245.8800012)
\curveto(248.379125,242.56800451)(247.34712277,240.5520012)(245.115125,240.5520012)
\curveto(242.93112718,240.5520012)(241.875125,242.56800451)(241.875125,245.8800012)
}
}
{
\newrgbcolor{curcolor}{0 0 0}
\pscustom[linestyle=none,fillstyle=solid,fillcolor=curcolor]
{
\newpath
\moveto(264.72725,248.2320012)
\curveto(264.72725,250.65599878)(263.40724779,252.0480012)(261.19925,252.0480012)
\curveto(259.42325178,252.0480012)(258.31924851,251.18400024)(256.83125,250.2240012)
\lineto(256.49525,252.0480012)
\lineto(252.91925,251.4480012)
\lineto(252.91925,250.8480012)
\lineto(253.83125,250.7280012)
\curveto(254.45524938,250.6320013)(254.59925,250.51200034)(254.59925,249.6480012)
\lineto(254.59925,242.0880012)
\curveto(254.59925,240.86400243)(254.52724827,240.81600108)(252.79925,240.6960012)
\lineto(252.79925,240.0000012)
\lineto(258.65525,240.0000012)
\lineto(258.65525,240.6960012)
\curveto(256.9512517,240.81600108)(256.85525,240.86400243)(256.85525,242.0880012)
\lineto(256.85525,247.7040012)
\curveto(256.85525,248.3040006)(256.90325019,248.64000159)(257.09525,249.0240012)
\curveto(257.5992495,249.96000027)(258.67925127,250.7520012)(259.95125,250.7520012)
\curveto(261.58324837,250.7520012)(262.47125,249.83999911)(262.47125,247.7520012)
\lineto(262.47125,242.0880012)
\curveto(262.47125,240.86400243)(262.39924827,240.81600108)(260.67125,240.6960012)
\lineto(260.67125,240.0000012)
\lineto(266.52725,240.0000012)
\lineto(266.52725,240.6960012)
\curveto(264.8232517,240.81600108)(264.72725,240.86400243)(264.72725,242.0880012)
\lineto(264.72725,248.2320012)
}
}
{
\newrgbcolor{curcolor}{0 0 0}
\pscustom[linestyle=none,fillstyle=solid,fillcolor=curcolor]
{
\newpath
\moveto(53.984,540.0000012)
\lineto(53.984,540.9280012)
\lineto(51.296,541.1520012)
\curveto(50.62400067,541.21600114)(50.24,541.47200245)(50.24,542.7200012)
\lineto(50.24,561.5680012)
\lineto(50.08,561.7600012)
\lineto(43.488,560.6400012)
\lineto(43.488,559.8400012)
\lineto(46.464,559.4880012)
\curveto(47.00799946,559.42400127)(47.232,559.16800027)(47.232,558.2400012)
\lineto(47.232,542.7200012)
\curveto(47.232,542.11200181)(47.13599981,541.72800098)(46.944,541.5040012)
\curveto(46.78400016,541.28000143)(46.52799965,541.18400117)(46.176,541.1520012)
\lineto(43.488,540.9280012)
\lineto(43.488,540.0000012)
\lineto(53.984,540.0000012)
}
}
{
\newrgbcolor{curcolor}{0 0 0}
\pscustom[linestyle=none,fillstyle=solid,fillcolor=curcolor]
{
\newpath
\moveto(55.52,464.2240012)
\lineto(54.624,464.3840012)
\curveto(53.95200067,462.65600293)(53.31199872,462.4320012)(52.032,462.4320012)
\lineto(43.84,462.4320012)
\curveto(44.35199949,464.09599954)(46.24000298,466.36800335)(49.216,468.5120012)
\curveto(52.28799693,470.75199896)(54.656,472.19200479)(54.656,475.7760012)
\curveto(54.656,480.22399675)(51.71199664,481.7600012)(48.352,481.7600012)
\curveto(43.96800438,481.7600012)(41.696,479.10399954)(41.696,477.4400012)
\curveto(41.696,476.35200229)(42.91200054,475.8400012)(43.456,475.8400012)
\curveto(44.03199942,475.8400012)(44.25600013,476.16000175)(44.384,476.7040012)
\curveto(44.89599949,478.87999903)(46.14400198,480.6400012)(48.128,480.6400012)
\curveto(50.59199754,480.6400012)(51.296,478.55999887)(51.296,476.2240012)
\curveto(51.296,472.80000463)(49.56799722,470.87999874)(46.784,468.4160012)
\curveto(42.84800394,464.99200463)(41.47199939,462.81599871)(40.864,460.3200012)
\lineto(41.184,460.0000012)
\lineto(54.432,460.0000012)
\lineto(55.52,464.2240012)
}
}
{
\newrgbcolor{curcolor}{0 0 0}
\pscustom[linestyle=none,fillstyle=solid,fillcolor=curcolor]
{
\newpath
\moveto(44.16,411.7120012)
\curveto(44.16,411.16800175)(44.32000042,410.8160012)(44.736,410.8160012)
\curveto(45.11999962,410.8160012)(45.98400154,411.1360012)(47.52,411.1360012)
\curveto(50.23999728,411.1360012)(51.776,408.67199842)(51.776,405.8880012)
\curveto(51.776,402.08000501)(49.88799773,400.7360012)(47.616,400.7360012)
\curveto(45.50400211,400.7360012)(44.12799939,402.33600299)(43.52,404.1280012)
\curveto(43.32800019,404.73600059)(43.00799955,405.0240012)(42.56,405.0240012)
\curveto(41.98400058,405.0240012)(40.864,404.41600005)(40.864,403.2640012)
\curveto(40.864,401.88800258)(43.07200451,399.6160012)(47.584,399.6160012)
\curveto(52.22399536,399.6160012)(55.136,401.9520053)(55.136,406.0480012)
\curveto(55.136,410.33599691)(51.45599824,411.61600136)(49.696,411.7760012)
\lineto(49.696,411.9040012)
\curveto(51.42399827,412.19200091)(54.208,413.47200437)(54.208,416.6400012)
\curveto(54.208,420.19199765)(51.42399658,421.7600012)(48,421.7600012)
\curveto(43.77600422,421.7600012)(41.696,419.32799983)(41.696,417.9520012)
\curveto(41.696,416.96000219)(42.81600045,416.4160012)(43.264,416.4160012)
\curveto(43.67999958,416.4160012)(43.93600013,416.64000162)(44.064,417.0560012)
\curveto(44.7679993,419.23199903)(45.98400179,420.6400012)(47.776,420.6400012)
\curveto(50.30399747,420.6400012)(50.912,418.39999941)(50.912,416.6080012)
\curveto(50.912,414.68800312)(50.23999728,412.2880012)(47.52,412.2880012)
\curveto(45.98400154,412.2880012)(45.11999962,412.6080012)(44.736,412.6080012)
\curveto(44.32000042,412.6080012)(44.16,412.28800063)(44.16,411.7120012)
}
}
{
\newrgbcolor{curcolor}{0 0 0}
\pscustom[linestyle=none,fillstyle=solid,fillcolor=curcolor]
{
\newpath
\moveto(509.6,355.7280012)
\lineto(509.6,352.4640012)
\curveto(509.6,351.37600229)(509.24799923,351.18400114)(508.48,351.1200012)
\lineto(506.432,350.9280012)
\lineto(506.432,350.0000012)
\lineto(514.88,350.0000012)
\lineto(514.88,350.9280012)
\lineto(513.44,351.0880012)
\curveto(512.70400074,351.18400111)(512.48,351.37600229)(512.48,352.4640012)
\lineto(512.48,355.7280012)
\lineto(515.776,355.7280012)
\lineto(515.776,357.2320012)
\lineto(512.48,357.2320012)
\lineto(512.48,371.3760012)
\lineto(510.144,371.3760012)
\curveto(507.13600301,366.96000562)(503.55199706,361.39199621)(500.608,356.4000012)
\lineto(500.896,355.7280012)
\lineto(509.6,355.7280012)
\moveto(502.912,357.2320012)
\curveto(504.83199808,360.71999771)(507.07200246,364.40000507)(509.536,368.2720012)
\lineto(509.6,368.2720012)
\lineto(509.6,357.2320012)
\lineto(502.912,357.2320012)
}
}
{
\newrgbcolor{curcolor}{1 1 1}
\pscustom[linestyle=none,fillstyle=solid,fillcolor=curcolor]
{
\newpath
\moveto(149.99850464,429.91890837)
\lineto(409.9803772,429.91890837)
\lineto(409.9803772,389.8669026)
\lineto(149.99850464,389.8669026)
\lineto(149.99850464,429.91890837)
\closepath
}
}
{
\newrgbcolor{curcolor}{1 0 0}
\pscustom[linewidth=2,linecolor=curcolor,linestyle=dashed,dash=8 8]
{
\newpath
\moveto(149.99850464,429.91890837)
\lineto(409.9803772,429.91890837)
\lineto(409.9803772,389.8669026)
\lineto(149.99850464,389.8669026)
\lineto(149.99850464,429.91890837)
\closepath
}
}
{
\newrgbcolor{curcolor}{0 0 0}
\pscustom[linestyle=none,fillstyle=solid,fillcolor=curcolor]
{
\newpath
\moveto(280.76009941,340.41146971)
\lineto(419.70257759,340.41146971)
\curveto(425.69124755,340.41146971)(430.51245117,345.23267332)(430.51245117,351.22134329)
\lineto(430.51245117,358.7882588)
\curveto(430.51245117,364.77692876)(425.69124755,369.59813238)(419.70257759,369.59813238)
\lineto(280.76009941,369.59813238)
\curveto(274.77142945,369.59813238)(269.95022583,364.77692876)(269.95022583,358.7882588)
\lineto(269.95022583,351.22134329)
\curveto(269.95022583,345.23267332)(274.77142945,340.41146971)(280.76009941,340.41146971)
\closepath
}
}
{
\newrgbcolor{curcolor}{0 0 0}
\pscustom[linewidth=2,linecolor=curcolor]
{
\newpath
\moveto(280.76009941,340.41146971)
\lineto(419.70257759,340.41146971)
\curveto(425.69124755,340.41146971)(430.51245117,345.23267332)(430.51245117,351.22134329)
\lineto(430.51245117,358.7882588)
\curveto(430.51245117,364.77692876)(425.69124755,369.59813238)(419.70257759,369.59813238)
\lineto(280.76009941,369.59813238)
\curveto(274.77142945,369.59813238)(269.95022583,364.77692876)(269.95022583,358.7882588)
\lineto(269.95022583,351.22134329)
\curveto(269.95022583,345.23267332)(274.77142945,340.41146971)(280.76009941,340.41146971)
\closepath
}
}
{
\newrgbcolor{curcolor}{0 0 0}
\pscustom[linestyle=none,fillstyle=solid,fillcolor=curcolor]
{
\newpath
\moveto(513.632,298.9440012)
\lineto(514.4,301.1200012)
\lineto(514.24,301.3760012)
\lineto(503.552,301.3760012)
\lineto(503.232,301.0880012)
\lineto(502.592,290.3680012)
\lineto(503.36,289.9840012)
\curveto(504.70399866,291.51999967)(506.01600173,292.2560012)(507.744,292.2560012)
\curveto(509.79199795,292.2560012)(512.032,290.75199698)(512.032,286.5280012)
\curveto(512.032,283.2320045)(510.68799722,280.7360012)(507.904,280.7360012)
\curveto(505.66400224,280.7360012)(504.51199936,282.36800303)(503.872,284.1920012)
\curveto(503.71200016,284.67200072)(503.42399955,284.9600012)(502.976,284.9600012)
\curveto(502.33600064,284.9600012)(501.28,284.32000015)(501.28,283.2640012)
\curveto(501.28,281.79200267)(503.52000416,279.6160012)(507.68,279.6160012)
\curveto(512.95999472,279.6160012)(515.36,282.81600527)(515.36,286.8800012)
\curveto(515.36,291.35999672)(512.54399619,293.6640012)(508.736,293.6640012)
\curveto(506.97600176,293.6640012)(505.08799898,292.89600027)(504.064,291.9680012)
\lineto(504,292.0320012)
\lineto(504.736,298.0160012)
\curveto(504.8319999,298.8160004)(505.05600061,298.9440012)(505.664,298.9440012)
\lineto(513.632,298.9440012)
}
}
{
\newrgbcolor{curcolor}{0 0 0}
\pscustom[linestyle=none,fillstyle=solid,fillcolor=curcolor]
{
\newpath
\moveto(290.2822094,272.23803831)
\lineto(420.36839485,272.23803831)
\curveto(426.06473886,272.23803831)(430.65060425,276.8239037)(430.65060425,282.52024771)
\lineto(430.65060425,289.71779753)
\curveto(430.65060425,295.41414153)(426.06473886,300.00000692)(420.36839485,300.00000692)
\lineto(290.2822094,300.00000692)
\curveto(284.58586539,300.00000692)(280,295.41414153)(280,289.71779753)
\lineto(280,282.52024771)
\curveto(280,276.8239037)(284.58586539,272.23803831)(290.2822094,272.23803831)
\closepath
}
}
{
\newrgbcolor{curcolor}{0 0 0}
\pscustom[linewidth=2,linecolor=curcolor]
{
\newpath
\moveto(290.2822094,272.23803831)
\lineto(420.36839485,272.23803831)
\curveto(426.06473886,272.23803831)(430.65060425,276.8239037)(430.65060425,282.52024771)
\lineto(430.65060425,289.71779753)
\curveto(430.65060425,295.41414153)(426.06473886,300.00000692)(420.36839485,300.00000692)
\lineto(290.2822094,300.00000692)
\curveto(284.58586539,300.00000692)(280,295.41414153)(280,289.71779753)
\lineto(280,282.52024771)
\curveto(280,276.8239037)(284.58586539,272.23803831)(290.2822094,272.23803831)
\closepath
}
}
{
\newrgbcolor{curcolor}{0 0 0}
\pscustom[linestyle=none,fillstyle=solid,fillcolor=curcolor]
{
\newpath
\moveto(291.03757477,230.0000012)
\lineto(419.01493073,230.0000012)
\curveto(425.12974715,230.0000012)(430.05250549,234.92275955)(430.05250549,241.03757597)
\lineto(430.05250549,248.76388098)
\curveto(430.05250549,254.8786974)(425.12974715,259.80145575)(419.01493073,259.80145575)
\lineto(291.03757477,259.80145575)
\curveto(284.92275835,259.80145575)(280,254.8786974)(280,248.76388098)
\lineto(280,241.03757597)
\curveto(280,234.92275955)(284.92275835,230.0000012)(291.03757477,230.0000012)
\closepath
}
}
{
\newrgbcolor{curcolor}{0 0 0}
\pscustom[linewidth=2,linecolor=curcolor]
{
\newpath
\moveto(291.03757477,230.0000012)
\lineto(419.01493073,230.0000012)
\curveto(425.12974715,230.0000012)(430.05250549,234.92275955)(430.05250549,241.03757597)
\lineto(430.05250549,248.76388098)
\curveto(430.05250549,254.8786974)(425.12974715,259.80145575)(419.01493073,259.80145575)
\lineto(291.03757477,259.80145575)
\curveto(284.92275835,259.80145575)(280,254.8786974)(280,248.76388098)
\lineto(280,241.03757597)
\curveto(280,234.92275955)(284.92275835,230.0000012)(291.03757477,230.0000012)
\closepath
}
}
{
\newrgbcolor{curcolor}{0 0 0}
\pscustom[linestyle=none,fillstyle=solid,fillcolor=curcolor]
{
\newpath
\moveto(141.62518787,99.99997068)
\lineto(248.98716259,99.99997068)
\curveto(255.08827452,99.99997068)(260,104.91169617)(260,111.01280809)
\lineto(260,118.7218001)
\curveto(260,124.82291202)(255.08827452,129.73463751)(248.98716259,129.73463751)
\lineto(141.62518787,129.73463751)
\curveto(135.52407595,129.73463751)(130.61235046,124.82291202)(130.61235046,118.7218001)
\lineto(130.61235046,111.01280809)
\curveto(130.61235046,104.91169617)(135.52407595,99.99997068)(141.62518787,99.99997068)
\closepath
}
}
{
\newrgbcolor{curcolor}{0 0 0}
\pscustom[linewidth=2,linecolor=curcolor]
{
\newpath
\moveto(141.62518787,99.99997068)
\lineto(248.98716259,99.99997068)
\curveto(255.08827452,99.99997068)(260,104.91169617)(260,111.01280809)
\lineto(260,118.7218001)
\curveto(260,124.82291202)(255.08827452,129.73463751)(248.98716259,129.73463751)
\lineto(141.62518787,129.73463751)
\curveto(135.52407595,129.73463751)(130.61235046,124.82291202)(130.61235046,118.7218001)
\lineto(130.61235046,111.01280809)
\curveto(130.61235046,104.91169617)(135.52407595,99.99997068)(141.62518787,99.99997068)
\closepath
}
}
{
\newrgbcolor{curcolor}{0 0 0}
\pscustom[linestyle=none,fillstyle=solid,fillcolor=curcolor]
{
\newpath
\moveto(322.16531563,99.73465086)
\lineto(418.9267025,99.73465086)
\curveto(425.40054725,99.73465086)(430.61234283,104.94644645)(430.61234283,111.42029119)
\lineto(430.61234283,119.60024382)
\curveto(430.61234283,126.07408856)(425.40054725,131.28588415)(418.9267025,131.28588415)
\lineto(322.16531563,131.28588415)
\curveto(315.69147088,131.28588415)(310.47967529,126.07408856)(310.47967529,119.60024382)
\lineto(310.47967529,111.42029119)
\curveto(310.47967529,104.94644645)(315.69147088,99.73465086)(322.16531563,99.73465086)
\closepath
}
}
{
\newrgbcolor{curcolor}{0 0 0}
\pscustom[linewidth=2,linecolor=curcolor]
{
\newpath
\moveto(322.16531563,99.73465086)
\lineto(418.9267025,99.73465086)
\curveto(425.40054725,99.73465086)(430.61234283,104.94644645)(430.61234283,111.42029119)
\lineto(430.61234283,119.60024382)
\curveto(430.61234283,126.07408856)(425.40054725,131.28588415)(418.9267025,131.28588415)
\lineto(322.16531563,131.28588415)
\curveto(315.69147088,131.28588415)(310.47967529,126.07408856)(310.47967529,119.60024382)
\lineto(310.47967529,111.42029119)
\curveto(310.47967529,104.94644645)(315.69147088,99.73465086)(322.16531563,99.73465086)
\closepath
}
}
{
\newrgbcolor{curcolor}{0 0 0}
\pscustom[linestyle=none,fillstyle=solid,fillcolor=curcolor]
{
\newpath
\moveto(508.704,239.6160012)
\curveto(513.43999526,239.6160012)(515.968,242.49600555)(515.968,246.8480012)
\curveto(515.968,251.16799688)(513.18399667,253.56800114)(509.856,253.5040012)
\curveto(507.42400243,253.56800114)(505.50399914,251.80800015)(504.64,250.7520012)
\lineto(504.576,250.7520012)
\curveto(504.60799997,257.79199416)(506.62400288,260.6400012)(509.504,260.6400012)
\curveto(510.97599853,260.6400012)(512.00000048,259.39199938)(512.48,257.5680012)
\curveto(512.5759999,257.18400159)(512.83200048,256.8960012)(513.312,256.8960012)
\curveto(514.0159993,256.8960012)(515.008,257.4400021)(515.008,258.3360012)
\curveto(515.008,259.61599992)(513.43999616,261.7600012)(509.6,261.7600012)
\curveto(507.07200253,261.7600012)(505.27999866,260.79999967)(503.936,259.2640012)
\curveto(502.49600144,257.60000287)(501.408,254.52799656)(501.408,249.8880012)
\curveto(501.408,242.97600811)(503.9040048,239.6160012)(508.704,239.6160012)
\moveto(508.768,251.8720012)
\curveto(510.78399798,251.8720012)(512.768,250.04799752)(512.768,246.3680012)
\curveto(512.768,243.00800456)(511.45599731,240.7360012)(508.768,240.7360012)
\curveto(505.98400278,240.7360012)(504.64,243.71200504)(504.64,247.5520012)
\curveto(504.64,249.50399925)(506.08000269,251.8720012)(508.768,251.8720012)
}
}
{
\newrgbcolor{curcolor}{0 0 0}
\pscustom[linestyle=none,fillstyle=solid,fillcolor=curcolor]
{
\newpath
\moveto(502.304,211.3760012)
\lineto(501.312,206.8960012)
\lineto(502.176,206.8960012)
\lineto(502.496,207.5680012)
\curveto(502.91199958,208.46400031)(503.32800144,208.9440012)(504.768,208.9440012)
\lineto(513.408,208.9440012)
\curveto(513.08800032,207.79200235)(512.51199741,205.93599749)(509.92,202.2240012)
\curveto(507.16800275,198.28800514)(505.344,194.70399743)(505.344,190.9280012)
\curveto(505.344,189.87200226)(506.08000099,189.6160012)(507.072,189.6160012)
\curveto(507.99999907,189.6160012)(508.67199997,189.90400213)(508.64,190.8320012)
\curveto(508.48000016,195.05599698)(509.3440017,198.44800418)(511.04,201.4240012)
\curveto(513.08799795,205.00799762)(514.78400112,206.99200523)(515.904,211.0240012)
\lineto(515.616,211.3760012)
\lineto(502.304,211.3760012)
}
}
{
\newrgbcolor{curcolor}{0 0 0}
\pscustom[linestyle=none,fillstyle=solid,fillcolor=curcolor]
{
\newpath
\moveto(508.48,160.8160012)
\curveto(511.03999744,160.8160012)(512.416,158.63999829)(512.416,155.7280012)
\curveto(512.416,152.91200402)(511.19999728,150.7360012)(508.48,150.7360012)
\curveto(505.82400266,150.7360012)(504.544,152.91200402)(504.544,155.7280012)
\curveto(504.544,158.63999829)(505.95200253,160.8160012)(508.48,160.8160012)
\moveto(508.48,149.6160012)
\curveto(513.66399482,149.6160012)(515.616,152.59200443)(515.616,155.8240012)
\curveto(515.616,159.08799794)(513.56799782,160.94400171)(511.392,161.4560012)
\lineto(511.392,161.5520012)
\curveto(513.34399805,162.06400069)(514.912,163.69600383)(514.912,166.3200012)
\curveto(514.912,169.80799771)(512.19199629,171.7600012)(508.48,171.7600012)
\curveto(504.83200365,171.7600012)(502.048,169.77599775)(502.048,166.3200012)
\curveto(502.048,163.69600383)(503.64800192,162.06400069)(505.568,161.5520012)
\lineto(505.568,161.4560012)
\curveto(503.42400214,160.94400171)(501.344,159.08799794)(501.344,155.8240012)
\curveto(501.344,152.59200443)(503.26400522,149.6160012)(508.48,149.6160012)
\moveto(508.48,161.9680012)
\curveto(506.1760023,161.9680012)(505.152,164.20800338)(505.152,166.3840012)
\curveto(505.152,168.84799874)(506.30400218,170.6400012)(508.48,170.6400012)
\curveto(510.65599782,170.6400012)(511.74400006,168.84799874)(511.808,166.3840012)
\curveto(511.808,164.20800338)(510.81599766,161.9680012)(508.48,161.9680012)
}
}
{
\newrgbcolor{curcolor}{0 0 0}
\pscustom[linewidth=2,linecolor=curcolor,linestyle=dashed,dash=8 8]
{
\newpath
\moveto(150,550)
\lineto(60,550)
}
}
{
\newrgbcolor{curcolor}{0 0 0}
\pscustom[linestyle=none,fillstyle=solid,fillcolor=curcolor]
{
\newpath
\moveto(139.53769464,554.84048224)
\lineto(152.6487474,550.01921591)
\lineto(139.53769392,545.19795064)
\curveto(141.632292,548.04442372)(141.62022288,551.93889292)(139.53769464,554.84048224)
\lineto(139.53769464,554.84048224)
\lineto(139.53769464,554.84048224)
\closepath
}
}
{
\newrgbcolor{curcolor}{0 0 0}
\pscustom[linewidth=2,linecolor=curcolor,linestyle=dashed,dash=8 8]
{
\newpath
\moveto(170,470)
\lineto(60,470)
}
}
{
\newrgbcolor{curcolor}{0 0 0}
\pscustom[linestyle=none,fillstyle=solid,fillcolor=curcolor]
{
\newpath
\moveto(159.53769464,474.84048224)
\lineto(172.6487474,470.01921591)
\lineto(159.53769392,465.19795064)
\curveto(161.632292,468.04442372)(161.62022288,471.93889292)(159.53769464,474.84048224)
\lineto(159.53769464,474.84048224)
\lineto(159.53769464,474.84048224)
\closepath
}
}
{
\newrgbcolor{curcolor}{0 0 0}
\pscustom[linewidth=2,linecolor=curcolor,linestyle=dashed,dash=8 8]
{
\newpath
\moveto(170,410)
\lineto(60,410)
}
}
{
\newrgbcolor{curcolor}{0 0 0}
\pscustom[linestyle=none,fillstyle=solid,fillcolor=curcolor]
{
\newpath
\moveto(159.53769464,414.84048224)
\lineto(172.6487474,410.01921591)
\lineto(159.53769392,405.19795064)
\curveto(161.632292,408.04442372)(161.62022288,411.93889292)(159.53769464,414.84048224)
\lineto(159.53769464,414.84048224)
\lineto(159.53769464,414.84048224)
\closepath
}
}
{
\newrgbcolor{curcolor}{0 0 0}
\pscustom[linewidth=2,linecolor=curcolor,linestyle=dashed,dash=8 8]
{
\newpath
\moveto(410,360)
\lineto(500,360)
}
}
{
\newrgbcolor{curcolor}{0 0 0}
\pscustom[linestyle=none,fillstyle=solid,fillcolor=curcolor]
{
\newpath
\moveto(420.46230536,355.15951776)
\lineto(407.3512526,359.98078409)
\lineto(420.46230608,364.80204936)
\curveto(418.367708,361.95557628)(418.37977712,358.06110708)(420.46230536,355.15951776)
\lineto(420.46230536,355.15951776)
\lineto(420.46230536,355.15951776)
\closepath
}
}
{
\newrgbcolor{curcolor}{0 0 0}
\pscustom[linewidth=2,linecolor=curcolor,linestyle=dashed,dash=8 8]
{
\newpath
\moveto(410,290)
\lineto(500,290)
}
}
{
\newrgbcolor{curcolor}{0 0 0}
\pscustom[linestyle=none,fillstyle=solid,fillcolor=curcolor]
{
\newpath
\moveto(420.46230536,285.15951776)
\lineto(407.3512526,289.98078409)
\lineto(420.46230608,294.80204936)
\curveto(418.367708,291.95557628)(418.37977712,288.06110708)(420.46230536,285.15951776)
\lineto(420.46230536,285.15951776)
\lineto(420.46230536,285.15951776)
\closepath
}
}
{
\newrgbcolor{curcolor}{0 0 0}
\pscustom[linewidth=2,linecolor=curcolor,linestyle=dashed,dash=8 8]
{
\newpath
\moveto(410,250)
\lineto(500,250)
}
}
{
\newrgbcolor{curcolor}{0 0 0}
\pscustom[linestyle=none,fillstyle=solid,fillcolor=curcolor]
{
\newpath
\moveto(420.46230536,245.15951776)
\lineto(407.3512526,249.98078409)
\lineto(420.46230608,254.80204936)
\curveto(418.367708,251.95557628)(418.37977712,248.06110708)(420.46230536,245.15951776)
\lineto(420.46230536,245.15951776)
\lineto(420.46230536,245.15951776)
\closepath
}
}
{
\newrgbcolor{curcolor}{0 0 0}
\pscustom[linewidth=2,linecolor=curcolor,linestyle=dashed,dash=8 8]
{
\newpath
\moveto(410,120)
\lineto(500,120)
}
}
{
\newrgbcolor{curcolor}{0 0 0}
\pscustom[linestyle=none,fillstyle=solid,fillcolor=curcolor]
{
\newpath
\moveto(420.46230536,115.15951776)
\lineto(407.3512526,119.98078409)
\lineto(420.46230608,124.80204936)
\curveto(418.367708,121.95557628)(418.37977712,118.06110708)(420.46230536,115.15951776)
\lineto(420.46230536,115.15951776)
\lineto(420.46230536,115.15951776)
\closepath
}
}
{
\newrgbcolor{curcolor}{0 0 0}
\pscustom[linewidth=2,linecolor=curcolor,linestyle=dashed,dash=8 8]
{
\newpath
\moveto(150,120)
\lineto(60,120)
}
}
{
\newrgbcolor{curcolor}{0 0 0}
\pscustom[linestyle=none,fillstyle=solid,fillcolor=curcolor]
{
\newpath
\moveto(139.53769464,124.84048224)
\lineto(152.6487474,120.01921591)
\lineto(139.53769392,115.19795064)
\curveto(141.632292,118.04442372)(141.62022288,121.93889292)(139.53769464,124.84048224)
\lineto(139.53769464,124.84048224)
\lineto(139.53769464,124.84048224)
\closepath
}
}
{
\newrgbcolor{curcolor}{0 0 0}
\pscustom[linestyle=none,fillstyle=solid,fillcolor=curcolor]
{
\newpath
\moveto(155.192,206.0320012)
\curveto(154.11200108,206.5360007)(152.28799782,206.9680012)(150.104,206.9680012)
\curveto(147.15200295,206.9680012)(144.77599856,206.10399959)(143.336,204.4960012)
\curveto(141.96800137,202.96000274)(141.152,200.89599849)(141.152,198.1840012)
\curveto(141.152,195.37600401)(141.99200144,193.11999976)(143.432,191.6800012)
\curveto(144.91999851,190.19200269)(147.27200269,189.6400012)(149.96,189.6400012)
\curveto(151.87999808,189.6400012)(154.08800115,190.16800183)(155.24,190.7920012)
\lineto(155.576,194.8000012)
\lineto(154.856,194.8000012)
\curveto(154.18400067,192.3040037)(152.95999686,190.4800012)(149.816,190.4800012)
\curveto(144.87200494,190.4800012)(143.84,195.35200411)(143.84,198.2560012)
\curveto(143.84,202.19199727)(145.40000446,206.15200118)(149.864,206.1280012)
\curveto(152.31199755,206.1280012)(154.01600046,205.11999837)(154.472,202.2880012)
\lineto(155.192,202.2880012)
\lineto(155.192,206.0320012)
}
}
{
\newrgbcolor{curcolor}{0 0 0}
\pscustom[linestyle=none,fillstyle=solid,fillcolor=curcolor]
{
\newpath
\moveto(163.31825,202.0480012)
\curveto(159.47825384,202.0480012)(157.58225,199.76799731)(157.58225,195.8800012)
\curveto(157.58225,191.99200509)(159.47825384,189.7120012)(163.31825,189.7120012)
\curveto(167.20624611,189.7120012)(169.07825,191.99200509)(169.07825,195.8800012)
\curveto(169.07825,199.76799731)(167.20624611,202.0480012)(163.31825,202.0480012)
\moveto(160.07825,195.8800012)
\curveto(160.07825,199.19199789)(161.13425218,201.2080012)(163.31825,201.2080012)
\curveto(165.55024777,201.2080012)(166.58225,199.19199789)(166.58225,195.8800012)
\curveto(166.58225,192.56800451)(165.55024777,190.5520012)(163.31825,190.5520012)
\curveto(161.13425218,190.5520012)(160.07825,192.56800451)(160.07825,195.8800012)
}
}
{
\newrgbcolor{curcolor}{0 0 0}
\pscustom[linestyle=none,fillstyle=solid,fillcolor=curcolor]
{
\newpath
\moveto(182.930375,198.2320012)
\curveto(182.930375,200.65599878)(181.61037279,202.0480012)(179.402375,202.0480012)
\curveto(177.62637678,202.0480012)(176.52237351,201.18400024)(175.034375,200.2240012)
\lineto(174.698375,202.0480012)
\lineto(171.122375,201.4480012)
\lineto(171.122375,200.8480012)
\lineto(172.034375,200.7280012)
\curveto(172.65837438,200.6320013)(172.802375,200.51200034)(172.802375,199.6480012)
\lineto(172.802375,192.0880012)
\curveto(172.802375,190.86400243)(172.73037327,190.81600108)(171.002375,190.6960012)
\lineto(171.002375,190.0000012)
\lineto(176.858375,190.0000012)
\lineto(176.858375,190.6960012)
\curveto(175.1543767,190.81600108)(175.058375,190.86400243)(175.058375,192.0880012)
\lineto(175.058375,197.7040012)
\curveto(175.058375,198.3040006)(175.10637519,198.64000159)(175.298375,199.0240012)
\curveto(175.8023745,199.96000027)(176.88237627,200.7520012)(178.154375,200.7520012)
\curveto(179.78637337,200.7520012)(180.674375,199.83999911)(180.674375,197.7520012)
\lineto(180.674375,192.0880012)
\curveto(180.674375,190.86400243)(180.60237327,190.81600108)(178.874375,190.6960012)
\lineto(178.874375,190.0000012)
\lineto(184.730375,190.0000012)
\lineto(184.730375,190.6960012)
\curveto(183.0263767,190.81600108)(182.930375,190.86400243)(182.930375,192.0880012)
\lineto(182.930375,198.2320012)
}
}
{
\newrgbcolor{curcolor}{0 0 0}
\pscustom[linestyle=none,fillstyle=solid,fillcolor=curcolor]
{
\newpath
\moveto(197.78975,198.2320012)
\curveto(197.78975,200.65599878)(196.46974779,202.0480012)(194.26175,202.0480012)
\curveto(192.48575178,202.0480012)(191.38174851,201.18400024)(189.89375,200.2240012)
\lineto(189.55775,202.0480012)
\lineto(185.98175,201.4480012)
\lineto(185.98175,200.8480012)
\lineto(186.89375,200.7280012)
\curveto(187.51774938,200.6320013)(187.66175,200.51200034)(187.66175,199.6480012)
\lineto(187.66175,192.0880012)
\curveto(187.66175,190.86400243)(187.58974827,190.81600108)(185.86175,190.6960012)
\lineto(185.86175,190.0000012)
\lineto(191.71775,190.0000012)
\lineto(191.71775,190.6960012)
\curveto(190.0137517,190.81600108)(189.91775,190.86400243)(189.91775,192.0880012)
\lineto(189.91775,197.7040012)
\curveto(189.91775,198.3040006)(189.96575019,198.64000159)(190.15775,199.0240012)
\curveto(190.6617495,199.96000027)(191.74175127,200.7520012)(193.01375,200.7520012)
\curveto(194.64574837,200.7520012)(195.53375,199.83999911)(195.53375,197.7520012)
\lineto(195.53375,192.0880012)
\curveto(195.53375,190.86400243)(195.46174827,190.81600108)(193.73375,190.6960012)
\lineto(193.73375,190.0000012)
\lineto(199.58975,190.0000012)
\lineto(199.58975,190.6960012)
\curveto(197.8857517,190.81600108)(197.78975,190.86400243)(197.78975,192.0880012)
\lineto(197.78975,198.2320012)
}
}
{
\newrgbcolor{curcolor}{0 0 0}
\pscustom[linestyle=none,fillstyle=solid,fillcolor=curcolor]
{
\newpath
\moveto(210.537125,196.3120012)
\curveto(211.0411245,196.3120012)(211.497125,196.43200199)(211.497125,197.2240012)
\curveto(211.497125,198.63999979)(211.04112061,202.0480012)(206.649125,202.0480012)
\curveto(202.90512874,202.0480012)(201.129125,199.3839976)(201.129125,195.7840012)
\curveto(201.129125,191.99200499)(202.76112891,189.66400125)(206.673125,189.7120012)
\curveto(209.33712234,189.73600118)(210.70512565,191.2000031)(211.353125,193.0960012)
\lineto(210.633125,193.4800012)
\curveto(209.96112567,192.08800259)(209.0491231,190.9360012)(207.153125,190.9360012)
\curveto(204.17712798,190.9360012)(203.57712505,193.8160037)(203.625125,196.3120012)
\lineto(210.537125,196.3120012)
\moveto(203.649125,197.2240012)
\curveto(203.649125,198.20800022)(204.0091275,201.2080012)(206.505125,201.2080012)
\curveto(208.73712277,201.2080012)(209.001125,198.92800027)(209.001125,197.9920012)
\curveto(209.001125,197.53600166)(208.85712433,197.2240012)(208.185125,197.2240012)
\lineto(203.649125,197.2240012)
}
}
{
\newrgbcolor{curcolor}{0 0 0}
\pscustom[linestyle=none,fillstyle=solid,fillcolor=curcolor]
{
\newpath
\moveto(216.91325,190.0000012)
\lineto(216.91325,190.6960012)
\lineto(215.90525,190.8160012)
\curveto(215.56925034,190.86400115)(215.59325014,191.12800147)(215.73725,191.3920012)
\curveto(216.26524947,192.42400017)(217.03325082,193.57600245)(217.84925,194.8240012)
\lineto(217.89725,194.8240012)
\lineto(220.03325,191.4880012)
\curveto(220.29724974,191.08000161)(220.22524964,190.86400115)(219.86525,190.8160012)
\lineto(218.85725,190.6960012)
\lineto(218.85725,190.0000012)
\lineto(224.49725,190.0000012)
\lineto(224.49725,190.6960012)
\curveto(223.41725108,190.74400115)(223.0092494,190.81600207)(222.40925,191.6800012)
\curveto(221.3052511,193.26399962)(220.17724894,194.94400288)(219.12125,196.6240012)
\curveto(219.91324921,197.824)(220.84925094,199.02400233)(221.78525,200.1520012)
\curveto(222.40924938,200.92000043)(222.72125115,201.01600125)(223.87325,201.0640012)
\lineto(223.87325,201.7600012)
\lineto(219.40925,201.7600012)
\lineto(219.40925,201.0640012)
\lineto(220.34525,200.9680012)
\curveto(220.72924962,200.92000125)(220.68124983,200.68000094)(220.51325,200.4160012)
\curveto(219.96125055,199.50400211)(219.31324928,198.4960001)(218.59325,197.3920012)
\lineto(218.56925,197.3920012)
\lineto(216.55325,200.3920012)
\curveto(216.36125019,200.68000091)(216.40925036,200.94400125)(216.76925,200.9920012)
\lineto(217.48925,201.0640012)
\lineto(217.48925,201.7600012)
\lineto(212.47325,201.7600012)
\lineto(212.47325,201.0640012)
\curveto(213.43324904,201.01600125)(213.67325058,200.96800041)(214.24925,200.1760012)
\curveto(215.25724899,198.78400259)(216.28925103,197.22399967)(217.32125,195.6880012)
\curveto(216.43325089,194.3920025)(215.49724902,193.02399991)(214.51325,191.7280012)
\curveto(213.88925062,190.88800204)(213.36124894,190.76800113)(212.30525,190.6960012)
\lineto(212.30525,190.0000012)
\lineto(216.91325,190.0000012)
}
}
{
\newrgbcolor{curcolor}{0 0 0}
\pscustom[linestyle=none,fillstyle=solid,fillcolor=curcolor]
{
\newpath
\moveto(224.841125,201.4480012)
\lineto(224.841125,200.8480012)
\lineto(225.753125,200.7280012)
\curveto(226.37712438,200.6320013)(226.521125,200.51200034)(226.521125,199.6480012)
\lineto(226.521125,192.0880012)
\curveto(226.521125,190.86400243)(226.44912327,190.81600108)(224.721125,190.6960012)
\lineto(224.721125,190.0000012)
\lineto(230.577125,190.0000012)
\lineto(230.577125,190.6960012)
\curveto(228.8731267,190.81600108)(228.777125,190.86400243)(228.777125,192.0880012)
\lineto(228.777125,201.9040012)
\lineto(228.609125,202.0480012)
\lineto(224.841125,201.4480012)
\moveto(227.625125,207.1600012)
\curveto(226.76112586,207.1600012)(226.161125,206.53600034)(226.161125,205.6720012)
\curveto(226.161125,204.83200204)(226.76112586,204.2320012)(227.625125,204.2320012)
\curveto(228.51312411,204.2320012)(229.06512502,204.83200204)(229.089125,205.6720012)
\curveto(229.089125,206.53600034)(228.51312411,207.1600012)(227.625125,207.1600012)
}
}
{
\newrgbcolor{curcolor}{0 0 0}
\pscustom[linestyle=none,fillstyle=solid,fillcolor=curcolor]
{
\newpath
\moveto(237.8495,202.0480012)
\curveto(234.00950384,202.0480012)(232.1135,199.76799731)(232.1135,195.8800012)
\curveto(232.1135,191.99200509)(234.00950384,189.7120012)(237.8495,189.7120012)
\curveto(241.73749611,189.7120012)(243.6095,191.99200509)(243.6095,195.8800012)
\curveto(243.6095,199.76799731)(241.73749611,202.0480012)(237.8495,202.0480012)
\moveto(234.6095,195.8800012)
\curveto(234.6095,199.19199789)(235.66550218,201.2080012)(237.8495,201.2080012)
\curveto(240.08149777,201.2080012)(241.1135,199.19199789)(241.1135,195.8800012)
\curveto(241.1135,192.56800451)(240.08149777,190.5520012)(237.8495,190.5520012)
\curveto(235.66550218,190.5520012)(234.6095,192.56800451)(234.6095,195.8800012)
}
}
{
\newrgbcolor{curcolor}{0 0 0}
\pscustom[linestyle=none,fillstyle=solid,fillcolor=curcolor]
{
\newpath
\moveto(257.461625,198.2320012)
\curveto(257.461625,200.65599878)(256.14162279,202.0480012)(253.933625,202.0480012)
\curveto(252.15762678,202.0480012)(251.05362351,201.18400024)(249.565625,200.2240012)
\lineto(249.229625,202.0480012)
\lineto(245.653625,201.4480012)
\lineto(245.653625,200.8480012)
\lineto(246.565625,200.7280012)
\curveto(247.18962438,200.6320013)(247.333625,200.51200034)(247.333625,199.6480012)
\lineto(247.333625,192.0880012)
\curveto(247.333625,190.86400243)(247.26162327,190.81600108)(245.533625,190.6960012)
\lineto(245.533625,190.0000012)
\lineto(251.389625,190.0000012)
\lineto(251.389625,190.6960012)
\curveto(249.6856267,190.81600108)(249.589625,190.86400243)(249.589625,192.0880012)
\lineto(249.589625,197.7040012)
\curveto(249.589625,198.3040006)(249.63762519,198.64000159)(249.829625,199.0240012)
\curveto(250.3336245,199.96000027)(251.41362627,200.7520012)(252.685625,200.7520012)
\curveto(254.31762337,200.7520012)(255.205625,199.83999911)(255.205625,197.7520012)
\lineto(255.205625,192.0880012)
\curveto(255.205625,190.86400243)(255.13362327,190.81600108)(253.405625,190.6960012)
\lineto(253.405625,190.0000012)
\lineto(259.261625,190.0000012)
\lineto(259.261625,190.6960012)
\curveto(257.5576267,190.81600108)(257.461625,190.86400243)(257.461625,192.0880012)
\lineto(257.461625,198.2320012)
}
}
{
\newrgbcolor{curcolor}{0 0 0}
\pscustom[linestyle=none,fillstyle=solid,fillcolor=curcolor]
{
}
}
{
\newrgbcolor{curcolor}{0 0 0}
\pscustom[linestyle=none,fillstyle=solid,fillcolor=curcolor]
{
\newpath
\moveto(273.247625,193.4320012)
\curveto(273.247625,191.48800315)(271.75962406,190.9840012)(270.823625,190.9840012)
\curveto(269.33562649,190.9840012)(268.615625,192.04000262)(268.615625,193.4560012)
\curveto(268.615625,194.58400007)(269.14362634,195.16000171)(270.487625,195.6640012)
\curveto(271.44762404,196.02400084)(272.71962553,196.48000154)(273.247625,196.8160012)
\lineto(273.247625,193.4320012)
\moveto(275.407625,198.8560012)
\curveto(275.407625,200.24799981)(275.09562138,202.0480012)(271.471625,202.0480012)
\curveto(268.75962771,202.0480012)(266.623625,200.63199988)(266.623625,199.3120012)
\curveto(266.623625,198.54400197)(267.51162546,198.1840012)(267.967625,198.1840012)
\curveto(268.4716245,198.1840012)(268.61562512,198.44800161)(268.735625,198.8560012)
\curveto(269.26362447,200.63199943)(270.24762596,201.2080012)(271.207625,201.2080012)
\curveto(272.14362406,201.2080012)(273.247625,200.72799928)(273.247625,198.8080012)
\lineto(273.247625,197.8000012)
\curveto(272.6476256,197.17600183)(270.3196231,196.5760006)(268.423625,195.9760012)
\curveto(266.69562673,195.44800173)(266.191625,194.24800007)(266.191625,193.1200012)
\curveto(266.191625,191.320003)(267.39162726,189.7120012)(269.647625,189.7120012)
\curveto(271.13562351,189.76000115)(272.45562577,190.64800171)(273.223625,191.1520012)
\curveto(273.55962466,190.26400209)(273.94362584,189.7120012)(274.783625,189.7120012)
\curveto(275.67162411,189.7120012)(276.70362591,189.97600166)(277.615625,190.4320012)
\lineto(277.471625,191.0080012)
\curveto(277.13562534,190.93600127)(276.60762464,190.8880013)(276.247625,190.9840012)
\curveto(275.81562543,191.08000111)(275.407625,191.53600271)(275.407625,193.0480012)
\lineto(275.407625,198.8560012)
}
}
{
\newrgbcolor{curcolor}{0 0 0}
\pscustom[linestyle=none,fillstyle=solid,fillcolor=curcolor]
{
\newpath
\moveto(291.762125,190.0000012)
\lineto(291.762125,190.6960012)
\curveto(290.10612666,190.81600108)(289.962125,190.86400243)(289.962125,192.0880012)
\lineto(289.962125,201.9040012)
\lineto(289.794125,202.0480012)
\lineto(286.026125,201.4480012)
\lineto(286.026125,200.8480012)
\lineto(286.938125,200.7280012)
\curveto(287.56212438,200.6320013)(287.706125,200.51200034)(287.706125,199.6480012)
\lineto(287.706125,194.2480012)
\curveto(287.706125,193.57600187)(287.63412488,193.04800096)(287.514125,192.8080012)
\curveto(286.96212555,191.70400231)(285.81012375,191.0080012)(284.562125,191.0080012)
\curveto(283.17012639,191.0080012)(282.138125,191.87200307)(282.138125,193.7440012)
\lineto(282.138125,201.9040012)
\lineto(281.970125,202.0480012)
\lineto(278.202125,201.4480012)
\lineto(278.202125,200.8480012)
\lineto(279.114125,200.7280012)
\curveto(279.73812438,200.6320013)(279.882125,200.51200034)(279.882125,199.6480012)
\lineto(279.882125,193.3120012)
\curveto(279.882125,190.55200396)(281.51412678,189.7120012)(283.290125,189.7120012)
\curveto(285.33012296,189.7120012)(286.9861257,191.10400154)(287.682125,191.4400012)
\lineto(287.922125,190.0000012)
\lineto(291.762125,190.0000012)
}
}
{
\newrgbcolor{curcolor}{0 0 0}
\pscustom[linestyle=none,fillstyle=solid,fillcolor=curcolor]
{
\newpath
\moveto(299.933375,200.8000012)
\lineto(299.933375,201.7600012)
\lineto(296.837375,201.7600012)
\lineto(296.837375,204.7840012)
\lineto(296.141375,204.7840012)
\lineto(294.677375,201.7600012)
\lineto(292.853375,201.7600012)
\lineto(292.853375,200.8000012)
\lineto(294.581375,200.8000012)
\lineto(294.581375,192.4480012)
\curveto(294.581375,190.00000365)(296.40537584,189.7120012)(297.245375,189.7120012)
\curveto(298.46937378,189.7120012)(299.6213757,190.33600163)(300.317375,190.7680012)
\lineto(300.125375,191.2720012)
\curveto(299.54937558,191.03200144)(299.0213744,190.9600012)(298.421375,190.9600012)
\curveto(297.60537582,190.9600012)(296.837375,191.53600303)(296.837375,193.3600012)
\lineto(296.837375,200.8000012)
\lineto(299.933375,200.8000012)
}
}
{
\newrgbcolor{curcolor}{0 0 0}
\pscustom[linestyle=none,fillstyle=solid,fillcolor=curcolor]
{
\newpath
\moveto(307.271375,202.0480012)
\curveto(303.43137884,202.0480012)(301.535375,199.76799731)(301.535375,195.8800012)
\curveto(301.535375,191.99200509)(303.43137884,189.7120012)(307.271375,189.7120012)
\curveto(311.15937111,189.7120012)(313.031375,191.99200509)(313.031375,195.8800012)
\curveto(313.031375,199.76799731)(311.15937111,202.0480012)(307.271375,202.0480012)
\moveto(304.031375,195.8800012)
\curveto(304.031375,199.19199789)(305.08737718,201.2080012)(307.271375,201.2080012)
\curveto(309.50337277,201.2080012)(310.535375,199.19199789)(310.535375,195.8800012)
\curveto(310.535375,192.56800451)(309.50337277,190.5520012)(307.271375,190.5520012)
\curveto(305.08737718,190.5520012)(304.031375,192.56800451)(304.031375,195.8800012)
}
}
{
\newrgbcolor{curcolor}{1 1 1}
\pscustom[linestyle=none,fillstyle=solid,fillcolor=curcolor]
{
\newpath
\moveto(330,210.0000012)
\lineto(350,210.0000012)
\lineto(350,190.0000012)
\lineto(330,190.0000012)
\lineto(330,210.0000012)
\closepath
}
}
{
\newrgbcolor{curcolor}{0 0 0}
\pscustom[linewidth=2,linecolor=curcolor]
{
\newpath
\moveto(330,210.0000012)
\lineto(350,210.0000012)
\lineto(350,190.0000012)
\lineto(330,190.0000012)
\lineto(330,210.0000012)
\closepath
}
}
{
\newrgbcolor{curcolor}{0 0 0}
\pscustom[linewidth=2,linecolor=curcolor,linestyle=dashed,dash=8 8]
{
\newpath
\moveto(340,200)
\lineto(500,200)
}
}
{
\newrgbcolor{curcolor}{0 0 0}
\pscustom[linestyle=none,fillstyle=solid,fillcolor=curcolor]
{
\newpath
\moveto(350.46230536,195.15951776)
\lineto(337.3512526,199.98078409)
\lineto(350.46230608,204.80204936)
\curveto(348.367708,201.95557628)(348.37977712,198.06110708)(350.46230536,195.15951776)
\lineto(350.46230536,195.15951776)
\lineto(350.46230536,195.15951776)
\closepath
}
}
{
\newrgbcolor{curcolor}{0 0 0}
\pscustom[linestyle=none,fillstyle=solid,fillcolor=curcolor]
{
\newpath
\moveto(47.328,118.5760012)
\curveto(49.31199802,118.5760012)(50.97600109,119.69600277)(52.064,121.2640012)
\lineto(52.128,121.2640012)
\curveto(52.128,119.24800322)(51.96799942,116.65599899)(51.392,114.4480012)
\curveto(50.81600058,112.33600331)(49.69599795,110.7360012)(47.648,110.7360012)
\curveto(45.47200218,110.7360012)(44.67199952,112.43200271)(44.192,113.9360012)
\curveto(44.06400013,114.35200079)(43.83999946,114.6080012)(43.296,114.6080012)
\curveto(42.68800061,114.6080012)(41.792,113.96800021)(41.792,112.9760012)
\curveto(41.792,111.56800261)(43.48800387,109.6160012)(47.36,109.6160012)
\curveto(49.95199741,109.6160012)(51.93600122,110.5760029)(53.152,112.2720012)
\curveto(54.46399869,114.09599938)(55.264,116.91200584)(55.264,121.5520012)
\curveto(55.264,125.51999723)(54.46399875,128.11200271)(53.216,129.6160012)
\curveto(52.00000122,131.05599976)(50.23999795,131.7600012)(48.192,131.7600012)
\curveto(44.0960041,131.7600012)(41.152,129.03999679)(41.152,124.6240012)
\curveto(41.152,120.6240052)(44.0320033,118.5760012)(47.328,118.5760012)
\moveto(48.288,120.2720012)
\curveto(45.7920025,120.2720012)(44.352,122.44800427)(44.352,125.5200012)
\curveto(44.352,128.46399826)(45.6640025,130.6400012)(48.16,130.6400012)
\curveto(50.71999744,130.6400012)(51.968,127.5999981)(51.968,124.4960012)
\curveto(51.968,123.95200175)(51.93599981,123.31200069)(51.744,122.8000012)
\curveto(51.23200051,121.29600271)(49.9839983,120.2720012)(48.288,120.2720012)
}
}
{
\newrgbcolor{curcolor}{0 0 0}
\pscustom[linestyle=none,fillstyle=solid,fillcolor=curcolor]
{
\newpath
\moveto(136.504,150.0000012)
\lineto(136.504,150.6960012)
\curveto(134.94400156,150.81600108)(134.36799986,150.88800221)(134.224,151.8960012)
\curveto(134.10400012,152.68800041)(134.056,153.60000247)(134.056,154.8720012)
\lineto(134.056,164.4000012)
\lineto(134.104,164.4000012)
\lineto(140.296,150.2880012)
\lineto(140.944,150.2880012)
\lineto(147.208,164.4000012)
\lineto(147.28,164.4000012)
\lineto(147.28,152.2320012)
\curveto(147.28,150.88800255)(147.23199789,150.84000106)(145.12,150.6960012)
\lineto(145.12,150.0000012)
\lineto(151.816,150.0000012)
\lineto(151.816,150.6960012)
\curveto(149.72800209,150.84000106)(149.656,150.88800255)(149.656,152.2320012)
\lineto(149.656,164.3760012)
\curveto(149.656,165.71999986)(149.72800209,165.76800135)(151.816,165.9120012)
\lineto(151.816,166.6080012)
\lineto(146.992,166.6080012)
\curveto(146.17600082,164.47200334)(145.19199904,162.35999909)(144.232,160.2480012)
\lineto(141.352,153.8160012)
\lineto(141.304,153.8160012)
\lineto(138.472,160.2240012)
\curveto(137.53600094,162.35999907)(136.52799914,164.47200334)(135.664,166.6080012)
\lineto(130.792,166.6080012)
\lineto(130.792,165.9120012)
\curveto(132.90399789,165.76800135)(132.952,165.71999986)(132.952,164.3760012)
\lineto(132.952,154.8720012)
\curveto(132.952,153.60000247)(132.90399988,152.68800041)(132.784,151.8960012)
\curveto(132.64000014,150.88800221)(132.06399875,150.79200111)(130.816,150.6960012)
\lineto(130.816,150.0000012)
\lineto(136.504,150.0000012)
}
}
{
\newrgbcolor{curcolor}{0 0 0}
\pscustom[linestyle=none,fillstyle=solid,fillcolor=curcolor]
{
\newpath
\moveto(163.224625,156.3120012)
\curveto(163.7286245,156.3120012)(164.184625,156.43200199)(164.184625,157.2240012)
\curveto(164.184625,158.63999979)(163.72862061,162.0480012)(159.336625,162.0480012)
\curveto(155.59262874,162.0480012)(153.816625,159.3839976)(153.816625,155.7840012)
\curveto(153.816625,151.99200499)(155.44862891,149.66400125)(159.360625,149.7120012)
\curveto(162.02462234,149.73600118)(163.39262565,151.2000031)(164.040625,153.0960012)
\lineto(163.320625,153.4800012)
\curveto(162.64862567,152.08800259)(161.7366231,150.9360012)(159.840625,150.9360012)
\curveto(156.86462798,150.9360012)(156.26462505,153.8160037)(156.312625,156.3120012)
\lineto(163.224625,156.3120012)
\moveto(156.336625,157.2240012)
\curveto(156.336625,158.20800022)(156.6966275,161.2080012)(159.192625,161.2080012)
\curveto(161.42462277,161.2080012)(161.688625,158.92800027)(161.688625,157.9920012)
\curveto(161.688625,157.53600166)(161.54462433,157.2240012)(160.872625,157.2240012)
\lineto(156.336625,157.2240012)
\moveto(157.824625,163.0080012)
\lineto(161.448625,165.0240012)
\curveto(162.16862428,165.40800082)(162.792625,165.81600171)(162.792625,166.3200012)
\curveto(162.792625,166.80000072)(162.21662464,167.3520012)(161.856625,167.3520012)
\curveto(161.44862541,167.3520012)(161.04062428,167.11200046)(160.320625,166.3680012)
\lineto(157.464625,163.4880012)
\lineto(157.824625,163.0080012)
}
}
{
\newrgbcolor{curcolor}{0 0 0}
\pscustom[linestyle=none,fillstyle=solid,fillcolor=curcolor]
{
\newpath
\moveto(165.85675,161.4480012)
\lineto(165.85675,160.8480012)
\lineto(166.76875,160.7280012)
\curveto(167.39274938,160.6320013)(167.53675,160.51200034)(167.53675,159.6480012)
\lineto(167.53675,152.0880012)
\curveto(167.53675,150.86400243)(167.46474827,150.81600108)(165.73675,150.6960012)
\lineto(165.73675,150.0000012)
\lineto(171.59275,150.0000012)
\lineto(171.59275,150.6960012)
\curveto(169.8887517,150.81600108)(169.79275,150.86400243)(169.79275,152.0880012)
\lineto(169.79275,157.7040012)
\curveto(169.79275,158.3040006)(169.81675017,158.64000154)(169.98475,158.9760012)
\curveto(170.44074954,159.93600024)(171.49675132,160.7520012)(172.81675,160.7520012)
\curveto(174.88074794,160.7520012)(175.12075,159.23999971)(175.12075,157.7520012)
\lineto(175.12075,152.0880012)
\curveto(175.12075,150.86400243)(175.04874827,150.81600108)(173.32075,150.6960012)
\lineto(173.32075,150.0000012)
\lineto(179.17675,150.0000012)
\lineto(179.17675,150.6960012)
\curveto(177.4727517,150.81600108)(177.37675,150.86400243)(177.37675,152.0880012)
\lineto(177.37675,157.7040012)
\curveto(177.37675,158.3040006)(177.40075014,158.64000159)(177.54475,159.0240012)
\curveto(177.90474964,159.98400024)(178.93675134,160.7520012)(180.28075,160.7520012)
\curveto(181.79274849,160.80000115)(182.70475,159.83999911)(182.70475,157.7520012)
\lineto(182.70475,152.0880012)
\curveto(182.70475,150.86400243)(182.63274827,150.81600108)(180.90475,150.6960012)
\lineto(180.90475,150.0000012)
\lineto(186.76075,150.0000012)
\lineto(186.76075,150.6960012)
\curveto(185.0567517,150.81600108)(184.96075,150.86400243)(184.96075,152.0880012)
\lineto(184.96075,158.2320012)
\curveto(184.96075,160.65599878)(183.66474772,162.0480012)(181.38475,162.0480012)
\curveto(179.41675197,162.0480012)(178.00074916,160.72800067)(177.16075,160.2000012)
\curveto(176.58475058,161.32800007)(175.67274832,162.0480012)(173.99275,162.0480012)
\curveto(172.14475185,162.0480012)(170.63274916,160.84800058)(169.79275,160.2240012)
\lineto(169.57675,162.0480012)
\lineto(165.85675,161.4480012)
}
}
{
\newrgbcolor{curcolor}{0 0 0}
\pscustom[linestyle=none,fillstyle=solid,fillcolor=curcolor]
{
\newpath
\moveto(194.052625,162.0480012)
\curveto(190.21262884,162.0480012)(188.316625,159.76799731)(188.316625,155.8800012)
\curveto(188.316625,151.99200509)(190.21262884,149.7120012)(194.052625,149.7120012)
\curveto(197.94062111,149.7120012)(199.812625,151.99200509)(199.812625,155.8800012)
\curveto(199.812625,159.76799731)(197.94062111,162.0480012)(194.052625,162.0480012)
\moveto(190.812625,155.8800012)
\curveto(190.812625,159.19199789)(191.86862718,161.2080012)(194.052625,161.2080012)
\curveto(196.28462277,161.2080012)(197.316625,159.19199789)(197.316625,155.8800012)
\curveto(197.316625,152.56800451)(196.28462277,150.5520012)(194.052625,150.5520012)
\curveto(191.86862718,150.5520012)(190.812625,152.56800451)(190.812625,155.8800012)
}
}
{
\newrgbcolor{curcolor}{0 0 0}
\pscustom[linestyle=none,fillstyle=solid,fillcolor=curcolor]
{
\newpath
\moveto(205.50475,162.0480012)
\lineto(201.85675,161.4480012)
\lineto(201.85675,160.8480012)
\lineto(202.76875,160.7280012)
\curveto(203.39274938,160.6320013)(203.53675,160.51200034)(203.53675,159.6480012)
\lineto(203.53675,152.0880012)
\curveto(203.53675,150.86400243)(203.41674832,150.81600108)(201.73675,150.6960012)
\lineto(201.73675,150.0000012)
\lineto(208.07275,150.0000012)
\lineto(208.07275,150.6960012)
\curveto(205.93675214,150.81600108)(205.79275,150.86400243)(205.79275,152.0880012)
\lineto(205.79275,157.7040012)
\curveto(205.79275,159.55199935)(206.60875055,160.1760012)(207.16075,160.1760012)
\curveto(207.54474962,160.1760012)(207.9527507,160.03200082)(208.64875,159.6480012)
\curveto(208.81674983,159.5520013)(209.00875012,159.5280012)(209.12875,159.5280012)
\curveto(209.70474942,159.5280012)(210.23275,160.12800195)(210.23275,160.8720012)
\curveto(210.23275,161.40000067)(209.89674906,162.0480012)(208.96075,162.0480012)
\curveto(208.09675086,162.0480012)(207.37674842,161.51999981)(205.79275,160.1280012)
\lineto(205.50475,162.0480012)
}
}
{
\newrgbcolor{curcolor}{0 0 0}
\pscustom[linestyle=none,fillstyle=solid,fillcolor=curcolor]
{
\newpath
\moveto(211.184875,161.4480012)
\lineto(211.184875,160.8480012)
\lineto(212.096875,160.7280012)
\curveto(212.72087438,160.6320013)(212.864875,160.51200034)(212.864875,159.6480012)
\lineto(212.864875,152.0880012)
\curveto(212.864875,150.86400243)(212.79287327,150.81600108)(211.064875,150.6960012)
\lineto(211.064875,150.0000012)
\lineto(216.920875,150.0000012)
\lineto(216.920875,150.6960012)
\curveto(215.2168767,150.81600108)(215.120875,150.86400243)(215.120875,152.0880012)
\lineto(215.120875,161.9040012)
\lineto(214.952875,162.0480012)
\lineto(211.184875,161.4480012)
\moveto(213.968875,167.1600012)
\curveto(213.10487586,167.1600012)(212.504875,166.53600034)(212.504875,165.6720012)
\curveto(212.504875,164.83200204)(213.10487586,164.2320012)(213.968875,164.2320012)
\curveto(214.85687411,164.2320012)(215.40887502,164.83200204)(215.432875,165.6720012)
\curveto(215.432875,166.53600034)(214.85687411,167.1600012)(213.968875,167.1600012)
}
}
{
\newrgbcolor{curcolor}{0 0 0}
\pscustom[linestyle=none,fillstyle=solid,fillcolor=curcolor]
{
\newpath
\moveto(226.08925,158.4240012)
\lineto(226.08925,161.2320012)
\curveto(225.29725079,161.80800063)(224.07324892,162.0480012)(222.99325,162.0480012)
\curveto(220.40125259,162.0480012)(218.64924998,160.82399892)(218.62525,158.5440012)
\curveto(218.64924998,156.55200319)(220.25725199,155.6640006)(222.24925,155.0640012)
\curveto(223.32924892,154.72800154)(224.84125,154.17599957)(224.84125,152.5440012)
\curveto(224.84125,151.32000243)(223.88124873,150.5520012)(222.60925,150.5520012)
\curveto(220.66525194,150.5520012)(219.58524952,151.96800324)(219.10525,154.0080012)
\lineto(218.40925,154.0080012)
\lineto(218.64925,150.8640012)
\curveto(219.51324914,150.09600197)(220.97725142,149.7120012)(222.39325,149.7120012)
\curveto(225.24924714,149.7120012)(226.88125,151.22400324)(226.88125,153.2640012)
\curveto(226.88125,155.37599909)(225.58524753,156.33600197)(223.11325,157.1040012)
\curveto(222.12925098,157.41600089)(220.54525,157.92000255)(220.54525,159.2640012)
\curveto(220.56924998,160.51199995)(221.5052512,161.2080012)(222.70525,161.2080012)
\curveto(224.4092483,161.2080012)(225.22525017,159.88799974)(225.39325,158.4240012)
\lineto(226.08925,158.4240012)
}
}
{
\newrgbcolor{curcolor}{0 0 0}
\pscustom[linestyle=none,fillstyle=solid,fillcolor=curcolor]
{
\newpath
\moveto(238.318375,156.3120012)
\curveto(238.8223745,156.3120012)(239.278375,156.43200199)(239.278375,157.2240012)
\curveto(239.278375,158.63999979)(238.82237061,162.0480012)(234.430375,162.0480012)
\curveto(230.68637874,162.0480012)(228.910375,159.3839976)(228.910375,155.7840012)
\curveto(228.910375,151.99200499)(230.54237891,149.66400125)(234.454375,149.7120012)
\curveto(237.11837234,149.73600118)(238.48637565,151.2000031)(239.134375,153.0960012)
\lineto(238.414375,153.4800012)
\curveto(237.74237567,152.08800259)(236.8303731,150.9360012)(234.934375,150.9360012)
\curveto(231.95837798,150.9360012)(231.35837505,153.8160037)(231.406375,156.3120012)
\lineto(238.318375,156.3120012)
\moveto(231.430375,157.2240012)
\curveto(231.430375,158.20800022)(231.7903775,161.2080012)(234.286375,161.2080012)
\curveto(236.51837277,161.2080012)(236.782375,158.92800027)(236.782375,157.9920012)
\curveto(236.782375,157.53600166)(236.63837433,157.2240012)(235.966375,157.2240012)
\lineto(231.430375,157.2240012)
}
}
{
\newrgbcolor{curcolor}{0 0 0}
\pscustom[linestyle=none,fillstyle=solid,fillcolor=curcolor]
{
\newpath
\moveto(244.5985,162.0480012)
\lineto(240.9505,161.4480012)
\lineto(240.9505,160.8480012)
\lineto(241.8625,160.7280012)
\curveto(242.48649938,160.6320013)(242.6305,160.51200034)(242.6305,159.6480012)
\lineto(242.6305,152.0880012)
\curveto(242.6305,150.86400243)(242.51049832,150.81600108)(240.8305,150.6960012)
\lineto(240.8305,150.0000012)
\lineto(247.1665,150.0000012)
\lineto(247.1665,150.6960012)
\curveto(245.03050214,150.81600108)(244.8865,150.86400243)(244.8865,152.0880012)
\lineto(244.8865,157.7040012)
\curveto(244.8865,159.55199935)(245.70250055,160.1760012)(246.2545,160.1760012)
\curveto(246.63849962,160.1760012)(247.0465007,160.03200082)(247.7425,159.6480012)
\curveto(247.91049983,159.5520013)(248.10250012,159.5280012)(248.2225,159.5280012)
\curveto(248.79849942,159.5280012)(249.3265,160.12800195)(249.3265,160.8720012)
\curveto(249.3265,161.40000067)(248.99049906,162.0480012)(248.0545,162.0480012)
\curveto(247.19050086,162.0480012)(246.47049842,161.51999981)(244.8865,160.1280012)
\lineto(244.5985,162.0480012)
}
}
{
\newrgbcolor{curcolor}{0 0 0}
\pscustom[linestyle=none,fillstyle=solid,fillcolor=curcolor]
{
}
}
{
\newrgbcolor{curcolor}{0 0 0}
\pscustom[linestyle=none,fillstyle=solid,fillcolor=curcolor]
{
\newpath
\moveto(255.66925,161.4480012)
\lineto(255.66925,160.8480012)
\lineto(256.58125,160.7280012)
\curveto(257.20524938,160.6320013)(257.34925,160.51200034)(257.34925,159.6480012)
\lineto(257.34925,152.0880012)
\curveto(257.34925,150.86400243)(257.27724827,150.81600108)(255.54925,150.6960012)
\lineto(255.54925,150.0000012)
\lineto(261.40525,150.0000012)
\lineto(261.40525,150.6960012)
\curveto(259.7012517,150.81600108)(259.60525,150.86400243)(259.60525,152.0880012)
\lineto(259.60525,157.7040012)
\curveto(259.60525,158.3040006)(259.62925017,158.64000154)(259.79725,158.9760012)
\curveto(260.25324954,159.93600024)(261.30925132,160.7520012)(262.62925,160.7520012)
\curveto(264.69324794,160.7520012)(264.93325,159.23999971)(264.93325,157.7520012)
\lineto(264.93325,152.0880012)
\curveto(264.93325,150.86400243)(264.86124827,150.81600108)(263.13325,150.6960012)
\lineto(263.13325,150.0000012)
\lineto(268.98925,150.0000012)
\lineto(268.98925,150.6960012)
\curveto(267.2852517,150.81600108)(267.18925,150.86400243)(267.18925,152.0880012)
\lineto(267.18925,157.7040012)
\curveto(267.18925,158.3040006)(267.21325014,158.64000159)(267.35725,159.0240012)
\curveto(267.71724964,159.98400024)(268.74925134,160.7520012)(270.09325,160.7520012)
\curveto(271.60524849,160.80000115)(272.51725,159.83999911)(272.51725,157.7520012)
\lineto(272.51725,152.0880012)
\curveto(272.51725,150.86400243)(272.44524827,150.81600108)(270.71725,150.6960012)
\lineto(270.71725,150.0000012)
\lineto(276.57325,150.0000012)
\lineto(276.57325,150.6960012)
\curveto(274.8692517,150.81600108)(274.77325,150.86400243)(274.77325,152.0880012)
\lineto(274.77325,158.2320012)
\curveto(274.77325,160.65599878)(273.47724772,162.0480012)(271.19725,162.0480012)
\curveto(269.22925197,162.0480012)(267.81324916,160.72800067)(266.97325,160.2000012)
\curveto(266.39725058,161.32800007)(265.48524832,162.0480012)(263.80525,162.0480012)
\curveto(261.95725185,162.0480012)(260.44524916,160.84800058)(259.60525,160.2240012)
\lineto(259.38925,162.0480012)
\lineto(255.66925,161.4480012)
}
}
{
\newrgbcolor{curcolor}{0 0 0}
\pscustom[linestyle=none,fillstyle=solid,fillcolor=curcolor]
{
\newpath
\moveto(283.865125,162.0480012)
\curveto(280.02512884,162.0480012)(278.129125,159.76799731)(278.129125,155.8800012)
\curveto(278.129125,151.99200509)(280.02512884,149.7120012)(283.865125,149.7120012)
\curveto(287.75312111,149.7120012)(289.625125,151.99200509)(289.625125,155.8800012)
\curveto(289.625125,159.76799731)(287.75312111,162.0480012)(283.865125,162.0480012)
\moveto(280.625125,155.8800012)
\curveto(280.625125,159.19199789)(281.68112718,161.2080012)(283.865125,161.2080012)
\curveto(286.09712277,161.2080012)(287.129125,159.19199789)(287.129125,155.8800012)
\curveto(287.129125,152.56800451)(286.09712277,150.5520012)(283.865125,150.5520012)
\curveto(281.68112718,150.5520012)(280.625125,152.56800451)(280.625125,155.8800012)
}
}
{
\newrgbcolor{curcolor}{0 0 0}
\pscustom[linestyle=none,fillstyle=solid,fillcolor=curcolor]
{
\newpath
\moveto(298.60525,160.8000012)
\lineto(298.60525,161.7600012)
\lineto(295.50925,161.7600012)
\lineto(295.50925,164.7840012)
\lineto(294.81325,164.7840012)
\lineto(293.34925,161.7600012)
\lineto(291.52525,161.7600012)
\lineto(291.52525,160.8000012)
\lineto(293.25325,160.8000012)
\lineto(293.25325,152.4480012)
\curveto(293.25325,150.00000365)(295.07725084,149.7120012)(295.91725,149.7120012)
\curveto(297.14124878,149.7120012)(298.2932507,150.33600163)(298.98925,150.7680012)
\lineto(298.79725,151.2720012)
\curveto(298.22125058,151.03200144)(297.6932494,150.9600012)(297.09325,150.9600012)
\curveto(296.27725082,150.9600012)(295.50925,151.53600303)(295.50925,153.3600012)
\lineto(295.50925,160.8000012)
\lineto(298.60525,160.8000012)
}
}
{
\newrgbcolor{curcolor}{0 0 0}
\pscustom[linestyle=none,fillstyle=solid,fillcolor=curcolor]
{
}
}
{
\newrgbcolor{curcolor}{0 0 0}
\pscustom[linestyle=none,fillstyle=solid,fillcolor=curcolor]
{
\newpath
\moveto(318.413875,150.0000012)
\lineto(318.413875,150.6960012)
\curveto(316.7098767,150.81600108)(316.613875,150.86400243)(316.613875,152.0880012)
\lineto(316.613875,167.6640012)
\lineto(316.445875,167.8080012)
\lineto(312.677875,167.2080012)
\lineto(312.677875,166.6080012)
\lineto(313.589875,166.5120012)
\curveto(314.21387438,166.44000127)(314.357875,166.32000029)(314.357875,165.4080012)
\lineto(314.357875,161.4720012)
\curveto(313.68587567,161.83200084)(312.7978739,162.0480012)(311.693875,162.0480012)
\curveto(309.72587697,162.0480012)(308.26187399,161.42400017)(307.253875,160.3920012)
\curveto(306.22187603,159.31200228)(305.597875,157.72799911)(305.597875,155.6400012)
\curveto(305.597875,152.08800475)(307.44587817,149.7120012)(310.613875,149.7120012)
\curveto(311.9098737,149.7120012)(313.0378763,150.28800226)(314.333875,151.3440012)
\lineto(314.333875,150.1680012)
\lineto(314.525875,150.0000012)
\lineto(318.413875,150.0000012)
\moveto(314.357875,154.0800012)
\curveto(314.357875,153.55200173)(314.33387486,153.14400079)(314.189875,152.7360012)
\curveto(313.70987548,151.48800245)(312.65387375,151.0080012)(311.405875,151.0080012)
\curveto(309.19787721,151.0080012)(308.093875,153.16800396)(308.093875,155.9280012)
\curveto(308.093875,159.04799808)(309.22187745,161.2080012)(311.669875,161.2080012)
\curveto(312.79787387,161.2080012)(313.70987543,160.68000022)(314.141875,159.6960012)
\curveto(314.33387481,159.28800161)(314.357875,158.95200055)(314.357875,158.3040012)
\lineto(314.357875,154.0800012)
}
}
{
\newrgbcolor{curcolor}{0 0 0}
\pscustom[linestyle=none,fillstyle=solid,fillcolor=curcolor]
{
\newpath
\moveto(329.349625,156.3120012)
\curveto(329.8536245,156.3120012)(330.309625,156.43200199)(330.309625,157.2240012)
\curveto(330.309625,158.63999979)(329.85362061,162.0480012)(325.461625,162.0480012)
\curveto(321.71762874,162.0480012)(319.941625,159.3839976)(319.941625,155.7840012)
\curveto(319.941625,151.99200499)(321.57362891,149.66400125)(325.485625,149.7120012)
\curveto(328.14962234,149.73600118)(329.51762565,151.2000031)(330.165625,153.0960012)
\lineto(329.445625,153.4800012)
\curveto(328.77362567,152.08800259)(327.8616231,150.9360012)(325.965625,150.9360012)
\curveto(322.98962798,150.9360012)(322.38962505,153.8160037)(322.437625,156.3120012)
\lineto(329.349625,156.3120012)
\moveto(322.461625,157.2240012)
\curveto(322.461625,158.20800022)(322.8216275,161.2080012)(325.317625,161.2080012)
\curveto(327.54962277,161.2080012)(327.813625,158.92800027)(327.813625,157.9920012)
\curveto(327.813625,157.53600166)(327.66962433,157.2240012)(326.997625,157.2240012)
\lineto(322.461625,157.2240012)
}
}
{
\newrgbcolor{curcolor}{0 0 0}
\pscustom[linestyle=none,fillstyle=solid,fillcolor=curcolor]
{
}
}
{
\newrgbcolor{curcolor}{0 0 0}
\pscustom[linestyle=none,fillstyle=solid,fillcolor=curcolor]
{
\newpath
\moveto(341.140375,157.7040012)
\curveto(341.140375,158.3040006)(341.23637522,158.64000163)(341.452375,159.0720012)
\curveto(342.00437445,160.1760001)(343.03637627,160.7520012)(344.308375,160.7520012)
\curveto(345.26837404,160.7520012)(347.404375,160.12799707)(347.404375,156.0000012)
\curveto(347.404375,152.47200473)(346.3003726,150.5520012)(343.900375,150.5520012)
\curveto(342.65237625,150.5520012)(341.69237462,151.20000231)(341.308375,152.3040012)
\curveto(341.16437514,152.73600077)(341.140375,153.21600175)(341.140375,153.7680012)
\lineto(341.140375,157.7040012)
\moveto(337.204375,161.4480012)
\lineto(337.204375,160.8480012)
\lineto(338.116375,160.7280012)
\curveto(338.74037438,160.6320013)(338.884375,160.51200034)(338.884375,159.6480012)
\lineto(338.884375,146.5680012)
\curveto(338.884375,145.34400243)(338.76437332,145.29600108)(337.084375,145.1760012)
\lineto(337.084375,144.4800012)
\lineto(343.300375,144.4800012)
\lineto(343.300375,145.1760012)
\curveto(341.28437702,145.29600108)(341.140375,145.34400243)(341.140375,146.5680012)
\lineto(341.140375,150.2640012)
\curveto(341.6443745,149.97600149)(342.7723761,149.7120012)(343.876375,149.7120012)
\curveto(347.11637176,149.7120012)(349.900375,151.20000615)(349.900375,156.1440012)
\curveto(349.900375,157.8479995)(349.58837063,162.0480012)(345.220375,162.0480012)
\curveto(343.46837675,162.0480012)(342.07637406,160.8720006)(341.140375,160.2720012)
\lineto(340.828375,162.0480012)
\lineto(337.204375,161.4480012)
}
}
{
\newrgbcolor{curcolor}{0 0 0}
\pscustom[linestyle=none,fillstyle=solid,fillcolor=curcolor]
{
\newpath
\moveto(359.2945,153.4320012)
\curveto(359.2945,151.48800315)(357.80649906,150.9840012)(356.8705,150.9840012)
\curveto(355.38250149,150.9840012)(354.6625,152.04000262)(354.6625,153.4560012)
\curveto(354.6625,154.58400007)(355.19050134,155.16000171)(356.5345,155.6640012)
\curveto(357.49449904,156.02400084)(358.76650053,156.48000154)(359.2945,156.8160012)
\lineto(359.2945,153.4320012)
\moveto(361.4545,158.8560012)
\curveto(361.4545,160.24799981)(361.14249638,162.0480012)(357.5185,162.0480012)
\curveto(354.80650271,162.0480012)(352.6705,160.63199988)(352.6705,159.3120012)
\curveto(352.6705,158.54400197)(353.55850046,158.1840012)(354.0145,158.1840012)
\curveto(354.5184995,158.1840012)(354.66250012,158.44800161)(354.7825,158.8560012)
\curveto(355.31049947,160.63199943)(356.29450096,161.2080012)(357.2545,161.2080012)
\curveto(358.19049906,161.2080012)(359.2945,160.72799928)(359.2945,158.8080012)
\lineto(359.2945,157.8000012)
\curveto(358.6945006,157.17600183)(356.3664981,156.5760006)(354.4705,155.9760012)
\curveto(352.74250173,155.44800173)(352.2385,154.24800007)(352.2385,153.1200012)
\curveto(352.2385,151.320003)(353.43850226,149.7120012)(355.6945,149.7120012)
\curveto(357.18249851,149.76000115)(358.50250077,150.64800171)(359.2705,151.1520012)
\curveto(359.60649966,150.26400209)(359.99050084,149.7120012)(360.8305,149.7120012)
\curveto(361.71849911,149.7120012)(362.75050091,149.97600166)(363.6625,150.4320012)
\lineto(363.5185,151.0080012)
\curveto(363.18250034,150.93600127)(362.65449964,150.8880013)(362.2945,150.9840012)
\curveto(361.86250043,151.08000111)(361.4545,151.53600271)(361.4545,153.0480012)
\lineto(361.4545,158.8560012)
}
}
{
\newrgbcolor{curcolor}{0 0 0}
\pscustom[linestyle=none,fillstyle=solid,fillcolor=curcolor]
{
\newpath
\moveto(372.433,158.4240012)
\lineto(372.433,161.2320012)
\curveto(371.64100079,161.80800063)(370.41699892,162.0480012)(369.337,162.0480012)
\curveto(366.74500259,162.0480012)(364.99299998,160.82399892)(364.969,158.5440012)
\curveto(364.99299998,156.55200319)(366.60100199,155.6640006)(368.593,155.0640012)
\curveto(369.67299892,154.72800154)(371.185,154.17599957)(371.185,152.5440012)
\curveto(371.185,151.32000243)(370.22499873,150.5520012)(368.953,150.5520012)
\curveto(367.00900194,150.5520012)(365.92899952,151.96800324)(365.449,154.0080012)
\lineto(364.753,154.0080012)
\lineto(364.993,150.8640012)
\curveto(365.85699914,150.09600197)(367.32100142,149.7120012)(368.737,149.7120012)
\curveto(371.59299714,149.7120012)(373.225,151.22400324)(373.225,153.2640012)
\curveto(373.225,155.37599909)(371.92899753,156.33600197)(369.457,157.1040012)
\curveto(368.47300098,157.41600089)(366.889,157.92000255)(366.889,159.2640012)
\curveto(366.91299998,160.51199995)(367.8490012,161.2080012)(369.049,161.2080012)
\curveto(370.7529983,161.2080012)(371.56900017,159.88799974)(371.737,158.4240012)
\lineto(372.433,158.4240012)
}
}
{
\newrgbcolor{curcolor}{0 0 0}
\pscustom[linestyle=none,fillstyle=solid,fillcolor=curcolor]
{
\newpath
\moveto(382.886125,158.4240012)
\lineto(382.886125,161.2320012)
\curveto(382.09412579,161.80800063)(380.87012392,162.0480012)(379.790125,162.0480012)
\curveto(377.19812759,162.0480012)(375.44612498,160.82399892)(375.422125,158.5440012)
\curveto(375.44612498,156.55200319)(377.05412699,155.6640006)(379.046125,155.0640012)
\curveto(380.12612392,154.72800154)(381.638125,154.17599957)(381.638125,152.5440012)
\curveto(381.638125,151.32000243)(380.67812373,150.5520012)(379.406125,150.5520012)
\curveto(377.46212694,150.5520012)(376.38212452,151.96800324)(375.902125,154.0080012)
\lineto(375.206125,154.0080012)
\lineto(375.446125,150.8640012)
\curveto(376.31012414,150.09600197)(377.77412642,149.7120012)(379.190125,149.7120012)
\curveto(382.04612214,149.7120012)(383.678125,151.22400324)(383.678125,153.2640012)
\curveto(383.678125,155.37599909)(382.38212253,156.33600197)(379.910125,157.1040012)
\curveto(378.92612598,157.41600089)(377.342125,157.92000255)(377.342125,159.2640012)
\curveto(377.36612498,160.51199995)(378.3021262,161.2080012)(379.502125,161.2080012)
\curveto(381.2061233,161.2080012)(382.02212517,159.88799974)(382.190125,158.4240012)
\lineto(382.886125,158.4240012)
}
}
{
\newrgbcolor{curcolor}{0 0 0}
\pscustom[linestyle=none,fillstyle=solid,fillcolor=curcolor]
{
\newpath
\moveto(395.11525,156.3120012)
\curveto(395.6192495,156.3120012)(396.07525,156.43200199)(396.07525,157.2240012)
\curveto(396.07525,158.63999979)(395.61924561,162.0480012)(391.22725,162.0480012)
\curveto(387.48325374,162.0480012)(385.70725,159.3839976)(385.70725,155.7840012)
\curveto(385.70725,151.99200499)(387.33925391,149.66400125)(391.25125,149.7120012)
\curveto(393.91524734,149.73600118)(395.28325065,151.2000031)(395.93125,153.0960012)
\lineto(395.21125,153.4800012)
\curveto(394.53925067,152.08800259)(393.6272481,150.9360012)(391.73125,150.9360012)
\curveto(388.75525298,150.9360012)(388.15525005,153.8160037)(388.20325,156.3120012)
\lineto(395.11525,156.3120012)
\moveto(388.22725,157.2240012)
\curveto(388.22725,158.20800022)(388.5872525,161.2080012)(391.08325,161.2080012)
\curveto(393.31524777,161.2080012)(393.57925,158.92800027)(393.57925,157.9920012)
\curveto(393.57925,157.53600166)(393.43524933,157.2240012)(392.76325,157.2240012)
\lineto(388.22725,157.2240012)
}
}
{
\newrgbcolor{curcolor}{1 1 1}
\pscustom[linestyle=none,fillstyle=solid,fillcolor=curcolor]
{
\newpath
\moveto(410,170.0000012)
\lineto(430,170.0000012)
\lineto(430,150.0000012)
\lineto(410,150.0000012)
\lineto(410,170.0000012)
\closepath
}
}
{
\newrgbcolor{curcolor}{0 0 0}
\pscustom[linewidth=2,linecolor=curcolor]
{
\newpath
\moveto(410,170.0000012)
\lineto(430,170.0000012)
\lineto(430,150.0000012)
\lineto(410,150.0000012)
\lineto(410,170.0000012)
\closepath
}
}
{
\newrgbcolor{curcolor}{0 0 0}
\pscustom[linewidth=2,linecolor=curcolor,linestyle=dashed,dash=8 8]
{
\newpath
\moveto(420,160)
\lineto(500,160)
}
}
{
\newrgbcolor{curcolor}{0 0 0}
\pscustom[linestyle=none,fillstyle=solid,fillcolor=curcolor]
{
\newpath
\moveto(430.46230536,155.15951776)
\lineto(417.3512526,159.98078409)
\lineto(430.46230608,164.80204936)
\curveto(428.367708,161.95557628)(428.37977712,158.06110708)(430.46230536,155.15951776)
\lineto(430.46230536,155.15951776)
\lineto(430.46230536,155.15951776)
\closepath
}
}
{
\newrgbcolor{curcolor}{0 0 0}
\pscustom[linestyle=none,fillstyle=solid,fillcolor=curcolor]
{
\newpath
\moveto(513.984,110.0000012)
\lineto(513.984,110.9280012)
\lineto(511.296,111.1520012)
\curveto(510.62400067,111.21600114)(510.24,111.47200245)(510.24,112.7200012)
\lineto(510.24,131.5680012)
\lineto(510.08,131.7600012)
\lineto(503.488,130.6400012)
\lineto(503.488,129.8400012)
\lineto(506.464,129.4880012)
\curveto(507.00799946,129.42400127)(507.232,129.16800027)(507.232,128.2400012)
\lineto(507.232,112.7200012)
\curveto(507.232,112.11200181)(507.13599981,111.72800098)(506.944,111.5040012)
\curveto(506.78400016,111.28000143)(506.52799965,111.18400117)(506.176,111.1520012)
\lineto(503.488,110.9280012)
\lineto(503.488,110.0000012)
\lineto(513.984,110.0000012)
}
}
{
\newrgbcolor{curcolor}{0 0 0}
\pscustom[linestyle=none,fillstyle=solid,fillcolor=curcolor]
{
\newpath
\moveto(525.4175,130.6400012)
\curveto(528.45749696,130.6400012)(529.3215,125.90399599)(529.3215,120.6880012)
\curveto(529.3215,115.47200642)(528.45749696,110.7360012)(525.4175,110.7360012)
\curveto(522.37750304,110.7360012)(521.5135,115.47200642)(521.5135,120.6880012)
\curveto(521.5135,125.90399599)(522.37750304,130.6400012)(525.4175,130.6400012)
\moveto(525.4175,131.7600012)
\curveto(520.13750528,131.7600012)(518.2495,127.0879948)(518.2495,120.6880012)
\curveto(518.2495,114.2880076)(520.13750528,109.6160012)(525.4175,109.6160012)
\curveto(530.69749472,109.6160012)(532.5855,114.2880076)(532.5855,120.6880012)
\curveto(532.5855,127.0879948)(530.69749472,131.7600012)(525.4175,131.7600012)
}
}
\end{pspicture}

		
		\begin{enumerate}
		  \item Fond d'écran
		  \item Pseudonyme du compte hors ligne utilisé
		  \item Cadre contenant l'erreur survenue
		  \item Champs de texte ``Nom utilisateur''
		  \item Champs de texte ``Mot de passe''
		  \item Champs de texte ``Vérification''
		  \item Check Box ``Connexion auto''
		  \item Bouton ``\hyperlink{Connexion multi-joueurs}{Retour}''
		  \item Bouton ``\hyperlink{Accueil multi-joueurs}{Valider}''
		\end{enumerate}
		
		$\,$
		
		Une fois le bouton ``Validé'' pressé une vérification des paramètres a alors
		lieu (caractères entrés, concordance des deux mots de passes, etc \ldots), si
		tout est valide une connexion a alors lieu avec le serveur afin de rajouter le
		nouvel utilisateur.
		Là aussi une nouvelle vérification a lieu (utilisateur non existant,
		concordance des deux mots de passes%
		\footnote[1]{
			Les mots de passes sont vérifiés du côté client pour éviter une connexion
			inutile au serveur et une fois sur le serveur au cas où une application
			client n'aurait pas fait ce test.
		}
		, etc \ldots).\\
		Si aucune erreur n'a lieu, le joueur est redirigé vers la page d'accueil
		multi-joueurs%
		\footnote[2]{
			\hyperlink{Accueil multi-joueurs}{Accueil multi-joueurs}
			\og voir section \ref{Accueil multi-joueurs}, page \pageref{Accueil multi-joueurs}.\fg
		}
		sinon le serveur la renvoie à l'application cliente qui l'affichera grâce au
		cadre défini à cette effigie (3).\\
		Le bouton ``Annuler'' permet à l'utilisateur de revenir à la page de connexion
		multi-joueurs
		\footnotemark[1]
		La check box ``Connexion auto'' permet à l'utilisateur d'être connecté
		automatiquement lors de ses prochaines parties en ligne.
				
		$\,$
	
\newpage

	\subsection{Connexion multi-joueurs}
	
		\hypertarget{Connexion multi-joueurs}{}
		\label{Connexion multi-joueurs}
	
		\input{./img/4bis_Connexion_multi_joueurs}
		
		\begin{enumerate}
		  \item Fond d'écran
		  \item Pseudonyme du compte hors ligne utilisé
		  \item Bouton ``\hyperlink{Creation compte multi-joueurs}{Inscription}''
		  \item Cadre contenant l'erreur survenue
		  \item Champs de texte ``Nom utilisateur''
		  \item Champs de texte ``Mot de passe''
		  \item Check Box ``Connexion auto''
		  \item Bouton ``\hyperlink{Page d'accueil}{Retour}''
		  \item Bouton ``\hyperlink{Accueil multi-joueurs}{Connexion}''
		\end{enumerate}
		
		
\newpage

	\subsection{Accueil multi-joueurs}

		\hypertarget{Accueil multi-joueurs}{}
		\label{Accueil multi-joueurs}
	
		%LaTeX with PSTricks extensions
%%Creator: inkscape 0.48.0
%%Please note this file requires PSTricks extensions
\psset{xunit=.5pt,yunit=.5pt,runit=.5pt}
\begin{pspicture}(560,600)
{
\newrgbcolor{curcolor}{1 1 1}
\pscustom[linestyle=none,fillstyle=solid,fillcolor=curcolor]
{
\newpath
\moveto(133.12401581,597.52220317)
\lineto(426.87598419,597.52220317)
\curveto(443.85397169,597.52220317)(457.52217102,583.85400385)(457.52217102,566.87601635)
\lineto(457.52217102,33.12401744)
\curveto(457.52217102,16.14602994)(443.85397169,2.47783062)(426.87598419,2.47783062)
\lineto(133.12401581,2.47783062)
\curveto(116.14602831,2.47783062)(102.47782898,16.14602994)(102.47782898,33.12401744)
\lineto(102.47782898,566.87601635)
\curveto(102.47782898,583.85400385)(116.14602831,597.52220317)(133.12401581,597.52220317)
\closepath
}
}
{
\newrgbcolor{curcolor}{0 0 0}
\pscustom[linewidth=4.95566034,linecolor=curcolor]
{
\newpath
\moveto(133.12401581,597.52220317)
\lineto(426.87598419,597.52220317)
\curveto(443.85397169,597.52220317)(457.52217102,583.85400385)(457.52217102,566.87601635)
\lineto(457.52217102,33.12401744)
\curveto(457.52217102,16.14602994)(443.85397169,2.47783062)(426.87598419,2.47783062)
\lineto(133.12401581,2.47783062)
\curveto(116.14602831,2.47783062)(102.47782898,16.14602994)(102.47782898,33.12401744)
\lineto(102.47782898,566.87601635)
\curveto(102.47782898,583.85400385)(116.14602831,597.52220317)(133.12401581,597.52220317)
\closepath
}
}
{
\newrgbcolor{curcolor}{1 1 1}
\pscustom[linestyle=none,fillstyle=solid,fillcolor=curcolor]
{
\newpath
\moveto(284.99628735,49.69105693)
\lineto(425.16983509,49.69105693)
\curveto(433.42142812,49.69105693)(440.06440735,43.04807771)(440.06440735,34.79648468)
\curveto(440.06440735,26.54489164)(433.42142812,19.90191242)(425.16983509,19.90191242)
\lineto(284.99628735,19.90191242)
\curveto(276.74469431,19.90191242)(270.10171509,26.54489164)(270.10171509,34.79648468)
\curveto(270.10171509,43.04807771)(276.74469431,49.69105693)(284.99628735,49.69105693)
\closepath
}
}
{
\newrgbcolor{curcolor}{0 0 0}
\pscustom[linewidth=2,linecolor=curcolor]
{
\newpath
\moveto(284.99628735,49.69105693)
\lineto(425.16983509,49.69105693)
\curveto(433.42142812,49.69105693)(440.06440735,43.04807771)(440.06440735,34.79648468)
\curveto(440.06440735,26.54489164)(433.42142812,19.90191242)(425.16983509,19.90191242)
\lineto(284.99628735,19.90191242)
\curveto(276.74469431,19.90191242)(270.10171509,26.54489164)(270.10171509,34.79648468)
\curveto(270.10171509,43.04807771)(276.74469431,49.69105693)(284.99628735,49.69105693)
\closepath
}
}
{
\newrgbcolor{curcolor}{0 0 0}
\pscustom[linestyle=none,fillstyle=solid,fillcolor=curcolor]
{
\newpath
\moveto(295.45703125,46.14842388)
\lineto(295.45703125,43.65233013)
\curveto(294.66014159,44.39450323)(293.80857994,44.94919018)(292.90234375,45.31639263)
\curveto(292.00389425,45.68356444)(291.04686395,45.86715801)(290.03125,45.86717388)
\curveto(288.03124197,45.86715801)(286.4999935,45.25387737)(285.4375,44.02733013)
\curveto(284.37499562,42.80856732)(283.84374616,41.04294408)(283.84375,38.73045513)
\curveto(283.84374616,36.4257612)(284.37499562,34.66013797)(285.4375,33.43358013)
\curveto(286.4999935,32.21482791)(288.03124197,31.60545352)(290.03125,31.60545513)
\curveto(291.04686395,31.60545352)(292.00389425,31.78904709)(292.90234375,32.15623638)
\curveto(293.80857994,32.52342135)(294.66014159,33.0781083)(295.45703125,33.82029888)
\lineto(295.45703125,31.34764263)
\curveto(294.62889162,30.78514184)(293.74998625,30.36326726)(292.8203125,30.08201763)
\curveto(291.8984256,29.80076783)(290.92186408,29.66014297)(289.890625,29.66014263)
\curveto(287.24218026,29.66014297)(285.15624484,30.46873591)(283.6328125,32.08592388)
\curveto(282.10937289,33.71092017)(281.3476549,35.9257617)(281.34765625,38.73045513)
\curveto(281.3476549,41.54294358)(282.10937289,43.75778512)(283.6328125,45.37498638)
\curveto(285.15624484,46.99996938)(287.24218026,47.81246856)(289.890625,47.81248638)
\curveto(290.93748906,47.81246856)(291.92186308,47.67184371)(292.84375,47.39061138)
\curveto(293.77342373,47.11715676)(294.64451661,46.70309467)(295.45703125,46.14842388)
}
}
{
\newrgbcolor{curcolor}{0 0 0}
\pscustom[linestyle=none,fillstyle=solid,fillcolor=curcolor]
{
\newpath
\moveto(306.6484375,41.10936138)
\curveto(306.40624037,41.24997513)(306.14061564,41.35153753)(305.8515625,41.41404888)
\curveto(305.57030371,41.48434989)(305.25780402,41.51950611)(304.9140625,41.51951763)
\curveto(303.69530559,41.51950611)(302.75780652,41.12106901)(302.1015625,40.32420513)
\curveto(301.45312033,39.53513309)(301.1289019,38.39841548)(301.12890625,36.91404888)
\lineto(301.12890625,29.99998638)
\lineto(298.9609375,29.99998638)
\lineto(298.9609375,43.12498638)
\lineto(301.12890625,43.12498638)
\lineto(301.12890625,41.08592388)
\curveto(301.58202645,41.88278699)(302.17186961,42.47263015)(302.8984375,42.85545513)
\curveto(303.62499316,43.24606688)(304.50780477,43.44137919)(305.546875,43.44139263)
\curveto(305.69530359,43.44137919)(305.85936592,43.42966045)(306.0390625,43.40623638)
\curveto(306.21874056,43.39059799)(306.41795911,43.36325426)(306.63671875,43.32420513)
\lineto(306.6484375,41.10936138)
}
}
{
\newrgbcolor{curcolor}{0 0 0}
\pscustom[linestyle=none,fillstyle=solid,fillcolor=curcolor]
{
\newpath
\moveto(319.64453125,37.10154888)
\lineto(319.64453125,36.04686138)
\lineto(309.73046875,36.04686138)
\curveto(309.82421508,34.56248181)(310.26952714,33.42967045)(311.06640625,32.64842388)
\curveto(311.87108804,31.8749845)(312.98827442,31.48826614)(314.41796875,31.48826763)
\curveto(315.24608466,31.48826614)(316.04686511,31.58982854)(316.8203125,31.79295513)
\curveto(317.60155105,31.99607813)(318.37498778,32.30076533)(319.140625,32.70701763)
\lineto(319.140625,30.66795513)
\curveto(318.36717529,30.33982979)(317.57420733,30.08983004)(316.76171875,29.91795513)
\curveto(315.94920896,29.74608038)(315.12499103,29.66014297)(314.2890625,29.66014263)
\curveto(312.19530646,29.66014297)(310.53515187,30.26951736)(309.30859375,31.48826763)
\curveto(308.08984182,32.70701492)(307.48046743,34.35545077)(307.48046875,36.43358013)
\curveto(307.48046743,38.58200904)(308.05859185,40.28513234)(309.21484375,41.54295513)
\curveto(310.37890203,42.80856732)(311.94530671,43.44137919)(313.9140625,43.44139263)
\curveto(315.67967798,43.44137919)(317.07420783,42.87106726)(318.09765625,41.73045513)
\curveto(319.12889328,40.59763203)(319.64451776,39.05466482)(319.64453125,37.10154888)
\moveto(317.48828125,37.73436138)
\curveto(317.47264493,38.91403996)(317.14061402,39.85544527)(316.4921875,40.55858013)
\curveto(315.8515528,41.26169387)(314.99999116,41.61325601)(313.9375,41.61326763)
\curveto(312.73436842,41.61325601)(311.76952564,41.2734126)(311.04296875,40.59373638)
\curveto(310.32421458,39.91403896)(309.9101525,38.95700867)(309.80078125,37.72264263)
\lineto(317.48828125,37.73436138)
\moveto(315.4140625,49.19529888)
\lineto(317.74609375,49.19529888)
\lineto(313.92578125,44.78904888)
\lineto(312.1328125,44.78904888)
\lineto(315.4140625,49.19529888)
}
}
{
\newrgbcolor{curcolor}{0 0 0}
\pscustom[linestyle=none,fillstyle=solid,fillcolor=curcolor]
{
\newpath
\moveto(334.41015625,37.10154888)
\lineto(334.41015625,36.04686138)
\lineto(324.49609375,36.04686138)
\curveto(324.58984008,34.56248181)(325.03515214,33.42967045)(325.83203125,32.64842388)
\curveto(326.63671304,31.8749845)(327.75389942,31.48826614)(329.18359375,31.48826763)
\curveto(330.01170966,31.48826614)(330.81249011,31.58982854)(331.5859375,31.79295513)
\curveto(332.36717605,31.99607813)(333.14061278,32.30076533)(333.90625,32.70701763)
\lineto(333.90625,30.66795513)
\curveto(333.13280029,30.33982979)(332.33983233,30.08983004)(331.52734375,29.91795513)
\curveto(330.71483396,29.74608038)(329.89061603,29.66014297)(329.0546875,29.66014263)
\curveto(326.96093146,29.66014297)(325.30077687,30.26951736)(324.07421875,31.48826763)
\curveto(322.85546682,32.70701492)(322.24609243,34.35545077)(322.24609375,36.43358013)
\curveto(322.24609243,38.58200904)(322.82421685,40.28513234)(323.98046875,41.54295513)
\curveto(325.14452703,42.80856732)(326.71093171,43.44137919)(328.6796875,43.44139263)
\curveto(330.44530298,43.44137919)(331.83983283,42.87106726)(332.86328125,41.73045513)
\curveto(333.89451828,40.59763203)(334.41014276,39.05466482)(334.41015625,37.10154888)
\moveto(332.25390625,37.73436138)
\curveto(332.23826993,38.91403996)(331.90623902,39.85544527)(331.2578125,40.55858013)
\curveto(330.6171778,41.26169387)(329.76561616,41.61325601)(328.703125,41.61326763)
\curveto(327.49999342,41.61325601)(326.53515064,41.2734126)(325.80859375,40.59373638)
\curveto(325.08983958,39.91403896)(324.6757775,38.95700867)(324.56640625,37.72264263)
\lineto(332.25390625,37.73436138)
}
}
{
\newrgbcolor{curcolor}{0 0 0}
\pscustom[linestyle=none,fillstyle=solid,fillcolor=curcolor]
{
\newpath
\moveto(345.5546875,41.10936138)
\curveto(345.31249037,41.24997513)(345.04686564,41.35153753)(344.7578125,41.41404888)
\curveto(344.47655371,41.48434989)(344.16405402,41.51950611)(343.8203125,41.51951763)
\curveto(342.60155559,41.51950611)(341.66405652,41.12106901)(341.0078125,40.32420513)
\curveto(340.35937033,39.53513309)(340.0351519,38.39841548)(340.03515625,36.91404888)
\lineto(340.03515625,29.99998638)
\lineto(337.8671875,29.99998638)
\lineto(337.8671875,43.12498638)
\lineto(340.03515625,43.12498638)
\lineto(340.03515625,41.08592388)
\curveto(340.48827645,41.88278699)(341.07811961,42.47263015)(341.8046875,42.85545513)
\curveto(342.53124316,43.24606688)(343.41405477,43.44137919)(344.453125,43.44139263)
\curveto(344.60155359,43.44137919)(344.76561592,43.42966045)(344.9453125,43.40623638)
\curveto(345.12499056,43.39059799)(345.32420911,43.36325426)(345.54296875,43.32420513)
\lineto(345.5546875,41.10936138)
}
}
{
\newrgbcolor{curcolor}{0 0 0}
\pscustom[linestyle=none,fillstyle=solid,fillcolor=curcolor]
{
}
}
{
\newrgbcolor{curcolor}{0 0 0}
\pscustom[linestyle=none,fillstyle=solid,fillcolor=curcolor]
{
\newpath
\moveto(357.56640625,31.96873638)
\lineto(357.56640625,25.00779888)
\lineto(355.3984375,25.00779888)
\lineto(355.3984375,43.12498638)
\lineto(357.56640625,43.12498638)
\lineto(357.56640625,41.13279888)
\curveto(358.01952645,41.91403696)(358.58983838,42.49216138)(359.27734375,42.86717388)
\curveto(359.9726495,43.24997313)(360.80077367,43.44137919)(361.76171875,43.44139263)
\curveto(363.35545861,43.44137919)(364.64842607,42.80856732)(365.640625,41.54295513)
\curveto(366.64061158,40.27731985)(367.14061108,38.61325901)(367.140625,36.55076763)
\curveto(367.14061108,34.48826314)(366.64061158,32.8242023)(365.640625,31.55858013)
\curveto(364.64842607,30.29295483)(363.35545861,29.66014297)(361.76171875,29.66014263)
\curveto(360.80077367,29.66014297)(359.9726495,29.84764278)(359.27734375,30.22264263)
\curveto(358.58983838,30.60545452)(358.01952645,31.18748519)(357.56640625,31.96873638)
\moveto(364.90234375,36.55076763)
\curveto(364.90233207,38.13669699)(364.57420739,39.37888325)(363.91796875,40.27733013)
\curveto(363.2695212,41.18356894)(362.37499084,41.63669349)(361.234375,41.63670513)
\curveto(360.09374313,41.63669349)(359.19530652,41.18356894)(358.5390625,40.27733013)
\curveto(357.89062033,39.37888325)(357.5664019,38.13669699)(357.56640625,36.55076763)
\curveto(357.5664019,34.96482516)(357.89062033,33.71873266)(358.5390625,32.81248638)
\curveto(359.19530652,31.91404696)(360.09374313,31.46482866)(361.234375,31.46483013)
\curveto(362.37499084,31.46482866)(363.2695212,31.91404696)(363.91796875,32.81248638)
\curveto(364.57420739,33.71873266)(364.90233207,34.96482516)(364.90234375,36.55076763)
}
}
{
\newrgbcolor{curcolor}{0 0 0}
\pscustom[linestyle=none,fillstyle=solid,fillcolor=curcolor]
{
\newpath
\moveto(376.6796875,36.59764263)
\curveto(374.93749352,36.59763603)(373.73046347,36.39841748)(373.05859375,35.99998638)
\curveto(372.38671482,35.60154328)(372.05077765,34.92185646)(372.05078125,33.96092388)
\curveto(372.05077765,33.19529568)(372.3007774,32.58592129)(372.80078125,32.13279888)
\curveto(373.30858889,31.68748469)(373.99608821,31.46482866)(374.86328125,31.46483013)
\curveto(376.05858614,31.46482866)(377.01561644,31.88670324)(377.734375,32.73045513)
\curveto(378.46092749,33.58201404)(378.82420838,34.71091917)(378.82421875,36.11717388)
\lineto(378.82421875,36.59764263)
\lineto(376.6796875,36.59764263)
\moveto(380.98046875,37.48826763)
\lineto(380.98046875,29.99998638)
\lineto(378.82421875,29.99998638)
\lineto(378.82421875,31.99217388)
\curveto(378.33202137,31.19529768)(377.71874073,30.60545452)(376.984375,30.22264263)
\curveto(376.2499922,29.84764278)(375.3515556,29.66014297)(374.2890625,29.66014263)
\curveto(372.94530801,29.66014297)(371.87499658,30.03514259)(371.078125,30.78514263)
\curveto(370.28906066,31.54295358)(369.89452981,32.55467132)(369.89453125,33.82029888)
\curveto(369.89452981,35.29685608)(370.38671682,36.41013622)(371.37109375,37.16014263)
\curveto(372.36327734,37.91013472)(373.83983836,38.28513434)(375.80078125,38.28514263)
\lineto(378.82421875,38.28514263)
\lineto(378.82421875,38.49608013)
\curveto(378.82420838,39.48825814)(378.49608371,40.25388237)(377.83984375,40.79295513)
\curveto(377.19139751,41.33981879)(376.27733593,41.61325601)(375.09765625,41.61326763)
\curveto(374.34765036,41.61325601)(373.61718234,41.52341235)(372.90625,41.34373638)
\curveto(372.19530876,41.16403771)(371.51171569,40.89450673)(370.85546875,40.53514263)
\lineto(370.85546875,42.52733013)
\curveto(371.64452806,42.83200479)(372.41015229,43.05856707)(373.15234375,43.20701763)
\curveto(373.89452581,43.36325426)(374.61718134,43.44137919)(375.3203125,43.44139263)
\curveto(377.21874123,43.44137919)(378.63670857,42.94919218)(379.57421875,41.96483013)
\curveto(380.51170669,40.98044415)(380.98045622,39.48825814)(380.98046875,37.48826763)
}
}
{
\newrgbcolor{curcolor}{0 0 0}
\pscustom[linestyle=none,fillstyle=solid,fillcolor=curcolor]
{
\newpath
\moveto(393.0390625,41.10936138)
\curveto(392.79686537,41.24997513)(392.53124064,41.35153753)(392.2421875,41.41404888)
\curveto(391.96092871,41.48434989)(391.64842902,41.51950611)(391.3046875,41.51951763)
\curveto(390.08593059,41.51950611)(389.14843152,41.12106901)(388.4921875,40.32420513)
\curveto(387.84374533,39.53513309)(387.5195269,38.39841548)(387.51953125,36.91404888)
\lineto(387.51953125,29.99998638)
\lineto(385.3515625,29.99998638)
\lineto(385.3515625,43.12498638)
\lineto(387.51953125,43.12498638)
\lineto(387.51953125,41.08592388)
\curveto(387.97265145,41.88278699)(388.56249461,42.47263015)(389.2890625,42.85545513)
\curveto(390.01561816,43.24606688)(390.89842977,43.44137919)(391.9375,43.44139263)
\curveto(392.08592859,43.44137919)(392.24999092,43.42966045)(392.4296875,43.40623638)
\curveto(392.60936556,43.39059799)(392.80858411,43.36325426)(393.02734375,43.32420513)
\lineto(393.0390625,41.10936138)
}
}
{
\newrgbcolor{curcolor}{0 0 0}
\pscustom[linestyle=none,fillstyle=solid,fillcolor=curcolor]
{
\newpath
\moveto(397.45703125,46.85154888)
\lineto(397.45703125,43.12498638)
\lineto(401.8984375,43.12498638)
\lineto(401.8984375,41.44920513)
\lineto(397.45703125,41.44920513)
\lineto(397.45703125,34.32420513)
\curveto(397.45702686,33.25388937)(397.60155796,32.56639006)(397.890625,32.26170513)
\curveto(398.18749488,31.95701567)(398.78515053,31.80467207)(399.68359375,31.80467388)
\lineto(401.8984375,31.80467388)
\lineto(401.8984375,29.99998638)
\lineto(399.68359375,29.99998638)
\curveto(398.01952629,29.99998638)(396.87108994,30.30857982)(396.23828125,30.92576763)
\curveto(395.60546621,31.55076608)(395.28906027,32.68357744)(395.2890625,34.32420513)
\lineto(395.2890625,41.44920513)
\lineto(393.70703125,41.44920513)
\lineto(393.70703125,43.12498638)
\lineto(395.2890625,43.12498638)
\lineto(395.2890625,46.85154888)
\lineto(397.45703125,46.85154888)
}
}
{
\newrgbcolor{curcolor}{0 0 0}
\pscustom[linestyle=none,fillstyle=solid,fillcolor=curcolor]
{
\newpath
\moveto(404.74609375,43.12498638)
\lineto(406.90234375,43.12498638)
\lineto(406.90234375,29.99998638)
\lineto(404.74609375,29.99998638)
\lineto(404.74609375,43.12498638)
\moveto(404.74609375,48.23436138)
\lineto(406.90234375,48.23436138)
\lineto(406.90234375,45.50389263)
\lineto(404.74609375,45.50389263)
\lineto(404.74609375,48.23436138)
}
}
{
\newrgbcolor{curcolor}{0 0 0}
\pscustom[linestyle=none,fillstyle=solid,fillcolor=curcolor]
{
\newpath
\moveto(422.62890625,37.10154888)
\lineto(422.62890625,36.04686138)
\lineto(412.71484375,36.04686138)
\curveto(412.80859008,34.56248181)(413.25390214,33.42967045)(414.05078125,32.64842388)
\curveto(414.85546304,31.8749845)(415.97264942,31.48826614)(417.40234375,31.48826763)
\curveto(418.23045966,31.48826614)(419.03124011,31.58982854)(419.8046875,31.79295513)
\curveto(420.58592605,31.99607813)(421.35936278,32.30076533)(422.125,32.70701763)
\lineto(422.125,30.66795513)
\curveto(421.35155029,30.33982979)(420.55858233,30.08983004)(419.74609375,29.91795513)
\curveto(418.93358396,29.74608038)(418.10936603,29.66014297)(417.2734375,29.66014263)
\curveto(415.17968146,29.66014297)(413.51952687,30.26951736)(412.29296875,31.48826763)
\curveto(411.07421682,32.70701492)(410.46484243,34.35545077)(410.46484375,36.43358013)
\curveto(410.46484243,38.58200904)(411.04296685,40.28513234)(412.19921875,41.54295513)
\curveto(413.36327703,42.80856732)(414.92968171,43.44137919)(416.8984375,43.44139263)
\curveto(418.66405298,43.44137919)(420.05858283,42.87106726)(421.08203125,41.73045513)
\curveto(422.11326828,40.59763203)(422.62889276,39.05466482)(422.62890625,37.10154888)
\moveto(420.47265625,37.73436138)
\curveto(420.45701993,38.91403996)(420.12498902,39.85544527)(419.4765625,40.55858013)
\curveto(418.8359278,41.26169387)(417.98436616,41.61325601)(416.921875,41.61326763)
\curveto(415.71874342,41.61325601)(414.75390064,41.2734126)(414.02734375,40.59373638)
\curveto(413.30858958,39.91403896)(412.8945275,38.95700867)(412.78515625,37.72264263)
\lineto(420.47265625,37.73436138)
}
}
{
\newrgbcolor{curcolor}{1 1 1}
\pscustom[linestyle=none,fillstyle=solid,fillcolor=curcolor]
{
\newpath
\moveto(120.1649704,449.97605487)
\lineto(440.11043549,449.97605487)
\lineto(440.11043549,69.95679828)
\lineto(120.1649704,69.95679828)
\closepath
}
}
{
\newrgbcolor{curcolor}{0 0 0}
\pscustom[linewidth=2,linecolor=curcolor]
{
\newpath
\moveto(120.1649704,449.97605487)
\lineto(440.11043549,449.97605487)
\lineto(440.11043549,69.95679828)
\lineto(120.1649704,69.95679828)
\closepath
}
}
{
\newrgbcolor{curcolor}{0 0 0}
\pscustom[linestyle=none,fillstyle=solid,fillcolor=curcolor,opacity=0.11612902]
{
\newpath
\moveto(134.33292294,489.64407131)
\lineto(355.89206791,489.64407131)
\curveto(363.9166436,489.64407131)(370.37686157,483.18385333)(370.37686157,475.15927764)
\lineto(370.37686157,474.18155357)
\curveto(370.37686157,466.15697788)(363.9166436,459.69675991)(355.89206791,459.69675991)
\lineto(134.33292294,459.69675991)
\curveto(126.30834725,459.69675991)(119.84812927,466.15697788)(119.84812927,474.18155357)
\lineto(119.84812927,475.15927764)
\curveto(119.84812927,483.18385333)(126.30834725,489.64407131)(134.33292294,489.64407131)
\closepath
}
}
{
\newrgbcolor{curcolor}{0 0 0}
\pscustom[linewidth=2,linecolor=curcolor]
{
\newpath
\moveto(134.33292294,489.64407131)
\lineto(355.89206791,489.64407131)
\curveto(363.9166436,489.64407131)(370.37686157,483.18385333)(370.37686157,475.15927764)
\lineto(370.37686157,474.18155357)
\curveto(370.37686157,466.15697788)(363.9166436,459.69675991)(355.89206791,459.69675991)
\lineto(134.33292294,459.69675991)
\curveto(126.30834725,459.69675991)(119.84812927,466.15697788)(119.84812927,474.18155357)
\lineto(119.84812927,475.15927764)
\curveto(119.84812927,483.18385333)(126.30834725,489.64407131)(134.33292294,489.64407131)
\closepath
}
}
{
\newrgbcolor{curcolor}{1 1 1}
\pscustom[linestyle=none,fillstyle=solid,fillcolor=curcolor]
{
\newpath
\moveto(135.06185913,49.95851299)
\lineto(214.99641418,49.95851299)
\curveto(223.31771634,49.95851299)(230.01681519,43.25941414)(230.01681519,34.93811199)
\curveto(230.01681519,26.61680983)(223.31771634,19.91771099)(214.99641418,19.91771099)
\lineto(135.06185913,19.91771099)
\curveto(126.74055698,19.91771099)(120.04145813,26.61680983)(120.04145813,34.93811199)
\curveto(120.04145813,43.25941414)(126.74055698,49.95851299)(135.06185913,49.95851299)
\closepath
}
}
{
\newrgbcolor{curcolor}{0 0 0}
\pscustom[linewidth=2,linecolor=curcolor]
{
\newpath
\moveto(135.06185913,49.95851299)
\lineto(214.99641418,49.95851299)
\curveto(223.31771634,49.95851299)(230.01681519,43.25941414)(230.01681519,34.93811199)
\curveto(230.01681519,26.61680983)(223.31771634,19.91771099)(214.99641418,19.91771099)
\lineto(135.06185913,19.91771099)
\curveto(126.74055698,19.91771099)(120.04145813,26.61680983)(120.04145813,34.93811199)
\curveto(120.04145813,43.25941414)(126.74055698,49.95851299)(135.06185913,49.95851299)
\closepath
}
}
{
\newrgbcolor{curcolor}{0 0 0}
\pscustom[linestyle=none,fillstyle=solid,fillcolor=curcolor]
{
\newpath
\moveto(150.65234375,38.20311138)
\curveto(151.16014509,38.03122835)(151.6523321,37.66404121)(152.12890625,37.10154888)
\curveto(152.61326864,36.53904234)(153.09764315,35.76560561)(153.58203125,34.78123638)
\lineto(155.984375,29.99998638)
\lineto(153.44140625,29.99998638)
\lineto(151.203125,34.48826763)
\curveto(150.62498938,35.66013697)(150.06248994,36.43747994)(149.515625,36.82029888)
\curveto(148.97655352,37.20310417)(148.23827301,37.39451023)(147.30078125,37.39451763)
\lineto(144.72265625,37.39451763)
\lineto(144.72265625,29.99998638)
\lineto(142.35546875,29.99998638)
\lineto(142.35546875,47.49608013)
\lineto(147.69921875,47.49608013)
\curveto(149.69920905,47.49606263)(151.19139506,47.0780943)(152.17578125,46.24217388)
\curveto(153.16014309,45.40622097)(153.6523301,44.14450348)(153.65234375,42.45701763)
\curveto(153.6523301,41.35544377)(153.39451786,40.44138219)(152.87890625,39.71483013)
\curveto(152.37108138,38.98825864)(151.62889462,38.48435289)(150.65234375,38.20311138)
\moveto(144.72265625,45.55076763)
\lineto(144.72265625,39.33983013)
\lineto(147.69921875,39.33983013)
\curveto(148.83983491,39.33982079)(149.69920905,39.60153928)(150.27734375,40.12498638)
\curveto(150.86327039,40.65622572)(151.15623884,41.43356869)(151.15625,42.45701763)
\curveto(151.15623884,43.48044165)(150.86327039,44.24997213)(150.27734375,44.76561138)
\curveto(149.69920905,45.28903359)(148.83983491,45.55075208)(147.69921875,45.55076763)
\lineto(144.72265625,45.55076763)
}
}
{
\newrgbcolor{curcolor}{0 0 0}
\pscustom[linestyle=none,fillstyle=solid,fillcolor=curcolor]
{
\newpath
\moveto(169.09765625,37.10154888)
\lineto(169.09765625,36.04686138)
\lineto(159.18359375,36.04686138)
\curveto(159.27734008,34.56248181)(159.72265214,33.42967045)(160.51953125,32.64842388)
\curveto(161.32421304,31.8749845)(162.44139942,31.48826614)(163.87109375,31.48826763)
\curveto(164.69920966,31.48826614)(165.49999011,31.58982854)(166.2734375,31.79295513)
\curveto(167.05467605,31.99607813)(167.82811278,32.30076533)(168.59375,32.70701763)
\lineto(168.59375,30.66795513)
\curveto(167.82030029,30.33982979)(167.02733233,30.08983004)(166.21484375,29.91795513)
\curveto(165.40233396,29.74608038)(164.57811603,29.66014297)(163.7421875,29.66014263)
\curveto(161.64843146,29.66014297)(159.98827687,30.26951736)(158.76171875,31.48826763)
\curveto(157.54296682,32.70701492)(156.93359243,34.35545077)(156.93359375,36.43358013)
\curveto(156.93359243,38.58200904)(157.51171685,40.28513234)(158.66796875,41.54295513)
\curveto(159.83202703,42.80856732)(161.39843171,43.44137919)(163.3671875,43.44139263)
\curveto(165.13280298,43.44137919)(166.52733283,42.87106726)(167.55078125,41.73045513)
\curveto(168.58201828,40.59763203)(169.09764276,39.05466482)(169.09765625,37.10154888)
\moveto(166.94140625,37.73436138)
\curveto(166.92576993,38.91403996)(166.59373902,39.85544527)(165.9453125,40.55858013)
\curveto(165.3046778,41.26169387)(164.45311616,41.61325601)(163.390625,41.61326763)
\curveto(162.18749342,41.61325601)(161.22265064,41.2734126)(160.49609375,40.59373638)
\curveto(159.77733958,39.91403896)(159.3632775,38.95700867)(159.25390625,37.72264263)
\lineto(166.94140625,37.73436138)
}
}
{
\newrgbcolor{curcolor}{0 0 0}
\pscustom[linestyle=none,fillstyle=solid,fillcolor=curcolor]
{
\newpath
\moveto(174.76953125,46.85154888)
\lineto(174.76953125,43.12498638)
\lineto(179.2109375,43.12498638)
\lineto(179.2109375,41.44920513)
\lineto(174.76953125,41.44920513)
\lineto(174.76953125,34.32420513)
\curveto(174.76952686,33.25388937)(174.91405796,32.56639006)(175.203125,32.26170513)
\curveto(175.49999487,31.95701567)(176.09765053,31.80467207)(176.99609375,31.80467388)
\lineto(179.2109375,31.80467388)
\lineto(179.2109375,29.99998638)
\lineto(176.99609375,29.99998638)
\curveto(175.33202629,29.99998638)(174.18358994,30.30857982)(173.55078125,30.92576763)
\curveto(172.91796621,31.55076608)(172.60156027,32.68357744)(172.6015625,34.32420513)
\lineto(172.6015625,41.44920513)
\lineto(171.01953125,41.44920513)
\lineto(171.01953125,43.12498638)
\lineto(172.6015625,43.12498638)
\lineto(172.6015625,46.85154888)
\lineto(174.76953125,46.85154888)
}
}
{
\newrgbcolor{curcolor}{0 0 0}
\pscustom[linestyle=none,fillstyle=solid,fillcolor=curcolor]
{
\newpath
\moveto(187.14453125,41.61326763)
\curveto(185.98827506,41.61325601)(185.07421347,41.16013147)(184.40234375,40.25389263)
\curveto(183.73046482,39.35544577)(183.39452765,38.12107201)(183.39453125,36.55076763)
\curveto(183.39452765,34.98045015)(183.72655857,33.74217013)(184.390625,32.83592388)
\curveto(185.06249473,31.93748444)(185.98046257,31.48826614)(187.14453125,31.48826763)
\curveto(188.29296025,31.48826614)(189.20311559,31.94139069)(189.875,32.84764263)
\curveto(190.54686425,33.75388887)(190.88280141,34.98826264)(190.8828125,36.55076763)
\curveto(190.88280141,38.10544702)(190.54686425,39.33591454)(189.875,40.24217388)
\curveto(189.20311559,41.15622522)(188.29296025,41.61325601)(187.14453125,41.61326763)
\moveto(187.14453125,43.44139263)
\curveto(189.01952203,43.44137919)(190.4921768,42.83200479)(191.5625,41.61326763)
\curveto(192.63279966,40.39450723)(193.16795538,38.70700892)(193.16796875,36.55076763)
\curveto(193.16795538,34.40232572)(192.63279966,32.71482741)(191.5625,31.48826763)
\curveto(190.4921768,30.26951736)(189.01952203,29.66014297)(187.14453125,29.66014263)
\curveto(185.26171329,29.66014297)(183.78515226,30.26951736)(182.71484375,31.48826763)
\curveto(181.65234189,32.71482741)(181.12109243,34.40232572)(181.12109375,36.55076763)
\curveto(181.12109243,38.70700892)(181.65234189,40.39450723)(182.71484375,41.61326763)
\curveto(183.78515226,42.83200479)(185.26171329,43.44137919)(187.14453125,43.44139263)
}
}
{
\newrgbcolor{curcolor}{0 0 0}
\pscustom[linestyle=none,fillstyle=solid,fillcolor=curcolor]
{
\newpath
\moveto(196.5078125,35.17967388)
\lineto(196.5078125,43.12498638)
\lineto(198.6640625,43.12498638)
\lineto(198.6640625,35.26170513)
\curveto(198.6640583,34.01951361)(198.90624556,33.08592079)(199.390625,32.46092388)
\curveto(199.87499459,31.84373453)(200.60155637,31.53514109)(201.5703125,31.53514263)
\curveto(202.73436673,31.53514109)(203.65233457,31.90623447)(204.32421875,32.64842388)
\curveto(205.00389571,33.39060799)(205.34373913,34.40232572)(205.34375,35.68358013)
\lineto(205.34375,43.12498638)
\lineto(207.5,43.12498638)
\lineto(207.5,29.99998638)
\lineto(205.34375,29.99998638)
\lineto(205.34375,32.01561138)
\curveto(204.82030215,31.21873516)(204.21092776,30.62498575)(203.515625,30.23436138)
\curveto(202.82811664,29.85154903)(202.02733619,29.66014297)(201.11328125,29.66014263)
\curveto(199.60546361,29.66014297)(198.46093351,30.1288925)(197.6796875,31.06639263)
\curveto(196.89843507,32.00389062)(196.50781046,33.374983)(196.5078125,35.17967388)
\moveto(201.93359375,43.44139263)
\lineto(201.93359375,43.44139263)
}
}
{
\newrgbcolor{curcolor}{0 0 0}
\pscustom[linestyle=none,fillstyle=solid,fillcolor=curcolor]
{
\newpath
\moveto(219.5703125,41.10936138)
\curveto(219.32811538,41.24997513)(219.06249064,41.35153753)(218.7734375,41.41404888)
\curveto(218.49217871,41.48434989)(218.17967902,41.51950611)(217.8359375,41.51951763)
\curveto(216.61718059,41.51950611)(215.67968152,41.12106901)(215.0234375,40.32420513)
\curveto(214.37499533,39.53513309)(214.0507769,38.39841548)(214.05078125,36.91404888)
\lineto(214.05078125,29.99998638)
\lineto(211.8828125,29.99998638)
\lineto(211.8828125,43.12498638)
\lineto(214.05078125,43.12498638)
\lineto(214.05078125,41.08592388)
\curveto(214.50390145,41.88278699)(215.09374461,42.47263015)(215.8203125,42.85545513)
\curveto(216.54686816,43.24606688)(217.42967977,43.44137919)(218.46875,43.44139263)
\curveto(218.61717859,43.44137919)(218.78124092,43.42966045)(218.9609375,43.40623638)
\curveto(219.14061556,43.39059799)(219.33983411,43.36325426)(219.55859375,43.32420513)
\lineto(219.5703125,41.10936138)
}
}
{
\newrgbcolor{curcolor}{1 1 1}
\pscustom[linestyle=none,fillstyle=solid,fillcolor=curcolor]
{
\newpath
\moveto(405.19685268,490.20115062)
\lineto(425.1650629,490.20115062)
\curveto(433.67793478,490.20115062)(440.53125763,483.34782778)(440.53125763,474.8349559)
\curveto(440.53125763,466.32208402)(433.67793478,459.46876117)(425.1650629,459.46876117)
\lineto(405.19685268,459.46876117)
\curveto(396.68398081,459.46876117)(389.83065796,466.32208402)(389.83065796,474.8349559)
\curveto(389.83065796,483.34782778)(396.68398081,490.20115062)(405.19685268,490.20115062)
\closepath
}
}
{
\newrgbcolor{curcolor}{0 0 0}
\pscustom[linewidth=2,linecolor=curcolor]
{
\newpath
\moveto(405.19685268,490.20115062)
\lineto(425.1650629,490.20115062)
\curveto(433.67793478,490.20115062)(440.53125763,483.34782778)(440.53125763,474.8349559)
\curveto(440.53125763,466.32208402)(433.67793478,459.46876117)(425.1650629,459.46876117)
\lineto(405.19685268,459.46876117)
\curveto(396.68398081,459.46876117)(389.83065796,466.32208402)(389.83065796,474.8349559)
\curveto(389.83065796,483.34782778)(396.68398081,490.20115062)(405.19685268,490.20115062)
\closepath
}
}
{
\newrgbcolor{curcolor}{0 0 0}
\pscustom[linewidth=2,linecolor=curcolor,linestyle=dashed,dash=8 8]
{
\newpath
\moveto(140,540)
\lineto(60,540)
}
}
{
\newrgbcolor{curcolor}{0 0 0}
\pscustom[linestyle=none,fillstyle=solid,fillcolor=curcolor]
{
\newpath
\moveto(129.53769464,544.84048224)
\lineto(142.6487474,540.01921591)
\lineto(129.53769392,535.19795064)
\curveto(131.632292,538.04442372)(131.62022288,541.93889292)(129.53769464,544.84048224)
\lineto(129.53769464,544.84048224)
\lineto(129.53769464,544.84048224)
\closepath
}
}
{
\newrgbcolor{curcolor}{0 0 0}
\pscustom[linewidth=2,linecolor=curcolor,linestyle=dashed,dash=8 8]
{
\newpath
\moveto(150.14286,480)
\lineto(60.142857,480)
}
}
{
\newrgbcolor{curcolor}{0 0 0}
\pscustom[linestyle=none,fillstyle=solid,fillcolor=curcolor]
{
\newpath
\moveto(139.68055464,484.84048224)
\lineto(152.7916074,480.01921591)
\lineto(139.68055392,475.19795064)
\curveto(141.775152,478.04442372)(141.76308288,481.93889292)(139.68055464,484.84048224)
\lineto(139.68055464,484.84048224)
\lineto(139.68055464,484.84048224)
\closepath
}
}
{
\newrgbcolor{curcolor}{0 0 0}
\pscustom[linewidth=2,linecolor=curcolor,linestyle=dashed,dash=8 8]
{
\newpath
\moveto(420,480)
\lineto(500,480)
}
}
{
\newrgbcolor{curcolor}{0 0 0}
\pscustom[linestyle=none,fillstyle=solid,fillcolor=curcolor]
{
\newpath
\moveto(430.46230536,475.15951776)
\lineto(417.3512526,479.98078409)
\lineto(430.46230608,484.80204936)
\curveto(428.367708,481.95557628)(428.37977712,478.06110708)(430.46230536,475.15951776)
\lineto(430.46230536,475.15951776)
\lineto(430.46230536,475.15951776)
\closepath
}
}
{
\newrgbcolor{curcolor}{0 0 0}
\pscustom[linewidth=2,linecolor=curcolor,linestyle=dashed,dash=8 8]
{
\newpath
\moveto(150,320)
\lineto(60,320)
}
}
{
\newrgbcolor{curcolor}{0 0 0}
\pscustom[linestyle=none,fillstyle=solid,fillcolor=curcolor]
{
\newpath
\moveto(139.53769464,324.84048224)
\lineto(152.6487474,320.01921591)
\lineto(139.53769392,315.19795064)
\curveto(141.632292,318.04442372)(141.62022288,321.93889292)(139.53769464,324.84048224)
\lineto(139.53769464,324.84048224)
\lineto(139.53769464,324.84048224)
\closepath
}
}
{
\newrgbcolor{curcolor}{0 0 0}
\pscustom[linewidth=2,linecolor=curcolor,linestyle=dashed,dash=8 8]
{
\newpath
\moveto(420,40)
\lineto(500,40)
}
}
{
\newrgbcolor{curcolor}{0 0 0}
\pscustom[linestyle=none,fillstyle=solid,fillcolor=curcolor]
{
\newpath
\moveto(430.46230536,35.15951776)
\lineto(417.3512526,39.98078409)
\lineto(430.46230608,44.80204936)
\curveto(428.367708,41.95557628)(428.37977712,38.06110708)(430.46230536,35.15951776)
\lineto(430.46230536,35.15951776)
\lineto(430.46230536,35.15951776)
\closepath
}
}
{
\newrgbcolor{curcolor}{0 0 0}
\pscustom[linewidth=2,linecolor=curcolor,linestyle=dashed,dash=8 8]
{
\newpath
\moveto(140,40)
\lineto(60,40)
}
}
{
\newrgbcolor{curcolor}{0 0 0}
\pscustom[linestyle=none,fillstyle=solid,fillcolor=curcolor]
{
\newpath
\moveto(129.53769464,44.84048224)
\lineto(142.6487474,40.01921591)
\lineto(129.53769392,35.19795064)
\curveto(131.632292,38.04442372)(131.62022288,41.93889292)(129.53769464,44.84048224)
\lineto(129.53769464,44.84048224)
\lineto(129.53769464,44.84048224)
\closepath
}
}
{
\newrgbcolor{curcolor}{0 0 0}
\pscustom[linestyle=none,fillstyle=solid,fillcolor=curcolor]
{
\newpath
\moveto(43.96875,532.65625164)
\lineto(49.125,532.65625164)
\lineto(49.125,550.45312664)
\lineto(43.515625,549.32812664)
\lineto(43.515625,552.20312664)
\lineto(49.09375,553.32812664)
\lineto(52.25,553.32812664)
\lineto(52.25,532.65625164)
\lineto(57.40625,532.65625164)
\lineto(57.40625,530.00000164)
\lineto(43.96875,530.00000164)
\lineto(43.96875,532.65625164)
}
}
{
\newrgbcolor{curcolor}{0 0 0}
\pscustom[linestyle=none,fillstyle=solid,fillcolor=curcolor]
{
\newpath
\moveto(46.140625,472.65624973)
\lineto(57.15625,472.65624973)
\lineto(57.15625,469.99999973)
\lineto(42.34375,469.99999973)
\lineto(42.34375,472.65624973)
\curveto(43.54166312,473.89582917)(45.17186983,475.55728584)(47.234375,477.64062473)
\curveto(49.30728236,479.73436499)(50.60936439,481.08332198)(51.140625,481.68749973)
\curveto(52.15102952,482.82290357)(52.85415381,483.78123595)(53.25,484.56249973)
\curveto(53.65623634,485.35415104)(53.85936114,486.13019193)(53.859375,486.89062473)
\curveto(53.85936114,488.13018993)(53.42186158,489.14060559)(52.546875,489.92187473)
\curveto(51.68227998,490.70310403)(50.55207278,491.09372863)(49.15625,491.09374973)
\curveto(48.1666585,491.09372863)(47.11978455,490.92185381)(46.015625,490.57812473)
\curveto(44.92187008,490.23435449)(43.74999625,489.71352168)(42.5,489.01562473)
\lineto(42.5,492.20312473)
\curveto(43.77082956,492.71351868)(44.95832838,493.09893496)(46.0625,493.35937473)
\curveto(47.1666595,493.61976778)(48.17707516,493.74997598)(49.09375,493.74999973)
\curveto(51.51040516,493.74997598)(53.43748656,493.14580992)(54.875,491.93749973)
\curveto(56.31248369,490.72914567)(57.03123297,489.11456395)(57.03125,487.09374973)
\curveto(57.03123297,486.13540026)(56.84894148,485.22394284)(56.484375,484.35937473)
\curveto(56.1301922,483.50519456)(55.47915119,482.4947789)(54.53125,481.32812473)
\curveto(54.27081906,481.02603037)(53.44269489,480.15103124)(52.046875,478.70312473)
\curveto(50.65103102,477.26561746)(48.68228298,475.24999448)(46.140625,472.65624973)
}
}
{
\newrgbcolor{curcolor}{0 0 0}
\pscustom[linestyle=none,fillstyle=solid,fillcolor=curcolor]
{
\newpath
\moveto(512.984375,482.57812473)
\curveto(514.49477717,482.25519581)(515.67185933,481.58332148)(516.515625,480.56249973)
\curveto(517.3697743,479.54165685)(517.7968572,478.28124145)(517.796875,476.78124973)
\curveto(517.7968572,474.47916192)(517.00519133,472.6979137)(515.421875,471.43749973)
\curveto(513.83852783,470.17708288)(511.58853008,469.54687518)(508.671875,469.54687473)
\curveto(507.69270064,469.54687518)(506.68228498,469.64583342)(505.640625,469.84374973)
\curveto(504.60937039,470.0312497)(503.54166313,470.31770774)(502.4375,470.70312473)
\lineto(502.4375,473.74999973)
\curveto(503.31249669,473.23957982)(504.27082906,472.85416354)(505.3125,472.59374973)
\curveto(506.35416031,472.33333073)(507.44270089,472.20312253)(508.578125,472.20312473)
\curveto(510.55728111,472.20312253)(512.06248794,472.59374713)(513.09375,473.37499973)
\curveto(514.13540253,474.15624557)(514.65623534,475.2916611)(514.65625,476.78124973)
\curveto(514.65623534,478.15624157)(514.17186083,479.22915717)(513.203125,479.99999973)
\curveto(512.24477942,480.78123895)(510.90623909,481.17186356)(509.1875,481.17187473)
\lineto(506.46875,481.17187473)
\lineto(506.46875,483.76562473)
\lineto(509.3125,483.76562473)
\curveto(510.86457247,483.76561096)(512.05207128,484.07290232)(512.875,484.68749973)
\curveto(513.69790297,485.31248442)(514.10936089,486.20831685)(514.109375,487.37499973)
\curveto(514.10936089,488.57289782)(513.68227798,489.48956357)(512.828125,490.12499973)
\curveto(511.98436302,490.77081229)(510.77082256,491.09372863)(509.1875,491.09374973)
\curveto(508.32290834,491.09372863)(507.39582594,490.99997873)(506.40625,490.81249973)
\curveto(505.41666125,490.6249791)(504.32812067,490.33331273)(503.140625,489.93749973)
\lineto(503.140625,492.74999973)
\curveto(504.33853733,493.08330998)(505.45832788,493.33330973)(506.5,493.49999973)
\curveto(507.55207578,493.66664273)(508.54165813,493.74997598)(509.46875,493.74999973)
\curveto(511.86457147,493.74997598)(513.76040291,493.20310153)(515.15625,492.10937473)
\curveto(516.55206678,491.02602037)(517.24998275,489.55727184)(517.25,487.70312473)
\curveto(517.24998275,486.41144165)(516.88019145,485.31769274)(516.140625,484.42187473)
\curveto(515.40102627,483.53644453)(514.34894398,482.92186181)(512.984375,482.57812473)
}
}
{
\newrgbcolor{curcolor}{0 0 0}
\pscustom[linestyle=none,fillstyle=solid,fillcolor=curcolor]
{
\newpath
\moveto(52.09375,330.57812664)
\lineto(44.125,318.12500164)
\lineto(52.09375,318.12500164)
\lineto(52.09375,330.57812664)
\moveto(51.265625,333.32812664)
\lineto(55.234375,333.32812664)
\lineto(55.234375,318.12500164)
\lineto(58.5625,318.12500164)
\lineto(58.5625,315.50000164)
\lineto(55.234375,315.50000164)
\lineto(55.234375,310.00000164)
\lineto(52.09375,310.00000164)
\lineto(52.09375,315.50000164)
\lineto(41.5625,315.50000164)
\lineto(41.5625,318.54687664)
\lineto(51.265625,333.32812664)
}
}
{
\newrgbcolor{curcolor}{0 0 0}
\pscustom[linestyle=none,fillstyle=solid,fillcolor=curcolor]
{
\newpath
\moveto(43.453125,53.32811138)
\lineto(55.84375,53.32811138)
\lineto(55.84375,50.67186138)
\lineto(46.34375,50.67186138)
\lineto(46.34375,44.95311138)
\curveto(46.80207653,45.10934627)(47.26040941,45.22392949)(47.71875,45.29686138)
\curveto(48.17707516,45.38017933)(48.63540803,45.42184596)(49.09375,45.42186138)
\curveto(51.69790497,45.42184596)(53.76040291,44.708305)(55.28125,43.28123638)
\curveto(56.80206653,41.85414119)(57.56248244,39.92185146)(57.5625,37.48436138)
\curveto(57.56248244,34.97393974)(56.78123322,33.02081669)(55.21875,31.62498638)
\curveto(53.65623634,30.23956947)(51.45311355,29.54686183)(48.609375,29.54686138)
\curveto(47.6302007,29.54686183)(46.6302017,29.63019508)(45.609375,29.79686138)
\curveto(44.59895373,29.96352808)(43.55207978,30.21352783)(42.46875,30.54686138)
\lineto(42.46875,33.71873638)
\curveto(43.40624659,33.2083165)(44.37499562,32.82810855)(45.375,32.57811138)
\curveto(46.37499363,32.32810905)(47.43228423,32.20310917)(48.546875,32.20311138)
\curveto(50.34894798,32.20310917)(51.77602989,32.67706703)(52.828125,33.62498638)
\curveto(53.88019445,34.57289847)(54.40623559,35.85935552)(54.40625,37.48436138)
\curveto(54.40623559,39.10935227)(53.88019445,40.39580931)(52.828125,41.34373638)
\curveto(51.77602989,42.29164075)(50.34894798,42.76559861)(48.546875,42.76561138)
\curveto(47.7031173,42.76559861)(46.85936814,42.67184871)(46.015625,42.48436138)
\curveto(45.18228648,42.29684908)(44.32812067,42.00518271)(43.453125,41.60936138)
\lineto(43.453125,53.32811138)
}
}
{
\newrgbcolor{curcolor}{0 0 0}
\pscustom[linestyle=none,fillstyle=solid,fillcolor=curcolor]
{
\newpath
\moveto(510.5625,42.92186138)
\curveto(509.14582419,42.92184846)(508.02082531,42.43747394)(507.1875,41.46873638)
\curveto(506.36457697,40.49997588)(505.95311905,39.17185221)(505.953125,37.48436138)
\curveto(505.95311905,35.80727224)(506.36457697,34.47914856)(507.1875,33.49998638)
\curveto(508.02082531,32.53123385)(509.14582419,32.04685933)(510.5625,32.04686138)
\curveto(511.97915469,32.04685933)(513.09894523,32.53123385)(513.921875,33.49998638)
\curveto(514.75519358,34.47914856)(515.17185983,35.80727224)(515.171875,37.48436138)
\curveto(515.17185983,39.17185221)(514.75519358,40.49997588)(513.921875,41.46873638)
\curveto(513.09894523,42.43747394)(511.97915469,42.92184846)(510.5625,42.92186138)
\moveto(516.828125,52.81248638)
\lineto(516.828125,49.93748638)
\curveto(516.0364423,50.31246606)(515.23435977,50.59892411)(514.421875,50.79686138)
\curveto(513.61977805,50.99475705)(512.82290384,51.09371528)(512.03125,51.09373638)
\curveto(509.94790672,51.09371528)(508.35415831,50.39059099)(507.25,48.98436138)
\curveto(506.15624384,47.5780938)(505.53124447,45.45309592)(505.375,42.60936138)
\curveto(505.98957734,43.51559786)(506.76040991,44.2083055)(507.6875,44.68748638)
\curveto(508.61457472,45.17705453)(509.63540703,45.42184596)(510.75,45.42186138)
\curveto(513.09373691,45.42184596)(514.94269339,44.708305)(516.296875,43.28123638)
\curveto(517.66144067,41.86455785)(518.34373166,39.93226811)(518.34375,37.48436138)
\curveto(518.34373166,35.08852296)(517.63539903,33.16664988)(516.21875,31.71873638)
\curveto(514.80206853,30.27081944)(512.91665375,29.54686183)(510.5625,29.54686138)
\curveto(507.86457547,29.54686183)(505.80207753,30.5781108)(504.375,32.64061138)
\curveto(502.94791372,34.71352333)(502.23437277,37.71352033)(502.234375,41.64061138)
\curveto(502.23437277,45.32809605)(503.10937189,48.26559311)(504.859375,50.45311138)
\curveto(506.60936839,52.65100539)(508.95832438,53.74996263)(511.90625,53.74998638)
\curveto(512.69790397,53.74996263)(513.49477817,53.67183771)(514.296875,53.51561138)
\curveto(515.10935989,53.35933802)(515.95310905,53.12496325)(516.828125,52.81248638)
}
}
\end{pspicture}

		
		\begin{enumerate}
		  \item Fond d'écran
		  \item Champs de texte ``Filtrer''
		  \item Bouton ``Rafraichir''
		  \item Cadre contenant les différentes parties en ligne en cours
		  \item Bouton ``\hyperlink{Page d'accueil}{Retour}''
		  \item Bouton ``\hyperlink{Creer partie multi-joueurs}{Créer partie}''
		\end{enumerate}
	
\newpage

	\subsection{Créer partie multi-joueurs}
	
		\hypertarget{Creer partie multi-joueurs}{}
		\label{Creer partie multi-joueurs}
	
		\input{./img/6_Creer_partie_multi_joueurs}
		
		\begin{enumerate}
		  \item Fond d'écran
		  \item Champs de texte ``Nom de la partie''
		  \item Carte sélectionnée
		  \item Carte précédente
		  \item Carte suivante
		  \item Nom de la carte sélectionnée
		  \item Liste déroulante ``Type de partie''
		  \item Liste déroulante ``Nombre d'ennemis''
		  \item Liste déroulante ``Difficulté des ennemis''
		  \item Bouton ``\hyperlink{Page d'accueil}{Retour}''
		  \item Bouton ``Lancer'' 
		\end{enumerate}
	
	
\newpage

	\subsection{Options}
	
		\hypertarget{Options}{}
		\label{Options}
	
		%LaTeX with PSTricks extensions
%%Creator: inkscape 0.48.0
%%Please note this file requires PSTricks extensions
\psset{xunit=.5pt,yunit=.5pt,runit=.5pt}
\begin{pspicture}(560,600)
{
\newrgbcolor{curcolor}{1 1 1}
\pscustom[linestyle=none,fillstyle=solid,fillcolor=curcolor]
{
\newpath
\moveto(133.12401581,597.52220317)
\lineto(426.87598419,597.52220317)
\curveto(443.85397169,597.52220317)(457.52217102,583.85400385)(457.52217102,566.87601635)
\lineto(457.52217102,33.12401744)
\curveto(457.52217102,16.14602994)(443.85397169,2.47783062)(426.87598419,2.47783062)
\lineto(133.12401581,2.47783062)
\curveto(116.14602831,2.47783062)(102.47782898,16.14602994)(102.47782898,33.12401744)
\lineto(102.47782898,566.87601635)
\curveto(102.47782898,583.85400385)(116.14602831,597.52220317)(133.12401581,597.52220317)
\closepath
}
}
{
\newrgbcolor{curcolor}{0 0 0}
\pscustom[linewidth=4.95566034,linecolor=curcolor]
{
\newpath
\moveto(133.12401581,597.52220317)
\lineto(426.87598419,597.52220317)
\curveto(443.85397169,597.52220317)(457.52217102,583.85400385)(457.52217102,566.87601635)
\lineto(457.52217102,33.12401744)
\curveto(457.52217102,16.14602994)(443.85397169,2.47783062)(426.87598419,2.47783062)
\lineto(133.12401581,2.47783062)
\curveto(116.14602831,2.47783062)(102.47782898,16.14602994)(102.47782898,33.12401744)
\lineto(102.47782898,566.87601635)
\curveto(102.47782898,583.85400385)(116.14602831,597.52220317)(133.12401581,597.52220317)
\closepath
}
}
{
\newrgbcolor{curcolor}{0 0 0}
\pscustom[linestyle=none,fillstyle=solid,fillcolor=curcolor,opacity=0.11612902]
{
\newpath
\moveto(213.84012413,420.00000164)
\lineto(392.82668495,420.00000164)
\curveto(394.95411372,420.00000164)(396.66680908,418.19472938)(396.66680908,415.952306)
\curveto(396.66680908,413.70988262)(394.95411372,411.90461036)(392.82668495,411.90461036)
\lineto(213.84012413,411.90461036)
\curveto(211.71269536,411.90461036)(210,413.70988262)(210,415.952306)
\curveto(210,418.19472938)(211.71269536,420.00000164)(213.84012413,420.00000164)
\closepath
}
}
{
\newrgbcolor{curcolor}{0 0 0}
\pscustom[linewidth=1.67362297,linecolor=curcolor]
{
\newpath
\moveto(213.84012413,420.00000164)
\lineto(392.82668495,420.00000164)
\curveto(394.95411372,420.00000164)(396.66680908,418.19472938)(396.66680908,415.952306)
\curveto(396.66680908,413.70988262)(394.95411372,411.90461036)(392.82668495,411.90461036)
\lineto(213.84012413,411.90461036)
\curveto(211.71269536,411.90461036)(210,413.70988262)(210,415.952306)
\curveto(210,418.19472938)(211.71269536,420.00000164)(213.84012413,420.00000164)
\closepath
}
}
{
\newrgbcolor{curcolor}{1 1 1}
\pscustom[linestyle=none,fillstyle=solid,fillcolor=curcolor]
{
\newpath
\moveto(201.36239243,319.87677166)
\lineto(368.26269913,319.87677166)
\curveto(374.76516381,319.87677166)(380,314.64193547)(380,308.13947078)
\lineto(380,301.73730632)
\curveto(380,295.23484164)(374.76516381,290.00000545)(368.26269913,290.00000545)
\lineto(201.36239243,290.00000545)
\curveto(194.85992774,290.00000545)(189.62509155,295.23484164)(189.62509155,301.73730632)
\lineto(189.62509155,308.13947078)
\curveto(189.62509155,314.64193547)(194.85992774,319.87677166)(201.36239243,319.87677166)
\closepath
}
}
{
\newrgbcolor{curcolor}{0 0 0}
\pscustom[linewidth=1.86126256,linecolor=curcolor]
{
\newpath
\moveto(201.36239243,319.87677166)
\lineto(368.26269913,319.87677166)
\curveto(374.76516381,319.87677166)(380,314.64193547)(380,308.13947078)
\lineto(380,301.73730632)
\curveto(380,295.23484164)(374.76516381,290.00000545)(368.26269913,290.00000545)
\lineto(201.36239243,290.00000545)
\curveto(194.85992774,290.00000545)(189.62509155,295.23484164)(189.62509155,301.73730632)
\lineto(189.62509155,308.13947078)
\curveto(189.62509155,314.64193547)(194.85992774,319.87677166)(201.36239243,319.87677166)
\closepath
}
}
{
\newrgbcolor{curcolor}{1 1 1}
\pscustom[linestyle=none,fillstyle=solid,fillcolor=curcolor]
{
\newpath
\moveto(131.97757626,49.48121807)
\lineto(218.50890446,49.48121807)
\curveto(224.95737906,49.48121807)(230.14875031,44.28984682)(230.14875031,37.84137222)
\lineto(230.14875031,31.37479087)
\curveto(230.14875031,24.92631627)(224.95737906,19.73494503)(218.50890446,19.73494503)
\lineto(131.97757626,19.73494503)
\curveto(125.52910166,19.73494503)(120.33773041,24.92631627)(120.33773041,31.37479087)
\lineto(120.33773041,37.84137222)
\curveto(120.33773041,44.28984682)(125.52910166,49.48121807)(131.97757626,49.48121807)
\closepath
}
}
{
\newrgbcolor{curcolor}{0 0 0}
\pscustom[linewidth=2.04669738,linecolor=curcolor]
{
\newpath
\moveto(131.97757626,49.48121807)
\lineto(218.50890446,49.48121807)
\curveto(224.95737906,49.48121807)(230.14875031,44.28984682)(230.14875031,37.84137222)
\lineto(230.14875031,31.37479087)
\curveto(230.14875031,24.92631627)(224.95737906,19.73494503)(218.50890446,19.73494503)
\lineto(131.97757626,19.73494503)
\curveto(125.52910166,19.73494503)(120.33773041,24.92631627)(120.33773041,31.37479087)
\lineto(120.33773041,37.84137222)
\curveto(120.33773041,44.28984682)(125.52910166,49.48121807)(131.97757626,49.48121807)
\closepath
}
}
{
\newrgbcolor{curcolor}{0 0 0}
\pscustom[linestyle=none,fillstyle=solid,fillcolor=curcolor]
{
\newpath
\moveto(297.91565156,431.14060947)
\lineto(298.75114799,431.14060947)
\curveto(300.60260608,431.14060947)(302.09313011,429.98939017)(302.09313011,428.55940029)
\lineto(302.09313011,403.34521266)
\curveto(302.09313011,401.91522278)(300.60260608,400.76400348)(298.75114799,400.76400348)
\lineto(297.91565156,400.76400348)
\curveto(296.06419346,400.76400348)(294.57366943,401.91522278)(294.57366943,403.34521266)
\lineto(294.57366943,428.55940029)
\curveto(294.57366943,429.98939017)(296.06419346,431.14060947)(297.91565156,431.14060947)
\closepath
}
}
{
\newrgbcolor{curcolor}{0 0 0}
\pscustom[linewidth=0.62339705,linecolor=curcolor]
{
\newpath
\moveto(297.91565156,431.14060947)
\lineto(298.75114799,431.14060947)
\curveto(300.60260608,431.14060947)(302.09313011,429.98939017)(302.09313011,428.55940029)
\lineto(302.09313011,403.34521266)
\curveto(302.09313011,401.91522278)(300.60260608,400.76400348)(298.75114799,400.76400348)
\lineto(297.91565156,400.76400348)
\curveto(296.06419346,400.76400348)(294.57366943,401.91522278)(294.57366943,403.34521266)
\lineto(294.57366943,428.55940029)
\curveto(294.57366943,429.98939017)(296.06419346,431.14060947)(297.91565156,431.14060947)
\closepath
}
}
{
\newrgbcolor{curcolor}{1 1 1}
\pscustom[linestyle=none,fillstyle=solid,fillcolor=curcolor]
{
\newpath
\moveto(411.40200806,428.59799549)
\lineto(430.00000381,428.59799549)
\lineto(430.00000381,409.99999973)
\lineto(411.40200806,409.99999973)
\closepath
}
}
{
\newrgbcolor{curcolor}{0 0 0}
\pscustom[linewidth=1.40200508,linecolor=curcolor]
{
\newpath
\moveto(411.40200806,428.59799549)
\lineto(430.00000381,428.59799549)
\lineto(430.00000381,409.99999973)
\lineto(411.40200806,409.99999973)
\closepath
}
}
{
\newrgbcolor{curcolor}{0 0 0}
\pscustom[linestyle=none,fillstyle=solid,fillcolor=curcolor]
{
\newpath
\moveto(162.84375,426.92187664)
\lineto(162.84375,424.61328289)
\curveto(161.94530055,425.04295534)(161.09764515,425.36326752)(160.30078125,425.57422039)
\curveto(159.50389675,425.7851421)(158.73436627,425.89061075)(157.9921875,425.89062664)
\curveto(156.7031183,425.89061075)(155.70702554,425.640611)(155.00390625,425.14062664)
\curveto(154.30858944,424.640612)(153.96093354,423.92967521)(153.9609375,423.00781414)
\curveto(153.96093354,422.2343644)(154.19140206,421.64842749)(154.65234375,421.25000164)
\curveto(155.12108863,420.85936578)(156.00390025,420.54295984)(157.30078125,420.30078289)
\lineto(158.73046875,420.00781414)
\curveto(160.49608325,419.67186696)(161.7968632,419.07811756)(162.6328125,418.22656414)
\curveto(163.47654902,417.38280675)(163.8984236,416.24999539)(163.8984375,414.82812664)
\curveto(163.8984236,413.132811)(163.32811167,411.84765604)(162.1875,410.97265789)
\curveto(161.05467645,410.09765779)(159.39061561,409.66015823)(157.1953125,409.66015789)
\curveto(156.36718113,409.66015823)(155.48436952,409.75390813)(154.546875,409.94140789)
\curveto(153.61718388,410.12890776)(152.6523411,410.40625123)(151.65234375,410.77343914)
\lineto(151.65234375,413.21093914)
\curveto(152.61327864,412.67187396)(153.55468395,412.26562437)(154.4765625,411.99218914)
\curveto(155.3984321,411.71874992)(156.3046812,411.5820313)(157.1953125,411.58203289)
\curveto(158.54686645,411.5820313)(159.58983416,411.84765604)(160.32421875,412.37890789)
\curveto(161.05858269,412.91015498)(161.42576982,413.66796672)(161.42578125,414.65234539)
\curveto(161.42576982,415.51171487)(161.16014509,416.1835892)(160.62890625,416.66797039)
\curveto(160.10545864,417.15233823)(159.24217826,417.51561912)(158.0390625,417.75781414)
\lineto(156.59765625,418.03906414)
\curveto(154.83202642,418.39061825)(153.55468395,418.94139894)(152.765625,419.69140789)
\curveto(151.97656052,420.44139744)(151.58202967,421.48436515)(151.58203125,422.82031414)
\curveto(151.58202967,424.36717477)(152.12499787,425.58592355)(153.2109375,426.47656414)
\curveto(154.3046832,427.36717177)(155.80858794,427.81248382)(157.72265625,427.81250164)
\curveto(158.54296021,427.81248382)(159.37889687,427.73826515)(160.23046875,427.58984539)
\curveto(161.08202017,427.44139044)(161.95311305,427.21873442)(162.84375,426.92187664)
}
}
{
\newrgbcolor{curcolor}{0 0 0}
\pscustom[linestyle=none,fillstyle=solid,fillcolor=curcolor]
{
\newpath
\moveto(172.58203125,421.61328289)
\curveto(171.42577506,421.61327127)(170.51171347,421.16014673)(169.83984375,420.25390789)
\curveto(169.16796482,419.35546103)(168.83202765,418.12108726)(168.83203125,416.55078289)
\curveto(168.83202765,414.98046541)(169.16405857,413.74218539)(169.828125,412.83593914)
\curveto(170.49999473,411.9374997)(171.41796257,411.4882814)(172.58203125,411.48828289)
\curveto(173.73046025,411.4882814)(174.64061559,411.94140594)(175.3125,412.84765789)
\curveto(175.98436425,413.75390413)(176.32030141,414.9882779)(176.3203125,416.55078289)
\curveto(176.32030141,418.10546228)(175.98436425,419.3359298)(175.3125,420.24218914)
\curveto(174.64061559,421.15624048)(173.73046025,421.61327127)(172.58203125,421.61328289)
\moveto(172.58203125,423.44140789)
\curveto(174.45702203,423.44139444)(175.9296768,422.83202005)(177,421.61328289)
\curveto(178.07029966,420.39452249)(178.60545538,418.70702418)(178.60546875,416.55078289)
\curveto(178.60545538,414.40234098)(178.07029966,412.71484267)(177,411.48828289)
\curveto(175.9296768,410.26953262)(174.45702203,409.66015823)(172.58203125,409.66015789)
\curveto(170.69921329,409.66015823)(169.22265226,410.26953262)(168.15234375,411.48828289)
\curveto(167.08984189,412.71484267)(166.55859243,414.40234098)(166.55859375,416.55078289)
\curveto(166.55859243,418.70702418)(167.08984189,420.39452249)(168.15234375,421.61328289)
\curveto(169.22265226,422.83202005)(170.69921329,423.44139444)(172.58203125,423.44140789)
}
}
{
\newrgbcolor{curcolor}{0 0 0}
\pscustom[linestyle=none,fillstyle=solid,fillcolor=curcolor]
{
\newpath
\moveto(193.078125,417.92187664)
\lineto(193.078125,410.00000164)
\lineto(190.921875,410.00000164)
\lineto(190.921875,417.85156414)
\curveto(190.92186398,419.09374254)(190.67967673,420.02342911)(190.1953125,420.64062664)
\curveto(189.7109277,421.25780288)(188.98436592,421.56639632)(188.015625,421.56640789)
\curveto(186.85155555,421.56639632)(185.93358772,421.19530294)(185.26171875,420.45312664)
\curveto(184.58983907,419.71092942)(184.2539019,418.69921169)(184.25390625,417.41797039)
\lineto(184.25390625,410.00000164)
\lineto(182.0859375,410.00000164)
\lineto(182.0859375,423.12500164)
\lineto(184.25390625,423.12500164)
\lineto(184.25390625,421.08593914)
\curveto(184.76952639,421.87498976)(185.37499453,422.46483292)(186.0703125,422.85547039)
\curveto(186.77343063,423.24608214)(187.58202357,423.44139444)(188.49609375,423.44140789)
\curveto(190.00389615,423.44139444)(191.14452001,422.97264491)(191.91796875,422.03515789)
\curveto(192.69139346,421.10545928)(193.07811183,419.7343669)(193.078125,417.92187664)
}
}
{
\newrgbcolor{curcolor}{0 0 0}
\pscustom[linestyle=none,fillstyle=solid,fillcolor=curcolor,opacity=0.11612902]
{
\newpath
\moveto(267.26809502,380.00000164)
\lineto(410.85629463,380.00000164)
\curveto(415.92190741,380.00000164)(420,375.92190904)(420,370.85629627)
\lineto(420,357.7938573)
\curveto(420,352.72824453)(415.92190741,348.65015193)(410.85629463,348.65015193)
\lineto(267.26809502,348.65015193)
\curveto(262.20248224,348.65015193)(258.12438965,352.72824453)(258.12438965,357.7938573)
\lineto(258.12438965,370.85629627)
\curveto(258.12438965,375.92190904)(262.20248224,380.00000164)(267.26809502,380.00000164)
\closepath
}
}
{
\newrgbcolor{curcolor}{0 0 0}
\pscustom[linewidth=2.02831459,linecolor=curcolor]
{
\newpath
\moveto(267.26809502,380.00000164)
\lineto(410.85629463,380.00000164)
\curveto(415.92190741,380.00000164)(420,375.92190904)(420,370.85629627)
\lineto(420,357.7938573)
\curveto(420,352.72824453)(415.92190741,348.65015193)(410.85629463,348.65015193)
\lineto(267.26809502,348.65015193)
\curveto(262.20248224,348.65015193)(258.12438965,352.72824453)(258.12438965,357.7938573)
\lineto(258.12438965,370.85629627)
\curveto(258.12438965,375.92190904)(262.20248224,380.00000164)(267.26809502,380.00000164)
\closepath
}
}
{
\newrgbcolor{curcolor}{0 0 0}
\pscustom[linestyle=none,fillstyle=solid,fillcolor=curcolor]
{
\newpath
\moveto(152.35546875,377.49609539)
\lineto(154.72265625,377.49609539)
\lineto(154.72265625,361.99218914)
\lineto(163.2421875,361.99218914)
\lineto(163.2421875,360.00000164)
\lineto(152.35546875,360.00000164)
\lineto(152.35546875,377.49609539)
}
}
{
\newrgbcolor{curcolor}{0 0 0}
\pscustom[linestyle=none,fillstyle=solid,fillcolor=curcolor]
{
\newpath
\moveto(171.5859375,366.59765789)
\curveto(169.84374352,366.59765129)(168.63671347,366.39843274)(167.96484375,366.00000164)
\curveto(167.29296482,365.60155853)(166.95702765,364.92187171)(166.95703125,363.96093914)
\curveto(166.95702765,363.19531094)(167.2070274,362.58593655)(167.70703125,362.13281414)
\curveto(168.21483889,361.68749995)(168.90233821,361.46484392)(169.76953125,361.46484539)
\curveto(170.96483614,361.46484392)(171.92186644,361.8867185)(172.640625,362.73047039)
\curveto(173.36717749,363.5820293)(173.73045838,364.71093442)(173.73046875,366.11718914)
\lineto(173.73046875,366.59765789)
\lineto(171.5859375,366.59765789)
\moveto(175.88671875,367.48828289)
\lineto(175.88671875,360.00000164)
\lineto(173.73046875,360.00000164)
\lineto(173.73046875,361.99218914)
\curveto(173.23827137,361.19531294)(172.62499073,360.60546978)(171.890625,360.22265789)
\curveto(171.1562422,359.84765804)(170.2578056,359.66015823)(169.1953125,359.66015789)
\curveto(167.85155801,359.66015823)(166.78124658,360.03515785)(165.984375,360.78515789)
\curveto(165.19531066,361.54296884)(164.80077981,362.55468658)(164.80078125,363.82031414)
\curveto(164.80077981,365.29687134)(165.29296682,366.41015148)(166.27734375,367.16015789)
\curveto(167.26952734,367.91014998)(168.74608836,368.2851496)(170.70703125,368.28515789)
\lineto(173.73046875,368.28515789)
\lineto(173.73046875,368.49609539)
\curveto(173.73045838,369.4882734)(173.40233371,370.25389763)(172.74609375,370.79297039)
\curveto(172.09764751,371.33983405)(171.18358593,371.61327127)(170.00390625,371.61328289)
\curveto(169.25390036,371.61327127)(168.52343234,371.52342761)(167.8125,371.34375164)
\curveto(167.10155876,371.16405297)(166.41796569,370.89452199)(165.76171875,370.53515789)
\lineto(165.76171875,372.52734539)
\curveto(166.55077806,372.83202005)(167.31640229,373.05858233)(168.05859375,373.20703289)
\curveto(168.80077581,373.36326952)(169.52343134,373.44139444)(170.2265625,373.44140789)
\curveto(172.12499123,373.44139444)(173.54295857,372.94920744)(174.48046875,371.96484539)
\curveto(175.41795669,370.98045941)(175.88670622,369.4882734)(175.88671875,367.48828289)
}
}
{
\newrgbcolor{curcolor}{0 0 0}
\pscustom[linestyle=none,fillstyle=solid,fillcolor=curcolor]
{
\newpath
\moveto(191.25,367.92187664)
\lineto(191.25,360.00000164)
\lineto(189.09375,360.00000164)
\lineto(189.09375,367.85156414)
\curveto(189.09373898,369.09374254)(188.85155173,370.02342911)(188.3671875,370.64062664)
\curveto(187.8828027,371.25780288)(187.15624092,371.56639632)(186.1875,371.56640789)
\curveto(185.02343055,371.56639632)(184.10546272,371.19530294)(183.43359375,370.45312664)
\curveto(182.76171407,369.71092942)(182.4257769,368.69921169)(182.42578125,367.41797039)
\lineto(182.42578125,360.00000164)
\lineto(180.2578125,360.00000164)
\lineto(180.2578125,373.12500164)
\lineto(182.42578125,373.12500164)
\lineto(182.42578125,371.08593914)
\curveto(182.94140139,371.87498976)(183.54686953,372.46483292)(184.2421875,372.85547039)
\curveto(184.94530563,373.24608214)(185.75389857,373.44139444)(186.66796875,373.44140789)
\curveto(188.17577115,373.44139444)(189.31639501,372.97264491)(190.08984375,372.03515789)
\curveto(190.86326846,371.10545928)(191.24998683,369.7343669)(191.25,367.92187664)
}
}
{
\newrgbcolor{curcolor}{0 0 0}
\pscustom[linestyle=none,fillstyle=solid,fillcolor=curcolor]
{
\newpath
\moveto(204.2109375,366.71484539)
\curveto(204.2109266,368.27733711)(203.88670818,369.4882734)(203.23828125,370.34765789)
\curveto(202.59764696,371.20702168)(201.69530412,371.63670875)(200.53125,371.63672039)
\curveto(199.37499394,371.63670875)(198.47265109,371.20702168)(197.82421875,370.34765789)
\curveto(197.18358988,369.4882734)(196.8632777,368.27733711)(196.86328125,366.71484539)
\curveto(196.8632777,365.16015273)(197.18358988,363.95312268)(197.82421875,363.09375164)
\curveto(198.47265109,362.2343744)(199.37499394,361.80468733)(200.53125,361.80468914)
\curveto(201.69530412,361.80468733)(202.59764696,362.2343744)(203.23828125,363.09375164)
\curveto(203.88670818,363.95312268)(204.2109266,365.16015273)(204.2109375,366.71484539)
\moveto(206.3671875,361.62890789)
\curveto(206.36717445,359.39453349)(205.87108119,357.7343789)(204.87890625,356.64843914)
\curveto(203.88670818,355.55469358)(202.36717845,355.00781913)(200.3203125,355.00781414)
\curveto(199.56249375,355.00781913)(198.84765071,355.06641282)(198.17578125,355.18359539)
\curveto(197.50390206,355.29297509)(196.85155896,355.46484992)(196.21875,355.69922039)
\lineto(196.21875,357.79687664)
\curveto(196.85155896,357.45312918)(197.47655834,357.19922319)(198.09375,357.03515789)
\curveto(198.7109321,356.87109851)(199.33983772,356.78906735)(199.98046875,356.78906414)
\curveto(201.39452317,356.78906735)(202.45311586,357.16016073)(203.15625,357.90234539)
\curveto(203.85936445,358.63672175)(204.2109266,359.75000189)(204.2109375,361.24218914)
\lineto(204.2109375,362.30859539)
\curveto(203.76561455,361.53515635)(203.19530262,360.95703193)(202.5,360.57422039)
\curveto(201.80467901,360.19140769)(200.97264859,360.00000164)(200.00390625,360.00000164)
\curveto(198.39452617,360.00000164)(197.09765246,360.61328227)(196.11328125,361.83984539)
\curveto(195.12890443,363.06640482)(194.63671743,364.69140319)(194.63671875,366.71484539)
\curveto(194.63671743,368.74608664)(195.12890443,370.37499126)(196.11328125,371.60156414)
\curveto(197.09765246,372.82811381)(198.39452617,373.44139444)(200.00390625,373.44140789)
\curveto(200.97264859,373.44139444)(201.80467901,373.24998839)(202.5,372.86718914)
\curveto(203.19530262,372.48436415)(203.76561455,371.90623973)(204.2109375,371.13281414)
\lineto(204.2109375,373.12500164)
\lineto(206.3671875,373.12500164)
\lineto(206.3671875,361.62890789)
}
}
{
\newrgbcolor{curcolor}{0 0 0}
\pscustom[linestyle=none,fillstyle=solid,fillcolor=curcolor]
{
\newpath
\moveto(210.5859375,365.17968914)
\lineto(210.5859375,373.12500164)
\lineto(212.7421875,373.12500164)
\lineto(212.7421875,365.26172039)
\curveto(212.7421833,364.01952887)(212.98437056,363.08593605)(213.46875,362.46093914)
\curveto(213.95311959,361.84374979)(214.67968137,361.53515635)(215.6484375,361.53515789)
\curveto(216.81249173,361.53515635)(217.73045957,361.90624973)(218.40234375,362.64843914)
\curveto(219.08202071,363.39062325)(219.42186413,364.40234098)(219.421875,365.68359539)
\lineto(219.421875,373.12500164)
\lineto(221.578125,373.12500164)
\lineto(221.578125,360.00000164)
\lineto(219.421875,360.00000164)
\lineto(219.421875,362.01562664)
\curveto(218.89842715,361.21875042)(218.28905276,360.62500101)(217.59375,360.23437664)
\curveto(216.90624164,359.85156428)(216.10546119,359.66015823)(215.19140625,359.66015789)
\curveto(213.68358861,359.66015823)(212.53905851,360.12890776)(211.7578125,361.06640789)
\curveto(210.97656007,362.00390588)(210.58593546,363.37499826)(210.5859375,365.17968914)
\moveto(216.01171875,373.44140789)
\lineto(216.01171875,373.44140789)
}
}
{
\newrgbcolor{curcolor}{0 0 0}
\pscustom[linestyle=none,fillstyle=solid,fillcolor=curcolor]
{
\newpath
\moveto(237.26953125,367.10156414)
\lineto(237.26953125,366.04687664)
\lineto(227.35546875,366.04687664)
\curveto(227.44921508,364.56249707)(227.89452714,363.42968571)(228.69140625,362.64843914)
\curveto(229.49608804,361.87499976)(230.61327442,361.4882814)(232.04296875,361.48828289)
\curveto(232.87108466,361.4882814)(233.67186511,361.5898438)(234.4453125,361.79297039)
\curveto(235.22655105,361.99609339)(235.99998778,362.30078058)(236.765625,362.70703289)
\lineto(236.765625,360.66797039)
\curveto(235.99217529,360.33984505)(235.19920733,360.0898453)(234.38671875,359.91797039)
\curveto(233.57420896,359.74609564)(232.74999103,359.66015823)(231.9140625,359.66015789)
\curveto(229.82030646,359.66015823)(228.16015187,360.26953262)(226.93359375,361.48828289)
\curveto(225.71484182,362.70703018)(225.10546743,364.35546603)(225.10546875,366.43359539)
\curveto(225.10546743,368.5820243)(225.68359185,370.2851476)(226.83984375,371.54297039)
\curveto(228.00390203,372.80858258)(229.57030671,373.44139444)(231.5390625,373.44140789)
\curveto(233.30467798,373.44139444)(234.69920783,372.87108251)(235.72265625,371.73047039)
\curveto(236.75389328,370.59764729)(237.26951776,369.05468008)(237.26953125,367.10156414)
\moveto(235.11328125,367.73437664)
\curveto(235.09764493,368.91405522)(234.76561402,369.85546053)(234.1171875,370.55859539)
\curveto(233.4765528,371.26170912)(232.62499116,371.61327127)(231.5625,371.61328289)
\curveto(230.35936842,371.61327127)(229.39452564,371.27342786)(228.66796875,370.59375164)
\curveto(227.94921458,369.91405422)(227.5351525,368.95702393)(227.42578125,367.72265789)
\lineto(235.11328125,367.73437664)
}
}
{
\newrgbcolor{curcolor}{0 0 0}
\pscustom[linestyle=none,fillstyle=solid,fillcolor=curcolor]
{
\newpath
\moveto(249.17578125,372.73828289)
\lineto(249.17578125,370.69922039)
\curveto(248.56639623,371.01170937)(247.93358436,371.24608414)(247.27734375,371.40234539)
\curveto(246.62108568,371.55858383)(245.94139886,371.63670875)(245.23828125,371.63672039)
\curveto(244.16796313,371.63670875)(243.36327643,371.47264641)(242.82421875,371.14453289)
\curveto(242.292965,370.81639707)(242.02734027,370.32421006)(242.02734375,369.66797039)
\curveto(242.02734027,369.16796122)(242.21874633,368.77343036)(242.6015625,368.48437664)
\curveto(242.98437056,368.20311843)(243.75390104,367.93358745)(244.91015625,367.67578289)
\lineto(245.6484375,367.51172039)
\curveto(247.17967887,367.1835882)(248.26561528,366.71874492)(248.90625,366.11718914)
\curveto(249.55467649,365.52343361)(249.87889492,364.69140319)(249.87890625,363.62109539)
\curveto(249.87889492,362.40234298)(249.3945204,361.4375002)(248.42578125,360.72656414)
\curveto(247.46483483,360.01562662)(246.14061741,359.66015823)(244.453125,359.66015789)
\curveto(243.7499948,359.66015823)(243.01562053,359.73047066)(242.25,359.87109539)
\curveto(241.49218455,360.00390788)(240.69140411,360.20703268)(239.84765625,360.48047039)
\lineto(239.84765625,362.70703289)
\curveto(240.64452915,362.29296809)(241.42968462,361.98046841)(242.203125,361.76953289)
\curveto(242.97655807,361.56640632)(243.7421823,361.46484392)(244.5,361.46484539)
\curveto(245.51561803,361.46484392)(246.29686725,361.63671875)(246.84375,361.98047039)
\curveto(247.39061616,362.33203055)(247.66405338,362.82421756)(247.6640625,363.45703289)
\curveto(247.66405338,364.04296634)(247.46483483,364.49218464)(247.06640625,364.80468914)
\curveto(246.67577312,365.11718402)(245.81249273,365.41796497)(244.4765625,365.70703289)
\lineto(243.7265625,365.88281414)
\curveto(242.39062116,366.16405797)(241.42577837,366.59374504)(240.83203125,367.17187664)
\curveto(240.23827956,367.75780638)(239.94140486,368.55858683)(239.94140625,369.57422039)
\curveto(239.94140486,370.80858458)(240.37890442,371.76170862)(241.25390625,372.43359539)
\curveto(242.12890267,373.10545728)(243.37108893,373.44139444)(244.98046875,373.44140789)
\curveto(245.77733652,373.44139444)(246.52733577,373.38280075)(247.23046875,373.26562664)
\curveto(247.93358436,373.14842599)(248.58202121,372.97264491)(249.17578125,372.73828289)
}
}
{
\newrgbcolor{curcolor}{0 0 0}
\pscustom[linestyle=none,fillstyle=solid,fillcolor=curcolor]
{
\newpath
\moveto(214.28515625,302.49609539)
\lineto(214.28515625,307.19531414)
\lineto(210.41796875,307.19531414)
\lineto(210.41796875,309.14062664)
\lineto(216.62890625,309.14062664)
\lineto(216.62890625,301.62890789)
\curveto(215.71482804,300.98046941)(214.70701654,300.4882824)(213.60546875,300.15234539)
\curveto(212.50389375,299.82422056)(211.32811367,299.66015823)(210.078125,299.66015789)
\curveto(207.34374266,299.66015823)(205.2031198,300.45703243)(203.65625,302.05078289)
\curveto(202.11718538,303.65234173)(201.3476549,305.87890201)(201.34765625,308.73047039)
\curveto(201.3476549,311.5898338)(202.11718538,313.81639407)(203.65625,315.41015789)
\curveto(205.2031198,317.01170337)(207.34374266,317.81248382)(210.078125,317.81250164)
\curveto(211.21873878,317.81248382)(212.30076895,317.67185896)(213.32421875,317.39062664)
\curveto(214.35545439,317.10935953)(215.3046722,316.69529744)(216.171875,316.14843914)
\lineto(216.171875,313.62890789)
\curveto(215.2968597,314.37108101)(214.36717313,314.92967421)(213.3828125,315.30468914)
\curveto(212.3984251,315.67967346)(211.36326989,315.86717327)(210.27734375,315.86718914)
\curveto(208.13671061,315.86717327)(206.52733722,315.26951762)(205.44921875,314.07422039)
\curveto(204.37890187,312.87889501)(203.84374616,311.09764679)(203.84375,308.73047039)
\curveto(203.84374616,306.37108901)(204.37890187,304.59374704)(205.44921875,303.39843914)
\curveto(206.52733722,302.20312443)(208.13671061,301.60546878)(210.27734375,301.60547039)
\curveto(211.11327014,301.60546878)(211.85936314,301.67578121)(212.515625,301.81640789)
\curveto(213.17186183,301.96484342)(213.76170499,302.19140569)(214.28515625,302.49609539)
}
}
{
\newrgbcolor{curcolor}{0 0 0}
\pscustom[linestyle=none,fillstyle=solid,fillcolor=curcolor]
{
\newpath
\moveto(232.09765625,307.10156414)
\lineto(232.09765625,306.04687664)
\lineto(222.18359375,306.04687664)
\curveto(222.27734008,304.56249707)(222.72265214,303.42968571)(223.51953125,302.64843914)
\curveto(224.32421304,301.87499976)(225.44139942,301.4882814)(226.87109375,301.48828289)
\curveto(227.69920966,301.4882814)(228.49999011,301.5898438)(229.2734375,301.79297039)
\curveto(230.05467605,301.99609339)(230.82811278,302.30078058)(231.59375,302.70703289)
\lineto(231.59375,300.66797039)
\curveto(230.82030029,300.33984505)(230.02733233,300.0898453)(229.21484375,299.91797039)
\curveto(228.40233396,299.74609564)(227.57811603,299.66015823)(226.7421875,299.66015789)
\curveto(224.64843146,299.66015823)(222.98827687,300.26953262)(221.76171875,301.48828289)
\curveto(220.54296682,302.70703018)(219.93359243,304.35546603)(219.93359375,306.43359539)
\curveto(219.93359243,308.5820243)(220.51171685,310.2851476)(221.66796875,311.54297039)
\curveto(222.83202703,312.80858258)(224.39843171,313.44139444)(226.3671875,313.44140789)
\curveto(228.13280298,313.44139444)(229.52733283,312.87108251)(230.55078125,311.73047039)
\curveto(231.58201828,310.59764729)(232.09764276,309.05468008)(232.09765625,307.10156414)
\moveto(229.94140625,307.73437664)
\curveto(229.92576993,308.91405522)(229.59373902,309.85546053)(228.9453125,310.55859539)
\curveto(228.3046778,311.26170912)(227.45311616,311.61327127)(226.390625,311.61328289)
\curveto(225.18749342,311.61327127)(224.22265064,311.27342786)(223.49609375,310.59375164)
\curveto(222.77733958,309.91405422)(222.3632775,308.95702393)(222.25390625,307.72265789)
\lineto(229.94140625,307.73437664)
}
}
{
\newrgbcolor{curcolor}{0 0 0}
\pscustom[linestyle=none,fillstyle=solid,fillcolor=curcolor]
{
\newpath
\moveto(244.00390625,312.73828289)
\lineto(244.00390625,310.69922039)
\curveto(243.39452123,311.01170937)(242.76170936,311.24608414)(242.10546875,311.40234539)
\curveto(241.44921068,311.55858383)(240.76952386,311.63670875)(240.06640625,311.63672039)
\curveto(238.99608813,311.63670875)(238.19140143,311.47264641)(237.65234375,311.14453289)
\curveto(237.12109,310.81639707)(236.85546527,310.32421006)(236.85546875,309.66797039)
\curveto(236.85546527,309.16796122)(237.04687133,308.77343036)(237.4296875,308.48437664)
\curveto(237.81249556,308.20311843)(238.58202604,307.93358745)(239.73828125,307.67578289)
\lineto(240.4765625,307.51172039)
\curveto(242.00780387,307.1835882)(243.09374028,306.71874492)(243.734375,306.11718914)
\curveto(244.38280149,305.52343361)(244.70701992,304.69140319)(244.70703125,303.62109539)
\curveto(244.70701992,302.40234298)(244.2226454,301.4375002)(243.25390625,300.72656414)
\curveto(242.29295983,300.01562662)(240.96874241,299.66015823)(239.28125,299.66015789)
\curveto(238.5781198,299.66015823)(237.84374553,299.73047066)(237.078125,299.87109539)
\curveto(236.32030955,300.00390788)(235.51952911,300.20703268)(234.67578125,300.48047039)
\lineto(234.67578125,302.70703289)
\curveto(235.47265415,302.29296809)(236.25780962,301.98046841)(237.03125,301.76953289)
\curveto(237.80468307,301.56640632)(238.5703073,301.46484392)(239.328125,301.46484539)
\curveto(240.34374303,301.46484392)(241.12499225,301.63671875)(241.671875,301.98047039)
\curveto(242.21874116,302.33203055)(242.49217838,302.82421756)(242.4921875,303.45703289)
\curveto(242.49217838,304.04296634)(242.29295983,304.49218464)(241.89453125,304.80468914)
\curveto(241.50389812,305.11718402)(240.64061773,305.41796497)(239.3046875,305.70703289)
\lineto(238.5546875,305.88281414)
\curveto(237.21874616,306.16405797)(236.25390337,306.59374504)(235.66015625,307.17187664)
\curveto(235.06640456,307.75780638)(234.76952986,308.55858683)(234.76953125,309.57422039)
\curveto(234.76952986,310.80858458)(235.20702942,311.76170862)(236.08203125,312.43359539)
\curveto(236.95702767,313.10545728)(238.19921393,313.44139444)(239.80859375,313.44140789)
\curveto(240.60546152,313.44139444)(241.35546077,313.38280075)(242.05859375,313.26562664)
\curveto(242.76170936,313.14842599)(243.41014621,312.97264491)(244.00390625,312.73828289)
}
}
{
\newrgbcolor{curcolor}{0 0 0}
\pscustom[linestyle=none,fillstyle=solid,fillcolor=curcolor]
{
\newpath
\moveto(250.28515625,316.85156414)
\lineto(250.28515625,313.12500164)
\lineto(254.7265625,313.12500164)
\lineto(254.7265625,311.44922039)
\lineto(250.28515625,311.44922039)
\lineto(250.28515625,304.32422039)
\curveto(250.28515186,303.25390463)(250.42968296,302.56640532)(250.71875,302.26172039)
\curveto(251.01561987,301.95703093)(251.61327553,301.80468733)(252.51171875,301.80468914)
\lineto(254.7265625,301.80468914)
\lineto(254.7265625,300.00000164)
\lineto(252.51171875,300.00000164)
\curveto(250.84765129,300.00000164)(249.69921494,300.30859508)(249.06640625,300.92578289)
\curveto(248.43359121,301.55078133)(248.11718527,302.6835927)(248.1171875,304.32422039)
\lineto(248.1171875,311.44922039)
\lineto(246.53515625,311.44922039)
\lineto(246.53515625,313.12500164)
\lineto(248.1171875,313.12500164)
\lineto(248.1171875,316.85156414)
\lineto(250.28515625,316.85156414)
}
}
{
\newrgbcolor{curcolor}{0 0 0}
\pscustom[linestyle=none,fillstyle=solid,fillcolor=curcolor]
{
\newpath
\moveto(257.57421875,313.12500164)
\lineto(259.73046875,313.12500164)
\lineto(259.73046875,300.00000164)
\lineto(257.57421875,300.00000164)
\lineto(257.57421875,313.12500164)
\moveto(257.57421875,318.23437664)
\lineto(259.73046875,318.23437664)
\lineto(259.73046875,315.50390789)
\lineto(257.57421875,315.50390789)
\lineto(257.57421875,318.23437664)
}
}
{
\newrgbcolor{curcolor}{0 0 0}
\pscustom[linestyle=none,fillstyle=solid,fillcolor=curcolor]
{
\newpath
\moveto(269.31640625,311.61328289)
\curveto(268.16015006,311.61327127)(267.24608847,311.16014673)(266.57421875,310.25390789)
\curveto(265.90233982,309.35546103)(265.56640265,308.12108726)(265.56640625,306.55078289)
\curveto(265.56640265,304.98046541)(265.89843357,303.74218539)(266.5625,302.83593914)
\curveto(267.23436973,301.9374997)(268.15233757,301.4882814)(269.31640625,301.48828289)
\curveto(270.46483525,301.4882814)(271.37499059,301.94140594)(272.046875,302.84765789)
\curveto(272.71873925,303.75390413)(273.05467641,304.9882779)(273.0546875,306.55078289)
\curveto(273.05467641,308.10546228)(272.71873925,309.3359298)(272.046875,310.24218914)
\curveto(271.37499059,311.15624048)(270.46483525,311.61327127)(269.31640625,311.61328289)
\moveto(269.31640625,313.44140789)
\curveto(271.19139703,313.44139444)(272.6640518,312.83202005)(273.734375,311.61328289)
\curveto(274.80467466,310.39452249)(275.33983038,308.70702418)(275.33984375,306.55078289)
\curveto(275.33983038,304.40234098)(274.80467466,302.71484267)(273.734375,301.48828289)
\curveto(272.6640518,300.26953262)(271.19139703,299.66015823)(269.31640625,299.66015789)
\curveto(267.43358829,299.66015823)(265.95702726,300.26953262)(264.88671875,301.48828289)
\curveto(263.82421689,302.71484267)(263.29296743,304.40234098)(263.29296875,306.55078289)
\curveto(263.29296743,308.70702418)(263.82421689,310.39452249)(264.88671875,311.61328289)
\curveto(265.95702726,312.83202005)(267.43358829,313.44139444)(269.31640625,313.44140789)
}
}
{
\newrgbcolor{curcolor}{0 0 0}
\pscustom[linestyle=none,fillstyle=solid,fillcolor=curcolor]
{
\newpath
\moveto(289.8125,307.92187664)
\lineto(289.8125,300.00000164)
\lineto(287.65625,300.00000164)
\lineto(287.65625,307.85156414)
\curveto(287.65623898,309.09374254)(287.41405173,310.02342911)(286.9296875,310.64062664)
\curveto(286.4453027,311.25780288)(285.71874092,311.56639632)(284.75,311.56640789)
\curveto(283.58593055,311.56639632)(282.66796272,311.19530294)(281.99609375,310.45312664)
\curveto(281.32421407,309.71092942)(280.9882769,308.69921169)(280.98828125,307.41797039)
\lineto(280.98828125,300.00000164)
\lineto(278.8203125,300.00000164)
\lineto(278.8203125,313.12500164)
\lineto(280.98828125,313.12500164)
\lineto(280.98828125,311.08593914)
\curveto(281.50390139,311.87498976)(282.10936953,312.46483292)(282.8046875,312.85547039)
\curveto(283.50780563,313.24608214)(284.31639857,313.44139444)(285.23046875,313.44140789)
\curveto(286.73827115,313.44139444)(287.87889501,312.97264491)(288.65234375,312.03515789)
\curveto(289.42576846,311.10545928)(289.81248683,309.7343669)(289.8125,307.92187664)
}
}
{
\newrgbcolor{curcolor}{0 0 0}
\pscustom[linestyle=none,fillstyle=solid,fillcolor=curcolor]
{
}
}
{
\newrgbcolor{curcolor}{0 0 0}
\pscustom[linestyle=none,fillstyle=solid,fillcolor=curcolor]
{
\newpath
\moveto(303.86328125,301.96875164)
\lineto(303.86328125,295.00781414)
\lineto(301.6953125,295.00781414)
\lineto(301.6953125,313.12500164)
\lineto(303.86328125,313.12500164)
\lineto(303.86328125,311.13281414)
\curveto(304.31640145,311.91405222)(304.88671338,312.49217664)(305.57421875,312.86718914)
\curveto(306.2695245,313.24998839)(307.09764867,313.44139444)(308.05859375,313.44140789)
\curveto(309.65233361,313.44139444)(310.94530107,312.80858258)(311.9375,311.54297039)
\curveto(312.93748658,310.27733511)(313.43748608,308.61327427)(313.4375,306.55078289)
\curveto(313.43748608,304.4882784)(312.93748658,302.82421756)(311.9375,301.55859539)
\curveto(310.94530107,300.29297009)(309.65233361,299.66015823)(308.05859375,299.66015789)
\curveto(307.09764867,299.66015823)(306.2695245,299.84765804)(305.57421875,300.22265789)
\curveto(304.88671338,300.60546978)(304.31640145,301.18750045)(303.86328125,301.96875164)
\moveto(311.19921875,306.55078289)
\curveto(311.19920707,308.13671225)(310.87108239,309.37889851)(310.21484375,310.27734539)
\curveto(309.5663962,311.1835842)(308.67186584,311.63670875)(307.53125,311.63672039)
\curveto(306.39061813,311.63670875)(305.49218152,311.1835842)(304.8359375,310.27734539)
\curveto(304.18749533,309.37889851)(303.8632769,308.13671225)(303.86328125,306.55078289)
\curveto(303.8632769,304.96484042)(304.18749533,303.71874792)(304.8359375,302.81250164)
\curveto(305.49218152,301.91406222)(306.39061813,301.46484392)(307.53125,301.46484539)
\curveto(308.67186584,301.46484392)(309.5663962,301.91406222)(310.21484375,302.81250164)
\curveto(310.87108239,303.71874792)(311.19920707,304.96484042)(311.19921875,306.55078289)
}
}
{
\newrgbcolor{curcolor}{0 0 0}
\pscustom[linestyle=none,fillstyle=solid,fillcolor=curcolor]
{
\newpath
\moveto(324.6171875,311.10937664)
\curveto(324.37499037,311.24999039)(324.10936564,311.35155278)(323.8203125,311.41406414)
\curveto(323.53905371,311.48436515)(323.22655402,311.51952137)(322.8828125,311.51953289)
\curveto(321.66405559,311.51952137)(320.72655652,311.12108426)(320.0703125,310.32422039)
\curveto(319.42187033,309.53514835)(319.0976519,308.39843074)(319.09765625,306.91406414)
\lineto(319.09765625,300.00000164)
\lineto(316.9296875,300.00000164)
\lineto(316.9296875,313.12500164)
\lineto(319.09765625,313.12500164)
\lineto(319.09765625,311.08593914)
\curveto(319.55077645,311.88280225)(320.14061961,312.47264541)(320.8671875,312.85547039)
\curveto(321.59374316,313.24608214)(322.47655477,313.44139444)(323.515625,313.44140789)
\curveto(323.66405359,313.44139444)(323.82811592,313.42967571)(324.0078125,313.40625164)
\curveto(324.18749056,313.39061325)(324.38670911,313.36326952)(324.60546875,313.32422039)
\lineto(324.6171875,311.10937664)
}
}
{
\newrgbcolor{curcolor}{0 0 0}
\pscustom[linestyle=none,fillstyle=solid,fillcolor=curcolor]
{
\newpath
\moveto(331.47265625,311.61328289)
\curveto(330.31640006,311.61327127)(329.40233847,311.16014673)(328.73046875,310.25390789)
\curveto(328.05858982,309.35546103)(327.72265265,308.12108726)(327.72265625,306.55078289)
\curveto(327.72265265,304.98046541)(328.05468357,303.74218539)(328.71875,302.83593914)
\curveto(329.39061973,301.9374997)(330.30858757,301.4882814)(331.47265625,301.48828289)
\curveto(332.62108525,301.4882814)(333.53124059,301.94140594)(334.203125,302.84765789)
\curveto(334.87498925,303.75390413)(335.21092641,304.9882779)(335.2109375,306.55078289)
\curveto(335.21092641,308.10546228)(334.87498925,309.3359298)(334.203125,310.24218914)
\curveto(333.53124059,311.15624048)(332.62108525,311.61327127)(331.47265625,311.61328289)
\moveto(331.47265625,313.44140789)
\curveto(333.34764703,313.44139444)(334.8203018,312.83202005)(335.890625,311.61328289)
\curveto(336.96092466,310.39452249)(337.49608038,308.70702418)(337.49609375,306.55078289)
\curveto(337.49608038,304.40234098)(336.96092466,302.71484267)(335.890625,301.48828289)
\curveto(334.8203018,300.26953262)(333.34764703,299.66015823)(331.47265625,299.66015789)
\curveto(329.58983829,299.66015823)(328.11327726,300.26953262)(327.04296875,301.48828289)
\curveto(325.98046689,302.71484267)(325.44921743,304.40234098)(325.44921875,306.55078289)
\curveto(325.44921743,308.70702418)(325.98046689,310.39452249)(327.04296875,311.61328289)
\curveto(328.11327726,312.83202005)(329.58983829,313.44139444)(331.47265625,313.44140789)
}
}
{
\newrgbcolor{curcolor}{0 0 0}
\pscustom[linestyle=none,fillstyle=solid,fillcolor=curcolor]
{
\newpath
\moveto(351.6640625,313.12500164)
\lineto(351.6640625,300.00000164)
\lineto(349.49609375,300.00000164)
\lineto(349.49609375,311.44922039)
\lineto(343.578125,311.44922039)
\lineto(343.578125,300.00000164)
\lineto(341.41015625,300.00000164)
\lineto(341.41015625,311.44922039)
\lineto(339.34765625,311.44922039)
\lineto(339.34765625,313.12500164)
\lineto(341.41015625,313.12500164)
\lineto(341.41015625,314.03906414)
\curveto(341.41015364,315.46873617)(341.7460908,316.52342261)(342.41796875,317.20312664)
\curveto(343.09765195,317.89060875)(344.13671341,318.2343584)(345.53515625,318.23437664)
\lineto(347.703125,318.23437664)
\lineto(347.703125,316.44140789)
\lineto(345.640625,316.44140789)
\curveto(344.86718143,316.44139144)(344.32811947,316.2851416)(344.0234375,315.97265789)
\curveto(343.72655757,315.66014223)(343.57812022,315.09764279)(343.578125,314.28515789)
\lineto(343.578125,313.12500164)
\lineto(351.6640625,313.12500164)
\moveto(349.49609375,318.21093914)
\lineto(351.6640625,318.21093914)
\lineto(351.6640625,315.48047039)
\lineto(349.49609375,315.48047039)
\lineto(349.49609375,318.21093914)
}
}
{
\newrgbcolor{curcolor}{0 0 0}
\pscustom[linestyle=none,fillstyle=solid,fillcolor=curcolor]
{
\newpath
\moveto(356.19921875,318.23437664)
\lineto(358.35546875,318.23437664)
\lineto(358.35546875,300.00000164)
\lineto(356.19921875,300.00000164)
\lineto(356.19921875,318.23437664)
}
}
{
\newrgbcolor{curcolor}{0 0 0}
\pscustom[linestyle=none,fillstyle=solid,fillcolor=curcolor]
{
\newpath
\moveto(150.65234375,38.20311138)
\curveto(151.16014509,38.03122835)(151.6523321,37.66404121)(152.12890625,37.10154888)
\curveto(152.61326864,36.53904234)(153.09764315,35.76560561)(153.58203125,34.78123638)
\lineto(155.984375,29.99998638)
\lineto(153.44140625,29.99998638)
\lineto(151.203125,34.48826763)
\curveto(150.62498938,35.66013697)(150.06248994,36.43747994)(149.515625,36.82029888)
\curveto(148.97655352,37.20310417)(148.23827301,37.39451023)(147.30078125,37.39451763)
\lineto(144.72265625,37.39451763)
\lineto(144.72265625,29.99998638)
\lineto(142.35546875,29.99998638)
\lineto(142.35546875,47.49608013)
\lineto(147.69921875,47.49608013)
\curveto(149.69920905,47.49606263)(151.19139506,47.0780943)(152.17578125,46.24217388)
\curveto(153.16014309,45.40622097)(153.6523301,44.14450348)(153.65234375,42.45701763)
\curveto(153.6523301,41.35544377)(153.39451786,40.44138219)(152.87890625,39.71483013)
\curveto(152.37108138,38.98825864)(151.62889462,38.48435289)(150.65234375,38.20311138)
\moveto(144.72265625,45.55076763)
\lineto(144.72265625,39.33983013)
\lineto(147.69921875,39.33983013)
\curveto(148.83983491,39.33982079)(149.69920905,39.60153928)(150.27734375,40.12498638)
\curveto(150.86327039,40.65622572)(151.15623884,41.43356869)(151.15625,42.45701763)
\curveto(151.15623884,43.48044165)(150.86327039,44.24997213)(150.27734375,44.76561138)
\curveto(149.69920905,45.28903359)(148.83983491,45.55075208)(147.69921875,45.55076763)
\lineto(144.72265625,45.55076763)
}
}
{
\newrgbcolor{curcolor}{0 0 0}
\pscustom[linestyle=none,fillstyle=solid,fillcolor=curcolor]
{
\newpath
\moveto(169.09765625,37.10154888)
\lineto(169.09765625,36.04686138)
\lineto(159.18359375,36.04686138)
\curveto(159.27734008,34.56248181)(159.72265214,33.42967045)(160.51953125,32.64842388)
\curveto(161.32421304,31.8749845)(162.44139942,31.48826614)(163.87109375,31.48826763)
\curveto(164.69920966,31.48826614)(165.49999011,31.58982854)(166.2734375,31.79295513)
\curveto(167.05467605,31.99607813)(167.82811278,32.30076533)(168.59375,32.70701763)
\lineto(168.59375,30.66795513)
\curveto(167.82030029,30.33982979)(167.02733233,30.08983004)(166.21484375,29.91795513)
\curveto(165.40233396,29.74608038)(164.57811603,29.66014297)(163.7421875,29.66014263)
\curveto(161.64843146,29.66014297)(159.98827687,30.26951736)(158.76171875,31.48826763)
\curveto(157.54296682,32.70701492)(156.93359243,34.35545077)(156.93359375,36.43358013)
\curveto(156.93359243,38.58200904)(157.51171685,40.28513234)(158.66796875,41.54295513)
\curveto(159.83202703,42.80856732)(161.39843171,43.44137919)(163.3671875,43.44139263)
\curveto(165.13280298,43.44137919)(166.52733283,42.87106726)(167.55078125,41.73045513)
\curveto(168.58201828,40.59763203)(169.09764276,39.05466482)(169.09765625,37.10154888)
\moveto(166.94140625,37.73436138)
\curveto(166.92576993,38.91403996)(166.59373902,39.85544527)(165.9453125,40.55858013)
\curveto(165.3046778,41.26169387)(164.45311616,41.61325601)(163.390625,41.61326763)
\curveto(162.18749342,41.61325601)(161.22265064,41.2734126)(160.49609375,40.59373638)
\curveto(159.77733958,39.91403896)(159.3632775,38.95700867)(159.25390625,37.72264263)
\lineto(166.94140625,37.73436138)
}
}
{
\newrgbcolor{curcolor}{0 0 0}
\pscustom[linestyle=none,fillstyle=solid,fillcolor=curcolor]
{
\newpath
\moveto(174.76953125,46.85154888)
\lineto(174.76953125,43.12498638)
\lineto(179.2109375,43.12498638)
\lineto(179.2109375,41.44920513)
\lineto(174.76953125,41.44920513)
\lineto(174.76953125,34.32420513)
\curveto(174.76952686,33.25388937)(174.91405796,32.56639006)(175.203125,32.26170513)
\curveto(175.49999487,31.95701567)(176.09765053,31.80467207)(176.99609375,31.80467388)
\lineto(179.2109375,31.80467388)
\lineto(179.2109375,29.99998638)
\lineto(176.99609375,29.99998638)
\curveto(175.33202629,29.99998638)(174.18358994,30.30857982)(173.55078125,30.92576763)
\curveto(172.91796621,31.55076608)(172.60156027,32.68357744)(172.6015625,34.32420513)
\lineto(172.6015625,41.44920513)
\lineto(171.01953125,41.44920513)
\lineto(171.01953125,43.12498638)
\lineto(172.6015625,43.12498638)
\lineto(172.6015625,46.85154888)
\lineto(174.76953125,46.85154888)
}
}
{
\newrgbcolor{curcolor}{0 0 0}
\pscustom[linestyle=none,fillstyle=solid,fillcolor=curcolor]
{
\newpath
\moveto(187.14453125,41.61326763)
\curveto(185.98827506,41.61325601)(185.07421347,41.16013147)(184.40234375,40.25389263)
\curveto(183.73046482,39.35544577)(183.39452765,38.12107201)(183.39453125,36.55076763)
\curveto(183.39452765,34.98045015)(183.72655857,33.74217013)(184.390625,32.83592388)
\curveto(185.06249473,31.93748444)(185.98046257,31.48826614)(187.14453125,31.48826763)
\curveto(188.29296025,31.48826614)(189.20311559,31.94139069)(189.875,32.84764263)
\curveto(190.54686425,33.75388887)(190.88280141,34.98826264)(190.8828125,36.55076763)
\curveto(190.88280141,38.10544702)(190.54686425,39.33591454)(189.875,40.24217388)
\curveto(189.20311559,41.15622522)(188.29296025,41.61325601)(187.14453125,41.61326763)
\moveto(187.14453125,43.44139263)
\curveto(189.01952203,43.44137919)(190.4921768,42.83200479)(191.5625,41.61326763)
\curveto(192.63279966,40.39450723)(193.16795538,38.70700892)(193.16796875,36.55076763)
\curveto(193.16795538,34.40232572)(192.63279966,32.71482741)(191.5625,31.48826763)
\curveto(190.4921768,30.26951736)(189.01952203,29.66014297)(187.14453125,29.66014263)
\curveto(185.26171329,29.66014297)(183.78515226,30.26951736)(182.71484375,31.48826763)
\curveto(181.65234189,32.71482741)(181.12109243,34.40232572)(181.12109375,36.55076763)
\curveto(181.12109243,38.70700892)(181.65234189,40.39450723)(182.71484375,41.61326763)
\curveto(183.78515226,42.83200479)(185.26171329,43.44137919)(187.14453125,43.44139263)
}
}
{
\newrgbcolor{curcolor}{0 0 0}
\pscustom[linestyle=none,fillstyle=solid,fillcolor=curcolor]
{
\newpath
\moveto(196.5078125,35.17967388)
\lineto(196.5078125,43.12498638)
\lineto(198.6640625,43.12498638)
\lineto(198.6640625,35.26170513)
\curveto(198.6640583,34.01951361)(198.90624556,33.08592079)(199.390625,32.46092388)
\curveto(199.87499459,31.84373453)(200.60155637,31.53514109)(201.5703125,31.53514263)
\curveto(202.73436673,31.53514109)(203.65233457,31.90623447)(204.32421875,32.64842388)
\curveto(205.00389571,33.39060799)(205.34373913,34.40232572)(205.34375,35.68358013)
\lineto(205.34375,43.12498638)
\lineto(207.5,43.12498638)
\lineto(207.5,29.99998638)
\lineto(205.34375,29.99998638)
\lineto(205.34375,32.01561138)
\curveto(204.82030215,31.21873516)(204.21092776,30.62498575)(203.515625,30.23436138)
\curveto(202.82811664,29.85154903)(202.02733619,29.66014297)(201.11328125,29.66014263)
\curveto(199.60546361,29.66014297)(198.46093351,30.1288925)(197.6796875,31.06639263)
\curveto(196.89843507,32.00389062)(196.50781046,33.374983)(196.5078125,35.17967388)
\moveto(201.93359375,43.44139263)
\lineto(201.93359375,43.44139263)
}
}
{
\newrgbcolor{curcolor}{0 0 0}
\pscustom[linestyle=none,fillstyle=solid,fillcolor=curcolor]
{
\newpath
\moveto(219.5703125,41.10936138)
\curveto(219.32811538,41.24997513)(219.06249064,41.35153753)(218.7734375,41.41404888)
\curveto(218.49217871,41.48434989)(218.17967902,41.51950611)(217.8359375,41.51951763)
\curveto(216.61718059,41.51950611)(215.67968152,41.12106901)(215.0234375,40.32420513)
\curveto(214.37499533,39.53513309)(214.0507769,38.39841548)(214.05078125,36.91404888)
\lineto(214.05078125,29.99998638)
\lineto(211.8828125,29.99998638)
\lineto(211.8828125,43.12498638)
\lineto(214.05078125,43.12498638)
\lineto(214.05078125,41.08592388)
\curveto(214.50390145,41.88278699)(215.09374461,42.47263015)(215.8203125,42.85545513)
\curveto(216.54686816,43.24606688)(217.42967977,43.44137919)(218.46875,43.44139263)
\curveto(218.61717859,43.44137919)(218.78124092,43.42966045)(218.9609375,43.40623638)
\curveto(219.14061556,43.39059799)(219.33983411,43.36325426)(219.55859375,43.32420513)
\lineto(219.5703125,41.10936138)
}
}
{
\newrgbcolor{curcolor}{0 0 0}
\pscustom[linestyle=none,fillstyle=solid,fillcolor=curcolor]
{
\newpath
\moveto(43.96875,532.65625164)
\lineto(49.125,532.65625164)
\lineto(49.125,550.45312664)
\lineto(43.515625,549.32812664)
\lineto(43.515625,552.20312664)
\lineto(49.09375,553.32812664)
\lineto(52.25,553.32812664)
\lineto(52.25,532.65625164)
\lineto(57.40625,532.65625164)
\lineto(57.40625,530.00000164)
\lineto(43.96875,530.00000164)
\lineto(43.96875,532.65625164)
}
}
{
\newrgbcolor{curcolor}{0 0 0}
\pscustom[linestyle=none,fillstyle=solid,fillcolor=curcolor]
{
\newpath
\moveto(506.140625,502.65625164)
\lineto(517.15625,502.65625164)
\lineto(517.15625,500.00000164)
\lineto(502.34375,500.00000164)
\lineto(502.34375,502.65625164)
\curveto(503.54166313,503.89583107)(505.17186983,505.55728775)(507.234375,507.64062664)
\curveto(509.30728236,509.7343669)(510.60936439,511.08332389)(511.140625,511.68750164)
\curveto(512.15102952,512.82290548)(512.85415381,513.78123785)(513.25,514.56250164)
\curveto(513.65623634,515.35415295)(513.85936114,516.13019384)(513.859375,516.89062664)
\curveto(513.85936114,518.13019184)(513.42186158,519.1406075)(512.546875,519.92187664)
\curveto(511.68227998,520.70310593)(510.55207278,521.09373054)(509.15625,521.09375164)
\curveto(508.1666585,521.09373054)(507.11978455,520.92185571)(506.015625,520.57812664)
\curveto(504.92187008,520.2343564)(503.74999625,519.71352359)(502.5,519.01562664)
\lineto(502.5,522.20312664)
\curveto(503.77082956,522.71352059)(504.95832838,523.09893687)(506.0625,523.35937664)
\curveto(507.1666595,523.61976968)(508.17707516,523.74997789)(509.09375,523.75000164)
\curveto(511.51040516,523.74997789)(513.43748656,523.14581182)(514.875,521.93750164)
\curveto(516.31248369,520.72914757)(517.03123297,519.11456585)(517.03125,517.09375164)
\curveto(517.03123297,516.13540217)(516.84894148,515.22394475)(516.484375,514.35937664)
\curveto(516.1301922,513.50519646)(515.47915119,512.49478081)(514.53125,511.32812664)
\curveto(514.27081906,511.02603228)(513.44269489,510.15103315)(512.046875,508.70312664)
\curveto(510.65103102,507.26561937)(508.68228298,505.24999639)(506.140625,502.65625164)
}
}
{
\newrgbcolor{curcolor}{0 0 0}
\pscustom[linestyle=none,fillstyle=solid,fillcolor=curcolor]
{
\newpath
\moveto(512.984375,422.57812664)
\curveto(514.49477717,422.25519771)(515.67185933,421.58332339)(516.515625,420.56250164)
\curveto(517.3697743,419.54165876)(517.7968572,418.28124335)(517.796875,416.78125164)
\curveto(517.7968572,414.47916382)(517.00519133,412.6979156)(515.421875,411.43750164)
\curveto(513.83852783,410.17708479)(511.58853008,409.54687709)(508.671875,409.54687664)
\curveto(507.69270064,409.54687709)(506.68228498,409.64583532)(505.640625,409.84375164)
\curveto(504.60937039,410.0312516)(503.54166313,410.31770965)(502.4375,410.70312664)
\lineto(502.4375,413.75000164)
\curveto(503.31249669,413.23958173)(504.27082906,412.85416545)(505.3125,412.59375164)
\curveto(506.35416031,412.33333264)(507.44270089,412.20312443)(508.578125,412.20312664)
\curveto(510.55728111,412.20312443)(512.06248794,412.59374904)(513.09375,413.37500164)
\curveto(514.13540253,414.15624748)(514.65623534,415.29166301)(514.65625,416.78125164)
\curveto(514.65623534,418.15624348)(514.17186083,419.22915907)(513.203125,420.00000164)
\curveto(512.24477942,420.78124085)(510.90623909,421.17186546)(509.1875,421.17187664)
\lineto(506.46875,421.17187664)
\lineto(506.46875,423.76562664)
\lineto(509.3125,423.76562664)
\curveto(510.86457247,423.76561287)(512.05207128,424.07290423)(512.875,424.68750164)
\curveto(513.69790297,425.31248632)(514.10936089,426.20831876)(514.109375,427.37500164)
\curveto(514.10936089,428.57289973)(513.68227798,429.48956548)(512.828125,430.12500164)
\curveto(511.98436302,430.7708142)(510.77082256,431.09373054)(509.1875,431.09375164)
\curveto(508.32290834,431.09373054)(507.39582594,430.99998064)(506.40625,430.81250164)
\curveto(505.41666125,430.62498101)(504.32812067,430.33331464)(503.140625,429.93750164)
\lineto(503.140625,432.75000164)
\curveto(504.33853733,433.08331189)(505.45832788,433.33331164)(506.5,433.50000164)
\curveto(507.55207578,433.66664464)(508.54165813,433.74997789)(509.46875,433.75000164)
\curveto(511.86457147,433.74997789)(513.76040291,433.20310343)(515.15625,432.10937664)
\curveto(516.55206678,431.02602228)(517.24998275,429.55727375)(517.25,427.70312664)
\curveto(517.24998275,426.41144356)(516.88019145,425.31769465)(516.140625,424.42187664)
\curveto(515.40102627,423.53644643)(514.34894398,422.92186371)(512.984375,422.57812664)
}
}
{
\newrgbcolor{curcolor}{0 0 0}
\pscustom[linestyle=none,fillstyle=solid,fillcolor=curcolor]
{
\newpath
\moveto(512.09375,370.57812664)
\lineto(504.125,358.12500164)
\lineto(512.09375,358.12500164)
\lineto(512.09375,370.57812664)
\moveto(511.265625,373.32812664)
\lineto(515.234375,373.32812664)
\lineto(515.234375,358.12500164)
\lineto(518.5625,358.12500164)
\lineto(518.5625,355.50000164)
\lineto(515.234375,355.50000164)
\lineto(515.234375,350.00000164)
\lineto(512.09375,350.00000164)
\lineto(512.09375,355.50000164)
\lineto(501.5625,355.50000164)
\lineto(501.5625,358.54687664)
\lineto(511.265625,373.32812664)
}
}
{
\newrgbcolor{curcolor}{0 0 0}
\pscustom[linestyle=none,fillstyle=solid,fillcolor=curcolor]
{
\newpath
\moveto(503.453125,323.32812664)
\lineto(515.84375,323.32812664)
\lineto(515.84375,320.67187664)
\lineto(506.34375,320.67187664)
\lineto(506.34375,314.95312664)
\curveto(506.80207653,315.10936153)(507.26040941,315.22394475)(507.71875,315.29687664)
\curveto(508.17707516,315.38019459)(508.63540803,315.42186121)(509.09375,315.42187664)
\curveto(511.69790497,315.42186121)(513.76040291,314.70832026)(515.28125,313.28125164)
\curveto(516.80206653,311.85415645)(517.56248244,309.92186671)(517.5625,307.48437664)
\curveto(517.56248244,304.973955)(516.78123322,303.02083195)(515.21875,301.62500164)
\curveto(513.65623634,300.23958473)(511.45311355,299.54687709)(508.609375,299.54687664)
\curveto(507.6302007,299.54687709)(506.6302017,299.63021034)(505.609375,299.79687664)
\curveto(504.59895373,299.96354334)(503.55207978,300.21354309)(502.46875,300.54687664)
\lineto(502.46875,303.71875164)
\curveto(503.40624659,303.20833176)(504.37499562,302.82812381)(505.375,302.57812664)
\curveto(506.37499363,302.32812431)(507.43228423,302.20312443)(508.546875,302.20312664)
\curveto(510.34894798,302.20312443)(511.77602989,302.67708229)(512.828125,303.62500164)
\curveto(513.88019445,304.57291373)(514.40623559,305.85937078)(514.40625,307.48437664)
\curveto(514.40623559,309.10936753)(513.88019445,310.39582457)(512.828125,311.34375164)
\curveto(511.77602989,312.29165601)(510.34894798,312.76561387)(508.546875,312.76562664)
\curveto(507.7031173,312.76561387)(506.85936814,312.67186396)(506.015625,312.48437664)
\curveto(505.18228648,312.29686434)(504.32812067,312.00519796)(503.453125,311.60937664)
\lineto(503.453125,323.32812664)
}
}
{
\newrgbcolor{curcolor}{0 0 0}
\pscustom[linestyle=none,fillstyle=solid,fillcolor=curcolor]
{
\newpath
\moveto(60.5625,42.92186138)
\curveto(59.14582419,42.92184846)(58.02082531,42.43747394)(57.1875,41.46873638)
\curveto(56.36457697,40.49997588)(55.95311905,39.17185221)(55.953125,37.48436138)
\curveto(55.95311905,35.80727224)(56.36457697,34.47914856)(57.1875,33.49998638)
\curveto(58.02082531,32.53123385)(59.14582419,32.04685933)(60.5625,32.04686138)
\curveto(61.97915469,32.04685933)(63.09894523,32.53123385)(63.921875,33.49998638)
\curveto(64.75519358,34.47914856)(65.17185983,35.80727224)(65.171875,37.48436138)
\curveto(65.17185983,39.17185221)(64.75519358,40.49997588)(63.921875,41.46873638)
\curveto(63.09894523,42.43747394)(61.97915469,42.92184846)(60.5625,42.92186138)
\moveto(66.828125,52.81248638)
\lineto(66.828125,49.93748638)
\curveto(66.0364423,50.31246606)(65.23435977,50.59892411)(64.421875,50.79686138)
\curveto(63.61977805,50.99475705)(62.82290384,51.09371528)(62.03125,51.09373638)
\curveto(59.94790672,51.09371528)(58.35415831,50.39059099)(57.25,48.98436138)
\curveto(56.15624384,47.5780938)(55.53124447,45.45309592)(55.375,42.60936138)
\curveto(55.98957734,43.51559786)(56.76040991,44.2083055)(57.6875,44.68748638)
\curveto(58.61457472,45.17705453)(59.63540703,45.42184596)(60.75,45.42186138)
\curveto(63.09373691,45.42184596)(64.94269339,44.708305)(66.296875,43.28123638)
\curveto(67.66144067,41.86455785)(68.34373166,39.93226811)(68.34375,37.48436138)
\curveto(68.34373166,35.08852296)(67.63539903,33.16664988)(66.21875,31.71873638)
\curveto(64.80206853,30.27081944)(62.91665375,29.54686183)(60.5625,29.54686138)
\curveto(57.86457547,29.54686183)(55.80207753,30.5781108)(54.375,32.64061138)
\curveto(52.94791372,34.71352333)(52.23437277,37.71352033)(52.234375,41.64061138)
\curveto(52.23437277,45.32809605)(53.10937189,48.26559311)(54.859375,50.45311138)
\curveto(56.60936839,52.65100539)(58.95832438,53.74996263)(61.90625,53.74998638)
\curveto(62.69790397,53.74996263)(63.49477817,53.67183771)(64.296875,53.51561138)
\curveto(65.10935989,53.35933802)(65.95310905,53.12496325)(66.828125,52.81248638)
}
}
{
\newrgbcolor{curcolor}{0 0 0}
\pscustom[linewidth=2,linecolor=curcolor,linestyle=dashed,dash=8 8]
{
\newpath
\moveto(150,540)
\lineto(60,540)
}
}
{
\newrgbcolor{curcolor}{0 0 0}
\pscustom[linestyle=none,fillstyle=solid,fillcolor=curcolor]
{
\newpath
\moveto(139.53769464,544.84048224)
\lineto(152.6487474,540.01921591)
\lineto(139.53769392,535.19795064)
\curveto(141.632292,538.04442372)(141.62022288,541.93889292)(139.53769464,544.84048224)
\lineto(139.53769464,544.84048224)
\lineto(139.53769464,544.84048224)
\closepath
}
}
{
\newrgbcolor{curcolor}{0 0 0}
\pscustom[linewidth=2,linecolor=curcolor,linestyle=dashed,dash=8 8]
{
\newpath
\moveto(420,420)
\lineto(500,420)
}
}
{
\newrgbcolor{curcolor}{0 0 0}
\pscustom[linestyle=none,fillstyle=solid,fillcolor=curcolor]
{
\newpath
\moveto(430.46230536,415.15951776)
\lineto(417.3512526,419.98078409)
\lineto(430.46230608,424.80204936)
\curveto(428.367708,421.95557628)(428.37977712,418.06110708)(430.46230536,415.15951776)
\lineto(430.46230536,415.15951776)
\lineto(430.46230536,415.15951776)
\closepath
}
}
{
\newrgbcolor{curcolor}{0 0 0}
\pscustom[linewidth=2,linecolor=curcolor,linestyle=dashed,dash=8 8]
{
\newpath
\moveto(400,360)
\lineto(500,360)
}
}
{
\newrgbcolor{curcolor}{0 0 0}
\pscustom[linestyle=none,fillstyle=solid,fillcolor=curcolor]
{
\newpath
\moveto(410.46230536,355.15951776)
\lineto(397.3512526,359.98078409)
\lineto(410.46230608,364.80204936)
\curveto(408.367708,361.95557628)(408.37977712,358.06110708)(410.46230536,355.15951776)
\lineto(410.46230536,355.15951776)
\lineto(410.46230536,355.15951776)
\closepath
}
}
{
\newrgbcolor{curcolor}{0 0 0}
\pscustom[linewidth=2,linecolor=curcolor,linestyle=dashed,dash=8 8]
{
\newpath
\moveto(359.96667,310.41667)
\lineto(499.96667,310.41667)
}
}
{
\newrgbcolor{curcolor}{0 0 0}
\pscustom[linestyle=none,fillstyle=solid,fillcolor=curcolor]
{
\newpath
\moveto(370.42897536,305.57618776)
\lineto(357.3179226,310.39745409)
\lineto(370.42897608,315.21871936)
\curveto(368.334378,312.37224628)(368.34644712,308.47777708)(370.42897536,305.57618776)
\lineto(370.42897536,305.57618776)
\lineto(370.42897536,305.57618776)
\closepath
}
}
{
\newrgbcolor{curcolor}{0 0 0}
\pscustom[linewidth=2,linecolor=curcolor,linestyle=dashed,dash=8 8]
{
\newpath
\moveto(140.07143,40)
\lineto(60.071429,40)
}
}
{
\newrgbcolor{curcolor}{0 0 0}
\pscustom[linewidth=2,linecolor=curcolor,linestyle=dashed,dash=8 8]
{
\newpath
\moveto(320,420)
\lineto(500,510)
}
}
{
\newrgbcolor{curcolor}{0 0 0}
\pscustom[linestyle=none,fillstyle=solid,fillcolor=curcolor]
{
\newpath
\moveto(331.52249986,420.34942626)
\lineto(317.63948192,418.79825692)
\lineto(327.21022928,428.97396904)
\curveto(326.60974526,425.49127338)(328.36219978,422.01335171)(331.52249986,420.34942626)
\lineto(331.52249986,420.34942626)
\lineto(331.52249986,420.34942626)
\closepath
}
}
\end{pspicture}

		
		\begin{enumerate}
		  \item Fond d'écran
		  \item Barre de volume
		  \item Check box ``Muet''
		  \item Liste déroulante ``Langues''
		  \item Bouton ``\hyperlink{Page d'accueil}{Gestion du profil}''
		  \item Bouton ``\hyperlink{Page d'accueil}{Retour}''
		\end{enumerate}

	
\newpage

	\subsection{Gestion du profil}
	
		\hypertarget{Gestion du profil}{}
		\label{Gestion du profil}
		
		\input{./img/10_Gestion_profil}
		
		\begin{enumerate}
		  \item Fond d'écran
		  \item Champs de texte ``Pseudo''
		  \item Liste déroulante ``Couleur joueur''
		  \item Champs de texte ``Nom d'utilisateur''
		  \item Bouton ``Modifier''
		  \item Liste déroulante ``Position menu de la partie''
		  \item Bouton ``\hyperlink{Options}{Valider}''
		  \item Bouton ``\hyperlink{Options}{Retour}''
		\end{enumerate}
	
\newpage

	\subsection{Mes statistiques}
	
		\input{./img/11_Mes_statistiques}
		
		\hypertarget{Statistiques}{}
		\label{Statistiques}
		
		\begin{enumerate}
		  \item Fond d'écran
		  \item Nom du compte hors ligne utilisé
		  \item Text Box ``Statistiques''
		  \item Bouton ``\hyperlink{Acceuil}{Retour}''
		\end{enumerate}
	
\newpage
	

\end{document}