\documentclass{report}

\usepackage[utf8]{inputenc}
\usepackage[frenchb]{babel}
\usepackage[T1]{fontenc}
\usepackage{lmodern}

\usepackage{a4wide}

\usepackage{pstricks}

\usepackage[colorlinks, linkcolor=blue]{hyperref}


\begin{document}

\chapter{Annexes}

	\section{Scenarios}
	
		Description de la partie \ldots

\newpage
	
	\subsection{Chargement de l'application}
	
		\hypertarget{Chargement de l'application}{}
		\label{Chargement de l'application}

		\begin{center}
			%LaTeX with PSTricks extensions
%%Creator: inkscape 0.48.0
%%Please note this file requires PSTricks extensions
\psset{xunit=.5pt,yunit=.5pt,runit=.5pt}
\begin{pspicture}(560,600)
{
\newrgbcolor{curcolor}{1 1 1}
\pscustom[linestyle=none,fillstyle=solid,fillcolor=curcolor]
{
\newpath
\moveto(133.12401716,597.5221417)
\lineto(426.87598554,597.5221417)
\curveto(443.85397304,597.5221417)(457.52217237,583.85394237)(457.52217237,566.87595487)
\lineto(457.52217237,33.124017)
\curveto(457.52217237,16.1460295)(443.85397304,2.47783017)(426.87598554,2.47783017)
\lineto(133.12401716,2.47783017)
\curveto(116.14602965,2.47783017)(102.47783033,16.1460295)(102.47783033,33.124017)
\lineto(102.47783033,566.87595487)
\curveto(102.47783033,583.85394237)(116.14602965,597.5221417)(133.12401716,597.5221417)
\closepath
}
}
{
\newrgbcolor{curcolor}{0 0 0}
\pscustom[linewidth=4.95566034,linecolor=curcolor]
{
\newpath
\moveto(133.12401716,597.5221417)
\lineto(426.87598554,597.5221417)
\curveto(443.85397304,597.5221417)(457.52217237,583.85394237)(457.52217237,566.87595487)
\lineto(457.52217237,33.124017)
\curveto(457.52217237,16.1460295)(443.85397304,2.47783017)(426.87598554,2.47783017)
\lineto(133.12401716,2.47783017)
\curveto(116.14602965,2.47783017)(102.47783033,16.1460295)(102.47783033,33.124017)
\lineto(102.47783033,566.87595487)
\curveto(102.47783033,583.85394237)(116.14602965,597.5221417)(133.12401716,597.5221417)
\closepath
}
}
{
\newrgbcolor{curcolor}{1 1 1}
\pscustom[linestyle=none,fillstyle=solid,fillcolor=curcolor]
{
\newpath
\moveto(260.00365964,158.92373777)
\lineto(309.96551649,158.92373777)
\lineto(309.96551649,110.05766797)
\lineto(260.00365964,110.05766797)
\closepath
}
}
{
\newrgbcolor{curcolor}{0 0 0}
\pscustom[linewidth=2,linecolor=curcolor]
{
\newpath
\moveto(260.00365964,158.92373777)
\lineto(309.96551649,158.92373777)
\lineto(309.96551649,110.05766797)
\lineto(260.00365964,110.05766797)
\closepath
}
}
{
\newrgbcolor{curcolor}{0 0 0}
\pscustom[linestyle=none,fillstyle=solid,fillcolor=curcolor]
{
\newpath
\moveto(43.96874944,502.65625119)
\lineto(49.12499944,502.65625119)
\lineto(49.12499944,520.45312619)
\lineto(43.51562444,519.32812619)
\lineto(43.51562444,522.20312619)
\lineto(49.09374944,523.32812619)
\lineto(52.24999944,523.32812619)
\lineto(52.24999944,502.65625119)
\lineto(57.40624944,502.65625119)
\lineto(57.40624944,500.00000119)
\lineto(43.96874944,500.00000119)
\lineto(43.96874944,502.65625119)
}
}
{
\newrgbcolor{curcolor}{0 0 0}
\pscustom[linestyle=none,fillstyle=solid,fillcolor=curcolor]
{
\newpath
\moveto(506.14062254,122.65622068)
\lineto(517.15624754,122.65622068)
\lineto(517.15624754,119.99997068)
\lineto(502.34374754,119.99997068)
\lineto(502.34374754,122.65622068)
\curveto(503.54166066,123.89580011)(505.17186736,125.55725679)(507.23437254,127.64059568)
\curveto(509.30727989,129.73433594)(510.60936193,131.08329293)(511.14062254,131.68747068)
\curveto(512.15102705,132.82287452)(512.85415135,133.78120689)(513.24999754,134.56247068)
\curveto(513.65623388,135.35412199)(513.85935868,136.13016288)(513.85937254,136.89059568)
\curveto(513.85935868,138.13016088)(513.42185911,139.14057654)(512.54687254,139.92184568)
\curveto(511.68227752,140.70307497)(510.55207032,141.09369958)(509.15624754,141.09372068)
\curveto(508.16665604,141.09369958)(507.11978208,140.92182475)(506.01562254,140.57809568)
\curveto(504.92186761,140.23432544)(503.74999379,139.71349263)(502.49999754,139.01559568)
\lineto(502.49999754,142.20309568)
\curveto(503.7708271,142.71348963)(504.95832591,143.09890591)(506.06249754,143.35934568)
\curveto(507.16665704,143.61973872)(508.17707269,143.74994693)(509.09374754,143.74997068)
\curveto(511.51040269,143.74994693)(513.4374841,143.14578086)(514.87499754,141.93747068)
\curveto(516.31248122,140.72911661)(517.0312305,139.11453489)(517.03124754,137.09372068)
\curveto(517.0312305,136.13537121)(516.84893902,135.22391379)(516.48437254,134.35934568)
\curveto(516.13018974,133.5051655)(515.47914872,132.49474985)(514.53124754,131.32809568)
\curveto(514.2708166,131.02600132)(513.44269243,130.15100219)(512.04687254,128.70309568)
\curveto(510.65102855,127.26558841)(508.68228052,125.24996543)(506.14062254,122.65622068)
}
}
{
\newrgbcolor{curcolor}{0 0 0}
\pscustom[linewidth=2,linecolor=curcolor,linestyle=dashed,dash=8 8]
{
\newpath
\moveto(150,510)
\lineto(60,510)
}
}
{
\newrgbcolor{curcolor}{0 0 0}
\pscustom[linestyle=none,fillstyle=solid,fillcolor=curcolor]
{
\newpath
\moveto(139.53769464,514.84048224)
\lineto(152.6487474,510.01921591)
\lineto(139.53769392,505.19795064)
\curveto(141.632292,508.04442372)(141.62022288,511.93889292)(139.53769464,514.84048224)
\lineto(139.53769464,514.84048224)
\closepath
}
}
{
\newrgbcolor{curcolor}{0 0 0}
\pscustom[linewidth=2,linecolor=curcolor,linestyle=dashed,dash=8 8]
{
\newpath
\moveto(290,130)
\lineto(500,130)
}
}
{
\newrgbcolor{curcolor}{0 0 0}
\pscustom[linestyle=none,fillstyle=solid,fillcolor=curcolor]
{
\newpath
\moveto(300.46230536,125.15951776)
\lineto(287.3512526,129.98078409)
\lineto(300.46230608,134.80204936)
\curveto(298.367708,131.95557628)(298.37977712,128.06110708)(300.46230536,125.15951776)
\lineto(300.46230536,125.15951776)
\closepath
}
}
\end{pspicture}

		\end{center}
		
		\begin{enumerate}
		  \item Fond d'écran (présent dans l'archive stocké sur le téléphone)
		  \item Spinner (présent dans l'archive stocké sur le téléphone)
		\end{enumerate}
		
		\subsubsection{Description des zones}

			\begin{tabular}{|c|c|c|c|c|} \hline
				Numéro de zone & Type  & Description & Evènement &	Règle \\\hline 
				2 & Barre de & Montre l'avancement du & Chargement de & RG0-01\\
				  & chargement & chargement de l'application & l'application &\\\hline
			\end{tabular}
		
		\subsubsection{Description des règles}
		Lancement de l’application 
		
		\underline{RG0-01 :}
		\begin{quote}
			Déclenchée au démarrage de l'application.\\
			Afficher le spinner montrant le chargement de l'application.\\
			\underline{Chargement de l'application :}
				\begin{quote}
					Vérification d'un compte local existant (stocké sur le téléphone).\\
					Si aucun compte n'est trouvé alors 
					\begin{quote}
						Créer la page de création du profil%
						\footnote[1]{
							\hyperlink{Création du profil}{Page de création du profil}
							\og voir section \ref{profil}, page \pageref{profil}.\fg
						}.\\
						Afficher la page de création du profil\footnotemark[1].
					\end{quote}
					Sinon
					\begin{quote}
						Créer la classe utilisateur.\\
						Mettre à jour les informations de la classe utilisateur.\\
						Charger la page d'accueil%
						\footnote[2]{
							\hyperlink{Page d'accueil}{Page d'accueil}
							\og voir section \ref{Accueil}, page \pageref{Accueil}.\fg
						}.\\
						Afficher la page d'accueil\footnotemark[2].
					\end{quote}
					Supprimer la page de chargement de l'application.
				\end{quote}
		\end{quote}

	
\newpage

	\subsection{Création du profil}
		\hypertarget{profil}{}
		\label{profil}

		\begin{center}
			%LaTeX with PSTricks extensions
%%Creator: inkscape 0.48.0
%%Please note this file requires PSTricks extensions
\psset{xunit=.4pt,yunit=.4pt,runit=.4pt}
\begin{pspicture}(560,600)
{
\newrgbcolor{curcolor}{1 1 1}
\pscustom[linestyle=none,fillstyle=solid,fillcolor=curcolor]
{
\newpath
\moveto(133.17790741,598.5302021)
\lineto(426.82210296,598.5302021)
\curveto(443.82879388,598.5302021)(457.52010101,584.83889497)(457.52010101,567.83220405)
\lineto(457.52010101,33.17790718)
\curveto(457.52010101,16.17121626)(443.82879388,2.47990913)(426.82210296,2.47990913)
\lineto(133.17790741,2.47990913)
\curveto(116.17121649,2.47990913)(102.47990936,16.17121626)(102.47990936,33.17790718)
\lineto(102.47990936,567.83220405)
\curveto(102.47990936,584.83889497)(116.17121649,598.5302021)(133.17790741,598.5302021)
\closepath
}
}
{
\newrgbcolor{curcolor}{0 0 0}
\pscustom[linewidth=4.95981836,linecolor=curcolor]
{
\newpath
\moveto(133.17790741,598.5302021)
\lineto(426.82210296,598.5302021)
\curveto(443.82879388,598.5302021)(457.52010101,584.83889497)(457.52010101,567.83220405)
\lineto(457.52010101,33.17790718)
\curveto(457.52010101,16.17121626)(443.82879388,2.47990913)(426.82210296,2.47990913)
\lineto(133.17790741,2.47990913)
\curveto(116.17121649,2.47990913)(102.47990936,16.17121626)(102.47990936,33.17790718)
\lineto(102.47990936,567.83220405)
\curveto(102.47990936,584.83889497)(116.17121649,598.5302021)(133.17790741,598.5302021)
\closepath
}
}
{
\newrgbcolor{curcolor}{1 1 1}
\pscustom[linestyle=none,fillstyle=solid,fillcolor=curcolor]
{
\newpath
\moveto(164.38434976,370.09907455)
\lineto(396.60310358,370.09907455)
\curveto(409.61615135,370.09907455)(420.09235948,359.62286642)(420.09235948,346.60981865)
\lineto(420.09235948,243.48925514)
\curveto(420.09235948,230.47620737)(409.61615135,219.99999924)(396.60310358,219.99999924)
\lineto(164.38434976,219.99999924)
\curveto(151.37130199,219.99999924)(140.89509386,230.47620737)(140.89509386,243.48925514)
\lineto(140.89509386,346.60981865)
\curveto(140.89509386,359.62286642)(151.37130199,370.09907455)(164.38434976,370.09907455)
\closepath
}
}
{
\newrgbcolor{curcolor}{0 0 0}
\pscustom[linewidth=1.75071895,linecolor=curcolor]
{
\newpath
\moveto(164.38434976,370.09907455)
\lineto(396.60310358,370.09907455)
\curveto(409.61615135,370.09907455)(420.09235948,359.62286642)(420.09235948,346.60981865)
\lineto(420.09235948,243.48925514)
\curveto(420.09235948,230.47620737)(409.61615135,219.99999924)(396.60310358,219.99999924)
\lineto(164.38434976,219.99999924)
\curveto(151.37130199,219.99999924)(140.89509386,230.47620737)(140.89509386,243.48925514)
\lineto(140.89509386,346.60981865)
\curveto(140.89509386,359.62286642)(151.37130199,370.09907455)(164.38434976,370.09907455)
\closepath
}
}
{
\newrgbcolor{curcolor}{0 0 0}
\pscustom[linestyle=none,fillstyle=solid,fillcolor=curcolor]
{
\newpath
\moveto(164.7226581,345.55079574)
\lineto(164.7226581,338.97657699)
\lineto(167.6992206,338.97657699)
\curveto(168.8007743,338.97656802)(169.65233595,339.26172398)(170.2539081,339.83204574)
\curveto(170.85545974,340.40234784)(171.15624069,341.21484703)(171.15625185,342.26954574)
\curveto(171.15624069,343.31640743)(170.85545974,344.12500037)(170.2539081,344.69532699)
\curveto(169.65233595,345.26562423)(168.8007743,345.55078019)(167.6992206,345.55079574)
\lineto(164.7226581,345.55079574)
\moveto(162.3554706,347.49610824)
\lineto(167.6992206,347.49610824)
\curveto(169.66014844,347.49609075)(171.14061571,347.05077869)(172.14062685,346.16017074)
\curveto(173.1484262,345.27734297)(173.65233195,343.98046926)(173.6523456,342.26954574)
\curveto(173.65233195,340.5429727)(173.1484262,339.23828651)(172.14062685,338.35548324)
\curveto(171.14061571,337.47266327)(169.66014844,337.03125746)(167.6992206,337.03126449)
\lineto(164.7226581,337.03126449)
\lineto(164.7226581,330.00001449)
\lineto(162.3554706,330.00001449)
\lineto(162.3554706,347.49610824)
}
}
{
\newrgbcolor{curcolor}{0 0 0}
\pscustom[linestyle=none,fillstyle=solid,fillcolor=curcolor]
{
\newpath
\moveto(184.6914081,342.73829574)
\lineto(184.6914081,340.69923324)
\curveto(184.08202308,341.01172223)(183.44921121,341.246097)(182.7929706,341.40235824)
\curveto(182.13671252,341.55859669)(181.4570257,341.63672161)(180.7539081,341.63673324)
\curveto(179.68358998,341.63672161)(178.87890328,341.47265927)(178.3398456,341.14454574)
\curveto(177.80859185,340.81640993)(177.54296712,340.32422292)(177.5429706,339.66798324)
\curveto(177.54296712,339.16797408)(177.73437318,338.77344322)(178.11718935,338.48438949)
\curveto(178.49999741,338.20313129)(179.26952789,337.93360031)(180.4257831,337.67579574)
\lineto(181.16406435,337.51173324)
\curveto(182.69530571,337.18360106)(183.78124213,336.71875778)(184.42187685,336.11720199)
\curveto(185.07030334,335.52344647)(185.39452177,334.69141605)(185.3945331,333.62110824)
\curveto(185.39452177,332.40235584)(184.91014725,331.43751306)(183.9414081,330.72657699)
\curveto(182.98046168,330.01563948)(181.65624425,329.66017108)(179.96875185,329.66017074)
\curveto(179.26562164,329.66017108)(178.53124738,329.73048351)(177.76562685,329.87110824)
\curveto(177.0078114,330.00392074)(176.20703095,330.20704554)(175.3632831,330.48048324)
\lineto(175.3632831,332.70704574)
\curveto(176.160156,332.29298095)(176.94531146,331.98048126)(177.71875185,331.76954574)
\curveto(178.49218492,331.56641918)(179.25780915,331.46485678)(180.01562685,331.46485824)
\curveto(181.03124488,331.46485678)(181.8124941,331.63673161)(182.35937685,331.98048324)
\curveto(182.906243,332.33204341)(183.17968023,332.82423042)(183.17968935,333.45704574)
\curveto(183.17968023,334.0429792)(182.98046168,334.4921975)(182.5820331,334.80470199)
\curveto(182.19139997,335.11719688)(181.32811958,335.41797783)(179.99218935,335.70704574)
\lineto(179.24218935,335.88282699)
\curveto(177.906248,336.16407083)(176.94140522,336.5937579)(176.3476581,337.17188949)
\curveto(175.75390641,337.75781924)(175.4570317,338.55859969)(175.4570331,339.57423324)
\curveto(175.4570317,340.80859744)(175.89453127,341.76172148)(176.7695331,342.43360824)
\curveto(177.64452952,343.10547014)(178.88671577,343.4414073)(180.4960956,343.44142074)
\curveto(181.29296337,343.4414073)(182.04296262,343.38281361)(182.7460956,343.26563949)
\curveto(183.44921121,343.14843885)(184.09764806,342.97265777)(184.6914081,342.73829574)
}
}
{
\newrgbcolor{curcolor}{0 0 0}
\pscustom[linestyle=none,fillstyle=solid,fillcolor=curcolor]
{
\newpath
\moveto(200.0664081,337.10157699)
\lineto(200.0664081,336.04688949)
\lineto(190.1523456,336.04688949)
\curveto(190.24609193,334.56250993)(190.69140398,333.42969856)(191.4882831,332.64845199)
\curveto(192.29296488,331.87501262)(193.41015127,331.48829426)(194.8398456,331.48829574)
\curveto(195.66796151,331.48829426)(196.46874196,331.58985665)(197.24218935,331.79298324)
\curveto(198.0234279,331.99610625)(198.79686463,332.30079344)(199.56250185,332.70704574)
\lineto(199.56250185,330.66798324)
\curveto(198.78905214,330.3398579)(197.99608418,330.08985815)(197.1835956,329.91798324)
\curveto(196.3710858,329.7461085)(195.54686788,329.66017108)(194.71093935,329.66017074)
\curveto(192.61718331,329.66017108)(190.95702872,330.26954547)(189.7304706,331.48829574)
\curveto(188.51171866,332.70704304)(187.90234427,334.35547889)(187.9023456,336.43360824)
\curveto(187.90234427,338.58203716)(188.4804687,340.28516046)(189.6367206,341.54298324)
\curveto(190.80077887,342.80859544)(192.36718356,343.4414073)(194.33593935,343.44142074)
\curveto(196.10155482,343.4414073)(197.49608468,342.87109537)(198.5195331,341.73048324)
\curveto(199.55077012,340.59766015)(200.06639461,339.05469294)(200.0664081,337.10157699)
\moveto(197.9101581,337.73438949)
\curveto(197.89452178,338.91406808)(197.56249086,339.85547339)(196.91406435,340.55860824)
\curveto(196.27342965,341.26172198)(195.421868,341.61328413)(194.35937685,341.61329574)
\curveto(193.15624527,341.61328413)(192.19140248,341.27344072)(191.4648456,340.59376449)
\curveto(190.74609143,339.91406708)(190.33202934,338.95703679)(190.2226581,337.72267074)
\lineto(197.9101581,337.73438949)
}
}
{
\newrgbcolor{curcolor}{0 0 0}
\pscustom[linestyle=none,fillstyle=solid,fillcolor=curcolor]
{
\newpath
\moveto(203.38281435,335.17970199)
\lineto(203.38281435,343.12501449)
\lineto(205.53906435,343.12501449)
\lineto(205.53906435,335.26173324)
\curveto(205.53906015,334.01954172)(205.78124741,333.08594891)(206.26562685,332.46095199)
\curveto(206.74999644,331.84376265)(207.47655821,331.53516921)(208.44531435,331.53517074)
\curveto(209.60936858,331.53516921)(210.52733641,331.90626259)(211.1992206,332.64845199)
\curveto(211.87889756,333.3906361)(212.21874097,334.40235384)(212.21875185,335.68360824)
\lineto(212.21875185,343.12501449)
\lineto(214.37500185,343.12501449)
\lineto(214.37500185,330.00001449)
\lineto(212.21875185,330.00001449)
\lineto(212.21875185,332.01563949)
\curveto(211.695304,331.21876328)(211.08592961,330.62501387)(210.39062685,330.23438949)
\curveto(209.70311849,329.85157714)(208.90233804,329.66017108)(207.9882831,329.66017074)
\curveto(206.48046546,329.66017108)(205.33593536,330.12892062)(204.55468935,331.06642074)
\curveto(203.77343692,332.00391874)(203.38281231,333.37501112)(203.38281435,335.17970199)
\moveto(208.8085956,343.44142074)
\lineto(208.8085956,343.44142074)
}
}
{
\newrgbcolor{curcolor}{0 0 0}
\pscustom[linestyle=none,fillstyle=solid,fillcolor=curcolor]
{
\newpath
\moveto(227.47656435,341.13282699)
\lineto(227.47656435,348.23438949)
\lineto(229.63281435,348.23438949)
\lineto(229.63281435,330.00001449)
\lineto(227.47656435,330.00001449)
\lineto(227.47656435,331.96876449)
\curveto(227.0234289,331.18751331)(226.44921073,330.60548264)(225.7539081,330.22267074)
\curveto(225.06639961,329.8476709)(224.23827544,329.66017108)(223.2695331,329.66017074)
\curveto(221.68359049,329.66017108)(220.39062304,330.29298295)(219.39062685,331.55860824)
\curveto(218.39843753,332.82423042)(217.90234427,334.48829126)(217.9023456,336.55079574)
\curveto(217.90234427,338.61328713)(218.39843753,340.27734797)(219.39062685,341.54298324)
\curveto(220.39062304,342.80859544)(221.68359049,343.4414073)(223.2695331,343.44142074)
\curveto(224.23827544,343.4414073)(225.06639961,343.25000124)(225.7539081,342.86720199)
\curveto(226.44921073,342.4921895)(227.0234289,341.91406508)(227.47656435,341.13282699)
\moveto(220.1289081,336.55079574)
\curveto(220.12890455,334.96485328)(220.45312297,333.71876078)(221.10156435,332.81251449)
\curveto(221.75780917,331.91407508)(222.65624577,331.46485678)(223.79687685,331.46485824)
\curveto(224.93749349,331.46485678)(225.83593009,331.91407508)(226.49218935,332.81251449)
\curveto(227.14842878,333.71876078)(227.47655345,334.96485328)(227.47656435,336.55079574)
\curveto(227.47655345,338.13672511)(227.14842878,339.37891137)(226.49218935,340.27735824)
\curveto(225.83593009,341.18359706)(224.93749349,341.63672161)(223.79687685,341.63673324)
\curveto(222.65624577,341.63672161)(221.75780917,341.18359706)(221.10156435,340.27735824)
\curveto(220.45312297,339.37891137)(220.12890455,338.13672511)(220.1289081,336.55079574)
}
}
{
\newrgbcolor{curcolor}{0 0 0}
\pscustom[linestyle=none,fillstyle=solid,fillcolor=curcolor]
{
\newpath
\moveto(239.1601581,341.61329574)
\curveto(238.00390191,341.61328413)(237.08984032,341.16015958)(236.4179706,340.25392074)
\curveto(235.74609166,339.35547389)(235.4101545,338.12110012)(235.4101581,336.55079574)
\curveto(235.4101545,334.98047826)(235.74218542,333.74219825)(236.40625185,332.83595199)
\curveto(237.07812158,331.93751256)(237.99608941,331.48829426)(239.1601581,331.48829574)
\curveto(240.3085871,331.48829426)(241.21874244,331.9414188)(241.89062685,332.84767074)
\curveto(242.5624911,333.75391699)(242.89842826,334.98829076)(242.89843935,336.55079574)
\curveto(242.89842826,338.10547514)(242.5624911,339.33594266)(241.89062685,340.24220199)
\curveto(241.21874244,341.15625334)(240.3085871,341.61328413)(239.1601581,341.61329574)
\moveto(239.1601581,343.44142074)
\curveto(241.03514888,343.4414073)(242.50780365,342.83203291)(243.57812685,341.61329574)
\curveto(244.64842651,340.39453535)(245.18358223,338.70703704)(245.1835956,336.55079574)
\curveto(245.18358223,334.40235384)(244.64842651,332.71485553)(243.57812685,331.48829574)
\curveto(242.50780365,330.26954547)(241.03514888,329.66017108)(239.1601581,329.66017074)
\curveto(237.27734013,329.66017108)(235.80077911,330.26954547)(234.7304706,331.48829574)
\curveto(233.66796874,332.71485553)(233.13671927,334.40235384)(233.1367206,336.55079574)
\curveto(233.13671927,338.70703704)(233.66796874,340.39453535)(234.7304706,341.61329574)
\curveto(235.80077911,342.83203291)(237.27734013,343.4414073)(239.1601581,343.44142074)
}
}
{
\newrgbcolor{curcolor}{0 0 0}
\pscustom[linestyle=none,fillstyle=solid,fillcolor=curcolor,opacity=0.15686275]
{
\newpath
\moveto(265.28100675,349.97837753)
\lineto(384.74003881,349.97837753)
\curveto(393.21111723,349.97837753)(400.03079408,346.49264783)(400.03079408,342.16284008)
\lineto(400.03079408,327.93718166)
\curveto(400.03079408,323.60737391)(393.21111723,320.12164421)(384.74003881,320.12164421)
\lineto(265.28100675,320.12164421)
\curveto(256.80992833,320.12164421)(249.99025148,323.60737391)(249.99025148,327.93718166)
\lineto(249.99025148,342.16284008)
\curveto(249.99025148,346.49264783)(256.80992833,349.97837753)(265.28100675,349.97837753)
\closepath
}
}
{
\newrgbcolor{curcolor}{0 0 0}
\pscustom[linewidth=2,linecolor=curcolor]
{
\newpath
\moveto(265.28100675,349.97837753)
\lineto(384.74003881,349.97837753)
\curveto(393.21111723,349.97837753)(400.03079408,346.49264783)(400.03079408,342.16284008)
\lineto(400.03079408,327.93718166)
\curveto(400.03079408,323.60737391)(393.21111723,320.12164421)(384.74003881,320.12164421)
\lineto(265.28100675,320.12164421)
\curveto(256.80992833,320.12164421)(249.99025148,323.60737391)(249.99025148,327.93718166)
\lineto(249.99025148,342.16284008)
\curveto(249.99025148,346.49264783)(256.80992833,349.97837753)(265.28100675,349.97837753)
\closepath
}
}
{
\newrgbcolor{curcolor}{1 1 1}
\pscustom[linestyle=none,fillstyle=solid,fillcolor=curcolor]
{
\newpath
\moveto(234.64676946,269.59635849)
\lineto(325.16059202,269.59635849)
\curveto(333.3281575,269.59635849)(339.90349001,263.02102599)(339.90349001,254.8534605)
\curveto(339.90349001,246.68589502)(333.3281575,240.11056251)(325.16059202,240.11056251)
\lineto(234.64676946,240.11056251)
\curveto(226.47920398,240.11056251)(219.90387148,246.68589502)(219.90387148,254.8534605)
\curveto(219.90387148,263.02102599)(226.47920398,269.59635849)(234.64676946,269.59635849)
\closepath
}
}
{
\newrgbcolor{curcolor}{0 0 0}
\pscustom[linewidth=2,linecolor=curcolor]
{
\newpath
\moveto(234.64676946,269.59635849)
\lineto(325.16059202,269.59635849)
\curveto(333.3281575,269.59635849)(339.90349001,263.02102599)(339.90349001,254.8534605)
\curveto(339.90349001,246.68589502)(333.3281575,240.11056251)(325.16059202,240.11056251)
\lineto(234.64676946,240.11056251)
\curveto(226.47920398,240.11056251)(219.90387148,246.68589502)(219.90387148,254.8534605)
\curveto(219.90387148,263.02102599)(226.47920398,269.59635849)(234.64676946,269.59635849)
\closepath
}
}
{
\newrgbcolor{curcolor}{0 0 0}
\pscustom[linestyle=none,fillstyle=solid,fillcolor=curcolor]
{
\newpath
\moveto(246.86718935,250.00001449)
\lineto(240.18750185,267.49610824)
\lineto(242.6601581,267.49610824)
\lineto(248.20312685,252.76563949)
\lineto(253.75781435,267.49610824)
\lineto(256.21875185,267.49610824)
\lineto(249.5507831,250.00001449)
\lineto(246.86718935,250.00001449)
}
}
{
\newrgbcolor{curcolor}{0 0 0}
\pscustom[linestyle=none,fillstyle=solid,fillcolor=curcolor]
{
\newpath
\moveto(262.75781435,256.59767074)
\curveto(261.01562036,256.59766415)(259.80859032,256.3984456)(259.1367206,256.00001449)
\curveto(258.46484166,255.60157139)(258.1289045,254.92188457)(258.1289081,253.96095199)
\curveto(258.1289045,253.1953238)(258.37890425,252.58594941)(258.8789081,252.13282699)
\curveto(259.38671574,251.68751281)(260.07421505,251.46485678)(260.9414081,251.46485824)
\curveto(262.13671299,251.46485678)(263.09374329,251.88673136)(263.81250185,252.73048324)
\curveto(264.53905434,253.58204216)(264.90233523,254.71094728)(264.9023456,256.11720199)
\lineto(264.9023456,256.59767074)
\lineto(262.75781435,256.59767074)
\moveto(267.0585956,257.48829574)
\lineto(267.0585956,250.00001449)
\lineto(264.9023456,250.00001449)
\lineto(264.9023456,251.99220199)
\curveto(264.41014822,251.1953258)(263.79686758,250.60548264)(263.06250185,250.22267074)
\curveto(262.32811905,249.8476709)(261.42968245,249.66017108)(260.36718935,249.66017074)
\curveto(259.02343486,249.66017108)(257.95312343,250.03517071)(257.15625185,250.78517074)
\curveto(256.36718751,251.5429817)(255.97265666,252.55469944)(255.9726581,253.82032699)
\curveto(255.97265666,255.2968842)(256.46484366,256.41016433)(257.4492206,257.16017074)
\curveto(258.44140419,257.91016283)(259.91796521,258.28516246)(261.8789081,258.28517074)
\lineto(264.9023456,258.28517074)
\lineto(264.9023456,258.49610824)
\curveto(264.90233523,259.48828626)(264.57421055,260.25391049)(263.9179706,260.79298324)
\curveto(263.26952436,261.3398469)(262.35546277,261.61328413)(261.1757831,261.61329574)
\curveto(260.4257772,261.61328413)(259.69530918,261.52344047)(258.98437685,261.34376449)
\curveto(258.27343561,261.16406583)(257.58984254,260.89453485)(256.9335956,260.53517074)
\lineto(256.9335956,262.52735824)
\curveto(257.72265491,262.83203291)(258.48827914,263.05859519)(259.2304706,263.20704574)
\curveto(259.97265266,263.36328238)(260.69530818,263.4414073)(261.39843935,263.44142074)
\curveto(263.29686808,263.4414073)(264.71483541,262.94922029)(265.6523456,261.96485824)
\curveto(266.58983354,260.98047226)(267.05858307,259.48828626)(267.0585956,257.48829574)
}
}
{
\newrgbcolor{curcolor}{0 0 0}
\pscustom[linestyle=none,fillstyle=solid,fillcolor=curcolor]
{
\newpath
\moveto(271.5117206,268.23438949)
\lineto(273.6679706,268.23438949)
\lineto(273.6679706,250.00001449)
\lineto(271.5117206,250.00001449)
\lineto(271.5117206,268.23438949)
}
}
{
\newrgbcolor{curcolor}{0 0 0}
\pscustom[linestyle=none,fillstyle=solid,fillcolor=curcolor]
{
\newpath
\moveto(278.1679706,263.12501449)
\lineto(280.3242206,263.12501449)
\lineto(280.3242206,250.00001449)
\lineto(278.1679706,250.00001449)
\lineto(278.1679706,263.12501449)
\moveto(278.1679706,268.23438949)
\lineto(280.3242206,268.23438949)
\lineto(280.3242206,265.50392074)
\lineto(278.1679706,265.50392074)
\lineto(278.1679706,268.23438949)
}
}
{
\newrgbcolor{curcolor}{0 0 0}
\pscustom[linestyle=none,fillstyle=solid,fillcolor=curcolor]
{
\newpath
\moveto(293.46093935,261.13282699)
\lineto(293.46093935,268.23438949)
\lineto(295.61718935,268.23438949)
\lineto(295.61718935,250.00001449)
\lineto(293.46093935,250.00001449)
\lineto(293.46093935,251.96876449)
\curveto(293.0078039,251.18751331)(292.43358573,250.60548264)(291.7382831,250.22267074)
\curveto(291.05077461,249.8476709)(290.22265044,249.66017108)(289.2539081,249.66017074)
\curveto(287.66796549,249.66017108)(286.37499804,250.29298295)(285.37500185,251.55860824)
\curveto(284.38281253,252.82423042)(283.88671927,254.48829126)(283.8867206,256.55079574)
\curveto(283.88671927,258.61328713)(284.38281253,260.27734797)(285.37500185,261.54298324)
\curveto(286.37499804,262.80859544)(287.66796549,263.4414073)(289.2539081,263.44142074)
\curveto(290.22265044,263.4414073)(291.05077461,263.25000124)(291.7382831,262.86720199)
\curveto(292.43358573,262.4921895)(293.0078039,261.91406508)(293.46093935,261.13282699)
\moveto(286.1132831,256.55079574)
\curveto(286.11327955,254.96485328)(286.43749797,253.71876078)(287.08593935,252.81251449)
\curveto(287.74218417,251.91407508)(288.64062077,251.46485678)(289.78125185,251.46485824)
\curveto(290.92186849,251.46485678)(291.82030509,251.91407508)(292.47656435,252.81251449)
\curveto(293.13280378,253.71876078)(293.46092845,254.96485328)(293.46093935,256.55079574)
\curveto(293.46092845,258.13672511)(293.13280378,259.37891137)(292.47656435,260.27735824)
\curveto(291.82030509,261.18359706)(290.92186849,261.63672161)(289.78125185,261.63673324)
\curveto(288.64062077,261.63672161)(287.74218417,261.18359706)(287.08593935,260.27735824)
\curveto(286.43749797,259.37891137)(286.11327955,258.13672511)(286.1132831,256.55079574)
}
}
{
\newrgbcolor{curcolor}{0 0 0}
\pscustom[linestyle=none,fillstyle=solid,fillcolor=curcolor]
{
\newpath
\moveto(311.2851581,257.10157699)
\lineto(311.2851581,256.04688949)
\lineto(301.3710956,256.04688949)
\curveto(301.46484193,254.56250993)(301.91015398,253.42969856)(302.7070331,252.64845199)
\curveto(303.51171488,251.87501262)(304.62890127,251.48829426)(306.0585956,251.48829574)
\curveto(306.88671151,251.48829426)(307.68749196,251.58985665)(308.46093935,251.79298324)
\curveto(309.2421779,251.99610625)(310.01561463,252.30079344)(310.78125185,252.70704574)
\lineto(310.78125185,250.66798324)
\curveto(310.00780214,250.3398579)(309.21483418,250.08985815)(308.4023456,249.91798324)
\curveto(307.5898358,249.7461085)(306.76561788,249.66017108)(305.92968935,249.66017074)
\curveto(303.83593331,249.66017108)(302.17577872,250.26954547)(300.9492206,251.48829574)
\curveto(299.73046866,252.70704304)(299.12109427,254.35547889)(299.1210956,256.43360824)
\curveto(299.12109427,258.58203716)(299.6992187,260.28516046)(300.8554706,261.54298324)
\curveto(302.01952887,262.80859544)(303.58593356,263.4414073)(305.55468935,263.44142074)
\curveto(307.32030482,263.4414073)(308.71483468,262.87109537)(309.7382831,261.73048324)
\curveto(310.76952012,260.59766015)(311.28514461,259.05469294)(311.2851581,257.10157699)
\moveto(309.1289081,257.73438949)
\curveto(309.11327178,258.91406808)(308.78124086,259.85547339)(308.13281435,260.55860824)
\curveto(307.49217965,261.26172198)(306.640618,261.61328413)(305.57812685,261.61329574)
\curveto(304.37499527,261.61328413)(303.41015248,261.27344072)(302.6835956,260.59376449)
\curveto(301.96484143,259.91406708)(301.55077934,258.95703679)(301.4414081,257.72267074)
\lineto(309.1289081,257.73438949)
}
}
{
\newrgbcolor{curcolor}{0 0 0}
\pscustom[linestyle=none,fillstyle=solid,fillcolor=curcolor]
{
\newpath
\moveto(322.42968935,261.10938949)
\curveto(322.18749222,261.25000324)(321.92186749,261.35156564)(321.63281435,261.41407699)
\curveto(321.35155556,261.48437801)(321.03905587,261.51953422)(320.69531435,261.51954574)
\curveto(319.47655743,261.51953422)(318.53905837,261.12109712)(317.88281435,260.32423324)
\curveto(317.23437218,259.53516121)(316.91015375,258.3984436)(316.9101581,256.91407699)
\lineto(316.9101581,250.00001449)
\lineto(314.74218935,250.00001449)
\lineto(314.74218935,263.12501449)
\lineto(316.9101581,263.12501449)
\lineto(316.9101581,261.08595199)
\curveto(317.3632783,261.88281511)(317.95312146,262.47265827)(318.67968935,262.85548324)
\curveto(319.406245,263.246095)(320.28905662,263.4414073)(321.32812685,263.44142074)
\curveto(321.47655543,263.4414073)(321.64061777,263.42968856)(321.82031435,263.40626449)
\curveto(321.99999241,263.3906261)(322.19921096,263.36328238)(322.4179706,263.32423324)
\lineto(322.42968935,261.10938949)
}
}
{
\newrgbcolor{curcolor}{0 0 0}
\pscustom[linestyle=none,fillstyle=solid,fillcolor=curcolor]
{
\newpath
\moveto(506.14062685,322.65626449)
\lineto(517.15625185,322.65626449)
\lineto(517.15625185,320.00001449)
\lineto(502.34375185,320.00001449)
\lineto(502.34375185,322.65626449)
\curveto(503.54166497,323.89584393)(505.17187168,325.5573006)(507.23437685,327.64063949)
\curveto(509.30728421,329.73437976)(510.60936624,331.08333674)(511.14062685,331.68751449)
\curveto(512.15103136,332.82291834)(512.85415566,333.78125071)(513.25000185,334.56251449)
\curveto(513.65623819,335.35416581)(513.85936299,336.1302067)(513.85937685,336.89063949)
\curveto(513.85936299,338.1302047)(513.42186343,339.14062035)(512.54687685,339.92188949)
\curveto(511.68228183,340.70311879)(510.55207463,341.0937434)(509.15625185,341.09376449)
\curveto(508.16666035,341.0937434)(507.11978639,340.92186857)(506.01562685,340.57813949)
\curveto(504.92187193,340.23436926)(503.7499981,339.71353645)(502.50000185,339.01563949)
\lineto(502.50000185,342.20313949)
\curveto(503.77083141,342.71353345)(504.95833022,343.09894973)(506.06250185,343.35938949)
\curveto(507.16666135,343.61978254)(508.177077,343.74999074)(509.09375185,343.75001449)
\curveto(511.510407,343.74999074)(513.43748841,343.14582468)(514.87500185,341.93751449)
\curveto(516.31248554,340.72916043)(517.03123482,339.11457871)(517.03125185,337.09376449)
\curveto(517.03123482,336.13541503)(516.84894333,335.2239576)(516.48437685,334.35938949)
\curveto(516.13019405,333.50520932)(515.47915304,332.49479367)(514.53125185,331.32813949)
\curveto(514.27082091,331.02604513)(513.44269674,330.15104601)(512.04687685,328.70313949)
\curveto(510.65103286,327.26563223)(508.68228483,325.25000924)(506.14062685,322.65626449)
}
}
{
\newrgbcolor{curcolor}{0 0 0}
\pscustom[linewidth=2,linecolor=curcolor,linestyle=dashed,dash=8 8]
{
\newpath
\moveto(150,530)
\lineto(60,530)
}
}
{
\newrgbcolor{curcolor}{0 0 0}
\pscustom[linestyle=none,fillstyle=solid,fillcolor=curcolor]
{
\newpath
\moveto(139.53769464,534.84048224)
\lineto(152.6487474,530.01921591)
\lineto(139.53769392,525.19795064)
\curveto(141.632292,528.04442372)(141.62022288,531.93889292)(139.53769464,534.84048224)
\lineto(139.53769464,534.84048224)
\closepath
}
}
{
\newrgbcolor{curcolor}{0 0 0}
\pscustom[linewidth=2,linecolor=curcolor,linestyle=dashed,dash=8 8]
{
\newpath
\moveto(370,330)
\lineto(500,330)
}
}
{
\newrgbcolor{curcolor}{0 0 0}
\pscustom[linestyle=none,fillstyle=solid,fillcolor=curcolor]
{
\newpath
\moveto(380.46230536,325.15951776)
\lineto(367.3512526,329.98078409)
\lineto(380.46230608,334.80204936)
\curveto(378.367708,331.95557628)(378.37977712,328.06110708)(380.46230536,325.15951776)
\lineto(380.46230536,325.15951776)
\closepath
}
}
{
\newrgbcolor{curcolor}{0 0 0}
\pscustom[linewidth=2,linecolor=curcolor,linestyle=dashed,dash=8 8]
{
\newpath
\moveto(230,260)
\lineto(60,260)
}
}
{
\newrgbcolor{curcolor}{0 0 0}
\pscustom[linestyle=none,fillstyle=solid,fillcolor=curcolor]
{
\newpath
\moveto(219.53769464,264.84048224)
\lineto(232.6487474,260.01921591)
\lineto(219.53769392,255.19795064)
\curveto(221.632292,258.04442372)(221.62022288,261.93889292)(219.53769464,264.84048224)
\lineto(219.53769464,264.84048224)
\closepath
}
}
{
\newrgbcolor{curcolor}{0 0 0}
\pscustom[linestyle=none,fillstyle=solid,fillcolor=curcolor]
{
\newpath
\moveto(43.96875185,522.65624924)
\lineto(49.12500185,522.65624924)
\lineto(49.12500185,540.45312424)
\lineto(43.51562685,539.32812424)
\lineto(43.51562685,542.20312424)
\lineto(49.09375185,543.32812424)
\lineto(52.25000185,543.32812424)
\lineto(52.25000185,522.65624924)
\lineto(57.40625185,522.65624924)
\lineto(57.40625185,519.99999924)
\lineto(43.96875185,519.99999924)
\lineto(43.96875185,522.65624924)
}
}
{
\newrgbcolor{curcolor}{0 0 0}
\pscustom[linestyle=none,fillstyle=solid,fillcolor=curcolor]
{
\newpath
\moveto(512.98437685,292.57813949)
\curveto(514.49477902,292.25521057)(515.67186118,291.58333624)(516.51562685,290.56251449)
\curveto(517.36977614,289.54167162)(517.79685905,288.28125621)(517.79687685,286.78126449)
\curveto(517.79685905,284.47917668)(517.00519318,282.69792846)(515.42187685,281.43751449)
\curveto(513.83852968,280.17709765)(511.58853193,279.54688995)(508.67187685,279.54688949)
\curveto(507.69270249,279.54688995)(506.68228683,279.64584818)(505.64062685,279.84376449)
\curveto(504.60937224,280.03126446)(503.54166497,280.31772251)(502.43750185,280.70313949)
\lineto(502.43750185,283.75001449)
\curveto(503.31249854,283.23959459)(504.27083091,282.85417831)(505.31250185,282.59376449)
\curveto(506.35416216,282.33334549)(507.44270274,282.20313729)(508.57812685,282.20313949)
\curveto(510.55728296,282.20313729)(512.06248979,282.5937619)(513.09375185,283.37501449)
\curveto(514.13540438,284.15626034)(514.65623719,285.29167587)(514.65625185,286.78126449)
\curveto(514.65623719,288.15625634)(514.17186268,289.22917193)(513.20312685,290.00001449)
\curveto(512.24478127,290.78125371)(510.90624094,291.17187832)(509.18750185,291.17188949)
\lineto(506.46875185,291.17188949)
\lineto(506.46875185,293.76563949)
\lineto(509.31250185,293.76563949)
\curveto(510.86457432,293.76562573)(512.05207313,294.07291709)(512.87500185,294.68751449)
\curveto(513.69790482,295.31249918)(514.10936274,296.20833162)(514.10937685,297.37501449)
\curveto(514.10936274,298.57291259)(513.68227983,299.48957834)(512.82812685,300.12501449)
\curveto(511.98436486,300.77082706)(510.77082441,301.0937434)(509.18750185,301.09376449)
\curveto(508.32291019,301.0937434)(507.39582779,300.99999349)(506.40625185,300.81251449)
\curveto(505.4166631,300.62499387)(504.32812252,300.33332749)(503.14062685,299.93751449)
\lineto(503.14062685,302.75001449)
\curveto(504.33853918,303.08332474)(505.45832972,303.33332449)(506.50000185,303.50001449)
\curveto(507.55207763,303.66665749)(508.54165997,303.74999074)(509.46875185,303.75001449)
\curveto(511.86457332,303.74999074)(513.76040475,303.20311629)(515.15625185,302.10938949)
\curveto(516.55206863,301.02603513)(517.2499846,299.5572866)(517.25000185,297.70313949)
\curveto(517.2499846,296.41145642)(516.8801933,295.31770751)(516.14062685,294.42188949)
\curveto(515.40102811,293.53645929)(514.34894583,292.92187657)(512.98437685,292.57813949)
}
}
{
\newrgbcolor{curcolor}{1 1 1}
\pscustom[linestyle=none,fillstyle=solid,fillcolor=curcolor]
{
\newpath
\moveto(160.00001634,310.00001449)
\lineto(399.99998735,310.00001449)
\lineto(400.00000185,310)
\lineto(400.00000185,280.00001945)
\lineto(399.99998735,280.00000496)
\lineto(160.00001634,280.00000496)
\lineto(160.00000185,280.00001945)
\lineto(160.00000185,310)
\lineto(160.00001634,310.00001449)
\closepath
}
}
{
\newrgbcolor{curcolor}{1 0 0}
\pscustom[linewidth=2,linecolor=curcolor,linestyle=dashed,dash=8 8]
{
\newpath
\moveto(160.00001634,310.00001449)
\lineto(399.99998735,310.00001449)
\lineto(400.00000185,310)
\lineto(400.00000185,280.00001945)
\lineto(399.99998735,280.00000496)
\lineto(160.00001634,280.00000496)
\lineto(160.00000185,280.00001945)
\lineto(160.00000185,310)
\lineto(160.00001634,310.00001449)
\closepath
}
}
{
\newrgbcolor{curcolor}{0 0 0}
\pscustom[linewidth=2,linecolor=curcolor,linestyle=dashed,dash=8 8]
{
\newpath
\moveto(380,290)
\lineto(500,290)
}
}
{
\newrgbcolor{curcolor}{0 0 0}
\pscustom[linestyle=none,fillstyle=solid,fillcolor=curcolor]
{
\newpath
\moveto(390.46230536,285.15951776)
\lineto(377.3512526,289.98078409)
\lineto(390.46230608,294.80204936)
\curveto(388.367708,291.95557628)(388.37977712,288.06110708)(390.46230536,285.15951776)
\lineto(390.46230536,285.15951776)
\closepath
}
}
{
\newrgbcolor{curcolor}{0 0 0}
\pscustom[linestyle=none,fillstyle=solid,fillcolor=curcolor]
{
\newpath
\moveto(52.09375185,270.57813949)
\lineto(44.12500185,258.12501449)
\lineto(52.09375185,258.12501449)
\lineto(52.09375185,270.57813949)
\moveto(51.26562685,273.32813949)
\lineto(55.23437685,273.32813949)
\lineto(55.23437685,258.12501449)
\lineto(58.56250185,258.12501449)
\lineto(58.56250185,255.50001449)
\lineto(55.23437685,255.50001449)
\lineto(55.23437685,250.00001449)
\lineto(52.09375185,250.00001449)
\lineto(52.09375185,255.50001449)
\lineto(41.56250185,255.50001449)
\lineto(41.56250185,258.54688949)
\lineto(51.26562685,273.32813949)
}
}
\end{pspicture}

		\end{center}

		\begin{enumerate}
		  \item Fond d'écran (présent dans l'archive stocké sur le téléphone)
		  \item Zone de saisie.
		  \item Label d'erreur.
		  \item Bouton ``\hyperlink{Accueil}{Valider}".
		\end{enumerate}

		\subsubsection{Description des zones}
				
			\begin{tabular}{|c|c|c|c|c|} \hline
				Numéro de zone & Type  & Description & Evènement &	Règle \\\hline 
				2 & Zone de saisie & Définition du pseudonyme & Perte du focus & RG1-01 \\\hline
				4 & Bouton         & Bouton de validation     & Cliqué & RG1-02 \\\hline
			\end{tabular}

		\subsubsection{Description des règles}
		
		\underline{RG1-01 :}
			\begin{quote}
				Vérification de la zone de saisie :
				\begin{itemize}
				  \item Non vide.
				  \item Pseudonyme entré inexistant.
				\end{itemize}
				Si la vérification est incorrecte alors
				\begin{quote}
					Afficher l'erreur via le label d'erreur (3).
				\end{quote}
			\end{quote}
			
		\underline{RG1-02 :}
			\begin{quote}
				RG1-01.\\
				Si la vérification est correcte alors
				\begin{quote}
					Créer un nouveau compte hors ligne ayant pour pseudonyme celui entré.\\
					Créer la classe utilisateur.\\
					Mettre à jour la classe utilisateur.\\
					Créer la page d'accueil%
						\footnote[1]{
							\hyperlink{Page d'accueil}{Page d'accueil}
							\og voir section \ref{Accueil}, page \pageref{Accueil}.\fg
						}.\\
					Afficher la page d'acceuil\footnotemark[1].\\
					Supprimer la page de création du profil.
				\end{quote}
			\end{quote}
	
\newpage

	\subsection{Page d'accueil}
		\hypertarget{Accueil}{}
		\label{Accueil}

		\begin{center}	
			%LaTeX with PSTricks extensions
%%Creator: inkscape 0.48.0
%%Please note this file requires PSTricks extensions
\psset{xunit=.5pt,yunit=.5pt,runit=.5pt}
\begin{pspicture}(560,600)
{
\newrgbcolor{curcolor}{1 1 1}
\pscustom[linestyle=none,fillstyle=solid,fillcolor=curcolor]
{
\newpath
\moveto(133.12418633,597.52237246)
\lineto(426.87615472,597.52237246)
\curveto(443.85414222,597.52237246)(457.52234155,583.85417314)(457.52234155,566.87618563)
\lineto(457.52234155,33.12418673)
\curveto(457.52234155,16.14619923)(443.85414222,2.4779999)(426.87615472,2.4779999)
\lineto(133.12418633,2.4779999)
\curveto(116.14619883,2.4779999)(102.4779995,16.14619923)(102.4779995,33.12418673)
\lineto(102.4779995,566.87618563)
\curveto(102.4779995,583.85417314)(116.14619883,597.52237246)(133.12418633,597.52237246)
\closepath
}
}
{
\newrgbcolor{curcolor}{0 0 0}
\pscustom[linewidth=4.95599985,linecolor=curcolor]
{
\newpath
\moveto(133.12418633,597.52237246)
\lineto(426.87615472,597.52237246)
\curveto(443.85414222,597.52237246)(457.52234155,583.85417314)(457.52234155,566.87618563)
\lineto(457.52234155,33.12418673)
\curveto(457.52234155,16.14619923)(443.85414222,2.4779999)(426.87615472,2.4779999)
\lineto(133.12418633,2.4779999)
\curveto(116.14619883,2.4779999)(102.4779995,16.14619923)(102.4779995,33.12418673)
\lineto(102.4779995,566.87618563)
\curveto(102.4779995,583.85417314)(116.14619883,597.52237246)(133.12418633,597.52237246)
\closepath
}
}
{
\newrgbcolor{curcolor}{1 1 1}
\pscustom[linestyle=none,fillstyle=solid,fillcolor=curcolor]
{
\newpath
\moveto(201.70414239,399.9999964)
\lineto(358.44696313,399.9999964)
\curveto(364.84734492,399.9999964)(369.99999886,394.84734246)(369.99999886,388.44696067)
\lineto(369.99999886,378.53398154)
\curveto(369.99999886,372.13359974)(364.84734492,366.9809458)(358.44696313,366.9809458)
\lineto(201.70414239,366.9809458)
\curveto(195.30376059,366.9809458)(190.15110665,372.13359974)(190.15110665,378.53398154)
\lineto(190.15110665,388.44696067)
\curveto(190.15110665,394.84734246)(195.30376059,399.9999964)(201.70414239,399.9999964)
\closepath
}
}
{
\newrgbcolor{curcolor}{0 0 0}
\pscustom[linewidth=2,linecolor=curcolor]
{
\newpath
\moveto(201.70414239,399.9999964)
\lineto(358.44696313,399.9999964)
\curveto(364.84734492,399.9999964)(369.99999886,394.84734246)(369.99999886,388.44696067)
\lineto(369.99999886,378.53398154)
\curveto(369.99999886,372.13359974)(364.84734492,366.9809458)(358.44696313,366.9809458)
\lineto(201.70414239,366.9809458)
\curveto(195.30376059,366.9809458)(190.15110665,372.13359974)(190.15110665,378.53398154)
\lineto(190.15110665,388.44696067)
\curveto(190.15110665,394.84734246)(195.30376059,399.9999964)(201.70414239,399.9999964)
\closepath
}
}
{
\newrgbcolor{curcolor}{0 0 0}
\pscustom[linestyle=none,fillstyle=solid,fillcolor=curcolor]
{
\newpath
\moveto(212.58688241,394.55735419)
\lineto(214.95406991,394.55735419)
\lineto(214.95406991,378.28001044)
\curveto(214.95406519,376.17063633)(214.55172184,374.63938786)(213.74703866,373.68626044)
\curveto(212.95016094,372.73313977)(211.66500597,372.25657775)(209.89156991,372.25657294)
\lineto(208.98922616,372.25657294)
\lineto(208.98922616,374.24876044)
\lineto(209.72750741,374.24876044)
\curveto(210.77438187,374.24876325)(211.51266238,374.54173171)(211.94235116,375.12766669)
\curveto(212.37203652,375.71360554)(212.58688005,376.76438574)(212.58688241,378.28001044)
\lineto(212.58688241,394.55735419)
}
}
{
\newrgbcolor{curcolor}{0 0 0}
\pscustom[linestyle=none,fillstyle=solid,fillcolor=curcolor]
{
\newpath
\moveto(224.65719491,388.67454169)
\curveto(223.50093872,388.67453008)(222.58687713,388.22140553)(221.91500741,387.31516669)
\curveto(221.24312847,386.41671983)(220.90719131,385.18234607)(220.90719491,383.61204169)
\curveto(220.90719131,382.04172421)(221.23922223,380.8034442)(221.90328866,379.89719794)
\curveto(222.57515839,378.9987585)(223.49312622,378.5495402)(224.65719491,378.54954169)
\curveto(225.80562391,378.5495402)(226.71577925,379.00266475)(227.38766366,379.90891669)
\curveto(228.05952791,380.81516294)(228.39546507,382.0495367)(228.39547616,383.61204169)
\curveto(228.39546507,385.16672108)(228.05952791,386.3971886)(227.38766366,387.30344794)
\curveto(226.71577925,388.21749928)(225.80562391,388.67453008)(224.65719491,388.67454169)
\moveto(224.65719491,390.50266669)
\curveto(226.53218569,390.50265325)(228.00484046,389.89327886)(229.07516366,388.67454169)
\curveto(230.14546332,387.4557813)(230.68061904,385.76828298)(230.68063241,383.61204169)
\curveto(230.68061904,381.46359979)(230.14546332,379.77610148)(229.07516366,378.54954169)
\curveto(228.00484046,377.33079142)(226.53218569,376.72141703)(224.65719491,376.72141669)
\curveto(222.77437694,376.72141703)(221.29781592,377.33079142)(220.22750741,378.54954169)
\curveto(219.16500555,379.77610148)(218.63375608,381.46359979)(218.63375741,383.61204169)
\curveto(218.63375608,385.76828298)(219.16500555,387.4557813)(220.22750741,388.67454169)
\curveto(221.29781592,389.89327886)(222.77437694,390.50265325)(224.65719491,390.50266669)
}
}
{
\newrgbcolor{curcolor}{0 0 0}
\pscustom[linestyle=none,fillstyle=solid,fillcolor=curcolor]
{
\newpath
\moveto(234.02047616,382.24094794)
\lineto(234.02047616,390.18626044)
\lineto(236.17672616,390.18626044)
\lineto(236.17672616,382.32297919)
\curveto(236.17672196,381.08078767)(236.41890922,380.14719485)(236.90328866,379.52219794)
\curveto(237.38765825,378.9050086)(238.11422003,378.59641516)(239.08297616,378.59641669)
\curveto(240.24703039,378.59641516)(241.16499822,378.96750853)(241.83688241,379.70969794)
\curveto(242.51655937,380.45188205)(242.85640278,381.46359979)(242.85641366,382.74485419)
\lineto(242.85641366,390.18626044)
\lineto(245.01266366,390.18626044)
\lineto(245.01266366,377.06126044)
\lineto(242.85641366,377.06126044)
\lineto(242.85641366,379.07688544)
\curveto(242.33296581,378.28000922)(241.72359142,377.68625982)(241.02828866,377.29563544)
\curveto(240.3407803,376.91282309)(239.53999985,376.72141703)(238.62594491,376.72141669)
\curveto(237.11812727,376.72141703)(235.97359717,377.19016656)(235.19235116,378.12766669)
\curveto(234.41109873,379.06516469)(234.02047412,380.43625707)(234.02047616,382.24094794)
\moveto(239.44625741,390.50266669)
\lineto(239.44625741,390.50266669)
}
}
{
\newrgbcolor{curcolor}{0 0 0}
\pscustom[linestyle=none,fillstyle=solid,fillcolor=curcolor]
{
\newpath
\moveto(260.70406991,384.16282294)
\lineto(260.70406991,383.10813544)
\lineto(250.79000741,383.10813544)
\curveto(250.88375374,381.62375588)(251.32906579,380.49094451)(252.12594491,379.70969794)
\curveto(252.93062669,378.93625857)(254.04781308,378.5495402)(255.47750741,378.54954169)
\curveto(256.30562332,378.5495402)(257.10640377,378.6511026)(257.87985116,378.85422919)
\curveto(258.66108971,379.05735219)(259.43452644,379.36203939)(260.20016366,379.76829169)
\lineto(260.20016366,377.72922919)
\curveto(259.42671395,377.40110385)(258.63374599,377.1511041)(257.82125741,376.97922919)
\curveto(257.00874762,376.80735444)(256.18452969,376.72141703)(255.34860116,376.72141669)
\curveto(253.25484512,376.72141703)(251.59469053,377.33079142)(250.36813241,378.54954169)
\curveto(249.14938047,379.76828898)(248.54000608,381.41672483)(248.54000741,383.49485419)
\curveto(248.54000608,385.64328311)(249.11813051,387.34640641)(250.27438241,388.60422919)
\curveto(251.43844069,389.86984138)(253.00484537,390.50265325)(254.97360116,390.50266669)
\curveto(256.73921663,390.50265325)(258.13374649,389.93234132)(259.15719491,388.79172919)
\curveto(260.18843194,387.65890609)(260.70405642,386.11593889)(260.70406991,384.16282294)
\moveto(258.54781991,384.79563544)
\curveto(258.53218359,385.97531403)(258.20015267,386.91671933)(257.55172616,387.61985419)
\curveto(256.91109146,388.32296793)(256.05952981,388.67453008)(254.99703866,388.67454169)
\curveto(253.79390708,388.67453008)(252.82906429,388.33468667)(252.10250741,387.65501044)
\curveto(251.38375324,386.97531303)(250.96969115,386.01828273)(250.86031991,384.78391669)
\lineto(258.54781991,384.79563544)
}
}
{
\newrgbcolor{curcolor}{0 0 0}
\pscustom[linestyle=none,fillstyle=solid,fillcolor=curcolor]
{
\newpath
\moveto(271.84860116,388.17063544)
\curveto(271.60640403,388.31124919)(271.3407793,388.41281159)(271.05172616,388.47532294)
\curveto(270.77046737,388.54562396)(270.45796768,388.58078017)(270.11422616,388.58079169)
\curveto(268.89546924,388.58078017)(267.95797018,388.18234307)(267.30172616,387.38547919)
\curveto(266.65328399,386.59640716)(266.32906556,385.45968954)(266.32906991,383.97532294)
\lineto(266.32906991,377.06126044)
\lineto(264.16110116,377.06126044)
\lineto(264.16110116,390.18626044)
\lineto(266.32906991,390.18626044)
\lineto(266.32906991,388.14719794)
\curveto(266.78219011,388.94406106)(267.37203327,389.53390422)(268.09860116,389.91672919)
\curveto(268.82515681,390.30734094)(269.70796843,390.50265325)(270.74703866,390.50266669)
\curveto(270.89546724,390.50265325)(271.05952958,390.49093451)(271.23922616,390.46751044)
\curveto(271.41890422,390.45187205)(271.61812277,390.42452833)(271.83688241,390.38547919)
\lineto(271.84860116,388.17063544)
}
}
{
\newrgbcolor{curcolor}{0 0 0}
\pscustom[linestyle=none,fillstyle=solid,fillcolor=curcolor]
{
}
}
{
\newrgbcolor{curcolor}{0 0 0}
\pscustom[linestyle=none,fillstyle=solid,fillcolor=curcolor]
{
\newpath
\moveto(293.00094491,384.16282294)
\lineto(293.00094491,383.10813544)
\lineto(283.08688241,383.10813544)
\curveto(283.18062874,381.62375588)(283.62594079,380.49094451)(284.42281991,379.70969794)
\curveto(285.22750169,378.93625857)(286.34468808,378.5495402)(287.77438241,378.54954169)
\curveto(288.60249832,378.5495402)(289.40327877,378.6511026)(290.17672616,378.85422919)
\curveto(290.95796471,379.05735219)(291.73140144,379.36203939)(292.49703866,379.76829169)
\lineto(292.49703866,377.72922919)
\curveto(291.72358895,377.40110385)(290.93062099,377.1511041)(290.11813241,376.97922919)
\curveto(289.30562262,376.80735444)(288.48140469,376.72141703)(287.64547616,376.72141669)
\curveto(285.55172012,376.72141703)(283.89156553,377.33079142)(282.66500741,378.54954169)
\curveto(281.44625547,379.76828898)(280.83688108,381.41672483)(280.83688241,383.49485419)
\curveto(280.83688108,385.64328311)(281.41500551,387.34640641)(282.57125741,388.60422919)
\curveto(283.73531569,389.86984138)(285.30172037,390.50265325)(287.27047616,390.50266669)
\curveto(289.03609163,390.50265325)(290.43062149,389.93234132)(291.45406991,388.79172919)
\curveto(292.48530694,387.65890609)(293.00093142,386.11593889)(293.00094491,384.16282294)
\moveto(290.84469491,384.79563544)
\curveto(290.82905859,385.97531403)(290.49702767,386.91671933)(289.84860116,387.61985419)
\curveto(289.20796646,388.32296793)(288.35640481,388.67453008)(287.29391366,388.67454169)
\curveto(286.09078208,388.67453008)(285.12593929,388.33468667)(284.39938241,387.65501044)
\curveto(283.68062824,386.97531303)(283.26656615,386.01828273)(283.15719491,384.78391669)
\lineto(290.84469491,384.79563544)
}
}
{
\newrgbcolor{curcolor}{0 0 0}
\pscustom[linestyle=none,fillstyle=solid,fillcolor=curcolor]
{
\newpath
\moveto(307.45016366,384.98313544)
\lineto(307.45016366,377.06126044)
\lineto(305.29391366,377.06126044)
\lineto(305.29391366,384.91282294)
\curveto(305.29390264,386.15500135)(305.05171538,387.08468792)(304.56735116,387.70188544)
\curveto(304.08296635,388.31906168)(303.35640458,388.62765512)(302.38766366,388.62766669)
\curveto(301.22359421,388.62765512)(300.30562638,388.25656175)(299.63375741,387.51438544)
\curveto(298.96187772,386.77218823)(298.62594056,385.76047049)(298.62594491,384.47922919)
\lineto(298.62594491,377.06126044)
\lineto(296.45797616,377.06126044)
\lineto(296.45797616,390.18626044)
\lineto(298.62594491,390.18626044)
\lineto(298.62594491,388.14719794)
\curveto(299.14156504,388.93624857)(299.74703319,389.52609173)(300.44235116,389.91672919)
\curveto(301.14546929,390.30734094)(301.95406223,390.50265325)(302.86813241,390.50266669)
\curveto(304.37593481,390.50265325)(305.51655867,390.03390372)(306.29000741,389.09641669)
\curveto(307.06343212,388.16671808)(307.45015049,386.79562571)(307.45016366,384.98313544)
}
}
{
\newrgbcolor{curcolor}{0 0 0}
\pscustom[linestyle=none,fillstyle=solid,fillcolor=curcolor]
{
}
}
{
\newrgbcolor{curcolor}{0 0 0}
\pscustom[linestyle=none,fillstyle=solid,fillcolor=curcolor]
{
\newpath
\moveto(327.78219491,389.79954169)
\lineto(327.78219491,387.76047919)
\curveto(327.17280989,388.07296818)(326.53999802,388.30734294)(325.88375741,388.46360419)
\curveto(325.22749933,388.61984263)(324.54781251,388.69796755)(323.84469491,388.69797919)
\curveto(322.77437679,388.69796755)(321.96969009,388.53390522)(321.43063241,388.20579169)
\curveto(320.89937866,387.87765587)(320.63375393,387.38546887)(320.63375741,386.72922919)
\curveto(320.63375393,386.22922002)(320.82515999,385.83468917)(321.20797616,385.54563544)
\curveto(321.59078422,385.26437724)(322.3603147,384.99484626)(323.51656991,384.73704169)
\lineto(324.25485116,384.57297919)
\curveto(325.78609253,384.24484701)(326.87202894,383.78000372)(327.51266366,383.17844794)
\curveto(328.16109015,382.58469242)(328.48530858,381.752662)(328.48531991,380.68235419)
\curveto(328.48530858,379.46360179)(328.00093406,378.498759)(327.03219491,377.78782294)
\curveto(326.07124849,377.07688542)(324.74703106,376.72141703)(323.05953866,376.72141669)
\curveto(322.35640846,376.72141703)(321.62203419,376.79172946)(320.85641366,376.93235419)
\curveto(320.09859821,377.06516669)(319.29781776,377.26829148)(318.45406991,377.54172919)
\lineto(318.45406991,379.76829169)
\curveto(319.25094281,379.3542269)(320.03609828,379.04172721)(320.80953866,378.83079169)
\curveto(321.58297173,378.62766512)(322.34859596,378.52610273)(323.10641366,378.52610419)
\curveto(324.12203169,378.52610273)(324.90328091,378.69797755)(325.45016366,379.04172919)
\curveto(325.99702981,379.39328936)(326.27046704,379.88547637)(326.27047616,380.51829169)
\curveto(326.27046704,381.10422515)(326.07124849,381.55344345)(325.67281991,381.86594794)
\curveto(325.28218678,382.17844282)(324.41890639,382.47922377)(323.08297616,382.76829169)
\lineto(322.33297616,382.94407294)
\curveto(320.99703481,383.22531678)(320.03219203,383.65500385)(319.43844491,384.23313544)
\curveto(318.84469322,384.81906518)(318.54781851,385.61984563)(318.54781991,386.63547919)
\curveto(318.54781851,387.86984338)(318.98531808,388.82296743)(319.86031991,389.49485419)
\curveto(320.73531633,390.16671608)(321.97750258,390.50265325)(323.58688241,390.50266669)
\curveto(324.38375018,390.50265325)(325.13374943,390.44405956)(325.83688241,390.32688544)
\curveto(326.53999802,390.20968479)(327.18843487,390.03390372)(327.78219491,389.79954169)
}
}
{
\newrgbcolor{curcolor}{0 0 0}
\pscustom[linestyle=none,fillstyle=solid,fillcolor=curcolor]
{
\newpath
\moveto(337.01656991,388.67454169)
\curveto(335.86031372,388.67453008)(334.94625213,388.22140553)(334.27438241,387.31516669)
\curveto(333.60250347,386.41671983)(333.26656631,385.18234607)(333.26656991,383.61204169)
\curveto(333.26656631,382.04172421)(333.59859723,380.8034442)(334.26266366,379.89719794)
\curveto(334.93453339,378.9987585)(335.85250122,378.5495402)(337.01656991,378.54954169)
\curveto(338.16499891,378.5495402)(339.07515425,379.00266475)(339.74703866,379.90891669)
\curveto(340.41890291,380.81516294)(340.75484007,382.0495367)(340.75485116,383.61204169)
\curveto(340.75484007,385.16672108)(340.41890291,386.3971886)(339.74703866,387.30344794)
\curveto(339.07515425,388.21749928)(338.16499891,388.67453008)(337.01656991,388.67454169)
\moveto(337.01656991,390.50266669)
\curveto(338.89156069,390.50265325)(340.36421546,389.89327886)(341.43453866,388.67454169)
\curveto(342.50483832,387.4557813)(343.03999404,385.76828298)(343.04000741,383.61204169)
\curveto(343.03999404,381.46359979)(342.50483832,379.77610148)(341.43453866,378.54954169)
\curveto(340.36421546,377.33079142)(338.89156069,376.72141703)(337.01656991,376.72141669)
\curveto(335.13375194,376.72141703)(333.65719092,377.33079142)(332.58688241,378.54954169)
\curveto(331.52438055,379.77610148)(330.99313108,381.46359979)(330.99313241,383.61204169)
\curveto(330.99313108,385.76828298)(331.52438055,387.4557813)(332.58688241,388.67454169)
\curveto(333.65719092,389.89327886)(335.13375194,390.50265325)(337.01656991,390.50266669)
}
}
{
\newrgbcolor{curcolor}{0 0 0}
\pscustom[linestyle=none,fillstyle=solid,fillcolor=curcolor]
{
\newpath
\moveto(346.60250741,395.29563544)
\lineto(348.75875741,395.29563544)
\lineto(348.75875741,377.06126044)
\lineto(346.60250741,377.06126044)
\lineto(346.60250741,395.29563544)
}
}
{
\newrgbcolor{curcolor}{0 0 0}
\pscustom[linestyle=none,fillstyle=solid,fillcolor=curcolor]
{
\newpath
\moveto(358.34469491,388.67454169)
\curveto(357.18843872,388.67453008)(356.27437713,388.22140553)(355.60250741,387.31516669)
\curveto(354.93062847,386.41671983)(354.59469131,385.18234607)(354.59469491,383.61204169)
\curveto(354.59469131,382.04172421)(354.92672223,380.8034442)(355.59078866,379.89719794)
\curveto(356.26265839,378.9987585)(357.18062622,378.5495402)(358.34469491,378.54954169)
\curveto(359.49312391,378.5495402)(360.40327925,379.00266475)(361.07516366,379.90891669)
\curveto(361.74702791,380.81516294)(362.08296507,382.0495367)(362.08297616,383.61204169)
\curveto(362.08296507,385.16672108)(361.74702791,386.3971886)(361.07516366,387.30344794)
\curveto(360.40327925,388.21749928)(359.49312391,388.67453008)(358.34469491,388.67454169)
\moveto(358.34469491,390.50266669)
\curveto(360.21968569,390.50265325)(361.69234046,389.89327886)(362.76266366,388.67454169)
\curveto(363.83296332,387.4557813)(364.36811904,385.76828298)(364.36813241,383.61204169)
\curveto(364.36811904,381.46359979)(363.83296332,379.77610148)(362.76266366,378.54954169)
\curveto(361.69234046,377.33079142)(360.21968569,376.72141703)(358.34469491,376.72141669)
\curveto(356.46187694,376.72141703)(354.98531592,377.33079142)(353.91500741,378.54954169)
\curveto(352.85250555,379.77610148)(352.32125608,381.46359979)(352.32125741,383.61204169)
\curveto(352.32125608,385.76828298)(352.85250555,387.4557813)(353.91500741,388.67454169)
\curveto(354.98531592,389.89327886)(356.46187694,390.50265325)(358.34469491,390.50266669)
}
}
{
\newrgbcolor{curcolor}{1 1 1}
\pscustom[linestyle=none,fillstyle=solid,fillcolor=curcolor]
{
\newpath
\moveto(151.8098362,349.9999964)
\lineto(401.00449258,349.9999964)
\curveto(406.79787809,349.9999964)(411.46186715,345.33600734)(411.46186715,339.54262183)
\lineto(411.46186715,330.68223021)
\curveto(411.46186715,324.8888447)(406.79787809,320.22485564)(401.00449258,320.22485564)
\lineto(151.8098362,320.22485564)
\curveto(146.01645069,320.22485564)(141.35246163,324.8888447)(141.35246163,330.68223021)
\lineto(141.35246163,339.54262183)
\curveto(141.35246163,345.33600734)(146.01645069,349.9999964)(151.8098362,349.9999964)
\closepath
}
}
{
\newrgbcolor{curcolor}{0 0 0}
\pscustom[linewidth=2,linecolor=curcolor]
{
\newpath
\moveto(151.8098362,349.9999964)
\lineto(401.00449258,349.9999964)
\curveto(406.79787809,349.9999964)(411.46186715,345.33600734)(411.46186715,339.54262183)
\lineto(411.46186715,330.68223021)
\curveto(411.46186715,324.8888447)(406.79787809,320.22485564)(401.00449258,320.22485564)
\lineto(151.8098362,320.22485564)
\curveto(146.01645069,320.22485564)(141.35246163,324.8888447)(141.35246163,330.68223021)
\lineto(141.35246163,339.54262183)
\curveto(141.35246163,345.33600734)(146.01645069,349.9999964)(151.8098362,349.9999964)
\closepath
}
}
{
\newrgbcolor{curcolor}{0 0 0}
\pscustom[linestyle=none,fillstyle=solid,fillcolor=curcolor]
{
\newpath
\moveto(152.35546761,347.49609015)
\lineto(154.72265511,347.49609015)
\lineto(154.72265511,331.2187464)
\curveto(154.72265039,329.10937229)(154.32030704,327.57812382)(153.51562386,326.6249964)
\curveto(152.71874614,325.67187573)(151.43359118,325.19531371)(149.66015511,325.1953089)
\lineto(148.75781136,325.1953089)
\lineto(148.75781136,327.1874964)
\lineto(149.49609261,327.1874964)
\curveto(150.54296707,327.18749921)(151.28124758,327.48046767)(151.71093636,328.06640265)
\curveto(152.14062172,328.6523415)(152.35546526,329.7031217)(152.35546761,331.2187464)
\lineto(152.35546761,347.49609015)
}
}
{
\newrgbcolor{curcolor}{0 0 0}
\pscustom[linestyle=none,fillstyle=solid,fillcolor=curcolor]
{
\newpath
\moveto(164.42578011,341.61327765)
\curveto(163.26952392,341.61326604)(162.35546234,341.16014149)(161.68359261,340.25390265)
\curveto(161.01171368,339.3554558)(160.67577652,338.12108203)(160.67578011,336.55077765)
\curveto(160.67577652,334.98046017)(161.00780743,333.74218016)(161.67187386,332.8359339)
\curveto(162.3437436,331.93749446)(163.26171143,331.48827616)(164.42578011,331.48827765)
\curveto(165.57420912,331.48827616)(166.48436446,331.94140071)(167.15624886,332.84765265)
\curveto(167.82811311,333.7538989)(168.16405028,334.98827266)(168.16406136,336.55077765)
\curveto(168.16405028,338.10545705)(167.82811311,339.33592457)(167.15624886,340.2421839)
\curveto(166.48436446,341.15623525)(165.57420912,341.61326604)(164.42578011,341.61327765)
\moveto(164.42578011,343.44140265)
\curveto(166.30077089,343.44138921)(167.77342567,342.83201482)(168.84374886,341.61327765)
\curveto(169.91404853,340.39451726)(170.44920424,338.70701895)(170.44921761,336.55077765)
\curveto(170.44920424,334.40233575)(169.91404853,332.71483744)(168.84374886,331.48827765)
\curveto(167.77342567,330.26952738)(166.30077089,329.66015299)(164.42578011,329.66015265)
\curveto(162.54296215,329.66015299)(161.06640112,330.26952738)(159.99609261,331.48827765)
\curveto(158.93359076,332.71483744)(158.40234129,334.40233575)(158.40234261,336.55077765)
\curveto(158.40234129,338.70701895)(158.93359076,340.39451726)(159.99609261,341.61327765)
\curveto(161.06640112,342.83201482)(162.54296215,343.44138921)(164.42578011,343.44140265)
}
}
{
\newrgbcolor{curcolor}{0 0 0}
\pscustom[linestyle=none,fillstyle=solid,fillcolor=curcolor]
{
\newpath
\moveto(173.78906136,335.1796839)
\lineto(173.78906136,343.1249964)
\lineto(175.94531136,343.1249964)
\lineto(175.94531136,335.26171515)
\curveto(175.94530717,334.01952363)(176.18749443,333.08593082)(176.67187386,332.4609339)
\curveto(177.15624346,331.84374456)(177.88280523,331.53515112)(178.85156136,331.53515265)
\curveto(180.0156156,331.53515112)(180.93358343,331.9062445)(181.60546761,332.6484339)
\curveto(182.28514458,333.39061801)(182.62498799,334.40233575)(182.62499886,335.68359015)
\lineto(182.62499886,343.1249964)
\lineto(184.78124886,343.1249964)
\lineto(184.78124886,329.9999964)
\lineto(182.62499886,329.9999964)
\lineto(182.62499886,332.0156214)
\curveto(182.10155101,331.21874518)(181.49217662,330.62499578)(180.79687386,330.2343714)
\curveto(180.1093655,329.85155905)(179.30858505,329.66015299)(178.39453011,329.66015265)
\curveto(176.88671248,329.66015299)(175.74218237,330.12890252)(174.96093636,331.06640265)
\curveto(174.17968393,332.00390065)(173.78905932,333.37499303)(173.78906136,335.1796839)
\moveto(179.21484261,343.44140265)
\lineto(179.21484261,343.44140265)
}
}
{
\newrgbcolor{curcolor}{0 0 0}
\pscustom[linestyle=none,fillstyle=solid,fillcolor=curcolor]
{
\newpath
\moveto(200.47265511,337.1015589)
\lineto(200.47265511,336.0468714)
\lineto(190.55859261,336.0468714)
\curveto(190.65233895,334.56249184)(191.097651,333.42968047)(191.89453011,332.6484339)
\curveto(192.6992119,331.87499453)(193.81639828,331.48827616)(195.24609261,331.48827765)
\curveto(196.07420852,331.48827616)(196.87498897,331.58983856)(197.64843636,331.79296515)
\curveto(198.42967492,331.99608816)(199.20311164,332.30077535)(199.96874886,332.70702765)
\lineto(199.96874886,330.66796515)
\curveto(199.19529915,330.33983981)(198.4023312,330.08984006)(197.58984261,329.91796515)
\curveto(196.77733282,329.74609041)(195.95311489,329.66015299)(195.11718636,329.66015265)
\curveto(193.02343032,329.66015299)(191.36327573,330.26952738)(190.13671761,331.48827765)
\curveto(188.91796568,332.70702495)(188.30859129,334.3554608)(188.30859261,336.43359015)
\curveto(188.30859129,338.58201907)(188.88671571,340.28514237)(190.04296761,341.54296515)
\curveto(191.20702589,342.80857734)(192.77343057,343.44138921)(194.74218636,343.44140265)
\curveto(196.50780184,343.44138921)(197.9023317,342.87107728)(198.92578011,341.73046515)
\curveto(199.95701714,340.59764205)(200.47264162,339.05467485)(200.47265511,337.1015589)
\moveto(198.31640511,337.7343714)
\curveto(198.3007688,338.91404999)(197.96873788,339.8554553)(197.32031136,340.55859015)
\curveto(196.67967667,341.26170389)(195.82811502,341.61326604)(194.76562386,341.61327765)
\curveto(193.56249229,341.61326604)(192.5976495,341.27342263)(191.87109261,340.5937464)
\curveto(191.15233845,339.91404899)(190.73827636,338.9570187)(190.62890511,337.72265265)
\lineto(198.31640511,337.7343714)
}
}
{
\newrgbcolor{curcolor}{0 0 0}
\pscustom[linestyle=none,fillstyle=solid,fillcolor=curcolor]
{
\newpath
\moveto(211.61718636,341.1093714)
\curveto(211.37498924,341.24998515)(211.1093645,341.35154755)(210.82031136,341.4140589)
\curveto(210.53905257,341.48435992)(210.22655289,341.51951613)(209.88281136,341.51952765)
\curveto(208.66405445,341.51951613)(207.72655539,341.12107903)(207.07031136,340.32421515)
\curveto(206.42186919,339.53514312)(206.09765077,338.3984255)(206.09765511,336.9140589)
\lineto(206.09765511,329.9999964)
\lineto(203.92968636,329.9999964)
\lineto(203.92968636,343.1249964)
\lineto(206.09765511,343.1249964)
\lineto(206.09765511,341.0859339)
\curveto(206.55077531,341.88279702)(207.14061847,342.47264018)(207.86718636,342.85546515)
\curveto(208.59374202,343.24607691)(209.47655364,343.44138921)(210.51562386,343.44140265)
\curveto(210.66405245,343.44138921)(210.82811479,343.42967047)(211.00781136,343.4062464)
\curveto(211.18748943,343.39060801)(211.38670798,343.36326429)(211.60546761,343.32421515)
\lineto(211.61718636,341.1093714)
}
}
{
\newrgbcolor{curcolor}{0 0 0}
\pscustom[linestyle=none,fillstyle=solid,fillcolor=curcolor]
{
}
}
{
\newrgbcolor{curcolor}{0 0 0}
\pscustom[linestyle=none,fillstyle=solid,fillcolor=curcolor]
{
\newpath
\moveto(232.76953011,337.1015589)
\lineto(232.76953011,336.0468714)
\lineto(222.85546761,336.0468714)
\curveto(222.94921395,334.56249184)(223.394526,333.42968047)(224.19140511,332.6484339)
\curveto(224.9960869,331.87499453)(226.11327328,331.48827616)(227.54296761,331.48827765)
\curveto(228.37108352,331.48827616)(229.17186397,331.58983856)(229.94531136,331.79296515)
\curveto(230.72654992,331.99608816)(231.49998664,332.30077535)(232.26562386,332.70702765)
\lineto(232.26562386,330.66796515)
\curveto(231.49217415,330.33983981)(230.6992062,330.08984006)(229.88671761,329.91796515)
\curveto(229.07420782,329.74609041)(228.24998989,329.66015299)(227.41406136,329.66015265)
\curveto(225.32030532,329.66015299)(223.66015073,330.26952738)(222.43359261,331.48827765)
\curveto(221.21484068,332.70702495)(220.60546629,334.3554608)(220.60546761,336.43359015)
\curveto(220.60546629,338.58201907)(221.18359071,340.28514237)(222.33984261,341.54296515)
\curveto(223.50390089,342.80857734)(225.07030557,343.44138921)(227.03906136,343.44140265)
\curveto(228.80467684,343.44138921)(230.1992067,342.87107728)(231.22265511,341.73046515)
\curveto(232.25389214,340.59764205)(232.76951662,339.05467485)(232.76953011,337.1015589)
\moveto(230.61328011,337.7343714)
\curveto(230.5976438,338.91404999)(230.26561288,339.8554553)(229.61718636,340.55859015)
\curveto(228.97655167,341.26170389)(228.12499002,341.61326604)(227.06249886,341.61327765)
\curveto(225.85936729,341.61326604)(224.8945245,341.27342263)(224.16796761,340.5937464)
\curveto(223.44921345,339.91404899)(223.03515136,338.9570187)(222.92578011,337.72265265)
\lineto(230.61328011,337.7343714)
}
}
{
\newrgbcolor{curcolor}{0 0 0}
\pscustom[linestyle=none,fillstyle=solid,fillcolor=curcolor]
{
\newpath
\moveto(247.21874886,337.9218714)
\lineto(247.21874886,329.9999964)
\lineto(245.06249886,329.9999964)
\lineto(245.06249886,337.8515589)
\curveto(245.06248785,339.09373731)(244.82030059,340.02342388)(244.33593636,340.6406214)
\curveto(243.85155156,341.25779764)(243.12498979,341.56639109)(242.15624886,341.56640265)
\curveto(240.99217942,341.56639109)(240.07421159,341.19529771)(239.40234261,340.4531214)
\curveto(238.73046293,339.71092419)(238.39452577,338.69920645)(238.39453011,337.41796515)
\lineto(238.39453011,329.9999964)
\lineto(236.22656136,329.9999964)
\lineto(236.22656136,343.1249964)
\lineto(238.39453011,343.1249964)
\lineto(238.39453011,341.0859339)
\curveto(238.91015025,341.87498453)(239.51561839,342.46482769)(240.21093636,342.85546515)
\curveto(240.9140545,343.24607691)(241.72264744,343.44138921)(242.63671761,343.44140265)
\curveto(244.14452002,343.44138921)(245.28514388,342.97263968)(246.05859261,342.03515265)
\curveto(246.83201733,341.10545405)(247.21873569,339.73436167)(247.21874886,337.9218714)
}
}
{
\newrgbcolor{curcolor}{0 0 0}
\pscustom[linestyle=none,fillstyle=solid,fillcolor=curcolor]
{
}
}
{
\newrgbcolor{curcolor}{0 0 0}
\pscustom[linestyle=none,fillstyle=solid,fillcolor=curcolor]
{
\newpath
\moveto(269.40234261,340.60546515)
\curveto(269.94139209,341.57420358)(270.5859227,342.28904661)(271.33593636,342.7499964)
\curveto(272.0859212,343.21092069)(272.96873282,343.44138921)(273.98437386,343.44140265)
\curveto(275.35154293,343.44138921)(276.40622938,342.96092094)(277.14843636,341.9999964)
\curveto(277.89060289,341.04686036)(278.26169627,339.68748671)(278.26171761,337.9218714)
\lineto(278.26171761,329.9999964)
\lineto(276.09374886,329.9999964)
\lineto(276.09374886,337.8515589)
\curveto(276.09372969,339.10936229)(275.87107366,340.04295511)(275.42578011,340.65234015)
\curveto(274.98044955,341.26170389)(274.30076273,341.56639109)(273.38671761,341.56640265)
\curveto(272.26951477,341.56639109)(271.38670315,341.19529771)(270.73828011,340.4531214)
\curveto(270.08982945,339.71092419)(269.76561102,338.69920645)(269.76562386,337.41796515)
\lineto(269.76562386,329.9999964)
\lineto(267.59765511,329.9999964)
\lineto(267.59765511,337.8515589)
\curveto(267.59764444,339.11717479)(267.37498841,340.0507676)(266.92968636,340.65234015)
\curveto(266.4843643,341.26170389)(265.79686499,341.56639109)(264.86718636,341.56640265)
\curveto(263.76561702,341.56639109)(262.89061789,341.19139146)(262.24218636,340.44140265)
\curveto(261.59374419,339.69920545)(261.26952577,338.69139396)(261.26953011,337.41796515)
\lineto(261.26953011,329.9999964)
\lineto(259.10156136,329.9999964)
\lineto(259.10156136,343.1249964)
\lineto(261.26953011,343.1249964)
\lineto(261.26953011,341.0859339)
\curveto(261.76171277,341.89060951)(262.35155593,342.48435892)(263.03906136,342.8671839)
\curveto(263.72655456,343.24998315)(264.54295999,343.44138921)(265.48828011,343.44140265)
\curveto(266.44139559,343.44138921)(267.24998854,343.19920195)(267.91406136,342.71484015)
\curveto(268.5859247,342.23045292)(269.08201795,341.52732862)(269.40234261,340.60546515)
}
}
{
\newrgbcolor{curcolor}{0 0 0}
\pscustom[linestyle=none,fillstyle=solid,fillcolor=curcolor]
{
\newpath
\moveto(282.35156136,335.1796839)
\lineto(282.35156136,343.1249964)
\lineto(284.50781136,343.1249964)
\lineto(284.50781136,335.26171515)
\curveto(284.50780717,334.01952363)(284.74999443,333.08593082)(285.23437386,332.4609339)
\curveto(285.71874346,331.84374456)(286.44530523,331.53515112)(287.41406136,331.53515265)
\curveto(288.5781156,331.53515112)(289.49608343,331.9062445)(290.16796761,332.6484339)
\curveto(290.84764458,333.39061801)(291.18748799,334.40233575)(291.18749886,335.68359015)
\lineto(291.18749886,343.1249964)
\lineto(293.34374886,343.1249964)
\lineto(293.34374886,329.9999964)
\lineto(291.18749886,329.9999964)
\lineto(291.18749886,332.0156214)
\curveto(290.66405101,331.21874518)(290.05467662,330.62499578)(289.35937386,330.2343714)
\curveto(288.6718655,329.85155905)(287.87108505,329.66015299)(286.95703011,329.66015265)
\curveto(285.44921248,329.66015299)(284.30468237,330.12890252)(283.52343636,331.06640265)
\curveto(282.74218393,332.00390065)(282.35155932,333.37499303)(282.35156136,335.1796839)
\moveto(287.77734261,343.44140265)
\lineto(287.77734261,343.44140265)
}
}
{
\newrgbcolor{curcolor}{0 0 0}
\pscustom[linestyle=none,fillstyle=solid,fillcolor=curcolor]
{
\newpath
\moveto(297.80859261,348.2343714)
\lineto(299.96484261,348.2343714)
\lineto(299.96484261,329.9999964)
\lineto(297.80859261,329.9999964)
\lineto(297.80859261,348.2343714)
}
}
{
\newrgbcolor{curcolor}{0 0 0}
\pscustom[linestyle=none,fillstyle=solid,fillcolor=curcolor]
{
\newpath
\moveto(306.59765511,346.8515589)
\lineto(306.59765511,343.1249964)
\lineto(311.03906136,343.1249964)
\lineto(311.03906136,341.44921515)
\lineto(306.59765511,341.44921515)
\lineto(306.59765511,334.32421515)
\curveto(306.59765072,333.2538994)(306.74218182,332.56640009)(307.03124886,332.26171515)
\curveto(307.32811874,331.9570257)(307.92577439,331.8046821)(308.82421761,331.8046839)
\lineto(311.03906136,331.8046839)
\lineto(311.03906136,329.9999964)
\lineto(308.82421761,329.9999964)
\curveto(307.16015016,329.9999964)(306.0117138,330.30858984)(305.37890511,330.92577765)
\curveto(304.74609007,331.5507761)(304.42968414,332.68358747)(304.42968636,334.32421515)
\lineto(304.42968636,341.44921515)
\lineto(302.84765511,341.44921515)
\lineto(302.84765511,343.1249964)
\lineto(304.42968636,343.1249964)
\lineto(304.42968636,346.8515589)
\lineto(306.59765511,346.8515589)
}
}
{
\newrgbcolor{curcolor}{0 0 0}
\pscustom[linestyle=none,fillstyle=solid,fillcolor=curcolor]
{
\newpath
\moveto(313.88671761,343.1249964)
\lineto(316.04296761,343.1249964)
\lineto(316.04296761,329.9999964)
\lineto(313.88671761,329.9999964)
\lineto(313.88671761,343.1249964)
\moveto(313.88671761,348.2343714)
\lineto(316.04296761,348.2343714)
\lineto(316.04296761,345.50390265)
\lineto(313.88671761,345.50390265)
\lineto(313.88671761,348.2343714)
}
}
{
\newrgbcolor{curcolor}{0 0 0}
\pscustom[linestyle=none,fillstyle=solid,fillcolor=curcolor]
{
\newpath
\moveto(320.54296761,343.1249964)
\lineto(322.69921761,343.1249964)
\lineto(322.69921761,329.7656214)
\curveto(322.6992132,328.09374831)(322.37890102,326.88281202)(321.73828011,326.1328089)
\curveto(321.10546479,325.38281352)(320.08202831,325.00781389)(318.66796761,325.0078089)
\lineto(317.84765511,325.0078089)
\lineto(317.84765511,326.8359339)
\lineto(318.42187386,326.8359339)
\curveto(319.2421854,326.83593707)(319.80077859,327.02734313)(320.09765511,327.41015265)
\curveto(320.394528,327.78515487)(320.54296535,328.57031033)(320.54296761,329.7656214)
\lineto(320.54296761,343.1249964)
\moveto(320.54296761,348.2343714)
\lineto(322.69921761,348.2343714)
\lineto(322.69921761,345.50390265)
\lineto(320.54296761,345.50390265)
\lineto(320.54296761,348.2343714)
}
}
{
\newrgbcolor{curcolor}{0 0 0}
\pscustom[linestyle=none,fillstyle=solid,fillcolor=curcolor]
{
\newpath
\moveto(332.28515511,341.61327765)
\curveto(331.12889892,341.61326604)(330.21483734,341.16014149)(329.54296761,340.25390265)
\curveto(328.87108868,339.3554558)(328.53515152,338.12108203)(328.53515511,336.55077765)
\curveto(328.53515152,334.98046017)(328.86718243,333.74218016)(329.53124886,332.8359339)
\curveto(330.2031186,331.93749446)(331.12108643,331.48827616)(332.28515511,331.48827765)
\curveto(333.43358412,331.48827616)(334.34373946,331.94140071)(335.01562386,332.84765265)
\curveto(335.68748811,333.7538989)(336.02342528,334.98827266)(336.02343636,336.55077765)
\curveto(336.02342528,338.10545705)(335.68748811,339.33592457)(335.01562386,340.2421839)
\curveto(334.34373946,341.15623525)(333.43358412,341.61326604)(332.28515511,341.61327765)
\moveto(332.28515511,343.44140265)
\curveto(334.16014589,343.44138921)(335.63280067,342.83201482)(336.70312386,341.61327765)
\curveto(337.77342353,340.39451726)(338.30857924,338.70701895)(338.30859261,336.55077765)
\curveto(338.30857924,334.40233575)(337.77342353,332.71483744)(336.70312386,331.48827765)
\curveto(335.63280067,330.26952738)(334.16014589,329.66015299)(332.28515511,329.66015265)
\curveto(330.40233715,329.66015299)(328.92577612,330.26952738)(327.85546761,331.48827765)
\curveto(326.79296576,332.71483744)(326.26171629,334.40233575)(326.26171761,336.55077765)
\curveto(326.26171629,338.70701895)(326.79296576,340.39451726)(327.85546761,341.61327765)
\curveto(328.92577612,342.83201482)(330.40233715,343.44138921)(332.28515511,343.44140265)
}
}
{
\newrgbcolor{curcolor}{0 0 0}
\pscustom[linestyle=none,fillstyle=solid,fillcolor=curcolor]
{
\newpath
\moveto(341.64843636,335.1796839)
\lineto(341.64843636,343.1249964)
\lineto(343.80468636,343.1249964)
\lineto(343.80468636,335.26171515)
\curveto(343.80468217,334.01952363)(344.04686943,333.08593082)(344.53124886,332.4609339)
\curveto(345.01561846,331.84374456)(345.74218023,331.53515112)(346.71093636,331.53515265)
\curveto(347.8749906,331.53515112)(348.79295843,331.9062445)(349.46484261,332.6484339)
\curveto(350.14451958,333.39061801)(350.48436299,334.40233575)(350.48437386,335.68359015)
\lineto(350.48437386,343.1249964)
\lineto(352.64062386,343.1249964)
\lineto(352.64062386,329.9999964)
\lineto(350.48437386,329.9999964)
\lineto(350.48437386,332.0156214)
\curveto(349.96092601,331.21874518)(349.35155162,330.62499578)(348.65624886,330.2343714)
\curveto(347.9687405,329.85155905)(347.16796005,329.66015299)(346.25390511,329.66015265)
\curveto(344.74608748,329.66015299)(343.60155737,330.12890252)(342.82031136,331.06640265)
\curveto(342.03905893,332.00390065)(341.64843432,333.37499303)(341.64843636,335.1796839)
\moveto(347.07421761,343.44140265)
\lineto(347.07421761,343.44140265)
}
}
{
\newrgbcolor{curcolor}{0 0 0}
\pscustom[linestyle=none,fillstyle=solid,fillcolor=curcolor]
{
\newpath
\moveto(368.33203011,337.1015589)
\lineto(368.33203011,336.0468714)
\lineto(358.41796761,336.0468714)
\curveto(358.51171395,334.56249184)(358.957026,333.42968047)(359.75390511,332.6484339)
\curveto(360.5585869,331.87499453)(361.67577328,331.48827616)(363.10546761,331.48827765)
\curveto(363.93358352,331.48827616)(364.73436397,331.58983856)(365.50781136,331.79296515)
\curveto(366.28904992,331.99608816)(367.06248664,332.30077535)(367.82812386,332.70702765)
\lineto(367.82812386,330.66796515)
\curveto(367.05467415,330.33983981)(366.2617062,330.08984006)(365.44921761,329.91796515)
\curveto(364.63670782,329.74609041)(363.81248989,329.66015299)(362.97656136,329.66015265)
\curveto(360.88280532,329.66015299)(359.22265073,330.26952738)(357.99609261,331.48827765)
\curveto(356.77734068,332.70702495)(356.16796629,334.3554608)(356.16796761,336.43359015)
\curveto(356.16796629,338.58201907)(356.74609071,340.28514237)(357.90234261,341.54296515)
\curveto(359.06640089,342.80857734)(360.63280557,343.44138921)(362.60156136,343.44140265)
\curveto(364.36717684,343.44138921)(365.7617067,342.87107728)(366.78515511,341.73046515)
\curveto(367.81639214,340.59764205)(368.33201662,339.05467485)(368.33203011,337.1015589)
\moveto(366.17578011,337.7343714)
\curveto(366.1601438,338.91404999)(365.82811288,339.8554553)(365.17968636,340.55859015)
\curveto(364.53905167,341.26170389)(363.68749002,341.61326604)(362.62499886,341.61327765)
\curveto(361.42186729,341.61326604)(360.4570245,341.27342263)(359.73046761,340.5937464)
\curveto(359.01171345,339.91404899)(358.59765136,338.9570187)(358.48828011,337.72265265)
\lineto(366.17578011,337.7343714)
}
}
{
\newrgbcolor{curcolor}{0 0 0}
\pscustom[linestyle=none,fillstyle=solid,fillcolor=curcolor]
{
\newpath
\moveto(371.64843636,335.1796839)
\lineto(371.64843636,343.1249964)
\lineto(373.80468636,343.1249964)
\lineto(373.80468636,335.26171515)
\curveto(373.80468217,334.01952363)(374.04686943,333.08593082)(374.53124886,332.4609339)
\curveto(375.01561846,331.84374456)(375.74218023,331.53515112)(376.71093636,331.53515265)
\curveto(377.8749906,331.53515112)(378.79295843,331.9062445)(379.46484261,332.6484339)
\curveto(380.14451958,333.39061801)(380.48436299,334.40233575)(380.48437386,335.68359015)
\lineto(380.48437386,343.1249964)
\lineto(382.64062386,343.1249964)
\lineto(382.64062386,329.9999964)
\lineto(380.48437386,329.9999964)
\lineto(380.48437386,332.0156214)
\curveto(379.96092601,331.21874518)(379.35155162,330.62499578)(378.65624886,330.2343714)
\curveto(377.9687405,329.85155905)(377.16796005,329.66015299)(376.25390511,329.66015265)
\curveto(374.74608748,329.66015299)(373.60155737,330.12890252)(372.82031136,331.06640265)
\curveto(372.03905893,332.00390065)(371.64843432,333.37499303)(371.64843636,335.1796839)
\moveto(377.07421761,343.44140265)
\lineto(377.07421761,343.44140265)
}
}
{
\newrgbcolor{curcolor}{0 0 0}
\pscustom[linestyle=none,fillstyle=solid,fillcolor=curcolor]
{
\newpath
\moveto(394.71093636,341.1093714)
\curveto(394.46873924,341.24998515)(394.2031145,341.35154755)(393.91406136,341.4140589)
\curveto(393.63280257,341.48435992)(393.32030289,341.51951613)(392.97656136,341.51952765)
\curveto(391.75780445,341.51951613)(390.82030539,341.12107903)(390.16406136,340.32421515)
\curveto(389.51561919,339.53514312)(389.19140077,338.3984255)(389.19140511,336.9140589)
\lineto(389.19140511,329.9999964)
\lineto(387.02343636,329.9999964)
\lineto(387.02343636,343.1249964)
\lineto(389.19140511,343.1249964)
\lineto(389.19140511,341.0859339)
\curveto(389.64452531,341.88279702)(390.23436847,342.47264018)(390.96093636,342.85546515)
\curveto(391.68749202,343.24607691)(392.57030364,343.44138921)(393.60937386,343.44140265)
\curveto(393.75780245,343.44138921)(393.92186479,343.42967047)(394.10156136,343.4062464)
\curveto(394.28123943,343.39060801)(394.48045798,343.36326429)(394.69921761,343.32421515)
\lineto(394.71093636,341.1093714)
}
}
{
\newrgbcolor{curcolor}{0 0 0}
\pscustom[linestyle=none,fillstyle=solid,fillcolor=curcolor]
{
\newpath
\moveto(405.36328011,342.73827765)
\lineto(405.36328011,340.69921515)
\curveto(404.75389509,341.01170414)(404.12108323,341.24607891)(403.46484261,341.40234015)
\curveto(402.80858454,341.55857859)(402.12889772,341.63670352)(401.42578011,341.63671515)
\curveto(400.35546199,341.63670352)(399.5507753,341.47264118)(399.01171761,341.14452765)
\curveto(398.48046387,340.81639184)(398.21483913,340.32420483)(398.21484261,339.66796515)
\curveto(398.21483913,339.16795598)(398.40624519,338.77342513)(398.78906136,338.4843714)
\curveto(399.17186943,338.2031132)(399.94139991,337.93358222)(401.09765511,337.67577765)
\lineto(401.83593636,337.51171515)
\curveto(403.36717773,337.18358297)(404.45311414,336.71873968)(405.09374886,336.1171839)
\curveto(405.74217536,335.52342838)(406.06639378,334.69139796)(406.06640511,333.62109015)
\curveto(406.06639378,332.40233775)(405.58201927,331.43749496)(404.61328011,330.7265589)
\curveto(403.6523337,330.01562139)(402.32811627,329.66015299)(400.64062386,329.66015265)
\curveto(399.93749366,329.66015299)(399.20311939,329.73046542)(398.43749886,329.87109015)
\curveto(397.67968342,330.00390265)(396.87890297,330.20702745)(396.03515511,330.48046515)
\lineto(396.03515511,332.70702765)
\curveto(396.83202802,332.29296286)(397.61718348,331.98046317)(398.39062386,331.76952765)
\curveto(399.16405693,331.56640109)(399.92968117,331.46483869)(400.68749886,331.46484015)
\curveto(401.70311689,331.46483869)(402.48436611,331.63671352)(403.03124886,331.98046515)
\curveto(403.57811502,332.33202532)(403.85155225,332.82421233)(403.85156136,333.45702765)
\curveto(403.85155225,334.04296111)(403.6523337,334.49217941)(403.25390511,334.8046839)
\curveto(402.86327198,335.11717879)(401.9999916,335.41795973)(400.66406136,335.70702765)
\lineto(399.91406136,335.8828089)
\curveto(398.57812002,336.16405274)(397.61327723,336.59373981)(397.01953011,337.1718714)
\curveto(396.42577842,337.75780114)(396.12890372,338.55858159)(396.12890511,339.57421515)
\curveto(396.12890372,340.80857934)(396.56640328,341.76170339)(397.44140511,342.43359015)
\curveto(398.31640153,343.10545205)(399.55858779,343.44138921)(401.16796761,343.44140265)
\curveto(401.96483538,343.44138921)(402.71483463,343.38279552)(403.41796761,343.2656214)
\curveto(404.12108323,343.14842075)(404.76952008,342.97263968)(405.36328011,342.73827765)
}
}
{
\newrgbcolor{curcolor}{1 1 1}
\pscustom[linestyle=none,fillstyle=solid,fillcolor=curcolor]
{
\newpath
\moveto(230.75260812,299.9999964)
\lineto(330.59985238,299.9999964)
\curveto(335.80753353,299.9999964)(339.99999886,295.80753107)(339.99999886,290.59984992)
\lineto(339.99999886,279.61122916)
\curveto(339.99999886,274.40354801)(335.80753353,270.21108268)(330.59985238,270.21108268)
\lineto(230.75260812,270.21108268)
\curveto(225.54492696,270.21108268)(221.35246163,274.40354801)(221.35246163,279.61122916)
\lineto(221.35246163,290.59984992)
\curveto(221.35246163,295.80753107)(225.54492696,299.9999964)(230.75260812,299.9999964)
\closepath
}
}
{
\newrgbcolor{curcolor}{0 0 0}
\pscustom[linewidth=2,linecolor=curcolor]
{
\newpath
\moveto(230.75260812,299.9999964)
\lineto(330.59985238,299.9999964)
\curveto(335.80753353,299.9999964)(339.99999886,295.80753107)(339.99999886,290.59984992)
\lineto(339.99999886,279.61122916)
\curveto(339.99999886,274.40354801)(335.80753353,270.21108268)(330.59985238,270.21108268)
\lineto(230.75260812,270.21108268)
\curveto(225.54492696,270.21108268)(221.35246163,274.40354801)(221.35246163,279.61122916)
\lineto(221.35246163,290.59984992)
\curveto(221.35246163,295.80753107)(225.54492696,299.9999964)(230.75260812,299.9999964)
\closepath
}
}
{
\newrgbcolor{curcolor}{0 0 0}
\pscustom[linestyle=none,fillstyle=solid,fillcolor=curcolor]
{
\newpath
\moveto(245.21453744,293.55099128)
\curveto(243.4957797,293.55097538)(242.12859357,292.91035103)(241.11297494,291.62911628)
\curveto(240.10515809,290.34785359)(239.60125234,288.60176158)(239.60125619,286.39083503)
\curveto(239.60125234,284.1877035)(240.10515809,282.44551774)(241.11297494,281.16427253)
\curveto(242.12859357,279.8830203)(243.4957797,279.24239594)(245.21453744,279.24239753)
\curveto(246.93327626,279.24239594)(248.2926499,279.8830203)(249.29266244,281.16427253)
\curveto(250.30046039,282.44551774)(250.80436614,284.1877035)(250.80438119,286.39083503)
\curveto(250.80436614,288.60176158)(250.30046039,290.34785359)(249.29266244,291.62911628)
\curveto(248.2926499,292.91035103)(246.93327626,293.55097538)(245.21453744,293.55099128)
\moveto(245.21453744,295.47286628)
\curveto(247.66765053,295.47284846)(249.62858607,294.64863054)(251.09734994,293.00021003)
\curveto(252.56608313,291.35957133)(253.30045739,289.15644853)(253.30047494,286.39083503)
\curveto(253.30045739,283.63301655)(252.56608313,281.42989376)(251.09734994,279.78146003)
\curveto(249.62858607,278.14083454)(247.66765053,277.32052287)(245.21453744,277.32052253)
\curveto(242.75359294,277.32052287)(240.78484491,278.14083454)(239.30828744,279.78146003)
\curveto(237.83953536,281.42208126)(237.10516109,283.62520406)(237.10516244,286.39083503)
\curveto(237.10516109,289.15644853)(237.83953536,291.35957133)(239.30828744,293.00021003)
\curveto(240.78484491,294.64863054)(242.75359294,295.47284846)(245.21453744,295.47286628)
}
}
{
\newrgbcolor{curcolor}{0 0 0}
\pscustom[linestyle=none,fillstyle=solid,fillcolor=curcolor]
{
\newpath
\moveto(258.99578744,279.62911628)
\lineto(258.99578744,272.66817878)
\lineto(256.82781869,272.66817878)
\lineto(256.82781869,290.78536628)
\lineto(258.99578744,290.78536628)
\lineto(258.99578744,288.79317878)
\curveto(259.44890764,289.57441686)(260.01921957,290.15254128)(260.70672494,290.52755378)
\curveto(261.40203068,290.91035303)(262.23015486,291.10175908)(263.19109994,291.10177253)
\curveto(264.7848398,291.10175908)(266.07780726,290.46894722)(267.07000619,289.20333503)
\curveto(268.06999277,287.93769975)(268.56999227,286.27363891)(268.57000619,284.21114753)
\curveto(268.56999227,282.14864304)(268.06999277,280.4845822)(267.07000619,279.21896003)
\curveto(266.07780726,277.95333473)(264.7848398,277.32052287)(263.19109994,277.32052253)
\curveto(262.23015486,277.32052287)(261.40203068,277.50802268)(260.70672494,277.88302253)
\curveto(260.01921957,278.26583442)(259.44890764,278.84786509)(258.99578744,279.62911628)
\moveto(266.33172494,284.21114753)
\curveto(266.33171325,285.79707689)(266.00358858,287.03926315)(265.34734994,287.93771003)
\curveto(264.69890239,288.84394884)(263.80437203,289.29707339)(262.66375619,289.29708503)
\curveto(261.52312431,289.29707339)(260.62468771,288.84394884)(259.96844369,287.93771003)
\curveto(259.32000152,287.03926315)(258.99578309,285.79707689)(258.99578744,284.21114753)
\curveto(258.99578309,282.62520506)(259.32000152,281.37911256)(259.96844369,280.47286628)
\curveto(260.62468771,279.57442686)(261.52312431,279.12520856)(262.66375619,279.12521003)
\curveto(263.80437203,279.12520856)(264.69890239,279.57442686)(265.34734994,280.47286628)
\curveto(266.00358858,281.37911256)(266.33171325,282.62520506)(266.33172494,284.21114753)
}
}
{
\newrgbcolor{curcolor}{0 0 0}
\pscustom[linestyle=none,fillstyle=solid,fillcolor=curcolor]
{
\newpath
\moveto(274.27703744,294.51192878)
\lineto(274.27703744,290.78536628)
\lineto(278.71844369,290.78536628)
\lineto(278.71844369,289.10958503)
\lineto(274.27703744,289.10958503)
\lineto(274.27703744,281.98458503)
\curveto(274.27703304,280.91426927)(274.42156415,280.22676996)(274.71063119,279.92208503)
\curveto(275.00750106,279.61739557)(275.60515671,279.46505197)(276.50359994,279.46505378)
\lineto(278.71844369,279.46505378)
\lineto(278.71844369,277.66036628)
\lineto(276.50359994,277.66036628)
\curveto(274.83953248,277.66036628)(273.69109613,277.96895972)(273.05828744,278.58614753)
\curveto(272.42547239,279.21114597)(272.10906646,280.34395734)(272.10906869,281.98458503)
\lineto(272.10906869,289.10958503)
\lineto(270.52703744,289.10958503)
\lineto(270.52703744,290.78536628)
\lineto(272.10906869,290.78536628)
\lineto(272.10906869,294.51192878)
\lineto(274.27703744,294.51192878)
}
}
{
\newrgbcolor{curcolor}{0 0 0}
\pscustom[linestyle=none,fillstyle=solid,fillcolor=curcolor]
{
\newpath
\moveto(281.56609994,290.78536628)
\lineto(283.72234994,290.78536628)
\lineto(283.72234994,277.66036628)
\lineto(281.56609994,277.66036628)
\lineto(281.56609994,290.78536628)
\moveto(281.56609994,295.89474128)
\lineto(283.72234994,295.89474128)
\lineto(283.72234994,293.16427253)
\lineto(281.56609994,293.16427253)
\lineto(281.56609994,295.89474128)
}
}
{
\newrgbcolor{curcolor}{0 0 0}
\pscustom[linestyle=none,fillstyle=solid,fillcolor=curcolor]
{
\newpath
\moveto(293.30828744,289.27364753)
\curveto(292.15203125,289.27363591)(291.23796966,288.82051137)(290.56609994,287.91427253)
\curveto(289.894221,287.01582567)(289.55828384,285.7814519)(289.55828744,284.21114753)
\curveto(289.55828384,282.64083004)(289.89031476,281.40255003)(290.55438119,280.49630378)
\curveto(291.22625092,279.59786434)(292.14421875,279.14864604)(293.30828744,279.14864753)
\curveto(294.45671644,279.14864604)(295.36687178,279.60177058)(296.03875619,280.50802253)
\curveto(296.71062044,281.41426877)(297.0465576,282.64864254)(297.04656869,284.21114753)
\curveto(297.0465576,285.76582692)(296.71062044,286.99629444)(296.03875619,287.90255378)
\curveto(295.36687178,288.81660512)(294.45671644,289.27363591)(293.30828744,289.27364753)
\moveto(293.30828744,291.10177253)
\curveto(295.18327821,291.10175908)(296.65593299,290.49238469)(297.72625619,289.27364753)
\curveto(298.79655585,288.05488713)(299.33171157,286.36738882)(299.33172494,284.21114753)
\curveto(299.33171157,282.06270562)(298.79655585,280.37520731)(297.72625619,279.14864753)
\curveto(296.65593299,277.92989726)(295.18327821,277.32052287)(293.30828744,277.32052253)
\curveto(291.42546947,277.32052287)(289.94890845,277.92989726)(288.87859994,279.14864753)
\curveto(287.81609808,280.37520731)(287.28484861,282.06270562)(287.28484994,284.21114753)
\curveto(287.28484861,286.36738882)(287.81609808,288.05488713)(288.87859994,289.27364753)
\curveto(289.94890845,290.49238469)(291.42546947,291.10175908)(293.30828744,291.10177253)
}
}
{
\newrgbcolor{curcolor}{0 0 0}
\pscustom[linestyle=none,fillstyle=solid,fillcolor=curcolor]
{
\newpath
\moveto(313.80438119,285.58224128)
\lineto(313.80438119,277.66036628)
\lineto(311.64813119,277.66036628)
\lineto(311.64813119,285.51192878)
\curveto(311.64812017,286.75410718)(311.40593291,287.68379375)(310.92156869,288.30099128)
\curveto(310.43718388,288.91816752)(309.71062211,289.22676096)(308.74188119,289.22677253)
\curveto(307.57781174,289.22676096)(306.65984391,288.85566758)(305.98797494,288.11349128)
\curveto(305.31609525,287.37129406)(304.98015809,286.35957633)(304.98016244,285.07833503)
\lineto(304.98016244,277.66036628)
\lineto(302.81219369,277.66036628)
\lineto(302.81219369,290.78536628)
\lineto(304.98016244,290.78536628)
\lineto(304.98016244,288.74630378)
\curveto(305.49578257,289.5353544)(306.10125072,290.12519756)(306.79656869,290.51583503)
\curveto(307.49968682,290.90644678)(308.30827976,291.10175908)(309.22234994,291.10177253)
\curveto(310.73015234,291.10175908)(311.8707762,290.63300955)(312.64422494,289.69552253)
\curveto(313.41764965,288.76582392)(313.80436802,287.39473154)(313.80438119,285.58224128)
}
}
{
\newrgbcolor{curcolor}{0 0 0}
\pscustom[linestyle=none,fillstyle=solid,fillcolor=curcolor]
{
\newpath
\moveto(326.49578744,290.39864753)
\lineto(326.49578744,288.35958503)
\curveto(325.88640242,288.67207401)(325.25359055,288.90644878)(324.59734994,289.06271003)
\curveto(323.94109186,289.21894847)(323.26140504,289.29707339)(322.55828744,289.29708503)
\curveto(321.48796932,289.29707339)(320.68328262,289.13301105)(320.14422494,288.80489753)
\curveto(319.61297119,288.47676171)(319.34734646,287.9845747)(319.34734994,287.32833503)
\curveto(319.34734646,286.82832586)(319.53875252,286.433795)(319.92156869,286.14474128)
\curveto(320.30437675,285.86348307)(321.07390723,285.59395209)(322.23016244,285.33614753)
\lineto(322.96844369,285.17208503)
\curveto(324.49968505,284.84395284)(325.58562147,284.37910956)(326.22625619,283.77755378)
\curveto(326.87468268,283.18379825)(327.19890111,282.35176783)(327.19891244,281.28146003)
\curveto(327.19890111,280.06270762)(326.71452659,279.09786484)(325.74578744,278.38692878)
\curveto(324.78484102,277.67599126)(323.46062359,277.32052287)(321.77313119,277.32052253)
\curveto(321.07000098,277.32052287)(320.33562672,277.39083529)(319.57000619,277.53146003)
\curveto(318.81219074,277.66427252)(318.01141029,277.86739732)(317.16766244,278.14083503)
\lineto(317.16766244,280.36739753)
\curveto(317.96453534,279.95333273)(318.7496908,279.64083304)(319.52313119,279.42989753)
\curveto(320.29656426,279.22677096)(321.06218849,279.12520856)(321.82000619,279.12521003)
\curveto(322.83562422,279.12520856)(323.61687344,279.29708339)(324.16375619,279.64083503)
\curveto(324.71062234,279.99239519)(324.98405957,280.4845822)(324.98406869,281.11739753)
\curveto(324.98405957,281.70333098)(324.78484102,282.15254928)(324.38641244,282.46505378)
\curveto(323.99577931,282.77754866)(323.13249892,283.07832961)(321.79656869,283.36739753)
\lineto(321.04656869,283.54317878)
\curveto(319.71062734,283.82442261)(318.74578456,284.25410968)(318.15203744,284.83224128)
\curveto(317.55828575,285.41817102)(317.26141104,286.21895147)(317.26141244,287.23458503)
\curveto(317.26141104,288.46894922)(317.69891061,289.42207326)(318.57391244,290.09396003)
\curveto(319.44890886,290.76582192)(320.69109511,291.10175908)(322.30047494,291.10177253)
\curveto(323.09734271,291.10175908)(323.84734196,291.04316539)(324.55047494,290.92599128)
\curveto(325.25359055,290.80879063)(325.9020274,290.63300955)(326.49578744,290.39864753)
}
}
{
\newrgbcolor{curcolor}{1 1 1}
\pscustom[linestyle=none,fillstyle=solid,fillcolor=curcolor]
{
\newpath
\moveto(182.63554459,249.84527228)
\lineto(378.59052735,249.84527228)
\curveto(384.94255905,249.84527228)(390.05628854,244.73154279)(390.05628854,238.3795111)
\lineto(390.05628854,231.39312384)
\curveto(390.05628854,225.04109215)(384.94255905,219.92736266)(378.59052735,219.92736266)
\lineto(182.63554459,219.92736266)
\curveto(176.2835129,219.92736266)(171.16978341,225.04109215)(171.16978341,231.39312384)
\lineto(171.16978341,238.3795111)
\curveto(171.16978341,244.73154279)(176.2835129,249.84527228)(182.63554459,249.84527228)
\closepath
}
}
{
\newrgbcolor{curcolor}{0 0 0}
\pscustom[linewidth=1.5483532,linecolor=curcolor]
{
\newpath
\moveto(182.63554459,249.84527228)
\lineto(378.59052735,249.84527228)
\curveto(384.94255905,249.84527228)(390.05628854,244.73154279)(390.05628854,238.3795111)
\lineto(390.05628854,231.39312384)
\curveto(390.05628854,225.04109215)(384.94255905,219.92736266)(378.59052735,219.92736266)
\lineto(182.63554459,219.92736266)
\curveto(176.2835129,219.92736266)(171.16978341,225.04109215)(171.16978341,231.39312384)
\lineto(171.16978341,238.3795111)
\curveto(171.16978341,244.73154279)(176.2835129,249.84527228)(182.63554459,249.84527228)
\closepath
}
}
{
\newrgbcolor{curcolor}{0 0 0}
\pscustom[linestyle=none,fillstyle=solid,fillcolor=curcolor]
{
\newpath
\moveto(192.35546761,247.49609015)
\lineto(195.88281136,247.49609015)
\lineto(200.34765511,235.58984015)
\lineto(204.83593636,247.49609015)
\lineto(208.36328011,247.49609015)
\lineto(208.36328011,229.9999964)
\lineto(206.05468636,229.9999964)
\lineto(206.05468636,245.36327765)
\lineto(201.54296761,233.36327765)
\lineto(199.16406136,233.36327765)
\lineto(194.65234261,245.36327765)
\lineto(194.65234261,229.9999964)
\lineto(192.35546761,229.9999964)
\lineto(192.35546761,247.49609015)
}
}
{
\newrgbcolor{curcolor}{0 0 0}
\pscustom[linestyle=none,fillstyle=solid,fillcolor=curcolor]
{
\newpath
\moveto(224.20703011,237.1015589)
\lineto(224.20703011,236.0468714)
\lineto(214.29296761,236.0468714)
\curveto(214.38671395,234.56249184)(214.832026,233.42968047)(215.62890511,232.6484339)
\curveto(216.4335869,231.87499453)(217.55077328,231.48827616)(218.98046761,231.48827765)
\curveto(219.80858352,231.48827616)(220.60936397,231.58983856)(221.38281136,231.79296515)
\curveto(222.16404992,231.99608816)(222.93748664,232.30077535)(223.70312386,232.70702765)
\lineto(223.70312386,230.66796515)
\curveto(222.92967415,230.33983981)(222.1367062,230.08984006)(221.32421761,229.91796515)
\curveto(220.51170782,229.74609041)(219.68748989,229.66015299)(218.85156136,229.66015265)
\curveto(216.75780532,229.66015299)(215.09765073,230.26952738)(213.87109261,231.48827765)
\curveto(212.65234068,232.70702495)(212.04296629,234.3554608)(212.04296761,236.43359015)
\curveto(212.04296629,238.58201907)(212.62109071,240.28514237)(213.77734261,241.54296515)
\curveto(214.94140089,242.80857734)(216.50780557,243.44138921)(218.47656136,243.44140265)
\curveto(220.24217684,243.44138921)(221.6367067,242.87107728)(222.66015511,241.73046515)
\curveto(223.69139214,240.59764205)(224.20701662,239.05467485)(224.20703011,237.1015589)
\moveto(222.05078011,237.7343714)
\curveto(222.0351438,238.91404999)(221.70311288,239.8554553)(221.05468636,240.55859015)
\curveto(220.41405167,241.26170389)(219.56249002,241.61326604)(218.49999886,241.61327765)
\curveto(217.29686729,241.61326604)(216.3320245,241.27342263)(215.60546761,240.5937464)
\curveto(214.88671345,239.91404899)(214.47265136,238.9570187)(214.36328011,237.72265265)
\lineto(222.05078011,237.7343714)
}
}
{
\newrgbcolor{curcolor}{0 0 0}
\pscustom[linestyle=none,fillstyle=solid,fillcolor=curcolor]
{
\newpath
\moveto(236.11328011,242.73827765)
\lineto(236.11328011,240.69921515)
\curveto(235.50389509,241.01170414)(234.87108323,241.24607891)(234.21484261,241.40234015)
\curveto(233.55858454,241.55857859)(232.87889772,241.63670352)(232.17578011,241.63671515)
\curveto(231.10546199,241.63670352)(230.3007753,241.47264118)(229.76171761,241.14452765)
\curveto(229.23046387,240.81639184)(228.96483913,240.32420483)(228.96484261,239.66796515)
\curveto(228.96483913,239.16795598)(229.15624519,238.77342513)(229.53906136,238.4843714)
\curveto(229.92186943,238.2031132)(230.69139991,237.93358222)(231.84765511,237.67577765)
\lineto(232.58593636,237.51171515)
\curveto(234.11717773,237.18358297)(235.20311414,236.71873968)(235.84374886,236.1171839)
\curveto(236.49217536,235.52342838)(236.81639378,234.69139796)(236.81640511,233.62109015)
\curveto(236.81639378,232.40233775)(236.33201927,231.43749496)(235.36328011,230.7265589)
\curveto(234.4023337,230.01562139)(233.07811627,229.66015299)(231.39062386,229.66015265)
\curveto(230.68749366,229.66015299)(229.95311939,229.73046542)(229.18749886,229.87109015)
\curveto(228.42968342,230.00390265)(227.62890297,230.20702745)(226.78515511,230.48046515)
\lineto(226.78515511,232.70702765)
\curveto(227.58202802,232.29296286)(228.36718348,231.98046317)(229.14062386,231.76952765)
\curveto(229.91405693,231.56640109)(230.67968117,231.46483869)(231.43749886,231.46484015)
\curveto(232.45311689,231.46483869)(233.23436611,231.63671352)(233.78124886,231.98046515)
\curveto(234.32811502,232.33202532)(234.60155225,232.82421233)(234.60156136,233.45702765)
\curveto(234.60155225,234.04296111)(234.4023337,234.49217941)(234.00390511,234.8046839)
\curveto(233.61327198,235.11717879)(232.7499916,235.41795973)(231.41406136,235.70702765)
\lineto(230.66406136,235.8828089)
\curveto(229.32812002,236.16405274)(228.36327723,236.59373981)(227.76953011,237.1718714)
\curveto(227.17577842,237.75780114)(226.87890372,238.55858159)(226.87890511,239.57421515)
\curveto(226.87890372,240.80857934)(227.31640328,241.76170339)(228.19140511,242.43359015)
\curveto(229.06640153,243.10545205)(230.30858779,243.44138921)(231.91796761,243.44140265)
\curveto(232.71483538,243.44138921)(233.46483463,243.38279552)(234.16796761,243.2656214)
\curveto(234.87108323,243.14842075)(235.51952008,242.97263968)(236.11328011,242.73827765)
}
}
{
\newrgbcolor{curcolor}{0 0 0}
\pscustom[linestyle=none,fillstyle=solid,fillcolor=curcolor]
{
}
}
{
\newrgbcolor{curcolor}{0 0 0}
\pscustom[linestyle=none,fillstyle=solid,fillcolor=curcolor]
{
\newpath
\moveto(256.26953011,242.73827765)
\lineto(256.26953011,240.69921515)
\curveto(255.66014509,241.01170414)(255.02733323,241.24607891)(254.37109261,241.40234015)
\curveto(253.71483454,241.55857859)(253.03514772,241.63670352)(252.33203011,241.63671515)
\curveto(251.26171199,241.63670352)(250.4570253,241.47264118)(249.91796761,241.14452765)
\curveto(249.38671387,240.81639184)(249.12108913,240.32420483)(249.12109261,239.66796515)
\curveto(249.12108913,239.16795598)(249.31249519,238.77342513)(249.69531136,238.4843714)
\curveto(250.07811943,238.2031132)(250.84764991,237.93358222)(252.00390511,237.67577765)
\lineto(252.74218636,237.51171515)
\curveto(254.27342773,237.18358297)(255.35936414,236.71873968)(255.99999886,236.1171839)
\curveto(256.64842536,235.52342838)(256.97264378,234.69139796)(256.97265511,233.62109015)
\curveto(256.97264378,232.40233775)(256.48826927,231.43749496)(255.51953011,230.7265589)
\curveto(254.5585837,230.01562139)(253.23436627,229.66015299)(251.54687386,229.66015265)
\curveto(250.84374366,229.66015299)(250.10936939,229.73046542)(249.34374886,229.87109015)
\curveto(248.58593342,230.00390265)(247.78515297,230.20702745)(246.94140511,230.48046515)
\lineto(246.94140511,232.70702765)
\curveto(247.73827802,232.29296286)(248.52343348,231.98046317)(249.29687386,231.76952765)
\curveto(250.07030693,231.56640109)(250.83593117,231.46483869)(251.59374886,231.46484015)
\curveto(252.60936689,231.46483869)(253.39061611,231.63671352)(253.93749886,231.98046515)
\curveto(254.48436502,232.33202532)(254.75780225,232.82421233)(254.75781136,233.45702765)
\curveto(254.75780225,234.04296111)(254.5585837,234.49217941)(254.16015511,234.8046839)
\curveto(253.76952198,235.11717879)(252.9062416,235.41795973)(251.57031136,235.70702765)
\lineto(250.82031136,235.8828089)
\curveto(249.48437002,236.16405274)(248.51952723,236.59373981)(247.92578011,237.1718714)
\curveto(247.33202842,237.75780114)(247.03515372,238.55858159)(247.03515511,239.57421515)
\curveto(247.03515372,240.80857934)(247.47265328,241.76170339)(248.34765511,242.43359015)
\curveto(249.22265153,243.10545205)(250.46483779,243.44138921)(252.07421761,243.44140265)
\curveto(252.87108538,243.44138921)(253.62108463,243.38279552)(254.32421761,243.2656214)
\curveto(255.02733323,243.14842075)(255.67577008,242.97263968)(256.26953011,242.73827765)
}
}
{
\newrgbcolor{curcolor}{0 0 0}
\pscustom[linestyle=none,fillstyle=solid,fillcolor=curcolor]
{
\newpath
\moveto(262.55078011,246.8515589)
\lineto(262.55078011,243.1249964)
\lineto(266.99218636,243.1249964)
\lineto(266.99218636,241.44921515)
\lineto(262.55078011,241.44921515)
\lineto(262.55078011,234.32421515)
\curveto(262.55077572,233.2538994)(262.69530682,232.56640009)(262.98437386,232.26171515)
\curveto(263.28124374,231.9570257)(263.87889939,231.8046821)(264.77734261,231.8046839)
\lineto(266.99218636,231.8046839)
\lineto(266.99218636,229.9999964)
\lineto(264.77734261,229.9999964)
\curveto(263.11327516,229.9999964)(261.9648388,230.30858984)(261.33203011,230.92577765)
\curveto(260.69921507,231.5507761)(260.38280914,232.68358747)(260.38281136,234.32421515)
\lineto(260.38281136,241.44921515)
\lineto(258.80078011,241.44921515)
\lineto(258.80078011,243.1249964)
\lineto(260.38281136,243.1249964)
\lineto(260.38281136,246.8515589)
\lineto(262.55078011,246.8515589)
}
}
{
\newrgbcolor{curcolor}{0 0 0}
\pscustom[linestyle=none,fillstyle=solid,fillcolor=curcolor]
{
\newpath
\moveto(275.80468636,236.59765265)
\curveto(274.06249238,236.59764605)(272.85546234,236.3984275)(272.18359261,235.9999964)
\curveto(271.51171368,235.6015533)(271.17577652,234.92186648)(271.17578011,233.9609339)
\curveto(271.17577652,233.19530571)(271.42577627,232.58593132)(271.92578011,232.1328089)
\curveto(272.43358776,231.68749471)(273.12108707,231.46483869)(273.98828011,231.46484015)
\curveto(275.18358501,231.46483869)(276.1406153,231.88671327)(276.85937386,232.73046515)
\curveto(277.58592636,233.58202407)(277.94920724,234.71092919)(277.94921761,236.1171839)
\lineto(277.94921761,236.59765265)
\lineto(275.80468636,236.59765265)
\moveto(280.10546761,237.48827765)
\lineto(280.10546761,229.9999964)
\lineto(277.94921761,229.9999964)
\lineto(277.94921761,231.9921839)
\curveto(277.45702023,231.19530771)(276.8437396,230.60546455)(276.10937386,230.22265265)
\curveto(275.37499107,229.8476528)(274.47655446,229.66015299)(273.41406136,229.66015265)
\curveto(272.07030687,229.66015299)(270.99999544,230.03515262)(270.20312386,230.78515265)
\curveto(269.41405953,231.54296361)(269.01952867,232.55468135)(269.01953011,233.8203089)
\curveto(269.01952867,235.29686611)(269.51171568,236.41014624)(270.49609261,237.16015265)
\curveto(271.4882762,237.91014474)(272.96483723,238.28514437)(274.92578011,238.28515265)
\lineto(277.94921761,238.28515265)
\lineto(277.94921761,238.49609015)
\curveto(277.94920724,239.48826816)(277.62108257,240.2538924)(276.96484261,240.79296515)
\curveto(276.31639638,241.33982881)(275.40233479,241.61326604)(274.22265511,241.61327765)
\curveto(273.47264922,241.61326604)(272.7421812,241.52342238)(272.03124886,241.3437464)
\curveto(271.32030762,241.16404774)(270.63671455,240.89451676)(269.98046761,240.53515265)
\lineto(269.98046761,242.52734015)
\curveto(270.76952692,242.83201482)(271.53515116,243.05857709)(272.27734261,243.20702765)
\curveto(273.01952467,243.36326429)(273.7421802,243.44138921)(274.44531136,243.44140265)
\curveto(276.3437401,243.44138921)(277.76170743,242.9492022)(278.69921761,241.96484015)
\curveto(279.63670555,240.98045417)(280.10545509,239.48826816)(280.10546761,237.48827765)
}
}
{
\newrgbcolor{curcolor}{0 0 0}
\pscustom[linestyle=none,fillstyle=solid,fillcolor=curcolor]
{
\newpath
\moveto(286.69140511,246.8515589)
\lineto(286.69140511,243.1249964)
\lineto(291.13281136,243.1249964)
\lineto(291.13281136,241.44921515)
\lineto(286.69140511,241.44921515)
\lineto(286.69140511,234.32421515)
\curveto(286.69140072,233.2538994)(286.83593182,232.56640009)(287.12499886,232.26171515)
\curveto(287.42186874,231.9570257)(288.01952439,231.8046821)(288.91796761,231.8046839)
\lineto(291.13281136,231.8046839)
\lineto(291.13281136,229.9999964)
\lineto(288.91796761,229.9999964)
\curveto(287.25390016,229.9999964)(286.1054638,230.30858984)(285.47265511,230.92577765)
\curveto(284.83984007,231.5507761)(284.52343414,232.68358747)(284.52343636,234.32421515)
\lineto(284.52343636,241.44921515)
\lineto(282.94140511,241.44921515)
\lineto(282.94140511,243.1249964)
\lineto(284.52343636,243.1249964)
\lineto(284.52343636,246.8515589)
\lineto(286.69140511,246.8515589)
}
}
{
\newrgbcolor{curcolor}{0 0 0}
\pscustom[linestyle=none,fillstyle=solid,fillcolor=curcolor]
{
\newpath
\moveto(293.98046761,243.1249964)
\lineto(296.13671761,243.1249964)
\lineto(296.13671761,229.9999964)
\lineto(293.98046761,229.9999964)
\lineto(293.98046761,243.1249964)
\moveto(293.98046761,248.2343714)
\lineto(296.13671761,248.2343714)
\lineto(296.13671761,245.50390265)
\lineto(293.98046761,245.50390265)
\lineto(293.98046761,248.2343714)
}
}
{
\newrgbcolor{curcolor}{0 0 0}
\pscustom[linestyle=none,fillstyle=solid,fillcolor=curcolor]
{
\newpath
\moveto(309.00390511,242.73827765)
\lineto(309.00390511,240.69921515)
\curveto(308.39452009,241.01170414)(307.76170823,241.24607891)(307.10546761,241.40234015)
\curveto(306.44920954,241.55857859)(305.76952272,241.63670352)(305.06640511,241.63671515)
\curveto(303.99608699,241.63670352)(303.1914003,241.47264118)(302.65234261,241.14452765)
\curveto(302.12108887,240.81639184)(301.85546413,240.32420483)(301.85546761,239.66796515)
\curveto(301.85546413,239.16795598)(302.04687019,238.77342513)(302.42968636,238.4843714)
\curveto(302.81249443,238.2031132)(303.58202491,237.93358222)(304.73828011,237.67577765)
\lineto(305.47656136,237.51171515)
\curveto(307.00780273,237.18358297)(308.09373914,236.71873968)(308.73437386,236.1171839)
\curveto(309.38280036,235.52342838)(309.70701878,234.69139796)(309.70703011,233.62109015)
\curveto(309.70701878,232.40233775)(309.22264427,231.43749496)(308.25390511,230.7265589)
\curveto(307.2929587,230.01562139)(305.96874127,229.66015299)(304.28124886,229.66015265)
\curveto(303.57811866,229.66015299)(302.84374439,229.73046542)(302.07812386,229.87109015)
\curveto(301.32030842,230.00390265)(300.51952797,230.20702745)(299.67578011,230.48046515)
\lineto(299.67578011,232.70702765)
\curveto(300.47265302,232.29296286)(301.25780848,231.98046317)(302.03124886,231.76952765)
\curveto(302.80468193,231.56640109)(303.57030617,231.46483869)(304.32812386,231.46484015)
\curveto(305.34374189,231.46483869)(306.12499111,231.63671352)(306.67187386,231.98046515)
\curveto(307.21874002,232.33202532)(307.49217725,232.82421233)(307.49218636,233.45702765)
\curveto(307.49217725,234.04296111)(307.2929587,234.49217941)(306.89453011,234.8046839)
\curveto(306.50389698,235.11717879)(305.6406166,235.41795973)(304.30468636,235.70702765)
\lineto(303.55468636,235.8828089)
\curveto(302.21874502,236.16405274)(301.25390223,236.59373981)(300.66015511,237.1718714)
\curveto(300.06640342,237.75780114)(299.76952872,238.55858159)(299.76953011,239.57421515)
\curveto(299.76952872,240.80857934)(300.20702828,241.76170339)(301.08203011,242.43359015)
\curveto(301.95702653,243.10545205)(303.19921279,243.44138921)(304.80859261,243.44140265)
\curveto(305.60546038,243.44138921)(306.35545963,243.38279552)(307.05859261,243.2656214)
\curveto(307.76170823,243.14842075)(308.41014508,242.97263968)(309.00390511,242.73827765)
}
}
{
\newrgbcolor{curcolor}{0 0 0}
\pscustom[linestyle=none,fillstyle=solid,fillcolor=curcolor]
{
\newpath
\moveto(315.28515511,246.8515589)
\lineto(315.28515511,243.1249964)
\lineto(319.72656136,243.1249964)
\lineto(319.72656136,241.44921515)
\lineto(315.28515511,241.44921515)
\lineto(315.28515511,234.32421515)
\curveto(315.28515072,233.2538994)(315.42968182,232.56640009)(315.71874886,232.26171515)
\curveto(316.01561874,231.9570257)(316.61327439,231.8046821)(317.51171761,231.8046839)
\lineto(319.72656136,231.8046839)
\lineto(319.72656136,229.9999964)
\lineto(317.51171761,229.9999964)
\curveto(315.84765016,229.9999964)(314.6992138,230.30858984)(314.06640511,230.92577765)
\curveto(313.43359007,231.5507761)(313.11718414,232.68358747)(313.11718636,234.32421515)
\lineto(313.11718636,241.44921515)
\lineto(311.53515511,241.44921515)
\lineto(311.53515511,243.1249964)
\lineto(313.11718636,243.1249964)
\lineto(313.11718636,246.8515589)
\lineto(315.28515511,246.8515589)
}
}
{
\newrgbcolor{curcolor}{0 0 0}
\pscustom[linestyle=none,fillstyle=solid,fillcolor=curcolor]
{
\newpath
\moveto(322.57421761,243.1249964)
\lineto(324.73046761,243.1249964)
\lineto(324.73046761,229.9999964)
\lineto(322.57421761,229.9999964)
\lineto(322.57421761,243.1249964)
\moveto(322.57421761,248.2343714)
\lineto(324.73046761,248.2343714)
\lineto(324.73046761,245.50390265)
\lineto(322.57421761,245.50390265)
\lineto(322.57421761,248.2343714)
}
}
{
\newrgbcolor{curcolor}{0 0 0}
\pscustom[linestyle=none,fillstyle=solid,fillcolor=curcolor]
{
\newpath
\moveto(330.51953011,236.55077765)
\curveto(330.51952656,234.96483519)(330.84374499,233.71874268)(331.49218636,232.8124964)
\curveto(332.14843118,231.91405699)(333.04686779,231.46483869)(334.18749886,231.46484015)
\curveto(335.3281155,231.46483869)(336.22655211,231.91405699)(336.88281136,232.8124964)
\curveto(337.53905079,233.71874268)(337.86717546,234.96483519)(337.86718636,236.55077765)
\curveto(337.86717546,238.13670702)(337.53905079,239.37889327)(336.88281136,240.27734015)
\curveto(336.22655211,241.18357897)(335.3281155,241.63670352)(334.18749886,241.63671515)
\curveto(333.04686779,241.63670352)(332.14843118,241.18357897)(331.49218636,240.27734015)
\curveto(330.84374499,239.37889327)(330.51952656,238.13670702)(330.51953011,236.55077765)
\moveto(337.86718636,231.9687464)
\curveto(337.41405092,231.18749521)(336.83983274,230.60546455)(336.14453011,230.22265265)
\curveto(335.45702162,229.8476528)(334.62889745,229.66015299)(333.66015511,229.66015265)
\curveto(332.07421251,229.66015299)(330.78124505,230.29296486)(329.78124886,231.55859015)
\curveto(328.78905954,232.82421233)(328.29296629,234.48827316)(328.29296761,236.55077765)
\curveto(328.29296629,238.61326904)(328.78905954,240.27732988)(329.78124886,241.54296515)
\curveto(330.78124505,242.80857734)(332.07421251,243.44138921)(333.66015511,243.44140265)
\curveto(334.62889745,243.44138921)(335.45702162,243.24998315)(336.14453011,242.8671839)
\curveto(336.83983274,242.49217141)(337.41405092,241.91404699)(337.86718636,241.1328089)
\lineto(337.86718636,243.1249964)
\lineto(340.02343636,243.1249964)
\lineto(340.02343636,225.0078089)
\lineto(337.86718636,225.0078089)
\lineto(337.86718636,231.9687464)
}
}
{
\newrgbcolor{curcolor}{0 0 0}
\pscustom[linestyle=none,fillstyle=solid,fillcolor=curcolor]
{
\newpath
\moveto(344.24218636,235.1796839)
\lineto(344.24218636,243.1249964)
\lineto(346.39843636,243.1249964)
\lineto(346.39843636,235.26171515)
\curveto(346.39843217,234.01952363)(346.64061943,233.08593082)(347.12499886,232.4609339)
\curveto(347.60936846,231.84374456)(348.33593023,231.53515112)(349.30468636,231.53515265)
\curveto(350.4687406,231.53515112)(351.38670843,231.9062445)(352.05859261,232.6484339)
\curveto(352.73826958,233.39061801)(353.07811299,234.40233575)(353.07812386,235.68359015)
\lineto(353.07812386,243.1249964)
\lineto(355.23437386,243.1249964)
\lineto(355.23437386,229.9999964)
\lineto(353.07812386,229.9999964)
\lineto(353.07812386,232.0156214)
\curveto(352.55467601,231.21874518)(351.94530162,230.62499578)(351.24999886,230.2343714)
\curveto(350.5624905,229.85155905)(349.76171005,229.66015299)(348.84765511,229.66015265)
\curveto(347.33983748,229.66015299)(346.19530737,230.12890252)(345.41406136,231.06640265)
\curveto(344.63280893,232.00390065)(344.24218432,233.37499303)(344.24218636,235.1796839)
\moveto(349.66796761,243.44140265)
\lineto(349.66796761,243.44140265)
}
}
{
\newrgbcolor{curcolor}{0 0 0}
\pscustom[linestyle=none,fillstyle=solid,fillcolor=curcolor]
{
\newpath
\moveto(370.92578011,237.1015589)
\lineto(370.92578011,236.0468714)
\lineto(361.01171761,236.0468714)
\curveto(361.10546395,234.56249184)(361.550776,233.42968047)(362.34765511,232.6484339)
\curveto(363.1523369,231.87499453)(364.26952328,231.48827616)(365.69921761,231.48827765)
\curveto(366.52733352,231.48827616)(367.32811397,231.58983856)(368.10156136,231.79296515)
\curveto(368.88279992,231.99608816)(369.65623664,232.30077535)(370.42187386,232.70702765)
\lineto(370.42187386,230.66796515)
\curveto(369.64842415,230.33983981)(368.8554562,230.08984006)(368.04296761,229.91796515)
\curveto(367.23045782,229.74609041)(366.40623989,229.66015299)(365.57031136,229.66015265)
\curveto(363.47655532,229.66015299)(361.81640073,230.26952738)(360.58984261,231.48827765)
\curveto(359.37109068,232.70702495)(358.76171629,234.3554608)(358.76171761,236.43359015)
\curveto(358.76171629,238.58201907)(359.33984071,240.28514237)(360.49609261,241.54296515)
\curveto(361.66015089,242.80857734)(363.22655557,243.44138921)(365.19531136,243.44140265)
\curveto(366.96092684,243.44138921)(368.3554567,242.87107728)(369.37890511,241.73046515)
\curveto(370.41014214,240.59764205)(370.92576662,239.05467485)(370.92578011,237.1015589)
\moveto(368.76953011,237.7343714)
\curveto(368.7538938,238.91404999)(368.42186288,239.8554553)(367.77343636,240.55859015)
\curveto(367.13280167,241.26170389)(366.28124002,241.61326604)(365.21874886,241.61327765)
\curveto(364.01561729,241.61326604)(363.0507745,241.27342263)(362.32421761,240.5937464)
\curveto(361.60546345,239.91404899)(361.19140136,238.9570187)(361.08203011,237.72265265)
\lineto(368.76953011,237.7343714)
}
}
{
\newrgbcolor{curcolor}{0 0 0}
\pscustom[linestyle=none,fillstyle=solid,fillcolor=curcolor]
{
\newpath
\moveto(382.83203011,242.73827765)
\lineto(382.83203011,240.69921515)
\curveto(382.22264509,241.01170414)(381.58983323,241.24607891)(380.93359261,241.40234015)
\curveto(380.27733454,241.55857859)(379.59764772,241.63670352)(378.89453011,241.63671515)
\curveto(377.82421199,241.63670352)(377.0195253,241.47264118)(376.48046761,241.14452765)
\curveto(375.94921387,240.81639184)(375.68358913,240.32420483)(375.68359261,239.66796515)
\curveto(375.68358913,239.16795598)(375.87499519,238.77342513)(376.25781136,238.4843714)
\curveto(376.64061943,238.2031132)(377.41014991,237.93358222)(378.56640511,237.67577765)
\lineto(379.30468636,237.51171515)
\curveto(380.83592773,237.18358297)(381.92186414,236.71873968)(382.56249886,236.1171839)
\curveto(383.21092536,235.52342838)(383.53514378,234.69139796)(383.53515511,233.62109015)
\curveto(383.53514378,232.40233775)(383.05076927,231.43749496)(382.08203011,230.7265589)
\curveto(381.1210837,230.01562139)(379.79686627,229.66015299)(378.10937386,229.66015265)
\curveto(377.40624366,229.66015299)(376.67186939,229.73046542)(375.90624886,229.87109015)
\curveto(375.14843342,230.00390265)(374.34765297,230.20702745)(373.50390511,230.48046515)
\lineto(373.50390511,232.70702765)
\curveto(374.30077802,232.29296286)(375.08593348,231.98046317)(375.85937386,231.76952765)
\curveto(376.63280693,231.56640109)(377.39843117,231.46483869)(378.15624886,231.46484015)
\curveto(379.17186689,231.46483869)(379.95311611,231.63671352)(380.49999886,231.98046515)
\curveto(381.04686502,232.33202532)(381.32030225,232.82421233)(381.32031136,233.45702765)
\curveto(381.32030225,234.04296111)(381.1210837,234.49217941)(380.72265511,234.8046839)
\curveto(380.33202198,235.11717879)(379.4687416,235.41795973)(378.13281136,235.70702765)
\lineto(377.38281136,235.8828089)
\curveto(376.04687002,236.16405274)(375.08202723,236.59373981)(374.48828011,237.1718714)
\curveto(373.89452842,237.75780114)(373.59765372,238.55858159)(373.59765511,239.57421515)
\curveto(373.59765372,240.80857934)(374.03515328,241.76170339)(374.91015511,242.43359015)
\curveto(375.78515153,243.10545205)(377.02733779,243.44138921)(378.63671761,243.44140265)
\curveto(379.43358538,243.44138921)(380.18358463,243.38279552)(380.88671761,243.2656214)
\curveto(381.58983323,243.14842075)(382.23827008,242.97263968)(382.83203011,242.73827765)
}
}
{
\newrgbcolor{curcolor}{1 1 1}
\pscustom[linestyle=none,fillstyle=solid,fillcolor=curcolor]
{
\newpath
\moveto(203.99147492,199.9999964)
\lineto(357.5801323,199.9999964)
\curveto(364.46073838,199.9999964)(369.99999886,194.46073592)(369.99999886,187.58012984)
\lineto(369.99999886,182.59711669)
\curveto(369.99999886,175.71651061)(364.46073838,170.17725013)(357.5801323,170.17725013)
\lineto(203.99147492,170.17725013)
\curveto(197.11086885,170.17725013)(191.57160836,175.71651061)(191.57160836,182.59711669)
\lineto(191.57160836,187.58012984)
\curveto(191.57160836,194.46073592)(197.11086885,199.9999964)(203.99147492,199.9999964)
\closepath
}
}
{
\newrgbcolor{curcolor}{0 0 0}
\pscustom[linewidth=2,linecolor=curcolor]
{
\newpath
\moveto(203.99147492,199.9999964)
\lineto(357.5801323,199.9999964)
\curveto(364.46073838,199.9999964)(369.99999886,194.46073592)(369.99999886,187.58012984)
\lineto(369.99999886,182.59711669)
\curveto(369.99999886,175.71651061)(364.46073838,170.17725013)(357.5801323,170.17725013)
\lineto(203.99147492,170.17725013)
\curveto(197.11086885,170.17725013)(191.57160836,175.71651061)(191.57160836,182.59711669)
\lineto(191.57160836,187.58012984)
\curveto(191.57160836,194.46073592)(197.11086885,199.9999964)(203.99147492,199.9999964)
\closepath
}
}
{
\newrgbcolor{curcolor}{0 0 0}
\pscustom[linestyle=none,fillstyle=solid,fillcolor=curcolor]
{
\newpath
\moveto(225.76302987,196.24282477)
\lineto(225.76302987,193.74673102)
\curveto(224.96614021,194.48890413)(224.11457856,195.04359107)(223.20834237,195.41079352)
\curveto(222.30989287,195.77796534)(221.35286257,195.9615589)(220.33724862,195.96157477)
\curveto(218.33724059,195.9615589)(216.80599212,195.34827827)(215.74349862,194.12173102)
\curveto(214.68099424,192.90296821)(214.14974478,191.13734498)(214.14974862,188.82485602)
\curveto(214.14974478,186.5201621)(214.68099424,184.75453886)(215.74349862,183.52798102)
\curveto(216.80599212,182.30922881)(218.33724059,181.69985442)(220.33724862,181.69985602)
\curveto(221.35286257,181.69985442)(222.30989287,181.88344798)(223.20834237,182.25063727)
\curveto(224.11457856,182.61782225)(224.96614021,183.17250919)(225.76302987,183.91469977)
\lineto(225.76302987,181.44204352)
\curveto(224.93489024,180.87954274)(224.05598487,180.45766816)(223.12631112,180.17641852)
\curveto(222.20442422,179.89516872)(221.2278627,179.75454386)(220.19662362,179.75454352)
\curveto(217.54817888,179.75454386)(215.46224346,180.5631368)(213.93881112,182.18032477)
\curveto(212.41537151,183.80532106)(211.65365352,186.0201626)(211.65365487,188.82485602)
\curveto(211.65365352,191.63734448)(212.41537151,193.85218601)(213.93881112,195.46938727)
\curveto(215.46224346,197.09437027)(217.54817888,197.90686946)(220.19662362,197.90688727)
\curveto(221.24348768,197.90686946)(222.2278617,197.7662446)(223.14974862,197.48501227)
\curveto(224.07942235,197.21155765)(224.95051522,196.79749557)(225.76302987,196.24282477)
}
}
{
\newrgbcolor{curcolor}{0 0 0}
\pscustom[linestyle=none,fillstyle=solid,fillcolor=curcolor]
{
\newpath
\moveto(236.95443612,191.20376227)
\curveto(236.71223899,191.34437602)(236.44661426,191.44593842)(236.15756112,191.50844977)
\curveto(235.87630233,191.57875079)(235.56380264,191.613907)(235.22006112,191.61391852)
\curveto(234.00130421,191.613907)(233.06380514,191.2154699)(232.40756112,190.41860602)
\curveto(231.75911895,189.62953399)(231.43490052,188.49281637)(231.43490487,187.00844977)
\lineto(231.43490487,180.09438727)
\lineto(229.26693612,180.09438727)
\lineto(229.26693612,193.21938727)
\lineto(231.43490487,193.21938727)
\lineto(231.43490487,191.18032477)
\curveto(231.88802507,191.97718789)(232.47786823,192.56703105)(233.20443612,192.94985602)
\curveto(233.93099178,193.34046778)(234.81380339,193.53578008)(235.85287362,193.53579352)
\curveto(236.00130221,193.53578008)(236.16536454,193.52406134)(236.34506112,193.50063727)
\curveto(236.52473918,193.48499888)(236.72395773,193.45765516)(236.94271737,193.41860602)
\lineto(236.95443612,191.20376227)
}
}
{
\newrgbcolor{curcolor}{0 0 0}
\pscustom[linestyle=none,fillstyle=solid,fillcolor=curcolor]
{
\newpath
\moveto(249.95052987,187.19594977)
\lineto(249.95052987,186.14126227)
\lineto(240.03646737,186.14126227)
\curveto(240.1302137,184.65688271)(240.57552576,183.52407134)(241.37240487,182.74282477)
\curveto(242.17708665,181.9693854)(243.29427304,181.58266703)(244.72396737,181.58266852)
\curveto(245.55208328,181.58266703)(246.35286373,181.68422943)(247.12631112,181.88735602)
\curveto(247.90754967,182.09047903)(248.6809864,182.39516622)(249.44662362,182.80141852)
\lineto(249.44662362,180.76235602)
\curveto(248.67317391,180.43423068)(247.88020595,180.18423093)(247.06771737,180.01235602)
\curveto(246.25520758,179.84048128)(245.43098965,179.75454386)(244.59506112,179.75454352)
\curveto(242.50130508,179.75454386)(240.84115049,180.36391825)(239.61459237,181.58266852)
\curveto(238.39584044,182.80141581)(237.78646604,184.44985167)(237.78646737,186.52798102)
\curveto(237.78646604,188.67640994)(238.36459047,190.37953324)(239.52084237,191.63735602)
\curveto(240.68490065,192.90296821)(242.25130533,193.53578008)(244.22006112,193.53579352)
\curveto(245.9856766,193.53578008)(247.38020645,192.96546815)(248.40365487,191.82485602)
\curveto(249.4348919,190.69203292)(249.95051638,189.14906572)(249.95052987,187.19594977)
\moveto(247.79427987,187.82876227)
\curveto(247.77864355,189.00844086)(247.44661263,189.94984617)(246.79818612,190.65298102)
\curveto(246.15755142,191.35609476)(245.30598978,191.70765691)(244.24349862,191.70766852)
\curveto(243.04036704,191.70765691)(242.07552426,191.3678135)(241.34896737,190.68813727)
\curveto(240.6302132,190.00843986)(240.21615112,189.05140956)(240.10677987,187.81704352)
\lineto(247.79427987,187.82876227)
\moveto(245.72006112,199.28969977)
\lineto(248.05209237,199.28969977)
\lineto(244.23177987,194.88344977)
\lineto(242.43881112,194.88344977)
\lineto(245.72006112,199.28969977)
}
}
{
\newrgbcolor{curcolor}{0 0 0}
\pscustom[linestyle=none,fillstyle=solid,fillcolor=curcolor]
{
\newpath
\moveto(264.71615487,187.19594977)
\lineto(264.71615487,186.14126227)
\lineto(254.80209237,186.14126227)
\curveto(254.8958387,184.65688271)(255.34115076,183.52407134)(256.13802987,182.74282477)
\curveto(256.94271165,181.9693854)(258.05989804,181.58266703)(259.48959237,181.58266852)
\curveto(260.31770828,181.58266703)(261.11848873,181.68422943)(261.89193612,181.88735602)
\curveto(262.67317467,182.09047903)(263.4466114,182.39516622)(264.21224862,182.80141852)
\lineto(264.21224862,180.76235602)
\curveto(263.43879891,180.43423068)(262.64583095,180.18423093)(261.83334237,180.01235602)
\curveto(261.02083258,179.84048128)(260.19661465,179.75454386)(259.36068612,179.75454352)
\curveto(257.26693008,179.75454386)(255.60677549,180.36391825)(254.38021737,181.58266852)
\curveto(253.16146544,182.80141581)(252.55209104,184.44985167)(252.55209237,186.52798102)
\curveto(252.55209104,188.67640994)(253.13021547,190.37953324)(254.28646737,191.63735602)
\curveto(255.45052565,192.90296821)(257.01693033,193.53578008)(258.98568612,193.53579352)
\curveto(260.7513016,193.53578008)(262.14583145,192.96546815)(263.16927987,191.82485602)
\curveto(264.2005169,190.69203292)(264.71614138,189.14906572)(264.71615487,187.19594977)
\moveto(262.55990487,187.82876227)
\curveto(262.54426855,189.00844086)(262.21223763,189.94984617)(261.56381112,190.65298102)
\curveto(260.92317642,191.35609476)(260.07161478,191.70765691)(259.00912362,191.70766852)
\curveto(257.80599204,191.70765691)(256.84114926,191.3678135)(256.11459237,190.68813727)
\curveto(255.3958382,190.00843986)(254.98177612,189.05140956)(254.87240487,187.81704352)
\lineto(262.55990487,187.82876227)
}
}
{
\newrgbcolor{curcolor}{0 0 0}
\pscustom[linestyle=none,fillstyle=solid,fillcolor=curcolor]
{
\newpath
\moveto(275.86068612,191.20376227)
\curveto(275.61848899,191.34437602)(275.35286426,191.44593842)(275.06381112,191.50844977)
\curveto(274.78255233,191.57875079)(274.47005264,191.613907)(274.12631112,191.61391852)
\curveto(272.90755421,191.613907)(271.97005514,191.2154699)(271.31381112,190.41860602)
\curveto(270.66536895,189.62953399)(270.34115052,188.49281637)(270.34115487,187.00844977)
\lineto(270.34115487,180.09438727)
\lineto(268.17318612,180.09438727)
\lineto(268.17318612,193.21938727)
\lineto(270.34115487,193.21938727)
\lineto(270.34115487,191.18032477)
\curveto(270.79427507,191.97718789)(271.38411823,192.56703105)(272.11068612,192.94985602)
\curveto(272.83724178,193.34046778)(273.72005339,193.53578008)(274.75912362,193.53579352)
\curveto(274.90755221,193.53578008)(275.07161454,193.52406134)(275.25131112,193.50063727)
\curveto(275.43098918,193.48499888)(275.63020773,193.45765516)(275.84896737,193.41860602)
\lineto(275.86068612,191.20376227)
}
}
{
\newrgbcolor{curcolor}{0 0 0}
\pscustom[linestyle=none,fillstyle=solid,fillcolor=curcolor]
{
}
}
{
\newrgbcolor{curcolor}{0 0 0}
\pscustom[linestyle=none,fillstyle=solid,fillcolor=curcolor]
{
\newpath
\moveto(296.69662362,188.01626227)
\lineto(296.69662362,180.09438727)
\lineto(294.54037362,180.09438727)
\lineto(294.54037362,187.94594977)
\curveto(294.5403626,189.18812818)(294.29817535,190.11781475)(293.81381112,190.73501227)
\curveto(293.32942631,191.35218851)(292.60286454,191.66078196)(291.63412362,191.66079352)
\curveto(290.47005417,191.66078196)(289.55208634,191.28968858)(288.88021737,190.54751227)
\curveto(288.20833769,189.80531506)(287.87240052,188.79359732)(287.87240487,187.51235602)
\lineto(287.87240487,180.09438727)
\lineto(285.70443612,180.09438727)
\lineto(285.70443612,193.21938727)
\lineto(287.87240487,193.21938727)
\lineto(287.87240487,191.18032477)
\curveto(288.38802501,191.9693754)(288.99349315,192.55921856)(289.68881112,192.94985602)
\curveto(290.39192925,193.34046778)(291.20052219,193.53578008)(292.11459237,193.53579352)
\curveto(293.62239477,193.53578008)(294.76301863,193.06703055)(295.53646737,192.12954352)
\curveto(296.30989208,191.19984492)(296.69661045,189.82875254)(296.69662362,188.01626227)
}
}
{
\newrgbcolor{curcolor}{0 0 0}
\pscustom[linestyle=none,fillstyle=solid,fillcolor=curcolor]
{
\newpath
\moveto(301.02084237,193.21938727)
\lineto(303.17709237,193.21938727)
\lineto(303.17709237,180.09438727)
\lineto(301.02084237,180.09438727)
\lineto(301.02084237,193.21938727)
\moveto(301.02084237,198.32876227)
\lineto(303.17709237,198.32876227)
\lineto(303.17709237,195.59829352)
\lineto(301.02084237,195.59829352)
\lineto(301.02084237,198.32876227)
}
}
{
\newrgbcolor{curcolor}{0 0 0}
\pscustom[linestyle=none,fillstyle=solid,fillcolor=curcolor]
{
\newpath
\moveto(306.13021737,193.21938727)
\lineto(308.41537362,193.21938727)
\lineto(312.51693612,182.20376227)
\lineto(316.61849862,193.21938727)
\lineto(318.90365487,193.21938727)
\lineto(313.98177987,180.09438727)
\lineto(311.05209237,180.09438727)
\lineto(306.13021737,193.21938727)
}
}
{
\newrgbcolor{curcolor}{0 0 0}
\pscustom[linestyle=none,fillstyle=solid,fillcolor=curcolor]
{
\newpath
\moveto(333.10677987,187.19594977)
\lineto(333.10677987,186.14126227)
\lineto(323.19271737,186.14126227)
\curveto(323.2864637,184.65688271)(323.73177576,183.52407134)(324.52865487,182.74282477)
\curveto(325.33333665,181.9693854)(326.45052304,181.58266703)(327.88021737,181.58266852)
\curveto(328.70833328,181.58266703)(329.50911373,181.68422943)(330.28256112,181.88735602)
\curveto(331.06379967,182.09047903)(331.8372364,182.39516622)(332.60287362,182.80141852)
\lineto(332.60287362,180.76235602)
\curveto(331.82942391,180.43423068)(331.03645595,180.18423093)(330.22396737,180.01235602)
\curveto(329.41145758,179.84048128)(328.58723965,179.75454386)(327.75131112,179.75454352)
\curveto(325.65755508,179.75454386)(323.99740049,180.36391825)(322.77084237,181.58266852)
\curveto(321.55209044,182.80141581)(320.94271604,184.44985167)(320.94271737,186.52798102)
\curveto(320.94271604,188.67640994)(321.52084047,190.37953324)(322.67709237,191.63735602)
\curveto(323.84115065,192.90296821)(325.40755533,193.53578008)(327.37631112,193.53579352)
\curveto(329.1419266,193.53578008)(330.53645645,192.96546815)(331.55990487,191.82485602)
\curveto(332.5911419,190.69203292)(333.10676638,189.14906572)(333.10677987,187.19594977)
\moveto(330.95052987,187.82876227)
\curveto(330.93489355,189.00844086)(330.60286263,189.94984617)(329.95443612,190.65298102)
\curveto(329.31380142,191.35609476)(328.46223978,191.70765691)(327.39974862,191.70766852)
\curveto(326.19661704,191.70765691)(325.23177426,191.3678135)(324.50521737,190.68813727)
\curveto(323.7864632,190.00843986)(323.37240112,189.05140956)(323.26302987,187.81704352)
\lineto(330.95052987,187.82876227)
}
}
{
\newrgbcolor{curcolor}{0 0 0}
\pscustom[linestyle=none,fillstyle=solid,fillcolor=curcolor]
{
\newpath
\moveto(342.61068612,186.69204352)
\curveto(340.86849213,186.69203692)(339.66146209,186.49281837)(338.98959237,186.09438727)
\curveto(338.31771344,185.69594417)(337.98177627,185.01625735)(337.98177987,184.05532477)
\curveto(337.98177627,183.28969658)(338.23177602,182.68032219)(338.73177987,182.22719977)
\curveto(339.23958751,181.78188558)(339.92708683,181.55922956)(340.79427987,181.55923102)
\curveto(341.98958476,181.55922956)(342.94661506,181.98110413)(343.66537362,182.82485602)
\curveto(344.39192611,183.67641494)(344.755207,184.80532006)(344.75521737,186.21157477)
\lineto(344.75521737,186.69204352)
\lineto(342.61068612,186.69204352)
\moveto(346.91146737,187.58266852)
\lineto(346.91146737,180.09438727)
\lineto(344.75521737,180.09438727)
\lineto(344.75521737,182.08657477)
\curveto(344.26301999,181.28969858)(343.64973935,180.69985542)(342.91537362,180.31704352)
\curveto(342.18099082,179.94204367)(341.28255422,179.75454386)(340.22006112,179.75454352)
\curveto(338.87630663,179.75454386)(337.8059952,180.12954349)(337.00912362,180.87954352)
\curveto(336.22005928,181.63735448)(335.82552843,182.64907222)(335.82552987,183.91469977)
\curveto(335.82552843,185.39125697)(336.31771544,186.50453711)(337.30209237,187.25454352)
\curveto(338.29427596,188.00453561)(339.77083698,188.37953524)(341.73177987,188.37954352)
\lineto(344.75521737,188.37954352)
\lineto(344.75521737,188.59048102)
\curveto(344.755207,189.58265903)(344.42708233,190.34828327)(343.77084237,190.88735602)
\curveto(343.12239613,191.43421968)(342.20833454,191.70765691)(341.02865487,191.70766852)
\curveto(340.27864897,191.70765691)(339.54818096,191.61781325)(338.83724862,191.43813727)
\curveto(338.12630738,191.25843861)(337.44271431,190.98890763)(336.78646737,190.62954352)
\lineto(336.78646737,192.62173102)
\curveto(337.57552668,192.92640569)(338.34115091,193.15296796)(339.08334237,193.30141852)
\curveto(339.82552443,193.45765516)(340.54817996,193.53578008)(341.25131112,193.53579352)
\curveto(343.14973985,193.53578008)(344.56770719,193.04359307)(345.50521737,192.05923102)
\curveto(346.44270531,191.07484504)(346.91145484,189.58265903)(346.91146737,187.58266852)
}
}
{
\newrgbcolor{curcolor}{0 0 0}
\pscustom[linestyle=none,fillstyle=solid,fillcolor=curcolor]
{
\newpath
\moveto(351.14193612,185.27407477)
\lineto(351.14193612,193.21938727)
\lineto(353.29818612,193.21938727)
\lineto(353.29818612,185.35610602)
\curveto(353.29818192,184.1139145)(353.54036918,183.18032169)(354.02474862,182.55532477)
\curveto(354.50911821,181.93813543)(355.23567999,181.62954199)(356.20443612,181.62954352)
\curveto(357.36849035,181.62954199)(358.28645819,182.00063537)(358.95834237,182.74282477)
\curveto(359.63801933,183.48500888)(359.97786274,184.49672662)(359.97787362,185.77798102)
\lineto(359.97787362,193.21938727)
\lineto(362.13412362,193.21938727)
\lineto(362.13412362,180.09438727)
\lineto(359.97787362,180.09438727)
\lineto(359.97787362,182.11001227)
\curveto(359.45442577,181.31313605)(358.84505138,180.71938665)(358.14974862,180.32876227)
\curveto(357.46224026,179.94594992)(356.66145981,179.75454386)(355.74740487,179.75454352)
\curveto(354.23958723,179.75454386)(353.09505713,180.22329339)(352.31381112,181.16079352)
\curveto(351.53255869,182.09829152)(351.14193408,183.4693839)(351.14193612,185.27407477)
\moveto(356.56771737,193.53579352)
\lineto(356.56771737,193.53579352)
}
}
{
\newrgbcolor{curcolor}{1 1 1}
\pscustom[linestyle=none,fillstyle=solid,fillcolor=curcolor]
{
\newpath
\moveto(254.69320374,150.5329859)
\lineto(315.21460801,150.5329859)
\curveto(323.33513314,150.5329859)(329.8725956,143.99552344)(329.8725956,135.87499831)
\lineto(329.8725956,134.83607695)
\curveto(329.8725956,126.71555183)(323.33513314,120.17808936)(315.21460801,120.17808936)
\lineto(254.69320374,120.17808936)
\curveto(246.57267862,120.17808936)(240.03521615,126.71555183)(240.03521615,134.83607695)
\lineto(240.03521615,135.87499831)
\curveto(240.03521615,143.99552344)(246.57267862,150.5329859)(254.69320374,150.5329859)
\closepath
}
}
{
\newrgbcolor{curcolor}{0 0 0}
\pscustom[linewidth=2,linecolor=curcolor]
{
\newpath
\moveto(254.69320374,150.5329859)
\lineto(315.21460801,150.5329859)
\curveto(323.33513314,150.5329859)(329.8725956,143.99552344)(329.8725956,135.87499831)
\lineto(329.8725956,134.83607695)
\curveto(329.8725956,126.71555183)(323.33513314,120.17808936)(315.21460801,120.17808936)
\lineto(254.69320374,120.17808936)
\curveto(246.57267862,120.17808936)(240.03521615,126.71555183)(240.03521615,134.83607695)
\lineto(240.03521615,135.87499831)
\curveto(240.03521615,143.99552344)(246.57267862,150.5329859)(254.69320374,150.5329859)
\closepath
}
}
{
\newrgbcolor{curcolor}{0 0 0}
\pscustom[linestyle=none,fillstyle=solid,fillcolor=curcolor]
{
\newpath
\moveto(268.20312386,145.1640589)
\lineto(264.99218636,136.45702765)
\lineto(271.42578011,136.45702765)
\lineto(268.20312386,145.1640589)
\moveto(266.86718636,147.49609015)
\lineto(269.55078011,147.49609015)
\lineto(276.21874886,129.9999964)
\lineto(273.75781136,129.9999964)
\lineto(272.16406136,134.48827765)
\lineto(264.27734261,134.48827765)
\lineto(262.68359261,129.9999964)
\lineto(260.18749886,129.9999964)
\lineto(266.86718636,147.49609015)
}
}
{
\newrgbcolor{curcolor}{0 0 0}
\pscustom[linestyle=none,fillstyle=solid,fillcolor=curcolor]
{
\newpath
\moveto(278.66796761,143.1249964)
\lineto(280.82421761,143.1249964)
\lineto(280.82421761,129.9999964)
\lineto(278.66796761,129.9999964)
\lineto(278.66796761,143.1249964)
\moveto(278.66796761,148.2343714)
\lineto(280.82421761,148.2343714)
\lineto(280.82421761,145.50390265)
\lineto(278.66796761,145.50390265)
\lineto(278.66796761,148.2343714)
}
}
{
\newrgbcolor{curcolor}{0 0 0}
\pscustom[linestyle=none,fillstyle=solid,fillcolor=curcolor]
{
\newpath
\moveto(293.96093636,141.1328089)
\lineto(293.96093636,148.2343714)
\lineto(296.11718636,148.2343714)
\lineto(296.11718636,129.9999964)
\lineto(293.96093636,129.9999964)
\lineto(293.96093636,131.9687464)
\curveto(293.50780092,131.18749521)(292.93358274,130.60546455)(292.23828011,130.22265265)
\curveto(291.55077162,129.8476528)(290.72264745,129.66015299)(289.75390511,129.66015265)
\curveto(288.16796251,129.66015299)(286.87499505,130.29296486)(285.87499886,131.55859015)
\curveto(284.88280954,132.82421233)(284.38671629,134.48827316)(284.38671761,136.55077765)
\curveto(284.38671629,138.61326904)(284.88280954,140.27732988)(285.87499886,141.54296515)
\curveto(286.87499505,142.80857734)(288.16796251,143.44138921)(289.75390511,143.44140265)
\curveto(290.72264745,143.44138921)(291.55077162,143.24998315)(292.23828011,142.8671839)
\curveto(292.93358274,142.49217141)(293.50780092,141.91404699)(293.96093636,141.1328089)
\moveto(286.61328011,136.55077765)
\curveto(286.61327656,134.96483519)(286.93749499,133.71874268)(287.58593636,132.8124964)
\curveto(288.24218118,131.91405699)(289.14061779,131.46483869)(290.28124886,131.46484015)
\curveto(291.4218655,131.46483869)(292.32030211,131.91405699)(292.97656136,132.8124964)
\curveto(293.63280079,133.71874268)(293.96092546,134.96483519)(293.96093636,136.55077765)
\curveto(293.96092546,138.13670702)(293.63280079,139.37889327)(292.97656136,140.27734015)
\curveto(292.32030211,141.18357897)(291.4218655,141.63670352)(290.28124886,141.63671515)
\curveto(289.14061779,141.63670352)(288.24218118,141.18357897)(287.58593636,140.27734015)
\curveto(286.93749499,139.37889327)(286.61327656,138.13670702)(286.61328011,136.55077765)
}
}
{
\newrgbcolor{curcolor}{0 0 0}
\pscustom[linestyle=none,fillstyle=solid,fillcolor=curcolor]
{
\newpath
\moveto(311.78515511,137.1015589)
\lineto(311.78515511,136.0468714)
\lineto(301.87109261,136.0468714)
\curveto(301.96483895,134.56249184)(302.410151,133.42968047)(303.20703011,132.6484339)
\curveto(304.0117119,131.87499453)(305.12889828,131.48827616)(306.55859261,131.48827765)
\curveto(307.38670852,131.48827616)(308.18748897,131.58983856)(308.96093636,131.79296515)
\curveto(309.74217492,131.99608816)(310.51561164,132.30077535)(311.28124886,132.70702765)
\lineto(311.28124886,130.66796515)
\curveto(310.50779915,130.33983981)(309.7148312,130.08984006)(308.90234261,129.91796515)
\curveto(308.08983282,129.74609041)(307.26561489,129.66015299)(306.42968636,129.66015265)
\curveto(304.33593032,129.66015299)(302.67577573,130.26952738)(301.44921761,131.48827765)
\curveto(300.23046568,132.70702495)(299.62109129,134.3554608)(299.62109261,136.43359015)
\curveto(299.62109129,138.58201907)(300.19921571,140.28514237)(301.35546761,141.54296515)
\curveto(302.51952589,142.80857734)(304.08593057,143.44138921)(306.05468636,143.44140265)
\curveto(307.82030184,143.44138921)(309.2148317,142.87107728)(310.23828011,141.73046515)
\curveto(311.26951714,140.59764205)(311.78514162,139.05467485)(311.78515511,137.1015589)
\moveto(309.62890511,137.7343714)
\curveto(309.6132688,138.91404999)(309.28123788,139.8554553)(308.63281136,140.55859015)
\curveto(307.99217667,141.26170389)(307.14061502,141.61326604)(306.07812386,141.61327765)
\curveto(304.87499229,141.61326604)(303.9101495,141.27342263)(303.18359261,140.5937464)
\curveto(302.46483845,139.91404899)(302.05077636,138.9570187)(301.94140511,137.72265265)
\lineto(309.62890511,137.7343714)
}
}
{
\newrgbcolor{curcolor}{0 0 0}
\pscustom[linestyle=none,fillstyle=solid,fillcolor=curcolor,opacity=0.11935484]
{
\newpath
\moveto(178.69516545,59.9999964)
\lineto(377.49469739,59.9999964)
\curveto(382.31181968,59.9999964)(386.18986398,56.1219521)(386.18986398,51.30482981)
\lineto(386.18986398,38.88316462)
\curveto(386.18986398,34.06604233)(382.31181968,30.18799804)(377.49469739,30.18799804)
\lineto(178.69516545,30.18799804)
\curveto(173.87804316,30.18799804)(169.99999886,34.06604233)(169.99999886,38.88316462)
\lineto(169.99999886,51.30482981)
\curveto(169.99999886,56.1219521)(173.87804316,59.9999964)(178.69516545,59.9999964)
\closepath
}
}
{
\newrgbcolor{curcolor}{0 0 0}
\pscustom[linewidth=2,linecolor=curcolor]
{
\newpath
\moveto(178.69516545,59.9999964)
\lineto(377.49469739,59.9999964)
\curveto(382.31181968,59.9999964)(386.18986398,56.1219521)(386.18986398,51.30482981)
\lineto(386.18986398,38.88316462)
\curveto(386.18986398,34.06604233)(382.31181968,30.18799804)(377.49469739,30.18799804)
\lineto(178.69516545,30.18799804)
\curveto(173.87804316,30.18799804)(169.99999886,34.06604233)(169.99999886,38.88316462)
\lineto(169.99999886,51.30482981)
\curveto(169.99999886,56.1219521)(173.87804316,59.9999964)(178.69516545,59.9999964)
\closepath
}
}
{
\newrgbcolor{curcolor}{0 0 0}
\pscustom[linestyle=none,fillstyle=solid,fillcolor=curcolor]
{
\newpath
\moveto(43.96874886,512.6562464)
\lineto(49.12499886,512.6562464)
\lineto(49.12499886,530.4531214)
\lineto(43.51562386,529.3281214)
\lineto(43.51562386,532.2031214)
\lineto(49.09374886,533.3281214)
\lineto(52.24999886,533.3281214)
\lineto(52.24999886,512.6562464)
\lineto(57.40624886,512.6562464)
\lineto(57.40624886,509.9999964)
\lineto(43.96874886,509.9999964)
\lineto(43.96874886,512.6562464)
}
}
{
\newrgbcolor{curcolor}{0 0 0}
\pscustom[linestyle=none,fillstyle=solid,fillcolor=curcolor]
{
\newpath
\moveto(46.14062386,382.6562464)
\lineto(57.15624886,382.6562464)
\lineto(57.15624886,379.9999964)
\lineto(42.34374886,379.9999964)
\lineto(42.34374886,382.6562464)
\curveto(43.54166199,383.89582584)(45.17186869,385.55728251)(47.23437386,387.6406214)
\curveto(49.30728122,389.73436167)(50.60936325,391.08331865)(51.14062386,391.6874964)
\curveto(52.15102838,392.82290025)(52.85415268,393.78123262)(53.24999886,394.5624964)
\curveto(53.65623521,395.35414771)(53.85936,396.13018861)(53.85937386,396.8906214)
\curveto(53.85936,398.13018661)(53.42186044,399.14060226)(52.54687386,399.9218714)
\curveto(51.68227885,400.7031007)(50.55207164,401.09372531)(49.15624886,401.0937464)
\curveto(48.16665736,401.09372531)(47.11978341,400.92185048)(46.01562386,400.5781214)
\curveto(44.92186894,400.23435117)(43.74999511,399.71351836)(42.49999886,399.0156214)
\lineto(42.49999886,402.2031214)
\curveto(43.77082843,402.71351536)(44.95832724,403.09893164)(46.06249886,403.3593714)
\curveto(47.16665836,403.61976445)(48.17707402,403.74997265)(49.09374886,403.7499964)
\curveto(51.51040402,403.74997265)(53.43748543,403.14580659)(54.87499886,401.9374964)
\curveto(56.31248255,400.72914234)(57.03123183,399.11456062)(57.03124886,397.0937464)
\curveto(57.03123183,396.13539693)(56.84894035,395.22393951)(56.48437386,394.3593714)
\curveto(56.13019107,393.50519123)(55.47915005,392.49477557)(54.53124886,391.3281214)
\curveto(54.27081793,391.02602704)(53.44269375,390.15102792)(52.04687386,388.7031214)
\curveto(50.65102988,387.26561414)(48.68228185,385.24999115)(46.14062386,382.6562464)
}
}
{
\newrgbcolor{curcolor}{0 0 0}
\pscustom[linestyle=none,fillstyle=solid,fillcolor=curcolor]
{
\newpath
\moveto(512.98437386,342.5781214)
\curveto(514.49477604,342.25519248)(515.67185819,341.58331815)(516.51562386,340.5624964)
\curveto(517.36977316,339.54165353)(517.79685607,338.28123812)(517.79687386,336.7812464)
\curveto(517.79685607,334.47915859)(517.00519019,332.69791037)(515.42187386,331.4374964)
\curveto(513.83852669,330.17707956)(511.58852894,329.54687186)(508.67187386,329.5468714)
\curveto(507.6926995,329.54687186)(506.68228385,329.64583009)(505.64062386,329.8437464)
\curveto(504.60936925,330.03124637)(503.54166199,330.31770442)(502.43749886,330.7031214)
\lineto(502.43749886,333.7499964)
\curveto(503.31249555,333.2395765)(504.27082793,332.85416021)(505.31249886,332.5937464)
\curveto(506.35415918,332.3333274)(507.44269975,332.2031192)(508.57812386,332.2031214)
\curveto(510.55727997,332.2031192)(512.0624868,332.59374381)(513.09374886,333.3749964)
\curveto(514.13540139,334.15624225)(514.65623421,335.29165778)(514.65624886,336.7812464)
\curveto(514.65623421,338.15623825)(514.17185969,339.22915384)(513.20312386,339.9999964)
\curveto(512.24477829,340.78123562)(510.90623796,341.17186023)(509.18749886,341.1718714)
\lineto(506.46874886,341.1718714)
\lineto(506.46874886,343.7656214)
\lineto(509.31249886,343.7656214)
\curveto(510.86457133,343.76560764)(512.05207014,344.072899)(512.87499886,344.6874964)
\curveto(513.69790183,345.31248109)(514.10935975,346.20831353)(514.10937386,347.3749964)
\curveto(514.10935975,348.5728945)(513.68227685,349.48956025)(512.82812386,350.1249964)
\curveto(511.98436188,350.77080896)(510.77082143,351.09372531)(509.18749886,351.0937464)
\curveto(508.32290721,351.09372531)(507.3958248,350.9999754)(506.40624886,350.8124964)
\curveto(505.41666011,350.62497578)(504.32811954,350.3333094)(503.14062386,349.9374964)
\lineto(503.14062386,352.7499964)
\curveto(504.33853619,353.08330665)(505.45832674,353.3333064)(506.49999886,353.4999964)
\curveto(507.55207464,353.6666394)(508.54165699,353.74997265)(509.46874886,353.7499964)
\curveto(511.86457033,353.74997265)(513.76040177,353.2030982)(515.15624886,352.1093714)
\curveto(516.55206564,351.02601704)(517.24998161,349.55726851)(517.24999886,347.7031214)
\curveto(517.24998161,346.41143832)(516.88019032,345.31768942)(516.14062386,344.4218714)
\curveto(515.40102513,343.5364412)(514.34894285,342.92185848)(512.98437386,342.5781214)
}
}
{
\newrgbcolor{curcolor}{0 0 0}
\pscustom[linestyle=none,fillstyle=solid,fillcolor=curcolor]
{
\newpath
\moveto(512.09374886,300.5781214)
\lineto(504.12499886,288.1249964)
\lineto(512.09374886,288.1249964)
\lineto(512.09374886,300.5781214)
\moveto(511.26562386,303.3281214)
\lineto(515.23437386,303.3281214)
\lineto(515.23437386,288.1249964)
\lineto(518.56249886,288.1249964)
\lineto(518.56249886,285.4999964)
\lineto(515.23437386,285.4999964)
\lineto(515.23437386,279.9999964)
\lineto(512.09374886,279.9999964)
\lineto(512.09374886,285.4999964)
\lineto(501.56249886,285.4999964)
\lineto(501.56249886,288.5468714)
\lineto(511.26562386,303.3281214)
}
}
{
\newrgbcolor{curcolor}{0 0 0}
\pscustom[linestyle=none,fillstyle=solid,fillcolor=curcolor]
{
\newpath
\moveto(503.45312386,253.3281214)
\lineto(515.84374886,253.3281214)
\lineto(515.84374886,250.6718714)
\lineto(506.34374886,250.6718714)
\lineto(506.34374886,244.9531214)
\curveto(506.80207539,245.10935629)(507.26040827,245.22393951)(507.71874886,245.2968714)
\curveto(508.17707402,245.38018936)(508.63540689,245.42185598)(509.09374886,245.4218714)
\curveto(511.69790383,245.42185598)(513.76040177,244.70831503)(515.28124886,243.2812464)
\curveto(516.80206539,241.85415121)(517.5624813,239.92186148)(517.56249886,237.4843714)
\curveto(517.5624813,234.97394976)(516.78123208,233.02082671)(515.21874886,231.6249964)
\curveto(513.65623521,230.2395795)(511.45311241,229.54687186)(508.60937386,229.5468714)
\curveto(507.63019957,229.54687186)(506.63020057,229.63020511)(505.60937386,229.7968714)
\curveto(504.5989526,229.96353811)(503.55207864,230.21353786)(502.46874886,230.5468714)
\lineto(502.46874886,233.7187464)
\curveto(503.40624546,233.20832653)(504.37499449,232.82811857)(505.37499886,232.5781214)
\curveto(506.37499249,232.32811907)(507.4322831,232.2031192)(508.54687386,232.2031214)
\curveto(510.34894685,232.2031192)(511.77602875,232.67707706)(512.82812386,233.6249964)
\curveto(513.88019332,234.5729085)(514.40623446,235.85936554)(514.40624886,237.4843714)
\curveto(514.40623446,239.10936229)(513.88019332,240.39581934)(512.82812386,241.3437464)
\curveto(511.77602875,242.29165078)(510.34894685,242.76560864)(508.54687386,242.7656214)
\curveto(507.70311616,242.76560864)(506.859367,242.67185873)(506.01562386,242.4843714)
\curveto(505.18228535,242.29685911)(504.32811954,242.00519273)(503.45312386,241.6093714)
\lineto(503.45312386,253.3281214)
}
}
{
\newrgbcolor{curcolor}{0 0 0}
\pscustom[linestyle=none,fillstyle=solid,fillcolor=curcolor]
{
\newpath
\moveto(510.56249886,192.9218714)
\curveto(509.14582305,192.92185848)(508.02082418,192.43748396)(507.18749886,191.4687464)
\curveto(506.36457583,190.4999859)(505.95311791,189.17186223)(505.95312386,187.4843714)
\curveto(505.95311791,185.80728226)(506.36457583,184.47915859)(507.18749886,183.4999964)
\curveto(508.02082418,182.53124387)(509.14582305,182.04686936)(510.56249886,182.0468714)
\curveto(511.97915355,182.04686936)(513.0989441,182.53124387)(513.92187386,183.4999964)
\curveto(514.75519244,184.47915859)(515.17185869,185.80728226)(515.17187386,187.4843714)
\curveto(515.17185869,189.17186223)(514.75519244,190.4999859)(513.92187386,191.4687464)
\curveto(513.0989441,192.43748396)(511.97915355,192.92185848)(510.56249886,192.9218714)
\moveto(516.82812386,202.8124964)
\lineto(516.82812386,199.9374964)
\curveto(516.03644116,200.31247609)(515.23435863,200.59893414)(514.42187386,200.7968714)
\curveto(513.61977691,200.99476707)(512.82290271,201.09372531)(512.03124886,201.0937464)
\curveto(509.94790558,201.09372531)(508.35415718,200.39060101)(507.24999886,198.9843714)
\curveto(506.15624271,197.57810382)(505.53124333,195.45310595)(505.37499886,192.6093714)
\curveto(505.98957621,193.51560789)(506.76040877,194.20831553)(507.68749886,194.6874964)
\curveto(508.61457358,195.17706456)(509.63540589,195.42185598)(510.74999886,195.4218714)
\curveto(513.09373577,195.42185598)(514.94269225,194.70831503)(516.29687386,193.2812464)
\curveto(517.66143954,191.86456787)(518.34373052,189.93227814)(518.34374886,187.4843714)
\curveto(518.34373052,185.08853298)(517.63539789,183.1666599)(516.21874886,181.7187464)
\curveto(514.80206739,180.27082946)(512.91665261,179.54687186)(510.56249886,179.5468714)
\curveto(507.86457433,179.54687186)(505.80207639,180.57812082)(504.37499886,182.6406214)
\curveto(502.94791258,184.71353336)(502.23437163,187.71353036)(502.23437386,191.6406214)
\curveto(502.23437163,195.32810607)(503.10937075,198.26560314)(504.85937386,200.4531214)
\curveto(506.60936725,202.65101542)(508.95832324,203.74997265)(511.90624886,203.7499964)
\curveto(512.69790283,203.74997265)(513.49477704,203.67184773)(514.29687386,203.5156214)
\curveto(515.10935875,203.35934804)(515.95310791,203.12497328)(516.82812386,202.8124964)
}
}
{
\newrgbcolor{curcolor}{0 0 0}
\pscustom[linestyle=none,fillstyle=solid,fillcolor=curcolor]
{
\newpath
\moveto(42.62499886,153.3281214)
\lineto(57.62499886,153.3281214)
\lineto(57.62499886,151.9843714)
\lineto(49.15624886,129.9999964)
\lineto(45.85937386,129.9999964)
\lineto(53.82812386,150.6718714)
\lineto(42.62499886,150.6718714)
\lineto(42.62499886,153.3281214)
}
}
{
\newrgbcolor{curcolor}{0 0 0}
\pscustom[linestyle=none,fillstyle=solid,fillcolor=curcolor]
{
\newpath
\moveto(50.17187386,51.0781214)
\curveto(48.67186519,51.07811032)(47.48957471,50.67706906)(46.62499886,49.8749964)
\curveto(45.77082643,49.072904)(45.34374352,47.96873843)(45.34374886,46.5624964)
\curveto(45.34374352,45.15624125)(45.77082643,44.05207568)(46.62499886,43.2499964)
\curveto(47.48957471,42.44791062)(48.67186519,42.04686936)(50.17187386,42.0468714)
\curveto(51.67186219,42.04686936)(52.85415268,42.44791062)(53.71874886,43.2499964)
\curveto(54.58331761,44.06249234)(55.01560885,45.1666579)(55.01562386,46.5624964)
\curveto(55.01560885,47.96873843)(54.58331761,49.072904)(53.71874886,49.8749964)
\curveto(52.86456933,50.67706906)(51.68227885,51.07811032)(50.17187386,51.0781214)
\moveto(47.01562386,52.4218714)
\curveto(45.66145154,52.75519198)(44.60416093,53.38539968)(43.84374886,54.3124964)
\curveto(43.09374577,55.2395645)(42.71874614,56.3697717)(42.71874886,57.7031214)
\curveto(42.71874614,59.56768517)(43.38020382,61.04164203)(44.70312386,62.1249964)
\curveto(46.03645116,63.20830653)(47.859366,63.74997265)(50.17187386,63.7499964)
\curveto(52.49477804,63.74997265)(54.31769288,63.20830653)(55.64062386,62.1249964)
\curveto(56.96352357,61.04164203)(57.62498124,59.56768517)(57.62499886,57.7031214)
\curveto(57.62498124,56.3697717)(57.24477329,55.2395645)(56.48437386,54.3124964)
\curveto(55.73435813,53.38539968)(54.68748418,52.75519198)(53.34374886,52.4218714)
\curveto(54.86456733,52.06769267)(56.04685782,51.37498503)(56.89062386,50.3437464)
\curveto(57.74477279,49.31248709)(58.17185569,48.05207168)(58.17187386,46.5624964)
\curveto(58.17185569,44.30207543)(57.47914805,42.56770217)(56.09374886,41.3593714)
\curveto(54.71873414,40.15103792)(52.74477779,39.54687186)(50.17187386,39.5468714)
\curveto(47.5989496,39.54687186)(45.61978491,40.15103792)(44.23437386,41.3593714)
\curveto(42.859371,42.56770217)(42.17187169,44.30207543)(42.17187386,46.5624964)
\curveto(42.17187169,48.05207168)(42.5989546,49.31248709)(43.45312386,50.3437464)
\curveto(44.30728622,51.37498503)(45.49478504,52.06769267)(47.01562386,52.4218714)
\moveto(45.85937386,57.4062464)
\curveto(45.859368,56.19789687)(46.23436763,55.25518948)(46.98437386,54.5781214)
\curveto(47.74478279,53.90102417)(48.80728172,53.56248284)(50.17187386,53.5624964)
\curveto(51.526029,53.56248284)(52.58331961,53.90102417)(53.34374886,54.5781214)
\curveto(54.11456808,55.25518948)(54.49998436,56.19789687)(54.49999886,57.4062464)
\curveto(54.49998436,58.61456112)(54.11456808,59.55726851)(53.34374886,60.2343714)
\curveto(52.58331961,60.91143382)(51.526029,61.24997515)(50.17187386,61.2499964)
\curveto(48.80728172,61.24997515)(47.74478279,60.91143382)(46.98437386,60.2343714)
\curveto(46.23436763,59.55726851)(45.859368,58.61456112)(45.85937386,57.4062464)
}
}
{
\newrgbcolor{curcolor}{0 0 0}
\pscustom[linewidth=1.80982327,linecolor=curcolor,linestyle=dashed,dash=7.23929296 7.23929296]
{
\newpath
\moveto(409.999999,339.999999)
\lineto(499.999999,339.999999)
}
}
{
\newrgbcolor{curcolor}{0 0 0}
\pscustom[linestyle=none,fillstyle=solid,fillcolor=curcolor]
{
\newpath
\moveto(419.46746087,335.61979029)
\lineto(407.60311665,339.9826103)
\lineto(419.46746152,344.34542935)
\curveto(417.57203535,341.76962273)(417.58295683,338.24547223)(419.46746087,335.61979029)
\lineto(419.46746087,335.61979029)
\closepath
}
}
{
\newrgbcolor{curcolor}{0 0 0}
\pscustom[linewidth=1.89703047,linecolor=curcolor,linestyle=dashed,dash=7.58812199 7.58812199]
{
\newpath
\moveto(339.999999,289.999819)
\lineto(500.262979,289.999819)
}
}
{
\newrgbcolor{curcolor}{0 0 0}
\pscustom[linestyle=none,fillstyle=solid,fillcolor=curcolor]
{
\newpath
\moveto(349.92365504,285.40854784)
\lineto(337.48762173,289.98159241)
\lineto(349.92365572,294.55463598)
\curveto(347.93689753,291.8547129)(347.94834528,288.16074952)(349.92365504,285.40854784)
\lineto(349.92365504,285.40854784)
\closepath
}
}
{
\newrgbcolor{curcolor}{0 0 0}
\pscustom[linewidth=1.86349273,linecolor=curcolor,linestyle=dashed,dash=7.4539707 7.4539707]
{
\newpath
\moveto(369.999999,189.999789)
\lineto(500.322879,189.999789)
}
}
{
\newrgbcolor{curcolor}{0 0 0}
\pscustom[linestyle=none,fillstyle=solid,fillcolor=curcolor]
{
\newpath
\moveto(379.74821397,185.48968727)
\lineto(367.53203824,189.98188464)
\lineto(379.74821465,194.47408103)
\curveto(377.7965805,191.82189009)(377.80782586,188.19323257)(379.74821397,185.48968727)
\lineto(379.74821397,185.48968727)
\closepath
}
}
{
\newrgbcolor{curcolor}{0 0 0}
\pscustom[linewidth=1.84759188,linecolor=curcolor,linestyle=dashed,dash=7.39036736 7.39036736]
{
\newpath
\moveto(389.999999,239.998149)
\lineto(501.401259,239.833959)
}
}
{
\newrgbcolor{curcolor}{0 0 0}
\pscustom[linestyle=none,fillstyle=solid,fillcolor=curcolor]
{
\newpath
\moveto(399.65843316,235.51229112)
\lineto(387.5530734,239.98400383)
\lineto(399.67156258,244.42001297)
\curveto(397.73270788,241.79330745)(397.73855477,238.19560009)(399.65843316,235.51229112)
\lineto(399.65843316,235.51229112)
\closepath
}
}
{
\newrgbcolor{curcolor}{0 0 0}
\pscustom[linewidth=1.85803509,linecolor=curcolor,linestyle=dashed,dash=7.43214029 7.43214029]
{
\newpath
\moveto(189.999999,389.999999)
\lineto(60,389.999999)
}
}
{
\newrgbcolor{curcolor}{0 0 0}
\pscustom[linestyle=none,fillstyle=solid,fillcolor=curcolor]
{
\newpath
\moveto(180.28033377,394.49689192)
\lineto(192.4607318,390.01785092)
\lineto(180.2803331,385.5388109)
\curveto(182.22625147,388.18323433)(182.21503904,391.80126454)(180.28033377,394.49689192)
\lineto(180.28033377,394.49689192)
\closepath
}
}
{
\newrgbcolor{curcolor}{0 0 0}
\pscustom[linewidth=1.89736664,linecolor=curcolor,linestyle=dashed,dash=7.58946638 7.58946638]
{
\newpath
\moveto(239.999999,139.999999)
\lineto(60,139.999999)
}
}
{
\newrgbcolor{curcolor}{0 0 0}
\pscustom[linestyle=none,fillstyle=solid,fillcolor=curcolor]
{
\newpath
\moveto(230.0745844,144.59208377)
\lineto(242.51282148,140.01822882)
\lineto(230.07458372,135.44437486)
\curveto(232.06169398,138.1447764)(232.05024421,141.83939438)(230.0745844,144.59208377)
\lineto(230.0745844,144.59208377)
\closepath
}
}
{
\newrgbcolor{curcolor}{0 0 0}
\pscustom[linewidth=2,linecolor=curcolor,linestyle=dashed,dash=8 8]
{
\newpath
\moveto(190,50)
\lineto(60,50)
}
}
{
\newrgbcolor{curcolor}{0 0 0}
\pscustom[linestyle=none,fillstyle=solid,fillcolor=curcolor]
{
\newpath
\moveto(179.53769464,54.84048224)
\lineto(192.6487474,50.01921591)
\lineto(179.53769392,45.19795064)
\curveto(181.632292,48.04442372)(181.62022288,51.93889292)(179.53769464,54.84048224)
\lineto(179.53769464,54.84048224)
\closepath
}
}
{
\newrgbcolor{curcolor}{0 0 0}
\pscustom[linewidth=2,linecolor=curcolor,linestyle=dashed,dash=8 8]
{
\newpath
\moveto(60,520)
\lineto(160,520)
}
}
{
\newrgbcolor{curcolor}{0 0 0}
\pscustom[linestyle=none,fillstyle=solid,fillcolor=curcolor]
{
\newpath
\moveto(149.53769464,524.84048224)
\lineto(162.6487474,520.01921591)
\lineto(149.53769392,515.19795064)
\curveto(151.632292,518.04442372)(151.62022288,521.93889292)(149.53769464,524.84048224)
\closepath
}
}
\end{pspicture}

		\end{center}

		\begin{enumerate}
		  \item Fond d'écran (présent dans l'archive stocké sur le téléphone)
		  \item Bouton ``\hyperlink{Creer partie solo}{Jouer en solo}''
		  \item Bouton ``\hyperlink{Connexion multi-joueurs}{Jouer en multi-joueurs}''
		  \item Bouton ``\hyperlink{Options}{Options}''
		  \item Bouton ``\hyperlink{Statistiques}{Mes statistiques}''
		  \item Bouton ``\hyperlink{Creer niveau}{Créer niveau}''
		  \item Bouton ``\hyperlink{Aide}{Aide}''
		  \item Liste déroulante ``Comptes hors lignes''
		  \item Bouton ``\hyperlink{profil}{+}'' 
		\end{enumerate}

		\subsubsection{Description des zones}
		
		\begin{tabular}{|c|c|c|c|c|} \hline
			Numéro de zone & Type  & Description & Evènement &	Règle \\\hline 
			2 & Bouton & Affiche l'écran de création d'un partie solitaire & Cliqué & RG2-01 \\\hline
			3 & Bouton & Affiche l'écran de connexion multi-joueurs & Cliqué & RG2-02 \\\hline
			4 & Bouton & Affiche l'écran des options & Cliqué & RG2-03 \\\hline
			5 & Bouton & Affiche l'écran des statistiques & Cliqué & RG2-04 \\\hline
			6 & Bouton & Affiche l'écran de création de niveaux & Cliqué & RG2-05 \\\hline
			7 & Bouton & Affiche l'écran d'aide & Cliqué & RG2-06 \\\hline
			8 & Liste & Permet de sélectionner un compte hors ligne & Cliqué & RG2-07 \\ 
			  & déroulante & Affiche le compte en cours d'utilisation & Perte du focus & RG2-08\\\hline
			9 & Bouton & Affiche l'écran de création du profil & Cliqué & RG2-08 \\\hline
			
		\end{tabular}
		
\newpage

		\subsubsection{Description des règles}
		
		\underline{RG2-01 :}
			\begin{quote}
				Charger la page de création d'une partie hors ligne
					\footnote{
						\hyperlink{Creer partie solo}{Création d'une partie hors ligne}
						\og voir section \ref{Creer partie solo}, page \pageref{Creer partie solo}.\fg
					}.\\
				Afficher la page de création d'une partie hors ligne\footnotemark[1].\\
				Supprimer la page d'accueil%
					\footnote{
						\hyperlink{Page d'accueil}{Page d'accueil}
						\og voir section \ref{Accueil}, page \pageref{Accueil}.\fg
					}.
			\end{quote}

		$\,$

		\underline{RG2-02 :}
			\begin{quote}
				Si la connexion automatique est activée alors
				\begin{quote}
					Si le nom d'utilisateur et le mot de passe existent alors
					\begin{quote}
						Se connecter au serveur distant.\\
						Si la connexion est validée alors
						\begin{quote}
							Charger la page d'accueil multi-joueurs%
								\footnote{
									\hyperlink{Accueil multi-joueurs}{Accueil multi-joueurs}
									\og voir section \ref{Accueil multi-joueurs}, page \pageref{Accueil multi-joueurs}.\fg
								}.\\
							Afficher la page d'accueil multi-joueurs\footnotemark[3].
						\end{quote}
					\end{quote}	
				\end{quote}
				Si la connexion automatique est desactivée ou qu'une erreur est survenue alors 
				\begin{quote}		
					Charger la page de connexion multi-joueurs%
						\footnote{
							\hyperlink{Connexion multi-joueurs}{Connexion multi-joueurs}
							\og voir section \ref{Connexion multi-joueurs}, page \pageref{Connexion multi-joueurs}.\fg
						}.\\
					Afficher la page de connexion multi-joueurs\footnotemark[4].
					Si une erreur est survenue
					\begin{quote}
						Afficher l'erreur dans la page de connexion multi-joueurs\footnotemark[4].
					\end{quote}
				\end{quote}
				Supprimer la page d'accueil\footnotemark[2].
			\end{quote}

		$\,$
		
		\underline{RG2-03 :}
			\begin{quote}
				Charger la page des options%
					\footnote{
						\hyperlink{Options}{Options}
						\og voir section \ref{Options}, page \pageref{Options}.\fg
					}.\\
				Afficher la page des options\footnotemark[5].\\
				Supprimer la page d'accueil\footnotemark[2].
			\end{quote}


		$\,$

		\underline{RG2-04 :}
			\begin{quote}
				Charger la page des statistiques%
					\footnote{
						\hyperlink{Statistiques}{Statistiques}
						\og voir section \ref{Statistiques}, page \pageref{Statistiques}.\fg
					}.\\
				Afficher la page des statistiques\footnotemark[6].\\
				Supprimer la page d'accueil\footnotemark[2].
			\end{quote}

		$\,$

		\underline{RG2-05 :}
			\begin{quote}
				Charger la page de création de niveaux%
					\footnote{
						\hyperlink{Creer niveau}{Créer niveau}
						\og voir section \ref{Creer niveau}, page \pageref{Creer niveau}.\fg
					}.\\
				Afficher la page de création de niveaux\footnotemark[7].\\
				Supprimer la page d'accueil\footnotemark[2].		
			\end{quote}

		$\,$

		\underline{RG2-06 :}
			\begin{quote}
				Charger la page d'aide%
					\footnote{
						\hyperlink{Aide}{Aide}
						\og voir section \ref{Aide}, page \pageref{Aide}.\fg
					}.\\
				Afficher la page d'aide\footnotemark[8].\\
				Supprimer la page d'accueil\footnotemark[2].\\		
			\end{quote}
			
			
		\underline{RG2-07 :}
			\begin{quote}
				Récupérer la liste des comptes hors ligne présents sur le téléphone.\\
				Les afficher en premier plan.\\
			\end{quote}


		\underline{RG2-08 :}
			\begin{quote}
				Mettre à jour la classe utilisateur.
			\end{quote}

		$\,$

		\underline{RG2-09 :}
			\begin{quote}
				Charger la page de création du profil%
					\footnote{
						\hyperlink{profil}{Création du profil}
						\og voir section \ref{profil}, page \pageref{profil}.\fg
					}.\\
				Afficher la page de création du profil\footnotemark[9].\\
				Supprimer la page d'accueil\footnotemark[2].
			\end{quote}

		$\,$
	
\newpage

	\subsection{Création d'une partie hors ligne}
	
		\hypertarget{Creer partie solo}{}
		\label{Creer partie solo}

		\begin{center}
			%LaTeX with PSTricks extensions
%%Creator: inkscape 0.48.0
%%Please note this file requires PSTricks extensions
\psset{xunit=.4pt,yunit=.4pt,runit=.4pt}
\begin{pspicture}(560,600)
{
\newrgbcolor{curcolor}{1 1 1}
\pscustom[linestyle=none,fillstyle=solid,fillcolor=curcolor]
{
\newpath
\moveto(133.12401581,597.52220273)
\lineto(426.87598419,597.52220273)
\curveto(443.85397169,597.52220273)(457.52217102,583.8540034)(457.52217102,566.8760159)
\lineto(457.52217102,33.124017)
\curveto(457.52217102,16.1460295)(443.85397169,2.47783017)(426.87598419,2.47783017)
\lineto(133.12401581,2.47783017)
\curveto(116.14602831,2.47783017)(102.47782898,16.1460295)(102.47782898,33.124017)
\lineto(102.47782898,566.8760159)
\curveto(102.47782898,583.8540034)(116.14602831,597.52220273)(133.12401581,597.52220273)
\closepath
}
}
{
\newrgbcolor{curcolor}{0 0 0}
\pscustom[linewidth=4.95566034,linecolor=curcolor]
{
\newpath
\moveto(133.12401581,597.52220273)
\lineto(426.87598419,597.52220273)
\curveto(443.85397169,597.52220273)(457.52217102,583.8540034)(457.52217102,566.8760159)
\lineto(457.52217102,33.124017)
\curveto(457.52217102,16.1460295)(443.85397169,2.47783017)(426.87598419,2.47783017)
\lineto(133.12401581,2.47783017)
\curveto(116.14602831,2.47783017)(102.47782898,16.1460295)(102.47782898,33.124017)
\lineto(102.47782898,566.8760159)
\curveto(102.47782898,583.8540034)(116.14602831,597.52220273)(133.12401581,597.52220273)
\closepath
}
}
{
\newrgbcolor{curcolor}{1 1 1}
\pscustom[linestyle=none,fillstyle=solid,fillcolor=curcolor]
{
\newpath
\moveto(135.10877228,299.91494751)
\lineto(423.47055817,299.91494751)
\curveto(437.43929419,299.91494751)(448.68488312,288.66935858)(448.68488312,274.70062256)
\lineto(448.68488312,165.27957153)
\curveto(448.68488312,151.31083551)(437.43929419,140.06524658)(423.47055817,140.06524658)
\lineto(135.10877228,140.06524658)
\curveto(121.14003625,140.06524658)(109.89444733,151.31083551)(109.89444733,165.27957153)
\lineto(109.89444733,274.70062256)
\curveto(109.89444733,288.66935858)(121.14003625,299.91494751)(135.10877228,299.91494751)
\closepath
}
}
{
\newrgbcolor{curcolor}{0 0 0}
\pscustom[linewidth=2,linecolor=curcolor]
{
\newpath
\moveto(135.10877228,299.91494751)
\lineto(423.47055817,299.91494751)
\curveto(437.43929419,299.91494751)(448.68488312,288.66935858)(448.68488312,274.70062256)
\lineto(448.68488312,165.27957153)
\curveto(448.68488312,151.31083551)(437.43929419,140.06524658)(423.47055817,140.06524658)
\lineto(135.10877228,140.06524658)
\curveto(121.14003625,140.06524658)(109.89444733,151.31083551)(109.89444733,165.27957153)
\lineto(109.89444733,274.70062256)
\curveto(109.89444733,288.66935858)(121.14003625,299.91494751)(135.10877228,299.91494751)
\closepath
}
}
{
\newrgbcolor{curcolor}{0 0 0}
\pscustom[linestyle=none,fillstyle=solid,fillcolor=curcolor,opacity=0.11935484]
{
\newpath
\moveto(312.75479603,289.96047974)
\lineto(417.33941174,289.96047974)
\curveto(424.44983058,289.96047974)(430.17410278,285.79401281)(430.17410278,280.61862564)
\lineto(430.17410278,269.59705544)
\curveto(430.17410278,264.42166827)(424.44983058,260.25520134)(417.33941174,260.25520134)
\lineto(312.75479603,260.25520134)
\curveto(305.64437719,260.25520134)(299.92010498,264.42166827)(299.92010498,269.59705544)
\lineto(299.92010498,280.61862564)
\curveto(299.92010498,285.79401281)(305.64437719,289.96047974)(312.75479603,289.96047974)
\closepath
}
}
{
\newrgbcolor{curcolor}{0 0 0}
\pscustom[linewidth=2,linecolor=curcolor]
{
\newpath
\moveto(312.75479603,289.96047974)
\lineto(417.33941174,289.96047974)
\curveto(424.44983058,289.96047974)(430.17410278,285.79401281)(430.17410278,280.61862564)
\lineto(430.17410278,269.59705544)
\curveto(430.17410278,264.42166827)(424.44983058,260.25520134)(417.33941174,260.25520134)
\lineto(312.75479603,260.25520134)
\curveto(305.64437719,260.25520134)(299.92010498,264.42166827)(299.92010498,269.59705544)
\lineto(299.92010498,280.61862564)
\curveto(299.92010498,285.79401281)(305.64437719,289.96047974)(312.75479603,289.96047974)
\closepath
}
}
{
\newrgbcolor{curcolor}{1 1 1}
\pscustom[linestyle=none,fillstyle=solid,fillcolor=curcolor]
{
\newpath
\moveto(190.0634613,350.19837952)
\lineto(369.92247009,350.19837952)
\lineto(369.92247009,319.90265656)
\lineto(190.0634613,319.90265656)
\closepath
}
}
{
\newrgbcolor{curcolor}{0 0 0}
\pscustom[linewidth=1.57047963,linecolor=curcolor]
{
\newpath
\moveto(190.0634613,350.19837952)
\lineto(369.92247009,350.19837952)
\lineto(369.92247009,319.90265656)
\lineto(190.0634613,319.90265656)
\closepath
}
}
{
\newrgbcolor{curcolor}{1 1 1}
\pscustom[linestyle=none,fillstyle=solid,fillcolor=curcolor]
{
\newpath
\moveto(220.09861755,479.90138245)
\lineto(340.15664673,479.90138245)
\lineto(340.15664673,359.90190125)
\lineto(220.09861755,359.90190125)
\closepath
}
}
{
\newrgbcolor{curcolor}{0 0 0}
\pscustom[linewidth=2,linecolor=curcolor]
{
\newpath
\moveto(220.09861755,479.90138245)
\lineto(340.15664673,479.90138245)
\lineto(340.15664673,359.90190125)
\lineto(220.09861755,359.90190125)
\closepath
}
}
{
\newrgbcolor{curcolor}{1 1 1}
\pscustom[linestyle=none,fillstyle=solid,fillcolor=curcolor]
{
\newpath
\moveto(360.10479736,460.10528564)
\lineto(440.10527039,460.10528564)
\lineto(440.10527039,380.14836884)
\lineto(360.10479736,380.14836884)
\closepath
}
}
{
\newrgbcolor{curcolor}{0 0 0}
\pscustom[linewidth=2,linecolor=curcolor]
{
\newpath
\moveto(360.10479736,460.10528564)
\lineto(440.10527039,460.10528564)
\lineto(440.10527039,380.14836884)
\lineto(360.10479736,380.14836884)
\closepath
}
}
{
\newrgbcolor{curcolor}{1 1 1}
\pscustom[linestyle=none,fillstyle=solid,fillcolor=curcolor]
{
\newpath
\moveto(119.89472198,460.10528564)
\lineto(199.89520264,460.10528564)
\lineto(199.89520264,380.14836884)
\lineto(119.89472198,380.14836884)
\closepath
}
}
{
\newrgbcolor{curcolor}{0 0 0}
\pscustom[linewidth=2,linecolor=curcolor]
{
\newpath
\moveto(119.89472198,460.10528564)
\lineto(199.89520264,460.10528564)
\lineto(199.89520264,380.14836884)
\lineto(119.89472198,380.14836884)
\closepath
}
}
{
\newrgbcolor{curcolor}{0 0 0}
\pscustom[linestyle=none,fillstyle=solid,fillcolor=curcolor]
{
\newpath
\moveto(119.93554688,286.03808594)
\lineto(133.50292969,286.03808594)
\lineto(133.50292969,284.21191406)
\lineto(127.80957031,284.21191406)
\lineto(127.80957031,270)
\lineto(125.62890625,270)
\lineto(125.62890625,284.21191406)
\lineto(119.93554688,284.21191406)
\lineto(119.93554688,286.03808594)
}
}
{
\newrgbcolor{curcolor}{0 0 0}
\pscustom[linestyle=none,fillstyle=solid,fillcolor=curcolor]
{
\newpath
\moveto(137.09082031,268.8828125)
\curveto(136.53222004,267.45052338)(135.98794975,266.515954)(135.45800781,266.07910156)
\curveto(134.92805498,265.64225696)(134.21907131,265.4238327)(133.33105469,265.42382812)
\lineto(131.75195312,265.42382812)
\lineto(131.75195312,267.078125)
\lineto(132.91210938,267.078125)
\curveto(133.45637676,267.07812792)(133.87890238,267.20703404)(134.1796875,267.46484375)
\curveto(134.48046428,267.72265853)(134.81347176,268.33138188)(135.17871094,269.29101562)
\lineto(135.53320312,270.19335938)
\lineto(130.66699219,282.03125)
\lineto(132.76171875,282.03125)
\lineto(136.52148438,272.62109375)
\lineto(140.28125,282.03125)
\lineto(142.37597656,282.03125)
\lineto(137.09082031,268.8828125)
}
}
{
\newrgbcolor{curcolor}{0 0 0}
\pscustom[linestyle=none,fillstyle=solid,fillcolor=curcolor]
{
\newpath
\moveto(147.01660156,271.8046875)
\lineto(147.01660156,265.42382812)
\lineto(145.02929688,265.42382812)
\lineto(145.02929688,282.03125)
\lineto(147.01660156,282.03125)
\lineto(147.01660156,280.20507812)
\curveto(147.43196175,280.92121304)(147.95474768,281.45116042)(148.58496094,281.79492188)
\curveto(149.22232454,282.14582119)(149.98143836,282.32127674)(150.86230469,282.32128906)
\curveto(152.3232329,282.32127674)(153.50845306,281.7411992)(154.41796875,280.58105469)
\curveto(155.33462311,279.42088902)(155.79295599,277.89549992)(155.79296875,276.00488281)
\curveto(155.79295599,274.1142537)(155.33462311,272.5888646)(154.41796875,271.42871094)
\curveto(153.50845306,270.26855442)(152.3232329,269.68847687)(150.86230469,269.68847656)
\curveto(149.98143836,269.68847687)(149.22232454,269.8603517)(148.58496094,270.20410156)
\curveto(147.95474768,270.55501247)(147.43196175,271.08854058)(147.01660156,271.8046875)
\moveto(153.74121094,276.00488281)
\curveto(153.74120023,277.4586514)(153.44041928,278.59732213)(152.83886719,279.42089844)
\curveto(152.24445693,280.25161735)(151.42447077,280.66698152)(150.37890625,280.66699219)
\curveto(149.33332703,280.66698152)(148.50976015,280.25161735)(147.90820312,279.42089844)
\curveto(147.3137978,278.59732213)(147.01659758,277.4586514)(147.01660156,276.00488281)
\curveto(147.01659758,274.55110222)(147.3137978,273.40885076)(147.90820312,272.578125)
\curveto(148.50976015,271.75455554)(149.33332703,271.34277209)(150.37890625,271.34277344)
\curveto(151.42447077,271.34277209)(152.24445693,271.75455554)(152.83886719,272.578125)
\curveto(153.44041928,273.40885076)(153.74120023,274.55110222)(153.74121094,276.00488281)
}
}
{
\newrgbcolor{curcolor}{0 0 0}
\pscustom[linestyle=none,fillstyle=solid,fillcolor=curcolor]
{
\newpath
\moveto(169.36035156,276.50976562)
\lineto(169.36035156,275.54296875)
\lineto(160.27246094,275.54296875)
\curveto(160.35839508,274.18228748)(160.76659779,273.14387706)(161.49707031,272.42773438)
\curveto(162.23469528,271.71874828)(163.2587828,271.36425645)(164.56933594,271.36425781)
\curveto(165.32844219,271.36425645)(166.06249093,271.45735531)(166.77148438,271.64355469)
\curveto(167.48761972,271.82975077)(168.19660338,272.10904737)(168.8984375,272.48144531)
\lineto(168.8984375,270.61230469)
\curveto(168.18944193,270.31152313)(167.46255464,270.08235669)(166.71777344,269.92480469)
\curveto(165.97297279,269.76725284)(165.2174397,269.68847687)(164.45117188,269.68847656)
\curveto(162.53189551,269.68847687)(161.01008713,270.24707007)(159.88574219,271.36425781)
\curveto(158.76855292,272.48144283)(158.20995972,273.99250903)(158.20996094,275.89746094)
\curveto(158.20995972,277.86685411)(158.73990711,279.42805047)(159.79980469,280.58105469)
\curveto(160.86685811,281.7411992)(162.30272907,282.32127674)(164.10742188,282.32128906)
\curveto(165.72590273,282.32127674)(167.00422176,281.79849081)(167.94238281,280.75292969)
\curveto(168.88768342,279.71450851)(169.3603392,278.30012191)(169.36035156,276.50976562)
\moveto(167.38378906,277.08984375)
\curveto(167.36945577,278.17121579)(167.0650941,279.03417065)(166.47070312,279.67871094)
\curveto(165.88345465,280.32323186)(165.10285648,280.64549717)(164.12890625,280.64550781)
\curveto(163.02603564,280.64549717)(162.14159642,280.33397404)(161.47558594,279.7109375)
\curveto(160.81672795,279.08788154)(160.43717104,278.21060377)(160.33691406,277.07910156)
\lineto(167.38378906,277.08984375)
}
}
{
\newrgbcolor{curcolor}{0 0 0}
\pscustom[linestyle=none,fillstyle=solid,fillcolor=curcolor]
{
}
}
{
\newrgbcolor{curcolor}{0 0 0}
\pscustom[linestyle=none,fillstyle=solid,fillcolor=curcolor]
{
\newpath
\moveto(187.52539062,280.20507812)
\lineto(187.52539062,286.71484375)
\lineto(189.50195312,286.71484375)
\lineto(189.50195312,270)
\lineto(187.52539062,270)
\lineto(187.52539062,271.8046875)
\curveto(187.11001647,271.08854058)(186.58364981,270.55501247)(185.94628906,270.20410156)
\curveto(185.31607295,269.8603517)(184.55695912,269.68847687)(183.66894531,269.68847656)
\curveto(182.21516459,269.68847687)(181.02994442,270.26855442)(180.11328125,271.42871094)
\curveto(179.20377437,272.5888646)(178.74902222,274.1142537)(178.74902344,276.00488281)
\curveto(178.74902222,277.89549992)(179.20377437,279.42088902)(180.11328125,280.58105469)
\curveto(181.02994442,281.7411992)(182.21516459,282.32127674)(183.66894531,282.32128906)
\curveto(184.55695912,282.32127674)(185.31607295,282.14582119)(185.94628906,281.79492188)
\curveto(186.58364981,281.45116042)(187.11001647,280.92121304)(187.52539062,280.20507812)
\moveto(180.79003906,276.00488281)
\curveto(180.79003581,274.55110222)(181.08723603,273.40885076)(181.68164062,272.578125)
\curveto(182.28319838,271.75455554)(183.10676526,271.34277209)(184.15234375,271.34277344)
\curveto(185.197909,271.34277209)(186.02147589,271.75455554)(186.62304688,272.578125)
\curveto(187.22459969,273.40885076)(187.52538063,274.55110222)(187.52539062,276.00488281)
\curveto(187.52538063,277.4586514)(187.22459969,278.59732213)(186.62304688,279.42089844)
\curveto(186.02147589,280.25161735)(185.197909,280.66698152)(184.15234375,280.66699219)
\curveto(183.10676526,280.66698152)(182.28319838,280.25161735)(181.68164062,279.42089844)
\curveto(181.08723603,278.59732213)(180.79003581,277.4586514)(180.79003906,276.00488281)
}
}
{
\newrgbcolor{curcolor}{0 0 0}
\pscustom[linestyle=none,fillstyle=solid,fillcolor=curcolor]
{
\newpath
\moveto(203.86425781,276.50976562)
\lineto(203.86425781,275.54296875)
\lineto(194.77636719,275.54296875)
\curveto(194.86230133,274.18228748)(195.27050404,273.14387706)(196.00097656,272.42773438)
\curveto(196.73860153,271.71874828)(197.76268905,271.36425645)(199.07324219,271.36425781)
\curveto(199.83234844,271.36425645)(200.56639718,271.45735531)(201.27539062,271.64355469)
\curveto(201.99152597,271.82975077)(202.70050963,272.10904737)(203.40234375,272.48144531)
\lineto(203.40234375,270.61230469)
\curveto(202.69334818,270.31152313)(201.96646089,270.08235669)(201.22167969,269.92480469)
\curveto(200.47687904,269.76725284)(199.72134595,269.68847687)(198.95507812,269.68847656)
\curveto(197.03580176,269.68847687)(195.51399338,270.24707007)(194.38964844,271.36425781)
\curveto(193.27245917,272.48144283)(192.71386597,273.99250903)(192.71386719,275.89746094)
\curveto(192.71386597,277.86685411)(193.24381336,279.42805047)(194.30371094,280.58105469)
\curveto(195.37076436,281.7411992)(196.80663532,282.32127674)(198.61132812,282.32128906)
\curveto(200.22980898,282.32127674)(201.50812801,281.79849081)(202.44628906,280.75292969)
\curveto(203.39158967,279.71450851)(203.86424545,278.30012191)(203.86425781,276.50976562)
\moveto(201.88769531,277.08984375)
\curveto(201.87336202,278.17121579)(201.56900035,279.03417065)(200.97460938,279.67871094)
\curveto(200.3873609,280.32323186)(199.60676273,280.64549717)(198.6328125,280.64550781)
\curveto(197.52994189,280.64549717)(196.64550267,280.33397404)(195.97949219,279.7109375)
\curveto(195.3206342,279.08788154)(194.94107729,278.21060377)(194.84082031,277.07910156)
\lineto(201.88769531,277.08984375)
}
}
{
\newrgbcolor{curcolor}{0 0 0}
\pscustom[linestyle=none,fillstyle=solid,fillcolor=curcolor]
{
}
}
{
\newrgbcolor{curcolor}{0 0 0}
\pscustom[linestyle=none,fillstyle=solid,fillcolor=curcolor]
{
\newpath
\moveto(216.02441406,271.8046875)
\lineto(216.02441406,265.42382812)
\lineto(214.03710938,265.42382812)
\lineto(214.03710938,282.03125)
\lineto(216.02441406,282.03125)
\lineto(216.02441406,280.20507812)
\curveto(216.43977425,280.92121304)(216.96256018,281.45116042)(217.59277344,281.79492188)
\curveto(218.23013704,282.14582119)(218.98925086,282.32127674)(219.87011719,282.32128906)
\curveto(221.3310454,282.32127674)(222.51626556,281.7411992)(223.42578125,280.58105469)
\curveto(224.34243561,279.42088902)(224.80076849,277.89549992)(224.80078125,276.00488281)
\curveto(224.80076849,274.1142537)(224.34243561,272.5888646)(223.42578125,271.42871094)
\curveto(222.51626556,270.26855442)(221.3310454,269.68847687)(219.87011719,269.68847656)
\curveto(218.98925086,269.68847687)(218.23013704,269.8603517)(217.59277344,270.20410156)
\curveto(216.96256018,270.55501247)(216.43977425,271.08854058)(216.02441406,271.8046875)
\moveto(222.74902344,276.00488281)
\curveto(222.74901273,277.4586514)(222.44823178,278.59732213)(221.84667969,279.42089844)
\curveto(221.25226943,280.25161735)(220.43228327,280.66698152)(219.38671875,280.66699219)
\curveto(218.34113953,280.66698152)(217.51757265,280.25161735)(216.91601562,279.42089844)
\curveto(216.3216103,278.59732213)(216.02441008,277.4586514)(216.02441406,276.00488281)
\curveto(216.02441008,274.55110222)(216.3216103,273.40885076)(216.91601562,272.578125)
\curveto(217.51757265,271.75455554)(218.34113953,271.34277209)(219.38671875,271.34277344)
\curveto(220.43228327,271.34277209)(221.25226943,271.75455554)(221.84667969,272.578125)
\curveto(222.44823178,273.40885076)(222.74901273,274.55110222)(222.74902344,276.00488281)
}
}
{
\newrgbcolor{curcolor}{0 0 0}
\pscustom[linestyle=none,fillstyle=solid,fillcolor=curcolor]
{
\newpath
\moveto(233.54492188,276.04785156)
\curveto(231.94791072,276.04784551)(230.84146652,275.86522851)(230.22558594,275.5)
\curveto(229.60969692,275.13476049)(229.30175451,274.51171424)(229.30175781,273.63085938)
\curveto(229.30175451,272.92903353)(229.53092095,272.37044034)(229.98925781,271.95507812)
\curveto(230.45474815,271.54687345)(231.08495586,271.34277209)(231.87988281,271.34277344)
\curveto(232.97557897,271.34277209)(233.85285673,271.72949046)(234.51171875,272.50292969)
\curveto(235.1777252,273.28352536)(235.51073268,274.31835506)(235.51074219,275.60742188)
\lineto(235.51074219,276.04785156)
\lineto(233.54492188,276.04785156)
\moveto(237.48730469,276.86425781)
\lineto(237.48730469,270)
\lineto(235.51074219,270)
\lineto(235.51074219,271.82617188)
\curveto(235.05956126,271.09570203)(234.49738734,270.55501247)(233.82421875,270.20410156)
\curveto(233.15103452,269.8603517)(232.32746763,269.68847687)(231.35351562,269.68847656)
\curveto(230.12174067,269.68847687)(229.14062186,270.03222653)(228.41015625,270.71972656)
\curveto(227.68684728,271.41438661)(227.32519399,272.34179453)(227.32519531,273.50195312)
\curveto(227.32519399,274.85546389)(227.77636542,275.87597069)(228.67871094,276.56347656)
\curveto(229.58821256,277.25096931)(230.94172683,277.59471897)(232.73925781,277.59472656)
\lineto(235.51074219,277.59472656)
\lineto(235.51074219,277.78808594)
\curveto(235.51073268,278.69758245)(235.20995173,279.39940466)(234.60839844,279.89355469)
\curveto(234.01398939,280.39484638)(233.1760996,280.64549717)(232.09472656,280.64550781)
\curveto(231.40722116,280.64549717)(230.73762547,280.56314048)(230.0859375,280.3984375)
\curveto(229.43424136,280.23371372)(228.80761438,279.98664366)(228.20605469,279.65722656)
\lineto(228.20605469,281.48339844)
\curveto(228.92935905,281.76268355)(229.63118127,281.97036563)(230.31152344,282.10644531)
\curveto(230.99185699,282.24966223)(231.65429122,282.32127674)(232.29882812,282.32128906)
\curveto(234.03905446,282.32127674)(235.33885785,281.87010532)(236.19824219,280.96777344)
\curveto(237.05760613,280.06541962)(237.4872932,278.69758245)(237.48730469,276.86425781)
}
}
{
\newrgbcolor{curcolor}{0 0 0}
\pscustom[linestyle=none,fillstyle=solid,fillcolor=curcolor]
{
\newpath
\moveto(248.54101562,280.18359375)
\curveto(248.31900159,280.31248969)(248.07551225,280.40558855)(247.81054688,280.46289062)
\curveto(247.55272632,280.52733322)(247.26626827,280.55955975)(246.95117188,280.55957031)
\curveto(245.83397804,280.55955975)(244.9746039,280.19432574)(244.37304688,279.46386719)
\curveto(243.77864155,278.74055116)(243.48144133,277.69856001)(243.48144531,276.33789062)
\lineto(243.48144531,270)
\lineto(241.49414062,270)
\lineto(241.49414062,282.03125)
\lineto(243.48144531,282.03125)
\lineto(243.48144531,280.16210938)
\curveto(243.8968055,280.89256723)(244.43749506,281.4332568)(245.10351562,281.78417969)
\curveto(245.76952498,282.14224046)(246.57876896,282.32127674)(247.53125,282.32128906)
\curveto(247.66730954,282.32127674)(247.81770001,282.31053456)(247.98242188,282.2890625)
\curveto(248.14712677,282.27472731)(248.32974377,282.24966223)(248.53027344,282.21386719)
\lineto(248.54101562,280.18359375)
}
}
{
\newrgbcolor{curcolor}{0 0 0}
\pscustom[linestyle=none,fillstyle=solid,fillcolor=curcolor]
{
\newpath
\moveto(252.59082031,285.44726562)
\lineto(252.59082031,282.03125)
\lineto(256.66210938,282.03125)
\lineto(256.66210938,280.49511719)
\lineto(252.59082031,280.49511719)
\lineto(252.59082031,273.96386719)
\curveto(252.59081628,272.98274441)(252.72330313,272.35253671)(252.98828125,272.07324219)
\curveto(253.26041197,271.79394352)(253.80826298,271.65429522)(254.63183594,271.65429688)
\lineto(256.66210938,271.65429688)
\lineto(256.66210938,270)
\lineto(254.63183594,270)
\curveto(253.10644077,270)(252.05370745,270.28287732)(251.47363281,270.84863281)
\curveto(250.89355236,271.42154806)(250.60351358,272.45995848)(250.60351562,273.96386719)
\lineto(250.60351562,280.49511719)
\lineto(249.15332031,280.49511719)
\lineto(249.15332031,282.03125)
\lineto(250.60351562,282.03125)
\lineto(250.60351562,285.44726562)
\lineto(252.59082031,285.44726562)
}
}
{
\newrgbcolor{curcolor}{0 0 0}
\pscustom[linestyle=none,fillstyle=solid,fillcolor=curcolor]
{
\newpath
\moveto(259.27246094,282.03125)
\lineto(261.24902344,282.03125)
\lineto(261.24902344,270)
\lineto(259.27246094,270)
\lineto(259.27246094,282.03125)
\moveto(259.27246094,286.71484375)
\lineto(261.24902344,286.71484375)
\lineto(261.24902344,284.21191406)
\lineto(259.27246094,284.21191406)
\lineto(259.27246094,286.71484375)
}
}
{
\newrgbcolor{curcolor}{0 0 0}
\pscustom[linestyle=none,fillstyle=solid,fillcolor=curcolor]
{
\newpath
\moveto(275.66503906,276.50976562)
\lineto(275.66503906,275.54296875)
\lineto(266.57714844,275.54296875)
\curveto(266.66308258,274.18228748)(267.07128529,273.14387706)(267.80175781,272.42773438)
\curveto(268.53938278,271.71874828)(269.5634703,271.36425645)(270.87402344,271.36425781)
\curveto(271.63312969,271.36425645)(272.36717843,271.45735531)(273.07617188,271.64355469)
\curveto(273.79230722,271.82975077)(274.50129088,272.10904737)(275.203125,272.48144531)
\lineto(275.203125,270.61230469)
\curveto(274.49412943,270.31152313)(273.76724214,270.08235669)(273.02246094,269.92480469)
\curveto(272.27766029,269.76725284)(271.5221272,269.68847687)(270.75585938,269.68847656)
\curveto(268.83658301,269.68847687)(267.31477463,270.24707007)(266.19042969,271.36425781)
\curveto(265.07324042,272.48144283)(264.51464722,273.99250903)(264.51464844,275.89746094)
\curveto(264.51464722,277.86685411)(265.04459461,279.42805047)(266.10449219,280.58105469)
\curveto(267.17154561,281.7411992)(268.60741657,282.32127674)(270.41210938,282.32128906)
\curveto(272.03059023,282.32127674)(273.30890926,281.79849081)(274.24707031,280.75292969)
\curveto(275.19237092,279.71450851)(275.6650267,278.30012191)(275.66503906,276.50976562)
\moveto(273.68847656,277.08984375)
\curveto(273.67414327,278.17121579)(273.3697816,279.03417065)(272.77539062,279.67871094)
\curveto(272.18814215,280.32323186)(271.40754398,280.64549717)(270.43359375,280.64550781)
\curveto(269.33072314,280.64549717)(268.44628392,280.33397404)(267.78027344,279.7109375)
\curveto(267.12141545,279.08788154)(266.74185854,278.21060377)(266.64160156,277.07910156)
\lineto(273.68847656,277.08984375)
}
}
{
\newrgbcolor{curcolor}{0 0 0}
\pscustom[linestyle=none,fillstyle=solid,fillcolor=curcolor]
{
\newpath
\moveto(112.111327,235.68264286)
\lineto(114.96844613,235.68264286)
\lineto(121.92216989,222.56300392)
\lineto(121.92216989,235.68264286)
\lineto(123.98097632,235.68264286)
\lineto(123.98097632,220)
\lineto(121.12385719,220)
\lineto(114.17013343,233.11963894)
\lineto(114.17013343,220)
\lineto(112.111327,220)
\lineto(112.111327,235.68264286)
}
}
{
\newrgbcolor{curcolor}{0 0 0}
\pscustom[linestyle=none,fillstyle=solid,fillcolor=curcolor]
{
\newpath
\moveto(132.67838398,230.40957741)
\curveto(131.64197248,230.409567)(130.82265237,230.00340832)(130.22042121,229.19110014)
\curveto(129.61818178,228.38577631)(129.31706414,227.27934403)(129.31706736,225.87179997)
\curveto(129.31706414,224.46424417)(129.61468042,223.35431052)(130.20991709,222.54199569)
\curveto(130.81214827,221.73667852)(131.63496974,221.3340212)(132.67838398,221.33402253)
\curveto(133.70777958,221.3340212)(134.52359832,221.74017989)(135.12584265,222.55249981)
\curveto(135.72806891,223.36481463)(136.02918655,224.47124691)(136.02919649,225.87179997)
\curveto(136.02918655,227.26533856)(135.72806891,228.36826947)(135.12584265,229.18059602)
\curveto(134.52359832,229.99990695)(133.70777958,230.409567)(132.67838398,230.40957741)
\moveto(132.67838398,232.04821926)
\curveto(134.35903403,232.04820722)(135.67904976,231.50199381)(136.63843513,230.40957741)
\curveto(137.59779941,229.31714019)(138.07748682,227.80454923)(138.07749881,225.87179997)
\curveto(138.07748682,223.94604171)(137.59779941,222.43345075)(136.63843513,221.33402253)
\curveto(135.67904976,220.24159439)(134.35903403,219.69538099)(132.67838398,219.69538068)
\curveto(130.99071803,219.69538099)(129.66720094,220.24159439)(128.70782873,221.33402253)
\curveto(127.75545402,222.43345075)(127.27926797,223.94604171)(127.27926916,225.87179997)
\curveto(127.27926797,227.80454923)(127.75545402,229.31714019)(128.70782873,230.40957741)
\curveto(129.66720094,231.50199381)(130.99071803,232.04820722)(132.67838398,232.04821926)
}
}
{
\newrgbcolor{curcolor}{0 0 0}
\pscustom[linestyle=none,fillstyle=solid,fillcolor=curcolor]
{
\newpath
\moveto(150.43033777,229.50622357)
\curveto(150.91351536,230.37455332)(151.49124108,231.01530366)(152.16351665,231.42847651)
\curveto(152.83576638,231.84162651)(153.62707555,232.04820722)(154.53744651,232.04821926)
\curveto(155.76291001,232.04820722)(156.70827937,231.61753895)(157.37355741,230.75621319)
\curveto(158.0387992,229.90186865)(158.37142916,228.68339259)(158.37144828,227.10078136)
\lineto(158.37144828,220)
\lineto(156.42818711,220)
\lineto(156.42818711,227.03775668)
\curveto(156.42816993,228.16519013)(156.22859195,229.00201707)(155.82945259,229.54824003)
\curveto(155.43028005,230.09444388)(154.82104203,230.36755059)(154.00173668,230.36756095)
\curveto(153.00033068,230.36755059)(152.20902151,230.03492063)(151.62780681,229.36967008)
\curveto(151.04656734,228.7044008)(150.7559538,227.79754649)(150.75596531,226.64910444)
\lineto(150.75596531,220)
\lineto(148.81270414,220)
\lineto(148.81270414,227.03775668)
\curveto(148.81269457,228.17219286)(148.6131166,229.00901981)(148.21396962,229.54824003)
\curveto(147.8148047,230.09444388)(147.19856394,230.36755059)(146.36524548,230.36756095)
\curveto(145.37785259,230.36755059)(144.59354616,230.03141926)(144.01232384,229.35916597)
\curveto(143.43109199,228.69389669)(143.14047845,227.79054375)(143.14048234,226.64910444)
\lineto(143.14048234,220)
\lineto(141.19722117,220)
\lineto(141.19722117,231.76460817)
\lineto(143.14048234,231.76460817)
\lineto(143.14048234,229.93689226)
\curveto(143.58165081,230.65816413)(144.11035738,231.19037206)(144.72660363,231.53351766)
\curveto(145.34283891,231.87664019)(146.07462481,232.04820722)(146.92196354,232.04821926)
\curveto(147.77628965,232.04820722)(148.50107282,231.8311224)(149.09631523,231.39696417)
\curveto(149.69854067,230.96278314)(150.14321441,230.33253691)(150.43033777,229.50622357)
}
}
{
\newrgbcolor{curcolor}{0 0 0}
\pscustom[linestyle=none,fillstyle=solid,fillcolor=curcolor]
{
\newpath
\moveto(170.68227312,225.87179997)
\curveto(170.68226265,227.2933495)(170.38814774,228.40678452)(169.79992751,229.21210837)
\curveto(169.21869083,230.02441652)(168.41687756,230.43057521)(167.3944853,230.43058564)
\curveto(166.37207866,230.43057521)(165.56676403,230.02441652)(164.97853898,229.21210837)
\curveto(164.39730712,228.40678452)(164.10669358,227.2933495)(164.10669748,225.87179997)
\curveto(164.10669358,224.4502387)(164.39730712,223.33330232)(164.97853898,222.52098747)
\curveto(165.56676403,221.71567031)(166.37207866,221.31301299)(167.3944853,221.31301431)
\curveto(168.41687756,221.31301299)(169.21869083,221.71567031)(169.79992751,222.52098747)
\curveto(170.38814774,223.33330232)(170.68226265,224.4502387)(170.68227312,225.87179997)
\moveto(164.10669748,229.97890872)
\curveto(164.51285227,230.67917234)(165.02405199,231.1973748)(165.64029819,231.53351766)
\curveto(166.26353626,231.87664019)(167.00582627,232.04820722)(167.86717045,232.04821926)
\curveto(169.29572093,232.04820722)(170.45467373,231.4809856)(171.34403233,230.34655272)
\curveto(172.2403714,229.21209915)(172.6885465,227.72051639)(172.68855898,225.87179997)
\curveto(172.6885465,224.02307181)(172.2403714,222.53148905)(171.34403233,221.39704722)
\curveto(170.45467373,220.2626026)(169.29572093,219.69538099)(167.86717045,219.69538068)
\curveto(167.00582627,219.69538099)(166.26353626,219.86344665)(165.64029819,220.19957817)
\curveto(165.02405199,220.54271204)(164.51285227,221.06441587)(164.10669748,221.76469123)
\lineto(164.10669748,220)
\lineto(162.16343631,220)
\lineto(162.16343631,236.34440207)
\lineto(164.10669748,236.34440207)
\lineto(164.10669748,229.97890872)
}
}
{
\newrgbcolor{curcolor}{0 0 0}
\pscustom[linestyle=none,fillstyle=solid,fillcolor=curcolor]
{
\newpath
\moveto(182.7094813,229.95790049)
\curveto(182.49238764,230.08393978)(182.25429462,230.17497535)(181.99520152,230.23100747)
\curveto(181.74309489,230.29402186)(181.46298545,230.32553417)(181.15487236,230.3255445)
\curveto(180.06243826,230.32553417)(179.22210995,229.96839464)(178.63388489,229.25412482)
\curveto(178.05265304,228.54683924)(177.7620395,227.52794116)(177.7620434,226.19742752)
\lineto(177.7620434,220)
\lineto(175.81878222,220)
\lineto(175.81878222,231.76460817)
\lineto(177.7620434,231.76460817)
\lineto(177.7620434,229.93689226)
\curveto(178.16819818,230.65116139)(178.69690475,231.17986796)(179.34816468,231.52301354)
\curveto(179.99941364,231.87313882)(180.7907228,232.04820722)(181.72209454,232.04821926)
\curveto(181.85513867,232.04820722)(182.00219612,232.03770311)(182.16326735,232.01670692)
\curveto(182.32432197,232.00268943)(182.50289174,231.97817986)(182.69897718,231.94317812)
\lineto(182.7094813,229.95790049)
}
}
{
\newrgbcolor{curcolor}{0 0 0}
\pscustom[linestyle=none,fillstyle=solid,fillcolor=curcolor]
{
\newpath
\moveto(194.35854627,226.36549335)
\lineto(194.35854627,225.42012305)
\lineto(185.47206545,225.42012305)
\curveto(185.55609508,224.0895978)(185.95525103,223.07420109)(186.6695345,222.37392986)
\curveto(187.3908119,221.68065663)(188.39220314,221.3340212)(189.67371123,221.33402253)
\curveto(190.41599383,221.3340212)(191.13377427,221.42505677)(191.82705469,221.60712951)
\curveto(192.52731873,221.78919904)(193.22058959,222.06230574)(193.90686935,222.42645044)
\lineto(193.90686935,220.59873452)
\curveto(193.21358685,220.30461901)(192.50280915,220.08053146)(191.77453412,219.9264712)
\curveto(191.04624007,219.77241108)(190.30745143,219.69538099)(189.55816597,219.69538068)
\curveto(187.68142544,219.69538099)(186.19334405,220.24159439)(185.09391733,221.33402253)
\curveto(184.0014877,222.42644801)(183.45527429,223.9040253)(183.45527548,225.76675883)
\curveto(183.45527429,227.69250545)(183.97347675,229.21910189)(185.00988442,230.34655272)
\curveto(186.05328933,231.4809856)(187.45733789,232.04820722)(189.22203431,232.04821926)
\curveto(190.80464568,232.04820722)(192.05463405,231.53700749)(192.97200316,230.51461856)
\curveto(193.89635361,229.49921133)(194.35853418,228.11617098)(194.35854627,226.36549335)
\moveto(192.42578921,226.93271553)
\curveto(192.41177358,227.99012173)(192.1141573,228.83395141)(191.53293948,229.46420711)
\curveto(190.95870587,230.09444388)(190.19540765,230.409567)(189.24304254,230.40957741)
\curveto(188.16461422,230.409567)(187.29977633,230.10494799)(186.64852627,229.49571946)
\curveto(186.00427018,228.88647193)(185.63312517,228.02863678)(185.53509014,226.92221142)
\lineto(192.42578921,226.93271553)
}
}
{
\newrgbcolor{curcolor}{0 0 0}
\pscustom[linestyle=none,fillstyle=solid,fillcolor=curcolor]
{
}
}
{
\newrgbcolor{curcolor}{0 0 0}
\pscustom[linestyle=none,fillstyle=solid,fillcolor=curcolor]
{
\newpath
\moveto(212.12099928,229.97890872)
\lineto(212.12099928,236.34440207)
\lineto(214.05375634,236.34440207)
\lineto(214.05375634,220)
\lineto(212.12099928,220)
\lineto(212.12099928,221.76469123)
\curveto(211.71483083,221.06441587)(211.20012973,220.54271204)(210.57689446,220.19957817)
\curveto(209.96064547,219.86344665)(209.21835546,219.69538099)(208.3500222,219.69538068)
\curveto(206.9284608,219.69538099)(205.769508,220.2626026)(204.87316032,221.39704722)
\curveto(203.98381033,222.53148905)(203.5391366,224.02307181)(203.53913778,225.87179997)
\curveto(203.5391366,227.72051639)(203.98381033,229.21209915)(204.87316032,230.34655272)
\curveto(205.769508,231.4809856)(206.9284608,232.04820722)(208.3500222,232.04821926)
\curveto(209.21835546,232.04820722)(209.96064547,231.87664019)(210.57689446,231.53351766)
\curveto(211.20012973,231.1973748)(211.71483083,230.67917234)(212.12099928,229.97890872)
\moveto(205.53491953,225.87179997)
\curveto(205.53491634,224.4502387)(205.82552989,223.33330232)(206.40676103,222.52098747)
\curveto(206.99498679,221.71567031)(207.80030143,221.31301299)(208.82270735,221.31301431)
\curveto(209.84510033,221.31301299)(210.65041496,221.71567031)(211.23865367,222.52098747)
\curveto(211.8268746,223.33330232)(212.12098951,224.4502387)(212.12099928,225.87179997)
\curveto(212.12098951,227.2933495)(211.8268746,228.40678452)(211.23865367,229.21210837)
\curveto(210.65041496,230.02441652)(209.84510033,230.43057521)(208.82270735,230.43058564)
\curveto(207.80030143,230.43057521)(206.99498679,230.02441652)(206.40676103,229.21210837)
\curveto(205.82552989,228.40678452)(205.53491634,227.2933495)(205.53491953,225.87179997)
}
}
{
\newrgbcolor{curcolor}{0 0 0}
\pscustom[linestyle=none,fillstyle=solid,fillcolor=curcolor]
{
\newpath
\moveto(219.86253258,235.68264286)
\lineto(219.86253258,229.85285935)
\lineto(218.07683313,229.85285935)
\lineto(218.07683313,235.68264286)
\lineto(219.86253258,235.68264286)
}
}
{
\newrgbcolor{curcolor}{0 0 0}
\pscustom[linestyle=none,fillstyle=solid,fillcolor=curcolor]
{
\newpath
\moveto(234.02208234,226.36549335)
\lineto(234.02208234,225.42012305)
\lineto(225.13560152,225.42012305)
\curveto(225.21963115,224.0895978)(225.6187871,223.07420109)(226.33307057,222.37392986)
\curveto(227.05434797,221.68065663)(228.05573921,221.3340212)(229.3372473,221.33402253)
\curveto(230.07952991,221.3340212)(230.79731034,221.42505677)(231.49059076,221.60712951)
\curveto(232.1908548,221.78919904)(232.88412566,222.06230574)(233.57040542,222.42645044)
\lineto(233.57040542,220.59873452)
\curveto(232.87712292,220.30461901)(232.16634522,220.08053146)(231.43807019,219.9264712)
\curveto(230.70977614,219.77241108)(229.9709875,219.69538099)(229.22170204,219.69538068)
\curveto(227.34496151,219.69538099)(225.85688012,220.24159439)(224.7574534,221.33402253)
\curveto(223.66502377,222.42644801)(223.11881036,223.9040253)(223.11881155,225.76675883)
\curveto(223.11881036,227.69250545)(223.63701282,229.21910189)(224.67342049,230.34655272)
\curveto(225.7168254,231.4809856)(227.12087396,232.04820722)(228.88557038,232.04821926)
\curveto(230.46818175,232.04820722)(231.71817012,231.53700749)(232.63553923,230.51461856)
\curveto(233.55988968,229.49921133)(234.02207025,228.11617098)(234.02208234,226.36549335)
\moveto(232.08932528,226.93271553)
\curveto(232.07530965,227.99012173)(231.77769338,228.83395141)(231.19647556,229.46420711)
\curveto(230.62224194,230.09444388)(229.85894372,230.409567)(228.90657861,230.40957741)
\curveto(227.8281503,230.409567)(226.9633124,230.10494799)(226.31206234,229.49571946)
\curveto(225.66780625,228.88647193)(225.29666125,228.02863678)(225.19862621,226.92221142)
\lineto(232.08932528,226.93271553)
}
}
{
\newrgbcolor{curcolor}{0 0 0}
\pscustom[linestyle=none,fillstyle=solid,fillcolor=curcolor]
{
\newpath
\moveto(246.97365497,227.10078136)
\lineto(246.97365497,220)
\lineto(245.04089791,220)
\lineto(245.04089791,227.03775668)
\curveto(245.04088804,228.15118466)(244.82380322,228.98451023)(244.38964281,229.53773591)
\curveto(243.95546396,230.09094252)(243.30420952,230.36755059)(242.43587753,230.36756095)
\curveto(241.3924626,230.36755059)(240.56964113,230.03492063)(239.96741063,229.36967008)
\curveto(239.36517054,228.7044008)(239.0640529,227.79754649)(239.06405679,226.64910444)
\lineto(239.06405679,220)
\lineto(237.12079562,220)
\lineto(237.12079562,231.76460817)
\lineto(239.06405679,231.76460817)
\lineto(239.06405679,229.93689226)
\curveto(239.52623347,230.64415866)(240.06894551,231.17286522)(240.69219453,231.52301354)
\curveto(241.32243524,231.87313882)(242.04721841,232.04820722)(242.86654622,232.04821926)
\curveto(244.21806656,232.04820722)(245.24046601,231.62804306)(245.93374764,230.78772553)
\curveto(246.62700773,229.95438916)(246.97364316,228.725409)(246.97365497,227.10078136)
}
}
{
\newrgbcolor{curcolor}{0 0 0}
\pscustom[linestyle=none,fillstyle=solid,fillcolor=curcolor]
{
\newpath
\moveto(260.62899707,227.10078136)
\lineto(260.62899707,220)
\lineto(258.69624001,220)
\lineto(258.69624001,227.03775668)
\curveto(258.69623014,228.15118466)(258.47914532,228.98451023)(258.04498492,229.53773591)
\curveto(257.61080606,230.09094252)(256.95955162,230.36755059)(256.09121963,230.36756095)
\curveto(255.0478047,230.36755059)(254.22498323,230.03492063)(253.62275274,229.36967008)
\curveto(253.02051264,228.7044008)(252.719395,227.79754649)(252.71939889,226.64910444)
\lineto(252.71939889,220)
\lineto(250.77613772,220)
\lineto(250.77613772,231.76460817)
\lineto(252.71939889,231.76460817)
\lineto(252.71939889,229.93689226)
\curveto(253.18157557,230.64415866)(253.72428761,231.17286522)(254.34753663,231.52301354)
\curveto(254.97777734,231.87313882)(255.70256052,232.04820722)(256.52188832,232.04821926)
\curveto(257.87340866,232.04820722)(258.89580811,231.62804306)(259.58908974,230.78772553)
\curveto(260.28234983,229.95438916)(260.62898526,228.725409)(260.62899707,227.10078136)
}
}
{
\newrgbcolor{curcolor}{0 0 0}
\pscustom[linestyle=none,fillstyle=solid,fillcolor=curcolor]
{
\newpath
\moveto(274.56795789,226.36549335)
\lineto(274.56795789,225.42012305)
\lineto(265.68147707,225.42012305)
\curveto(265.7655067,224.0895978)(266.16466265,223.07420109)(266.87894612,222.37392986)
\curveto(267.60022352,221.68065663)(268.60161476,221.3340212)(269.88312285,221.33402253)
\curveto(270.62540546,221.3340212)(271.34318589,221.42505677)(272.03646631,221.60712951)
\curveto(272.73673035,221.78919904)(273.43000121,222.06230574)(274.11628097,222.42645044)
\lineto(274.11628097,220.59873452)
\curveto(273.42299847,220.30461901)(272.71222077,220.08053146)(271.98394574,219.9264712)
\curveto(271.25565169,219.77241108)(270.51686305,219.69538099)(269.76757759,219.69538068)
\curveto(267.89083706,219.69538099)(266.40275567,220.24159439)(265.30332895,221.33402253)
\curveto(264.21089932,222.42644801)(263.66468591,223.9040253)(263.6646871,225.76675883)
\curveto(263.66468591,227.69250545)(264.18288837,229.21910189)(265.21929604,230.34655272)
\curveto(266.26270095,231.4809856)(267.66674951,232.04820722)(269.43144593,232.04821926)
\curveto(271.0140573,232.04820722)(272.26404567,231.53700749)(273.18141478,230.51461856)
\curveto(274.10576523,229.49921133)(274.5679458,228.11617098)(274.56795789,226.36549335)
\moveto(272.63520083,226.93271553)
\curveto(272.6211852,227.99012173)(272.32356893,228.83395141)(271.74235111,229.46420711)
\curveto(271.16811749,230.09444388)(270.40481927,230.409567)(269.45245416,230.40957741)
\curveto(268.37402584,230.409567)(267.50918795,230.10494799)(266.85793789,229.49571946)
\curveto(266.2136818,228.88647193)(265.8425368,228.02863678)(265.74450176,226.92221142)
\lineto(272.63520083,226.93271553)
}
}
{
\newrgbcolor{curcolor}{0 0 0}
\pscustom[linestyle=none,fillstyle=solid,fillcolor=curcolor]
{
\newpath
\moveto(286.89978776,229.50622357)
\curveto(287.38296536,230.37455332)(287.96069107,231.01530366)(288.63296665,231.42847651)
\curveto(289.30521638,231.84162651)(290.09652554,232.04820722)(291.00689651,232.04821926)
\curveto(292.23236001,232.04820722)(293.17772936,231.61753895)(293.84300741,230.75621319)
\curveto(294.5082492,229.90186865)(294.84087915,228.68339259)(294.84089828,227.10078136)
\lineto(294.84089828,220)
\lineto(292.89763711,220)
\lineto(292.89763711,227.03775668)
\curveto(292.89761993,228.16519013)(292.69804195,229.00201707)(292.29890259,229.54824003)
\curveto(291.89973005,230.09444388)(291.29049202,230.36755059)(290.47118667,230.36756095)
\curveto(289.46978067,230.36755059)(288.67847151,230.03492063)(288.09725681,229.36967008)
\curveto(287.51601734,228.7044008)(287.2254038,227.79754649)(287.22541531,226.64910444)
\lineto(287.22541531,220)
\lineto(285.28215414,220)
\lineto(285.28215414,227.03775668)
\curveto(285.28214457,228.17219286)(285.0825666,229.00901981)(284.68341962,229.54824003)
\curveto(284.2842547,230.09444388)(283.66801393,230.36755059)(282.83469548,230.36756095)
\curveto(281.84730258,230.36755059)(281.06299616,230.03141926)(280.48177384,229.35916597)
\curveto(279.90054199,228.69389669)(279.60992844,227.79054375)(279.60993234,226.64910444)
\lineto(279.60993234,220)
\lineto(277.66667117,220)
\lineto(277.66667117,231.76460817)
\lineto(279.60993234,231.76460817)
\lineto(279.60993234,229.93689226)
\curveto(280.05110081,230.65816413)(280.57980737,231.19037206)(281.19605362,231.53351766)
\curveto(281.8122889,231.87664019)(282.54407481,232.04820722)(283.39141354,232.04821926)
\curveto(284.24573965,232.04820722)(284.97052282,231.8311224)(285.56576523,231.39696417)
\curveto(286.16799067,230.96278314)(286.6126644,230.33253691)(286.89978776,229.50622357)
}
}
{
\newrgbcolor{curcolor}{0 0 0}
\pscustom[linestyle=none,fillstyle=solid,fillcolor=curcolor]
{
\newpath
\moveto(298.70641701,231.76460817)
\lineto(300.63917407,231.76460817)
\lineto(300.63917407,220)
\lineto(298.70641701,220)
\lineto(298.70641701,231.76460817)
\moveto(298.70641701,236.34440207)
\lineto(300.63917407,236.34440207)
\lineto(300.63917407,233.89694341)
\lineto(298.70641701,233.89694341)
\lineto(298.70641701,236.34440207)
}
}
{
\newrgbcolor{curcolor}{0 0 0}
\pscustom[linestyle=none,fillstyle=solid,fillcolor=curcolor]
{
\newpath
\moveto(312.17269383,231.4179724)
\lineto(312.17269383,229.59025649)
\curveto(311.6264709,229.87035633)(311.05924929,230.08043841)(310.4710273,230.22050335)
\curveto(309.88278965,230.36054785)(309.27355162,230.43057521)(308.64331138,230.43058564)
\curveto(307.68393056,230.43057521)(306.96264875,230.28351775)(306.47946381,229.98941283)
\curveto(306.00327393,229.69528793)(305.76518091,229.25411557)(305.76518403,228.66589441)
\curveto(305.76518091,228.21771065)(305.93674794,227.86407248)(306.27988563,227.60497886)
\curveto(306.62301606,227.35287276)(307.31278555,227.11127837)(308.34919618,226.88019496)
\lineto(309.01095539,226.73313736)
\curveto(310.38348527,226.43901571)(311.35686557,226.02235292)(311.9310992,225.48314774)
\curveto(312.512317,224.95093432)(312.80293054,224.20514294)(312.8029407,223.24577136)
\curveto(312.80293054,222.15334131)(312.36876091,221.28850342)(311.50043051,220.6512551)
\curveto(310.63908513,220.01400547)(309.45212139,219.69538099)(307.93953571,219.69538068)
\curveto(307.30928418,219.69538099)(306.651027,219.75840561)(305.9647622,219.88445474)
\curveto(305.28549349,220.00350137)(304.56771306,220.1855725)(303.81141874,220.43066869)
\lineto(303.81141874,222.42645044)
\curveto(304.52569664,222.055303)(305.2294716,221.77519357)(305.92274574,221.58612128)
\curveto(306.61601332,221.40404856)(307.30228145,221.31301299)(307.98155217,221.31301431)
\curveto(308.89190251,221.31301299)(309.59217611,221.46707318)(310.08237506,221.77519534)
\curveto(310.57255914,222.09031668)(310.8176549,222.53148905)(310.81766307,223.09871376)
\curveto(310.8176549,223.62391586)(310.63908513,224.02657318)(310.28195324,224.30668692)
\curveto(309.9318088,224.58679205)(309.15800648,224.85639739)(307.96054394,225.11550373)
\lineto(307.28828062,225.27306545)
\curveto(306.09080813,225.52515867)(305.22597024,225.91030915)(304.69376435,226.42851804)
\curveto(304.16155437,226.95371681)(303.8954504,227.67149724)(303.89545165,228.5818615)
\curveto(303.8954504,229.6882852)(304.28760362,230.54261899)(305.07191247,231.14486542)
\curveto(305.85621647,231.74708957)(306.96965149,232.04820722)(308.41222086,232.04821926)
\curveto(309.12649417,232.04820722)(309.79875682,231.9956867)(310.42901084,231.89065755)
\curveto(311.05924929,231.78560462)(311.64047637,231.62804306)(312.17269383,231.4179724)
}
}
{
\newrgbcolor{curcolor}{0 0 0}
\pscustom[linestyle=none,fillstyle=solid,fillcolor=curcolor,opacity=0.11935484]
{
\newpath
\moveto(333.90655899,240)
\lineto(419.49635696,240)
\curveto(425.31537098,240)(429.99999237,235.80657851)(429.99999237,230.59770966)
\lineto(429.99999237,219.50483704)
\curveto(429.99999237,214.29596819)(425.31537098,210.10254669)(419.49635696,210.10254669)
\lineto(333.90655899,210.10254669)
\curveto(328.08754498,210.10254669)(323.40292358,214.29596819)(323.40292358,219.50483704)
\lineto(323.40292358,230.59770966)
\curveto(323.40292358,235.80657851)(328.08754498,240)(333.90655899,240)
\closepath
}
}
{
\newrgbcolor{curcolor}{0 0 0}
\pscustom[linewidth=2,linecolor=curcolor]
{
\newpath
\moveto(333.90655899,240)
\lineto(419.49635696,240)
\curveto(425.31537098,240)(429.99999237,235.80657851)(429.99999237,230.59770966)
\lineto(429.99999237,219.50483704)
\curveto(429.99999237,214.29596819)(425.31537098,210.10254669)(419.49635696,210.10254669)
\lineto(333.90655899,210.10254669)
\curveto(328.08754498,210.10254669)(323.40292358,214.29596819)(323.40292358,219.50483704)
\lineto(323.40292358,230.59770966)
\curveto(323.40292358,235.80657851)(328.08754498,240)(333.90655899,240)
\closepath
}
}
{
\newrgbcolor{curcolor}{0 0 0}
\pscustom[linestyle=none,fillstyle=solid,fillcolor=curcolor,opacity=0.11935484]
{
\newpath
\moveto(312.70388412,189.96066284)
\lineto(417.25014114,189.96066284)
\curveto(424.35795159,189.96066284)(430.0801239,185.79414402)(430.0801239,180.6186924)
\lineto(430.0801239,169.59698677)
\curveto(430.0801239,164.42153514)(424.35795159,160.25501633)(417.25014114,160.25501633)
\lineto(312.70388412,160.25501633)
\curveto(305.59607368,160.25501633)(299.87390137,164.42153514)(299.87390137,169.59698677)
\lineto(299.87390137,180.6186924)
\curveto(299.87390137,185.79414402)(305.59607368,189.96066284)(312.70388412,189.96066284)
\closepath
}
}
{
\newrgbcolor{curcolor}{0 0 0}
\pscustom[linewidth=2,linecolor=curcolor]
{
\newpath
\moveto(312.70388412,189.96066284)
\lineto(417.25014114,189.96066284)
\curveto(424.35795159,189.96066284)(430.0801239,185.79414402)(430.0801239,180.6186924)
\lineto(430.0801239,169.59698677)
\curveto(430.0801239,164.42153514)(424.35795159,160.25501633)(417.25014114,160.25501633)
\lineto(312.70388412,160.25501633)
\curveto(305.59607368,160.25501633)(299.87390137,164.42153514)(299.87390137,169.59698677)
\lineto(299.87390137,180.6186924)
\curveto(299.87390137,185.79414402)(305.59607368,189.96066284)(312.70388412,189.96066284)
\closepath
}
}
{
\newrgbcolor{curcolor}{1 1 1}
\pscustom[linestyle=none,fillstyle=solid,fillcolor=curcolor]
{
\newpath
\moveto(329.26973534,49.81427002)
\lineto(420.7846241,49.81427002)
\curveto(425.93425914,49.81427002)(430.0799942,46.91140156)(430.0799942,43.30559635)
\lineto(430.0799942,26.42860699)
\curveto(430.0799942,22.82280177)(425.93425914,19.91993332)(420.7846241,19.91993332)
\lineto(329.26973534,19.91993332)
\curveto(324.1201003,19.91993332)(319.97436523,22.82280177)(319.97436523,26.42860699)
\lineto(319.97436523,43.30559635)
\curveto(319.97436523,46.91140156)(324.1201003,49.81427002)(329.26973534,49.81427002)
\closepath
}
}
{
\newrgbcolor{curcolor}{0 0 0}
\pscustom[linewidth=2,linecolor=curcolor]
{
\newpath
\moveto(329.26973534,49.81427002)
\lineto(420.7846241,49.81427002)
\curveto(425.93425914,49.81427002)(430.0799942,46.91140156)(430.0799942,43.30559635)
\lineto(430.0799942,26.42860699)
\curveto(430.0799942,22.82280177)(425.93425914,19.91993332)(420.7846241,19.91993332)
\lineto(329.26973534,19.91993332)
\curveto(324.1201003,19.91993332)(319.97436523,22.82280177)(319.97436523,26.42860699)
\lineto(319.97436523,43.30559635)
\curveto(319.97436523,46.91140156)(324.1201003,49.81427002)(329.26973534,49.81427002)
\closepath
}
}
{
\newrgbcolor{curcolor}{0 0 0}
\pscustom[linestyle=none,fillstyle=solid,fillcolor=curcolor]
{
\newpath
\moveto(342.35546875,47.49609375)
\lineto(344.72265625,47.49609375)
\lineto(344.72265625,31.9921875)
\lineto(353.2421875,31.9921875)
\lineto(353.2421875,30)
\lineto(342.35546875,30)
\lineto(342.35546875,47.49609375)
}
}
{
\newrgbcolor{curcolor}{0 0 0}
\pscustom[linestyle=none,fillstyle=solid,fillcolor=curcolor]
{
\newpath
\moveto(361.5859375,36.59765625)
\curveto(359.84374352,36.59764965)(358.63671347,36.3984311)(357.96484375,36)
\curveto(357.29296482,35.6015569)(356.95702765,34.92187008)(356.95703125,33.9609375)
\curveto(356.95702765,33.1953093)(357.2070274,32.58593491)(357.70703125,32.1328125)
\curveto(358.21483889,31.68749831)(358.90233821,31.46484229)(359.76953125,31.46484375)
\curveto(360.96483614,31.46484229)(361.92186644,31.88671686)(362.640625,32.73046875)
\curveto(363.36717749,33.58202767)(363.73045838,34.71093279)(363.73046875,36.1171875)
\lineto(363.73046875,36.59765625)
\lineto(361.5859375,36.59765625)
\moveto(365.88671875,37.48828125)
\lineto(365.88671875,30)
\lineto(363.73046875,30)
\lineto(363.73046875,31.9921875)
\curveto(363.23827137,31.1953113)(362.62499073,30.60546814)(361.890625,30.22265625)
\curveto(361.1562422,29.8476564)(360.2578056,29.66015659)(359.1953125,29.66015625)
\curveto(357.85155801,29.66015659)(356.78124658,30.03515621)(355.984375,30.78515625)
\curveto(355.19531066,31.54296721)(354.80077981,32.55468495)(354.80078125,33.8203125)
\curveto(354.80077981,35.2968697)(355.29296682,36.41014984)(356.27734375,37.16015625)
\curveto(357.26952734,37.91014834)(358.74608836,38.28514796)(360.70703125,38.28515625)
\lineto(363.73046875,38.28515625)
\lineto(363.73046875,38.49609375)
\curveto(363.73045838,39.48827176)(363.40233371,40.253896)(362.74609375,40.79296875)
\curveto(362.09764751,41.33983241)(361.18358593,41.61326964)(360.00390625,41.61328125)
\curveto(359.25390036,41.61326964)(358.52343234,41.52342598)(357.8125,41.34375)
\curveto(357.10155876,41.16405134)(356.41796569,40.89452036)(355.76171875,40.53515625)
\lineto(355.76171875,42.52734375)
\curveto(356.55077806,42.83201842)(357.31640229,43.05858069)(358.05859375,43.20703125)
\curveto(358.80077581,43.36326789)(359.52343134,43.44139281)(360.2265625,43.44140625)
\curveto(362.12499123,43.44139281)(363.54295857,42.9492058)(364.48046875,41.96484375)
\curveto(365.41795669,40.98045777)(365.88670622,39.48827176)(365.88671875,37.48828125)
}
}
{
\newrgbcolor{curcolor}{0 0 0}
\pscustom[linestyle=none,fillstyle=solid,fillcolor=curcolor]
{
\newpath
\moveto(381.25,37.921875)
\lineto(381.25,30)
\lineto(379.09375,30)
\lineto(379.09375,37.8515625)
\curveto(379.09373898,39.09374091)(378.85155173,40.02342748)(378.3671875,40.640625)
\curveto(377.8828027,41.25780124)(377.15624092,41.56639468)(376.1875,41.56640625)
\curveto(375.02343055,41.56639468)(374.10546272,41.1953013)(373.43359375,40.453125)
\curveto(372.76171407,39.71092779)(372.4257769,38.69921005)(372.42578125,37.41796875)
\lineto(372.42578125,30)
\lineto(370.2578125,30)
\lineto(370.2578125,43.125)
\lineto(372.42578125,43.125)
\lineto(372.42578125,41.0859375)
\curveto(372.94140139,41.87498812)(373.54686953,42.46483129)(374.2421875,42.85546875)
\curveto(374.94530563,43.2460805)(375.75389857,43.44139281)(376.66796875,43.44140625)
\curveto(378.17577115,43.44139281)(379.31639501,42.97264328)(380.08984375,42.03515625)
\curveto(380.86326846,41.10545764)(381.24998683,39.73436527)(381.25,37.921875)
}
}
{
\newrgbcolor{curcolor}{0 0 0}
\pscustom[linestyle=none,fillstyle=solid,fillcolor=curcolor]
{
\newpath
\moveto(395.01953125,42.62109375)
\lineto(395.01953125,40.60546875)
\curveto(394.41014515,40.94139531)(393.79686452,41.19139506)(393.1796875,41.35546875)
\curveto(392.57030324,41.52733222)(391.95311636,41.61326964)(391.328125,41.61328125)
\curveto(389.92968088,41.61326964)(388.84374447,41.16795758)(388.0703125,40.27734375)
\curveto(387.29687102,39.39452186)(386.91015265,38.1523356)(386.91015625,36.55078125)
\curveto(386.91015265,34.9492138)(387.29687102,33.7031213)(388.0703125,32.8125)
\curveto(388.84374447,31.92968557)(389.92968088,31.48827976)(391.328125,31.48828125)
\curveto(391.95311636,31.48827976)(392.57030324,31.57031093)(393.1796875,31.734375)
\curveto(393.79686452,31.90624809)(394.41014515,32.16015409)(395.01953125,32.49609375)
\lineto(395.01953125,30.50390625)
\curveto(394.41795764,30.22265603)(393.79295827,30.01171874)(393.14453125,29.87109375)
\curveto(392.50389706,29.73046902)(391.82030399,29.66015659)(391.09375,29.66015625)
\curveto(389.1171817,29.66015659)(387.54687077,30.28124972)(386.3828125,31.5234375)
\curveto(385.21874809,32.76562223)(384.63671743,34.44140181)(384.63671875,36.55078125)
\curveto(384.63671743,38.69139756)(385.22265434,40.37498963)(386.39453125,41.6015625)
\curveto(387.57421449,42.82811217)(389.18749413,43.44139281)(391.234375,43.44140625)
\curveto(391.89842891,43.44139281)(392.54686577,43.37108038)(393.1796875,43.23046875)
\curveto(393.8124895,43.09764315)(394.42577014,42.89451836)(395.01953125,42.62109375)
}
}
{
\newrgbcolor{curcolor}{0 0 0}
\pscustom[linestyle=none,fillstyle=solid,fillcolor=curcolor]
{
\newpath
\moveto(410.01953125,37.1015625)
\lineto(410.01953125,36.046875)
\lineto(400.10546875,36.046875)
\curveto(400.19921508,34.56249544)(400.64452714,33.42968407)(401.44140625,32.6484375)
\curveto(402.24608804,31.87499813)(403.36327442,31.48827976)(404.79296875,31.48828125)
\curveto(405.62108466,31.48827976)(406.42186511,31.58984216)(407.1953125,31.79296875)
\curveto(407.97655105,31.99609175)(408.74998778,32.30077895)(409.515625,32.70703125)
\lineto(409.515625,30.66796875)
\curveto(408.74217529,30.33984341)(407.94920733,30.08984366)(407.13671875,29.91796875)
\curveto(406.32420896,29.746094)(405.49999103,29.66015659)(404.6640625,29.66015625)
\curveto(402.57030646,29.66015659)(400.91015187,30.26953098)(399.68359375,31.48828125)
\curveto(398.46484182,32.70702854)(397.85546743,34.35546439)(397.85546875,36.43359375)
\curveto(397.85546743,38.58202267)(398.43359185,40.28514596)(399.58984375,41.54296875)
\curveto(400.75390203,42.80858094)(402.32030671,43.44139281)(404.2890625,43.44140625)
\curveto(406.05467798,43.44139281)(407.44920783,42.87108088)(408.47265625,41.73046875)
\curveto(409.50389328,40.59764565)(410.01951776,39.05467845)(410.01953125,37.1015625)
\moveto(407.86328125,37.734375)
\curveto(407.84764493,38.91405359)(407.51561402,39.85545889)(406.8671875,40.55859375)
\curveto(406.2265528,41.26170749)(405.37499116,41.61326964)(404.3125,41.61328125)
\curveto(403.10936842,41.61326964)(402.14452564,41.27342623)(401.41796875,40.59375)
\curveto(400.69921458,39.91405259)(400.2851525,38.95702229)(400.17578125,37.72265625)
\lineto(407.86328125,37.734375)
}
}
{
\newrgbcolor{curcolor}{0 0 0}
\pscustom[linestyle=none,fillstyle=solid,fillcolor=curcolor]
{
\newpath
\moveto(421.1640625,41.109375)
\curveto(420.92186537,41.24998875)(420.65624064,41.35155115)(420.3671875,41.4140625)
\curveto(420.08592871,41.48436352)(419.77342902,41.51951973)(419.4296875,41.51953125)
\curveto(418.21093059,41.51951973)(417.27343152,41.12108263)(416.6171875,40.32421875)
\curveto(415.96874533,39.53514671)(415.6445269,38.3984291)(415.64453125,36.9140625)
\lineto(415.64453125,30)
\lineto(413.4765625,30)
\lineto(413.4765625,43.125)
\lineto(415.64453125,43.125)
\lineto(415.64453125,41.0859375)
\curveto(416.09765145,41.88280062)(416.68749461,42.47264378)(417.4140625,42.85546875)
\curveto(418.14061816,43.2460805)(419.02342977,43.44139281)(420.0625,43.44140625)
\curveto(420.21092859,43.44139281)(420.37499092,43.42967407)(420.5546875,43.40625)
\curveto(420.73436556,43.39061161)(420.93358411,43.36326789)(421.15234375,43.32421875)
\lineto(421.1640625,41.109375)
}
}
{
\newrgbcolor{curcolor}{1 1 1}
\pscustom[linestyle=none,fillstyle=solid,fillcolor=curcolor]
{
\newpath
\moveto(139.20729923,49.62908936)
\lineto(230.67112637,49.62908936)
\curveto(235.81788884,49.62908936)(239.96131134,46.74393137)(239.96131134,43.16012526)
\lineto(239.96131134,26.38610125)
\curveto(239.96131134,22.80229513)(235.81788884,19.91713715)(230.67112637,19.91713715)
\lineto(139.20729923,19.91713715)
\curveto(134.06053676,19.91713715)(129.91711426,22.80229513)(129.91711426,26.38610125)
\lineto(129.91711426,43.16012526)
\curveto(129.91711426,46.74393137)(134.06053676,49.62908936)(139.20729923,49.62908936)
\closepath
}
}
{
\newrgbcolor{curcolor}{0 0 0}
\pscustom[linewidth=1.72500002,linecolor=curcolor]
{
\newpath
\moveto(139.20729923,49.62908936)
\lineto(230.67112637,49.62908936)
\curveto(235.81788884,49.62908936)(239.96131134,46.74393137)(239.96131134,43.16012526)
\lineto(239.96131134,26.38610125)
\curveto(239.96131134,22.80229513)(235.81788884,19.91713715)(230.67112637,19.91713715)
\lineto(139.20729923,19.91713715)
\curveto(134.06053676,19.91713715)(129.91711426,22.80229513)(129.91711426,26.38610125)
\lineto(129.91711426,43.16012526)
\curveto(129.91711426,46.74393137)(134.06053676,49.62908936)(139.20729923,49.62908936)
\closepath
}
}
{
\newrgbcolor{curcolor}{0 0 0}
\pscustom[linestyle=none,fillstyle=solid,fillcolor=curcolor]
{
\newpath
\moveto(160.65234375,38.203125)
\curveto(161.16014509,38.03124197)(161.6523321,37.66405484)(162.12890625,37.1015625)
\curveto(162.61326864,36.53905596)(163.09764315,35.76561923)(163.58203125,34.78125)
\lineto(165.984375,30)
\lineto(163.44140625,30)
\lineto(161.203125,34.48828125)
\curveto(160.62498938,35.66015059)(160.06248994,36.43749356)(159.515625,36.8203125)
\curveto(158.97655352,37.2031178)(158.23827301,37.39452386)(157.30078125,37.39453125)
\lineto(154.72265625,37.39453125)
\lineto(154.72265625,30)
\lineto(152.35546875,30)
\lineto(152.35546875,47.49609375)
\lineto(157.69921875,47.49609375)
\curveto(159.69920905,47.49607625)(161.19139506,47.07810792)(162.17578125,46.2421875)
\curveto(163.16014309,45.40623459)(163.6523301,44.14451711)(163.65234375,42.45703125)
\curveto(163.6523301,41.35545739)(163.39451786,40.44139581)(162.87890625,39.71484375)
\curveto(162.37108138,38.98827226)(161.62889462,38.48436652)(160.65234375,38.203125)
\moveto(154.72265625,45.55078125)
\lineto(154.72265625,39.33984375)
\lineto(157.69921875,39.33984375)
\curveto(158.83983491,39.33983441)(159.69920905,39.6015529)(160.27734375,40.125)
\curveto(160.86327039,40.65623934)(161.15623884,41.43358232)(161.15625,42.45703125)
\curveto(161.15623884,43.48045527)(160.86327039,44.24998575)(160.27734375,44.765625)
\curveto(159.69920905,45.28904721)(158.83983491,45.5507657)(157.69921875,45.55078125)
\lineto(154.72265625,45.55078125)
}
}
{
\newrgbcolor{curcolor}{0 0 0}
\pscustom[linestyle=none,fillstyle=solid,fillcolor=curcolor]
{
\newpath
\moveto(179.09765625,37.1015625)
\lineto(179.09765625,36.046875)
\lineto(169.18359375,36.046875)
\curveto(169.27734008,34.56249544)(169.72265214,33.42968407)(170.51953125,32.6484375)
\curveto(171.32421304,31.87499813)(172.44139942,31.48827976)(173.87109375,31.48828125)
\curveto(174.69920966,31.48827976)(175.49999011,31.58984216)(176.2734375,31.79296875)
\curveto(177.05467605,31.99609175)(177.82811278,32.30077895)(178.59375,32.70703125)
\lineto(178.59375,30.66796875)
\curveto(177.82030029,30.33984341)(177.02733233,30.08984366)(176.21484375,29.91796875)
\curveto(175.40233396,29.746094)(174.57811603,29.66015659)(173.7421875,29.66015625)
\curveto(171.64843146,29.66015659)(169.98827687,30.26953098)(168.76171875,31.48828125)
\curveto(167.54296682,32.70702854)(166.93359243,34.35546439)(166.93359375,36.43359375)
\curveto(166.93359243,38.58202267)(167.51171685,40.28514596)(168.66796875,41.54296875)
\curveto(169.83202703,42.80858094)(171.39843171,43.44139281)(173.3671875,43.44140625)
\curveto(175.13280298,43.44139281)(176.52733283,42.87108088)(177.55078125,41.73046875)
\curveto(178.58201828,40.59764565)(179.09764276,39.05467845)(179.09765625,37.1015625)
\moveto(176.94140625,37.734375)
\curveto(176.92576993,38.91405359)(176.59373902,39.85545889)(175.9453125,40.55859375)
\curveto(175.3046778,41.26170749)(174.45311616,41.61326964)(173.390625,41.61328125)
\curveto(172.18749342,41.61326964)(171.22265064,41.27342623)(170.49609375,40.59375)
\curveto(169.77733958,39.91405259)(169.3632775,38.95702229)(169.25390625,37.72265625)
\lineto(176.94140625,37.734375)
}
}
{
\newrgbcolor{curcolor}{0 0 0}
\pscustom[linestyle=none,fillstyle=solid,fillcolor=curcolor]
{
\newpath
\moveto(184.76953125,46.8515625)
\lineto(184.76953125,43.125)
\lineto(189.2109375,43.125)
\lineto(189.2109375,41.44921875)
\lineto(184.76953125,41.44921875)
\lineto(184.76953125,34.32421875)
\curveto(184.76952686,33.253903)(184.91405796,32.56640368)(185.203125,32.26171875)
\curveto(185.49999487,31.95702929)(186.09765053,31.8046857)(186.99609375,31.8046875)
\lineto(189.2109375,31.8046875)
\lineto(189.2109375,30)
\lineto(186.99609375,30)
\curveto(185.33202629,30)(184.18358994,30.30859344)(183.55078125,30.92578125)
\curveto(182.91796621,31.5507797)(182.60156027,32.68359107)(182.6015625,34.32421875)
\lineto(182.6015625,41.44921875)
\lineto(181.01953125,41.44921875)
\lineto(181.01953125,43.125)
\lineto(182.6015625,43.125)
\lineto(182.6015625,46.8515625)
\lineto(184.76953125,46.8515625)
}
}
{
\newrgbcolor{curcolor}{0 0 0}
\pscustom[linestyle=none,fillstyle=solid,fillcolor=curcolor]
{
\newpath
\moveto(197.14453125,41.61328125)
\curveto(195.98827506,41.61326964)(195.07421347,41.16014509)(194.40234375,40.25390625)
\curveto(193.73046482,39.35545939)(193.39452765,38.12108563)(193.39453125,36.55078125)
\curveto(193.39452765,34.98046377)(193.72655857,33.74218376)(194.390625,32.8359375)
\curveto(195.06249473,31.93749806)(195.98046257,31.48827976)(197.14453125,31.48828125)
\curveto(198.29296025,31.48827976)(199.20311559,31.94140431)(199.875,32.84765625)
\curveto(200.54686425,33.7539025)(200.88280141,34.98827626)(200.8828125,36.55078125)
\curveto(200.88280141,38.10546064)(200.54686425,39.33592816)(199.875,40.2421875)
\curveto(199.20311559,41.15623884)(198.29296025,41.61326964)(197.14453125,41.61328125)
\moveto(197.14453125,43.44140625)
\curveto(199.01952203,43.44139281)(200.4921768,42.83201842)(201.5625,41.61328125)
\curveto(202.63279966,40.39452086)(203.16795538,38.70702254)(203.16796875,36.55078125)
\curveto(203.16795538,34.40233935)(202.63279966,32.71484104)(201.5625,31.48828125)
\curveto(200.4921768,30.26953098)(199.01952203,29.66015659)(197.14453125,29.66015625)
\curveto(195.26171329,29.66015659)(193.78515226,30.26953098)(192.71484375,31.48828125)
\curveto(191.65234189,32.71484104)(191.12109243,34.40233935)(191.12109375,36.55078125)
\curveto(191.12109243,38.70702254)(191.65234189,40.39452086)(192.71484375,41.61328125)
\curveto(193.78515226,42.83201842)(195.26171329,43.44139281)(197.14453125,43.44140625)
}
}
{
\newrgbcolor{curcolor}{0 0 0}
\pscustom[linestyle=none,fillstyle=solid,fillcolor=curcolor]
{
\newpath
\moveto(206.5078125,35.1796875)
\lineto(206.5078125,43.125)
\lineto(208.6640625,43.125)
\lineto(208.6640625,35.26171875)
\curveto(208.6640583,34.01952723)(208.90624556,33.08593441)(209.390625,32.4609375)
\curveto(209.87499459,31.84374816)(210.60155637,31.53515471)(211.5703125,31.53515625)
\curveto(212.73436673,31.53515471)(213.65233457,31.90624809)(214.32421875,32.6484375)
\curveto(215.00389571,33.39062161)(215.34373913,34.40233935)(215.34375,35.68359375)
\lineto(215.34375,43.125)
\lineto(217.5,43.125)
\lineto(217.5,30)
\lineto(215.34375,30)
\lineto(215.34375,32.015625)
\curveto(214.82030215,31.21874878)(214.21092776,30.62499938)(213.515625,30.234375)
\curveto(212.82811664,29.85156265)(212.02733619,29.66015659)(211.11328125,29.66015625)
\curveto(209.60546361,29.66015659)(208.46093351,30.12890612)(207.6796875,31.06640625)
\curveto(206.89843507,32.00390425)(206.50781046,33.37499662)(206.5078125,35.1796875)
\moveto(211.93359375,43.44140625)
\lineto(211.93359375,43.44140625)
}
}
{
\newrgbcolor{curcolor}{0 0 0}
\pscustom[linestyle=none,fillstyle=solid,fillcolor=curcolor]
{
\newpath
\moveto(229.5703125,41.109375)
\curveto(229.32811538,41.24998875)(229.06249064,41.35155115)(228.7734375,41.4140625)
\curveto(228.49217871,41.48436352)(228.17967902,41.51951973)(227.8359375,41.51953125)
\curveto(226.61718059,41.51951973)(225.67968152,41.12108263)(225.0234375,40.32421875)
\curveto(224.37499533,39.53514671)(224.0507769,38.3984291)(224.05078125,36.9140625)
\lineto(224.05078125,30)
\lineto(221.8828125,30)
\lineto(221.8828125,43.125)
\lineto(224.05078125,43.125)
\lineto(224.05078125,41.0859375)
\curveto(224.50390145,41.88280062)(225.09374461,42.47264378)(225.8203125,42.85546875)
\curveto(226.54686816,43.2460805)(227.42967977,43.44139281)(228.46875,43.44140625)
\curveto(228.61717859,43.44139281)(228.78124092,43.42967407)(228.9609375,43.40625)
\curveto(229.14061556,43.39061161)(229.33983411,43.36326789)(229.55859375,43.32421875)
\lineto(229.5703125,41.109375)
}
}
{
\newrgbcolor{curcolor}{0 0 0}
\pscustom[linewidth=2,linecolor=curcolor,linestyle=dashed,dash=8 8]
{
\newpath
\moveto(299.8133,460.48668)
\lineto(500.32522,499.8184)
}
}
{
\newrgbcolor{curcolor}{0 0 0}
\pscustom[linestyle=none,fillstyle=solid,fillcolor=curcolor]
{
\newpath
\moveto(311.01168865,457.75058911)
\lineto(297.21778492,459.95797069)
\lineto(309.15561533,467.21279855)
\curveto(307.64810037,464.01637106)(308.40958335,460.1970544)(311.01168865,457.75058911)
\lineto(311.01168865,457.75058911)
\closepath
}
}
{
\newrgbcolor{curcolor}{0 0 0}
\pscustom[linewidth=2,linecolor=curcolor,linestyle=dashed,dash=8 8]
{
\newpath
\moveto(140,40)
\lineto(70.11406,39.73606)
}
}
{
\newrgbcolor{curcolor}{0 0 0}
\pscustom[linestyle=none,fillstyle=solid,fillcolor=curcolor]
{
\newpath
\moveto(129.51948821,44.80093475)
\lineto(142.64865594,40.0292193)
\lineto(129.55590443,35.15847191)
\curveto(131.6397373,38.01283536)(131.61296003,41.9072312)(129.51948821,44.80093475)
\lineto(129.51948821,44.80093475)
\closepath
}
}
{
\newrgbcolor{curcolor}{0 0 0}
\pscustom[linewidth=2,linecolor=curcolor,linestyle=dashed,dash=8 8]
{
\newpath
\moveto(220,330)
\lineto(71.232745,330)
}
}
{
\newrgbcolor{curcolor}{0 0 0}
\pscustom[linestyle=none,fillstyle=solid,fillcolor=curcolor]
{
\newpath
\moveto(209.53769464,334.84048224)
\lineto(222.6487474,330.01921591)
\lineto(209.53769392,325.19795064)
\curveto(211.632292,328.04442372)(211.62022288,331.93889292)(209.53769464,334.84048224)
\lineto(209.53769464,334.84048224)
\closepath
}
}
{
\newrgbcolor{curcolor}{0 0 0}
\pscustom[linewidth=2.21782422,linecolor=curcolor,linestyle=dashed,dash=8.87129678 8.87129678]
{
\newpath
\moveto(63.318808,420)
\lineto(160,420)
}
}
{
\newrgbcolor{curcolor}{0 0 0}
\pscustom[linestyle=none,fillstyle=solid,fillcolor=curcolor]
{
\newpath
\moveto(148.39822288,425.36766938)
\lineto(162.93722807,420.02130876)
\lineto(148.39822209,414.67494931)
\curveto(150.72094726,417.83143778)(150.70756367,422.15006184)(148.39822288,425.36766938)
\closepath
}
}
{
\newrgbcolor{curcolor}{0 0 0}
\pscustom[linewidth=2,linecolor=curcolor,linestyle=dashed,dash=8 8]
{
\newpath
\moveto(500,420)
\lineto(399.70808,420)
}
}
{
\newrgbcolor{curcolor}{0 0 0}
\pscustom[linestyle=none,fillstyle=solid,fillcolor=curcolor]
{
\newpath
\moveto(410.17038536,415.15951776)
\lineto(397.0593326,419.98078409)
\lineto(410.17038608,424.80204936)
\curveto(408.075788,421.95557628)(408.08785712,418.06110708)(410.17038536,415.15951776)
\closepath
}
}
{
\newrgbcolor{curcolor}{0 0 0}
\pscustom[linewidth=2,linecolor=curcolor,linestyle=dashed,dash=8 8]
{
\newpath
\moveto(500,270)
\lineto(399.86062,270)
}
}
{
\newrgbcolor{curcolor}{0 0 0}
\pscustom[linestyle=none,fillstyle=solid,fillcolor=curcolor]
{
\newpath
\moveto(410.32292536,265.15951776)
\lineto(397.2118726,269.98078409)
\lineto(410.32292608,274.80204936)
\curveto(408.228328,271.95557628)(408.24039712,268.06110708)(410.32292536,265.15951776)
\closepath
}
}
{
\newrgbcolor{curcolor}{0 0 0}
\pscustom[linewidth=2.12666917,linecolor=curcolor,linestyle=dashed,dash=8.50667644 8.50667644]
{
\newpath
\moveto(500.47505,219.83946)
\lineto(413.6734,219.83946)
}
}
{
\newrgbcolor{curcolor}{0 0 0}
\pscustom[linestyle=none,fillstyle=solid,fillcolor=curcolor]
{
\newpath
\moveto(424.79833112,214.69240783)
\lineto(410.85689528,219.81902706)
\lineto(424.79833189,224.94564516)
\curveto(422.57107331,221.91889189)(422.58390682,217.7777681)(424.79833112,214.69240783)
\closepath
}
}
{
\newrgbcolor{curcolor}{0 0 0}
\pscustom[linewidth=2.10198331,linecolor=curcolor,linestyle=dashed,dash=8.40793346 8.40793346]
{
\newpath
\moveto(499.6268,171.30557)
\lineto(400,170)
}
}
{
\newrgbcolor{curcolor}{0 0 0}
\pscustom[linestyle=none,fillstyle=solid,fillcolor=curcolor]
{
\newpath
\moveto(411.06151301,165.05721353)
\lineto(397.21669222,169.94332835)
\lineto(410.9287202,175.19056371)
\curveto(408.76670473,172.17035504)(408.83302145,168.07781803)(411.06151301,165.05721353)
\closepath
}
}
{
\newrgbcolor{curcolor}{0 0 0}
\pscustom[linewidth=2,linecolor=curcolor,linestyle=dashed,dash=8 8]
{
\newpath
\moveto(490,40)
\lineto(411.0551,41.27788)
}
}
{
\newrgbcolor{curcolor}{0 0 0}
\pscustom[linestyle=none,fillstyle=solid,fillcolor=curcolor]
{
\newpath
\moveto(421.43769241,36.26870077)
\lineto(408.40638854,41.30153623)
\lineto(421.59375622,45.90996935)
\curveto(419.45336271,43.0977699)(419.40239879,39.20361547)(421.43769241,36.26870077)
\closepath
}
}
{
\newrgbcolor{curcolor}{0 0 0}
\pscustom[linestyle=none,fillstyle=solid,fillcolor=curcolor]
{
\newpath
\moveto(52.984375,422.578125)
\curveto(54.49477717,422.25519608)(55.67185933,421.58332175)(56.515625,420.5625)
\curveto(57.3697743,419.54165713)(57.7968572,418.28124172)(57.796875,416.78125)
\curveto(57.7968572,414.47916219)(57.00519133,412.69791397)(55.421875,411.4375)
\curveto(53.83852783,410.17708316)(51.58853008,409.54687545)(48.671875,409.546875)
\curveto(47.69270064,409.54687545)(46.68228498,409.64583369)(45.640625,409.84375)
\curveto(44.60937039,410.03124997)(43.54166312,410.31770802)(42.4375,410.703125)
\lineto(42.4375,413.75)
\curveto(43.31249669,413.23958009)(44.27082906,412.85416381)(45.3125,412.59375)
\curveto(46.35416031,412.333331)(47.44270089,412.2031228)(48.578125,412.203125)
\curveto(50.55728111,412.2031228)(52.06248794,412.59374741)(53.09375,413.375)
\curveto(54.13540253,414.15624584)(54.65623534,415.29166137)(54.65625,416.78125)
\curveto(54.65623534,418.15624184)(54.17186083,419.22915744)(53.203125,420)
\curveto(52.24477942,420.78123922)(50.90623909,421.17186383)(49.1875,421.171875)
\lineto(46.46875,421.171875)
\lineto(46.46875,423.765625)
\lineto(49.3125,423.765625)
\curveto(50.86457247,423.76561123)(52.05207128,424.07290259)(52.875,424.6875)
\curveto(53.69790297,425.31248469)(54.10936089,426.20831713)(54.109375,427.375)
\curveto(54.10936089,428.57289809)(53.68227798,429.48956384)(52.828125,430.125)
\curveto(51.98436302,430.77081256)(50.77082256,431.09372891)(49.1875,431.09375)
\curveto(48.32290834,431.09372891)(47.39582594,430.999979)(46.40625,430.8125)
\curveto(45.41666125,430.62497938)(44.32812067,430.333313)(43.140625,429.9375)
\lineto(43.140625,432.75)
\curveto(44.33853733,433.08331025)(45.45832788,433.33331)(46.5,433.5)
\curveto(47.55207578,433.666643)(48.54165812,433.74997625)(49.46875,433.75)
\curveto(51.86457147,433.74997625)(53.76040291,433.2031018)(55.15625,432.109375)
\curveto(56.55206678,431.02602064)(57.24998275,429.55727211)(57.25,427.703125)
\curveto(57.24998275,426.41144192)(56.88019145,425.31769302)(56.140625,424.421875)
\curveto(55.40102627,423.5364448)(54.34894398,422.92186208)(52.984375,422.578125)
}
}
{
\newrgbcolor{curcolor}{0 0 0}
\pscustom[linestyle=none,fillstyle=solid,fillcolor=curcolor]
{
\newpath
\moveto(512.09375,430.578125)
\lineto(504.125,418.125)
\lineto(512.09375,418.125)
\lineto(512.09375,430.578125)
\moveto(511.265625,433.328125)
\lineto(515.234375,433.328125)
\lineto(515.234375,418.125)
\lineto(518.5625,418.125)
\lineto(518.5625,415.5)
\lineto(515.234375,415.5)
\lineto(515.234375,410)
\lineto(512.09375,410)
\lineto(512.09375,415.5)
\lineto(501.5625,415.5)
\lineto(501.5625,418.546875)
\lineto(511.265625,433.328125)
}
}
{
\newrgbcolor{curcolor}{0 0 0}
\pscustom[linestyle=none,fillstyle=solid,fillcolor=curcolor]
{
\newpath
\moveto(53.453125,343.328125)
\lineto(65.84375,343.328125)
\lineto(65.84375,340.671875)
\lineto(56.34375,340.671875)
\lineto(56.34375,334.953125)
\curveto(56.80207653,335.10935989)(57.26040941,335.22394311)(57.71875,335.296875)
\curveto(58.17707516,335.38019295)(58.63540803,335.42185958)(59.09375,335.421875)
\curveto(61.69790497,335.42185958)(63.76040291,334.70831863)(65.28125,333.28125)
\curveto(66.80206653,331.85415481)(67.56248244,329.92186508)(67.5625,327.484375)
\curveto(67.56248244,324.97395336)(66.78123322,323.02083031)(65.21875,321.625)
\curveto(63.65623634,320.23958309)(61.45311355,319.54687545)(58.609375,319.546875)
\curveto(57.6302007,319.54687545)(56.6302017,319.6302087)(55.609375,319.796875)
\curveto(54.59895373,319.9635417)(53.55207978,320.21354145)(52.46875,320.546875)
\lineto(52.46875,323.71875)
\curveto(53.40624659,323.20833012)(54.37499562,322.82812217)(55.375,322.578125)
\curveto(56.37499363,322.32812267)(57.43228423,322.2031228)(58.546875,322.203125)
\curveto(60.34894798,322.2031228)(61.77602989,322.67708066)(62.828125,323.625)
\curveto(63.88019445,324.57291209)(64.40623559,325.85936914)(64.40625,327.484375)
\curveto(64.40623559,329.10936589)(63.88019445,330.39582294)(62.828125,331.34375)
\curveto(61.77602989,332.29165438)(60.34894798,332.76561223)(58.546875,332.765625)
\curveto(57.7031173,332.76561223)(56.85936814,332.67186233)(56.015625,332.484375)
\curveto(55.18228648,332.2968627)(54.32812067,332.00519633)(53.453125,331.609375)
\lineto(53.453125,343.328125)
}
}
{
\newrgbcolor{curcolor}{0 0 0}
\pscustom[linestyle=none,fillstyle=solid,fillcolor=curcolor]
{
\newpath
\moveto(510.5625,272.921875)
\curveto(509.14582419,272.92186208)(508.02082531,272.43748756)(507.1875,271.46875)
\curveto(506.36457697,270.4999895)(505.95311905,269.17186583)(505.953125,267.484375)
\curveto(505.95311905,265.80728586)(506.36457697,264.47916219)(507.1875,263.5)
\curveto(508.02082531,262.53124747)(509.14582419,262.04687295)(510.5625,262.046875)
\curveto(511.97915469,262.04687295)(513.09894523,262.53124747)(513.921875,263.5)
\curveto(514.75519358,264.47916219)(515.17185983,265.80728586)(515.171875,267.484375)
\curveto(515.17185983,269.17186583)(514.75519358,270.4999895)(513.921875,271.46875)
\curveto(513.09894523,272.43748756)(511.97915469,272.92186208)(510.5625,272.921875)
\moveto(516.828125,282.8125)
\lineto(516.828125,279.9375)
\curveto(516.0364423,280.31247969)(515.23435977,280.59893773)(514.421875,280.796875)
\curveto(513.61977805,280.99477067)(512.82290384,281.09372891)(512.03125,281.09375)
\curveto(509.94790672,281.09372891)(508.35415831,280.39060461)(507.25,278.984375)
\curveto(506.15624384,277.57810742)(505.53124447,275.45310955)(505.375,272.609375)
\curveto(505.98957734,273.51561148)(506.76040991,274.20831913)(507.6875,274.6875)
\curveto(508.61457472,275.17706816)(509.63540703,275.42185958)(510.75,275.421875)
\curveto(513.09373691,275.42185958)(514.94269339,274.70831863)(516.296875,273.28125)
\curveto(517.66144067,271.86457147)(518.34373166,269.93228173)(518.34375,267.484375)
\curveto(518.34373166,265.08853658)(517.63539903,263.1666635)(516.21875,261.71875)
\curveto(514.80206853,260.27083306)(512.91665375,259.54687545)(510.5625,259.546875)
\curveto(507.86457547,259.54687545)(505.80207753,260.57812442)(504.375,262.640625)
\curveto(502.94791372,264.71353695)(502.23437277,267.71353395)(502.234375,271.640625)
\curveto(502.23437277,275.32810967)(503.10937189,278.26560673)(504.859375,280.453125)
\curveto(506.60936839,282.65101902)(508.95832438,283.74997625)(511.90625,283.75)
\curveto(512.69790397,283.74997625)(513.49477817,283.67185133)(514.296875,283.515625)
\curveto(515.10935989,283.35935164)(515.95310905,283.12497688)(516.828125,282.8125)
}
}
{
\newrgbcolor{curcolor}{0 0 0}
\pscustom[linestyle=none,fillstyle=solid,fillcolor=curcolor]
{
\newpath
\moveto(502.625,233.328125)
\lineto(517.625,233.328125)
\lineto(517.625,231.984375)
\lineto(509.15625,210)
\lineto(505.859375,210)
\lineto(513.828125,230.671875)
\lineto(502.625,230.671875)
\lineto(502.625,233.328125)
}
}
{
\newrgbcolor{curcolor}{0 0 0}
\pscustom[linestyle=none,fillstyle=solid,fillcolor=curcolor]
{
\newpath
\moveto(510.171875,171.078125)
\curveto(508.67186633,171.07811392)(507.48957584,170.67707266)(506.625,169.875)
\curveto(505.77082756,169.07290759)(505.34374466,167.96874203)(505.34375,166.5625)
\curveto(505.34374466,165.15624484)(505.77082756,164.05207928)(506.625,163.25)
\curveto(507.48957584,162.44791422)(508.67186633,162.04687295)(510.171875,162.046875)
\curveto(511.67186333,162.04687295)(512.85415381,162.44791422)(513.71875,163.25)
\curveto(514.58331875,164.06249594)(515.01560998,165.1666615)(515.015625,166.5625)
\curveto(515.01560998,167.96874203)(514.58331875,169.07290759)(513.71875,169.875)
\curveto(512.86457047,170.67707266)(511.68227998,171.07811392)(510.171875,171.078125)
\moveto(507.015625,172.421875)
\curveto(505.66145267,172.75519558)(504.60416206,173.38540328)(503.84375,174.3125)
\curveto(503.09374691,175.23956809)(502.71874728,176.3697753)(502.71875,177.703125)
\curveto(502.71874728,179.56768877)(503.38020495,181.04164563)(504.703125,182.125)
\curveto(506.0364523,183.20831013)(507.85936714,183.74997625)(510.171875,183.75)
\curveto(512.49477917,183.74997625)(514.31769402,183.20831013)(515.640625,182.125)
\curveto(516.9635247,181.04164563)(517.62498237,179.56768877)(517.625,177.703125)
\curveto(517.62498237,176.3697753)(517.24477442,175.23956809)(516.484375,174.3125)
\curveto(515.73435927,173.38540328)(514.68748531,172.75519558)(513.34375,172.421875)
\curveto(514.86456847,172.06769627)(516.04685895,171.37498863)(516.890625,170.34375)
\curveto(517.74477392,169.31249069)(518.17185683,168.05207528)(518.171875,166.5625)
\curveto(518.17185683,164.30207903)(517.47914919,162.56770577)(516.09375,161.359375)
\curveto(514.71873528,160.15104152)(512.74477892,159.54687545)(510.171875,159.546875)
\curveto(507.59895073,159.54687545)(505.61978605,160.15104152)(504.234375,161.359375)
\curveto(502.85937214,162.56770577)(502.17187283,164.30207903)(502.171875,166.5625)
\curveto(502.17187283,168.05207528)(502.59895573,169.31249069)(503.453125,170.34375)
\curveto(504.30728736,171.37498863)(505.49478617,172.06769627)(507.015625,172.421875)
\moveto(505.859375,177.40625)
\curveto(505.85936914,176.19790047)(506.23436877,175.25519308)(506.984375,174.578125)
\curveto(507.74478392,173.90102777)(508.80728286,173.56248644)(510.171875,173.5625)
\curveto(511.52603014,173.56248644)(512.58332075,173.90102777)(513.34375,174.578125)
\curveto(514.11456922,175.25519308)(514.4999855,176.19790047)(514.5,177.40625)
\curveto(514.4999855,178.61456472)(514.11456922,179.55727211)(513.34375,180.234375)
\curveto(512.58332075,180.91143742)(511.52603014,181.24997875)(510.171875,181.25)
\curveto(508.80728286,181.24997875)(507.74478392,180.91143742)(506.984375,180.234375)
\curveto(506.23436877,179.55727211)(505.85936914,178.61456472)(505.859375,177.40625)
}
}
{
\newrgbcolor{curcolor}{0 0 0}
\pscustom[linestyle=none,fillstyle=solid,fillcolor=curcolor]
{
\newpath
\moveto(53.515625,30.484375)
\lineto(53.515625,33.359375)
\curveto(54.30728736,32.98437202)(55.10936989,32.69791397)(55.921875,32.5)
\curveto(56.73436827,32.30208103)(57.53124247,32.2031228)(58.3125,32.203125)
\curveto(60.39582294,32.2031228)(61.98436302,32.90103877)(63.078125,34.296875)
\curveto(64.18227748,35.7031193)(64.81248519,37.8333255)(64.96875,40.6875)
\curveto(64.36456897,39.79165687)(63.59894473,39.10415756)(62.671875,38.625)
\curveto(61.74477992,38.14582519)(60.71873928,37.90624209)(59.59375,37.90625)
\curveto(57.26040941,37.90624209)(55.41145292,38.60936639)(54.046875,40.015625)
\curveto(52.69270564,41.43228023)(52.01562298,43.36456997)(52.015625,45.8125)
\curveto(52.01562298,48.20831513)(52.72395561,50.1301882)(54.140625,51.578125)
\curveto(55.55728611,53.02601864)(57.44270089,53.74997625)(59.796875,53.75)
\curveto(62.49477917,53.74997625)(64.55206878,52.71351895)(65.96875,50.640625)
\curveto(67.39581594,48.57810642)(68.10935689,45.57810942)(68.109375,41.640625)
\curveto(68.10935689,37.9635337)(67.23435777,35.02603664)(65.484375,32.828125)
\curveto(63.74477792,30.64062436)(61.40103027,29.54687545)(58.453125,29.546875)
\curveto(57.66145067,29.54687545)(56.85936814,29.62500038)(56.046875,29.78125)
\curveto(55.23436977,29.93750006)(54.39062061,30.17187483)(53.515625,30.484375)
\moveto(59.796875,40.375)
\curveto(61.21353045,40.37498963)(62.333321,40.85936414)(63.15625,41.828125)
\curveto(63.98956934,42.7968622)(64.40623559,44.12498587)(64.40625,45.8125)
\curveto(64.40623559,47.48956584)(63.98956934,48.81248119)(63.15625,49.78125)
\curveto(62.333321,50.76039591)(61.21353045,51.24997875)(59.796875,51.25)
\curveto(58.38019995,51.24997875)(57.25520108,50.76039591)(56.421875,49.78125)
\curveto(55.59895273,48.81248119)(55.18749481,47.48956584)(55.1875,45.8125)
\curveto(55.18749481,44.12498587)(55.59895273,42.7968622)(56.421875,41.828125)
\curveto(57.25520108,40.85936414)(58.38019995,40.37498963)(59.796875,40.375)
}
}
{
\newrgbcolor{curcolor}{0 0 0}
\pscustom[linewidth=2,linecolor=curcolor,linestyle=dashed,dash=8 8]
{
\newpath
\moveto(59.189408,539.999593)
\lineto(139.51115,539.999593)
}
}
{
\newrgbcolor{curcolor}{0 0 0}
\pscustom[linestyle=none,fillstyle=solid,fillcolor=curcolor]
{
\newpath
\moveto(129.04884464,544.84007524)
\lineto(142.1598974,540.01880891)
\lineto(129.04884392,535.19754364)
\curveto(131.143442,538.04401672)(131.13137288,541.93848592)(129.04884464,544.84007524)
\closepath
}
}
{
\newrgbcolor{curcolor}{0 0 0}
\pscustom[linestyle=none,fillstyle=solid,fillcolor=curcolor]
{
\newpath
\moveto(43.96875,532.65625)
\lineto(49.125,532.65625)
\lineto(49.125,550.453125)
\lineto(43.515625,549.328125)
\lineto(43.515625,552.203125)
\lineto(49.09375,553.328125)
\lineto(52.25,553.328125)
\lineto(52.25,532.65625)
\lineto(57.40625,532.65625)
\lineto(57.40625,530)
\lineto(43.96875,530)
\lineto(43.96875,532.65625)
}
}
{
\newrgbcolor{curcolor}{0 0 0}
\pscustom[linestyle=none,fillstyle=solid,fillcolor=curcolor]
{
\newpath
\moveto(506.140625,492.65625)
\lineto(517.15625,492.65625)
\lineto(517.15625,490)
\lineto(502.34375,490)
\lineto(502.34375,492.65625)
\curveto(503.54166313,493.89582944)(505.17186983,495.55728611)(507.234375,497.640625)
\curveto(509.30728236,499.73436527)(510.60936439,501.08332225)(511.140625,501.6875)
\curveto(512.15102952,502.82290384)(512.85415381,503.78123622)(513.25,504.5625)
\curveto(513.65623634,505.35415131)(513.85936114,506.1301922)(513.859375,506.890625)
\curveto(513.85936114,508.1301902)(513.42186158,509.14060586)(512.546875,509.921875)
\curveto(511.68227998,510.7031043)(510.55207278,511.09372891)(509.15625,511.09375)
\curveto(508.1666585,511.09372891)(507.11978455,510.92185408)(506.015625,510.578125)
\curveto(504.92187008,510.23435477)(503.74999625,509.71352195)(502.5,509.015625)
\lineto(502.5,512.203125)
\curveto(503.77082956,512.71351895)(504.95832838,513.09893523)(506.0625,513.359375)
\curveto(507.1666595,513.61976805)(508.17707516,513.74997625)(509.09375,513.75)
\curveto(511.51040516,513.74997625)(513.43748656,513.14581019)(514.875,511.9375)
\curveto(516.31248369,510.72914594)(517.03123297,509.11456422)(517.03125,507.09375)
\curveto(517.03123297,506.13540053)(516.84894148,505.22394311)(516.484375,504.359375)
\curveto(516.1301922,503.50519483)(515.47915119,502.49477917)(514.53125,501.328125)
\curveto(514.27081906,501.02603064)(513.44269489,500.15103152)(512.046875,498.703125)
\curveto(510.65103102,497.26561773)(508.68228298,495.24999475)(506.140625,492.65625)
}
}
{
\newrgbcolor{curcolor}{0 0 0}
\pscustom[linestyle=none,fillstyle=solid,fillcolor=curcolor]
{
\newpath
\moveto(493.96875,32.65625)
\lineto(499.125,32.65625)
\lineto(499.125,50.453125)
\lineto(493.515625,49.328125)
\lineto(493.515625,52.203125)
\lineto(499.09375,53.328125)
\lineto(502.25,53.328125)
\lineto(502.25,32.65625)
\lineto(507.40625,32.65625)
\lineto(507.40625,30)
\lineto(493.96875,30)
\lineto(493.96875,32.65625)
}
}
{
\newrgbcolor{curcolor}{0 0 0}
\pscustom[linestyle=none,fillstyle=solid,fillcolor=curcolor]
{
\newpath
\moveto(520.546875,51.25)
\curveto(518.92186645,51.24997875)(517.69790934,50.44789622)(516.875,48.84375)
\curveto(516.06249431,47.24998275)(515.65624472,44.84894348)(515.65625,41.640625)
\curveto(515.65624472,38.44269989)(516.06249431,36.04166062)(516.875,34.4375)
\curveto(517.69790934,32.84374716)(518.92186645,32.04687295)(520.546875,32.046875)
\curveto(522.18227986,32.04687295)(523.40623697,32.84374716)(524.21875,34.4375)
\curveto(525.041652,36.04166062)(525.45310992,38.44269989)(525.453125,41.640625)
\curveto(525.45310992,44.84894348)(525.041652,47.24998275)(524.21875,48.84375)
\curveto(523.40623697,50.44789622)(522.18227986,51.24997875)(520.546875,51.25)
\moveto(520.546875,53.75)
\curveto(523.16144555,53.74997625)(525.15623522,52.71351895)(526.53125,50.640625)
\curveto(527.91664913,48.57810642)(528.60935677,45.57810942)(528.609375,41.640625)
\curveto(528.60935677,37.71353395)(527.91664913,34.71353695)(526.53125,32.640625)
\curveto(525.15623522,30.57812442)(523.16144555,29.54687545)(520.546875,29.546875)
\curveto(517.93228411,29.54687545)(515.93228611,30.57812442)(514.546875,32.640625)
\curveto(513.1718722,34.71353695)(512.48437289,37.71353395)(512.484375,41.640625)
\curveto(512.48437289,45.57810942)(513.1718722,48.57810642)(514.546875,50.640625)
\curveto(515.93228611,52.71351895)(517.93228411,53.74997625)(520.546875,53.75)
}
}
{
\newrgbcolor{curcolor}{0 0 0}
\pscustom[linestyle=none,fillstyle=solid,fillcolor=curcolor]
{
\newpath
\moveto(142.01269531,194.25488281)
\lineto(142.01269531,181.78320312)
\lineto(144.63378906,181.78320312)
\curveto(146.84667052,181.78320134)(148.46515849,182.28450292)(149.48925781,183.28710938)
\curveto(150.52049498,184.28970925)(151.03611946,185.87238996)(151.03613281,188.03515625)
\curveto(151.03611946,190.18358357)(150.52049498,191.7555221)(149.48925781,192.75097656)
\curveto(148.46515849,193.75356698)(146.84667052,194.25486856)(144.63378906,194.25488281)
\lineto(142.01269531,194.25488281)
\moveto(139.84277344,196.03808594)
\lineto(144.30078125,196.03808594)
\curveto(147.40884444,196.0380699)(149.68976664,195.38995857)(151.14355469,194.09375)
\curveto(152.59731582,192.8046747)(153.32420311,190.78514546)(153.32421875,188.03515625)
\curveto(153.32420311,185.27082806)(152.59373509,183.24055666)(151.1328125,181.94433594)
\curveto(149.67186301,180.64811133)(147.39452154,180)(144.30078125,180)
\lineto(139.84277344,180)
\lineto(139.84277344,196.03808594)
}
}
{
\newrgbcolor{curcolor}{0 0 0}
\pscustom[linestyle=none,fillstyle=solid,fillcolor=curcolor]
{
\newpath
\moveto(156.68652344,192.03125)
\lineto(158.66308594,192.03125)
\lineto(158.66308594,180)
\lineto(156.68652344,180)
\lineto(156.68652344,192.03125)
\moveto(156.68652344,196.71484375)
\lineto(158.66308594,196.71484375)
\lineto(158.66308594,194.21191406)
\lineto(156.68652344,194.21191406)
\lineto(156.68652344,196.71484375)
}
}
{
\newrgbcolor{curcolor}{0 0 0}
\pscustom[linestyle=none,fillstyle=solid,fillcolor=curcolor]
{
\newpath
\moveto(168.87890625,196.71484375)
\lineto(168.87890625,195.07128906)
\lineto(166.98828125,195.07128906)
\curveto(166.27929131,195.07127399)(165.78515118,194.92804497)(165.50585938,194.64160156)
\curveto(165.23371944,194.35512887)(165.09765187,193.83950439)(165.09765625,193.09472656)
\lineto(165.09765625,192.03125)
\lineto(170.52246094,192.03125)
\lineto(170.52246094,192.86914062)
\curveto(170.52245113,194.20831913)(170.83397426,195.18227648)(171.45703125,195.79101562)
\curveto(171.65754114,195.99152047)(171.89744975,196.15981457)(172.17675781,196.29589844)
\curveto(172.73533954,196.57517874)(173.48371119,196.71482704)(174.421875,196.71484375)
\lineto(176.29101562,196.71484375)
\lineto(176.29101562,195.07128906)
\lineto(174.40039062,195.07128906)
\curveto(173.69139327,195.07127399)(173.19725314,194.92804497)(172.91796875,194.64160156)
\curveto(172.6458214,194.35512887)(172.50975383,193.83950439)(172.50976562,193.09472656)
\lineto(172.50976562,192.03125)
\lineto(179.921875,192.03125)
\lineto(179.921875,180)
\lineto(177.93457031,180)
\lineto(177.93457031,190.49511719)
\lineto(172.50976562,190.49511719)
\lineto(172.50976562,180)
\lineto(170.52246094,180)
\lineto(170.52246094,190.49511719)
\lineto(165.09765625,190.49511719)
\lineto(165.09765625,180)
\lineto(163.11035156,180)
\lineto(163.11035156,190.49511719)
\lineto(161.21972656,190.49511719)
\lineto(161.21972656,192.03125)
\lineto(163.11035156,192.03125)
\lineto(163.11035156,192.86914062)
\curveto(163.11034917,194.20831913)(163.42187229,195.18227648)(164.04492188,195.79101562)
\curveto(164.6679648,196.40688463)(165.65624506,196.71482704)(167.00976562,196.71484375)
\lineto(168.87890625,196.71484375)
\moveto(177.93457031,196.69335938)
\lineto(179.921875,196.69335938)
\lineto(179.921875,194.19042969)
\lineto(177.93457031,194.19042969)
\lineto(177.93457031,196.69335938)
}
}
{
\newrgbcolor{curcolor}{0 0 0}
\pscustom[linestyle=none,fillstyle=solid,fillcolor=curcolor]
{
\newpath
\moveto(192.71582031,191.56933594)
\lineto(192.71582031,189.72167969)
\curveto(192.15721639,190.02961237)(191.59504247,190.2587788)(191.02929688,190.40917969)
\curveto(190.47069464,190.5667212)(189.90494,190.64549717)(189.33203125,190.64550781)
\curveto(188.05012414,190.64549717)(187.05468243,190.23729445)(186.34570312,189.42089844)
\curveto(185.6367151,188.61164503)(185.28222326,187.4729743)(185.28222656,186.00488281)
\curveto(185.28222326,184.53677932)(185.6367151,183.39452786)(186.34570312,182.578125)
\curveto(187.05468243,181.76887844)(188.05012414,181.36425645)(189.33203125,181.36425781)
\curveto(189.90494,181.36425645)(190.47069464,181.43945169)(191.02929688,181.58984375)
\curveto(191.59504247,181.74739409)(192.15721639,181.98014125)(192.71582031,182.28808594)
\lineto(192.71582031,180.46191406)
\curveto(192.16437784,180.20410136)(191.59146175,180.01074218)(190.99707031,179.88183594)
\curveto(190.4098223,179.75292993)(189.78319533,179.68847687)(189.1171875,179.68847656)
\curveto(187.30533322,179.68847687)(185.86588154,180.25781224)(184.79882812,181.39648438)
\curveto(183.73176909,182.53515371)(183.19824097,184.07128499)(183.19824219,186.00488281)
\curveto(183.19824097,187.96711443)(183.73534981,189.51040716)(184.80957031,190.63476562)
\curveto(185.89094661,191.75910282)(187.36978628,192.32127674)(189.24609375,192.32128906)
\curveto(189.85480984,192.32127674)(190.44921029,192.25682368)(191.02929688,192.12792969)
\curveto(191.60936538,192.00617289)(192.17153929,191.81997516)(192.71582031,191.56933594)
}
}
{
\newrgbcolor{curcolor}{0 0 0}
\pscustom[linestyle=none,fillstyle=solid,fillcolor=curcolor]
{
\newpath
\moveto(195.97070312,184.74804688)
\lineto(195.97070312,192.03125)
\lineto(197.94726562,192.03125)
\lineto(197.94726562,184.82324219)
\curveto(197.94726178,183.68456663)(198.16926677,182.82877321)(198.61328125,182.25585938)
\curveto(199.05728671,181.69010248)(199.72330167,181.40722516)(200.61132812,181.40722656)
\curveto(201.67837784,181.40722516)(202.51984835,181.74739409)(203.13574219,182.42773438)
\curveto(203.75877941,183.10806981)(204.07030253,184.03547774)(204.0703125,185.20996094)
\lineto(204.0703125,192.03125)
\lineto(206.046875,192.03125)
\lineto(206.046875,180)
\lineto(204.0703125,180)
\lineto(204.0703125,181.84765625)
\curveto(203.5904853,181.11718638)(203.03189211,180.57291609)(202.39453125,180.21484375)
\curveto(201.76431525,179.86393243)(201.03026651,179.68847687)(200.19238281,179.68847656)
\curveto(198.81021665,179.68847687)(197.76106405,180.11816394)(197.04492188,180.97753906)
\curveto(196.32877381,181.83691223)(195.97070126,183.09374691)(195.97070312,184.74804688)
\moveto(200.94433594,192.32128906)
\lineto(200.94433594,192.32128906)
}
}
{
\newrgbcolor{curcolor}{0 0 0}
\pscustom[linestyle=none,fillstyle=solid,fillcolor=curcolor]
{
\newpath
\moveto(210.13964844,196.71484375)
\lineto(212.11621094,196.71484375)
\lineto(212.11621094,180)
\lineto(210.13964844,180)
\lineto(210.13964844,196.71484375)
}
}
{
\newrgbcolor{curcolor}{0 0 0}
\pscustom[linestyle=none,fillstyle=solid,fillcolor=curcolor]
{
\newpath
\moveto(218.19628906,195.44726562)
\lineto(218.19628906,192.03125)
\lineto(222.26757812,192.03125)
\lineto(222.26757812,190.49511719)
\lineto(218.19628906,190.49511719)
\lineto(218.19628906,183.96386719)
\curveto(218.19628503,182.98274441)(218.32877188,182.35253671)(218.59375,182.07324219)
\curveto(218.86588072,181.79394352)(219.41373173,181.65429522)(220.23730469,181.65429688)
\lineto(222.26757812,181.65429688)
\lineto(222.26757812,180)
\lineto(220.23730469,180)
\curveto(218.71190952,180)(217.6591762,180.28287732)(217.07910156,180.84863281)
\curveto(216.49902111,181.42154806)(216.20898233,182.45995848)(216.20898438,183.96386719)
\lineto(216.20898438,190.49511719)
\lineto(214.75878906,190.49511719)
\lineto(214.75878906,192.03125)
\lineto(216.20898438,192.03125)
\lineto(216.20898438,195.44726562)
\lineto(218.19628906,195.44726562)
}
}
{
\newrgbcolor{curcolor}{0 0 0}
\pscustom[linestyle=none,fillstyle=solid,fillcolor=curcolor]
{
\newpath
\moveto(235.16894531,186.50976562)
\lineto(235.16894531,185.54296875)
\lineto(226.08105469,185.54296875)
\curveto(226.16698883,184.18228748)(226.57519154,183.14387706)(227.30566406,182.42773438)
\curveto(228.04328903,181.71874828)(229.06737655,181.36425645)(230.37792969,181.36425781)
\curveto(231.13703594,181.36425645)(231.87108468,181.45735531)(232.58007812,181.64355469)
\curveto(233.29621347,181.82975077)(234.00519713,182.10904737)(234.70703125,182.48144531)
\lineto(234.70703125,180.61230469)
\curveto(233.99803568,180.31152313)(233.27114839,180.08235669)(232.52636719,179.92480469)
\curveto(231.78156654,179.76725284)(231.02603345,179.68847687)(230.25976562,179.68847656)
\curveto(228.34048926,179.68847687)(226.81868088,180.24707007)(225.69433594,181.36425781)
\curveto(224.57714667,182.48144283)(224.01855347,183.99250903)(224.01855469,185.89746094)
\curveto(224.01855347,187.86685411)(224.54850086,189.42805047)(225.60839844,190.58105469)
\curveto(226.67545186,191.7411992)(228.11132282,192.32127674)(229.91601562,192.32128906)
\curveto(231.53449648,192.32127674)(232.81281551,191.79849081)(233.75097656,190.75292969)
\curveto(234.69627717,189.71450851)(235.16893295,188.30012191)(235.16894531,186.50976562)
\moveto(233.19238281,187.08984375)
\curveto(233.17804952,188.17121579)(232.87368785,189.03417065)(232.27929688,189.67871094)
\curveto(231.6920484,190.32323186)(230.91145023,190.64549717)(229.9375,190.64550781)
\curveto(228.83462939,190.64549717)(227.95019017,190.33397404)(227.28417969,189.7109375)
\curveto(226.6253217,189.08788154)(226.24576479,188.21060377)(226.14550781,187.07910156)
\lineto(233.19238281,187.08984375)
\moveto(231.29101562,197.59570312)
\lineto(233.42871094,197.59570312)
\lineto(229.92675781,193.55664062)
\lineto(228.28320312,193.55664062)
\lineto(231.29101562,197.59570312)
}
}
{
\newrgbcolor{curcolor}{0 0 0}
\pscustom[linestyle=none,fillstyle=solid,fillcolor=curcolor]
{
}
}
{
\newrgbcolor{curcolor}{0 0 0}
\pscustom[linestyle=none,fillstyle=solid,fillcolor=curcolor]
{
\newpath
\moveto(253.33398438,190.20507812)
\lineto(253.33398438,196.71484375)
\lineto(255.31054688,196.71484375)
\lineto(255.31054688,180)
\lineto(253.33398438,180)
\lineto(253.33398438,181.8046875)
\curveto(252.91861022,181.08854058)(252.39224356,180.55501247)(251.75488281,180.20410156)
\curveto(251.1246667,179.8603517)(250.36555287,179.68847687)(249.47753906,179.68847656)
\curveto(248.02375834,179.68847687)(246.83853817,180.26855442)(245.921875,181.42871094)
\curveto(245.01236812,182.5888646)(244.55761597,184.1142537)(244.55761719,186.00488281)
\curveto(244.55761597,187.89549992)(245.01236812,189.42088902)(245.921875,190.58105469)
\curveto(246.83853817,191.7411992)(248.02375834,192.32127674)(249.47753906,192.32128906)
\curveto(250.36555287,192.32127674)(251.1246667,192.14582119)(251.75488281,191.79492188)
\curveto(252.39224356,191.45116042)(252.91861022,190.92121304)(253.33398438,190.20507812)
\moveto(246.59863281,186.00488281)
\curveto(246.59862956,184.55110222)(246.89582978,183.40885076)(247.49023438,182.578125)
\curveto(248.09179213,181.75455554)(248.91535901,181.34277209)(249.9609375,181.34277344)
\curveto(251.00650275,181.34277209)(251.83006964,181.75455554)(252.43164062,182.578125)
\curveto(253.03319344,183.40885076)(253.33397438,184.55110222)(253.33398438,186.00488281)
\curveto(253.33397438,187.4586514)(253.03319344,188.59732213)(252.43164062,189.42089844)
\curveto(251.83006964,190.25161735)(251.00650275,190.66698152)(249.9609375,190.66699219)
\curveto(248.91535901,190.66698152)(248.09179213,190.25161735)(247.49023438,189.42089844)
\curveto(246.89582978,188.59732213)(246.59862956,187.4586514)(246.59863281,186.00488281)
}
}
{
\newrgbcolor{curcolor}{0 0 0}
\pscustom[linestyle=none,fillstyle=solid,fillcolor=curcolor]
{
\newpath
\moveto(269.67285156,186.50976562)
\lineto(269.67285156,185.54296875)
\lineto(260.58496094,185.54296875)
\curveto(260.67089508,184.18228748)(261.07909779,183.14387706)(261.80957031,182.42773438)
\curveto(262.54719528,181.71874828)(263.5712828,181.36425645)(264.88183594,181.36425781)
\curveto(265.64094219,181.36425645)(266.37499093,181.45735531)(267.08398438,181.64355469)
\curveto(267.80011972,181.82975077)(268.50910338,182.10904737)(269.2109375,182.48144531)
\lineto(269.2109375,180.61230469)
\curveto(268.50194193,180.31152313)(267.77505464,180.08235669)(267.03027344,179.92480469)
\curveto(266.28547279,179.76725284)(265.5299397,179.68847687)(264.76367188,179.68847656)
\curveto(262.84439551,179.68847687)(261.32258713,180.24707007)(260.19824219,181.36425781)
\curveto(259.08105292,182.48144283)(258.52245972,183.99250903)(258.52246094,185.89746094)
\curveto(258.52245972,187.86685411)(259.05240711,189.42805047)(260.11230469,190.58105469)
\curveto(261.17935811,191.7411992)(262.61522907,192.32127674)(264.41992188,192.32128906)
\curveto(266.03840273,192.32127674)(267.31672176,191.79849081)(268.25488281,190.75292969)
\curveto(269.20018342,189.71450851)(269.6728392,188.30012191)(269.67285156,186.50976562)
\moveto(267.69628906,187.08984375)
\curveto(267.68195577,188.17121579)(267.3775941,189.03417065)(266.78320312,189.67871094)
\curveto(266.19595465,190.32323186)(265.41535648,190.64549717)(264.44140625,190.64550781)
\curveto(263.33853564,190.64549717)(262.45409642,190.33397404)(261.78808594,189.7109375)
\curveto(261.12922795,189.08788154)(260.74967104,188.21060377)(260.64941406,187.07910156)
\lineto(267.69628906,187.08984375)
}
}
{
\newrgbcolor{curcolor}{0 0 0}
\pscustom[linestyle=none,fillstyle=solid,fillcolor=curcolor]
{
\newpath
\moveto(280.58691406,191.67675781)
\lineto(280.58691406,189.80761719)
\curveto(280.02831113,190.09406543)(279.44823358,190.30890896)(278.84667969,190.45214844)
\curveto(278.24510979,190.59536701)(277.62206353,190.66698152)(276.97753906,190.66699219)
\curveto(275.99641412,190.66698152)(275.25878465,190.51659105)(274.76464844,190.21582031)
\curveto(274.27766584,189.91502915)(274.0341765,189.46385772)(274.03417969,188.86230469)
\curveto(274.0341765,188.40396295)(274.20963205,188.04230967)(274.56054688,187.77734375)
\curveto(274.91145427,187.51952373)(275.61685721,187.27245367)(276.67675781,187.03613281)
\lineto(277.35351562,186.88574219)
\curveto(278.75715354,186.58495435)(279.75259526,186.15884801)(280.33984375,185.60742188)
\curveto(280.9342347,185.06314598)(281.23143492,184.30045143)(281.23144531,183.31933594)
\curveto(281.23143492,182.20214624)(280.78742495,181.31770702)(279.89941406,180.66601562)
\curveto(279.01854651,180.0143229)(277.80468054,179.68847687)(276.2578125,179.68847656)
\curveto(275.61327648,179.68847687)(274.94010007,179.75292993)(274.23828125,179.88183594)
\curveto(273.54361709,180.00358073)(272.80956835,180.18977846)(272.03613281,180.44042969)
\lineto(272.03613281,182.48144531)
\curveto(272.76659964,182.10188592)(273.48632548,181.81542787)(274.1953125,181.62207031)
\curveto(274.90429281,181.43587096)(275.60611503,181.34277209)(276.30078125,181.34277344)
\curveto(277.23176445,181.34277209)(277.94790956,181.50032402)(278.44921875,181.81542969)
\curveto(278.95051273,182.13769317)(279.20116352,182.5888646)(279.20117188,183.16894531)
\curveto(279.20116352,183.70605098)(279.01854651,184.11783442)(278.65332031,184.40429688)
\curveto(278.29523994,184.69075052)(277.50389959,184.96646639)(276.27929688,185.23144531)
\lineto(275.59179688,185.39257812)
\curveto(274.36718398,185.65038497)(273.48274476,186.04426479)(272.93847656,186.57421875)
\curveto(272.39420418,187.11132101)(272.12206903,187.84536976)(272.12207031,188.77636719)
\curveto(272.12206903,189.9078677)(272.5231103,190.78156474)(273.32519531,191.39746094)
\curveto(274.12727536,192.01333434)(275.2659461,192.32127674)(276.74121094,192.32128906)
\curveto(277.47167306,192.32127674)(278.15917237,192.26756586)(278.80371094,192.16015625)
\curveto(279.44823358,192.05272232)(280.04263403,191.89158967)(280.58691406,191.67675781)
}
}
{
\newrgbcolor{curcolor}{0 0 0}
\pscustom[linestyle=none,fillstyle=solid,fillcolor=curcolor]
{
\newpath
\moveto(175.35644531,159.00976562)
\lineto(175.35644531,158.04296875)
\lineto(166.26855469,158.04296875)
\curveto(166.35448883,156.68228748)(166.76269154,155.64387706)(167.49316406,154.92773438)
\curveto(168.23078903,154.21874828)(169.25487655,153.86425645)(170.56542969,153.86425781)
\curveto(171.32453594,153.86425645)(172.05858468,153.95735531)(172.76757812,154.14355469)
\curveto(173.48371347,154.32975077)(174.19269713,154.60904737)(174.89453125,154.98144531)
\lineto(174.89453125,153.11230469)
\curveto(174.18553568,152.81152313)(173.45864839,152.58235669)(172.71386719,152.42480469)
\curveto(171.96906654,152.26725284)(171.21353345,152.18847687)(170.44726562,152.18847656)
\curveto(168.52798926,152.18847687)(167.00618088,152.74707007)(165.88183594,153.86425781)
\curveto(164.76464667,154.98144283)(164.20605347,156.49250903)(164.20605469,158.39746094)
\curveto(164.20605347,160.36685411)(164.73600086,161.92805047)(165.79589844,163.08105469)
\curveto(166.86295186,164.2411992)(168.29882282,164.82127674)(170.10351562,164.82128906)
\curveto(171.72199648,164.82127674)(173.00031551,164.29849081)(173.93847656,163.25292969)
\curveto(174.88377717,162.21450851)(175.35643295,160.80012191)(175.35644531,159.00976562)
\moveto(173.37988281,159.58984375)
\curveto(173.36554952,160.67121579)(173.06118785,161.53417065)(172.46679688,162.17871094)
\curveto(171.8795484,162.82323186)(171.09895023,163.14549717)(170.125,163.14550781)
\curveto(169.02212939,163.14549717)(168.13769017,162.83397404)(167.47167969,162.2109375)
\curveto(166.8128217,161.58788154)(166.43326479,160.71060377)(166.33300781,159.57910156)
\lineto(173.37988281,159.58984375)
}
}
{
\newrgbcolor{curcolor}{0 0 0}
\pscustom[linestyle=none,fillstyle=solid,fillcolor=curcolor]
{
\newpath
\moveto(188.6015625,159.76171875)
\lineto(188.6015625,152.5)
\lineto(186.625,152.5)
\lineto(186.625,159.69726562)
\curveto(186.6249899,160.83592916)(186.40298492,161.68814185)(185.95898438,162.25390625)
\curveto(185.51496497,162.81965114)(184.84895001,163.10252846)(183.9609375,163.10253906)
\curveto(182.89387384,163.10252846)(182.05240333,162.76235953)(181.43652344,162.08203125)
\curveto(180.82063373,161.40168381)(180.51269133,160.47427588)(180.51269531,159.29980469)
\lineto(180.51269531,152.5)
\lineto(178.52539062,152.5)
\lineto(178.52539062,164.53125)
\lineto(180.51269531,164.53125)
\lineto(180.51269531,162.66210938)
\curveto(180.9853471,163.38540578)(181.54035957,163.92609534)(182.17773438,164.28417969)
\curveto(182.82225933,164.64224046)(183.56346953,164.82127674)(184.40136719,164.82128906)
\curveto(185.78351939,164.82127674)(186.82909126,164.39158967)(187.53808594,163.53222656)
\curveto(188.24705859,162.68000284)(188.60155043,161.42316816)(188.6015625,159.76171875)
}
}
{
\newrgbcolor{curcolor}{0 0 0}
\pscustom[linestyle=none,fillstyle=solid,fillcolor=curcolor]
{
\newpath
\moveto(202.56640625,159.76171875)
\lineto(202.56640625,152.5)
\lineto(200.58984375,152.5)
\lineto(200.58984375,159.69726562)
\curveto(200.58983365,160.83592916)(200.36782867,161.68814185)(199.92382812,162.25390625)
\curveto(199.47980872,162.81965114)(198.81379376,163.10252846)(197.92578125,163.10253906)
\curveto(196.85871759,163.10252846)(196.01724708,162.76235953)(195.40136719,162.08203125)
\curveto(194.78547748,161.40168381)(194.47753508,160.47427588)(194.47753906,159.29980469)
\lineto(194.47753906,152.5)
\lineto(192.49023438,152.5)
\lineto(192.49023438,164.53125)
\lineto(194.47753906,164.53125)
\lineto(194.47753906,162.66210938)
\curveto(194.95019085,163.38540578)(195.50520332,163.92609534)(196.14257812,164.28417969)
\curveto(196.78710308,164.64224046)(197.52831328,164.82127674)(198.36621094,164.82128906)
\curveto(199.74836314,164.82127674)(200.79393501,164.39158967)(201.50292969,163.53222656)
\curveto(202.21190234,162.68000284)(202.56639418,161.42316816)(202.56640625,159.76171875)
}
}
{
\newrgbcolor{curcolor}{0 0 0}
\pscustom[linestyle=none,fillstyle=solid,fillcolor=curcolor]
{
\newpath
\moveto(216.82128906,159.00976562)
\lineto(216.82128906,158.04296875)
\lineto(207.73339844,158.04296875)
\curveto(207.81933258,156.68228748)(208.22753529,155.64387706)(208.95800781,154.92773438)
\curveto(209.69563278,154.21874828)(210.7197203,153.86425645)(212.03027344,153.86425781)
\curveto(212.78937969,153.86425645)(213.52342843,153.95735531)(214.23242188,154.14355469)
\curveto(214.94855722,154.32975077)(215.65754088,154.60904737)(216.359375,154.98144531)
\lineto(216.359375,153.11230469)
\curveto(215.65037943,152.81152313)(214.92349214,152.58235669)(214.17871094,152.42480469)
\curveto(213.43391029,152.26725284)(212.6783772,152.18847687)(211.91210938,152.18847656)
\curveto(209.99283301,152.18847687)(208.47102463,152.74707007)(207.34667969,153.86425781)
\curveto(206.22949042,154.98144283)(205.67089722,156.49250903)(205.67089844,158.39746094)
\curveto(205.67089722,160.36685411)(206.20084461,161.92805047)(207.26074219,163.08105469)
\curveto(208.32779561,164.2411992)(209.76366657,164.82127674)(211.56835938,164.82128906)
\curveto(213.18684023,164.82127674)(214.46515926,164.29849081)(215.40332031,163.25292969)
\curveto(216.34862092,162.21450851)(216.8212767,160.80012191)(216.82128906,159.00976562)
\moveto(214.84472656,159.58984375)
\curveto(214.83039327,160.67121579)(214.5260316,161.53417065)(213.93164062,162.17871094)
\curveto(213.34439215,162.82323186)(212.56379398,163.14549717)(211.58984375,163.14550781)
\curveto(210.48697314,163.14549717)(209.60253392,162.83397404)(208.93652344,162.2109375)
\curveto(208.27766545,161.58788154)(207.89810854,160.71060377)(207.79785156,159.57910156)
\lineto(214.84472656,159.58984375)
}
}
{
\newrgbcolor{curcolor}{0 0 0}
\pscustom[linestyle=none,fillstyle=solid,fillcolor=curcolor]
{
\newpath
\moveto(229.43261719,162.22167969)
\curveto(229.92674588,163.10968991)(230.5175656,163.76496269)(231.20507812,164.1875)
\curveto(231.89256422,164.61001393)(232.70180821,164.82127674)(233.6328125,164.82128906)
\curveto(234.88605081,164.82127674)(235.85284672,164.38084749)(236.53320312,163.5)
\curveto(237.21352245,162.62629196)(237.55369138,161.38019945)(237.55371094,159.76171875)
\lineto(237.55371094,152.5)
\lineto(235.56640625,152.5)
\lineto(235.56640625,159.69726562)
\curveto(235.56638868,160.85025207)(235.36228732,161.70604548)(234.95410156,162.26464844)
\curveto(234.54588188,162.82323186)(233.92283563,163.10252846)(233.08496094,163.10253906)
\curveto(232.06085833,163.10252846)(231.25161434,162.76235953)(230.65722656,162.08203125)
\curveto(230.06281345,161.40168381)(229.76561323,160.47427588)(229.765625,159.29980469)
\lineto(229.765625,152.5)
\lineto(227.77832031,152.5)
\lineto(227.77832031,159.69726562)
\curveto(227.77831053,160.85741352)(227.57420917,161.71320693)(227.16601562,162.26464844)
\curveto(226.75780373,162.82323186)(226.12759603,163.10252846)(225.27539062,163.10253906)
\curveto(224.26561873,163.10252846)(223.4635362,162.7587788)(222.86914062,162.07128906)
\curveto(222.2747353,161.39094163)(221.97753508,160.46711443)(221.97753906,159.29980469)
\lineto(221.97753906,152.5)
\lineto(219.99023438,152.5)
\lineto(219.99023438,164.53125)
\lineto(221.97753906,164.53125)
\lineto(221.97753906,162.66210938)
\curveto(222.4287065,163.39972868)(222.96939606,163.94399897)(223.59960938,164.29492188)
\curveto(224.22981147,164.64582119)(224.97818312,164.82127674)(225.84472656,164.82128906)
\curveto(226.71841575,164.82127674)(227.45962595,164.59927175)(228.06835938,164.15527344)
\curveto(228.6842341,163.71125181)(229.13898625,163.0667212)(229.43261719,162.22167969)
}
}
{
\newrgbcolor{curcolor}{0 0 0}
\pscustom[linestyle=none,fillstyle=solid,fillcolor=curcolor]
{
\newpath
\moveto(241.50683594,164.53125)
\lineto(243.48339844,164.53125)
\lineto(243.48339844,152.5)
\lineto(241.50683594,152.5)
\lineto(241.50683594,164.53125)
\moveto(241.50683594,169.21484375)
\lineto(243.48339844,169.21484375)
\lineto(243.48339844,166.71191406)
\lineto(241.50683594,166.71191406)
\lineto(241.50683594,169.21484375)
}
}
{
\newrgbcolor{curcolor}{0 0 0}
\pscustom[linestyle=none,fillstyle=solid,fillcolor=curcolor]
{
\newpath
\moveto(255.27832031,164.17675781)
\lineto(255.27832031,162.30761719)
\curveto(254.71971738,162.59406543)(254.13963983,162.80890896)(253.53808594,162.95214844)
\curveto(252.93651604,163.09536701)(252.31346978,163.16698152)(251.66894531,163.16699219)
\curveto(250.68782037,163.16698152)(249.9501909,163.01659105)(249.45605469,162.71582031)
\curveto(248.96907209,162.41502915)(248.72558275,161.96385772)(248.72558594,161.36230469)
\curveto(248.72558275,160.90396295)(248.9010383,160.54230967)(249.25195312,160.27734375)
\curveto(249.60286052,160.01952373)(250.30826346,159.77245367)(251.36816406,159.53613281)
\lineto(252.04492188,159.38574219)
\curveto(253.44855979,159.08495435)(254.44400151,158.65884801)(255.03125,158.10742188)
\curveto(255.62564095,157.56314598)(255.92284117,156.80045143)(255.92285156,155.81933594)
\curveto(255.92284117,154.70214624)(255.4788312,153.81770702)(254.59082031,153.16601562)
\curveto(253.70995276,152.5143229)(252.49608679,152.18847687)(250.94921875,152.18847656)
\curveto(250.30468273,152.18847687)(249.63150632,152.25292993)(248.9296875,152.38183594)
\curveto(248.23502334,152.50358073)(247.5009746,152.68977846)(246.72753906,152.94042969)
\lineto(246.72753906,154.98144531)
\curveto(247.45800589,154.60188592)(248.17773173,154.31542787)(248.88671875,154.12207031)
\curveto(249.59569906,153.93587096)(250.29752128,153.84277209)(250.9921875,153.84277344)
\curveto(251.9231707,153.84277209)(252.63931581,154.00032402)(253.140625,154.31542969)
\curveto(253.64191898,154.63769317)(253.89256977,155.0888646)(253.89257812,155.66894531)
\curveto(253.89256977,156.20605098)(253.70995276,156.61783442)(253.34472656,156.90429688)
\curveto(252.98664619,157.19075052)(252.19530584,157.46646639)(250.97070312,157.73144531)
\lineto(250.28320312,157.89257812)
\curveto(249.05859023,158.15038497)(248.17415101,158.54426479)(247.62988281,159.07421875)
\curveto(247.08561043,159.61132101)(246.81347528,160.34536976)(246.81347656,161.27636719)
\curveto(246.81347528,162.4078677)(247.21451655,163.28156474)(248.01660156,163.89746094)
\curveto(248.81868161,164.51333434)(249.95735235,164.82127674)(251.43261719,164.82128906)
\curveto(252.16307931,164.82127674)(252.85057862,164.76756586)(253.49511719,164.66015625)
\curveto(254.13963983,164.55272232)(254.73404028,164.39158967)(255.27832031,164.17675781)
}
}
\end{pspicture}
		
		\end{center}
		
		\begin{enumerate}
		  \item Fond d'écran (présent dans l'archive stocké sur le téléphone).
		  \item Carte sélectionnée
		  \item Carte précédente
		  \item Carte suivante
		  \item Nom de la carte sélectionnée
		  \item Liste déroulante ``Type de partie''
		  \item Liste déroulante ``Nombre d'ennemis''
		  \item Liste déroulante ``Difficulté des ennemis''
		  \item Bouton ``\hyperlink{Accueil}{Retour}''
		  \item Bouton ``Lancer'' 
		\end{enumerate}
		
		\subsubsection{Description des zones}
		
			\begin{tabular}{|c|c|c|c|c|} \hline
				Numéro de zone & Type  & Description & Evènement &	Règle \\\hline
				2,3,4 & Sélecteur & Permet de sélectionner la carte voulue & Cliqué & RG3-01 \\\hline
				5 & Label & Affiche le nom de la carte choisie & RG3-01 & RG3-02 \\\hline
				6 & Liste déroulante & Permet de choisir le type de partie & Cliqué & RG3-03 \\\hline
				7 & Liste déroulante & Permet de choisir le nombre d'ennemis & Cliqué & RG3-04 \\
				  &                  & (1, 2 ou 3). & & \\\hline
				8 & Liste      & Permet de choisir la difficulté  & Cliqué & RG3-05 \\
				  & déroulante & de l'intelligence artificielle   &        & \\
				  &            & (facile/moyenne/difficile). & & \\\hline				  
				9 & Bouton & Affiche l'écran d'accueil & Cliqué & RG3-06 \\\hline
				10& Bouton & Lance une partie selon les paramètres choisis & Cliqué & RG3-07 \\\hline
			\end{tabular}

\newpage
			
		\subsubsection{Description des règles}

			\underline{RG3-01 :}
				\begin{quote}
					Décaler les cartes vers la droite ou la gauche.\\
					RG3-02\\
				\end{quote}
				
				
			\underline{RG3-02 :}
				\begin{quote}
					Afficher le nom de la carte choisie.\\
				\end{quote}


			\underline{RG3-03 :}
				\begin{quote}
					Dérouler la liste.\\
				\end{quote}


			\underline{RG3-04 :}
				\begin{quote}
					Dérouler la liste.\\				
				\end{quote}	
				

			\underline{RG3-05 :}
				\begin{quote}
					Dérouler la liste.\\				
				\end{quote}	


			\underline{RG3-06 :}
				\begin{quote}
					Afficher la page d'accueil%
						\footnote[1]{
							\hyperlink{Page d'accueil}{Page d'accueil}
							\og voir section \ref{Accueil}, page \pageref{Accueil}.\fg
						}.\\
					Supprimer la page de création d'une partie solitaire.\\
				\end{quote}


			\underline{RG3-07 :}
				\begin{quote}
					 Lancer une partie selon les paramètres choisis.\\
					 Supprimer la page de création d'une partie solitaire.\\
				\end{quote}	


\newpage

	\subsection{Création d'un compte multi-joueurs}
	
		\hypertarget{Creation compte multi-joueurs}{}
		\label{Creation compte multi-joueurs}
	
		\begin{center}
			%LaTeX with PSTricks extensions
%%Creator: inkscape 0.47
%%Please note this file requires PSTricks extensions
\psset{xunit=.5pt,yunit=.5pt,runit=.5pt}
\begin{pspicture}(560,600)
{
\newrgbcolor{curcolor}{1 1 1}
\pscustom[linestyle=none,fillstyle=solid,fillcolor=curcolor]
{
\newpath
\moveto(133.12401581,597.52220274)
\lineto(426.87598419,597.52220274)
\curveto(443.85397169,597.52220274)(457.52217102,583.85400341)(457.52217102,566.87601591)
\lineto(457.52217102,33.12401701)
\curveto(457.52217102,16.14602951)(443.85397169,2.47783018)(426.87598419,2.47783018)
\lineto(133.12401581,2.47783018)
\curveto(116.14602831,2.47783018)(102.47782898,16.14602951)(102.47782898,33.12401701)
\lineto(102.47782898,566.87601591)
\curveto(102.47782898,583.85400341)(116.14602831,597.52220274)(133.12401581,597.52220274)
\closepath
}
}
{
\newrgbcolor{curcolor}{0 0 0}
\pscustom[linewidth=4.95566034,linecolor=curcolor]
{
\newpath
\moveto(133.12401581,597.52220274)
\lineto(426.87598419,597.52220274)
\curveto(443.85397169,597.52220274)(457.52217102,583.85400341)(457.52217102,566.87601591)
\lineto(457.52217102,33.12401701)
\curveto(457.52217102,16.14602951)(443.85397169,2.47783018)(426.87598419,2.47783018)
\lineto(133.12401581,2.47783018)
\curveto(116.14602831,2.47783018)(102.47782898,16.14602951)(102.47782898,33.12401701)
\lineto(102.47782898,566.87601591)
\curveto(102.47782898,583.85400341)(116.14602831,597.52220274)(133.12401581,597.52220274)
\closepath
}
}
{
\newrgbcolor{curcolor}{1 1 1}
\pscustom[linestyle=none,fillstyle=solid,fillcolor=curcolor]
{
\newpath
\moveto(165.03649902,459.92251707)
\lineto(394.86647034,459.92251707)
\curveto(419.81245988,459.92251707)(439.89533234,439.83964462)(439.89533234,414.89365507)
\lineto(439.89533234,125.0288632)
\curveto(439.89533234,100.08287365)(419.81245988,80.0000012)(394.86647034,80.0000012)
\lineto(165.03649902,80.0000012)
\curveto(140.09050948,80.0000012)(120.00763702,100.08287365)(120.00763702,125.0288632)
\lineto(120.00763702,414.89365507)
\curveto(120.00763702,439.83964462)(140.09050948,459.92251707)(165.03649902,459.92251707)
\closepath
}
}
{
\newrgbcolor{curcolor}{0 0 0}
\pscustom[linewidth=2.08383346,linecolor=curcolor]
{
\newpath
\moveto(165.03649902,459.92251707)
\lineto(394.86647034,459.92251707)
\curveto(419.81245988,459.92251707)(439.89533234,439.83964462)(439.89533234,414.89365507)
\lineto(439.89533234,125.0288632)
\curveto(439.89533234,100.08287365)(419.81245988,80.0000012)(394.86647034,80.0000012)
\lineto(165.03649902,80.0000012)
\curveto(140.09050948,80.0000012)(120.00763702,100.08287365)(120.00763702,125.0288632)
\lineto(120.00763702,414.89365507)
\curveto(120.00763702,439.83964462)(140.09050948,459.92251707)(165.03649902,459.92251707)
\closepath
}
}
{
\newrgbcolor{curcolor}{0 0 0}
\pscustom[linestyle=none,fillstyle=solid,fillcolor=curcolor]
{
\newpath
\moveto(329.568,126.60797068)
\lineto(329.568,125.91197068)
\curveto(330.91199866,125.86397073)(331.20000048,125.76796941)(331.68,124.49597068)
\curveto(332.44799923,122.4799727)(333.33600084,120.07996828)(334.176,117.67997068)
\lineto(336.888,109.99997068)
\lineto(338.592,109.99997068)
\lineto(341.664,118.27997068)
\curveto(342.64799902,120.94396802)(343.60800058,123.39197196)(344.184,124.66397068)
\curveto(344.6879995,125.79196956)(344.9760013,125.83997076)(346.272,125.91197068)
\lineto(346.272,126.60797068)
\lineto(340.968,126.60797068)
\lineto(340.968,125.91197068)
\lineto(342.288,125.79197068)
\curveto(342.91199938,125.74397073)(342.93599988,125.50397023)(342.816,125.04797068)
\curveto(342.55200026,123.99197174)(341.80799914,121.87996828)(340.944,119.47997068)
\lineto(338.448,112.59197068)
\lineto(338.4,112.63997068)
\lineto(336.168,118.80797068)
\curveto(335.42400074,120.8959686)(334.67999938,123.0079726)(334.056,124.92797068)
\curveto(333.86400019,125.52797008)(333.93600058,125.74397073)(334.512,125.79197068)
\lineto(335.76,125.91197068)
\lineto(335.76,126.60797068)
\lineto(329.568,126.60797068)
}
}
{
\newrgbcolor{curcolor}{0 0 0}
\pscustom[linestyle=none,fillstyle=solid,fillcolor=curcolor]
{
\newpath
\moveto(353.466375,113.43197068)
\curveto(353.466375,111.48797263)(351.97837406,110.98397068)(351.042375,110.98397068)
\curveto(349.55437649,110.98397068)(348.834375,112.0399721)(348.834375,113.45597068)
\curveto(348.834375,114.58396956)(349.36237634,115.15997119)(350.706375,115.66397068)
\curveto(351.66637404,116.02397032)(352.93837553,116.47997102)(353.466375,116.81597068)
\lineto(353.466375,113.43197068)
\moveto(355.626375,118.85597068)
\curveto(355.626375,120.24796929)(355.31437138,122.04797068)(351.690375,122.04797068)
\curveto(348.97837771,122.04797068)(346.842375,120.63196936)(346.842375,119.31197068)
\curveto(346.842375,118.54397145)(347.73037546,118.18397068)(348.186375,118.18397068)
\curveto(348.6903745,118.18397068)(348.83437512,118.44797109)(348.954375,118.85597068)
\curveto(349.48237447,120.63196891)(350.46637596,121.20797068)(351.426375,121.20797068)
\curveto(352.36237406,121.20797068)(353.466375,120.72796876)(353.466375,118.80797068)
\lineto(353.466375,117.79997068)
\curveto(352.8663756,117.17597131)(350.5383731,116.57597008)(348.642375,115.97597068)
\curveto(346.91437673,115.44797121)(346.410375,114.24796956)(346.410375,113.11997068)
\curveto(346.410375,111.31997248)(347.61037726,109.71197068)(349.866375,109.71197068)
\curveto(351.35437351,109.75997064)(352.67437577,110.64797119)(353.442375,111.15197068)
\curveto(353.77837466,110.26397157)(354.16237584,109.71197068)(355.002375,109.71197068)
\curveto(355.89037411,109.71197068)(356.92237591,109.97597114)(357.834375,110.43197068)
\lineto(357.690375,111.00797068)
\curveto(357.35437534,110.93597076)(356.82637464,110.88797078)(356.466375,110.98397068)
\curveto(356.03437543,111.07997059)(355.626375,111.5359722)(355.626375,113.04797068)
\lineto(355.626375,118.85597068)
}
}
{
\newrgbcolor{curcolor}{0 0 0}
\pscustom[linestyle=none,fillstyle=solid,fillcolor=curcolor]
{
\newpath
\moveto(362.332875,127.66397068)
\lineto(362.164875,127.80797068)
\lineto(358.396875,127.20797068)
\lineto(358.396875,126.60797068)
\lineto(359.308875,126.51197068)
\curveto(359.93287438,126.43997076)(360.076875,126.31996977)(360.076875,125.40797068)
\lineto(360.076875,112.08797068)
\curveto(360.076875,110.86397191)(360.00487327,110.81597056)(358.276875,110.69597068)
\lineto(358.276875,109.99997068)
\lineto(364.132875,109.99997068)
\lineto(364.132875,110.69597068)
\curveto(362.4288767,110.81597056)(362.332875,110.86397191)(362.332875,112.08797068)
\lineto(362.332875,127.66397068)
}
}
{
\newrgbcolor{curcolor}{0 0 0}
\pscustom[linestyle=none,fillstyle=solid,fillcolor=curcolor]
{
\newpath
\moveto(365.388,121.44797068)
\lineto(365.388,120.84797068)
\lineto(366.3,120.72797068)
\curveto(366.92399938,120.63197078)(367.068,120.51196982)(367.068,119.64797068)
\lineto(367.068,112.08797068)
\curveto(367.068,110.86397191)(366.99599827,110.81597056)(365.268,110.69597068)
\lineto(365.268,109.99997068)
\lineto(371.124,109.99997068)
\lineto(371.124,110.69597068)
\curveto(369.4200017,110.81597056)(369.324,110.86397191)(369.324,112.08797068)
\lineto(369.324,121.90397068)
\lineto(369.156,122.04797068)
\lineto(365.388,121.44797068)
\moveto(368.172,127.15997068)
\curveto(367.30800086,127.15997068)(366.708,126.53596982)(366.708,125.67197068)
\curveto(366.708,124.83197152)(367.30800086,124.23197068)(368.172,124.23197068)
\curveto(369.05999911,124.23197068)(369.61200002,124.83197152)(369.636,125.67197068)
\curveto(369.636,126.53596982)(369.05999911,127.15997068)(368.172,127.15997068)
}
}
{
\newrgbcolor{curcolor}{0 0 0}
\pscustom[linestyle=none,fillstyle=solid,fillcolor=curcolor]
{
\newpath
\moveto(385.476375,109.99997068)
\lineto(385.476375,110.69597068)
\curveto(383.7723767,110.81597056)(383.676375,110.86397191)(383.676375,112.08797068)
\lineto(383.676375,127.66397068)
\lineto(383.508375,127.80797068)
\lineto(379.740375,127.20797068)
\lineto(379.740375,126.60797068)
\lineto(380.652375,126.51197068)
\curveto(381.27637438,126.43997076)(381.420375,126.31996977)(381.420375,125.40797068)
\lineto(381.420375,121.47197068)
\curveto(380.74837567,121.83197032)(379.8603739,122.04797068)(378.756375,122.04797068)
\curveto(376.78837697,122.04797068)(375.32437399,121.42396965)(374.316375,120.39197068)
\curveto(373.28437603,119.31197176)(372.660375,117.7279686)(372.660375,115.63997068)
\curveto(372.660375,112.08797424)(374.50837817,109.71197068)(377.676375,109.71197068)
\curveto(378.9723737,109.71197068)(380.1003763,110.28797174)(381.396375,111.34397068)
\lineto(381.396375,110.16797068)
\lineto(381.588375,109.99997068)
\lineto(385.476375,109.99997068)
\moveto(381.420375,114.07997068)
\curveto(381.420375,113.55197121)(381.39637486,113.14397028)(381.252375,112.73597068)
\curveto(380.77237548,111.48797193)(379.71637375,111.00797068)(378.468375,111.00797068)
\curveto(376.26037721,111.00797068)(375.156375,113.16797344)(375.156375,115.92797068)
\curveto(375.156375,119.04796756)(376.28437745,121.20797068)(378.732375,121.20797068)
\curveto(379.86037387,121.20797068)(380.77237543,120.6799697)(381.204375,119.69597068)
\curveto(381.39637481,119.28797109)(381.420375,118.95197004)(381.420375,118.30397068)
\lineto(381.420375,114.07997068)
}
}
{
\newrgbcolor{curcolor}{0 0 0}
\pscustom[linestyle=none,fillstyle=solid,fillcolor=curcolor]
{
\newpath
\moveto(396.412125,116.31197068)
\curveto(396.9161245,116.31197068)(397.372125,116.43197148)(397.372125,117.22397068)
\curveto(397.372125,118.63996927)(396.91612061,122.04797068)(392.524125,122.04797068)
\curveto(388.78012874,122.04797068)(387.004125,119.38396708)(387.004125,115.78397068)
\curveto(387.004125,111.99197448)(388.63612891,109.66397073)(392.548125,109.71197068)
\curveto(395.21212234,109.73597066)(396.58012565,111.19997258)(397.228125,113.09597068)
\lineto(396.508125,113.47997068)
\curveto(395.83612567,112.08797208)(394.9241231,110.93597068)(393.028125,110.93597068)
\curveto(390.05212798,110.93597068)(389.45212505,113.81597318)(389.500125,116.31197068)
\lineto(396.412125,116.31197068)
\moveto(389.524125,117.22397068)
\curveto(389.524125,118.2079697)(389.8841275,121.20797068)(392.380125,121.20797068)
\curveto(394.61212277,121.20797068)(394.876125,118.92796975)(394.876125,117.99197068)
\curveto(394.876125,117.53597114)(394.73212433,117.22397068)(394.060125,117.22397068)
\lineto(389.524125,117.22397068)
}
}
{
\newrgbcolor{curcolor}{0 0 0}
\pscustom[linestyle=none,fillstyle=solid,fillcolor=curcolor]
{
\newpath
\moveto(402.69225,122.04797068)
\lineto(399.04425,121.44797068)
\lineto(399.04425,120.84797068)
\lineto(399.95625,120.72797068)
\curveto(400.58024938,120.63197078)(400.72425,120.51196982)(400.72425,119.64797068)
\lineto(400.72425,112.08797068)
\curveto(400.72425,110.86397191)(400.60424832,110.81597056)(398.92425,110.69597068)
\lineto(398.92425,109.99997068)
\lineto(405.26025,109.99997068)
\lineto(405.26025,110.69597068)
\curveto(403.12425214,110.81597056)(402.98025,110.86397191)(402.98025,112.08797068)
\lineto(402.98025,117.70397068)
\curveto(402.98025,119.55196884)(403.79625055,120.17597068)(404.34825,120.17597068)
\curveto(404.73224962,120.17597068)(405.1402507,120.0319703)(405.83625,119.64797068)
\curveto(406.00424983,119.55197078)(406.19625012,119.52797068)(406.31625,119.52797068)
\curveto(406.89224942,119.52797068)(407.42025,120.12797143)(407.42025,120.87197068)
\curveto(407.42025,121.39997016)(407.08424906,122.04797068)(406.14825,122.04797068)
\curveto(405.28425086,122.04797068)(404.56424842,121.51996929)(402.98025,120.12797068)
\lineto(402.69225,122.04797068)
}
}
{
\newrgbcolor{curcolor}{0 0 0}
\pscustom[linestyle=none,fillstyle=solid,fillcolor=curcolor]
{
\newpath
\moveto(165.68358765,109.88037229)
\lineto(165.68358765,110.57637229)
\curveto(164.29158904,110.69637217)(164.02758721,110.76837354)(163.59558765,112.01637229)
\lineto(158.50758765,126.48837229)
\lineto(156.73158765,126.48837229)
\lineto(154.21158765,119.38437229)
\curveto(153.46758839,117.29637438)(152.45958681,114.41637011)(151.61958765,112.23237229)
\curveto(151.11558815,110.93637359)(150.89958613,110.64837222)(149.38758765,110.57637229)
\lineto(149.38758765,109.88037229)
\lineto(154.66758765,109.88037229)
\lineto(154.66758765,110.57637229)
\lineto(153.41958765,110.69637229)
\curveto(152.69958837,110.76837222)(152.62758784,111.00837294)(152.81958765,111.65637229)
\curveto(153.22758724,113.09637085)(153.7315882,114.584374)(154.28358765,116.28837229)
\lineto(159.65958765,116.28837229)
\lineto(161.26758765,111.70437229)
\curveto(161.50758741,111.00837299)(161.38758688,110.74437222)(160.61958765,110.67237229)
\lineto(159.53958765,110.57637229)
\lineto(159.53958765,109.88037229)
\lineto(165.68358765,109.88037229)
\moveto(159.37158765,117.20037229)
\lineto(154.59558765,117.20037229)
\curveto(155.36358688,119.60036989)(156.20358837,121.95237441)(156.92358765,124.06437229)
\lineto(156.99558765,124.06437229)
\lineto(159.37158765,117.20037229)
}
}
{
\newrgbcolor{curcolor}{0 0 0}
\pscustom[linestyle=none,fillstyle=solid,fillcolor=curcolor]
{
\newpath
\moveto(178.01396265,118.11237229)
\curveto(178.01396265,120.53636987)(176.69396044,121.92837229)(174.48596265,121.92837229)
\curveto(172.70996442,121.92837229)(171.60596116,121.06437133)(170.11796265,120.10437229)
\lineto(169.78196265,121.92837229)
\lineto(166.20596265,121.32837229)
\lineto(166.20596265,120.72837229)
\lineto(167.11796265,120.60837229)
\curveto(167.74196202,120.51237239)(167.88596265,120.39237143)(167.88596265,119.52837229)
\lineto(167.88596265,111.96837229)
\curveto(167.88596265,110.74437352)(167.81396092,110.69637217)(166.08596265,110.57637229)
\lineto(166.08596265,109.88037229)
\lineto(171.94196265,109.88037229)
\lineto(171.94196265,110.57637229)
\curveto(170.23796435,110.69637217)(170.14196265,110.74437352)(170.14196265,111.96837229)
\lineto(170.14196265,117.58437229)
\curveto(170.14196265,118.18437169)(170.18996284,118.52037268)(170.38196265,118.90437229)
\curveto(170.88596214,119.84037136)(171.96596392,120.63237229)(173.23796265,120.63237229)
\curveto(174.86996101,120.63237229)(175.75796265,119.72037021)(175.75796265,117.63237229)
\lineto(175.75796265,111.96837229)
\curveto(175.75796265,110.74437352)(175.68596092,110.69637217)(173.95796265,110.57637229)
\lineto(173.95796265,109.88037229)
\lineto(179.81396265,109.88037229)
\lineto(179.81396265,110.57637229)
\curveto(178.10996435,110.69637217)(178.01396265,110.74437352)(178.01396265,111.96837229)
\lineto(178.01396265,118.11237229)
}
}
{
\newrgbcolor{curcolor}{0 0 0}
\pscustom[linestyle=none,fillstyle=solid,fillcolor=curcolor]
{
\newpath
\moveto(192.87333765,118.11237229)
\curveto(192.87333765,120.53636987)(191.55333544,121.92837229)(189.34533765,121.92837229)
\curveto(187.56933942,121.92837229)(186.46533616,121.06437133)(184.97733765,120.10437229)
\lineto(184.64133765,121.92837229)
\lineto(181.06533765,121.32837229)
\lineto(181.06533765,120.72837229)
\lineto(181.97733765,120.60837229)
\curveto(182.60133702,120.51237239)(182.74533765,120.39237143)(182.74533765,119.52837229)
\lineto(182.74533765,111.96837229)
\curveto(182.74533765,110.74437352)(182.67333592,110.69637217)(180.94533765,110.57637229)
\lineto(180.94533765,109.88037229)
\lineto(186.80133765,109.88037229)
\lineto(186.80133765,110.57637229)
\curveto(185.09733935,110.69637217)(185.00133765,110.74437352)(185.00133765,111.96837229)
\lineto(185.00133765,117.58437229)
\curveto(185.00133765,118.18437169)(185.04933784,118.52037268)(185.24133765,118.90437229)
\curveto(185.74533714,119.84037136)(186.82533892,120.63237229)(188.09733765,120.63237229)
\curveto(189.72933601,120.63237229)(190.61733765,119.72037021)(190.61733765,117.63237229)
\lineto(190.61733765,111.96837229)
\curveto(190.61733765,110.74437352)(190.54533592,110.69637217)(188.81733765,110.57637229)
\lineto(188.81733765,109.88037229)
\lineto(194.67333765,109.88037229)
\lineto(194.67333765,110.57637229)
\curveto(192.96933935,110.69637217)(192.87333765,110.74437352)(192.87333765,111.96837229)
\lineto(192.87333765,118.11237229)
}
}
{
\newrgbcolor{curcolor}{0 0 0}
\pscustom[linestyle=none,fillstyle=solid,fillcolor=curcolor]
{
\newpath
\moveto(209.22071265,109.88037229)
\lineto(209.22071265,110.57637229)
\curveto(207.5647143,110.69637217)(207.42071265,110.74437352)(207.42071265,111.96837229)
\lineto(207.42071265,121.78437229)
\lineto(207.25271265,121.92837229)
\lineto(203.48471265,121.32837229)
\lineto(203.48471265,120.72837229)
\lineto(204.39671265,120.60837229)
\curveto(205.02071202,120.51237239)(205.16471265,120.39237143)(205.16471265,119.52837229)
\lineto(205.16471265,114.12837229)
\curveto(205.16471265,113.45637297)(205.09271253,112.92837205)(204.97271265,112.68837229)
\curveto(204.4207132,111.5843734)(203.2687114,110.88837229)(202.02071265,110.88837229)
\curveto(200.62871404,110.88837229)(199.59671265,111.75237417)(199.59671265,113.62437229)
\lineto(199.59671265,121.78437229)
\lineto(199.42871265,121.92837229)
\lineto(195.66071265,121.32837229)
\lineto(195.66071265,120.72837229)
\lineto(196.57271265,120.60837229)
\curveto(197.19671202,120.51237239)(197.34071265,120.39237143)(197.34071265,119.52837229)
\lineto(197.34071265,113.19237229)
\curveto(197.34071265,110.43237505)(198.97271442,109.59237229)(200.74871265,109.59237229)
\curveto(202.78871061,109.59237229)(204.44471334,110.98437263)(205.14071265,111.32037229)
\lineto(205.38071265,109.88037229)
\lineto(209.22071265,109.88037229)
}
}
{
\newrgbcolor{curcolor}{0 0 0}
\pscustom[linestyle=none,fillstyle=solid,fillcolor=curcolor]
{
\newpath
\moveto(214.10396265,127.54437229)
\lineto(213.93596265,127.68837229)
\lineto(210.16796265,127.08837229)
\lineto(210.16796265,126.48837229)
\lineto(211.07996265,126.39237229)
\curveto(211.70396202,126.32037237)(211.84796265,126.20037138)(211.84796265,125.28837229)
\lineto(211.84796265,111.96837229)
\curveto(211.84796265,110.74437352)(211.77596092,110.69637217)(210.04796265,110.57637229)
\lineto(210.04796265,109.88037229)
\lineto(215.90396265,109.88037229)
\lineto(215.90396265,110.57637229)
\curveto(214.19996435,110.69637217)(214.10396265,110.74437352)(214.10396265,111.96837229)
\lineto(214.10396265,127.54437229)
}
}
{
\newrgbcolor{curcolor}{0 0 0}
\pscustom[linestyle=none,fillstyle=solid,fillcolor=curcolor]
{
\newpath
\moveto(226.85508765,116.19237229)
\curveto(227.35908714,116.19237229)(227.81508765,116.31237309)(227.81508765,117.10437229)
\curveto(227.81508765,118.52037088)(227.35908325,121.92837229)(222.96708765,121.92837229)
\curveto(219.22309139,121.92837229)(217.44708765,119.26436869)(217.44708765,115.66437229)
\curveto(217.44708765,111.87237609)(219.07909156,109.54437234)(222.99108765,109.59237229)
\curveto(225.65508498,109.61637227)(227.02308829,111.08037419)(227.67108765,112.97637229)
\lineto(226.95108765,113.36037229)
\curveto(226.27908832,111.96837369)(225.36708575,110.81637229)(223.47108765,110.81637229)
\curveto(220.49509062,110.81637229)(219.89508769,113.69637479)(219.94308765,116.19237229)
\lineto(226.85508765,116.19237229)
\moveto(219.96708765,117.10437229)
\curveto(219.96708765,118.08837131)(220.32709014,121.08837229)(222.82308765,121.08837229)
\curveto(225.05508541,121.08837229)(225.31908765,118.80837136)(225.31908765,117.87237229)
\curveto(225.31908765,117.41637275)(225.17508697,117.10437229)(224.50308765,117.10437229)
\lineto(219.96708765,117.10437229)
}
}
{
\newrgbcolor{curcolor}{0 0 0}
\pscustom[linestyle=none,fillstyle=solid,fillcolor=curcolor]
{
\newpath
\moveto(233.13521265,121.92837229)
\lineto(229.48721265,121.32837229)
\lineto(229.48721265,120.72837229)
\lineto(230.39921265,120.60837229)
\curveto(231.02321202,120.51237239)(231.16721265,120.39237143)(231.16721265,119.52837229)
\lineto(231.16721265,111.96837229)
\curveto(231.16721265,110.74437352)(231.04721097,110.69637217)(229.36721265,110.57637229)
\lineto(229.36721265,109.88037229)
\lineto(235.70321265,109.88037229)
\lineto(235.70321265,110.57637229)
\curveto(233.56721478,110.69637217)(233.42321265,110.74437352)(233.42321265,111.96837229)
\lineto(233.42321265,117.58437229)
\curveto(233.42321265,119.43237045)(234.2392132,120.05637229)(234.79121265,120.05637229)
\curveto(235.17521226,120.05637229)(235.58321334,119.91237191)(236.27921265,119.52837229)
\curveto(236.44721248,119.43237239)(236.63921277,119.40837229)(236.75921265,119.40837229)
\curveto(237.33521207,119.40837229)(237.86321265,120.00837304)(237.86321265,120.75237229)
\curveto(237.86321265,121.28037177)(237.52721171,121.92837229)(236.59121265,121.92837229)
\curveto(235.72721351,121.92837229)(235.00721106,121.4003709)(233.42321265,120.00837229)
\lineto(233.13521265,121.92837229)
}
}
{
\newrgbcolor{curcolor}{1 1 1}
\pscustom[linestyle=none,fillstyle=solid,fillcolor=curcolor]
{
\newpath
\moveto(149.98284912,480.0171578)
\lineto(329.94728088,480.0171578)
\lineto(329.94728088,450.20041967)
\lineto(149.98284912,450.20041967)
\lineto(149.98284912,480.0171578)
\closepath
}
}
{
\newrgbcolor{curcolor}{0 0 0}
\pscustom[linewidth=2,linecolor=curcolor]
{
\newpath
\moveto(149.98284912,480.0171578)
\lineto(329.94728088,480.0171578)
\lineto(329.94728088,450.20041967)
\lineto(149.98284912,450.20041967)
\lineto(149.98284912,480.0171578)
\closepath
}
}
{
\newrgbcolor{curcolor}{0 0 0}
\pscustom[linestyle=none,fillstyle=solid,fillcolor=curcolor]
{
}
}
{
\newrgbcolor{curcolor}{0 0 0}
\pscustom[linestyle=none,fillstyle=solid,fillcolor=curcolor]
{
}
}
{
\newrgbcolor{curcolor}{0 0 0}
\pscustom[linestyle=none,fillstyle=solid,fillcolor=curcolor]
{
}
}
{
\newrgbcolor{curcolor}{0 0 0}
\pscustom[linestyle=none,fillstyle=solid,fillcolor=curcolor]
{
}
}
{
\newrgbcolor{curcolor}{0 0 0}
\pscustom[linestyle=none,fillstyle=solid,fillcolor=curcolor]
{
}
}
{
\newrgbcolor{curcolor}{0 0 0}
\pscustom[linestyle=none,fillstyle=solid,fillcolor=curcolor]
{
}
}
{
\newrgbcolor{curcolor}{0 0 0}
\pscustom[linestyle=none,fillstyle=solid,fillcolor=curcolor]
{
}
}
{
\newrgbcolor{curcolor}{0 0 0}
\pscustom[linestyle=none,fillstyle=solid,fillcolor=curcolor]
{
\newpath
\moveto(168.550375,376.6080012)
\lineto(168.550375,375.9120012)
\curveto(170.66237289,375.76800135)(170.710375,375.71999986)(170.710375,374.3760012)
\lineto(170.710375,364.8720012)
\curveto(170.710375,363.60000247)(170.66237488,362.68800041)(170.542375,361.8960012)
\curveto(170.39837514,360.88800221)(169.82237375,360.79200111)(168.574375,360.6960012)
\lineto(168.574375,360.0000012)
\lineto(174.262375,360.0000012)
\lineto(174.262375,360.6960012)
\curveto(172.70237656,360.81600108)(172.12637486,360.88800221)(171.982375,361.8960012)
\curveto(171.86237512,362.68800041)(171.814375,363.60000247)(171.814375,364.8720012)
\lineto(171.814375,374.4480012)
\lineto(171.862375,374.4480012)
\curveto(175.29437157,369.62400603)(178.70237838,364.82399638)(182.086375,360.0000012)
\lineto(183.670375,360.0000012)
\lineto(183.670375,371.7360012)
\curveto(183.670375,373.00799993)(183.71837512,373.92000199)(183.838375,374.7120012)
\curveto(183.98237486,375.72000019)(184.55837625,375.8160013)(185.806375,375.9120012)
\lineto(185.806375,376.6080012)
\lineto(180.118375,376.6080012)
\lineto(180.118375,375.9120012)
\curveto(181.67837344,375.79200132)(182.25437514,375.72000019)(182.398375,374.7120012)
\curveto(182.51837488,373.92000199)(182.566375,373.00799993)(182.566375,371.7360012)
\lineto(182.566375,363.3120012)
\lineto(182.518375,363.3600012)
\curveto(179.37437814,367.77599679)(176.2543719,372.19200562)(173.158375,376.6080012)
\lineto(168.550375,376.6080012)
}
}
{
\newrgbcolor{curcolor}{0 0 0}
\pscustom[linestyle=none,fillstyle=solid,fillcolor=curcolor]
{
\newpath
\moveto(193.3495,372.0480012)
\curveto(189.50950384,372.0480012)(187.6135,369.76799731)(187.6135,365.8800012)
\curveto(187.6135,361.99200509)(189.50950384,359.7120012)(193.3495,359.7120012)
\curveto(197.23749611,359.7120012)(199.1095,361.99200509)(199.1095,365.8800012)
\curveto(199.1095,369.76799731)(197.23749611,372.0480012)(193.3495,372.0480012)
\moveto(190.1095,365.8800012)
\curveto(190.1095,369.19199789)(191.16550218,371.2080012)(193.3495,371.2080012)
\curveto(195.58149777,371.2080012)(196.6135,369.19199789)(196.6135,365.8800012)
\curveto(196.6135,362.56800451)(195.58149777,360.5520012)(193.3495,360.5520012)
\curveto(191.16550218,360.5520012)(190.1095,362.56800451)(190.1095,365.8800012)
}
}
{
\newrgbcolor{curcolor}{0 0 0}
\pscustom[linestyle=none,fillstyle=solid,fillcolor=curcolor]
{
\newpath
\moveto(201.153625,371.4480012)
\lineto(201.153625,370.8480012)
\lineto(202.065625,370.7280012)
\curveto(202.68962438,370.6320013)(202.833625,370.51200034)(202.833625,369.6480012)
\lineto(202.833625,362.0880012)
\curveto(202.833625,360.86400243)(202.76162327,360.81600108)(201.033625,360.6960012)
\lineto(201.033625,360.0000012)
\lineto(206.889625,360.0000012)
\lineto(206.889625,360.6960012)
\curveto(205.1856267,360.81600108)(205.089625,360.86400243)(205.089625,362.0880012)
\lineto(205.089625,367.7040012)
\curveto(205.089625,368.3040006)(205.11362517,368.64000154)(205.281625,368.9760012)
\curveto(205.73762454,369.93600024)(206.79362632,370.7520012)(208.113625,370.7520012)
\curveto(210.17762294,370.7520012)(210.417625,369.23999971)(210.417625,367.7520012)
\lineto(210.417625,362.0880012)
\curveto(210.417625,360.86400243)(210.34562327,360.81600108)(208.617625,360.6960012)
\lineto(208.617625,360.0000012)
\lineto(214.473625,360.0000012)
\lineto(214.473625,360.6960012)
\curveto(212.7696267,360.81600108)(212.673625,360.86400243)(212.673625,362.0880012)
\lineto(212.673625,367.7040012)
\curveto(212.673625,368.3040006)(212.69762514,368.64000159)(212.841625,369.0240012)
\curveto(213.20162464,369.98400024)(214.23362634,370.7520012)(215.577625,370.7520012)
\curveto(217.08962349,370.80000115)(218.001625,369.83999911)(218.001625,367.7520012)
\lineto(218.001625,362.0880012)
\curveto(218.001625,360.86400243)(217.92962327,360.81600108)(216.201625,360.6960012)
\lineto(216.201625,360.0000012)
\lineto(222.057625,360.0000012)
\lineto(222.057625,360.6960012)
\curveto(220.3536267,360.81600108)(220.257625,360.86400243)(220.257625,362.0880012)
\lineto(220.257625,368.2320012)
\curveto(220.257625,370.65599878)(218.96162272,372.0480012)(216.681625,372.0480012)
\curveto(214.71362697,372.0480012)(213.29762416,370.72800067)(212.457625,370.2000012)
\curveto(211.88162558,371.32800007)(210.96962332,372.0480012)(209.289625,372.0480012)
\curveto(207.44162685,372.0480012)(205.92962416,370.84800058)(205.089625,370.2240012)
\lineto(204.873625,372.0480012)
\lineto(201.153625,371.4480012)
}
}
{
\newrgbcolor{curcolor}{0 0 0}
\pscustom[linestyle=none,fillstyle=solid,fillcolor=curcolor]
{
}
}
{
\newrgbcolor{curcolor}{0 0 0}
\pscustom[linestyle=none,fillstyle=solid,fillcolor=curcolor]
{
\newpath
\moveto(143.992,330.0000012)
\lineto(143.992,330.6960012)
\curveto(142.2880017,330.81600108)(142.192,330.86400243)(142.192,332.0880012)
\lineto(142.192,347.6640012)
\lineto(142.024,347.8080012)
\lineto(138.256,347.2080012)
\lineto(138.256,346.6080012)
\lineto(139.168,346.5120012)
\curveto(139.79199938,346.44000127)(139.936,346.32000029)(139.936,345.4080012)
\lineto(139.936,341.4720012)
\curveto(139.26400067,341.83200084)(138.3759989,342.0480012)(137.272,342.0480012)
\curveto(135.30400197,342.0480012)(133.83999899,341.42400017)(132.832,340.3920012)
\curveto(131.80000103,339.31200228)(131.176,337.72799911)(131.176,335.6400012)
\curveto(131.176,332.08800475)(133.02400317,329.7120012)(136.192,329.7120012)
\curveto(137.4879987,329.7120012)(138.6160013,330.28800226)(139.912,331.3440012)
\lineto(139.912,330.1680012)
\lineto(140.104,330.0000012)
\lineto(143.992,330.0000012)
\moveto(139.936,334.0800012)
\curveto(139.936,333.55200173)(139.91199986,333.14400079)(139.768,332.7360012)
\curveto(139.28800048,331.48800245)(138.23199875,331.0080012)(136.984,331.0080012)
\curveto(134.77600221,331.0080012)(133.672,333.16800396)(133.672,335.9280012)
\curveto(133.672,339.04799808)(134.80000245,341.2080012)(137.248,341.2080012)
\curveto(138.37599887,341.2080012)(139.28800043,340.68000022)(139.72,339.6960012)
\curveto(139.91199981,339.28800161)(139.936,338.95200055)(139.936,338.3040012)
\lineto(139.936,334.0800012)
}
}
{
\newrgbcolor{curcolor}{0 0 0}
\pscustom[linestyle=none,fillstyle=solid,fillcolor=curcolor]
{
\newpath
\moveto(147.91975,341.1360012)
\lineto(148.75975,346.5360012)
\curveto(148.87974988,347.40000034)(148.56774911,347.8080012)(147.67975,347.8080012)
\curveto(146.79175089,347.8080012)(146.47975012,347.40000034)(146.59975,346.5360012)
\lineto(147.43975,341.1360012)
\lineto(147.91975,341.1360012)
}
}
{
\newrgbcolor{curcolor}{0 0 0}
\pscustom[linestyle=none,fillstyle=solid,fillcolor=curcolor]
{
\newpath
\moveto(165.184,330.0000012)
\lineto(165.184,330.6960012)
\curveto(163.52800166,330.81600108)(163.384,330.86400243)(163.384,332.0880012)
\lineto(163.384,341.9040012)
\lineto(163.216,342.0480012)
\lineto(159.448,341.4480012)
\lineto(159.448,340.8480012)
\lineto(160.36,340.7280012)
\curveto(160.98399938,340.6320013)(161.128,340.51200034)(161.128,339.6480012)
\lineto(161.128,334.2480012)
\curveto(161.128,333.57600187)(161.05599988,333.04800096)(160.936,332.8080012)
\curveto(160.38400055,331.70400231)(159.23199875,331.0080012)(157.984,331.0080012)
\curveto(156.59200139,331.0080012)(155.56,331.87200307)(155.56,333.7440012)
\lineto(155.56,341.9040012)
\lineto(155.392,342.0480012)
\lineto(151.624,341.4480012)
\lineto(151.624,340.8480012)
\lineto(152.536,340.7280012)
\curveto(153.15999938,340.6320013)(153.304,340.51200034)(153.304,339.6480012)
\lineto(153.304,333.3120012)
\curveto(153.304,330.55200396)(154.93600178,329.7120012)(156.712,329.7120012)
\curveto(158.75199796,329.7120012)(160.4080007,331.10400154)(161.104,331.4400012)
\lineto(161.344,330.0000012)
\lineto(165.184,330.0000012)
}
}
{
\newrgbcolor{curcolor}{0 0 0}
\pscustom[linestyle=none,fillstyle=solid,fillcolor=curcolor]
{
\newpath
\moveto(173.35525,340.8000012)
\lineto(173.35525,341.7600012)
\lineto(170.25925,341.7600012)
\lineto(170.25925,344.7840012)
\lineto(169.56325,344.7840012)
\lineto(168.09925,341.7600012)
\lineto(166.27525,341.7600012)
\lineto(166.27525,340.8000012)
\lineto(168.00325,340.8000012)
\lineto(168.00325,332.4480012)
\curveto(168.00325,330.00000365)(169.82725084,329.7120012)(170.66725,329.7120012)
\curveto(171.89124878,329.7120012)(173.0432507,330.33600163)(173.73925,330.7680012)
\lineto(173.54725,331.2720012)
\curveto(172.97125058,331.03200144)(172.4432494,330.9600012)(171.84325,330.9600012)
\curveto(171.02725082,330.9600012)(170.25925,331.53600303)(170.25925,333.3600012)
\lineto(170.25925,340.8000012)
\lineto(173.35525,340.8000012)
}
}
{
\newrgbcolor{curcolor}{0 0 0}
\pscustom[linestyle=none,fillstyle=solid,fillcolor=curcolor]
{
\newpath
\moveto(174.66925,341.4480012)
\lineto(174.66925,340.8480012)
\lineto(175.58125,340.7280012)
\curveto(176.20524938,340.6320013)(176.34925,340.51200034)(176.34925,339.6480012)
\lineto(176.34925,332.0880012)
\curveto(176.34925,330.86400243)(176.27724827,330.81600108)(174.54925,330.6960012)
\lineto(174.54925,330.0000012)
\lineto(180.40525,330.0000012)
\lineto(180.40525,330.6960012)
\curveto(178.7012517,330.81600108)(178.60525,330.86400243)(178.60525,332.0880012)
\lineto(178.60525,341.9040012)
\lineto(178.43725,342.0480012)
\lineto(174.66925,341.4480012)
\moveto(177.45325,347.1600012)
\curveto(176.58925086,347.1600012)(175.98925,346.53600034)(175.98925,345.6720012)
\curveto(175.98925,344.83200204)(176.58925086,344.2320012)(177.45325,344.2320012)
\curveto(178.34124911,344.2320012)(178.89325002,344.83200204)(178.91725,345.6720012)
\curveto(178.91725,346.53600034)(178.34124911,347.1600012)(177.45325,347.1600012)
}
}
{
\newrgbcolor{curcolor}{0 0 0}
\pscustom[linestyle=none,fillstyle=solid,fillcolor=curcolor]
{
\newpath
\moveto(185.301625,347.6640012)
\lineto(185.133625,347.8080012)
\lineto(181.365625,347.2080012)
\lineto(181.365625,346.6080012)
\lineto(182.277625,346.5120012)
\curveto(182.90162438,346.44000127)(183.045625,346.32000029)(183.045625,345.4080012)
\lineto(183.045625,332.0880012)
\curveto(183.045625,330.86400243)(182.97362327,330.81600108)(181.245625,330.6960012)
\lineto(181.245625,330.0000012)
\lineto(187.101625,330.0000012)
\lineto(187.101625,330.6960012)
\curveto(185.3976267,330.81600108)(185.301625,330.86400243)(185.301625,332.0880012)
\lineto(185.301625,347.6640012)
}
}
{
\newrgbcolor{curcolor}{0 0 0}
\pscustom[linestyle=none,fillstyle=solid,fillcolor=curcolor]
{
\newpath
\moveto(188.35675,341.4480012)
\lineto(188.35675,340.8480012)
\lineto(189.26875,340.7280012)
\curveto(189.89274938,340.6320013)(190.03675,340.51200034)(190.03675,339.6480012)
\lineto(190.03675,332.0880012)
\curveto(190.03675,330.86400243)(189.96474827,330.81600108)(188.23675,330.6960012)
\lineto(188.23675,330.0000012)
\lineto(194.09275,330.0000012)
\lineto(194.09275,330.6960012)
\curveto(192.3887517,330.81600108)(192.29275,330.86400243)(192.29275,332.0880012)
\lineto(192.29275,341.9040012)
\lineto(192.12475,342.0480012)
\lineto(188.35675,341.4480012)
\moveto(191.14075,347.1600012)
\curveto(190.27675086,347.1600012)(189.67675,346.53600034)(189.67675,345.6720012)
\curveto(189.67675,344.83200204)(190.27675086,344.2320012)(191.14075,344.2320012)
\curveto(192.02874911,344.2320012)(192.58075002,344.83200204)(192.60475,345.6720012)
\curveto(192.60475,346.53600034)(192.02874911,347.1600012)(191.14075,347.1600012)
}
}
{
\newrgbcolor{curcolor}{0 0 0}
\pscustom[linestyle=none,fillstyle=solid,fillcolor=curcolor]
{
\newpath
\moveto(203.261125,338.4240012)
\lineto(203.261125,341.2320012)
\curveto(202.46912579,341.80800063)(201.24512392,342.0480012)(200.165125,342.0480012)
\curveto(197.57312759,342.0480012)(195.82112498,340.82399892)(195.797125,338.5440012)
\curveto(195.82112498,336.55200319)(197.42912699,335.6640006)(199.421125,335.0640012)
\curveto(200.50112392,334.72800154)(202.013125,334.17599957)(202.013125,332.5440012)
\curveto(202.013125,331.32000243)(201.05312373,330.5520012)(199.781125,330.5520012)
\curveto(197.83712694,330.5520012)(196.75712452,331.96800324)(196.277125,334.0080012)
\lineto(195.581125,334.0080012)
\lineto(195.821125,330.8640012)
\curveto(196.68512414,330.09600197)(198.14912642,329.7120012)(199.565125,329.7120012)
\curveto(202.42112214,329.7120012)(204.053125,331.22400324)(204.053125,333.2640012)
\curveto(204.053125,335.37599909)(202.75712253,336.33600197)(200.285125,337.1040012)
\curveto(199.30112598,337.41600089)(197.717125,337.92000255)(197.717125,339.2640012)
\curveto(197.74112498,340.51199995)(198.6771262,341.2080012)(199.877125,341.2080012)
\curveto(201.5811233,341.2080012)(202.39712517,339.88799974)(202.565125,338.4240012)
\lineto(203.261125,338.4240012)
}
}
{
\newrgbcolor{curcolor}{0 0 0}
\pscustom[linestyle=none,fillstyle=solid,fillcolor=curcolor]
{
\newpath
\moveto(213.13825,333.4320012)
\curveto(213.13825,331.48800315)(211.65024906,330.9840012)(210.71425,330.9840012)
\curveto(209.22625149,330.9840012)(208.50625,332.04000262)(208.50625,333.4560012)
\curveto(208.50625,334.58400007)(209.03425134,335.16000171)(210.37825,335.6640012)
\curveto(211.33824904,336.02400084)(212.61025053,336.48000154)(213.13825,336.8160012)
\lineto(213.13825,333.4320012)
\moveto(215.29825,338.8560012)
\curveto(215.29825,340.24799981)(214.98624638,342.0480012)(211.36225,342.0480012)
\curveto(208.65025271,342.0480012)(206.51425,340.63199988)(206.51425,339.3120012)
\curveto(206.51425,338.54400197)(207.40225046,338.1840012)(207.85825,338.1840012)
\curveto(208.3622495,338.1840012)(208.50625012,338.44800161)(208.62625,338.8560012)
\curveto(209.15424947,340.63199943)(210.13825096,341.2080012)(211.09825,341.2080012)
\curveto(212.03424906,341.2080012)(213.13825,340.72799928)(213.13825,338.8080012)
\lineto(213.13825,337.8000012)
\curveto(212.5382506,337.17600183)(210.2102481,336.5760006)(208.31425,335.9760012)
\curveto(206.58625173,335.44800173)(206.08225,334.24800007)(206.08225,333.1200012)
\curveto(206.08225,331.320003)(207.28225226,329.7120012)(209.53825,329.7120012)
\curveto(211.02624851,329.76000115)(212.34625077,330.64800171)(213.11425,331.1520012)
\curveto(213.45024966,330.26400209)(213.83425084,329.7120012)(214.67425,329.7120012)
\curveto(215.56224911,329.7120012)(216.59425091,329.97600166)(217.50625,330.4320012)
\lineto(217.36225,331.0080012)
\curveto(217.02625034,330.93600127)(216.49824964,330.8880013)(216.13825,330.9840012)
\curveto(215.70625043,331.08000111)(215.29825,331.53600271)(215.29825,333.0480012)
\lineto(215.29825,338.8560012)
}
}
{
\newrgbcolor{curcolor}{0 0 0}
\pscustom[linestyle=none,fillstyle=solid,fillcolor=curcolor]
{
\newpath
\moveto(225.29275,340.8000012)
\lineto(225.29275,341.7600012)
\lineto(222.19675,341.7600012)
\lineto(222.19675,344.7840012)
\lineto(221.50075,344.7840012)
\lineto(220.03675,341.7600012)
\lineto(218.21275,341.7600012)
\lineto(218.21275,340.8000012)
\lineto(219.94075,340.8000012)
\lineto(219.94075,332.4480012)
\curveto(219.94075,330.00000365)(221.76475084,329.7120012)(222.60475,329.7120012)
\curveto(223.82874878,329.7120012)(224.9807507,330.33600163)(225.67675,330.7680012)
\lineto(225.48475,331.2720012)
\curveto(224.90875058,331.03200144)(224.3807494,330.9600012)(223.78075,330.9600012)
\curveto(222.96475082,330.9600012)(222.19675,331.53600303)(222.19675,333.3600012)
\lineto(222.19675,340.8000012)
\lineto(225.29275,340.8000012)
}
}
{
\newrgbcolor{curcolor}{0 0 0}
\pscustom[linestyle=none,fillstyle=solid,fillcolor=curcolor]
{
\newpath
\moveto(236.30275,336.3120012)
\curveto(236.8067495,336.3120012)(237.26275,336.43200199)(237.26275,337.2240012)
\curveto(237.26275,338.63999979)(236.80674561,342.0480012)(232.41475,342.0480012)
\curveto(228.67075374,342.0480012)(226.89475,339.3839976)(226.89475,335.7840012)
\curveto(226.89475,331.99200499)(228.52675391,329.66400125)(232.43875,329.7120012)
\curveto(235.10274734,329.73600118)(236.47075065,331.2000031)(237.11875,333.0960012)
\lineto(236.39875,333.4800012)
\curveto(235.72675067,332.08800259)(234.8147481,330.9360012)(232.91875,330.9360012)
\curveto(229.94275298,330.9360012)(229.34275005,333.8160037)(229.39075,336.3120012)
\lineto(236.30275,336.3120012)
\moveto(229.41475,337.2240012)
\curveto(229.41475,338.20800022)(229.7747525,341.2080012)(232.27075,341.2080012)
\curveto(234.50274777,341.2080012)(234.76675,338.92800027)(234.76675,337.9920012)
\curveto(234.76675,337.53600166)(234.62274933,337.2240012)(233.95075,337.2240012)
\lineto(229.41475,337.2240012)
}
}
{
\newrgbcolor{curcolor}{0 0 0}
\pscustom[linestyle=none,fillstyle=solid,fillcolor=curcolor]
{
\newpath
\moveto(252.230875,330.0000012)
\lineto(252.230875,330.6960012)
\curveto(250.57487666,330.81600108)(250.430875,330.86400243)(250.430875,332.0880012)
\lineto(250.430875,341.9040012)
\lineto(250.262875,342.0480012)
\lineto(246.494875,341.4480012)
\lineto(246.494875,340.8480012)
\lineto(247.406875,340.7280012)
\curveto(248.03087438,340.6320013)(248.174875,340.51200034)(248.174875,339.6480012)
\lineto(248.174875,334.2480012)
\curveto(248.174875,333.57600187)(248.10287488,333.04800096)(247.982875,332.8080012)
\curveto(247.43087555,331.70400231)(246.27887375,331.0080012)(245.030875,331.0080012)
\curveto(243.63887639,331.0080012)(242.606875,331.87200307)(242.606875,333.7440012)
\lineto(242.606875,341.9040012)
\lineto(242.438875,342.0480012)
\lineto(238.670875,341.4480012)
\lineto(238.670875,340.8480012)
\lineto(239.582875,340.7280012)
\curveto(240.20687438,340.6320013)(240.350875,340.51200034)(240.350875,339.6480012)
\lineto(240.350875,333.3120012)
\curveto(240.350875,330.55200396)(241.98287678,329.7120012)(243.758875,329.7120012)
\curveto(245.79887296,329.7120012)(247.4548757,331.10400154)(248.150875,331.4400012)
\lineto(248.390875,330.0000012)
\lineto(252.230875,330.0000012)
}
}
{
\newrgbcolor{curcolor}{0 0 0}
\pscustom[linestyle=none,fillstyle=solid,fillcolor=curcolor]
{
\newpath
\moveto(257.114125,342.0480012)
\lineto(253.466125,341.4480012)
\lineto(253.466125,340.8480012)
\lineto(254.378125,340.7280012)
\curveto(255.00212438,340.6320013)(255.146125,340.51200034)(255.146125,339.6480012)
\lineto(255.146125,332.0880012)
\curveto(255.146125,330.86400243)(255.02612332,330.81600108)(253.346125,330.6960012)
\lineto(253.346125,330.0000012)
\lineto(259.682125,330.0000012)
\lineto(259.682125,330.6960012)
\curveto(257.54612714,330.81600108)(257.402125,330.86400243)(257.402125,332.0880012)
\lineto(257.402125,337.7040012)
\curveto(257.402125,339.55199935)(258.21812555,340.1760012)(258.770125,340.1760012)
\curveto(259.15412462,340.1760012)(259.5621257,340.03200082)(260.258125,339.6480012)
\curveto(260.42612483,339.5520013)(260.61812512,339.5280012)(260.738125,339.5280012)
\curveto(261.31412442,339.5280012)(261.842125,340.12800195)(261.842125,340.8720012)
\curveto(261.842125,341.40000067)(261.50612406,342.0480012)(260.570125,342.0480012)
\curveto(259.70612586,342.0480012)(258.98612342,341.51999981)(257.402125,340.1280012)
\lineto(257.114125,342.0480012)
}
}
{
\newrgbcolor{curcolor}{0 0 0}
\pscustom[linestyle=none,fillstyle=solid,fillcolor=curcolor]
{
\newpath
\moveto(136.504,280.0000012)
\lineto(136.504,280.6960012)
\curveto(134.94400156,280.81600108)(134.36799986,280.88800221)(134.224,281.8960012)
\curveto(134.10400012,282.68800041)(134.056,283.60000247)(134.056,284.8720012)
\lineto(134.056,294.4000012)
\lineto(134.104,294.4000012)
\lineto(140.296,280.2880012)
\lineto(140.944,280.2880012)
\lineto(147.208,294.4000012)
\lineto(147.28,294.4000012)
\lineto(147.28,282.2320012)
\curveto(147.28,280.88800255)(147.23199789,280.84000106)(145.12,280.6960012)
\lineto(145.12,280.0000012)
\lineto(151.816,280.0000012)
\lineto(151.816,280.6960012)
\curveto(149.72800209,280.84000106)(149.656,280.88800255)(149.656,282.2320012)
\lineto(149.656,294.3760012)
\curveto(149.656,295.71999986)(149.72800209,295.76800135)(151.816,295.9120012)
\lineto(151.816,296.6080012)
\lineto(146.992,296.6080012)
\curveto(146.17600082,294.47200334)(145.19199904,292.35999909)(144.232,290.2480012)
\lineto(141.352,283.8160012)
\lineto(141.304,283.8160012)
\lineto(138.472,290.2240012)
\curveto(137.53600094,292.35999907)(136.52799914,294.47200334)(135.664,296.6080012)
\lineto(130.792,296.6080012)
\lineto(130.792,295.9120012)
\curveto(132.90399789,295.76800135)(132.952,295.71999986)(132.952,294.3760012)
\lineto(132.952,284.8720012)
\curveto(132.952,283.60000247)(132.90399988,282.68800041)(132.784,281.8960012)
\curveto(132.64000014,280.88800221)(132.06399875,280.79200111)(130.816,280.6960012)
\lineto(130.816,280.0000012)
\lineto(136.504,280.0000012)
}
}
{
\newrgbcolor{curcolor}{0 0 0}
\pscustom[linestyle=none,fillstyle=solid,fillcolor=curcolor]
{
\newpath
\moveto(159.552625,292.0480012)
\curveto(155.71262884,292.0480012)(153.816625,289.76799731)(153.816625,285.8800012)
\curveto(153.816625,281.99200509)(155.71262884,279.7120012)(159.552625,279.7120012)
\curveto(163.44062111,279.7120012)(165.312625,281.99200509)(165.312625,285.8800012)
\curveto(165.312625,289.76799731)(163.44062111,292.0480012)(159.552625,292.0480012)
\moveto(156.312625,285.8800012)
\curveto(156.312625,289.19199789)(157.36862718,291.2080012)(159.552625,291.2080012)
\curveto(161.78462277,291.2080012)(162.816625,289.19199789)(162.816625,285.8800012)
\curveto(162.816625,282.56800451)(161.78462277,280.5520012)(159.552625,280.5520012)
\curveto(157.36862718,280.5520012)(156.312625,282.56800451)(156.312625,285.8800012)
}
}
{
\newrgbcolor{curcolor}{0 0 0}
\pscustom[linestyle=none,fillstyle=solid,fillcolor=curcolor]
{
\newpath
\moveto(174.29275,290.8000012)
\lineto(174.29275,291.7600012)
\lineto(171.19675,291.7600012)
\lineto(171.19675,294.7840012)
\lineto(170.50075,294.7840012)
\lineto(169.03675,291.7600012)
\lineto(167.21275,291.7600012)
\lineto(167.21275,290.8000012)
\lineto(168.94075,290.8000012)
\lineto(168.94075,282.4480012)
\curveto(168.94075,280.00000365)(170.76475084,279.7120012)(171.60475,279.7120012)
\curveto(172.82874878,279.7120012)(173.9807507,280.33600163)(174.67675,280.7680012)
\lineto(174.48475,281.2720012)
\curveto(173.90875058,281.03200144)(173.3807494,280.9600012)(172.78075,280.9600012)
\curveto(171.96475082,280.9600012)(171.19675,281.53600303)(171.19675,283.3600012)
\lineto(171.19675,290.8000012)
\lineto(174.29275,290.8000012)
}
}
{
\newrgbcolor{curcolor}{0 0 0}
\pscustom[linestyle=none,fillstyle=solid,fillcolor=curcolor]
{
}
}
{
\newrgbcolor{curcolor}{0 0 0}
\pscustom[linestyle=none,fillstyle=solid,fillcolor=curcolor]
{
\newpath
\moveto(194.101375,280.0000012)
\lineto(194.101375,280.6960012)
\curveto(192.3973767,280.81600108)(192.301375,280.86400243)(192.301375,282.0880012)
\lineto(192.301375,297.6640012)
\lineto(192.133375,297.8080012)
\lineto(188.365375,297.2080012)
\lineto(188.365375,296.6080012)
\lineto(189.277375,296.5120012)
\curveto(189.90137438,296.44000127)(190.045375,296.32000029)(190.045375,295.4080012)
\lineto(190.045375,291.4720012)
\curveto(189.37337567,291.83200084)(188.4853739,292.0480012)(187.381375,292.0480012)
\curveto(185.41337697,292.0480012)(183.94937399,291.42400017)(182.941375,290.3920012)
\curveto(181.90937603,289.31200228)(181.285375,287.72799911)(181.285375,285.6400012)
\curveto(181.285375,282.08800475)(183.13337817,279.7120012)(186.301375,279.7120012)
\curveto(187.5973737,279.7120012)(188.7253763,280.28800226)(190.021375,281.3440012)
\lineto(190.021375,280.1680012)
\lineto(190.213375,280.0000012)
\lineto(194.101375,280.0000012)
\moveto(190.045375,284.0800012)
\curveto(190.045375,283.55200173)(190.02137486,283.14400079)(189.877375,282.7360012)
\curveto(189.39737548,281.48800245)(188.34137375,281.0080012)(187.093375,281.0080012)
\curveto(184.88537721,281.0080012)(183.781375,283.16800396)(183.781375,285.9280012)
\curveto(183.781375,289.04799808)(184.90937745,291.2080012)(187.357375,291.2080012)
\curveto(188.48537387,291.2080012)(189.39737543,290.68000022)(189.829375,289.6960012)
\curveto(190.02137481,289.28800161)(190.045375,288.95200055)(190.045375,288.3040012)
\lineto(190.045375,284.0800012)
}
}
{
\newrgbcolor{curcolor}{0 0 0}
\pscustom[linestyle=none,fillstyle=solid,fillcolor=curcolor]
{
\newpath
\moveto(205.037125,286.3120012)
\curveto(205.5411245,286.3120012)(205.997125,286.43200199)(205.997125,287.2240012)
\curveto(205.997125,288.63999979)(205.54112061,292.0480012)(201.149125,292.0480012)
\curveto(197.40512874,292.0480012)(195.629125,289.3839976)(195.629125,285.7840012)
\curveto(195.629125,281.99200499)(197.26112891,279.66400125)(201.173125,279.7120012)
\curveto(203.83712234,279.73600118)(205.20512565,281.2000031)(205.853125,283.0960012)
\lineto(205.133125,283.4800012)
\curveto(204.46112567,282.08800259)(203.5491231,280.9360012)(201.653125,280.9360012)
\curveto(198.67712798,280.9360012)(198.07712505,283.8160037)(198.125125,286.3120012)
\lineto(205.037125,286.3120012)
\moveto(198.149125,287.2240012)
\curveto(198.149125,288.20800022)(198.5091275,291.2080012)(201.005125,291.2080012)
\curveto(203.23712277,291.2080012)(203.501125,288.92800027)(203.501125,287.9920012)
\curveto(203.501125,287.53600166)(203.35712433,287.2240012)(202.685125,287.2240012)
\lineto(198.149125,287.2240012)
}
}
{
\newrgbcolor{curcolor}{0 0 0}
\pscustom[linestyle=none,fillstyle=solid,fillcolor=curcolor]
{
}
}
{
\newrgbcolor{curcolor}{0 0 0}
\pscustom[linestyle=none,fillstyle=solid,fillcolor=curcolor]
{
\newpath
\moveto(216.827875,287.7040012)
\curveto(216.827875,288.3040006)(216.92387522,288.64000163)(217.139875,289.0720012)
\curveto(217.69187445,290.1760001)(218.72387627,290.7520012)(219.995875,290.7520012)
\curveto(220.95587404,290.7520012)(223.091875,290.12799707)(223.091875,286.0000012)
\curveto(223.091875,282.47200473)(221.9878726,280.5520012)(219.587875,280.5520012)
\curveto(218.33987625,280.5520012)(217.37987462,281.20000231)(216.995875,282.3040012)
\curveto(216.85187514,282.73600077)(216.827875,283.21600175)(216.827875,283.7680012)
\lineto(216.827875,287.7040012)
\moveto(212.891875,291.4480012)
\lineto(212.891875,290.8480012)
\lineto(213.803875,290.7280012)
\curveto(214.42787438,290.6320013)(214.571875,290.51200034)(214.571875,289.6480012)
\lineto(214.571875,276.5680012)
\curveto(214.571875,275.34400243)(214.45187332,275.29600108)(212.771875,275.1760012)
\lineto(212.771875,274.4800012)
\lineto(218.987875,274.4800012)
\lineto(218.987875,275.1760012)
\curveto(216.97187702,275.29600108)(216.827875,275.34400243)(216.827875,276.5680012)
\lineto(216.827875,280.2640012)
\curveto(217.3318745,279.97600149)(218.4598761,279.7120012)(219.563875,279.7120012)
\curveto(222.80387176,279.7120012)(225.587875,281.20000615)(225.587875,286.1440012)
\curveto(225.587875,287.8479995)(225.27587063,292.0480012)(220.907875,292.0480012)
\curveto(219.15587675,292.0480012)(217.76387406,290.8720006)(216.827875,290.2720012)
\lineto(216.515875,292.0480012)
\lineto(212.891875,291.4480012)
}
}
{
\newrgbcolor{curcolor}{0 0 0}
\pscustom[linestyle=none,fillstyle=solid,fillcolor=curcolor]
{
\newpath
\moveto(234.982,283.4320012)
\curveto(234.982,281.48800315)(233.49399906,280.9840012)(232.558,280.9840012)
\curveto(231.07000149,280.9840012)(230.35,282.04000262)(230.35,283.4560012)
\curveto(230.35,284.58400007)(230.87800134,285.16000171)(232.222,285.6640012)
\curveto(233.18199904,286.02400084)(234.45400053,286.48000154)(234.982,286.8160012)
\lineto(234.982,283.4320012)
\moveto(237.142,288.8560012)
\curveto(237.142,290.24799981)(236.82999638,292.0480012)(233.206,292.0480012)
\curveto(230.49400271,292.0480012)(228.358,290.63199988)(228.358,289.3120012)
\curveto(228.358,288.54400197)(229.24600046,288.1840012)(229.702,288.1840012)
\curveto(230.2059995,288.1840012)(230.35000012,288.44800161)(230.47,288.8560012)
\curveto(230.99799947,290.63199943)(231.98200096,291.2080012)(232.942,291.2080012)
\curveto(233.87799906,291.2080012)(234.982,290.72799928)(234.982,288.8080012)
\lineto(234.982,287.8000012)
\curveto(234.3820006,287.17600183)(232.0539981,286.5760006)(230.158,285.9760012)
\curveto(228.43000173,285.44800173)(227.926,284.24800007)(227.926,283.1200012)
\curveto(227.926,281.320003)(229.12600226,279.7120012)(231.382,279.7120012)
\curveto(232.86999851,279.76000115)(234.19000077,280.64800171)(234.958,281.1520012)
\curveto(235.29399966,280.26400209)(235.67800084,279.7120012)(236.518,279.7120012)
\curveto(237.40599911,279.7120012)(238.43800091,279.97600166)(239.35,280.4320012)
\lineto(239.206,281.0080012)
\curveto(238.87000034,280.93600127)(238.34199964,280.8880013)(237.982,280.9840012)
\curveto(237.55000043,281.08000111)(237.142,281.53600271)(237.142,283.0480012)
\lineto(237.142,288.8560012)
}
}
{
\newrgbcolor{curcolor}{0 0 0}
\pscustom[linestyle=none,fillstyle=solid,fillcolor=curcolor]
{
\newpath
\moveto(248.1205,288.4240012)
\lineto(248.1205,291.2320012)
\curveto(247.32850079,291.80800063)(246.10449892,292.0480012)(245.0245,292.0480012)
\curveto(242.43250259,292.0480012)(240.68049998,290.82399892)(240.6565,288.5440012)
\curveto(240.68049998,286.55200319)(242.28850199,285.6640006)(244.2805,285.0640012)
\curveto(245.36049892,284.72800154)(246.8725,284.17599957)(246.8725,282.5440012)
\curveto(246.8725,281.32000243)(245.91249873,280.5520012)(244.6405,280.5520012)
\curveto(242.69650194,280.5520012)(241.61649952,281.96800324)(241.1365,284.0080012)
\lineto(240.4405,284.0080012)
\lineto(240.6805,280.8640012)
\curveto(241.54449914,280.09600197)(243.00850142,279.7120012)(244.4245,279.7120012)
\curveto(247.28049714,279.7120012)(248.9125,281.22400324)(248.9125,283.2640012)
\curveto(248.9125,285.37599909)(247.61649753,286.33600197)(245.1445,287.1040012)
\curveto(244.16050098,287.41600089)(242.5765,287.92000255)(242.5765,289.2640012)
\curveto(242.60049998,290.51199995)(243.5365012,291.2080012)(244.7365,291.2080012)
\curveto(246.4404983,291.2080012)(247.25650017,289.88799974)(247.4245,288.4240012)
\lineto(248.1205,288.4240012)
}
}
{
\newrgbcolor{curcolor}{0 0 0}
\pscustom[linestyle=none,fillstyle=solid,fillcolor=curcolor]
{
\newpath
\moveto(258.573625,288.4240012)
\lineto(258.573625,291.2320012)
\curveto(257.78162579,291.80800063)(256.55762392,292.0480012)(255.477625,292.0480012)
\curveto(252.88562759,292.0480012)(251.13362498,290.82399892)(251.109625,288.5440012)
\curveto(251.13362498,286.55200319)(252.74162699,285.6640006)(254.733625,285.0640012)
\curveto(255.81362392,284.72800154)(257.325625,284.17599957)(257.325625,282.5440012)
\curveto(257.325625,281.32000243)(256.36562373,280.5520012)(255.093625,280.5520012)
\curveto(253.14962694,280.5520012)(252.06962452,281.96800324)(251.589625,284.0080012)
\lineto(250.893625,284.0080012)
\lineto(251.133625,280.8640012)
\curveto(251.99762414,280.09600197)(253.46162642,279.7120012)(254.877625,279.7120012)
\curveto(257.73362214,279.7120012)(259.365625,281.22400324)(259.365625,283.2640012)
\curveto(259.365625,285.37599909)(258.06962253,286.33600197)(255.597625,287.1040012)
\curveto(254.61362598,287.41600089)(253.029625,287.92000255)(253.029625,289.2640012)
\curveto(253.05362498,290.51199995)(253.9896262,291.2080012)(255.189625,291.2080012)
\curveto(256.8936233,291.2080012)(257.70962517,289.88799974)(257.877625,288.4240012)
\lineto(258.573625,288.4240012)
}
}
{
\newrgbcolor{curcolor}{0 0 0}
\pscustom[linestyle=none,fillstyle=solid,fillcolor=curcolor]
{
\newpath
\moveto(270.80275,286.3120012)
\curveto(271.3067495,286.3120012)(271.76275,286.43200199)(271.76275,287.2240012)
\curveto(271.76275,288.63999979)(271.30674561,292.0480012)(266.91475,292.0480012)
\curveto(263.17075374,292.0480012)(261.39475,289.3839976)(261.39475,285.7840012)
\curveto(261.39475,281.99200499)(263.02675391,279.66400125)(266.93875,279.7120012)
\curveto(269.60274734,279.73600118)(270.97075065,281.2000031)(271.61875,283.0960012)
\lineto(270.89875,283.4800012)
\curveto(270.22675067,282.08800259)(269.3147481,280.9360012)(267.41875,280.9360012)
\curveto(264.44275298,280.9360012)(263.84275005,283.8160037)(263.89075,286.3120012)
\lineto(270.80275,286.3120012)
\moveto(263.91475,287.2240012)
\curveto(263.91475,288.20800022)(264.2747525,291.2080012)(266.77075,291.2080012)
\curveto(269.00274777,291.2080012)(269.26675,288.92800027)(269.26675,287.9920012)
\curveto(269.26675,287.53600166)(269.12274933,287.2240012)(268.45075,287.2240012)
\lineto(263.91475,287.2240012)
}
}
{
\newrgbcolor{curcolor}{0 0 0}
\pscustom[linestyle=none,fillstyle=solid,fillcolor=curcolor]
{
\newpath
\moveto(139.568,256.6080012)
\lineto(139.568,255.9120012)
\curveto(140.91199866,255.86400125)(141.20000048,255.76799993)(141.68,254.4960012)
\curveto(142.44799923,252.48000322)(143.33600084,250.0799988)(144.176,247.6800012)
\lineto(146.888,240.0000012)
\lineto(148.592,240.0000012)
\lineto(151.664,248.2800012)
\curveto(152.64799902,250.94399854)(153.60800058,253.39200247)(154.184,254.6640012)
\curveto(154.6879995,255.79200007)(154.9760013,255.84000127)(156.272,255.9120012)
\lineto(156.272,256.6080012)
\lineto(150.968,256.6080012)
\lineto(150.968,255.9120012)
\lineto(152.288,255.7920012)
\curveto(152.91199938,255.74400125)(152.93599988,255.50400075)(152.816,255.0480012)
\curveto(152.55200026,253.99200226)(151.80799914,251.8799988)(150.944,249.4800012)
\lineto(148.448,242.5920012)
\lineto(148.4,242.6400012)
\lineto(146.168,248.8080012)
\curveto(145.42400074,250.89599911)(144.67999938,253.00800312)(144.056,254.9280012)
\curveto(143.86400019,255.5280006)(143.93600058,255.74400125)(144.512,255.7920012)
\lineto(145.76,255.9120012)
\lineto(145.76,256.6080012)
\lineto(139.568,256.6080012)
}
}
{
\newrgbcolor{curcolor}{0 0 0}
\pscustom[linestyle=none,fillstyle=solid,fillcolor=curcolor]
{
\newpath
\moveto(165.818375,246.3120012)
\curveto(166.3223745,246.3120012)(166.778375,246.43200199)(166.778375,247.2240012)
\curveto(166.778375,248.63999979)(166.32237061,252.0480012)(161.930375,252.0480012)
\curveto(158.18637874,252.0480012)(156.410375,249.3839976)(156.410375,245.7840012)
\curveto(156.410375,241.99200499)(158.04237891,239.66400125)(161.954375,239.7120012)
\curveto(164.61837234,239.73600118)(165.98637565,241.2000031)(166.634375,243.0960012)
\lineto(165.914375,243.4800012)
\curveto(165.24237567,242.08800259)(164.3303731,240.9360012)(162.434375,240.9360012)
\curveto(159.45837798,240.9360012)(158.85837505,243.8160037)(158.906375,246.3120012)
\lineto(165.818375,246.3120012)
\moveto(158.930375,247.2240012)
\curveto(158.930375,248.20800022)(159.2903775,251.2080012)(161.786375,251.2080012)
\curveto(164.01837277,251.2080012)(164.282375,248.92800027)(164.282375,247.9920012)
\curveto(164.282375,247.53600166)(164.13837433,247.2240012)(163.466375,247.2240012)
\lineto(158.930375,247.2240012)
\moveto(160.418375,253.0080012)
\lineto(164.042375,255.0240012)
\curveto(164.76237428,255.40800082)(165.386375,255.81600171)(165.386375,256.3200012)
\curveto(165.386375,256.80000072)(164.81037464,257.3520012)(164.450375,257.3520012)
\curveto(164.04237541,257.3520012)(163.63437428,257.11200046)(162.914375,256.3680012)
\lineto(160.058375,253.4880012)
\lineto(160.418375,253.0080012)
}
}
{
\newrgbcolor{curcolor}{0 0 0}
\pscustom[linestyle=none,fillstyle=solid,fillcolor=curcolor]
{
\newpath
\moveto(172.0985,252.0480012)
\lineto(168.4505,251.4480012)
\lineto(168.4505,250.8480012)
\lineto(169.3625,250.7280012)
\curveto(169.98649938,250.6320013)(170.1305,250.51200034)(170.1305,249.6480012)
\lineto(170.1305,242.0880012)
\curveto(170.1305,240.86400243)(170.01049832,240.81600108)(168.3305,240.6960012)
\lineto(168.3305,240.0000012)
\lineto(174.6665,240.0000012)
\lineto(174.6665,240.6960012)
\curveto(172.53050214,240.81600108)(172.3865,240.86400243)(172.3865,242.0880012)
\lineto(172.3865,247.7040012)
\curveto(172.3865,249.55199935)(173.20250055,250.1760012)(173.7545,250.1760012)
\curveto(174.13849962,250.1760012)(174.5465007,250.03200082)(175.2425,249.6480012)
\curveto(175.41049983,249.5520013)(175.60250012,249.5280012)(175.7225,249.5280012)
\curveto(176.29849942,249.5280012)(176.8265,250.12800195)(176.8265,250.8720012)
\curveto(176.8265,251.40000067)(176.49049906,252.0480012)(175.5545,252.0480012)
\curveto(174.69050086,252.0480012)(173.97049842,251.51999981)(172.3865,250.1280012)
\lineto(172.0985,252.0480012)
}
}
{
\newrgbcolor{curcolor}{0 0 0}
\pscustom[linestyle=none,fillstyle=solid,fillcolor=curcolor]
{
\newpath
\moveto(177.778625,251.4480012)
\lineto(177.778625,250.8480012)
\lineto(178.690625,250.7280012)
\curveto(179.31462438,250.6320013)(179.458625,250.51200034)(179.458625,249.6480012)
\lineto(179.458625,242.0880012)
\curveto(179.458625,240.86400243)(179.38662327,240.81600108)(177.658625,240.6960012)
\lineto(177.658625,240.0000012)
\lineto(183.514625,240.0000012)
\lineto(183.514625,240.6960012)
\curveto(181.8106267,240.81600108)(181.714625,240.86400243)(181.714625,242.0880012)
\lineto(181.714625,251.9040012)
\lineto(181.546625,252.0480012)
\lineto(177.778625,251.4480012)
\moveto(180.562625,257.1600012)
\curveto(179.69862586,257.1600012)(179.098625,256.53600034)(179.098625,255.6720012)
\curveto(179.098625,254.83200204)(179.69862586,254.2320012)(180.562625,254.2320012)
\curveto(181.45062411,254.2320012)(182.00262502,254.83200204)(182.026625,255.6720012)
\curveto(182.026625,256.53600034)(181.45062411,257.1600012)(180.562625,257.1600012)
}
}
{
\newrgbcolor{curcolor}{0 0 0}
\pscustom[linestyle=none,fillstyle=solid,fillcolor=curcolor]
{
\newpath
\moveto(188.699,250.8000012)
\lineto(191.627,250.8000012)
\lineto(191.627,251.7600012)
\lineto(188.699,251.7600012)
\lineto(188.603,254.4480012)
\curveto(188.53100007,256.72799892)(189.37100038,256.9680012)(189.755,256.9680012)
\curveto(190.2589995,256.9680012)(190.7870006,256.72799981)(191.387,255.3360012)
\curveto(191.50699988,255.09600144)(191.67500036,254.9280012)(192.035,254.9280012)
\curveto(192.51499952,254.9280012)(193.211,255.33600192)(193.211,256.0560012)
\curveto(193.211,256.84800041)(192.20299808,257.8080012)(190.283,257.8080012)
\curveto(188.96300132,257.8080012)(187.8349994,257.20800029)(187.235,256.2960012)
\curveto(186.68300055,255.45600204)(186.443,254.18399962)(186.443,252.6000012)
\lineto(186.443,251.7600012)
\lineto(184.595,251.7600012)
\lineto(184.595,250.8000012)
\lineto(186.443,250.8000012)
\lineto(186.443,242.0880012)
\curveto(186.443,240.86400243)(186.37099827,240.81600108)(184.643,240.6960012)
\lineto(184.643,240.0000012)
\lineto(190.979,240.0000012)
\lineto(190.979,240.6960012)
\curveto(188.79500218,240.79200111)(188.699,240.84000243)(188.699,242.0640012)
\lineto(188.699,250.8000012)
}
}
{
\newrgbcolor{curcolor}{0 0 0}
\pscustom[linestyle=none,fillstyle=solid,fillcolor=curcolor]
{
\newpath
\moveto(192.403625,251.4480012)
\lineto(192.403625,250.8480012)
\lineto(193.315625,250.7280012)
\curveto(193.93962438,250.6320013)(194.083625,250.51200034)(194.083625,249.6480012)
\lineto(194.083625,242.0880012)
\curveto(194.083625,240.86400243)(194.01162327,240.81600108)(192.283625,240.6960012)
\lineto(192.283625,240.0000012)
\lineto(198.139625,240.0000012)
\lineto(198.139625,240.6960012)
\curveto(196.4356267,240.81600108)(196.339625,240.86400243)(196.339625,242.0880012)
\lineto(196.339625,251.9040012)
\lineto(196.171625,252.0480012)
\lineto(192.403625,251.4480012)
\moveto(195.187625,257.1600012)
\curveto(194.32362586,257.1600012)(193.723625,256.53600034)(193.723625,255.6720012)
\curveto(193.723625,254.83200204)(194.32362586,254.2320012)(195.187625,254.2320012)
\curveto(196.07562411,254.2320012)(196.62762502,254.83200204)(196.651625,255.6720012)
\curveto(196.651625,256.53600034)(196.07562411,257.1600012)(195.187625,257.1600012)
}
}
{
\newrgbcolor{curcolor}{0 0 0}
\pscustom[linestyle=none,fillstyle=solid,fillcolor=curcolor]
{
\newpath
\moveto(205.7,240.9600012)
\curveto(202.84400286,240.9600012)(202.172,243.93600346)(202.172,246.1920012)
\curveto(202.172,249.76799763)(203.58800166,251.2080012)(205.244,251.2080012)
\curveto(206.3479989,251.2080012)(207.04400048,250.41599991)(207.524,249.1200012)
\curveto(207.66799986,248.73600159)(207.81200046,248.4960012)(208.268,248.4960012)
\curveto(208.74799952,248.4960012)(209.684,248.80800211)(209.684,249.7200012)
\curveto(209.684,250.8240001)(208.09999741,252.0480012)(205.508,252.0480012)
\curveto(201.14000437,252.0480012)(199.676,248.95199801)(199.676,245.7600012)
\curveto(199.676,241.75200521)(201.6440036,239.7120012)(205.244,239.7120012)
\curveto(206.92399832,239.7120012)(209.18000067,240.60000377)(209.852,243.1680012)
\lineto(209.156,243.5040012)
\curveto(208.41200074,241.82400288)(207.45199825,240.9600012)(205.7,240.9600012)
}
}
{
\newrgbcolor{curcolor}{0 0 0}
\pscustom[linestyle=none,fillstyle=solid,fillcolor=curcolor]
{
\newpath
\moveto(218.63825,243.4320012)
\curveto(218.63825,241.48800315)(217.15024906,240.9840012)(216.21425,240.9840012)
\curveto(214.72625149,240.9840012)(214.00625,242.04000262)(214.00625,243.4560012)
\curveto(214.00625,244.58400007)(214.53425134,245.16000171)(215.87825,245.6640012)
\curveto(216.83824904,246.02400084)(218.11025053,246.48000154)(218.63825,246.8160012)
\lineto(218.63825,243.4320012)
\moveto(220.79825,248.8560012)
\curveto(220.79825,250.24799981)(220.48624638,252.0480012)(216.86225,252.0480012)
\curveto(214.15025271,252.0480012)(212.01425,250.63199988)(212.01425,249.3120012)
\curveto(212.01425,248.54400197)(212.90225046,248.1840012)(213.35825,248.1840012)
\curveto(213.8622495,248.1840012)(214.00625012,248.44800161)(214.12625,248.8560012)
\curveto(214.65424947,250.63199943)(215.63825096,251.2080012)(216.59825,251.2080012)
\curveto(217.53424906,251.2080012)(218.63825,250.72799928)(218.63825,248.8080012)
\lineto(218.63825,247.8000012)
\curveto(218.0382506,247.17600183)(215.7102481,246.5760006)(213.81425,245.9760012)
\curveto(212.08625173,245.44800173)(211.58225,244.24800007)(211.58225,243.1200012)
\curveto(211.58225,241.320003)(212.78225226,239.7120012)(215.03825,239.7120012)
\curveto(216.52624851,239.76000115)(217.84625077,240.64800171)(218.61425,241.1520012)
\curveto(218.95024966,240.26400209)(219.33425084,239.7120012)(220.17425,239.7120012)
\curveto(221.06224911,239.7120012)(222.09425091,239.97600166)(223.00625,240.4320012)
\lineto(222.86225,241.0080012)
\curveto(222.52625034,240.93600127)(221.99824964,240.8880013)(221.63825,240.9840012)
\curveto(221.20625043,241.08000111)(220.79825,241.53600271)(220.79825,243.0480012)
\lineto(220.79825,248.8560012)
}
}
{
\newrgbcolor{curcolor}{0 0 0}
\pscustom[linestyle=none,fillstyle=solid,fillcolor=curcolor]
{
\newpath
\moveto(230.79275,250.8000012)
\lineto(230.79275,251.7600012)
\lineto(227.69675,251.7600012)
\lineto(227.69675,254.7840012)
\lineto(227.00075,254.7840012)
\lineto(225.53675,251.7600012)
\lineto(223.71275,251.7600012)
\lineto(223.71275,250.8000012)
\lineto(225.44075,250.8000012)
\lineto(225.44075,242.4480012)
\curveto(225.44075,240.00000365)(227.26475084,239.7120012)(228.10475,239.7120012)
\curveto(229.32874878,239.7120012)(230.4807507,240.33600163)(231.17675,240.7680012)
\lineto(230.98475,241.2720012)
\curveto(230.40875058,241.03200144)(229.8807494,240.9600012)(229.28075,240.9600012)
\curveto(228.46475082,240.9600012)(227.69675,241.53600303)(227.69675,243.3600012)
\lineto(227.69675,250.8000012)
\lineto(230.79275,250.8000012)
}
}
{
\newrgbcolor{curcolor}{0 0 0}
\pscustom[linestyle=none,fillstyle=solid,fillcolor=curcolor]
{
\newpath
\moveto(232.10675,251.4480012)
\lineto(232.10675,250.8480012)
\lineto(233.01875,250.7280012)
\curveto(233.64274938,250.6320013)(233.78675,250.51200034)(233.78675,249.6480012)
\lineto(233.78675,242.0880012)
\curveto(233.78675,240.86400243)(233.71474827,240.81600108)(231.98675,240.6960012)
\lineto(231.98675,240.0000012)
\lineto(237.84275,240.0000012)
\lineto(237.84275,240.6960012)
\curveto(236.1387517,240.81600108)(236.04275,240.86400243)(236.04275,242.0880012)
\lineto(236.04275,251.9040012)
\lineto(235.87475,252.0480012)
\lineto(232.10675,251.4480012)
\moveto(234.89075,257.1600012)
\curveto(234.02675086,257.1600012)(233.42675,256.53600034)(233.42675,255.6720012)
\curveto(233.42675,254.83200204)(234.02675086,254.2320012)(234.89075,254.2320012)
\curveto(235.77874911,254.2320012)(236.33075002,254.83200204)(236.35475,255.6720012)
\curveto(236.35475,256.53600034)(235.77874911,257.1600012)(234.89075,257.1600012)
}
}
{
\newrgbcolor{curcolor}{0 0 0}
\pscustom[linestyle=none,fillstyle=solid,fillcolor=curcolor]
{
\newpath
\moveto(245.115125,252.0480012)
\curveto(241.27512884,252.0480012)(239.379125,249.76799731)(239.379125,245.8800012)
\curveto(239.379125,241.99200509)(241.27512884,239.7120012)(245.115125,239.7120012)
\curveto(249.00312111,239.7120012)(250.875125,241.99200509)(250.875125,245.8800012)
\curveto(250.875125,249.76799731)(249.00312111,252.0480012)(245.115125,252.0480012)
\moveto(241.875125,245.8800012)
\curveto(241.875125,249.19199789)(242.93112718,251.2080012)(245.115125,251.2080012)
\curveto(247.34712277,251.2080012)(248.379125,249.19199789)(248.379125,245.8800012)
\curveto(248.379125,242.56800451)(247.34712277,240.5520012)(245.115125,240.5520012)
\curveto(242.93112718,240.5520012)(241.875125,242.56800451)(241.875125,245.8800012)
}
}
{
\newrgbcolor{curcolor}{0 0 0}
\pscustom[linestyle=none,fillstyle=solid,fillcolor=curcolor]
{
\newpath
\moveto(264.72725,248.2320012)
\curveto(264.72725,250.65599878)(263.40724779,252.0480012)(261.19925,252.0480012)
\curveto(259.42325178,252.0480012)(258.31924851,251.18400024)(256.83125,250.2240012)
\lineto(256.49525,252.0480012)
\lineto(252.91925,251.4480012)
\lineto(252.91925,250.8480012)
\lineto(253.83125,250.7280012)
\curveto(254.45524938,250.6320013)(254.59925,250.51200034)(254.59925,249.6480012)
\lineto(254.59925,242.0880012)
\curveto(254.59925,240.86400243)(254.52724827,240.81600108)(252.79925,240.6960012)
\lineto(252.79925,240.0000012)
\lineto(258.65525,240.0000012)
\lineto(258.65525,240.6960012)
\curveto(256.9512517,240.81600108)(256.85525,240.86400243)(256.85525,242.0880012)
\lineto(256.85525,247.7040012)
\curveto(256.85525,248.3040006)(256.90325019,248.64000159)(257.09525,249.0240012)
\curveto(257.5992495,249.96000027)(258.67925127,250.7520012)(259.95125,250.7520012)
\curveto(261.58324837,250.7520012)(262.47125,249.83999911)(262.47125,247.7520012)
\lineto(262.47125,242.0880012)
\curveto(262.47125,240.86400243)(262.39924827,240.81600108)(260.67125,240.6960012)
\lineto(260.67125,240.0000012)
\lineto(266.52725,240.0000012)
\lineto(266.52725,240.6960012)
\curveto(264.8232517,240.81600108)(264.72725,240.86400243)(264.72725,242.0880012)
\lineto(264.72725,248.2320012)
}
}
{
\newrgbcolor{curcolor}{0 0 0}
\pscustom[linestyle=none,fillstyle=solid,fillcolor=curcolor]
{
\newpath
\moveto(53.984,540.0000012)
\lineto(53.984,540.9280012)
\lineto(51.296,541.1520012)
\curveto(50.62400067,541.21600114)(50.24,541.47200245)(50.24,542.7200012)
\lineto(50.24,561.5680012)
\lineto(50.08,561.7600012)
\lineto(43.488,560.6400012)
\lineto(43.488,559.8400012)
\lineto(46.464,559.4880012)
\curveto(47.00799946,559.42400127)(47.232,559.16800027)(47.232,558.2400012)
\lineto(47.232,542.7200012)
\curveto(47.232,542.11200181)(47.13599981,541.72800098)(46.944,541.5040012)
\curveto(46.78400016,541.28000143)(46.52799965,541.18400117)(46.176,541.1520012)
\lineto(43.488,540.9280012)
\lineto(43.488,540.0000012)
\lineto(53.984,540.0000012)
}
}
{
\newrgbcolor{curcolor}{0 0 0}
\pscustom[linestyle=none,fillstyle=solid,fillcolor=curcolor]
{
\newpath
\moveto(55.52,464.2240012)
\lineto(54.624,464.3840012)
\curveto(53.95200067,462.65600293)(53.31199872,462.4320012)(52.032,462.4320012)
\lineto(43.84,462.4320012)
\curveto(44.35199949,464.09599954)(46.24000298,466.36800335)(49.216,468.5120012)
\curveto(52.28799693,470.75199896)(54.656,472.19200479)(54.656,475.7760012)
\curveto(54.656,480.22399675)(51.71199664,481.7600012)(48.352,481.7600012)
\curveto(43.96800438,481.7600012)(41.696,479.10399954)(41.696,477.4400012)
\curveto(41.696,476.35200229)(42.91200054,475.8400012)(43.456,475.8400012)
\curveto(44.03199942,475.8400012)(44.25600013,476.16000175)(44.384,476.7040012)
\curveto(44.89599949,478.87999903)(46.14400198,480.6400012)(48.128,480.6400012)
\curveto(50.59199754,480.6400012)(51.296,478.55999887)(51.296,476.2240012)
\curveto(51.296,472.80000463)(49.56799722,470.87999874)(46.784,468.4160012)
\curveto(42.84800394,464.99200463)(41.47199939,462.81599871)(40.864,460.3200012)
\lineto(41.184,460.0000012)
\lineto(54.432,460.0000012)
\lineto(55.52,464.2240012)
}
}
{
\newrgbcolor{curcolor}{0 0 0}
\pscustom[linestyle=none,fillstyle=solid,fillcolor=curcolor]
{
\newpath
\moveto(44.16,411.7120012)
\curveto(44.16,411.16800175)(44.32000042,410.8160012)(44.736,410.8160012)
\curveto(45.11999962,410.8160012)(45.98400154,411.1360012)(47.52,411.1360012)
\curveto(50.23999728,411.1360012)(51.776,408.67199842)(51.776,405.8880012)
\curveto(51.776,402.08000501)(49.88799773,400.7360012)(47.616,400.7360012)
\curveto(45.50400211,400.7360012)(44.12799939,402.33600299)(43.52,404.1280012)
\curveto(43.32800019,404.73600059)(43.00799955,405.0240012)(42.56,405.0240012)
\curveto(41.98400058,405.0240012)(40.864,404.41600005)(40.864,403.2640012)
\curveto(40.864,401.88800258)(43.07200451,399.6160012)(47.584,399.6160012)
\curveto(52.22399536,399.6160012)(55.136,401.9520053)(55.136,406.0480012)
\curveto(55.136,410.33599691)(51.45599824,411.61600136)(49.696,411.7760012)
\lineto(49.696,411.9040012)
\curveto(51.42399827,412.19200091)(54.208,413.47200437)(54.208,416.6400012)
\curveto(54.208,420.19199765)(51.42399658,421.7600012)(48,421.7600012)
\curveto(43.77600422,421.7600012)(41.696,419.32799983)(41.696,417.9520012)
\curveto(41.696,416.96000219)(42.81600045,416.4160012)(43.264,416.4160012)
\curveto(43.67999958,416.4160012)(43.93600013,416.64000162)(44.064,417.0560012)
\curveto(44.7679993,419.23199903)(45.98400179,420.6400012)(47.776,420.6400012)
\curveto(50.30399747,420.6400012)(50.912,418.39999941)(50.912,416.6080012)
\curveto(50.912,414.68800312)(50.23999728,412.2880012)(47.52,412.2880012)
\curveto(45.98400154,412.2880012)(45.11999962,412.6080012)(44.736,412.6080012)
\curveto(44.32000042,412.6080012)(44.16,412.28800063)(44.16,411.7120012)
}
}
{
\newrgbcolor{curcolor}{0 0 0}
\pscustom[linestyle=none,fillstyle=solid,fillcolor=curcolor]
{
\newpath
\moveto(509.6,355.7280012)
\lineto(509.6,352.4640012)
\curveto(509.6,351.37600229)(509.24799923,351.18400114)(508.48,351.1200012)
\lineto(506.432,350.9280012)
\lineto(506.432,350.0000012)
\lineto(514.88,350.0000012)
\lineto(514.88,350.9280012)
\lineto(513.44,351.0880012)
\curveto(512.70400074,351.18400111)(512.48,351.37600229)(512.48,352.4640012)
\lineto(512.48,355.7280012)
\lineto(515.776,355.7280012)
\lineto(515.776,357.2320012)
\lineto(512.48,357.2320012)
\lineto(512.48,371.3760012)
\lineto(510.144,371.3760012)
\curveto(507.13600301,366.96000562)(503.55199706,361.39199621)(500.608,356.4000012)
\lineto(500.896,355.7280012)
\lineto(509.6,355.7280012)
\moveto(502.912,357.2320012)
\curveto(504.83199808,360.71999771)(507.07200246,364.40000507)(509.536,368.2720012)
\lineto(509.6,368.2720012)
\lineto(509.6,357.2320012)
\lineto(502.912,357.2320012)
}
}
{
\newrgbcolor{curcolor}{1 1 1}
\pscustom[linestyle=none,fillstyle=solid,fillcolor=curcolor]
{
\newpath
\moveto(149.99850464,429.91890837)
\lineto(409.9803772,429.91890837)
\lineto(409.9803772,389.8669026)
\lineto(149.99850464,389.8669026)
\lineto(149.99850464,429.91890837)
\closepath
}
}
{
\newrgbcolor{curcolor}{1 0 0}
\pscustom[linewidth=2,linecolor=curcolor,linestyle=dashed,dash=8 8]
{
\newpath
\moveto(149.99850464,429.91890837)
\lineto(409.9803772,429.91890837)
\lineto(409.9803772,389.8669026)
\lineto(149.99850464,389.8669026)
\lineto(149.99850464,429.91890837)
\closepath
}
}
{
\newrgbcolor{curcolor}{0 0 0}
\pscustom[linestyle=none,fillstyle=solid,fillcolor=curcolor]
{
\newpath
\moveto(280.76009941,340.41146971)
\lineto(419.70257759,340.41146971)
\curveto(425.69124755,340.41146971)(430.51245117,345.23267332)(430.51245117,351.22134329)
\lineto(430.51245117,358.7882588)
\curveto(430.51245117,364.77692876)(425.69124755,369.59813238)(419.70257759,369.59813238)
\lineto(280.76009941,369.59813238)
\curveto(274.77142945,369.59813238)(269.95022583,364.77692876)(269.95022583,358.7882588)
\lineto(269.95022583,351.22134329)
\curveto(269.95022583,345.23267332)(274.77142945,340.41146971)(280.76009941,340.41146971)
\closepath
}
}
{
\newrgbcolor{curcolor}{0 0 0}
\pscustom[linewidth=2,linecolor=curcolor]
{
\newpath
\moveto(280.76009941,340.41146971)
\lineto(419.70257759,340.41146971)
\curveto(425.69124755,340.41146971)(430.51245117,345.23267332)(430.51245117,351.22134329)
\lineto(430.51245117,358.7882588)
\curveto(430.51245117,364.77692876)(425.69124755,369.59813238)(419.70257759,369.59813238)
\lineto(280.76009941,369.59813238)
\curveto(274.77142945,369.59813238)(269.95022583,364.77692876)(269.95022583,358.7882588)
\lineto(269.95022583,351.22134329)
\curveto(269.95022583,345.23267332)(274.77142945,340.41146971)(280.76009941,340.41146971)
\closepath
}
}
{
\newrgbcolor{curcolor}{0 0 0}
\pscustom[linestyle=none,fillstyle=solid,fillcolor=curcolor]
{
\newpath
\moveto(513.632,298.9440012)
\lineto(514.4,301.1200012)
\lineto(514.24,301.3760012)
\lineto(503.552,301.3760012)
\lineto(503.232,301.0880012)
\lineto(502.592,290.3680012)
\lineto(503.36,289.9840012)
\curveto(504.70399866,291.51999967)(506.01600173,292.2560012)(507.744,292.2560012)
\curveto(509.79199795,292.2560012)(512.032,290.75199698)(512.032,286.5280012)
\curveto(512.032,283.2320045)(510.68799722,280.7360012)(507.904,280.7360012)
\curveto(505.66400224,280.7360012)(504.51199936,282.36800303)(503.872,284.1920012)
\curveto(503.71200016,284.67200072)(503.42399955,284.9600012)(502.976,284.9600012)
\curveto(502.33600064,284.9600012)(501.28,284.32000015)(501.28,283.2640012)
\curveto(501.28,281.79200267)(503.52000416,279.6160012)(507.68,279.6160012)
\curveto(512.95999472,279.6160012)(515.36,282.81600527)(515.36,286.8800012)
\curveto(515.36,291.35999672)(512.54399619,293.6640012)(508.736,293.6640012)
\curveto(506.97600176,293.6640012)(505.08799898,292.89600027)(504.064,291.9680012)
\lineto(504,292.0320012)
\lineto(504.736,298.0160012)
\curveto(504.8319999,298.8160004)(505.05600061,298.9440012)(505.664,298.9440012)
\lineto(513.632,298.9440012)
}
}
{
\newrgbcolor{curcolor}{0 0 0}
\pscustom[linestyle=none,fillstyle=solid,fillcolor=curcolor]
{
\newpath
\moveto(290.2822094,272.23803831)
\lineto(420.36839485,272.23803831)
\curveto(426.06473886,272.23803831)(430.65060425,276.8239037)(430.65060425,282.52024771)
\lineto(430.65060425,289.71779753)
\curveto(430.65060425,295.41414153)(426.06473886,300.00000692)(420.36839485,300.00000692)
\lineto(290.2822094,300.00000692)
\curveto(284.58586539,300.00000692)(280,295.41414153)(280,289.71779753)
\lineto(280,282.52024771)
\curveto(280,276.8239037)(284.58586539,272.23803831)(290.2822094,272.23803831)
\closepath
}
}
{
\newrgbcolor{curcolor}{0 0 0}
\pscustom[linewidth=2,linecolor=curcolor]
{
\newpath
\moveto(290.2822094,272.23803831)
\lineto(420.36839485,272.23803831)
\curveto(426.06473886,272.23803831)(430.65060425,276.8239037)(430.65060425,282.52024771)
\lineto(430.65060425,289.71779753)
\curveto(430.65060425,295.41414153)(426.06473886,300.00000692)(420.36839485,300.00000692)
\lineto(290.2822094,300.00000692)
\curveto(284.58586539,300.00000692)(280,295.41414153)(280,289.71779753)
\lineto(280,282.52024771)
\curveto(280,276.8239037)(284.58586539,272.23803831)(290.2822094,272.23803831)
\closepath
}
}
{
\newrgbcolor{curcolor}{0 0 0}
\pscustom[linestyle=none,fillstyle=solid,fillcolor=curcolor]
{
\newpath
\moveto(291.03757477,230.0000012)
\lineto(419.01493073,230.0000012)
\curveto(425.12974715,230.0000012)(430.05250549,234.92275955)(430.05250549,241.03757597)
\lineto(430.05250549,248.76388098)
\curveto(430.05250549,254.8786974)(425.12974715,259.80145575)(419.01493073,259.80145575)
\lineto(291.03757477,259.80145575)
\curveto(284.92275835,259.80145575)(280,254.8786974)(280,248.76388098)
\lineto(280,241.03757597)
\curveto(280,234.92275955)(284.92275835,230.0000012)(291.03757477,230.0000012)
\closepath
}
}
{
\newrgbcolor{curcolor}{0 0 0}
\pscustom[linewidth=2,linecolor=curcolor]
{
\newpath
\moveto(291.03757477,230.0000012)
\lineto(419.01493073,230.0000012)
\curveto(425.12974715,230.0000012)(430.05250549,234.92275955)(430.05250549,241.03757597)
\lineto(430.05250549,248.76388098)
\curveto(430.05250549,254.8786974)(425.12974715,259.80145575)(419.01493073,259.80145575)
\lineto(291.03757477,259.80145575)
\curveto(284.92275835,259.80145575)(280,254.8786974)(280,248.76388098)
\lineto(280,241.03757597)
\curveto(280,234.92275955)(284.92275835,230.0000012)(291.03757477,230.0000012)
\closepath
}
}
{
\newrgbcolor{curcolor}{0 0 0}
\pscustom[linestyle=none,fillstyle=solid,fillcolor=curcolor]
{
\newpath
\moveto(141.62518787,99.99997068)
\lineto(248.98716259,99.99997068)
\curveto(255.08827452,99.99997068)(260,104.91169617)(260,111.01280809)
\lineto(260,118.7218001)
\curveto(260,124.82291202)(255.08827452,129.73463751)(248.98716259,129.73463751)
\lineto(141.62518787,129.73463751)
\curveto(135.52407595,129.73463751)(130.61235046,124.82291202)(130.61235046,118.7218001)
\lineto(130.61235046,111.01280809)
\curveto(130.61235046,104.91169617)(135.52407595,99.99997068)(141.62518787,99.99997068)
\closepath
}
}
{
\newrgbcolor{curcolor}{0 0 0}
\pscustom[linewidth=2,linecolor=curcolor]
{
\newpath
\moveto(141.62518787,99.99997068)
\lineto(248.98716259,99.99997068)
\curveto(255.08827452,99.99997068)(260,104.91169617)(260,111.01280809)
\lineto(260,118.7218001)
\curveto(260,124.82291202)(255.08827452,129.73463751)(248.98716259,129.73463751)
\lineto(141.62518787,129.73463751)
\curveto(135.52407595,129.73463751)(130.61235046,124.82291202)(130.61235046,118.7218001)
\lineto(130.61235046,111.01280809)
\curveto(130.61235046,104.91169617)(135.52407595,99.99997068)(141.62518787,99.99997068)
\closepath
}
}
{
\newrgbcolor{curcolor}{0 0 0}
\pscustom[linestyle=none,fillstyle=solid,fillcolor=curcolor]
{
\newpath
\moveto(322.16531563,99.73465086)
\lineto(418.9267025,99.73465086)
\curveto(425.40054725,99.73465086)(430.61234283,104.94644645)(430.61234283,111.42029119)
\lineto(430.61234283,119.60024382)
\curveto(430.61234283,126.07408856)(425.40054725,131.28588415)(418.9267025,131.28588415)
\lineto(322.16531563,131.28588415)
\curveto(315.69147088,131.28588415)(310.47967529,126.07408856)(310.47967529,119.60024382)
\lineto(310.47967529,111.42029119)
\curveto(310.47967529,104.94644645)(315.69147088,99.73465086)(322.16531563,99.73465086)
\closepath
}
}
{
\newrgbcolor{curcolor}{0 0 0}
\pscustom[linewidth=2,linecolor=curcolor]
{
\newpath
\moveto(322.16531563,99.73465086)
\lineto(418.9267025,99.73465086)
\curveto(425.40054725,99.73465086)(430.61234283,104.94644645)(430.61234283,111.42029119)
\lineto(430.61234283,119.60024382)
\curveto(430.61234283,126.07408856)(425.40054725,131.28588415)(418.9267025,131.28588415)
\lineto(322.16531563,131.28588415)
\curveto(315.69147088,131.28588415)(310.47967529,126.07408856)(310.47967529,119.60024382)
\lineto(310.47967529,111.42029119)
\curveto(310.47967529,104.94644645)(315.69147088,99.73465086)(322.16531563,99.73465086)
\closepath
}
}
{
\newrgbcolor{curcolor}{0 0 0}
\pscustom[linestyle=none,fillstyle=solid,fillcolor=curcolor]
{
\newpath
\moveto(508.704,239.6160012)
\curveto(513.43999526,239.6160012)(515.968,242.49600555)(515.968,246.8480012)
\curveto(515.968,251.16799688)(513.18399667,253.56800114)(509.856,253.5040012)
\curveto(507.42400243,253.56800114)(505.50399914,251.80800015)(504.64,250.7520012)
\lineto(504.576,250.7520012)
\curveto(504.60799997,257.79199416)(506.62400288,260.6400012)(509.504,260.6400012)
\curveto(510.97599853,260.6400012)(512.00000048,259.39199938)(512.48,257.5680012)
\curveto(512.5759999,257.18400159)(512.83200048,256.8960012)(513.312,256.8960012)
\curveto(514.0159993,256.8960012)(515.008,257.4400021)(515.008,258.3360012)
\curveto(515.008,259.61599992)(513.43999616,261.7600012)(509.6,261.7600012)
\curveto(507.07200253,261.7600012)(505.27999866,260.79999967)(503.936,259.2640012)
\curveto(502.49600144,257.60000287)(501.408,254.52799656)(501.408,249.8880012)
\curveto(501.408,242.97600811)(503.9040048,239.6160012)(508.704,239.6160012)
\moveto(508.768,251.8720012)
\curveto(510.78399798,251.8720012)(512.768,250.04799752)(512.768,246.3680012)
\curveto(512.768,243.00800456)(511.45599731,240.7360012)(508.768,240.7360012)
\curveto(505.98400278,240.7360012)(504.64,243.71200504)(504.64,247.5520012)
\curveto(504.64,249.50399925)(506.08000269,251.8720012)(508.768,251.8720012)
}
}
{
\newrgbcolor{curcolor}{0 0 0}
\pscustom[linestyle=none,fillstyle=solid,fillcolor=curcolor]
{
\newpath
\moveto(502.304,211.3760012)
\lineto(501.312,206.8960012)
\lineto(502.176,206.8960012)
\lineto(502.496,207.5680012)
\curveto(502.91199958,208.46400031)(503.32800144,208.9440012)(504.768,208.9440012)
\lineto(513.408,208.9440012)
\curveto(513.08800032,207.79200235)(512.51199741,205.93599749)(509.92,202.2240012)
\curveto(507.16800275,198.28800514)(505.344,194.70399743)(505.344,190.9280012)
\curveto(505.344,189.87200226)(506.08000099,189.6160012)(507.072,189.6160012)
\curveto(507.99999907,189.6160012)(508.67199997,189.90400213)(508.64,190.8320012)
\curveto(508.48000016,195.05599698)(509.3440017,198.44800418)(511.04,201.4240012)
\curveto(513.08799795,205.00799762)(514.78400112,206.99200523)(515.904,211.0240012)
\lineto(515.616,211.3760012)
\lineto(502.304,211.3760012)
}
}
{
\newrgbcolor{curcolor}{0 0 0}
\pscustom[linestyle=none,fillstyle=solid,fillcolor=curcolor]
{
\newpath
\moveto(508.48,160.8160012)
\curveto(511.03999744,160.8160012)(512.416,158.63999829)(512.416,155.7280012)
\curveto(512.416,152.91200402)(511.19999728,150.7360012)(508.48,150.7360012)
\curveto(505.82400266,150.7360012)(504.544,152.91200402)(504.544,155.7280012)
\curveto(504.544,158.63999829)(505.95200253,160.8160012)(508.48,160.8160012)
\moveto(508.48,149.6160012)
\curveto(513.66399482,149.6160012)(515.616,152.59200443)(515.616,155.8240012)
\curveto(515.616,159.08799794)(513.56799782,160.94400171)(511.392,161.4560012)
\lineto(511.392,161.5520012)
\curveto(513.34399805,162.06400069)(514.912,163.69600383)(514.912,166.3200012)
\curveto(514.912,169.80799771)(512.19199629,171.7600012)(508.48,171.7600012)
\curveto(504.83200365,171.7600012)(502.048,169.77599775)(502.048,166.3200012)
\curveto(502.048,163.69600383)(503.64800192,162.06400069)(505.568,161.5520012)
\lineto(505.568,161.4560012)
\curveto(503.42400214,160.94400171)(501.344,159.08799794)(501.344,155.8240012)
\curveto(501.344,152.59200443)(503.26400522,149.6160012)(508.48,149.6160012)
\moveto(508.48,161.9680012)
\curveto(506.1760023,161.9680012)(505.152,164.20800338)(505.152,166.3840012)
\curveto(505.152,168.84799874)(506.30400218,170.6400012)(508.48,170.6400012)
\curveto(510.65599782,170.6400012)(511.74400006,168.84799874)(511.808,166.3840012)
\curveto(511.808,164.20800338)(510.81599766,161.9680012)(508.48,161.9680012)
}
}
{
\newrgbcolor{curcolor}{0 0 0}
\pscustom[linewidth=2,linecolor=curcolor,linestyle=dashed,dash=8 8]
{
\newpath
\moveto(150,550)
\lineto(60,550)
}
}
{
\newrgbcolor{curcolor}{0 0 0}
\pscustom[linestyle=none,fillstyle=solid,fillcolor=curcolor]
{
\newpath
\moveto(139.53769464,554.84048224)
\lineto(152.6487474,550.01921591)
\lineto(139.53769392,545.19795064)
\curveto(141.632292,548.04442372)(141.62022288,551.93889292)(139.53769464,554.84048224)
\lineto(139.53769464,554.84048224)
\lineto(139.53769464,554.84048224)
\closepath
}
}
{
\newrgbcolor{curcolor}{0 0 0}
\pscustom[linewidth=2,linecolor=curcolor,linestyle=dashed,dash=8 8]
{
\newpath
\moveto(170,470)
\lineto(60,470)
}
}
{
\newrgbcolor{curcolor}{0 0 0}
\pscustom[linestyle=none,fillstyle=solid,fillcolor=curcolor]
{
\newpath
\moveto(159.53769464,474.84048224)
\lineto(172.6487474,470.01921591)
\lineto(159.53769392,465.19795064)
\curveto(161.632292,468.04442372)(161.62022288,471.93889292)(159.53769464,474.84048224)
\lineto(159.53769464,474.84048224)
\lineto(159.53769464,474.84048224)
\closepath
}
}
{
\newrgbcolor{curcolor}{0 0 0}
\pscustom[linewidth=2,linecolor=curcolor,linestyle=dashed,dash=8 8]
{
\newpath
\moveto(170,410)
\lineto(60,410)
}
}
{
\newrgbcolor{curcolor}{0 0 0}
\pscustom[linestyle=none,fillstyle=solid,fillcolor=curcolor]
{
\newpath
\moveto(159.53769464,414.84048224)
\lineto(172.6487474,410.01921591)
\lineto(159.53769392,405.19795064)
\curveto(161.632292,408.04442372)(161.62022288,411.93889292)(159.53769464,414.84048224)
\lineto(159.53769464,414.84048224)
\lineto(159.53769464,414.84048224)
\closepath
}
}
{
\newrgbcolor{curcolor}{0 0 0}
\pscustom[linewidth=2,linecolor=curcolor,linestyle=dashed,dash=8 8]
{
\newpath
\moveto(410,360)
\lineto(500,360)
}
}
{
\newrgbcolor{curcolor}{0 0 0}
\pscustom[linestyle=none,fillstyle=solid,fillcolor=curcolor]
{
\newpath
\moveto(420.46230536,355.15951776)
\lineto(407.3512526,359.98078409)
\lineto(420.46230608,364.80204936)
\curveto(418.367708,361.95557628)(418.37977712,358.06110708)(420.46230536,355.15951776)
\lineto(420.46230536,355.15951776)
\lineto(420.46230536,355.15951776)
\closepath
}
}
{
\newrgbcolor{curcolor}{0 0 0}
\pscustom[linewidth=2,linecolor=curcolor,linestyle=dashed,dash=8 8]
{
\newpath
\moveto(410,290)
\lineto(500,290)
}
}
{
\newrgbcolor{curcolor}{0 0 0}
\pscustom[linestyle=none,fillstyle=solid,fillcolor=curcolor]
{
\newpath
\moveto(420.46230536,285.15951776)
\lineto(407.3512526,289.98078409)
\lineto(420.46230608,294.80204936)
\curveto(418.367708,291.95557628)(418.37977712,288.06110708)(420.46230536,285.15951776)
\lineto(420.46230536,285.15951776)
\lineto(420.46230536,285.15951776)
\closepath
}
}
{
\newrgbcolor{curcolor}{0 0 0}
\pscustom[linewidth=2,linecolor=curcolor,linestyle=dashed,dash=8 8]
{
\newpath
\moveto(410,250)
\lineto(500,250)
}
}
{
\newrgbcolor{curcolor}{0 0 0}
\pscustom[linestyle=none,fillstyle=solid,fillcolor=curcolor]
{
\newpath
\moveto(420.46230536,245.15951776)
\lineto(407.3512526,249.98078409)
\lineto(420.46230608,254.80204936)
\curveto(418.367708,251.95557628)(418.37977712,248.06110708)(420.46230536,245.15951776)
\lineto(420.46230536,245.15951776)
\lineto(420.46230536,245.15951776)
\closepath
}
}
{
\newrgbcolor{curcolor}{0 0 0}
\pscustom[linewidth=2,linecolor=curcolor,linestyle=dashed,dash=8 8]
{
\newpath
\moveto(410,120)
\lineto(500,120)
}
}
{
\newrgbcolor{curcolor}{0 0 0}
\pscustom[linestyle=none,fillstyle=solid,fillcolor=curcolor]
{
\newpath
\moveto(420.46230536,115.15951776)
\lineto(407.3512526,119.98078409)
\lineto(420.46230608,124.80204936)
\curveto(418.367708,121.95557628)(418.37977712,118.06110708)(420.46230536,115.15951776)
\lineto(420.46230536,115.15951776)
\lineto(420.46230536,115.15951776)
\closepath
}
}
{
\newrgbcolor{curcolor}{0 0 0}
\pscustom[linewidth=2,linecolor=curcolor,linestyle=dashed,dash=8 8]
{
\newpath
\moveto(150,120)
\lineto(60,120)
}
}
{
\newrgbcolor{curcolor}{0 0 0}
\pscustom[linestyle=none,fillstyle=solid,fillcolor=curcolor]
{
\newpath
\moveto(139.53769464,124.84048224)
\lineto(152.6487474,120.01921591)
\lineto(139.53769392,115.19795064)
\curveto(141.632292,118.04442372)(141.62022288,121.93889292)(139.53769464,124.84048224)
\lineto(139.53769464,124.84048224)
\lineto(139.53769464,124.84048224)
\closepath
}
}
{
\newrgbcolor{curcolor}{0 0 0}
\pscustom[linestyle=none,fillstyle=solid,fillcolor=curcolor]
{
\newpath
\moveto(155.192,206.0320012)
\curveto(154.11200108,206.5360007)(152.28799782,206.9680012)(150.104,206.9680012)
\curveto(147.15200295,206.9680012)(144.77599856,206.10399959)(143.336,204.4960012)
\curveto(141.96800137,202.96000274)(141.152,200.89599849)(141.152,198.1840012)
\curveto(141.152,195.37600401)(141.99200144,193.11999976)(143.432,191.6800012)
\curveto(144.91999851,190.19200269)(147.27200269,189.6400012)(149.96,189.6400012)
\curveto(151.87999808,189.6400012)(154.08800115,190.16800183)(155.24,190.7920012)
\lineto(155.576,194.8000012)
\lineto(154.856,194.8000012)
\curveto(154.18400067,192.3040037)(152.95999686,190.4800012)(149.816,190.4800012)
\curveto(144.87200494,190.4800012)(143.84,195.35200411)(143.84,198.2560012)
\curveto(143.84,202.19199727)(145.40000446,206.15200118)(149.864,206.1280012)
\curveto(152.31199755,206.1280012)(154.01600046,205.11999837)(154.472,202.2880012)
\lineto(155.192,202.2880012)
\lineto(155.192,206.0320012)
}
}
{
\newrgbcolor{curcolor}{0 0 0}
\pscustom[linestyle=none,fillstyle=solid,fillcolor=curcolor]
{
\newpath
\moveto(163.31825,202.0480012)
\curveto(159.47825384,202.0480012)(157.58225,199.76799731)(157.58225,195.8800012)
\curveto(157.58225,191.99200509)(159.47825384,189.7120012)(163.31825,189.7120012)
\curveto(167.20624611,189.7120012)(169.07825,191.99200509)(169.07825,195.8800012)
\curveto(169.07825,199.76799731)(167.20624611,202.0480012)(163.31825,202.0480012)
\moveto(160.07825,195.8800012)
\curveto(160.07825,199.19199789)(161.13425218,201.2080012)(163.31825,201.2080012)
\curveto(165.55024777,201.2080012)(166.58225,199.19199789)(166.58225,195.8800012)
\curveto(166.58225,192.56800451)(165.55024777,190.5520012)(163.31825,190.5520012)
\curveto(161.13425218,190.5520012)(160.07825,192.56800451)(160.07825,195.8800012)
}
}
{
\newrgbcolor{curcolor}{0 0 0}
\pscustom[linestyle=none,fillstyle=solid,fillcolor=curcolor]
{
\newpath
\moveto(182.930375,198.2320012)
\curveto(182.930375,200.65599878)(181.61037279,202.0480012)(179.402375,202.0480012)
\curveto(177.62637678,202.0480012)(176.52237351,201.18400024)(175.034375,200.2240012)
\lineto(174.698375,202.0480012)
\lineto(171.122375,201.4480012)
\lineto(171.122375,200.8480012)
\lineto(172.034375,200.7280012)
\curveto(172.65837438,200.6320013)(172.802375,200.51200034)(172.802375,199.6480012)
\lineto(172.802375,192.0880012)
\curveto(172.802375,190.86400243)(172.73037327,190.81600108)(171.002375,190.6960012)
\lineto(171.002375,190.0000012)
\lineto(176.858375,190.0000012)
\lineto(176.858375,190.6960012)
\curveto(175.1543767,190.81600108)(175.058375,190.86400243)(175.058375,192.0880012)
\lineto(175.058375,197.7040012)
\curveto(175.058375,198.3040006)(175.10637519,198.64000159)(175.298375,199.0240012)
\curveto(175.8023745,199.96000027)(176.88237627,200.7520012)(178.154375,200.7520012)
\curveto(179.78637337,200.7520012)(180.674375,199.83999911)(180.674375,197.7520012)
\lineto(180.674375,192.0880012)
\curveto(180.674375,190.86400243)(180.60237327,190.81600108)(178.874375,190.6960012)
\lineto(178.874375,190.0000012)
\lineto(184.730375,190.0000012)
\lineto(184.730375,190.6960012)
\curveto(183.0263767,190.81600108)(182.930375,190.86400243)(182.930375,192.0880012)
\lineto(182.930375,198.2320012)
}
}
{
\newrgbcolor{curcolor}{0 0 0}
\pscustom[linestyle=none,fillstyle=solid,fillcolor=curcolor]
{
\newpath
\moveto(197.78975,198.2320012)
\curveto(197.78975,200.65599878)(196.46974779,202.0480012)(194.26175,202.0480012)
\curveto(192.48575178,202.0480012)(191.38174851,201.18400024)(189.89375,200.2240012)
\lineto(189.55775,202.0480012)
\lineto(185.98175,201.4480012)
\lineto(185.98175,200.8480012)
\lineto(186.89375,200.7280012)
\curveto(187.51774938,200.6320013)(187.66175,200.51200034)(187.66175,199.6480012)
\lineto(187.66175,192.0880012)
\curveto(187.66175,190.86400243)(187.58974827,190.81600108)(185.86175,190.6960012)
\lineto(185.86175,190.0000012)
\lineto(191.71775,190.0000012)
\lineto(191.71775,190.6960012)
\curveto(190.0137517,190.81600108)(189.91775,190.86400243)(189.91775,192.0880012)
\lineto(189.91775,197.7040012)
\curveto(189.91775,198.3040006)(189.96575019,198.64000159)(190.15775,199.0240012)
\curveto(190.6617495,199.96000027)(191.74175127,200.7520012)(193.01375,200.7520012)
\curveto(194.64574837,200.7520012)(195.53375,199.83999911)(195.53375,197.7520012)
\lineto(195.53375,192.0880012)
\curveto(195.53375,190.86400243)(195.46174827,190.81600108)(193.73375,190.6960012)
\lineto(193.73375,190.0000012)
\lineto(199.58975,190.0000012)
\lineto(199.58975,190.6960012)
\curveto(197.8857517,190.81600108)(197.78975,190.86400243)(197.78975,192.0880012)
\lineto(197.78975,198.2320012)
}
}
{
\newrgbcolor{curcolor}{0 0 0}
\pscustom[linestyle=none,fillstyle=solid,fillcolor=curcolor]
{
\newpath
\moveto(210.537125,196.3120012)
\curveto(211.0411245,196.3120012)(211.497125,196.43200199)(211.497125,197.2240012)
\curveto(211.497125,198.63999979)(211.04112061,202.0480012)(206.649125,202.0480012)
\curveto(202.90512874,202.0480012)(201.129125,199.3839976)(201.129125,195.7840012)
\curveto(201.129125,191.99200499)(202.76112891,189.66400125)(206.673125,189.7120012)
\curveto(209.33712234,189.73600118)(210.70512565,191.2000031)(211.353125,193.0960012)
\lineto(210.633125,193.4800012)
\curveto(209.96112567,192.08800259)(209.0491231,190.9360012)(207.153125,190.9360012)
\curveto(204.17712798,190.9360012)(203.57712505,193.8160037)(203.625125,196.3120012)
\lineto(210.537125,196.3120012)
\moveto(203.649125,197.2240012)
\curveto(203.649125,198.20800022)(204.0091275,201.2080012)(206.505125,201.2080012)
\curveto(208.73712277,201.2080012)(209.001125,198.92800027)(209.001125,197.9920012)
\curveto(209.001125,197.53600166)(208.85712433,197.2240012)(208.185125,197.2240012)
\lineto(203.649125,197.2240012)
}
}
{
\newrgbcolor{curcolor}{0 0 0}
\pscustom[linestyle=none,fillstyle=solid,fillcolor=curcolor]
{
\newpath
\moveto(216.91325,190.0000012)
\lineto(216.91325,190.6960012)
\lineto(215.90525,190.8160012)
\curveto(215.56925034,190.86400115)(215.59325014,191.12800147)(215.73725,191.3920012)
\curveto(216.26524947,192.42400017)(217.03325082,193.57600245)(217.84925,194.8240012)
\lineto(217.89725,194.8240012)
\lineto(220.03325,191.4880012)
\curveto(220.29724974,191.08000161)(220.22524964,190.86400115)(219.86525,190.8160012)
\lineto(218.85725,190.6960012)
\lineto(218.85725,190.0000012)
\lineto(224.49725,190.0000012)
\lineto(224.49725,190.6960012)
\curveto(223.41725108,190.74400115)(223.0092494,190.81600207)(222.40925,191.6800012)
\curveto(221.3052511,193.26399962)(220.17724894,194.94400288)(219.12125,196.6240012)
\curveto(219.91324921,197.824)(220.84925094,199.02400233)(221.78525,200.1520012)
\curveto(222.40924938,200.92000043)(222.72125115,201.01600125)(223.87325,201.0640012)
\lineto(223.87325,201.7600012)
\lineto(219.40925,201.7600012)
\lineto(219.40925,201.0640012)
\lineto(220.34525,200.9680012)
\curveto(220.72924962,200.92000125)(220.68124983,200.68000094)(220.51325,200.4160012)
\curveto(219.96125055,199.50400211)(219.31324928,198.4960001)(218.59325,197.3920012)
\lineto(218.56925,197.3920012)
\lineto(216.55325,200.3920012)
\curveto(216.36125019,200.68000091)(216.40925036,200.94400125)(216.76925,200.9920012)
\lineto(217.48925,201.0640012)
\lineto(217.48925,201.7600012)
\lineto(212.47325,201.7600012)
\lineto(212.47325,201.0640012)
\curveto(213.43324904,201.01600125)(213.67325058,200.96800041)(214.24925,200.1760012)
\curveto(215.25724899,198.78400259)(216.28925103,197.22399967)(217.32125,195.6880012)
\curveto(216.43325089,194.3920025)(215.49724902,193.02399991)(214.51325,191.7280012)
\curveto(213.88925062,190.88800204)(213.36124894,190.76800113)(212.30525,190.6960012)
\lineto(212.30525,190.0000012)
\lineto(216.91325,190.0000012)
}
}
{
\newrgbcolor{curcolor}{0 0 0}
\pscustom[linestyle=none,fillstyle=solid,fillcolor=curcolor]
{
\newpath
\moveto(224.841125,201.4480012)
\lineto(224.841125,200.8480012)
\lineto(225.753125,200.7280012)
\curveto(226.37712438,200.6320013)(226.521125,200.51200034)(226.521125,199.6480012)
\lineto(226.521125,192.0880012)
\curveto(226.521125,190.86400243)(226.44912327,190.81600108)(224.721125,190.6960012)
\lineto(224.721125,190.0000012)
\lineto(230.577125,190.0000012)
\lineto(230.577125,190.6960012)
\curveto(228.8731267,190.81600108)(228.777125,190.86400243)(228.777125,192.0880012)
\lineto(228.777125,201.9040012)
\lineto(228.609125,202.0480012)
\lineto(224.841125,201.4480012)
\moveto(227.625125,207.1600012)
\curveto(226.76112586,207.1600012)(226.161125,206.53600034)(226.161125,205.6720012)
\curveto(226.161125,204.83200204)(226.76112586,204.2320012)(227.625125,204.2320012)
\curveto(228.51312411,204.2320012)(229.06512502,204.83200204)(229.089125,205.6720012)
\curveto(229.089125,206.53600034)(228.51312411,207.1600012)(227.625125,207.1600012)
}
}
{
\newrgbcolor{curcolor}{0 0 0}
\pscustom[linestyle=none,fillstyle=solid,fillcolor=curcolor]
{
\newpath
\moveto(237.8495,202.0480012)
\curveto(234.00950384,202.0480012)(232.1135,199.76799731)(232.1135,195.8800012)
\curveto(232.1135,191.99200509)(234.00950384,189.7120012)(237.8495,189.7120012)
\curveto(241.73749611,189.7120012)(243.6095,191.99200509)(243.6095,195.8800012)
\curveto(243.6095,199.76799731)(241.73749611,202.0480012)(237.8495,202.0480012)
\moveto(234.6095,195.8800012)
\curveto(234.6095,199.19199789)(235.66550218,201.2080012)(237.8495,201.2080012)
\curveto(240.08149777,201.2080012)(241.1135,199.19199789)(241.1135,195.8800012)
\curveto(241.1135,192.56800451)(240.08149777,190.5520012)(237.8495,190.5520012)
\curveto(235.66550218,190.5520012)(234.6095,192.56800451)(234.6095,195.8800012)
}
}
{
\newrgbcolor{curcolor}{0 0 0}
\pscustom[linestyle=none,fillstyle=solid,fillcolor=curcolor]
{
\newpath
\moveto(257.461625,198.2320012)
\curveto(257.461625,200.65599878)(256.14162279,202.0480012)(253.933625,202.0480012)
\curveto(252.15762678,202.0480012)(251.05362351,201.18400024)(249.565625,200.2240012)
\lineto(249.229625,202.0480012)
\lineto(245.653625,201.4480012)
\lineto(245.653625,200.8480012)
\lineto(246.565625,200.7280012)
\curveto(247.18962438,200.6320013)(247.333625,200.51200034)(247.333625,199.6480012)
\lineto(247.333625,192.0880012)
\curveto(247.333625,190.86400243)(247.26162327,190.81600108)(245.533625,190.6960012)
\lineto(245.533625,190.0000012)
\lineto(251.389625,190.0000012)
\lineto(251.389625,190.6960012)
\curveto(249.6856267,190.81600108)(249.589625,190.86400243)(249.589625,192.0880012)
\lineto(249.589625,197.7040012)
\curveto(249.589625,198.3040006)(249.63762519,198.64000159)(249.829625,199.0240012)
\curveto(250.3336245,199.96000027)(251.41362627,200.7520012)(252.685625,200.7520012)
\curveto(254.31762337,200.7520012)(255.205625,199.83999911)(255.205625,197.7520012)
\lineto(255.205625,192.0880012)
\curveto(255.205625,190.86400243)(255.13362327,190.81600108)(253.405625,190.6960012)
\lineto(253.405625,190.0000012)
\lineto(259.261625,190.0000012)
\lineto(259.261625,190.6960012)
\curveto(257.5576267,190.81600108)(257.461625,190.86400243)(257.461625,192.0880012)
\lineto(257.461625,198.2320012)
}
}
{
\newrgbcolor{curcolor}{0 0 0}
\pscustom[linestyle=none,fillstyle=solid,fillcolor=curcolor]
{
}
}
{
\newrgbcolor{curcolor}{0 0 0}
\pscustom[linestyle=none,fillstyle=solid,fillcolor=curcolor]
{
\newpath
\moveto(273.247625,193.4320012)
\curveto(273.247625,191.48800315)(271.75962406,190.9840012)(270.823625,190.9840012)
\curveto(269.33562649,190.9840012)(268.615625,192.04000262)(268.615625,193.4560012)
\curveto(268.615625,194.58400007)(269.14362634,195.16000171)(270.487625,195.6640012)
\curveto(271.44762404,196.02400084)(272.71962553,196.48000154)(273.247625,196.8160012)
\lineto(273.247625,193.4320012)
\moveto(275.407625,198.8560012)
\curveto(275.407625,200.24799981)(275.09562138,202.0480012)(271.471625,202.0480012)
\curveto(268.75962771,202.0480012)(266.623625,200.63199988)(266.623625,199.3120012)
\curveto(266.623625,198.54400197)(267.51162546,198.1840012)(267.967625,198.1840012)
\curveto(268.4716245,198.1840012)(268.61562512,198.44800161)(268.735625,198.8560012)
\curveto(269.26362447,200.63199943)(270.24762596,201.2080012)(271.207625,201.2080012)
\curveto(272.14362406,201.2080012)(273.247625,200.72799928)(273.247625,198.8080012)
\lineto(273.247625,197.8000012)
\curveto(272.6476256,197.17600183)(270.3196231,196.5760006)(268.423625,195.9760012)
\curveto(266.69562673,195.44800173)(266.191625,194.24800007)(266.191625,193.1200012)
\curveto(266.191625,191.320003)(267.39162726,189.7120012)(269.647625,189.7120012)
\curveto(271.13562351,189.76000115)(272.45562577,190.64800171)(273.223625,191.1520012)
\curveto(273.55962466,190.26400209)(273.94362584,189.7120012)(274.783625,189.7120012)
\curveto(275.67162411,189.7120012)(276.70362591,189.97600166)(277.615625,190.4320012)
\lineto(277.471625,191.0080012)
\curveto(277.13562534,190.93600127)(276.60762464,190.8880013)(276.247625,190.9840012)
\curveto(275.81562543,191.08000111)(275.407625,191.53600271)(275.407625,193.0480012)
\lineto(275.407625,198.8560012)
}
}
{
\newrgbcolor{curcolor}{0 0 0}
\pscustom[linestyle=none,fillstyle=solid,fillcolor=curcolor]
{
\newpath
\moveto(291.762125,190.0000012)
\lineto(291.762125,190.6960012)
\curveto(290.10612666,190.81600108)(289.962125,190.86400243)(289.962125,192.0880012)
\lineto(289.962125,201.9040012)
\lineto(289.794125,202.0480012)
\lineto(286.026125,201.4480012)
\lineto(286.026125,200.8480012)
\lineto(286.938125,200.7280012)
\curveto(287.56212438,200.6320013)(287.706125,200.51200034)(287.706125,199.6480012)
\lineto(287.706125,194.2480012)
\curveto(287.706125,193.57600187)(287.63412488,193.04800096)(287.514125,192.8080012)
\curveto(286.96212555,191.70400231)(285.81012375,191.0080012)(284.562125,191.0080012)
\curveto(283.17012639,191.0080012)(282.138125,191.87200307)(282.138125,193.7440012)
\lineto(282.138125,201.9040012)
\lineto(281.970125,202.0480012)
\lineto(278.202125,201.4480012)
\lineto(278.202125,200.8480012)
\lineto(279.114125,200.7280012)
\curveto(279.73812438,200.6320013)(279.882125,200.51200034)(279.882125,199.6480012)
\lineto(279.882125,193.3120012)
\curveto(279.882125,190.55200396)(281.51412678,189.7120012)(283.290125,189.7120012)
\curveto(285.33012296,189.7120012)(286.9861257,191.10400154)(287.682125,191.4400012)
\lineto(287.922125,190.0000012)
\lineto(291.762125,190.0000012)
}
}
{
\newrgbcolor{curcolor}{0 0 0}
\pscustom[linestyle=none,fillstyle=solid,fillcolor=curcolor]
{
\newpath
\moveto(299.933375,200.8000012)
\lineto(299.933375,201.7600012)
\lineto(296.837375,201.7600012)
\lineto(296.837375,204.7840012)
\lineto(296.141375,204.7840012)
\lineto(294.677375,201.7600012)
\lineto(292.853375,201.7600012)
\lineto(292.853375,200.8000012)
\lineto(294.581375,200.8000012)
\lineto(294.581375,192.4480012)
\curveto(294.581375,190.00000365)(296.40537584,189.7120012)(297.245375,189.7120012)
\curveto(298.46937378,189.7120012)(299.6213757,190.33600163)(300.317375,190.7680012)
\lineto(300.125375,191.2720012)
\curveto(299.54937558,191.03200144)(299.0213744,190.9600012)(298.421375,190.9600012)
\curveto(297.60537582,190.9600012)(296.837375,191.53600303)(296.837375,193.3600012)
\lineto(296.837375,200.8000012)
\lineto(299.933375,200.8000012)
}
}
{
\newrgbcolor{curcolor}{0 0 0}
\pscustom[linestyle=none,fillstyle=solid,fillcolor=curcolor]
{
\newpath
\moveto(307.271375,202.0480012)
\curveto(303.43137884,202.0480012)(301.535375,199.76799731)(301.535375,195.8800012)
\curveto(301.535375,191.99200509)(303.43137884,189.7120012)(307.271375,189.7120012)
\curveto(311.15937111,189.7120012)(313.031375,191.99200509)(313.031375,195.8800012)
\curveto(313.031375,199.76799731)(311.15937111,202.0480012)(307.271375,202.0480012)
\moveto(304.031375,195.8800012)
\curveto(304.031375,199.19199789)(305.08737718,201.2080012)(307.271375,201.2080012)
\curveto(309.50337277,201.2080012)(310.535375,199.19199789)(310.535375,195.8800012)
\curveto(310.535375,192.56800451)(309.50337277,190.5520012)(307.271375,190.5520012)
\curveto(305.08737718,190.5520012)(304.031375,192.56800451)(304.031375,195.8800012)
}
}
{
\newrgbcolor{curcolor}{1 1 1}
\pscustom[linestyle=none,fillstyle=solid,fillcolor=curcolor]
{
\newpath
\moveto(330,210.0000012)
\lineto(350,210.0000012)
\lineto(350,190.0000012)
\lineto(330,190.0000012)
\lineto(330,210.0000012)
\closepath
}
}
{
\newrgbcolor{curcolor}{0 0 0}
\pscustom[linewidth=2,linecolor=curcolor]
{
\newpath
\moveto(330,210.0000012)
\lineto(350,210.0000012)
\lineto(350,190.0000012)
\lineto(330,190.0000012)
\lineto(330,210.0000012)
\closepath
}
}
{
\newrgbcolor{curcolor}{0 0 0}
\pscustom[linewidth=2,linecolor=curcolor,linestyle=dashed,dash=8 8]
{
\newpath
\moveto(340,200)
\lineto(500,200)
}
}
{
\newrgbcolor{curcolor}{0 0 0}
\pscustom[linestyle=none,fillstyle=solid,fillcolor=curcolor]
{
\newpath
\moveto(350.46230536,195.15951776)
\lineto(337.3512526,199.98078409)
\lineto(350.46230608,204.80204936)
\curveto(348.367708,201.95557628)(348.37977712,198.06110708)(350.46230536,195.15951776)
\lineto(350.46230536,195.15951776)
\lineto(350.46230536,195.15951776)
\closepath
}
}
{
\newrgbcolor{curcolor}{0 0 0}
\pscustom[linestyle=none,fillstyle=solid,fillcolor=curcolor]
{
\newpath
\moveto(47.328,118.5760012)
\curveto(49.31199802,118.5760012)(50.97600109,119.69600277)(52.064,121.2640012)
\lineto(52.128,121.2640012)
\curveto(52.128,119.24800322)(51.96799942,116.65599899)(51.392,114.4480012)
\curveto(50.81600058,112.33600331)(49.69599795,110.7360012)(47.648,110.7360012)
\curveto(45.47200218,110.7360012)(44.67199952,112.43200271)(44.192,113.9360012)
\curveto(44.06400013,114.35200079)(43.83999946,114.6080012)(43.296,114.6080012)
\curveto(42.68800061,114.6080012)(41.792,113.96800021)(41.792,112.9760012)
\curveto(41.792,111.56800261)(43.48800387,109.6160012)(47.36,109.6160012)
\curveto(49.95199741,109.6160012)(51.93600122,110.5760029)(53.152,112.2720012)
\curveto(54.46399869,114.09599938)(55.264,116.91200584)(55.264,121.5520012)
\curveto(55.264,125.51999723)(54.46399875,128.11200271)(53.216,129.6160012)
\curveto(52.00000122,131.05599976)(50.23999795,131.7600012)(48.192,131.7600012)
\curveto(44.0960041,131.7600012)(41.152,129.03999679)(41.152,124.6240012)
\curveto(41.152,120.6240052)(44.0320033,118.5760012)(47.328,118.5760012)
\moveto(48.288,120.2720012)
\curveto(45.7920025,120.2720012)(44.352,122.44800427)(44.352,125.5200012)
\curveto(44.352,128.46399826)(45.6640025,130.6400012)(48.16,130.6400012)
\curveto(50.71999744,130.6400012)(51.968,127.5999981)(51.968,124.4960012)
\curveto(51.968,123.95200175)(51.93599981,123.31200069)(51.744,122.8000012)
\curveto(51.23200051,121.29600271)(49.9839983,120.2720012)(48.288,120.2720012)
}
}
{
\newrgbcolor{curcolor}{0 0 0}
\pscustom[linestyle=none,fillstyle=solid,fillcolor=curcolor]
{
\newpath
\moveto(136.504,150.0000012)
\lineto(136.504,150.6960012)
\curveto(134.94400156,150.81600108)(134.36799986,150.88800221)(134.224,151.8960012)
\curveto(134.10400012,152.68800041)(134.056,153.60000247)(134.056,154.8720012)
\lineto(134.056,164.4000012)
\lineto(134.104,164.4000012)
\lineto(140.296,150.2880012)
\lineto(140.944,150.2880012)
\lineto(147.208,164.4000012)
\lineto(147.28,164.4000012)
\lineto(147.28,152.2320012)
\curveto(147.28,150.88800255)(147.23199789,150.84000106)(145.12,150.6960012)
\lineto(145.12,150.0000012)
\lineto(151.816,150.0000012)
\lineto(151.816,150.6960012)
\curveto(149.72800209,150.84000106)(149.656,150.88800255)(149.656,152.2320012)
\lineto(149.656,164.3760012)
\curveto(149.656,165.71999986)(149.72800209,165.76800135)(151.816,165.9120012)
\lineto(151.816,166.6080012)
\lineto(146.992,166.6080012)
\curveto(146.17600082,164.47200334)(145.19199904,162.35999909)(144.232,160.2480012)
\lineto(141.352,153.8160012)
\lineto(141.304,153.8160012)
\lineto(138.472,160.2240012)
\curveto(137.53600094,162.35999907)(136.52799914,164.47200334)(135.664,166.6080012)
\lineto(130.792,166.6080012)
\lineto(130.792,165.9120012)
\curveto(132.90399789,165.76800135)(132.952,165.71999986)(132.952,164.3760012)
\lineto(132.952,154.8720012)
\curveto(132.952,153.60000247)(132.90399988,152.68800041)(132.784,151.8960012)
\curveto(132.64000014,150.88800221)(132.06399875,150.79200111)(130.816,150.6960012)
\lineto(130.816,150.0000012)
\lineto(136.504,150.0000012)
}
}
{
\newrgbcolor{curcolor}{0 0 0}
\pscustom[linestyle=none,fillstyle=solid,fillcolor=curcolor]
{
\newpath
\moveto(163.224625,156.3120012)
\curveto(163.7286245,156.3120012)(164.184625,156.43200199)(164.184625,157.2240012)
\curveto(164.184625,158.63999979)(163.72862061,162.0480012)(159.336625,162.0480012)
\curveto(155.59262874,162.0480012)(153.816625,159.3839976)(153.816625,155.7840012)
\curveto(153.816625,151.99200499)(155.44862891,149.66400125)(159.360625,149.7120012)
\curveto(162.02462234,149.73600118)(163.39262565,151.2000031)(164.040625,153.0960012)
\lineto(163.320625,153.4800012)
\curveto(162.64862567,152.08800259)(161.7366231,150.9360012)(159.840625,150.9360012)
\curveto(156.86462798,150.9360012)(156.26462505,153.8160037)(156.312625,156.3120012)
\lineto(163.224625,156.3120012)
\moveto(156.336625,157.2240012)
\curveto(156.336625,158.20800022)(156.6966275,161.2080012)(159.192625,161.2080012)
\curveto(161.42462277,161.2080012)(161.688625,158.92800027)(161.688625,157.9920012)
\curveto(161.688625,157.53600166)(161.54462433,157.2240012)(160.872625,157.2240012)
\lineto(156.336625,157.2240012)
\moveto(157.824625,163.0080012)
\lineto(161.448625,165.0240012)
\curveto(162.16862428,165.40800082)(162.792625,165.81600171)(162.792625,166.3200012)
\curveto(162.792625,166.80000072)(162.21662464,167.3520012)(161.856625,167.3520012)
\curveto(161.44862541,167.3520012)(161.04062428,167.11200046)(160.320625,166.3680012)
\lineto(157.464625,163.4880012)
\lineto(157.824625,163.0080012)
}
}
{
\newrgbcolor{curcolor}{0 0 0}
\pscustom[linestyle=none,fillstyle=solid,fillcolor=curcolor]
{
\newpath
\moveto(165.85675,161.4480012)
\lineto(165.85675,160.8480012)
\lineto(166.76875,160.7280012)
\curveto(167.39274938,160.6320013)(167.53675,160.51200034)(167.53675,159.6480012)
\lineto(167.53675,152.0880012)
\curveto(167.53675,150.86400243)(167.46474827,150.81600108)(165.73675,150.6960012)
\lineto(165.73675,150.0000012)
\lineto(171.59275,150.0000012)
\lineto(171.59275,150.6960012)
\curveto(169.8887517,150.81600108)(169.79275,150.86400243)(169.79275,152.0880012)
\lineto(169.79275,157.7040012)
\curveto(169.79275,158.3040006)(169.81675017,158.64000154)(169.98475,158.9760012)
\curveto(170.44074954,159.93600024)(171.49675132,160.7520012)(172.81675,160.7520012)
\curveto(174.88074794,160.7520012)(175.12075,159.23999971)(175.12075,157.7520012)
\lineto(175.12075,152.0880012)
\curveto(175.12075,150.86400243)(175.04874827,150.81600108)(173.32075,150.6960012)
\lineto(173.32075,150.0000012)
\lineto(179.17675,150.0000012)
\lineto(179.17675,150.6960012)
\curveto(177.4727517,150.81600108)(177.37675,150.86400243)(177.37675,152.0880012)
\lineto(177.37675,157.7040012)
\curveto(177.37675,158.3040006)(177.40075014,158.64000159)(177.54475,159.0240012)
\curveto(177.90474964,159.98400024)(178.93675134,160.7520012)(180.28075,160.7520012)
\curveto(181.79274849,160.80000115)(182.70475,159.83999911)(182.70475,157.7520012)
\lineto(182.70475,152.0880012)
\curveto(182.70475,150.86400243)(182.63274827,150.81600108)(180.90475,150.6960012)
\lineto(180.90475,150.0000012)
\lineto(186.76075,150.0000012)
\lineto(186.76075,150.6960012)
\curveto(185.0567517,150.81600108)(184.96075,150.86400243)(184.96075,152.0880012)
\lineto(184.96075,158.2320012)
\curveto(184.96075,160.65599878)(183.66474772,162.0480012)(181.38475,162.0480012)
\curveto(179.41675197,162.0480012)(178.00074916,160.72800067)(177.16075,160.2000012)
\curveto(176.58475058,161.32800007)(175.67274832,162.0480012)(173.99275,162.0480012)
\curveto(172.14475185,162.0480012)(170.63274916,160.84800058)(169.79275,160.2240012)
\lineto(169.57675,162.0480012)
\lineto(165.85675,161.4480012)
}
}
{
\newrgbcolor{curcolor}{0 0 0}
\pscustom[linestyle=none,fillstyle=solid,fillcolor=curcolor]
{
\newpath
\moveto(194.052625,162.0480012)
\curveto(190.21262884,162.0480012)(188.316625,159.76799731)(188.316625,155.8800012)
\curveto(188.316625,151.99200509)(190.21262884,149.7120012)(194.052625,149.7120012)
\curveto(197.94062111,149.7120012)(199.812625,151.99200509)(199.812625,155.8800012)
\curveto(199.812625,159.76799731)(197.94062111,162.0480012)(194.052625,162.0480012)
\moveto(190.812625,155.8800012)
\curveto(190.812625,159.19199789)(191.86862718,161.2080012)(194.052625,161.2080012)
\curveto(196.28462277,161.2080012)(197.316625,159.19199789)(197.316625,155.8800012)
\curveto(197.316625,152.56800451)(196.28462277,150.5520012)(194.052625,150.5520012)
\curveto(191.86862718,150.5520012)(190.812625,152.56800451)(190.812625,155.8800012)
}
}
{
\newrgbcolor{curcolor}{0 0 0}
\pscustom[linestyle=none,fillstyle=solid,fillcolor=curcolor]
{
\newpath
\moveto(205.50475,162.0480012)
\lineto(201.85675,161.4480012)
\lineto(201.85675,160.8480012)
\lineto(202.76875,160.7280012)
\curveto(203.39274938,160.6320013)(203.53675,160.51200034)(203.53675,159.6480012)
\lineto(203.53675,152.0880012)
\curveto(203.53675,150.86400243)(203.41674832,150.81600108)(201.73675,150.6960012)
\lineto(201.73675,150.0000012)
\lineto(208.07275,150.0000012)
\lineto(208.07275,150.6960012)
\curveto(205.93675214,150.81600108)(205.79275,150.86400243)(205.79275,152.0880012)
\lineto(205.79275,157.7040012)
\curveto(205.79275,159.55199935)(206.60875055,160.1760012)(207.16075,160.1760012)
\curveto(207.54474962,160.1760012)(207.9527507,160.03200082)(208.64875,159.6480012)
\curveto(208.81674983,159.5520013)(209.00875012,159.5280012)(209.12875,159.5280012)
\curveto(209.70474942,159.5280012)(210.23275,160.12800195)(210.23275,160.8720012)
\curveto(210.23275,161.40000067)(209.89674906,162.0480012)(208.96075,162.0480012)
\curveto(208.09675086,162.0480012)(207.37674842,161.51999981)(205.79275,160.1280012)
\lineto(205.50475,162.0480012)
}
}
{
\newrgbcolor{curcolor}{0 0 0}
\pscustom[linestyle=none,fillstyle=solid,fillcolor=curcolor]
{
\newpath
\moveto(211.184875,161.4480012)
\lineto(211.184875,160.8480012)
\lineto(212.096875,160.7280012)
\curveto(212.72087438,160.6320013)(212.864875,160.51200034)(212.864875,159.6480012)
\lineto(212.864875,152.0880012)
\curveto(212.864875,150.86400243)(212.79287327,150.81600108)(211.064875,150.6960012)
\lineto(211.064875,150.0000012)
\lineto(216.920875,150.0000012)
\lineto(216.920875,150.6960012)
\curveto(215.2168767,150.81600108)(215.120875,150.86400243)(215.120875,152.0880012)
\lineto(215.120875,161.9040012)
\lineto(214.952875,162.0480012)
\lineto(211.184875,161.4480012)
\moveto(213.968875,167.1600012)
\curveto(213.10487586,167.1600012)(212.504875,166.53600034)(212.504875,165.6720012)
\curveto(212.504875,164.83200204)(213.10487586,164.2320012)(213.968875,164.2320012)
\curveto(214.85687411,164.2320012)(215.40887502,164.83200204)(215.432875,165.6720012)
\curveto(215.432875,166.53600034)(214.85687411,167.1600012)(213.968875,167.1600012)
}
}
{
\newrgbcolor{curcolor}{0 0 0}
\pscustom[linestyle=none,fillstyle=solid,fillcolor=curcolor]
{
\newpath
\moveto(226.08925,158.4240012)
\lineto(226.08925,161.2320012)
\curveto(225.29725079,161.80800063)(224.07324892,162.0480012)(222.99325,162.0480012)
\curveto(220.40125259,162.0480012)(218.64924998,160.82399892)(218.62525,158.5440012)
\curveto(218.64924998,156.55200319)(220.25725199,155.6640006)(222.24925,155.0640012)
\curveto(223.32924892,154.72800154)(224.84125,154.17599957)(224.84125,152.5440012)
\curveto(224.84125,151.32000243)(223.88124873,150.5520012)(222.60925,150.5520012)
\curveto(220.66525194,150.5520012)(219.58524952,151.96800324)(219.10525,154.0080012)
\lineto(218.40925,154.0080012)
\lineto(218.64925,150.8640012)
\curveto(219.51324914,150.09600197)(220.97725142,149.7120012)(222.39325,149.7120012)
\curveto(225.24924714,149.7120012)(226.88125,151.22400324)(226.88125,153.2640012)
\curveto(226.88125,155.37599909)(225.58524753,156.33600197)(223.11325,157.1040012)
\curveto(222.12925098,157.41600089)(220.54525,157.92000255)(220.54525,159.2640012)
\curveto(220.56924998,160.51199995)(221.5052512,161.2080012)(222.70525,161.2080012)
\curveto(224.4092483,161.2080012)(225.22525017,159.88799974)(225.39325,158.4240012)
\lineto(226.08925,158.4240012)
}
}
{
\newrgbcolor{curcolor}{0 0 0}
\pscustom[linestyle=none,fillstyle=solid,fillcolor=curcolor]
{
\newpath
\moveto(238.318375,156.3120012)
\curveto(238.8223745,156.3120012)(239.278375,156.43200199)(239.278375,157.2240012)
\curveto(239.278375,158.63999979)(238.82237061,162.0480012)(234.430375,162.0480012)
\curveto(230.68637874,162.0480012)(228.910375,159.3839976)(228.910375,155.7840012)
\curveto(228.910375,151.99200499)(230.54237891,149.66400125)(234.454375,149.7120012)
\curveto(237.11837234,149.73600118)(238.48637565,151.2000031)(239.134375,153.0960012)
\lineto(238.414375,153.4800012)
\curveto(237.74237567,152.08800259)(236.8303731,150.9360012)(234.934375,150.9360012)
\curveto(231.95837798,150.9360012)(231.35837505,153.8160037)(231.406375,156.3120012)
\lineto(238.318375,156.3120012)
\moveto(231.430375,157.2240012)
\curveto(231.430375,158.20800022)(231.7903775,161.2080012)(234.286375,161.2080012)
\curveto(236.51837277,161.2080012)(236.782375,158.92800027)(236.782375,157.9920012)
\curveto(236.782375,157.53600166)(236.63837433,157.2240012)(235.966375,157.2240012)
\lineto(231.430375,157.2240012)
}
}
{
\newrgbcolor{curcolor}{0 0 0}
\pscustom[linestyle=none,fillstyle=solid,fillcolor=curcolor]
{
\newpath
\moveto(244.5985,162.0480012)
\lineto(240.9505,161.4480012)
\lineto(240.9505,160.8480012)
\lineto(241.8625,160.7280012)
\curveto(242.48649938,160.6320013)(242.6305,160.51200034)(242.6305,159.6480012)
\lineto(242.6305,152.0880012)
\curveto(242.6305,150.86400243)(242.51049832,150.81600108)(240.8305,150.6960012)
\lineto(240.8305,150.0000012)
\lineto(247.1665,150.0000012)
\lineto(247.1665,150.6960012)
\curveto(245.03050214,150.81600108)(244.8865,150.86400243)(244.8865,152.0880012)
\lineto(244.8865,157.7040012)
\curveto(244.8865,159.55199935)(245.70250055,160.1760012)(246.2545,160.1760012)
\curveto(246.63849962,160.1760012)(247.0465007,160.03200082)(247.7425,159.6480012)
\curveto(247.91049983,159.5520013)(248.10250012,159.5280012)(248.2225,159.5280012)
\curveto(248.79849942,159.5280012)(249.3265,160.12800195)(249.3265,160.8720012)
\curveto(249.3265,161.40000067)(248.99049906,162.0480012)(248.0545,162.0480012)
\curveto(247.19050086,162.0480012)(246.47049842,161.51999981)(244.8865,160.1280012)
\lineto(244.5985,162.0480012)
}
}
{
\newrgbcolor{curcolor}{0 0 0}
\pscustom[linestyle=none,fillstyle=solid,fillcolor=curcolor]
{
}
}
{
\newrgbcolor{curcolor}{0 0 0}
\pscustom[linestyle=none,fillstyle=solid,fillcolor=curcolor]
{
\newpath
\moveto(255.66925,161.4480012)
\lineto(255.66925,160.8480012)
\lineto(256.58125,160.7280012)
\curveto(257.20524938,160.6320013)(257.34925,160.51200034)(257.34925,159.6480012)
\lineto(257.34925,152.0880012)
\curveto(257.34925,150.86400243)(257.27724827,150.81600108)(255.54925,150.6960012)
\lineto(255.54925,150.0000012)
\lineto(261.40525,150.0000012)
\lineto(261.40525,150.6960012)
\curveto(259.7012517,150.81600108)(259.60525,150.86400243)(259.60525,152.0880012)
\lineto(259.60525,157.7040012)
\curveto(259.60525,158.3040006)(259.62925017,158.64000154)(259.79725,158.9760012)
\curveto(260.25324954,159.93600024)(261.30925132,160.7520012)(262.62925,160.7520012)
\curveto(264.69324794,160.7520012)(264.93325,159.23999971)(264.93325,157.7520012)
\lineto(264.93325,152.0880012)
\curveto(264.93325,150.86400243)(264.86124827,150.81600108)(263.13325,150.6960012)
\lineto(263.13325,150.0000012)
\lineto(268.98925,150.0000012)
\lineto(268.98925,150.6960012)
\curveto(267.2852517,150.81600108)(267.18925,150.86400243)(267.18925,152.0880012)
\lineto(267.18925,157.7040012)
\curveto(267.18925,158.3040006)(267.21325014,158.64000159)(267.35725,159.0240012)
\curveto(267.71724964,159.98400024)(268.74925134,160.7520012)(270.09325,160.7520012)
\curveto(271.60524849,160.80000115)(272.51725,159.83999911)(272.51725,157.7520012)
\lineto(272.51725,152.0880012)
\curveto(272.51725,150.86400243)(272.44524827,150.81600108)(270.71725,150.6960012)
\lineto(270.71725,150.0000012)
\lineto(276.57325,150.0000012)
\lineto(276.57325,150.6960012)
\curveto(274.8692517,150.81600108)(274.77325,150.86400243)(274.77325,152.0880012)
\lineto(274.77325,158.2320012)
\curveto(274.77325,160.65599878)(273.47724772,162.0480012)(271.19725,162.0480012)
\curveto(269.22925197,162.0480012)(267.81324916,160.72800067)(266.97325,160.2000012)
\curveto(266.39725058,161.32800007)(265.48524832,162.0480012)(263.80525,162.0480012)
\curveto(261.95725185,162.0480012)(260.44524916,160.84800058)(259.60525,160.2240012)
\lineto(259.38925,162.0480012)
\lineto(255.66925,161.4480012)
}
}
{
\newrgbcolor{curcolor}{0 0 0}
\pscustom[linestyle=none,fillstyle=solid,fillcolor=curcolor]
{
\newpath
\moveto(283.865125,162.0480012)
\curveto(280.02512884,162.0480012)(278.129125,159.76799731)(278.129125,155.8800012)
\curveto(278.129125,151.99200509)(280.02512884,149.7120012)(283.865125,149.7120012)
\curveto(287.75312111,149.7120012)(289.625125,151.99200509)(289.625125,155.8800012)
\curveto(289.625125,159.76799731)(287.75312111,162.0480012)(283.865125,162.0480012)
\moveto(280.625125,155.8800012)
\curveto(280.625125,159.19199789)(281.68112718,161.2080012)(283.865125,161.2080012)
\curveto(286.09712277,161.2080012)(287.129125,159.19199789)(287.129125,155.8800012)
\curveto(287.129125,152.56800451)(286.09712277,150.5520012)(283.865125,150.5520012)
\curveto(281.68112718,150.5520012)(280.625125,152.56800451)(280.625125,155.8800012)
}
}
{
\newrgbcolor{curcolor}{0 0 0}
\pscustom[linestyle=none,fillstyle=solid,fillcolor=curcolor]
{
\newpath
\moveto(298.60525,160.8000012)
\lineto(298.60525,161.7600012)
\lineto(295.50925,161.7600012)
\lineto(295.50925,164.7840012)
\lineto(294.81325,164.7840012)
\lineto(293.34925,161.7600012)
\lineto(291.52525,161.7600012)
\lineto(291.52525,160.8000012)
\lineto(293.25325,160.8000012)
\lineto(293.25325,152.4480012)
\curveto(293.25325,150.00000365)(295.07725084,149.7120012)(295.91725,149.7120012)
\curveto(297.14124878,149.7120012)(298.2932507,150.33600163)(298.98925,150.7680012)
\lineto(298.79725,151.2720012)
\curveto(298.22125058,151.03200144)(297.6932494,150.9600012)(297.09325,150.9600012)
\curveto(296.27725082,150.9600012)(295.50925,151.53600303)(295.50925,153.3600012)
\lineto(295.50925,160.8000012)
\lineto(298.60525,160.8000012)
}
}
{
\newrgbcolor{curcolor}{0 0 0}
\pscustom[linestyle=none,fillstyle=solid,fillcolor=curcolor]
{
}
}
{
\newrgbcolor{curcolor}{0 0 0}
\pscustom[linestyle=none,fillstyle=solid,fillcolor=curcolor]
{
\newpath
\moveto(318.413875,150.0000012)
\lineto(318.413875,150.6960012)
\curveto(316.7098767,150.81600108)(316.613875,150.86400243)(316.613875,152.0880012)
\lineto(316.613875,167.6640012)
\lineto(316.445875,167.8080012)
\lineto(312.677875,167.2080012)
\lineto(312.677875,166.6080012)
\lineto(313.589875,166.5120012)
\curveto(314.21387438,166.44000127)(314.357875,166.32000029)(314.357875,165.4080012)
\lineto(314.357875,161.4720012)
\curveto(313.68587567,161.83200084)(312.7978739,162.0480012)(311.693875,162.0480012)
\curveto(309.72587697,162.0480012)(308.26187399,161.42400017)(307.253875,160.3920012)
\curveto(306.22187603,159.31200228)(305.597875,157.72799911)(305.597875,155.6400012)
\curveto(305.597875,152.08800475)(307.44587817,149.7120012)(310.613875,149.7120012)
\curveto(311.9098737,149.7120012)(313.0378763,150.28800226)(314.333875,151.3440012)
\lineto(314.333875,150.1680012)
\lineto(314.525875,150.0000012)
\lineto(318.413875,150.0000012)
\moveto(314.357875,154.0800012)
\curveto(314.357875,153.55200173)(314.33387486,153.14400079)(314.189875,152.7360012)
\curveto(313.70987548,151.48800245)(312.65387375,151.0080012)(311.405875,151.0080012)
\curveto(309.19787721,151.0080012)(308.093875,153.16800396)(308.093875,155.9280012)
\curveto(308.093875,159.04799808)(309.22187745,161.2080012)(311.669875,161.2080012)
\curveto(312.79787387,161.2080012)(313.70987543,160.68000022)(314.141875,159.6960012)
\curveto(314.33387481,159.28800161)(314.357875,158.95200055)(314.357875,158.3040012)
\lineto(314.357875,154.0800012)
}
}
{
\newrgbcolor{curcolor}{0 0 0}
\pscustom[linestyle=none,fillstyle=solid,fillcolor=curcolor]
{
\newpath
\moveto(329.349625,156.3120012)
\curveto(329.8536245,156.3120012)(330.309625,156.43200199)(330.309625,157.2240012)
\curveto(330.309625,158.63999979)(329.85362061,162.0480012)(325.461625,162.0480012)
\curveto(321.71762874,162.0480012)(319.941625,159.3839976)(319.941625,155.7840012)
\curveto(319.941625,151.99200499)(321.57362891,149.66400125)(325.485625,149.7120012)
\curveto(328.14962234,149.73600118)(329.51762565,151.2000031)(330.165625,153.0960012)
\lineto(329.445625,153.4800012)
\curveto(328.77362567,152.08800259)(327.8616231,150.9360012)(325.965625,150.9360012)
\curveto(322.98962798,150.9360012)(322.38962505,153.8160037)(322.437625,156.3120012)
\lineto(329.349625,156.3120012)
\moveto(322.461625,157.2240012)
\curveto(322.461625,158.20800022)(322.8216275,161.2080012)(325.317625,161.2080012)
\curveto(327.54962277,161.2080012)(327.813625,158.92800027)(327.813625,157.9920012)
\curveto(327.813625,157.53600166)(327.66962433,157.2240012)(326.997625,157.2240012)
\lineto(322.461625,157.2240012)
}
}
{
\newrgbcolor{curcolor}{0 0 0}
\pscustom[linestyle=none,fillstyle=solid,fillcolor=curcolor]
{
}
}
{
\newrgbcolor{curcolor}{0 0 0}
\pscustom[linestyle=none,fillstyle=solid,fillcolor=curcolor]
{
\newpath
\moveto(341.140375,157.7040012)
\curveto(341.140375,158.3040006)(341.23637522,158.64000163)(341.452375,159.0720012)
\curveto(342.00437445,160.1760001)(343.03637627,160.7520012)(344.308375,160.7520012)
\curveto(345.26837404,160.7520012)(347.404375,160.12799707)(347.404375,156.0000012)
\curveto(347.404375,152.47200473)(346.3003726,150.5520012)(343.900375,150.5520012)
\curveto(342.65237625,150.5520012)(341.69237462,151.20000231)(341.308375,152.3040012)
\curveto(341.16437514,152.73600077)(341.140375,153.21600175)(341.140375,153.7680012)
\lineto(341.140375,157.7040012)
\moveto(337.204375,161.4480012)
\lineto(337.204375,160.8480012)
\lineto(338.116375,160.7280012)
\curveto(338.74037438,160.6320013)(338.884375,160.51200034)(338.884375,159.6480012)
\lineto(338.884375,146.5680012)
\curveto(338.884375,145.34400243)(338.76437332,145.29600108)(337.084375,145.1760012)
\lineto(337.084375,144.4800012)
\lineto(343.300375,144.4800012)
\lineto(343.300375,145.1760012)
\curveto(341.28437702,145.29600108)(341.140375,145.34400243)(341.140375,146.5680012)
\lineto(341.140375,150.2640012)
\curveto(341.6443745,149.97600149)(342.7723761,149.7120012)(343.876375,149.7120012)
\curveto(347.11637176,149.7120012)(349.900375,151.20000615)(349.900375,156.1440012)
\curveto(349.900375,157.8479995)(349.58837063,162.0480012)(345.220375,162.0480012)
\curveto(343.46837675,162.0480012)(342.07637406,160.8720006)(341.140375,160.2720012)
\lineto(340.828375,162.0480012)
\lineto(337.204375,161.4480012)
}
}
{
\newrgbcolor{curcolor}{0 0 0}
\pscustom[linestyle=none,fillstyle=solid,fillcolor=curcolor]
{
\newpath
\moveto(359.2945,153.4320012)
\curveto(359.2945,151.48800315)(357.80649906,150.9840012)(356.8705,150.9840012)
\curveto(355.38250149,150.9840012)(354.6625,152.04000262)(354.6625,153.4560012)
\curveto(354.6625,154.58400007)(355.19050134,155.16000171)(356.5345,155.6640012)
\curveto(357.49449904,156.02400084)(358.76650053,156.48000154)(359.2945,156.8160012)
\lineto(359.2945,153.4320012)
\moveto(361.4545,158.8560012)
\curveto(361.4545,160.24799981)(361.14249638,162.0480012)(357.5185,162.0480012)
\curveto(354.80650271,162.0480012)(352.6705,160.63199988)(352.6705,159.3120012)
\curveto(352.6705,158.54400197)(353.55850046,158.1840012)(354.0145,158.1840012)
\curveto(354.5184995,158.1840012)(354.66250012,158.44800161)(354.7825,158.8560012)
\curveto(355.31049947,160.63199943)(356.29450096,161.2080012)(357.2545,161.2080012)
\curveto(358.19049906,161.2080012)(359.2945,160.72799928)(359.2945,158.8080012)
\lineto(359.2945,157.8000012)
\curveto(358.6945006,157.17600183)(356.3664981,156.5760006)(354.4705,155.9760012)
\curveto(352.74250173,155.44800173)(352.2385,154.24800007)(352.2385,153.1200012)
\curveto(352.2385,151.320003)(353.43850226,149.7120012)(355.6945,149.7120012)
\curveto(357.18249851,149.76000115)(358.50250077,150.64800171)(359.2705,151.1520012)
\curveto(359.60649966,150.26400209)(359.99050084,149.7120012)(360.8305,149.7120012)
\curveto(361.71849911,149.7120012)(362.75050091,149.97600166)(363.6625,150.4320012)
\lineto(363.5185,151.0080012)
\curveto(363.18250034,150.93600127)(362.65449964,150.8880013)(362.2945,150.9840012)
\curveto(361.86250043,151.08000111)(361.4545,151.53600271)(361.4545,153.0480012)
\lineto(361.4545,158.8560012)
}
}
{
\newrgbcolor{curcolor}{0 0 0}
\pscustom[linestyle=none,fillstyle=solid,fillcolor=curcolor]
{
\newpath
\moveto(372.433,158.4240012)
\lineto(372.433,161.2320012)
\curveto(371.64100079,161.80800063)(370.41699892,162.0480012)(369.337,162.0480012)
\curveto(366.74500259,162.0480012)(364.99299998,160.82399892)(364.969,158.5440012)
\curveto(364.99299998,156.55200319)(366.60100199,155.6640006)(368.593,155.0640012)
\curveto(369.67299892,154.72800154)(371.185,154.17599957)(371.185,152.5440012)
\curveto(371.185,151.32000243)(370.22499873,150.5520012)(368.953,150.5520012)
\curveto(367.00900194,150.5520012)(365.92899952,151.96800324)(365.449,154.0080012)
\lineto(364.753,154.0080012)
\lineto(364.993,150.8640012)
\curveto(365.85699914,150.09600197)(367.32100142,149.7120012)(368.737,149.7120012)
\curveto(371.59299714,149.7120012)(373.225,151.22400324)(373.225,153.2640012)
\curveto(373.225,155.37599909)(371.92899753,156.33600197)(369.457,157.1040012)
\curveto(368.47300098,157.41600089)(366.889,157.92000255)(366.889,159.2640012)
\curveto(366.91299998,160.51199995)(367.8490012,161.2080012)(369.049,161.2080012)
\curveto(370.7529983,161.2080012)(371.56900017,159.88799974)(371.737,158.4240012)
\lineto(372.433,158.4240012)
}
}
{
\newrgbcolor{curcolor}{0 0 0}
\pscustom[linestyle=none,fillstyle=solid,fillcolor=curcolor]
{
\newpath
\moveto(382.886125,158.4240012)
\lineto(382.886125,161.2320012)
\curveto(382.09412579,161.80800063)(380.87012392,162.0480012)(379.790125,162.0480012)
\curveto(377.19812759,162.0480012)(375.44612498,160.82399892)(375.422125,158.5440012)
\curveto(375.44612498,156.55200319)(377.05412699,155.6640006)(379.046125,155.0640012)
\curveto(380.12612392,154.72800154)(381.638125,154.17599957)(381.638125,152.5440012)
\curveto(381.638125,151.32000243)(380.67812373,150.5520012)(379.406125,150.5520012)
\curveto(377.46212694,150.5520012)(376.38212452,151.96800324)(375.902125,154.0080012)
\lineto(375.206125,154.0080012)
\lineto(375.446125,150.8640012)
\curveto(376.31012414,150.09600197)(377.77412642,149.7120012)(379.190125,149.7120012)
\curveto(382.04612214,149.7120012)(383.678125,151.22400324)(383.678125,153.2640012)
\curveto(383.678125,155.37599909)(382.38212253,156.33600197)(379.910125,157.1040012)
\curveto(378.92612598,157.41600089)(377.342125,157.92000255)(377.342125,159.2640012)
\curveto(377.36612498,160.51199995)(378.3021262,161.2080012)(379.502125,161.2080012)
\curveto(381.2061233,161.2080012)(382.02212517,159.88799974)(382.190125,158.4240012)
\lineto(382.886125,158.4240012)
}
}
{
\newrgbcolor{curcolor}{0 0 0}
\pscustom[linestyle=none,fillstyle=solid,fillcolor=curcolor]
{
\newpath
\moveto(395.11525,156.3120012)
\curveto(395.6192495,156.3120012)(396.07525,156.43200199)(396.07525,157.2240012)
\curveto(396.07525,158.63999979)(395.61924561,162.0480012)(391.22725,162.0480012)
\curveto(387.48325374,162.0480012)(385.70725,159.3839976)(385.70725,155.7840012)
\curveto(385.70725,151.99200499)(387.33925391,149.66400125)(391.25125,149.7120012)
\curveto(393.91524734,149.73600118)(395.28325065,151.2000031)(395.93125,153.0960012)
\lineto(395.21125,153.4800012)
\curveto(394.53925067,152.08800259)(393.6272481,150.9360012)(391.73125,150.9360012)
\curveto(388.75525298,150.9360012)(388.15525005,153.8160037)(388.20325,156.3120012)
\lineto(395.11525,156.3120012)
\moveto(388.22725,157.2240012)
\curveto(388.22725,158.20800022)(388.5872525,161.2080012)(391.08325,161.2080012)
\curveto(393.31524777,161.2080012)(393.57925,158.92800027)(393.57925,157.9920012)
\curveto(393.57925,157.53600166)(393.43524933,157.2240012)(392.76325,157.2240012)
\lineto(388.22725,157.2240012)
}
}
{
\newrgbcolor{curcolor}{1 1 1}
\pscustom[linestyle=none,fillstyle=solid,fillcolor=curcolor]
{
\newpath
\moveto(410,170.0000012)
\lineto(430,170.0000012)
\lineto(430,150.0000012)
\lineto(410,150.0000012)
\lineto(410,170.0000012)
\closepath
}
}
{
\newrgbcolor{curcolor}{0 0 0}
\pscustom[linewidth=2,linecolor=curcolor]
{
\newpath
\moveto(410,170.0000012)
\lineto(430,170.0000012)
\lineto(430,150.0000012)
\lineto(410,150.0000012)
\lineto(410,170.0000012)
\closepath
}
}
{
\newrgbcolor{curcolor}{0 0 0}
\pscustom[linewidth=2,linecolor=curcolor,linestyle=dashed,dash=8 8]
{
\newpath
\moveto(420,160)
\lineto(500,160)
}
}
{
\newrgbcolor{curcolor}{0 0 0}
\pscustom[linestyle=none,fillstyle=solid,fillcolor=curcolor]
{
\newpath
\moveto(430.46230536,155.15951776)
\lineto(417.3512526,159.98078409)
\lineto(430.46230608,164.80204936)
\curveto(428.367708,161.95557628)(428.37977712,158.06110708)(430.46230536,155.15951776)
\lineto(430.46230536,155.15951776)
\lineto(430.46230536,155.15951776)
\closepath
}
}
{
\newrgbcolor{curcolor}{0 0 0}
\pscustom[linestyle=none,fillstyle=solid,fillcolor=curcolor]
{
\newpath
\moveto(513.984,110.0000012)
\lineto(513.984,110.9280012)
\lineto(511.296,111.1520012)
\curveto(510.62400067,111.21600114)(510.24,111.47200245)(510.24,112.7200012)
\lineto(510.24,131.5680012)
\lineto(510.08,131.7600012)
\lineto(503.488,130.6400012)
\lineto(503.488,129.8400012)
\lineto(506.464,129.4880012)
\curveto(507.00799946,129.42400127)(507.232,129.16800027)(507.232,128.2400012)
\lineto(507.232,112.7200012)
\curveto(507.232,112.11200181)(507.13599981,111.72800098)(506.944,111.5040012)
\curveto(506.78400016,111.28000143)(506.52799965,111.18400117)(506.176,111.1520012)
\lineto(503.488,110.9280012)
\lineto(503.488,110.0000012)
\lineto(513.984,110.0000012)
}
}
{
\newrgbcolor{curcolor}{0 0 0}
\pscustom[linestyle=none,fillstyle=solid,fillcolor=curcolor]
{
\newpath
\moveto(525.4175,130.6400012)
\curveto(528.45749696,130.6400012)(529.3215,125.90399599)(529.3215,120.6880012)
\curveto(529.3215,115.47200642)(528.45749696,110.7360012)(525.4175,110.7360012)
\curveto(522.37750304,110.7360012)(521.5135,115.47200642)(521.5135,120.6880012)
\curveto(521.5135,125.90399599)(522.37750304,130.6400012)(525.4175,130.6400012)
\moveto(525.4175,131.7600012)
\curveto(520.13750528,131.7600012)(518.2495,127.0879948)(518.2495,120.6880012)
\curveto(518.2495,114.2880076)(520.13750528,109.6160012)(525.4175,109.6160012)
\curveto(530.69749472,109.6160012)(532.5855,114.2880076)(532.5855,120.6880012)
\curveto(532.5855,127.0879948)(530.69749472,131.7600012)(525.4175,131.7600012)
}
}
\end{pspicture}

		\end{center}
		
		\begin{enumerate}
		  \item Fond d'écran (présent dans l'archive stocké sur le téléphone).
		  \item Pseudonyme du compte hors ligne utilisé
		  \item Cadre contenant l'erreur survenue
		  \item Champs de texte ``Nom utilisateur''
		  \item Champs de texte ``Mot de passe''
		  \item Champs de texte ``Vérification''
		  \item Check Box ``Connexion auto''
		  \item Bouton ``\hyperlink{Connexion multi-joueurs}{Retour}''
		  \item Bouton ``\hyperlink{Accueil multi-joueurs}{Valider}''
		\end{enumerate}
		
		\subsubsection{Description des zones}
		
			\begin{tabular}{|c|c|c|c|c|} \hline
				Numéro de zone & Type  & Description & Evènement &	Règle \\\hline
				2 & Label & Affiche le pseudonyme du compte & Chargement de la page & RG4-01 \\
				  &       & hors ligne en cours d'utilisation & & \\\hline
				3 & Label & Affiche l'erreur & RG4-03 & RG4-02 \\
				  &       & rencontré par l'utilisateur & RG4-04 & \\\hline
				4 & Champs de texte & & Perte du focus & RG4-03 \\
				 & & & RG4-07 & \\\hline 
				6 & Champs de texte & & Perte du focus & RG4-04 \\
				 & & & RG4-07 & \\\hline 
				7 & Check box & Permet la connexion automatique & Cliqué & RG4-05 \\
				  &           & lors des prochaines utilisations&        & \\				
				  &           & du multi-joueurs                &        & \\\hline
				8 & Bouton & Permet de revenir à la page & Cliqué & RG4-06 \\
				  &        & de connexion multi-joueurs \footnotemark[1] & & \\\hline
				9 & Bouton & Valide les paramètres entrés & Cliqué & RG4-07 \\\hline
			\end{tabular}
			
		\subsubsection{Description des règles}

			\underline{RG4-01 :}
				\begin{quote}
					Récupérer le pseudonyme du compte hors ligne en cours d'utilisation dans la classe utilisateur.\\
					Afficher le pseudonyme.\\
				\end{quote}

			\underline{RG4-02 :}
				\begin{quote}
					Afficher l'erreur rencontrée.\\
				\end{quote}
				
			\underline{RG4-03 :}
				\begin{quote}
					Vérifier les caractères entrés.\\
					Si une erreur est rencontrée alors
					\begin{quote}
						RG4-02.
					\end{quote}
				\end{quote}
				
			\underline{RG4-04 :}
				\begin{quote}
					Vérifier que les caractères entrés soient identiques à ceux entrés dans la zone 5.\\
					Si ce n'est pas le cas alors
					\begin{quote}
						RG4-02.
					\end{quote}					
				\end{quote}	
				
			\underline{RG4-05 :}
				\begin{quote}
					
				\end{quote}
				
			\underline{RG4-06 :}
				\begin{quote}
					Afficher la page de connexion multi-joueurs%
						\footnote[1]{
							\hyperlink{Connexion multi-joueurs}{Connexion multi-joueurs}
							\og voir section \ref{Connexion multi-joueurs}, page \pageref{Connexion multi-joueurs}.\fg
						}.\\
					Supprimer la page de création d'un compte multi-joueurs.\\
				\end{quote}

			\underline{RG4-07 :}
				\begin{quote}
					RG4-03.\\
					RG4-04.\\
					Si aucune erreur n'a été détectée alors
						\begin{quote}
							Une connexion avec le serveur distant a lieu afin d'ajouter le nouveau compte.\\
							Si le serveur renvoie une erreur alors \\
							\begin{quote}
								Elle sera affichée sous la forme d'un pop-up.
							\end{quote}	
							Sinon
							\begin{quote}
								L'utilisateur sera redirigé vers l'accueil multi-joueurs%
								\footnote[2]{
									\hyperlink{Accueil multi-joueurs}{Accueil multi-joueurs}
									\og voir section \ref{Accueil multi-joueurs}, page \pageref{Accueil multi-joueurs}.\fg
								}.\\
								RG4-05.\\
								Supprimer toutes les pages existantes sauf la page d'acceuil multi-joueurs\footnotemark[2].					
							\end{quote}
						\end{quote}
				\end{quote}
	
\newpage

	\subsection{Connexion multi-joueurs}
	
		\hypertarget{Connexion multi-joueurs}{}
		\label{Connexion multi-joueurs}
	
		\begin{center}
			%LaTeX with PSTricks extensions
%%Creator: inkscape 0.48.0
%%Please note this file requires PSTricks extensions
\psset{xunit=.5pt,yunit=.5pt,runit=.5pt}
\begin{pspicture}(560,600)
{
\newrgbcolor{curcolor}{1 1 1}
\pscustom[linestyle=none,fillstyle=solid,fillcolor=curcolor]
{
\newpath
\moveto(133.12401581,597.52220317)
\lineto(426.87598419,597.52220317)
\curveto(443.85397169,597.52220317)(457.52217102,583.85400385)(457.52217102,566.87601635)
\lineto(457.52217102,33.12401744)
\curveto(457.52217102,16.14602994)(443.85397169,2.47783062)(426.87598419,2.47783062)
\lineto(133.12401581,2.47783062)
\curveto(116.14602831,2.47783062)(102.47782898,16.14602994)(102.47782898,33.12401744)
\lineto(102.47782898,566.87601635)
\curveto(102.47782898,583.85400385)(116.14602831,597.52220317)(133.12401581,597.52220317)
\closepath
}
}
{
\newrgbcolor{curcolor}{0 0 0}
\pscustom[linewidth=4.95566034,linecolor=curcolor]
{
\newpath
\moveto(133.12401581,597.52220317)
\lineto(426.87598419,597.52220317)
\curveto(443.85397169,597.52220317)(457.52217102,583.85400385)(457.52217102,566.87601635)
\lineto(457.52217102,33.12401744)
\curveto(457.52217102,16.14602994)(443.85397169,2.47783062)(426.87598419,2.47783062)
\lineto(133.12401581,2.47783062)
\curveto(116.14602831,2.47783062)(102.47782898,16.14602994)(102.47782898,33.12401744)
\lineto(102.47782898,566.87601635)
\curveto(102.47782898,583.85400385)(116.14602831,597.52220317)(133.12401581,597.52220317)
\closepath
}
}
{
\newrgbcolor{curcolor}{0 0 0}
\pscustom[linestyle=none,fillstyle=solid,fillcolor=curcolor,opacity=0]
{
\newpath
\moveto(150.43106842,457.69565174)
\lineto(407.85224152,457.69565174)
\curveto(426.47576608,457.69565174)(441.46871185,442.70270597)(441.46871185,424.0791814)
\lineto(441.46871185,133.61644336)
\curveto(441.46871185,114.9929188)(426.47576608,99.99997303)(407.85224152,99.99997303)
\lineto(150.43106842,99.99997303)
\curveto(131.80754385,99.99997303)(116.81459808,114.9929188)(116.81459808,133.61644336)
\lineto(116.81459808,424.0791814)
\curveto(116.81459808,442.70270597)(131.80754385,457.69565174)(150.43106842,457.69565174)
\closepath
}
}
{
\newrgbcolor{curcolor}{0 0 0}
\pscustom[linewidth=2.23000407,linecolor=curcolor]
{
\newpath
\moveto(150.43106842,457.69565174)
\lineto(407.85224152,457.69565174)
\curveto(426.47576608,457.69565174)(441.46871185,442.70270597)(441.46871185,424.0791814)
\lineto(441.46871185,133.61644336)
\curveto(441.46871185,114.9929188)(426.47576608,99.99997303)(407.85224152,99.99997303)
\lineto(150.43106842,99.99997303)
\curveto(131.80754385,99.99997303)(116.81459808,114.9929188)(116.81459808,133.61644336)
\lineto(116.81459808,424.0791814)
\curveto(116.81459808,442.70270597)(131.80754385,457.69565174)(150.43106842,457.69565174)
\closepath
}
}
{
\newrgbcolor{curcolor}{0 0 0}
\pscustom[linestyle=none,fillstyle=solid,fillcolor=curcolor,opacity=0.11636364]
{
\newpath
\moveto(290.35070133,330.00074932)
\lineto(419.84194088,330.00074932)
\curveto(425.53073339,330.00074932)(430.11051941,325.42096329)(430.11051941,319.73217079)
\lineto(430.11051941,310.37671157)
\curveto(430.11051941,304.68791907)(425.53073339,300.10813304)(419.84194088,300.10813304)
\lineto(290.35070133,300.10813304)
\curveto(284.66190883,300.10813304)(280.0821228,304.68791907)(280.0821228,310.37671157)
\lineto(280.0821228,319.73217079)
\curveto(280.0821228,325.42096329)(284.66190883,330.00074932)(290.35070133,330.00074932)
\closepath
}
}
{
\newrgbcolor{curcolor}{0 0 0}
\pscustom[linewidth=2,linecolor=curcolor]
{
\newpath
\moveto(290.35070133,330.00074932)
\lineto(419.84194088,330.00074932)
\curveto(425.53073339,330.00074932)(430.11051941,325.42096329)(430.11051941,319.73217079)
\lineto(430.11051941,310.37671157)
\curveto(430.11051941,304.68791907)(425.53073339,300.10813304)(419.84194088,300.10813304)
\lineto(290.35070133,300.10813304)
\curveto(284.66190883,300.10813304)(280.0821228,304.68791907)(280.0821228,310.37671157)
\lineto(280.0821228,319.73217079)
\curveto(280.0821228,325.42096329)(284.66190883,330.00074932)(290.35070133,330.00074932)
\closepath
}
}
{
\newrgbcolor{curcolor}{0 0 0}
\pscustom[linestyle=none,fillstyle=solid,fillcolor=curcolor,opacity=0.11636364]
{
\newpath
\moveto(300.28209114,280.21498271)
\lineto(419.86942863,280.21498271)
\curveto(425.52066758,280.21498271)(430.07022095,275.66542934)(430.07022095,270.0141904)
\lineto(430.07022095,260.72049495)
\curveto(430.07022095,255.06925601)(425.52066758,250.51970264)(419.86942863,250.51970264)
\lineto(300.28209114,250.51970264)
\curveto(294.6308522,250.51970264)(290.08129883,255.06925601)(290.08129883,260.72049495)
\lineto(290.08129883,270.0141904)
\curveto(290.08129883,275.66542934)(294.6308522,280.21498271)(300.28209114,280.21498271)
\closepath
}
}
{
\newrgbcolor{curcolor}{0 0 0}
\pscustom[linewidth=2,linecolor=curcolor]
{
\newpath
\moveto(300.28209114,280.21498271)
\lineto(419.86942863,280.21498271)
\curveto(425.52066758,280.21498271)(430.07022095,275.66542934)(430.07022095,270.0141904)
\lineto(430.07022095,260.72049495)
\curveto(430.07022095,255.06925601)(425.52066758,250.51970264)(419.86942863,250.51970264)
\lineto(300.28209114,250.51970264)
\curveto(294.6308522,250.51970264)(290.08129883,255.06925601)(290.08129883,260.72049495)
\lineto(290.08129883,270.0141904)
\curveto(290.08129883,275.66542934)(294.6308522,280.21498271)(300.28209114,280.21498271)
\closepath
}
}
{
\newrgbcolor{curcolor}{0 0 0}
\pscustom[linestyle=none,fillstyle=solid,fillcolor=curcolor,opacity=0]
{
\newpath
\moveto(294.88945961,140.00001689)
\lineto(415.27087975,140.00001689)
\curveto(423.51964037,140.00001689)(430.16033936,133.35931791)(430.16033936,125.11055728)
\curveto(430.16033936,116.86179666)(423.51964037,110.22109767)(415.27087975,110.22109767)
\lineto(294.88945961,110.22109767)
\curveto(286.64069899,110.22109767)(280,116.86179666)(280,125.11055728)
\curveto(280,133.35931791)(286.64069899,140.00001689)(294.88945961,140.00001689)
\closepath
}
}
{
\newrgbcolor{curcolor}{0 0 0}
\pscustom[linewidth=2,linecolor=curcolor]
{
\newpath
\moveto(294.88945961,140.00001689)
\lineto(415.27087975,140.00001689)
\curveto(423.51964037,140.00001689)(430.16033936,133.35931791)(430.16033936,125.11055728)
\curveto(430.16033936,116.86179666)(423.51964037,110.22109767)(415.27087975,110.22109767)
\lineto(294.88945961,110.22109767)
\curveto(286.64069899,110.22109767)(280,116.86179666)(280,125.11055728)
\curveto(280,133.35931791)(286.64069899,140.00001689)(294.88945961,140.00001689)
\closepath
}
}
{
\newrgbcolor{curcolor}{0 0 0}
\pscustom[linestyle=none,fillstyle=solid,fillcolor=curcolor,opacity=0]
{
\newpath
\moveto(144.83261013,139.6113145)
\lineto(245.30122471,139.6113145)
\curveto(253.51662253,139.6113145)(260.13046265,132.99747481)(260.13046265,124.78207752)
\curveto(260.13046265,116.56668023)(253.51662253,109.95284053)(245.30122471,109.95284053)
\lineto(144.83261013,109.95284053)
\curveto(136.61721231,109.95284053)(130.00337219,116.56668023)(130.00337219,124.78207752)
\curveto(130.00337219,132.99747481)(136.61721231,139.6113145)(144.83261013,139.6113145)
\closepath
}
}
{
\newrgbcolor{curcolor}{0 0 0}
\pscustom[linewidth=2,linecolor=curcolor]
{
\newpath
\moveto(144.83261013,139.6113145)
\lineto(245.30122471,139.6113145)
\curveto(253.51662253,139.6113145)(260.13046265,132.99747481)(260.13046265,124.78207752)
\curveto(260.13046265,116.56668023)(253.51662253,109.95284053)(245.30122471,109.95284053)
\lineto(144.83261013,109.95284053)
\curveto(136.61721231,109.95284053)(130.00337219,116.56668023)(130.00337219,124.78207752)
\curveto(130.00337219,132.99747481)(136.61721231,139.6113145)(144.83261013,139.6113145)
\closepath
}
}
{
\newrgbcolor{curcolor}{0 0 0}
\pscustom[linestyle=none,fillstyle=solid,fillcolor=curcolor]
{
\newpath
\moveto(158.30810547,135.10872051)
\lineto(155.09716797,126.40168926)
\lineto(161.53076172,126.40168926)
\lineto(158.30810547,135.10872051)
\moveto(156.97216797,137.44075176)
\lineto(159.65576172,137.44075176)
\lineto(166.32373047,119.94465801)
\lineto(163.86279297,119.94465801)
\lineto(162.26904297,124.43293926)
\lineto(154.38232422,124.43293926)
\lineto(152.78857422,119.94465801)
\lineto(150.29248047,119.94465801)
\lineto(156.97216797,137.44075176)
}
}
{
\newrgbcolor{curcolor}{0 0 0}
\pscustom[linestyle=none,fillstyle=solid,fillcolor=curcolor]
{
\newpath
\moveto(179.68310547,127.86653301)
\lineto(179.68310547,119.94465801)
\lineto(177.52685547,119.94465801)
\lineto(177.52685547,127.79622051)
\curveto(177.52684445,129.03839891)(177.2846572,129.96808548)(176.80029297,130.58528301)
\curveto(176.31590816,131.20245925)(175.58934639,131.51105269)(174.62060547,131.51106426)
\curveto(173.45653602,131.51105269)(172.53856819,131.13995931)(171.86669922,130.39778301)
\curveto(171.19481954,129.6555858)(170.85888237,128.64386806)(170.85888672,127.36262676)
\lineto(170.85888672,119.94465801)
\lineto(168.69091797,119.94465801)
\lineto(168.69091797,133.06965801)
\lineto(170.85888672,133.06965801)
\lineto(170.85888672,131.03059551)
\curveto(171.37450686,131.81964613)(171.979975,132.40948929)(172.67529297,132.80012676)
\curveto(173.3784111,133.19073851)(174.18700404,133.38605082)(175.10107422,133.38606426)
\curveto(176.60887662,133.38605082)(177.74950048,132.91730129)(178.52294922,131.97981426)
\curveto(179.29637393,131.05011565)(179.6830923,129.67902327)(179.68310547,127.86653301)
}
}
{
\newrgbcolor{curcolor}{0 0 0}
\pscustom[linestyle=none,fillstyle=solid,fillcolor=curcolor]
{
\newpath
\moveto(194.91748047,127.86653301)
\lineto(194.91748047,119.94465801)
\lineto(192.76123047,119.94465801)
\lineto(192.76123047,127.79622051)
\curveto(192.76121945,129.03839891)(192.5190322,129.96808548)(192.03466797,130.58528301)
\curveto(191.55028316,131.20245925)(190.82372139,131.51105269)(189.85498047,131.51106426)
\curveto(188.69091102,131.51105269)(187.77294319,131.13995931)(187.10107422,130.39778301)
\curveto(186.42919454,129.6555858)(186.09325737,128.64386806)(186.09326172,127.36262676)
\lineto(186.09326172,119.94465801)
\lineto(183.92529297,119.94465801)
\lineto(183.92529297,133.06965801)
\lineto(186.09326172,133.06965801)
\lineto(186.09326172,131.03059551)
\curveto(186.60888186,131.81964613)(187.21435,132.40948929)(187.90966797,132.80012676)
\curveto(188.6127861,133.19073851)(189.42137904,133.38605082)(190.33544922,133.38606426)
\curveto(191.84325162,133.38605082)(192.98387548,132.91730129)(193.75732422,131.97981426)
\curveto(194.53074893,131.05011565)(194.9174673,129.67902327)(194.91748047,127.86653301)
}
}
{
\newrgbcolor{curcolor}{0 0 0}
\pscustom[linestyle=none,fillstyle=solid,fillcolor=curcolor]
{
\newpath
\moveto(199.01904297,125.12434551)
\lineto(199.01904297,133.06965801)
\lineto(201.17529297,133.06965801)
\lineto(201.17529297,125.20637676)
\curveto(201.17528877,123.96418524)(201.41747603,123.03059242)(201.90185547,122.40559551)
\curveto(202.38622506,121.78840616)(203.11278684,121.47981272)(204.08154297,121.47981426)
\curveto(205.2455972,121.47981272)(206.16356504,121.8509061)(206.83544922,122.59309551)
\curveto(207.51512618,123.33527962)(207.85496959,124.34699736)(207.85498047,125.62825176)
\lineto(207.85498047,133.06965801)
\lineto(210.01123047,133.06965801)
\lineto(210.01123047,119.94465801)
\lineto(207.85498047,119.94465801)
\lineto(207.85498047,121.96028301)
\curveto(207.33153262,121.16340679)(206.72215823,120.56965738)(206.02685547,120.17903301)
\curveto(205.33934711,119.79622066)(204.53856666,119.6048146)(203.62451172,119.60481426)
\curveto(202.11669408,119.6048146)(200.97216398,120.07356413)(200.19091797,121.01106426)
\curveto(199.40966554,121.94856225)(199.01904093,123.31965463)(199.01904297,125.12434551)
\moveto(204.44482422,133.38606426)
\lineto(204.44482422,133.38606426)
}
}
{
\newrgbcolor{curcolor}{0 0 0}
\pscustom[linestyle=none,fillstyle=solid,fillcolor=curcolor]
{
\newpath
\moveto(214.47607422,138.17903301)
\lineto(216.63232422,138.17903301)
\lineto(216.63232422,119.94465801)
\lineto(214.47607422,119.94465801)
\lineto(214.47607422,138.17903301)
}
}
{
\newrgbcolor{curcolor}{0 0 0}
\pscustom[linestyle=none,fillstyle=solid,fillcolor=curcolor]
{
\newpath
\moveto(232.35888672,127.04622051)
\lineto(232.35888672,125.99153301)
\lineto(222.44482422,125.99153301)
\curveto(222.53857055,124.50715345)(222.98388261,123.37434208)(223.78076172,122.59309551)
\curveto(224.5854435,121.81965613)(225.70262989,121.43293777)(227.13232422,121.43293926)
\curveto(227.96044013,121.43293777)(228.76122058,121.53450017)(229.53466797,121.73762676)
\curveto(230.31590652,121.94074976)(231.08934325,122.24543696)(231.85498047,122.65168926)
\lineto(231.85498047,120.61262676)
\curveto(231.08153076,120.28450142)(230.2885628,120.03450167)(229.47607422,119.86262676)
\curveto(228.66356443,119.69075201)(227.8393465,119.6048146)(227.00341797,119.60481426)
\curveto(224.90966193,119.6048146)(223.24950734,120.21418899)(222.02294922,121.43293926)
\curveto(220.80419729,122.65168655)(220.19482289,124.3001224)(220.19482422,126.37825176)
\curveto(220.19482289,128.52668068)(220.77294732,130.22980397)(221.92919922,131.48762676)
\curveto(223.0932575,132.75323895)(224.65966218,133.38605082)(226.62841797,133.38606426)
\curveto(228.39403345,133.38605082)(229.7885633,132.81573889)(230.81201172,131.67512676)
\curveto(231.84324875,130.54230366)(232.35887323,128.99933645)(232.35888672,127.04622051)
\moveto(230.20263672,127.67903301)
\curveto(230.1870004,128.85871159)(229.85496948,129.8001169)(229.20654297,130.50325176)
\curveto(228.56590827,131.2063655)(227.71434662,131.55792764)(226.65185547,131.55793926)
\curveto(225.44872389,131.55792764)(224.48388111,131.21808423)(223.75732422,130.53840801)
\curveto(223.03857005,129.85871059)(222.62450796,128.9016803)(222.51513672,127.66731426)
\lineto(230.20263672,127.67903301)
}
}
{
\newrgbcolor{curcolor}{0 0 0}
\pscustom[linestyle=none,fillstyle=solid,fillcolor=curcolor]
{
\newpath
\moveto(243.50341797,131.05403301)
\curveto(243.26122084,131.19464676)(242.99559611,131.29620916)(242.70654297,131.35872051)
\curveto(242.42528418,131.42902152)(242.11278449,131.46417774)(241.76904297,131.46418926)
\curveto(240.55028605,131.46417774)(239.61278699,131.06574064)(238.95654297,130.26887676)
\curveto(238.3081008,129.47980472)(237.98388237,128.34308711)(237.98388672,126.85872051)
\lineto(237.98388672,119.94465801)
\lineto(235.81591797,119.94465801)
\lineto(235.81591797,133.06965801)
\lineto(237.98388672,133.06965801)
\lineto(237.98388672,131.03059551)
\curveto(238.43700692,131.82745863)(239.02685008,132.41730179)(239.75341797,132.80012676)
\curveto(240.47997363,133.19073851)(241.36278524,133.38605082)(242.40185547,133.38606426)
\curveto(242.55028405,133.38605082)(242.71434639,133.37433208)(242.89404297,133.35090801)
\curveto(243.07372103,133.33526962)(243.27293958,133.30792589)(243.49169922,133.26887676)
\lineto(243.50341797,131.05403301)
}
}
{
\newrgbcolor{curcolor}{0 0 0}
\pscustom[linestyle=none,fillstyle=solid,fillcolor=curcolor]
{
\newpath
\moveto(304.36921127,135.01194477)
\lineto(304.36921127,132.69151869)
\curveto(303.62840434,133.38145986)(302.83677307,133.89710958)(301.99431509,134.23846941)
\curveto(301.1590958,134.57980077)(300.26941846,134.75047356)(299.3252804,134.75048832)
\curveto(297.466027,134.75047356)(296.04254325,134.1803538)(295.05482489,133.04012731)
\curveto(294.06709642,131.90713741)(293.57323471,130.2657735)(293.57323828,128.11603065)
\curveto(293.57323471,125.97353424)(294.06709642,124.33217033)(295.05482489,123.19193399)
\curveto(296.04254325,122.05895394)(297.466027,121.49246551)(299.3252804,121.492467)
\curveto(300.26941846,121.49246551)(301.1590958,121.66313831)(301.99431509,122.0044859)
\curveto(302.83677307,122.34582949)(303.62840434,122.86147921)(304.36921127,123.55143662)
\lineto(304.36921127,121.25279858)
\curveto(303.59935365,120.72988493)(302.78230303,120.33770063)(301.91805696,120.07624451)
\curveto(301.06104972,119.81478824)(300.1532157,119.68406014)(299.19455217,119.68405982)
\curveto(296.73249711,119.68406014)(294.79336364,120.43574671)(293.37714594,121.93912179)
\curveto(291.96092149,123.44975566)(291.25281095,125.50872322)(291.2528122,128.11603065)
\curveto(291.25281095,130.73058452)(291.96092149,132.78955208)(293.37714594,134.29293951)
\curveto(294.79336364,135.80356103)(296.73249711,136.55887894)(299.19455217,136.5588955)
\curveto(300.16774105,136.55887894)(301.08283774,136.42815084)(301.939845,136.16671081)
\curveto(302.80409104,135.91250111)(303.61387899,135.52757949)(304.36921127,135.01194477)
}
}
{
\newrgbcolor{curcolor}{0 0 0}
\pscustom[linestyle=none,fillstyle=solid,fillcolor=curcolor]
{
\newpath
\moveto(312.430785,130.79595936)
\curveto(311.35590268,130.79594857)(310.50617004,130.37471358)(309.88158451,129.53225314)
\curveto(309.25699043,128.69703631)(308.94469552,127.5495341)(308.94469887,126.08974309)
\curveto(308.94469552,124.62993989)(309.25335909,123.47880635)(309.87069049,122.63633901)
\curveto(310.49527603,121.80112908)(311.34864001,121.38352543)(312.430785,121.38352681)
\curveto(313.49839098,121.38352543)(314.34449228,121.80476041)(314.96909146,122.64723303)
\curveto(315.5936719,123.48970036)(315.9059668,124.63720256)(315.90597711,126.08974309)
\curveto(315.9059668,127.53500876)(315.5936719,128.67887963)(314.96909146,129.52135912)
\curveto(314.34449228,130.37108224)(313.49839098,130.79594857)(312.430785,130.79595936)
\moveto(312.430785,132.49542635)
\curveto(314.17381949,132.49541385)(315.54283319,131.92892542)(316.53783022,130.79595936)
\curveto(317.53280537,129.66297171)(318.03029841,128.09423452)(318.03031084,126.08974309)
\curveto(318.03029841,124.09250215)(317.53280537,122.52376496)(316.53783022,121.38352681)
\curveto(315.54283319,120.25054857)(314.17381949,119.68406014)(312.430785,119.68405982)
\curveto(310.68047417,119.68406014)(309.30782913,120.25054857)(308.31284576,121.38352681)
\curveto(307.32511963,122.52376496)(306.83125792,124.09250215)(306.83125915,126.08974309)
\curveto(306.83125792,128.09423452)(307.32511963,129.66297171)(308.31284576,130.79595936)
\curveto(309.30782913,131.92892542)(310.68047417,132.49541385)(312.430785,132.49542635)
}
}
{
\newrgbcolor{curcolor}{0 0 0}
\pscustom[linestyle=none,fillstyle=solid,fillcolor=curcolor]
{
\newpath
\moveto(331.48442526,127.36434333)
\lineto(331.48442526,119.99998638)
\lineto(329.47992574,119.99998638)
\lineto(329.47992574,127.29897921)
\curveto(329.4799155,128.45373679)(329.25477266,129.31799478)(328.80449655,129.89175577)
\curveto(328.35420131,130.46549698)(327.6787728,130.75237253)(326.77820899,130.75238328)
\curveto(325.69606329,130.75237253)(324.84269931,130.4073956)(324.21811448,129.71745146)
\curveto(323.5935197,129.02748789)(323.28122479,128.08697184)(323.28122884,126.8959005)
\lineto(323.28122884,119.99998638)
\lineto(321.26583529,119.99998638)
\lineto(321.26583529,132.20128783)
\lineto(323.28122884,132.20128783)
\lineto(323.28122884,130.3057285)
\curveto(323.76056116,131.03924808)(324.32341825,131.58757983)(324.96980181,131.95072539)
\curveto(325.62343657,132.31384705)(326.37512314,132.49541385)(327.22486377,132.49542635)
\curveto(328.62655152,132.49541385)(329.68690165,132.05965352)(330.40591737,131.18814405)
\curveto(331.12491074,130.32387487)(331.48441302,129.04927591)(331.48442526,127.36434333)
}
}
{
\newrgbcolor{curcolor}{0 0 0}
\pscustom[linestyle=none,fillstyle=solid,fillcolor=curcolor]
{
\newpath
\moveto(345.64664908,127.36434333)
\lineto(345.64664908,119.99998638)
\lineto(343.64214955,119.99998638)
\lineto(343.64214955,127.29897921)
\curveto(343.64213931,128.45373679)(343.41699647,129.31799478)(342.96672036,129.89175577)
\curveto(342.51642513,130.46549698)(341.84099661,130.75237253)(340.9404328,130.75238328)
\curveto(339.85828711,130.75237253)(339.00492313,130.4073956)(338.3803383,129.71745146)
\curveto(337.75574351,129.02748789)(337.44344861,128.08697184)(337.44345265,126.8959005)
\lineto(337.44345265,119.99998638)
\lineto(335.42805911,119.99998638)
\lineto(335.42805911,132.20128783)
\lineto(337.44345265,132.20128783)
\lineto(337.44345265,130.3057285)
\curveto(337.92278497,131.03924808)(338.48564207,131.58757983)(339.13202562,131.95072539)
\curveto(339.78566039,132.31384705)(340.53734696,132.49541385)(341.38708759,132.49542635)
\curveto(342.78877533,132.49541385)(343.84912547,132.05965352)(344.56814118,131.18814405)
\curveto(345.28713456,130.32387487)(345.64663683,129.04927591)(345.64664908,127.36434333)
}
}
{
\newrgbcolor{curcolor}{0 0 0}
\pscustom[linestyle=none,fillstyle=solid,fillcolor=curcolor]
{
\newpath
\moveto(360.10301141,126.60176199)
\lineto(360.10301141,125.62130026)
\lineto(350.8866712,125.62130026)
\curveto(350.97381995,124.24138693)(351.38779226,123.18829946)(352.12858939,122.46203471)
\curveto(352.87664006,121.7430277)(353.91520218,121.38352543)(355.24427887,121.38352681)
\curveto(356.01411444,121.38352543)(356.75853834,121.47794016)(357.47755279,121.66677131)
\curveto(358.2038101,121.85559912)(358.92281464,122.13884333)(359.63456859,122.5165048)
\lineto(359.63456859,120.62094547)
\curveto(358.91555197,120.31591262)(358.17839075,120.08350711)(357.4230827,119.92372824)
\curveto(356.66775493,119.76394953)(355.90154302,119.68406014)(355.12444466,119.68405982)
\curveto(353.17804095,119.68406014)(351.63472312,120.25054857)(350.49448651,121.38352681)
\curveto(349.36150672,122.51650229)(348.7950183,124.04892611)(348.79501953,125.98080289)
\curveto(348.7950183,127.97803176)(349.33245604,129.5612943)(350.40733436,130.73059525)
\curveto(351.48946967,131.90713741)(352.94563544,132.49541385)(354.77583604,132.49542635)
\curveto(356.41719274,132.49541385)(357.71357973,131.96523879)(358.66500088,130.90489955)
\curveto(359.62366251,129.85180118)(360.10299887,128.41742343)(360.10301141,126.60176199)
\moveto(358.09851189,127.19003902)
\curveto(358.08397601,128.28669533)(357.77531244,129.16184733)(357.17252026,129.81549764)
\curveto(356.57697153,130.46912832)(355.78534026,130.79594857)(354.79762408,130.79595936)
\curveto(353.67916533,130.79594857)(352.78222532,130.48002233)(352.10680135,129.84817969)
\curveto(351.43863097,129.21631737)(351.05370934,128.32664003)(350.95203532,127.179145)
\lineto(358.09851189,127.19003902)
}
}
{
\newrgbcolor{curcolor}{0 0 0}
\pscustom[linestyle=none,fillstyle=solid,fillcolor=curcolor]
{
\newpath
\moveto(373.14315481,132.20128783)
\lineto(368.73107706,126.26404739)
\lineto(373.37192922,119.99998638)
\lineto(371.00792706,119.99998638)
\lineto(367.45647682,124.79335481)
\lineto(363.90502657,119.99998638)
\lineto(361.54102441,119.99998638)
\lineto(366.27992275,126.3838816)
\lineto(361.94410312,132.20128783)
\lineto(364.30810528,132.20128783)
\lineto(367.54362897,127.85457419)
\lineto(370.77915266,132.20128783)
\lineto(373.14315481,132.20128783)
}
}
{
\newrgbcolor{curcolor}{0 0 0}
\pscustom[linestyle=none,fillstyle=solid,fillcolor=curcolor]
{
\newpath
\moveto(376.20437523,132.20128783)
\lineto(378.20887475,132.20128783)
\lineto(378.20887475,119.99998638)
\lineto(376.20437523,119.99998638)
\lineto(376.20437523,132.20128783)
\moveto(376.20437523,136.95108018)
\lineto(378.20887475,136.95108018)
\lineto(378.20887475,134.41277372)
\lineto(376.20437523,134.41277372)
\lineto(376.20437523,136.95108018)
}
}
{
\newrgbcolor{curcolor}{0 0 0}
\pscustom[linestyle=none,fillstyle=solid,fillcolor=curcolor]
{
\newpath
\moveto(387.12017709,130.79595936)
\curveto(386.04529477,130.79594857)(385.19556213,130.37471358)(384.5709766,129.53225314)
\curveto(383.94638252,128.69703631)(383.63408761,127.5495341)(383.63409096,126.08974309)
\curveto(383.63408761,124.62993989)(383.94275118,123.47880635)(384.56008258,122.63633901)
\curveto(385.18466812,121.80112908)(386.0380321,121.38352543)(387.12017709,121.38352681)
\curveto(388.18778307,121.38352543)(389.03388437,121.80476041)(389.65848355,122.64723303)
\curveto(390.28306399,123.48970036)(390.59535889,124.63720256)(390.5953692,126.08974309)
\curveto(390.59535889,127.53500876)(390.28306399,128.67887963)(389.65848355,129.52135912)
\curveto(389.03388437,130.37108224)(388.18778307,130.79594857)(387.12017709,130.79595936)
\moveto(387.12017709,132.49542635)
\curveto(388.86321158,132.49541385)(390.23222528,131.92892542)(391.22722231,130.79595936)
\curveto(392.22219746,129.66297171)(392.7196905,128.09423452)(392.71970293,126.08974309)
\curveto(392.7196905,124.09250215)(392.22219746,122.52376496)(391.22722231,121.38352681)
\curveto(390.23222528,120.25054857)(388.86321158,119.68406014)(387.12017709,119.68405982)
\curveto(385.36986626,119.68406014)(383.99722122,120.25054857)(383.00223785,121.38352681)
\curveto(382.01451172,122.52376496)(381.52065001,124.09250215)(381.52065124,126.08974309)
\curveto(381.52065001,128.09423452)(382.01451172,129.66297171)(383.00223785,130.79595936)
\curveto(383.99722122,131.92892542)(385.36986626,132.49541385)(387.12017709,132.49542635)
}
}
{
\newrgbcolor{curcolor}{0 0 0}
\pscustom[linestyle=none,fillstyle=solid,fillcolor=curcolor]
{
\newpath
\moveto(406.17381735,127.36434333)
\lineto(406.17381735,119.99998638)
\lineto(404.16931783,119.99998638)
\lineto(404.16931783,127.29897921)
\curveto(404.16930759,128.45373679)(403.94416475,129.31799478)(403.49388864,129.89175577)
\curveto(403.0435934,130.46549698)(402.36816489,130.75237253)(401.46760108,130.75238328)
\curveto(400.38545538,130.75237253)(399.5320914,130.4073956)(398.90750657,129.71745146)
\curveto(398.28291179,129.02748789)(397.97061688,128.08697184)(397.97062093,126.8959005)
\lineto(397.97062093,119.99998638)
\lineto(395.95522738,119.99998638)
\lineto(395.95522738,132.20128783)
\lineto(397.97062093,132.20128783)
\lineto(397.97062093,130.3057285)
\curveto(398.44995325,131.03924808)(399.01281034,131.58757983)(399.6591939,131.95072539)
\curveto(400.31282866,132.31384705)(401.06451523,132.49541385)(401.91425586,132.49542635)
\curveto(403.31594361,132.49541385)(404.37629374,132.05965352)(405.09530945,131.18814405)
\curveto(405.81430283,130.32387487)(406.17380511,129.04927591)(406.17381735,127.36434333)
}
}
{
\newrgbcolor{curcolor}{1 1 1}
\pscustom[linestyle=none,fillstyle=solid,fillcolor=curcolor]
{
\newpath
\moveto(129.960495,479.98731395)
\lineto(279.98731995,479.98731395)
\lineto(279.98731995,450.11458942)
\lineto(129.960495,450.11458942)
\closepath
}
}
{
\newrgbcolor{curcolor}{0 0 0}
\pscustom[linewidth=2,linecolor=curcolor]
{
\newpath
\moveto(129.960495,479.98731395)
\lineto(279.98731995,479.98731395)
\lineto(279.98731995,450.11458942)
\lineto(129.960495,450.11458942)
\closepath
}
}
{
\newrgbcolor{curcolor}{0 0 0}
\pscustom[linestyle=none,fillstyle=solid,fillcolor=curcolor,opacity=0]
{
\newpath
\moveto(294.7643919,450.05025288)
\lineto(415.16532612,450.05025288)
\curveto(423.41158085,450.05025288)(430.05026245,443.41157128)(430.05026245,435.16531655)
\curveto(430.05026245,426.91906182)(423.41158085,420.28038022)(415.16532612,420.28038022)
\lineto(294.7643919,420.28038022)
\curveto(286.51813717,420.28038022)(279.87945557,426.91906182)(279.87945557,435.16531655)
\curveto(279.87945557,443.41157128)(286.51813717,450.05025288)(294.7643919,450.05025288)
\closepath
}
}
{
\newrgbcolor{curcolor}{0 0 0}
\pscustom[linewidth=2,linecolor=curcolor]
{
\newpath
\moveto(294.7643919,450.05025288)
\lineto(415.16532612,450.05025288)
\curveto(423.41158085,450.05025288)(430.05026245,443.41157128)(430.05026245,435.16531655)
\curveto(430.05026245,426.91906182)(423.41158085,420.28038022)(415.16532612,420.28038022)
\lineto(294.7643919,420.28038022)
\curveto(286.51813717,420.28038022)(279.87945557,426.91906182)(279.87945557,435.16531655)
\curveto(279.87945557,443.41157128)(286.51813717,450.05025288)(294.7643919,450.05025288)
\closepath
}
}
{
\newrgbcolor{curcolor}{0 0 0}
\pscustom[linestyle=none,fillstyle=solid,fillcolor=curcolor]
{
\newpath
\moveto(292.15917969,446.03808567)
\lineto(294.32910156,446.03808567)
\lineto(294.32910156,429.99999973)
\lineto(292.15917969,429.99999973)
\lineto(292.15917969,446.03808567)
}
}
{
\newrgbcolor{curcolor}{0 0 0}
\pscustom[linestyle=none,fillstyle=solid,fillcolor=curcolor]
{
\newpath
\moveto(308.5625,437.26171848)
\lineto(308.5625,429.99999973)
\lineto(306.5859375,429.99999973)
\lineto(306.5859375,437.19726535)
\curveto(306.5859274,438.33592889)(306.36392242,439.18814158)(305.91992188,439.75390598)
\curveto(305.47590247,440.31965087)(304.80988751,440.60252819)(303.921875,440.60253879)
\curveto(302.85481134,440.60252819)(302.01334083,440.26235926)(301.39746094,439.58203098)
\curveto(300.78157123,438.90168354)(300.47362883,437.97427561)(300.47363281,436.79980442)
\lineto(300.47363281,429.99999973)
\lineto(298.48632812,429.99999973)
\lineto(298.48632812,442.03124973)
\lineto(300.47363281,442.03124973)
\lineto(300.47363281,440.1621091)
\curveto(300.9462846,440.88540551)(301.50129707,441.42609507)(302.13867188,441.78417942)
\curveto(302.78319683,442.14224019)(303.52440703,442.32127647)(304.36230469,442.32128879)
\curveto(305.74445689,442.32127647)(306.79002876,441.8915894)(307.49902344,441.03222629)
\curveto(308.20799609,440.18000257)(308.56248793,438.92316789)(308.5625,437.26171848)
}
}
{
\newrgbcolor{curcolor}{0 0 0}
\pscustom[linestyle=none,fillstyle=solid,fillcolor=curcolor]
{
\newpath
\moveto(320.19628906,441.67675754)
\lineto(320.19628906,439.80761692)
\curveto(319.63768613,440.09406516)(319.05760858,440.30890869)(318.45605469,440.45214817)
\curveto(317.85448479,440.59536674)(317.23143853,440.66698125)(316.58691406,440.66699192)
\curveto(315.60578912,440.66698125)(314.86815965,440.51659077)(314.37402344,440.21582004)
\curveto(313.88704084,439.91502888)(313.6435515,439.46385745)(313.64355469,438.86230442)
\curveto(313.6435515,438.40396268)(313.81900705,438.04230939)(314.16992188,437.77734348)
\curveto(314.52082927,437.51952346)(315.22623221,437.27245339)(316.28613281,437.03613254)
\lineto(316.96289062,436.88574192)
\curveto(318.36652854,436.58495408)(319.36197026,436.15884774)(319.94921875,435.6074216)
\curveto(320.5436097,435.06314571)(320.84080992,434.30045116)(320.84082031,433.31933567)
\curveto(320.84080992,432.20214596)(320.39679995,431.31770674)(319.50878906,430.66601535)
\curveto(318.62792151,430.01432263)(317.41405554,429.6884766)(315.8671875,429.68847629)
\curveto(315.22265148,429.6884766)(314.54947507,429.75292966)(313.84765625,429.88183567)
\curveto(313.15299209,430.00358045)(312.41894335,430.18977818)(311.64550781,430.44042942)
\lineto(311.64550781,432.48144504)
\curveto(312.37597464,432.10188565)(313.09570048,431.8154276)(313.8046875,431.62207004)
\curveto(314.51366781,431.43587069)(315.21549003,431.34277182)(315.91015625,431.34277317)
\curveto(316.84113945,431.34277182)(317.55728456,431.50032375)(318.05859375,431.81542942)
\curveto(318.55988773,432.1376929)(318.81053852,432.58886433)(318.81054688,433.16894504)
\curveto(318.81053852,433.70605071)(318.62792151,434.11783415)(318.26269531,434.4042966)
\curveto(317.90461494,434.69075025)(317.11327459,434.96646612)(315.88867188,435.23144504)
\lineto(315.20117188,435.39257785)
\curveto(313.97655898,435.6503847)(313.09211976,436.04426452)(312.54785156,436.57421848)
\curveto(312.00357918,437.11132074)(311.73144403,437.84536949)(311.73144531,438.77636692)
\curveto(311.73144403,439.90786742)(312.1324853,440.78156447)(312.93457031,441.39746067)
\curveto(313.73665036,442.01333407)(314.8753211,442.32127647)(316.35058594,442.32128879)
\curveto(317.08104806,442.32127647)(317.76854737,442.26756559)(318.41308594,442.16015598)
\curveto(319.05760858,442.05272205)(319.65200903,441.8915894)(320.19628906,441.67675754)
}
}
{
\newrgbcolor{curcolor}{0 0 0}
\pscustom[linestyle=none,fillstyle=solid,fillcolor=curcolor]
{
\newpath
\moveto(332.65722656,441.56933567)
\lineto(332.65722656,439.72167942)
\curveto(332.09862264,440.02961209)(331.53644872,440.25877853)(330.97070312,440.40917942)
\curveto(330.41210089,440.56672093)(329.84634625,440.6454969)(329.2734375,440.64550754)
\curveto(327.99153039,440.6454969)(326.99608868,440.23729418)(326.28710938,439.42089817)
\curveto(325.57812135,438.61164476)(325.22362951,437.47297403)(325.22363281,436.00488254)
\curveto(325.22362951,434.53677905)(325.57812135,433.39452758)(326.28710938,432.57812473)
\curveto(326.99608868,431.76887817)(327.99153039,431.36425618)(329.2734375,431.36425754)
\curveto(329.84634625,431.36425618)(330.41210089,431.43945141)(330.97070312,431.58984348)
\curveto(331.53644872,431.74739381)(332.09862264,431.98014098)(332.65722656,432.28808567)
\lineto(332.65722656,430.46191379)
\curveto(332.10578409,430.20410109)(331.532868,430.01074191)(330.93847656,429.88183567)
\curveto(330.35122855,429.75292966)(329.72460158,429.6884766)(329.05859375,429.68847629)
\curveto(327.24673947,429.6884766)(325.80728779,430.25781197)(324.74023438,431.3964841)
\curveto(323.67317534,432.53515344)(323.13964722,434.07128472)(323.13964844,436.00488254)
\curveto(323.13964722,437.96711416)(323.67675606,439.51040688)(324.75097656,440.63476535)
\curveto(325.83235286,441.75910255)(327.31119253,442.32127647)(329.1875,442.32128879)
\curveto(329.79621609,442.32127647)(330.39061654,442.25682341)(330.97070312,442.12792942)
\curveto(331.55077162,442.00617262)(332.11294554,441.81997489)(332.65722656,441.56933567)
}
}
{
\newrgbcolor{curcolor}{0 0 0}
\pscustom[linestyle=none,fillstyle=solid,fillcolor=curcolor]
{
\newpath
\moveto(343.08789062,440.18359348)
\curveto(342.86587659,440.31248942)(342.62238725,440.40558828)(342.35742188,440.46289035)
\curveto(342.09960132,440.52733295)(341.81314327,440.55955948)(341.49804688,440.55957004)
\curveto(340.38085304,440.55955948)(339.5214789,440.19432547)(338.91992188,439.46386692)
\curveto(338.32551655,438.74055088)(338.02831633,437.69855974)(338.02832031,436.33789035)
\lineto(338.02832031,429.99999973)
\lineto(336.04101562,429.99999973)
\lineto(336.04101562,442.03124973)
\lineto(338.02832031,442.03124973)
\lineto(338.02832031,440.1621091)
\curveto(338.4436805,440.89256696)(338.98437006,441.43325652)(339.65039062,441.78417942)
\curveto(340.31639998,442.14224019)(341.12564396,442.32127647)(342.078125,442.32128879)
\curveto(342.21418454,442.32127647)(342.36457501,442.31053429)(342.52929688,442.28906223)
\curveto(342.69400177,442.27472704)(342.87661877,442.24966196)(343.07714844,442.21386692)
\lineto(343.08789062,440.18359348)
}
}
{
\newrgbcolor{curcolor}{0 0 0}
\pscustom[linestyle=none,fillstyle=solid,fillcolor=curcolor]
{
\newpath
\moveto(345.18261719,442.03124973)
\lineto(347.15917969,442.03124973)
\lineto(347.15917969,429.99999973)
\lineto(345.18261719,429.99999973)
\lineto(345.18261719,442.03124973)
\moveto(345.18261719,446.71484348)
\lineto(347.15917969,446.71484348)
\lineto(347.15917969,444.21191379)
\lineto(345.18261719,444.21191379)
\lineto(345.18261719,446.71484348)
}
}
{
\newrgbcolor{curcolor}{0 0 0}
\pscustom[linestyle=none,fillstyle=solid,fillcolor=curcolor]
{
\newpath
\moveto(353.19628906,431.80468723)
\lineto(353.19628906,425.42382785)
\lineto(351.20898438,425.42382785)
\lineto(351.20898438,442.03124973)
\lineto(353.19628906,442.03124973)
\lineto(353.19628906,440.20507785)
\curveto(353.61164925,440.92121277)(354.13443518,441.45116015)(354.76464844,441.7949216)
\curveto(355.40201204,442.14582092)(356.16112586,442.32127647)(357.04199219,442.32128879)
\curveto(358.5029204,442.32127647)(359.68814056,441.74119892)(360.59765625,440.58105442)
\curveto(361.51431061,439.42088874)(361.97264349,437.89549965)(361.97265625,436.00488254)
\curveto(361.97264349,434.11425343)(361.51431061,432.58886433)(360.59765625,431.42871067)
\curveto(359.68814056,430.26855415)(358.5029204,429.6884766)(357.04199219,429.68847629)
\curveto(356.16112586,429.6884766)(355.40201204,429.86035143)(354.76464844,430.20410129)
\curveto(354.13443518,430.55501219)(353.61164925,431.08854031)(353.19628906,431.80468723)
\moveto(359.92089844,436.00488254)
\curveto(359.92088773,437.45865112)(359.62010678,438.59732186)(359.01855469,439.42089817)
\curveto(358.42414443,440.25161708)(357.60415827,440.66698125)(356.55859375,440.66699192)
\curveto(355.51301453,440.66698125)(354.68944765,440.25161708)(354.08789062,439.42089817)
\curveto(353.4934853,438.59732186)(353.19628508,437.45865112)(353.19628906,436.00488254)
\curveto(353.19628508,434.55110195)(353.4934853,433.40885049)(354.08789062,432.57812473)
\curveto(354.68944765,431.75455527)(355.51301453,431.34277182)(356.55859375,431.34277317)
\curveto(357.60415827,431.34277182)(358.42414443,431.75455527)(359.01855469,432.57812473)
\curveto(359.62010678,433.40885049)(359.92088773,434.55110195)(359.92089844,436.00488254)
}
}
{
\newrgbcolor{curcolor}{0 0 0}
\pscustom[linestyle=none,fillstyle=solid,fillcolor=curcolor]
{
\newpath
\moveto(367.20410156,445.44726535)
\lineto(367.20410156,442.03124973)
\lineto(371.27539062,442.03124973)
\lineto(371.27539062,440.49511692)
\lineto(367.20410156,440.49511692)
\lineto(367.20410156,433.96386692)
\curveto(367.20409753,432.98274414)(367.33658438,432.35253644)(367.6015625,432.07324192)
\curveto(367.87369322,431.79394325)(368.42154423,431.65429495)(369.24511719,431.6542966)
\lineto(371.27539062,431.6542966)
\lineto(371.27539062,429.99999973)
\lineto(369.24511719,429.99999973)
\curveto(367.71972202,429.99999973)(366.6669887,430.28287705)(366.08691406,430.84863254)
\curveto(365.50683361,431.42154779)(365.21679483,432.45995821)(365.21679688,433.96386692)
\lineto(365.21679688,440.49511692)
\lineto(363.76660156,440.49511692)
\lineto(363.76660156,442.03124973)
\lineto(365.21679688,442.03124973)
\lineto(365.21679688,445.44726535)
\lineto(367.20410156,445.44726535)
}
}
{
\newrgbcolor{curcolor}{0 0 0}
\pscustom[linestyle=none,fillstyle=solid,fillcolor=curcolor]
{
\newpath
\moveto(373.88574219,442.03124973)
\lineto(375.86230469,442.03124973)
\lineto(375.86230469,429.99999973)
\lineto(373.88574219,429.99999973)
\lineto(373.88574219,442.03124973)
\moveto(373.88574219,446.71484348)
\lineto(375.86230469,446.71484348)
\lineto(375.86230469,444.21191379)
\lineto(373.88574219,444.21191379)
\lineto(373.88574219,446.71484348)
}
}
{
\newrgbcolor{curcolor}{0 0 0}
\pscustom[linestyle=none,fillstyle=solid,fillcolor=curcolor]
{
\newpath
\moveto(384.64941406,440.64550754)
\curveto(383.58951255,440.6454969)(382.75162277,440.23013273)(382.13574219,439.39941379)
\curveto(381.51985317,438.57583751)(381.21191076,437.44432822)(381.21191406,436.00488254)
\curveto(381.21191076,434.56542485)(381.51627244,433.43033484)(382.125,432.5996091)
\curveto(382.74088059,431.77603962)(383.5823511,431.36425618)(384.64941406,431.36425754)
\curveto(385.70214065,431.36425618)(386.53644971,431.77962034)(387.15234375,432.61035129)
\curveto(387.76821931,433.44107702)(388.07616171,434.5725863)(388.07617188,436.00488254)
\curveto(388.07616171,437.43000532)(387.76821931,438.55793388)(387.15234375,439.3886716)
\curveto(386.53644971,440.226552)(385.70214065,440.6454969)(384.64941406,440.64550754)
\moveto(384.64941406,442.32128879)
\curveto(386.36815561,442.32127647)(387.71808915,441.76268328)(388.69921875,440.64550754)
\curveto(389.68032678,439.52831051)(390.17088618,437.98143706)(390.17089844,436.00488254)
\curveto(390.17088618,434.03547746)(389.68032678,432.48860401)(388.69921875,431.36425754)
\curveto(387.71808915,430.24706979)(386.36815561,429.6884766)(384.64941406,429.68847629)
\curveto(382.92349759,429.6884766)(381.56998332,430.24706979)(380.58886719,431.36425754)
\curveto(379.61490715,432.48860401)(379.12792847,434.03547746)(379.12792969,436.00488254)
\curveto(379.12792847,437.98143706)(379.61490715,439.52831051)(380.58886719,440.64550754)
\curveto(381.56998332,441.76268328)(382.92349759,442.32127647)(384.64941406,442.32128879)
}
}
{
\newrgbcolor{curcolor}{0 0 0}
\pscustom[linestyle=none,fillstyle=solid,fillcolor=curcolor]
{
\newpath
\moveto(403.4375,437.26171848)
\lineto(403.4375,429.99999973)
\lineto(401.4609375,429.99999973)
\lineto(401.4609375,437.19726535)
\curveto(401.4609274,438.33592889)(401.23892242,439.18814158)(400.79492188,439.75390598)
\curveto(400.35090247,440.31965087)(399.68488751,440.60252819)(398.796875,440.60253879)
\curveto(397.72981134,440.60252819)(396.88834083,440.26235926)(396.27246094,439.58203098)
\curveto(395.65657123,438.90168354)(395.34862883,437.97427561)(395.34863281,436.79980442)
\lineto(395.34863281,429.99999973)
\lineto(393.36132812,429.99999973)
\lineto(393.36132812,442.03124973)
\lineto(395.34863281,442.03124973)
\lineto(395.34863281,440.1621091)
\curveto(395.8212846,440.88540551)(396.37629707,441.42609507)(397.01367188,441.78417942)
\curveto(397.65819683,442.14224019)(398.39940703,442.32127647)(399.23730469,442.32128879)
\curveto(400.61945689,442.32127647)(401.66502876,441.8915894)(402.37402344,441.03222629)
\curveto(403.08299609,440.18000257)(403.43748793,438.92316789)(403.4375,437.26171848)
}
}
{
\newrgbcolor{curcolor}{1 1 1}
\pscustom[linestyle=none,fillstyle=solid,fillcolor=curcolor]
{
\newpath
\moveto(149.97337341,399.69833919)
\lineto(410.03535461,399.69833919)
\lineto(410.03535461,360.11021587)
\lineto(149.97337341,360.11021587)
\closepath
}
}
{
\newrgbcolor{curcolor}{1 0 0}
\pscustom[linewidth=2,linecolor=curcolor,linestyle=dashed,dash=8 8]
{
\newpath
\moveto(149.97337341,399.69833919)
\lineto(410.03535461,399.69833919)
\lineto(410.03535461,360.11021587)
\lineto(149.97337341,360.11021587)
\closepath
}
}
{
\newrgbcolor{curcolor}{0 0 0}
\pscustom[linestyle=none,fillstyle=solid,fillcolor=curcolor]
{
\newpath
\moveto(43.96875,542.65625164)
\lineto(49.125,542.65625164)
\lineto(49.125,560.45312664)
\lineto(43.515625,559.32812664)
\lineto(43.515625,562.20312664)
\lineto(49.09375,563.32812664)
\lineto(52.25,563.32812664)
\lineto(52.25,542.65625164)
\lineto(57.40625,542.65625164)
\lineto(57.40625,540.00000164)
\lineto(43.96875,540.00000164)
\lineto(43.96875,542.65625164)
}
}
{
\newrgbcolor{curcolor}{0 0 0}
\pscustom[linestyle=none,fillstyle=solid,fillcolor=curcolor]
{
\newpath
\moveto(46.140625,462.65625021)
\lineto(57.15625,462.65625021)
\lineto(57.15625,460.00000021)
\lineto(42.34375,460.00000021)
\lineto(42.34375,462.65625021)
\curveto(43.54166312,463.89582964)(45.17186983,465.55728631)(47.234375,467.64062521)
\curveto(49.30728236,469.73436547)(50.60936439,471.08332246)(51.140625,471.68750021)
\curveto(52.15102952,472.82290405)(52.85415381,473.78123642)(53.25,474.56250021)
\curveto(53.65623634,475.35415152)(53.85936114,476.13019241)(53.859375,476.89062521)
\curveto(53.85936114,478.13019041)(53.42186158,479.14060606)(52.546875,479.92187521)
\curveto(51.68227998,480.7031045)(50.55207278,481.09372911)(49.15625,481.09375021)
\curveto(48.1666585,481.09372911)(47.11978455,480.92185428)(46.015625,480.57812521)
\curveto(44.92187008,480.23435497)(43.74999625,479.71352216)(42.5,479.01562521)
\lineto(42.5,482.20312521)
\curveto(43.77082956,482.71351916)(44.95832838,483.09893544)(46.0625,483.35937521)
\curveto(47.1666595,483.61976825)(48.17707516,483.74997646)(49.09375,483.75000021)
\curveto(51.51040516,483.74997646)(53.43748656,483.14581039)(54.875,481.93750021)
\curveto(56.31248369,480.72914614)(57.03123297,479.11456442)(57.03125,477.09375021)
\curveto(57.03123297,476.13540074)(56.84894148,475.22394331)(56.484375,474.35937521)
\curveto(56.1301922,473.50519503)(55.47915119,472.49477938)(54.53125,471.32812521)
\curveto(54.27081906,471.02603085)(53.44269489,470.15103172)(52.046875,468.70312521)
\curveto(50.65103102,467.26561794)(48.68228298,465.24999496)(46.140625,462.65625021)
}
}
{
\newrgbcolor{curcolor}{0 0 0}
\pscustom[linestyle=none,fillstyle=solid,fillcolor=curcolor]
{
\newpath
\moveto(52.09375,390.57812664)
\lineto(44.125,378.12500164)
\lineto(52.09375,378.12500164)
\lineto(52.09375,390.57812664)
\moveto(51.265625,393.32812664)
\lineto(55.234375,393.32812664)
\lineto(55.234375,378.12500164)
\lineto(58.5625,378.12500164)
\lineto(58.5625,375.50000164)
\lineto(55.234375,375.50000164)
\lineto(55.234375,370.00000164)
\lineto(52.09375,370.00000164)
\lineto(52.09375,375.50000164)
\lineto(41.5625,375.50000164)
\lineto(41.5625,378.54687664)
\lineto(51.265625,393.32812664)
}
}
{
\newrgbcolor{curcolor}{0 0 0}
\pscustom[linestyle=none,fillstyle=solid,fillcolor=curcolor]
{
\newpath
\moveto(503.453125,333.32812664)
\lineto(515.84375,333.32812664)
\lineto(515.84375,330.67187664)
\lineto(506.34375,330.67187664)
\lineto(506.34375,324.95312664)
\curveto(506.80207653,325.10936153)(507.26040941,325.22394475)(507.71875,325.29687664)
\curveto(508.17707516,325.38019459)(508.63540803,325.42186121)(509.09375,325.42187664)
\curveto(511.69790497,325.42186121)(513.76040291,324.70832026)(515.28125,323.28125164)
\curveto(516.80206653,321.85415645)(517.56248244,319.92186671)(517.5625,317.48437664)
\curveto(517.56248244,314.973955)(516.78123322,313.02083195)(515.21875,311.62500164)
\curveto(513.65623634,310.23958473)(511.45311355,309.54687709)(508.609375,309.54687664)
\curveto(507.6302007,309.54687709)(506.6302017,309.63021034)(505.609375,309.79687664)
\curveto(504.59895373,309.96354334)(503.55207978,310.21354309)(502.46875,310.54687664)
\lineto(502.46875,313.71875164)
\curveto(503.40624659,313.20833176)(504.37499562,312.82812381)(505.375,312.57812664)
\curveto(506.37499363,312.32812431)(507.43228423,312.20312443)(508.546875,312.20312664)
\curveto(510.34894798,312.20312443)(511.77602989,312.67708229)(512.828125,313.62500164)
\curveto(513.88019445,314.57291373)(514.40623559,315.85937078)(514.40625,317.48437664)
\curveto(514.40623559,319.10936753)(513.88019445,320.39582457)(512.828125,321.34375164)
\curveto(511.77602989,322.29165601)(510.34894798,322.76561387)(508.546875,322.76562664)
\curveto(507.7031173,322.76561387)(506.85936814,322.67186396)(506.015625,322.48437664)
\curveto(505.18228648,322.29686434)(504.32812067,322.00519796)(503.453125,321.60937664)
\lineto(503.453125,333.32812664)
}
}
{
\newrgbcolor{curcolor}{0 0 0}
\pscustom[linestyle=none,fillstyle=solid,fillcolor=curcolor]
{
\newpath
\moveto(510.5625,272.92187664)
\curveto(509.14582419,272.92186371)(508.02082531,272.4374892)(507.1875,271.46875164)
\curveto(506.36457697,270.49999114)(505.95311905,269.17186746)(505.953125,267.48437664)
\curveto(505.95311905,265.8072875)(506.36457697,264.47916382)(507.1875,263.50000164)
\curveto(508.02082531,262.5312491)(509.14582419,262.04687459)(510.5625,262.04687664)
\curveto(511.97915469,262.04687459)(513.09894523,262.5312491)(513.921875,263.50000164)
\curveto(514.75519358,264.47916382)(515.17185983,265.8072875)(515.171875,267.48437664)
\curveto(515.17185983,269.17186746)(514.75519358,270.49999114)(513.921875,271.46875164)
\curveto(513.09894523,272.4374892)(511.97915469,272.92186371)(510.5625,272.92187664)
\moveto(516.828125,282.81250164)
\lineto(516.828125,279.93750164)
\curveto(516.0364423,280.31248132)(515.23435977,280.59893937)(514.421875,280.79687664)
\curveto(513.61977805,280.99477231)(512.82290384,281.09373054)(512.03125,281.09375164)
\curveto(509.94790672,281.09373054)(508.35415831,280.39060625)(507.25,278.98437664)
\curveto(506.15624384,277.57810906)(505.53124447,275.45311118)(505.375,272.60937664)
\curveto(505.98957734,273.51561312)(506.76040991,274.20832076)(507.6875,274.68750164)
\curveto(508.61457472,275.17706979)(509.63540703,275.42186121)(510.75,275.42187664)
\curveto(513.09373691,275.42186121)(514.94269339,274.70832026)(516.296875,273.28125164)
\curveto(517.66144067,271.8645731)(518.34373166,269.93228337)(518.34375,267.48437664)
\curveto(518.34373166,265.08853821)(517.63539903,263.16666514)(516.21875,261.71875164)
\curveto(514.80206853,260.2708347)(512.91665375,259.54687709)(510.5625,259.54687664)
\curveto(507.86457547,259.54687709)(505.80207753,260.57812606)(504.375,262.64062664)
\curveto(502.94791372,264.71353859)(502.23437277,267.71353559)(502.234375,271.64062664)
\curveto(502.23437277,275.32811131)(503.10937189,278.26560837)(504.859375,280.45312664)
\curveto(506.60936839,282.65102065)(508.95832438,283.74997789)(511.90625,283.75000164)
\curveto(512.69790397,283.74997789)(513.49477817,283.67185296)(514.296875,283.51562664)
\curveto(515.10935989,283.35935328)(515.95310905,283.12497851)(516.828125,282.81250164)
}
}
{
\newrgbcolor{curcolor}{0 0 0}
\pscustom[linestyle=none,fillstyle=solid,fillcolor=curcolor]
{
\newpath
\moveto(512.984375,442.57812473)
\curveto(514.49477717,442.25519581)(515.67185933,441.58332148)(516.515625,440.56249973)
\curveto(517.3697743,439.54165685)(517.7968572,438.28124145)(517.796875,436.78124973)
\curveto(517.7968572,434.47916192)(517.00519133,432.6979137)(515.421875,431.43749973)
\curveto(513.83852783,430.17708288)(511.58853008,429.54687518)(508.671875,429.54687473)
\curveto(507.69270064,429.54687518)(506.68228498,429.64583342)(505.640625,429.84374973)
\curveto(504.60937039,430.0312497)(503.54166313,430.31770774)(502.4375,430.70312473)
\lineto(502.4375,433.74999973)
\curveto(503.31249669,433.23957982)(504.27082906,432.85416354)(505.3125,432.59374973)
\curveto(506.35416031,432.33333073)(507.44270089,432.20312253)(508.578125,432.20312473)
\curveto(510.55728111,432.20312253)(512.06248794,432.59374713)(513.09375,433.37499973)
\curveto(514.13540253,434.15624557)(514.65623534,435.2916611)(514.65625,436.78124973)
\curveto(514.65623534,438.15624157)(514.17186083,439.22915717)(513.203125,439.99999973)
\curveto(512.24477942,440.78123895)(510.90623909,441.17186356)(509.1875,441.17187473)
\lineto(506.46875,441.17187473)
\lineto(506.46875,443.76562473)
\lineto(509.3125,443.76562473)
\curveto(510.86457247,443.76561096)(512.05207128,444.07290232)(512.875,444.68749973)
\curveto(513.69790297,445.31248442)(514.10936089,446.20831685)(514.109375,447.37499973)
\curveto(514.10936089,448.57289782)(513.68227798,449.48956357)(512.828125,450.12499973)
\curveto(511.98436302,450.77081229)(510.77082256,451.09372863)(509.1875,451.09374973)
\curveto(508.32290834,451.09372863)(507.39582594,450.99997873)(506.40625,450.81249973)
\curveto(505.41666125,450.6249791)(504.32812067,450.33331273)(503.140625,449.93749973)
\lineto(503.140625,452.74999973)
\curveto(504.33853733,453.08330998)(505.45832788,453.33330973)(506.5,453.49999973)
\curveto(507.55207578,453.66664273)(508.54165813,453.74997598)(509.46875,453.74999973)
\curveto(511.86457147,453.74997598)(513.76040291,453.20310153)(515.15625,452.10937473)
\curveto(516.55206678,451.02602037)(517.24998275,449.55727184)(517.25,447.70312473)
\curveto(517.24998275,446.41144165)(516.88019145,445.31769274)(516.140625,444.42187473)
\curveto(515.40102627,443.53644453)(514.34894398,442.92186181)(512.984375,442.57812473)
}
}
{
\newrgbcolor{curcolor}{0 0 0}
\pscustom[linestyle=none,fillstyle=solid,fillcolor=curcolor]
{
\newpath
\moveto(502.625,233.32812664)
\lineto(517.625,233.32812664)
\lineto(517.625,231.98437664)
\lineto(509.15625,210.00000164)
\lineto(505.859375,210.00000164)
\lineto(513.828125,230.67187664)
\lineto(502.625,230.67187664)
\lineto(502.625,233.32812664)
}
}
{
\newrgbcolor{curcolor}{0 0 0}
\pscustom[linestyle=none,fillstyle=solid,fillcolor=curcolor]
{
\newpath
\moveto(510.171875,181.07811138)
\curveto(508.67186633,181.0781003)(507.48957584,180.67705903)(506.625,179.87498638)
\curveto(505.77082756,179.07289397)(505.34374466,177.96872841)(505.34375,176.56248638)
\curveto(505.34374466,175.15623122)(505.77082756,174.05206566)(506.625,173.24998638)
\curveto(507.48957584,172.4479006)(508.67186633,172.04685933)(510.171875,172.04686138)
\curveto(511.67186333,172.04685933)(512.85415381,172.4479006)(513.71875,173.24998638)
\curveto(514.58331875,174.06248231)(515.01560998,175.16664788)(515.015625,176.56248638)
\curveto(515.01560998,177.96872841)(514.58331875,179.07289397)(513.71875,179.87498638)
\curveto(512.86457047,180.67705903)(511.68227998,181.0781003)(510.171875,181.07811138)
\moveto(507.015625,182.42186138)
\curveto(505.66145267,182.75518196)(504.60416206,183.38538966)(503.84375,184.31248638)
\curveto(503.09374691,185.23955447)(502.71874728,186.36976167)(502.71875,187.70311138)
\curveto(502.71874728,189.56767514)(503.38020495,191.041632)(504.703125,192.12498638)
\curveto(506.0364523,193.2082965)(507.85936714,193.74996263)(510.171875,193.74998638)
\curveto(512.49477917,193.74996263)(514.31769402,193.2082965)(515.640625,192.12498638)
\curveto(516.9635247,191.041632)(517.62498237,189.56767514)(517.625,187.70311138)
\curveto(517.62498237,186.36976167)(517.24477442,185.23955447)(516.484375,184.31248638)
\curveto(515.73435927,183.38538966)(514.68748531,182.75518196)(513.34375,182.42186138)
\curveto(514.86456847,182.06768264)(516.04685895,181.374975)(516.890625,180.34373638)
\curveto(517.74477392,179.31247706)(518.17185683,178.05206166)(518.171875,176.56248638)
\curveto(518.17185683,174.30206541)(517.47914919,172.56769214)(516.09375,171.35936138)
\curveto(514.71873528,170.15102789)(512.74477892,169.54686183)(510.171875,169.54686138)
\curveto(507.59895073,169.54686183)(505.61978605,170.15102789)(504.234375,171.35936138)
\curveto(502.85937214,172.56769214)(502.17187283,174.30206541)(502.171875,176.56248638)
\curveto(502.17187283,178.05206166)(502.59895573,179.31247706)(503.453125,180.34373638)
\curveto(504.30728736,181.374975)(505.49478617,182.06768264)(507.015625,182.42186138)
\moveto(505.859375,187.40623638)
\curveto(505.85936914,186.19788685)(506.23436877,185.25517946)(506.984375,184.57811138)
\curveto(507.74478392,183.90101414)(508.80728286,183.56247281)(510.171875,183.56248638)
\curveto(511.52603014,183.56247281)(512.58332075,183.90101414)(513.34375,184.57811138)
\curveto(514.11456922,185.25517946)(514.4999855,186.19788685)(514.5,187.40623638)
\curveto(514.4999855,188.6145511)(514.11456922,189.55725849)(513.34375,190.23436138)
\curveto(512.58332075,190.9114238)(511.52603014,191.24996513)(510.171875,191.24998638)
\curveto(508.80728286,191.24996513)(507.74478392,190.9114238)(506.984375,190.23436138)
\curveto(506.23436877,189.55725849)(505.85936914,188.6145511)(505.859375,187.40623638)
}
}
{
\newrgbcolor{curcolor}{0 0 0}
\pscustom[linewidth=2,linecolor=curcolor,linestyle=dashed,dash=8 8]
{
\newpath
\moveto(150,550)
\lineto(60,550)
}
}
{
\newrgbcolor{curcolor}{0 0 0}
\pscustom[linestyle=none,fillstyle=solid,fillcolor=curcolor]
{
\newpath
\moveto(139.53769464,554.84048224)
\lineto(152.6487474,550.01921591)
\lineto(139.53769392,545.19795064)
\curveto(141.632292,548.04442372)(141.62022288,551.93889292)(139.53769464,554.84048224)
\lineto(139.53769464,554.84048224)
\lineto(139.53769464,554.84048224)
\closepath
}
}
{
\newrgbcolor{curcolor}{0 0 0}
\pscustom[linewidth=2,linecolor=curcolor,linestyle=dashed,dash=8 8]
{
\newpath
\moveto(150,469.999997)
\lineto(60,469.999997)
}
}
{
\newrgbcolor{curcolor}{0 0 0}
\pscustom[linestyle=none,fillstyle=solid,fillcolor=curcolor]
{
\newpath
\moveto(139.53769464,474.84047924)
\lineto(152.6487474,470.01921291)
\lineto(139.53769392,465.19794764)
\curveto(141.632292,468.04442072)(141.62022288,471.93888992)(139.53769464,474.84047924)
\lineto(139.53769464,474.84047924)
\lineto(139.53769464,474.84047924)
\closepath
}
}
{
\newrgbcolor{curcolor}{0 0 0}
\pscustom[linewidth=2,linecolor=curcolor,linestyle=dashed,dash=8 8]
{
\newpath
\moveto(150,130)
\lineto(60,130)
}
}
{
\newrgbcolor{curcolor}{0 0 0}
\pscustom[linestyle=none,fillstyle=solid,fillcolor=curcolor]
{
\newpath
\moveto(139.53769464,134.84048224)
\lineto(152.6487474,130.01921591)
\lineto(139.53769392,125.19795064)
\curveto(141.632292,128.04442372)(141.62022288,131.93889292)(139.53769464,134.84048224)
\lineto(139.53769464,134.84048224)
\lineto(139.53769464,134.84048224)
\closepath
}
}
{
\newrgbcolor{curcolor}{0 0 0}
\pscustom[linewidth=2,linecolor=curcolor,linestyle=dashed,dash=8 8]
{
\newpath
\moveto(409.9375,439.999996)
\lineto(499.9375,439.999996)
}
}
{
\newrgbcolor{curcolor}{0 0 0}
\pscustom[linestyle=none,fillstyle=solid,fillcolor=curcolor]
{
\newpath
\moveto(420.39980536,435.15951376)
\lineto(407.2887526,439.98078009)
\lineto(420.39980608,444.80204536)
\curveto(418.305208,441.95557228)(418.31727712,438.06110308)(420.39980536,435.15951376)
\lineto(420.39980536,435.15951376)
\lineto(420.39980536,435.15951376)
\closepath
}
}
{
\newrgbcolor{curcolor}{0 0 0}
\pscustom[linewidth=2,linecolor=curcolor,linestyle=dashed,dash=8 8]
{
\newpath
\moveto(60,380)
\lineto(180,380)
}
}
{
\newrgbcolor{curcolor}{0 0 0}
\pscustom[linestyle=none,fillstyle=solid,fillcolor=curcolor]
{
\newpath
\moveto(169.53769464,384.84048224)
\lineto(182.6487474,380.01921591)
\lineto(169.53769392,375.19795064)
\curveto(171.632292,378.04442372)(171.62022288,381.93889292)(169.53769464,384.84048224)
\closepath
}
}
{
\newrgbcolor{curcolor}{0 0 0}
\pscustom[linewidth=2,linecolor=curcolor,linestyle=dashed,dash=8 8]
{
\newpath
\moveto(400,320)
\lineto(500,320)
}
}
{
\newrgbcolor{curcolor}{0 0 0}
\pscustom[linestyle=none,fillstyle=solid,fillcolor=curcolor]
{
\newpath
\moveto(410.46230536,315.15951776)
\lineto(397.3512526,319.98078409)
\lineto(410.46230608,324.80204936)
\curveto(408.367708,321.95557628)(408.37977712,318.06110708)(410.46230536,315.15951776)
\lineto(410.46230536,315.15951776)
\lineto(410.46230536,315.15951776)
\closepath
}
}
{
\newrgbcolor{curcolor}{0 0 0}
\pscustom[linewidth=2,linecolor=curcolor,linestyle=dashed,dash=8 8]
{
\newpath
\moveto(400.04545,270)
\lineto(500.04545,270)
}
}
{
\newrgbcolor{curcolor}{0 0 0}
\pscustom[linestyle=none,fillstyle=solid,fillcolor=curcolor]
{
\newpath
\moveto(410.50775536,265.15951776)
\lineto(397.3967026,269.98078409)
\lineto(410.50775608,274.80204936)
\curveto(408.413158,271.95557628)(408.42522712,268.06110708)(410.50775536,265.15951776)
\lineto(410.50775536,265.15951776)
\lineto(410.50775536,265.15951776)
\closepath
}
}
{
\newrgbcolor{curcolor}{0 0 0}
\pscustom[linewidth=2,linecolor=curcolor,linestyle=dashed,dash=8 8]
{
\newpath
\moveto(410,130)
\lineto(500,130)
}
}
{
\newrgbcolor{curcolor}{0 0 0}
\pscustom[linestyle=none,fillstyle=solid,fillcolor=curcolor]
{
\newpath
\moveto(420.46230536,125.15951776)
\lineto(407.3512526,129.98078409)
\lineto(420.46230608,134.80204936)
\curveto(418.367708,131.95557628)(418.37977712,128.06110708)(420.46230536,125.15951776)
\lineto(420.46230536,125.15951776)
\lineto(420.46230536,125.15951776)
\closepath
}
}
{
\newrgbcolor{curcolor}{1 1 1}
\pscustom[linestyle=none,fillstyle=solid,fillcolor=curcolor]
{
\newpath
\moveto(320,230)
\lineto(340,230)
\lineto(340,210)
\lineto(320,210)
\closepath
}
}
{
\newrgbcolor{curcolor}{0 0 0}
\pscustom[linewidth=2,linecolor=curcolor]
{
\newpath
\moveto(320,230)
\lineto(340,230)
\lineto(340,210)
\lineto(320,210)
\closepath
}
}
{
\newrgbcolor{curcolor}{0 0 0}
\pscustom[linewidth=2,linecolor=curcolor,linestyle=dashed,dash=8 8]
{
\newpath
\moveto(330,220)
\lineto(500,220)
}
}
{
\newrgbcolor{curcolor}{0 0 0}
\pscustom[linestyle=none,fillstyle=solid,fillcolor=curcolor]
{
\newpath
\moveto(340.46230536,215.15951776)
\lineto(327.3512526,219.98078409)
\lineto(340.46230608,224.80204936)
\curveto(338.367708,221.95557628)(338.37977712,218.06110708)(340.46230536,215.15951776)
\lineto(340.46230536,215.15951776)
\lineto(340.46230536,215.15951776)
\closepath
}
}
{
\newrgbcolor{curcolor}{0 0 0}
\pscustom[linestyle=none,fillstyle=solid,fillcolor=curcolor]
{
\newpath
\moveto(43.515625,120.48436138)
\lineto(43.515625,123.35936138)
\curveto(44.30728736,122.98435839)(45.10936989,122.69790035)(45.921875,122.49998638)
\curveto(46.73436827,122.30206741)(47.53124247,122.20310917)(48.3125,122.20311138)
\curveto(50.39582294,122.20310917)(51.98436302,122.90102514)(53.078125,124.29686138)
\curveto(54.18227748,125.70310567)(54.81248519,127.83331188)(54.96875,130.68748638)
\curveto(54.36456897,129.79164325)(53.59894473,129.10414394)(52.671875,128.62498638)
\curveto(51.74477992,128.14581156)(50.71873928,127.90622847)(49.59375,127.90623638)
\curveto(47.26040941,127.90622847)(45.41145292,128.60935277)(44.046875,130.01561138)
\curveto(42.69270564,131.43226661)(42.01562298,133.36455635)(42.015625,135.81248638)
\curveto(42.01562298,138.2083015)(42.72395561,140.13017458)(44.140625,141.57811138)
\curveto(45.55728611,143.02600502)(47.44270089,143.74996263)(49.796875,143.74998638)
\curveto(52.49477917,143.74996263)(54.55206878,142.71350533)(55.96875,140.64061138)
\curveto(57.39581594,138.5780928)(58.10935689,135.5780958)(58.109375,131.64061138)
\curveto(58.10935689,127.96352008)(57.23435777,125.02602302)(55.484375,122.82811138)
\curveto(53.74477792,120.64061074)(51.40103027,119.54686183)(48.453125,119.54686138)
\curveto(47.66145067,119.54686183)(46.85936814,119.62498675)(46.046875,119.78123638)
\curveto(45.23436977,119.93748644)(44.39062061,120.17186121)(43.515625,120.48436138)
\moveto(49.796875,130.37498638)
\curveto(51.21353045,130.374976)(52.333321,130.85935052)(53.15625,131.82811138)
\curveto(53.98956934,132.79684858)(54.40623559,134.12497225)(54.40625,135.81248638)
\curveto(54.40623559,137.48955222)(53.98956934,138.81246756)(53.15625,139.78123638)
\curveto(52.333321,140.76038228)(51.21353045,141.24996513)(49.796875,141.24998638)
\curveto(48.38019995,141.24996513)(47.25520108,140.76038228)(46.421875,139.78123638)
\curveto(45.59895273,138.81246756)(45.18749481,137.48955222)(45.1875,135.81248638)
\curveto(45.18749481,134.12497225)(45.59895273,132.79684858)(46.421875,131.82811138)
\curveto(47.25520108,130.85935052)(48.38019995,130.374976)(49.796875,130.37498638)
}
}
{
\newrgbcolor{curcolor}{1 1 1}
\pscustom[linestyle=none,fillstyle=solid,fillcolor=curcolor]
{
\newpath
\moveto(410,189.99998638)
\lineto(430,189.99998638)
\lineto(430,169.99998638)
\lineto(410,169.99998638)
\closepath
}
}
{
\newrgbcolor{curcolor}{0 0 0}
\pscustom[linewidth=2,linecolor=curcolor]
{
\newpath
\moveto(410,189.99998638)
\lineto(430,189.99998638)
\lineto(430,169.99998638)
\lineto(410,169.99998638)
\closepath
}
}
{
\newrgbcolor{curcolor}{0 0 0}
\pscustom[linewidth=2,linecolor=curcolor,linestyle=dashed,dash=8 8]
{
\newpath
\moveto(420,180)
\lineto(500,180)
}
}
{
\newrgbcolor{curcolor}{0 0 0}
\pscustom[linestyle=none,fillstyle=solid,fillcolor=curcolor]
{
\newpath
\moveto(430.46230536,175.15951776)
\lineto(417.3512526,179.98078409)
\lineto(430.46230608,184.80204936)
\curveto(428.367708,181.95557628)(428.37977712,178.06110708)(430.46230536,175.15951776)
\lineto(430.46230536,175.15951776)
\lineto(430.46230536,175.15951776)
\closepath
}
}
{
\newrgbcolor{curcolor}{0 0 0}
\pscustom[linestyle=none,fillstyle=solid,fillcolor=curcolor]
{
\newpath
\moveto(503.96875,122.65623638)
\lineto(509.125,122.65623638)
\lineto(509.125,140.45311138)
\lineto(503.515625,139.32811138)
\lineto(503.515625,142.20311138)
\lineto(509.09375,143.32811138)
\lineto(512.25,143.32811138)
\lineto(512.25,122.65623638)
\lineto(517.40625,122.65623638)
\lineto(517.40625,119.99998638)
\lineto(503.96875,119.99998638)
\lineto(503.96875,122.65623638)
}
}
{
\newrgbcolor{curcolor}{0 0 0}
\pscustom[linestyle=none,fillstyle=solid,fillcolor=curcolor]
{
\newpath
\moveto(530.546875,141.24998638)
\curveto(528.92186645,141.24996513)(527.69790934,140.4478826)(526.875,138.84373638)
\curveto(526.06249431,137.24996913)(525.65624472,134.84892986)(525.65625,131.64061138)
\curveto(525.65624472,128.44268627)(526.06249431,126.041647)(526.875,124.43748638)
\curveto(527.69790934,122.84373353)(528.92186645,122.04685933)(530.546875,122.04686138)
\curveto(532.18227986,122.04685933)(533.40623697,122.84373353)(534.21875,124.43748638)
\curveto(535.041652,126.041647)(535.45310992,128.44268627)(535.453125,131.64061138)
\curveto(535.45310992,134.84892986)(535.041652,137.24996913)(534.21875,138.84373638)
\curveto(533.40623697,140.4478826)(532.18227986,141.24996513)(530.546875,141.24998638)
\moveto(530.546875,143.74998638)
\curveto(533.16144555,143.74996263)(535.15623522,142.71350533)(536.53125,140.64061138)
\curveto(537.91664913,138.5780928)(538.60935677,135.5780958)(538.609375,131.64061138)
\curveto(538.60935677,127.71352033)(537.91664913,124.71352333)(536.53125,122.64061138)
\curveto(535.15623522,120.5781108)(533.16144555,119.54686183)(530.546875,119.54686138)
\curveto(527.93228411,119.54686183)(525.93228611,120.5781108)(524.546875,122.64061138)
\curveto(523.1718722,124.71352333)(522.48437289,127.71352033)(522.484375,131.64061138)
\curveto(522.48437289,135.5780958)(523.1718722,138.5780928)(524.546875,140.64061138)
\curveto(525.93228611,142.71350533)(527.93228411,143.74996263)(530.546875,143.74998638)
}
}
{
\newrgbcolor{curcolor}{0 0 0}
\pscustom[linestyle=none,fillstyle=solid,fillcolor=curcolor]
{
\newpath
\moveto(176.48535156,336.03808757)
\lineto(179.40722656,336.03808757)
\lineto(186.51855469,322.62109539)
\lineto(186.51855469,336.03808757)
\lineto(188.62402344,336.03808757)
\lineto(188.62402344,320.00000164)
\lineto(185.70214844,320.00000164)
\lineto(178.59082031,333.41699382)
\lineto(178.59082031,320.00000164)
\lineto(176.48535156,320.00000164)
\lineto(176.48535156,336.03808757)
}
}
{
\newrgbcolor{curcolor}{0 0 0}
\pscustom[linestyle=none,fillstyle=solid,fillcolor=curcolor]
{
\newpath
\moveto(197.51855469,330.64550945)
\curveto(196.45865318,330.6454988)(195.62076339,330.23013463)(195.00488281,329.3994157)
\curveto(194.38899379,328.57583941)(194.08105139,327.44433013)(194.08105469,326.00488445)
\curveto(194.08105139,324.56542676)(194.38541306,323.43033675)(194.99414062,322.59961101)
\curveto(195.61002121,321.77604153)(196.45149173,321.36425808)(197.51855469,321.36425945)
\curveto(198.57128127,321.36425808)(199.40559034,321.77962225)(200.02148438,322.6103532)
\curveto(200.63735994,323.44107892)(200.94530234,324.57258821)(200.9453125,326.00488445)
\curveto(200.94530234,327.43000723)(200.63735994,328.55793579)(200.02148438,329.38867351)
\curveto(199.40559034,330.22655391)(198.57128127,330.6454988)(197.51855469,330.64550945)
\moveto(197.51855469,332.3212907)
\curveto(199.23729623,332.32127838)(200.58722978,331.76268519)(201.56835938,330.64550945)
\curveto(202.5494674,329.52831242)(203.04002681,327.98143897)(203.04003906,326.00488445)
\curveto(203.04002681,324.03547937)(202.5494674,322.48860592)(201.56835938,321.36425945)
\curveto(200.58722978,320.2470717)(199.23729623,319.68847851)(197.51855469,319.6884782)
\curveto(195.79263822,319.68847851)(194.43912395,320.2470717)(193.45800781,321.36425945)
\curveto(192.48404778,322.48860592)(191.9970691,324.03547937)(191.99707031,326.00488445)
\curveto(191.9970691,327.98143897)(192.48404778,329.52831242)(193.45800781,330.64550945)
\curveto(194.43912395,331.76268519)(195.79263822,332.32127838)(197.51855469,332.3212907)
}
}
{
\newrgbcolor{curcolor}{0 0 0}
\pscustom[linestyle=none,fillstyle=solid,fillcolor=curcolor]
{
\newpath
\moveto(215.67285156,329.72168132)
\curveto(216.16698025,330.60969155)(216.75779997,331.26496433)(217.4453125,331.68750164)
\curveto(218.1327986,332.11001557)(218.94204258,332.32127838)(219.87304688,332.3212907)
\curveto(221.12628519,332.32127838)(222.0930811,331.88084913)(222.7734375,331.00000164)
\curveto(223.45375682,330.12629359)(223.79392575,328.88020109)(223.79394531,327.26172039)
\lineto(223.79394531,320.00000164)
\lineto(221.80664062,320.00000164)
\lineto(221.80664062,327.19726726)
\curveto(221.80662305,328.3502537)(221.60252169,329.20604712)(221.19433594,329.76465007)
\curveto(220.78611626,330.3232335)(220.16307001,330.6025301)(219.32519531,330.6025407)
\curveto(218.3010927,330.6025301)(217.49184872,330.26236117)(216.89746094,329.58203289)
\curveto(216.30304783,328.90168544)(216.0058476,327.97427752)(216.00585938,326.79980632)
\lineto(216.00585938,320.00000164)
\lineto(214.01855469,320.00000164)
\lineto(214.01855469,327.19726726)
\curveto(214.0185449,328.35741515)(213.81444354,329.21320857)(213.40625,329.76465007)
\curveto(212.99803811,330.3232335)(212.36783041,330.6025301)(211.515625,330.6025407)
\curveto(210.5058531,330.6025301)(209.70377057,330.25878044)(209.109375,329.5712907)
\curveto(208.51496968,328.89094327)(208.21776945,327.96711606)(208.21777344,326.79980632)
\lineto(208.21777344,320.00000164)
\lineto(206.23046875,320.00000164)
\lineto(206.23046875,332.03125164)
\lineto(208.21777344,332.03125164)
\lineto(208.21777344,330.16211101)
\curveto(208.66894088,330.89973032)(209.20963044,331.44400061)(209.83984375,331.79492351)
\curveto(210.47004585,332.14582282)(211.21841749,332.32127838)(212.08496094,332.3212907)
\curveto(212.95865013,332.32127838)(213.69986032,332.09927339)(214.30859375,331.65527507)
\curveto(214.92446847,331.21125345)(215.37922062,330.56672284)(215.67285156,329.72168132)
}
}
{
\newrgbcolor{curcolor}{0 0 0}
\pscustom[linestyle=none,fillstyle=solid,fillcolor=curcolor]
{
\newpath
\moveto(144.44140625,302.70507976)
\lineto(144.44140625,309.21484539)
\lineto(146.41796875,309.21484539)
\lineto(146.41796875,292.50000164)
\lineto(144.44140625,292.50000164)
\lineto(144.44140625,294.30468914)
\curveto(144.02603209,293.58854221)(143.49966543,293.0550141)(142.86230469,292.7041032)
\curveto(142.23208857,292.36035334)(141.47297475,292.18847851)(140.58496094,292.1884782)
\curveto(139.13118022,292.18847851)(137.94596005,292.76855605)(137.02929688,293.92871257)
\curveto(136.11979,295.08886623)(135.66503785,296.61425533)(135.66503906,298.50488445)
\curveto(135.66503785,300.39550155)(136.11979,301.92089065)(137.02929688,303.08105632)
\curveto(137.94596005,304.24120083)(139.13118022,304.82127838)(140.58496094,304.8212907)
\curveto(141.47297475,304.82127838)(142.23208857,304.64582282)(142.86230469,304.29492351)
\curveto(143.49966543,303.95116206)(144.02603209,303.42121467)(144.44140625,302.70507976)
\moveto(137.70605469,298.50488445)
\curveto(137.70605143,297.05110386)(138.00325166,295.90885239)(138.59765625,295.07812664)
\curveto(139.199214,294.25455717)(140.02278089,293.84277373)(141.06835938,293.84277507)
\curveto(142.11392463,293.84277373)(142.93749151,294.25455717)(143.5390625,295.07812664)
\curveto(144.14061531,295.90885239)(144.44139626,297.05110386)(144.44140625,298.50488445)
\curveto(144.44139626,299.95865303)(144.14061531,301.09732377)(143.5390625,301.92090007)
\curveto(142.93749151,302.75161899)(142.11392463,303.16698316)(141.06835938,303.16699382)
\curveto(140.02278089,303.16698316)(139.199214,302.75161899)(138.59765625,301.92090007)
\curveto(138.00325166,301.09732377)(137.70605143,299.95865303)(137.70605469,298.50488445)
}
}
{
\newrgbcolor{curcolor}{0 0 0}
\pscustom[linestyle=none,fillstyle=solid,fillcolor=curcolor]
{
\newpath
\moveto(152.35839844,308.53808757)
\lineto(152.35839844,302.57617351)
\lineto(150.53222656,302.57617351)
\lineto(150.53222656,308.53808757)
\lineto(152.35839844,308.53808757)
}
}
{
\newrgbcolor{curcolor}{0 0 0}
\pscustom[linestyle=none,fillstyle=solid,fillcolor=curcolor]
{
\newpath
\moveto(156.34375,297.24804851)
\lineto(156.34375,304.53125164)
\lineto(158.3203125,304.53125164)
\lineto(158.3203125,297.32324382)
\curveto(158.32030865,296.18456826)(158.54231364,295.32877485)(158.98632812,294.75586101)
\curveto(159.43033359,294.19010411)(160.09634854,293.90722679)(160.984375,293.9072282)
\curveto(162.05142471,293.90722679)(162.89289523,294.24739572)(163.50878906,294.92773601)
\curveto(164.13182628,295.60807144)(164.44334941,296.53547937)(164.44335938,297.70996257)
\lineto(164.44335938,304.53125164)
\lineto(166.41992188,304.53125164)
\lineto(166.41992188,292.50000164)
\lineto(164.44335938,292.50000164)
\lineto(164.44335938,294.34765789)
\curveto(163.96353218,293.61718802)(163.40493899,293.07291773)(162.76757812,292.71484539)
\curveto(162.13736213,292.36393406)(161.40331338,292.18847851)(160.56542969,292.1884782)
\curveto(159.18326352,292.18847851)(158.13411092,292.61816558)(157.41796875,293.4775407)
\curveto(156.70182069,294.33691386)(156.34374813,295.59374854)(156.34375,297.24804851)
\moveto(161.31738281,304.8212907)
\lineto(161.31738281,304.8212907)
}
}
{
\newrgbcolor{curcolor}{0 0 0}
\pscustom[linestyle=none,fillstyle=solid,fillcolor=curcolor]
{
\newpath
\moveto(172.46777344,307.94726726)
\lineto(172.46777344,304.53125164)
\lineto(176.5390625,304.53125164)
\lineto(176.5390625,302.99511882)
\lineto(172.46777344,302.99511882)
\lineto(172.46777344,296.46386882)
\curveto(172.46776941,295.48274605)(172.60025626,294.85253835)(172.86523438,294.57324382)
\curveto(173.13736509,294.29394515)(173.68521611,294.15429686)(174.50878906,294.15429851)
\lineto(176.5390625,294.15429851)
\lineto(176.5390625,292.50000164)
\lineto(174.50878906,292.50000164)
\curveto(172.98339389,292.50000164)(171.93066057,292.78287896)(171.35058594,293.34863445)
\curveto(170.77050548,293.92154969)(170.48046671,294.95996011)(170.48046875,296.46386882)
\lineto(170.48046875,302.99511882)
\lineto(169.03027344,302.99511882)
\lineto(169.03027344,304.53125164)
\lineto(170.48046875,304.53125164)
\lineto(170.48046875,307.94726726)
\lineto(172.46777344,307.94726726)
}
}
{
\newrgbcolor{curcolor}{0 0 0}
\pscustom[linestyle=none,fillstyle=solid,fillcolor=curcolor]
{
\newpath
\moveto(179.14941406,304.53125164)
\lineto(181.12597656,304.53125164)
\lineto(181.12597656,292.50000164)
\lineto(179.14941406,292.50000164)
\lineto(179.14941406,304.53125164)
\moveto(179.14941406,309.21484539)
\lineto(181.12597656,309.21484539)
\lineto(181.12597656,306.7119157)
\lineto(179.14941406,306.7119157)
\lineto(179.14941406,309.21484539)
}
}
{
\newrgbcolor{curcolor}{0 0 0}
\pscustom[linestyle=none,fillstyle=solid,fillcolor=curcolor]
{
\newpath
\moveto(185.25097656,309.21484539)
\lineto(187.22753906,309.21484539)
\lineto(187.22753906,292.50000164)
\lineto(185.25097656,292.50000164)
\lineto(185.25097656,309.21484539)
}
}
{
\newrgbcolor{curcolor}{0 0 0}
\pscustom[linestyle=none,fillstyle=solid,fillcolor=curcolor]
{
\newpath
\moveto(191.35253906,304.53125164)
\lineto(193.32910156,304.53125164)
\lineto(193.32910156,292.50000164)
\lineto(191.35253906,292.50000164)
\lineto(191.35253906,304.53125164)
\moveto(191.35253906,309.21484539)
\lineto(193.32910156,309.21484539)
\lineto(193.32910156,306.7119157)
\lineto(191.35253906,306.7119157)
\lineto(191.35253906,309.21484539)
}
}
{
\newrgbcolor{curcolor}{0 0 0}
\pscustom[linestyle=none,fillstyle=solid,fillcolor=curcolor]
{
\newpath
\moveto(205.12402344,304.17675945)
\lineto(205.12402344,302.30761882)
\curveto(204.5654205,302.59406706)(203.98534296,302.8089106)(203.38378906,302.95215007)
\curveto(202.78221916,303.09536864)(202.15917291,303.16698316)(201.51464844,303.16699382)
\curveto(200.53352349,303.16698316)(199.79589402,303.01659268)(199.30175781,302.71582195)
\curveto(198.81477521,302.41503078)(198.57128587,301.96385936)(198.57128906,301.36230632)
\curveto(198.57128587,300.90396459)(198.74674143,300.5423113)(199.09765625,300.27734539)
\curveto(199.44856364,300.01952537)(200.15396658,299.7724553)(201.21386719,299.53613445)
\lineto(201.890625,299.38574382)
\curveto(203.29426292,299.08495599)(204.28970463,298.65884964)(204.87695312,298.10742351)
\curveto(205.47134408,297.56314761)(205.7685443,296.80045306)(205.76855469,295.81933757)
\curveto(205.7685443,294.70214787)(205.32453433,293.81770865)(204.43652344,293.16601726)
\curveto(203.55565589,292.51432454)(202.34178991,292.18847851)(200.79492188,292.1884782)
\curveto(200.15038586,292.18847851)(199.47720945,292.25293157)(198.77539062,292.38183757)
\curveto(198.08072647,292.50358236)(197.34667772,292.68978009)(196.57324219,292.94043132)
\lineto(196.57324219,294.98144695)
\curveto(197.30370901,294.60188755)(198.02343486,294.31542951)(198.73242188,294.12207195)
\curveto(199.44140219,293.9358726)(200.1432244,293.84277373)(200.83789062,293.84277507)
\curveto(201.76887382,293.84277373)(202.48501894,294.00032566)(202.98632812,294.31543132)
\curveto(203.4876221,294.63769481)(203.73827289,295.08886623)(203.73828125,295.66894695)
\curveto(203.73827289,296.20605262)(203.55565589,296.61783606)(203.19042969,296.90429851)
\curveto(202.83234932,297.19075215)(202.04100896,297.46646802)(200.81640625,297.73144695)
\lineto(200.12890625,297.89257976)
\curveto(198.90429335,298.15038661)(198.01985413,298.54426642)(197.47558594,299.07422039)
\curveto(196.93131355,299.61132265)(196.65917841,300.34537139)(196.65917969,301.27636882)
\curveto(196.65917841,302.40786933)(197.06021967,303.28156638)(197.86230469,303.89746257)
\curveto(198.66438474,304.51333598)(199.80305547,304.82127838)(201.27832031,304.8212907)
\curveto(202.00878243,304.82127838)(202.69628175,304.76756749)(203.34082031,304.66015789)
\curveto(203.98534296,304.55272396)(204.57974341,304.39159131)(205.12402344,304.17675945)
}
}
{
\newrgbcolor{curcolor}{0 0 0}
\pscustom[linestyle=none,fillstyle=solid,fillcolor=curcolor]
{
\newpath
\moveto(214.39453125,298.5478532)
\curveto(212.7975201,298.54784715)(211.69107589,298.36523015)(211.07519531,298.00000164)
\curveto(210.45930629,297.63476213)(210.15136389,297.01171587)(210.15136719,296.13086101)
\curveto(210.15136389,295.42903517)(210.38053033,294.87044197)(210.83886719,294.45507976)
\curveto(211.30435753,294.04687509)(211.93456523,293.84277373)(212.72949219,293.84277507)
\curveto(213.82518834,293.84277373)(214.70246611,294.22949209)(215.36132812,295.00293132)
\curveto(216.02733458,295.783527)(216.36034206,296.81835669)(216.36035156,298.10742351)
\lineto(216.36035156,298.5478532)
\lineto(214.39453125,298.5478532)
\moveto(218.33691406,299.36425945)
\lineto(218.33691406,292.50000164)
\lineto(216.36035156,292.50000164)
\lineto(216.36035156,294.32617351)
\curveto(215.90917063,293.59570367)(215.34699671,293.0550141)(214.67382812,292.7041032)
\curveto(214.00064389,292.36035334)(213.17707701,292.18847851)(212.203125,292.1884782)
\curveto(210.97135005,292.18847851)(209.99023124,292.53222817)(209.25976562,293.2197282)
\curveto(208.53645665,293.91438824)(208.17480337,294.84179617)(208.17480469,296.00195476)
\curveto(208.17480337,297.35546553)(208.62597479,298.37597232)(209.52832031,299.0634782)
\curveto(210.43782194,299.75097095)(211.79133621,300.0947206)(213.58886719,300.0947282)
\lineto(216.36035156,300.0947282)
\lineto(216.36035156,300.28808757)
\curveto(216.36034206,301.19758408)(216.05956111,301.8994063)(215.45800781,302.39355632)
\curveto(214.86359876,302.89484801)(214.02570897,303.1454988)(212.94433594,303.14550945)
\curveto(212.25683053,303.1454988)(211.58723485,303.06314211)(210.93554688,302.89843914)
\curveto(210.28385074,302.73371536)(209.65722376,302.48664529)(209.05566406,302.1572282)
\lineto(209.05566406,303.98340007)
\curveto(209.77896843,304.26268519)(210.48079064,304.47036727)(211.16113281,304.60644695)
\curveto(211.84146637,304.74966387)(212.5039006,304.82127838)(213.1484375,304.8212907)
\curveto(214.88866384,304.82127838)(216.18846723,304.37010695)(217.04785156,303.46777507)
\curveto(217.90721551,302.56542126)(218.33690258,301.19758408)(218.33691406,299.36425945)
}
}
{
\newrgbcolor{curcolor}{0 0 0}
\pscustom[linestyle=none,fillstyle=solid,fillcolor=curcolor]
{
\newpath
\moveto(224.37402344,307.94726726)
\lineto(224.37402344,304.53125164)
\lineto(228.4453125,304.53125164)
\lineto(228.4453125,302.99511882)
\lineto(224.37402344,302.99511882)
\lineto(224.37402344,296.46386882)
\curveto(224.37401941,295.48274605)(224.50650626,294.85253835)(224.77148438,294.57324382)
\curveto(225.04361509,294.29394515)(225.59146611,294.15429686)(226.41503906,294.15429851)
\lineto(228.4453125,294.15429851)
\lineto(228.4453125,292.50000164)
\lineto(226.41503906,292.50000164)
\curveto(224.88964389,292.50000164)(223.83691057,292.78287896)(223.25683594,293.34863445)
\curveto(222.67675548,293.92154969)(222.38671671,294.95996011)(222.38671875,296.46386882)
\lineto(222.38671875,302.99511882)
\lineto(220.93652344,302.99511882)
\lineto(220.93652344,304.53125164)
\lineto(222.38671875,304.53125164)
\lineto(222.38671875,307.94726726)
\lineto(224.37402344,307.94726726)
}
}
{
\newrgbcolor{curcolor}{0 0 0}
\pscustom[linestyle=none,fillstyle=solid,fillcolor=curcolor]
{
\newpath
\moveto(241.34667969,299.00976726)
\lineto(241.34667969,298.04297039)
\lineto(232.25878906,298.04297039)
\curveto(232.3447232,296.68228912)(232.75292592,295.6438787)(233.48339844,294.92773601)
\curveto(234.22102341,294.21874992)(235.24511092,293.86425808)(236.55566406,293.86425945)
\curveto(237.31477031,293.86425808)(238.04881906,293.95735695)(238.7578125,294.14355632)
\curveto(239.47394784,294.32975241)(240.18293151,294.60904901)(240.88476562,294.98144695)
\lineto(240.88476562,293.11230632)
\curveto(240.17577006,292.81152476)(239.44888276,292.58235832)(238.70410156,292.42480632)
\curveto(237.95930092,292.26725447)(237.20376782,292.18847851)(236.4375,292.1884782)
\curveto(234.51822363,292.18847851)(232.99641526,292.7470717)(231.87207031,293.86425945)
\curveto(230.75488104,294.98144447)(230.19628785,296.49251066)(230.19628906,298.39746257)
\curveto(230.19628785,300.36685575)(230.72623524,301.9280521)(231.78613281,303.08105632)
\curveto(232.85318623,304.24120083)(234.28905719,304.82127838)(236.09375,304.8212907)
\curveto(237.71223085,304.82127838)(238.99054989,304.29849244)(239.92871094,303.25293132)
\curveto(240.87401155,302.21451015)(241.34666732,300.80012354)(241.34667969,299.00976726)
\moveto(239.37011719,299.58984539)
\curveto(239.3557839,300.67121742)(239.05142222,301.53417229)(238.45703125,302.17871257)
\curveto(237.86978278,302.8232335)(237.0891846,303.1454988)(236.11523438,303.14550945)
\curveto(235.01236376,303.1454988)(234.12792454,302.83397568)(233.46191406,302.21093914)
\curveto(232.80305608,301.58788317)(232.42349916,300.7106054)(232.32324219,299.5791032)
\lineto(239.37011719,299.58984539)
}
}
{
\newrgbcolor{curcolor}{0 0 0}
\pscustom[linestyle=none,fillstyle=solid,fillcolor=curcolor]
{
\newpath
\moveto(244.38671875,297.24804851)
\lineto(244.38671875,304.53125164)
\lineto(246.36328125,304.53125164)
\lineto(246.36328125,297.32324382)
\curveto(246.3632774,296.18456826)(246.58528239,295.32877485)(247.02929688,294.75586101)
\curveto(247.47330234,294.19010411)(248.13931729,293.90722679)(249.02734375,293.9072282)
\curveto(250.09439346,293.90722679)(250.93586398,294.24739572)(251.55175781,294.92773601)
\curveto(252.17479503,295.60807144)(252.48631816,296.53547937)(252.48632812,297.70996257)
\lineto(252.48632812,304.53125164)
\lineto(254.46289062,304.53125164)
\lineto(254.46289062,292.50000164)
\lineto(252.48632812,292.50000164)
\lineto(252.48632812,294.34765789)
\curveto(252.00650093,293.61718802)(251.44790774,293.07291773)(250.81054688,292.71484539)
\curveto(250.18033088,292.36393406)(249.44628213,292.18847851)(248.60839844,292.1884782)
\curveto(247.22623227,292.18847851)(246.17707967,292.61816558)(245.4609375,293.4775407)
\curveto(244.74478944,294.33691386)(244.38671688,295.59374854)(244.38671875,297.24804851)
\moveto(249.36035156,304.8212907)
\lineto(249.36035156,304.8212907)
}
}
{
\newrgbcolor{curcolor}{0 0 0}
\pscustom[linestyle=none,fillstyle=solid,fillcolor=curcolor]
{
\newpath
\moveto(265.52734375,302.68359539)
\curveto(265.30532972,302.81249132)(265.06184038,302.90559019)(264.796875,302.96289226)
\curveto(264.53905444,303.02733486)(264.2525964,303.05956139)(263.9375,303.05957195)
\curveto(262.82030616,303.05956139)(261.96093202,302.69432738)(261.359375,301.96386882)
\curveto(260.76496968,301.24055279)(260.46776945,300.19856165)(260.46777344,298.83789226)
\lineto(260.46777344,292.50000164)
\lineto(258.48046875,292.50000164)
\lineto(258.48046875,304.53125164)
\lineto(260.46777344,304.53125164)
\lineto(260.46777344,302.66211101)
\curveto(260.88313362,303.39256887)(261.42382318,303.93325843)(262.08984375,304.28418132)
\curveto(262.7558531,304.6422421)(263.56509708,304.82127838)(264.51757812,304.8212907)
\curveto(264.65363766,304.82127838)(264.80402814,304.8105362)(264.96875,304.78906414)
\curveto(265.13345489,304.77472894)(265.3160719,304.74966387)(265.51660156,304.71386882)
\lineto(265.52734375,302.68359539)
}
}
{
\newrgbcolor{curcolor}{0 0 0}
\pscustom[linestyle=none,fillstyle=solid,fillcolor=curcolor]
{
\newpath
\moveto(132.15917969,276.03808757)
\lineto(135.39257812,276.03808757)
\lineto(139.48535156,265.12402507)
\lineto(143.59960938,276.03808757)
\lineto(146.83300781,276.03808757)
\lineto(146.83300781,260.00000164)
\lineto(144.71679688,260.00000164)
\lineto(144.71679688,274.08300945)
\lineto(140.58105469,263.08300945)
\lineto(138.40039062,263.08300945)
\lineto(134.26464844,274.08300945)
\lineto(134.26464844,260.00000164)
\lineto(132.15917969,260.00000164)
\lineto(132.15917969,276.03808757)
}
}
{
\newrgbcolor{curcolor}{0 0 0}
\pscustom[linestyle=none,fillstyle=solid,fillcolor=curcolor]
{
\newpath
\moveto(155.72753906,270.64550945)
\curveto(154.66763755,270.6454988)(153.82974777,270.23013463)(153.21386719,269.3994157)
\curveto(152.59797817,268.57583941)(152.29003576,267.44433013)(152.29003906,266.00488445)
\curveto(152.29003576,264.56542676)(152.59439744,263.43033675)(153.203125,262.59961101)
\curveto(153.81900559,261.77604153)(154.6604761,261.36425808)(155.72753906,261.36425945)
\curveto(156.78026565,261.36425808)(157.61457471,261.77962225)(158.23046875,262.6103532)
\curveto(158.84634431,263.44107892)(159.15428671,264.57258821)(159.15429688,266.00488445)
\curveto(159.15428671,267.43000723)(158.84634431,268.55793579)(158.23046875,269.38867351)
\curveto(157.61457471,270.22655391)(156.78026565,270.6454988)(155.72753906,270.64550945)
\moveto(155.72753906,272.3212907)
\curveto(157.44628061,272.32127838)(158.79621415,271.76268519)(159.77734375,270.64550945)
\curveto(160.75845178,269.52831242)(161.24901118,267.98143897)(161.24902344,266.00488445)
\curveto(161.24901118,264.03547937)(160.75845178,262.48860592)(159.77734375,261.36425945)
\curveto(158.79621415,260.2470717)(157.44628061,259.68847851)(155.72753906,259.6884782)
\curveto(154.00162259,259.68847851)(152.64810832,260.2470717)(151.66699219,261.36425945)
\curveto(150.69303215,262.48860592)(150.20605347,264.03547937)(150.20605469,266.00488445)
\curveto(150.20605347,267.98143897)(150.69303215,269.52831242)(151.66699219,270.64550945)
\curveto(152.64810832,271.76268519)(154.00162259,272.32127838)(155.72753906,272.3212907)
}
}
{
\newrgbcolor{curcolor}{0 0 0}
\pscustom[linestyle=none,fillstyle=solid,fillcolor=curcolor]
{
\newpath
\moveto(166.46972656,275.44726726)
\lineto(166.46972656,272.03125164)
\lineto(170.54101562,272.03125164)
\lineto(170.54101562,270.49511882)
\lineto(166.46972656,270.49511882)
\lineto(166.46972656,263.96386882)
\curveto(166.46972253,262.98274605)(166.60220938,262.35253835)(166.8671875,262.07324382)
\curveto(167.13931822,261.79394515)(167.68716923,261.65429686)(168.51074219,261.65429851)
\lineto(170.54101562,261.65429851)
\lineto(170.54101562,260.00000164)
\lineto(168.51074219,260.00000164)
\curveto(166.98534702,260.00000164)(165.9326137,260.28287896)(165.35253906,260.84863445)
\curveto(164.77245861,261.42154969)(164.48241983,262.45996011)(164.48242188,263.96386882)
\lineto(164.48242188,270.49511882)
\lineto(163.03222656,270.49511882)
\lineto(163.03222656,272.03125164)
\lineto(164.48242188,272.03125164)
\lineto(164.48242188,275.44726726)
\lineto(166.46972656,275.44726726)
}
}
{
\newrgbcolor{curcolor}{0 0 0}
\pscustom[linestyle=none,fillstyle=solid,fillcolor=curcolor]
{
}
}
{
\newrgbcolor{curcolor}{0 0 0}
\pscustom[linestyle=none,fillstyle=solid,fillcolor=curcolor]
{
\newpath
\moveto(188.07226562,270.20507976)
\lineto(188.07226562,276.71484539)
\lineto(190.04882812,276.71484539)
\lineto(190.04882812,260.00000164)
\lineto(188.07226562,260.00000164)
\lineto(188.07226562,261.80468914)
\curveto(187.65689147,261.08854221)(187.13052481,260.5550141)(186.49316406,260.2041032)
\curveto(185.86294795,259.86035334)(185.10383412,259.68847851)(184.21582031,259.6884782)
\curveto(182.76203959,259.68847851)(181.57681942,260.26855605)(180.66015625,261.42871257)
\curveto(179.75064937,262.58886623)(179.29589722,264.11425533)(179.29589844,266.00488445)
\curveto(179.29589722,267.89550155)(179.75064937,269.42089065)(180.66015625,270.58105632)
\curveto(181.57681942,271.74120083)(182.76203959,272.32127838)(184.21582031,272.3212907)
\curveto(185.10383412,272.32127838)(185.86294795,272.14582282)(186.49316406,271.79492351)
\curveto(187.13052481,271.45116206)(187.65689147,270.92121467)(188.07226562,270.20507976)
\moveto(181.33691406,266.00488445)
\curveto(181.33691081,264.55110386)(181.63411103,263.40885239)(182.22851562,262.57812664)
\curveto(182.83007338,261.75455717)(183.65364026,261.34277373)(184.69921875,261.34277507)
\curveto(185.744784,261.34277373)(186.56835089,261.75455717)(187.16992188,262.57812664)
\curveto(187.77147469,263.40885239)(188.07225563,264.55110386)(188.07226562,266.00488445)
\curveto(188.07225563,267.45865303)(187.77147469,268.59732377)(187.16992188,269.42090007)
\curveto(186.56835089,270.25161899)(185.744784,270.66698316)(184.69921875,270.66699382)
\curveto(183.65364026,270.66698316)(182.83007338,270.25161899)(182.22851562,269.42090007)
\curveto(181.63411103,268.59732377)(181.33691081,267.45865303)(181.33691406,266.00488445)
}
}
{
\newrgbcolor{curcolor}{0 0 0}
\pscustom[linestyle=none,fillstyle=solid,fillcolor=curcolor]
{
\newpath
\moveto(204.41113281,266.50976726)
\lineto(204.41113281,265.54297039)
\lineto(195.32324219,265.54297039)
\curveto(195.40917633,264.18228912)(195.81737904,263.1438787)(196.54785156,262.42773601)
\curveto(197.28547653,261.71874992)(198.30956405,261.36425808)(199.62011719,261.36425945)
\curveto(200.37922344,261.36425808)(201.11327218,261.45735695)(201.82226562,261.64355632)
\curveto(202.53840097,261.82975241)(203.24738463,262.10904901)(203.94921875,262.48144695)
\lineto(203.94921875,260.61230632)
\curveto(203.24022318,260.31152476)(202.51333589,260.08235832)(201.76855469,259.92480632)
\curveto(201.02375404,259.76725447)(200.26822095,259.68847851)(199.50195312,259.6884782)
\curveto(197.58267676,259.68847851)(196.06086838,260.2470717)(194.93652344,261.36425945)
\curveto(193.81933417,262.48144447)(193.26074097,263.99251066)(193.26074219,265.89746257)
\curveto(193.26074097,267.86685575)(193.79068836,269.4280521)(194.85058594,270.58105632)
\curveto(195.91763936,271.74120083)(197.35351032,272.32127838)(199.15820312,272.3212907)
\curveto(200.77668398,272.32127838)(202.05500301,271.79849244)(202.99316406,270.75293132)
\curveto(203.93846467,269.71451015)(204.41112045,268.30012354)(204.41113281,266.50976726)
\moveto(202.43457031,267.08984539)
\curveto(202.42023702,268.17121742)(202.11587535,269.03417229)(201.52148438,269.67871257)
\curveto(200.9342359,270.3232335)(200.15363773,270.6454988)(199.1796875,270.64550945)
\curveto(198.07681689,270.6454988)(197.19237767,270.33397568)(196.52636719,269.71093914)
\curveto(195.8675092,269.08788317)(195.48795229,268.2106054)(195.38769531,267.0791032)
\lineto(202.43457031,267.08984539)
}
}
{
\newrgbcolor{curcolor}{0 0 0}
\pscustom[linestyle=none,fillstyle=solid,fillcolor=curcolor]
{
}
}
{
\newrgbcolor{curcolor}{0 0 0}
\pscustom[linestyle=none,fillstyle=solid,fillcolor=curcolor]
{
\newpath
\moveto(216.57128906,261.80468914)
\lineto(216.57128906,255.42382976)
\lineto(214.58398438,255.42382976)
\lineto(214.58398438,272.03125164)
\lineto(216.57128906,272.03125164)
\lineto(216.57128906,270.20507976)
\curveto(216.98664925,270.92121467)(217.50943518,271.45116206)(218.13964844,271.79492351)
\curveto(218.77701204,272.14582282)(219.53612586,272.32127838)(220.41699219,272.3212907)
\curveto(221.8779204,272.32127838)(223.06314056,271.74120083)(223.97265625,270.58105632)
\curveto(224.88931061,269.42089065)(225.34764349,267.89550155)(225.34765625,266.00488445)
\curveto(225.34764349,264.11425533)(224.88931061,262.58886623)(223.97265625,261.42871257)
\curveto(223.06314056,260.26855605)(221.8779204,259.68847851)(220.41699219,259.6884782)
\curveto(219.53612586,259.68847851)(218.77701204,259.86035334)(218.13964844,260.2041032)
\curveto(217.50943518,260.5550141)(216.98664925,261.08854221)(216.57128906,261.80468914)
\moveto(223.29589844,266.00488445)
\curveto(223.29588773,267.45865303)(222.99510678,268.59732377)(222.39355469,269.42090007)
\curveto(221.79914443,270.25161899)(220.97915827,270.66698316)(219.93359375,270.66699382)
\curveto(218.88801453,270.66698316)(218.06444765,270.25161899)(217.46289062,269.42090007)
\curveto(216.8684853,268.59732377)(216.57128508,267.45865303)(216.57128906,266.00488445)
\curveto(216.57128508,264.55110386)(216.8684853,263.40885239)(217.46289062,262.57812664)
\curveto(218.06444765,261.75455717)(218.88801453,261.34277373)(219.93359375,261.34277507)
\curveto(220.97915827,261.34277373)(221.79914443,261.75455717)(222.39355469,262.57812664)
\curveto(222.99510678,263.40885239)(223.29588773,264.55110386)(223.29589844,266.00488445)
}
}
{
\newrgbcolor{curcolor}{0 0 0}
\pscustom[linestyle=none,fillstyle=solid,fillcolor=curcolor]
{
\newpath
\moveto(234.09179688,266.0478532)
\curveto(232.49478572,266.04784715)(231.38834152,265.86523015)(230.77246094,265.50000164)
\curveto(230.15657192,265.13476213)(229.84862951,264.51171587)(229.84863281,263.63086101)
\curveto(229.84862951,262.92903517)(230.07779595,262.37044197)(230.53613281,261.95507976)
\curveto(231.00162315,261.54687509)(231.63183086,261.34277373)(232.42675781,261.34277507)
\curveto(233.52245397,261.34277373)(234.39973173,261.72949209)(235.05859375,262.50293132)
\curveto(235.7246002,263.283527)(236.05760768,264.31835669)(236.05761719,265.60742351)
\lineto(236.05761719,266.0478532)
\lineto(234.09179688,266.0478532)
\moveto(238.03417969,266.86425945)
\lineto(238.03417969,260.00000164)
\lineto(236.05761719,260.00000164)
\lineto(236.05761719,261.82617351)
\curveto(235.60643626,261.09570367)(235.04426234,260.5550141)(234.37109375,260.2041032)
\curveto(233.69790952,259.86035334)(232.87434263,259.68847851)(231.90039062,259.6884782)
\curveto(230.66861567,259.68847851)(229.68749686,260.03222817)(228.95703125,260.7197282)
\curveto(228.23372228,261.41438824)(227.87206899,262.34179617)(227.87207031,263.50195476)
\curveto(227.87206899,264.85546553)(228.32324042,265.87597232)(229.22558594,266.5634782)
\curveto(230.13508756,267.25097095)(231.48860183,267.5947206)(233.28613281,267.5947282)
\lineto(236.05761719,267.5947282)
\lineto(236.05761719,267.78808757)
\curveto(236.05760768,268.69758408)(235.75682673,269.3994063)(235.15527344,269.89355632)
\curveto(234.56086439,270.39484801)(233.7229746,270.6454988)(232.64160156,270.64550945)
\curveto(231.95409616,270.6454988)(231.28450047,270.56314211)(230.6328125,270.39843914)
\curveto(229.98111636,270.23371536)(229.35448938,269.98664529)(228.75292969,269.6572282)
\lineto(228.75292969,271.48340007)
\curveto(229.47623405,271.76268519)(230.17805627,271.97036727)(230.85839844,272.10644695)
\curveto(231.53873199,272.24966387)(232.20116622,272.32127838)(232.84570312,272.3212907)
\curveto(234.58592946,272.32127838)(235.88573285,271.87010695)(236.74511719,270.96777507)
\curveto(237.60448113,270.06542126)(238.0341682,268.69758408)(238.03417969,266.86425945)
}
}
{
\newrgbcolor{curcolor}{0 0 0}
\pscustom[linestyle=none,fillstyle=solid,fillcolor=curcolor]
{
\newpath
\moveto(249.78613281,271.67675945)
\lineto(249.78613281,269.80761882)
\curveto(249.22752988,270.09406706)(248.64745233,270.3089106)(248.04589844,270.45215007)
\curveto(247.44432854,270.59536864)(246.82128228,270.66698316)(246.17675781,270.66699382)
\curveto(245.19563287,270.66698316)(244.4580034,270.51659268)(243.96386719,270.21582195)
\curveto(243.47688459,269.91503078)(243.23339525,269.46385936)(243.23339844,268.86230632)
\curveto(243.23339525,268.40396459)(243.4088508,268.0423113)(243.75976562,267.77734539)
\curveto(244.11067302,267.51952537)(244.81607596,267.2724553)(245.87597656,267.03613445)
\lineto(246.55273438,266.88574382)
\curveto(247.95637229,266.58495599)(248.95181401,266.15884964)(249.5390625,265.60742351)
\curveto(250.13345345,265.06314761)(250.43065367,264.30045306)(250.43066406,263.31933757)
\curveto(250.43065367,262.20214787)(249.9866437,261.31770865)(249.09863281,260.66601726)
\curveto(248.21776526,260.01432454)(247.00389929,259.68847851)(245.45703125,259.6884782)
\curveto(244.81249523,259.68847851)(244.13931882,259.75293157)(243.4375,259.88183757)
\curveto(242.74283584,260.00358236)(242.0087871,260.18978009)(241.23535156,260.44043132)
\lineto(241.23535156,262.48144695)
\curveto(241.96581839,262.10188755)(242.68554423,261.81542951)(243.39453125,261.62207195)
\curveto(244.10351156,261.4358726)(244.80533378,261.34277373)(245.5,261.34277507)
\curveto(246.4309832,261.34277373)(247.14712831,261.50032566)(247.6484375,261.81543132)
\curveto(248.14973148,262.13769481)(248.40038227,262.58886623)(248.40039062,263.16894695)
\curveto(248.40038227,263.70605262)(248.21776526,264.11783606)(247.85253906,264.40429851)
\curveto(247.49445869,264.69075215)(246.70311834,264.96646802)(245.47851562,265.23144695)
\lineto(244.79101562,265.39257976)
\curveto(243.56640273,265.65038661)(242.68196351,266.04426642)(242.13769531,266.57422039)
\curveto(241.59342293,267.11132265)(241.32128778,267.84537139)(241.32128906,268.77636882)
\curveto(241.32128778,269.90786933)(241.72232905,270.78156638)(242.52441406,271.39746257)
\curveto(243.32649411,272.01333598)(244.46516485,272.32127838)(245.94042969,272.3212907)
\curveto(246.67089181,272.32127838)(247.35839112,272.26756749)(248.00292969,272.16015789)
\curveto(248.64745233,272.05272396)(249.24185278,271.89159131)(249.78613281,271.67675945)
}
}
{
\newrgbcolor{curcolor}{0 0 0}
\pscustom[linestyle=none,fillstyle=solid,fillcolor=curcolor]
{
\newpath
\moveto(261.25878906,271.67675945)
\lineto(261.25878906,269.80761882)
\curveto(260.70018613,270.09406706)(260.12010858,270.3089106)(259.51855469,270.45215007)
\curveto(258.91698479,270.59536864)(258.29393853,270.66698316)(257.64941406,270.66699382)
\curveto(256.66828912,270.66698316)(255.93065965,270.51659268)(255.43652344,270.21582195)
\curveto(254.94954084,269.91503078)(254.7060515,269.46385936)(254.70605469,268.86230632)
\curveto(254.7060515,268.40396459)(254.88150705,268.0423113)(255.23242188,267.77734539)
\curveto(255.58332927,267.51952537)(256.28873221,267.2724553)(257.34863281,267.03613445)
\lineto(258.02539062,266.88574382)
\curveto(259.42902854,266.58495599)(260.42447026,266.15884964)(261.01171875,265.60742351)
\curveto(261.6061097,265.06314761)(261.90330992,264.30045306)(261.90332031,263.31933757)
\curveto(261.90330992,262.20214787)(261.45929995,261.31770865)(260.57128906,260.66601726)
\curveto(259.69042151,260.01432454)(258.47655554,259.68847851)(256.9296875,259.6884782)
\curveto(256.28515148,259.68847851)(255.61197507,259.75293157)(254.91015625,259.88183757)
\curveto(254.21549209,260.00358236)(253.48144335,260.18978009)(252.70800781,260.44043132)
\lineto(252.70800781,262.48144695)
\curveto(253.43847464,262.10188755)(254.15820048,261.81542951)(254.8671875,261.62207195)
\curveto(255.57616781,261.4358726)(256.27799003,261.34277373)(256.97265625,261.34277507)
\curveto(257.90363945,261.34277373)(258.61978456,261.50032566)(259.12109375,261.81543132)
\curveto(259.62238773,262.13769481)(259.87303852,262.58886623)(259.87304688,263.16894695)
\curveto(259.87303852,263.70605262)(259.69042151,264.11783606)(259.32519531,264.40429851)
\curveto(258.96711494,264.69075215)(258.17577459,264.96646802)(256.95117188,265.23144695)
\lineto(256.26367188,265.39257976)
\curveto(255.03905898,265.65038661)(254.15461976,266.04426642)(253.61035156,266.57422039)
\curveto(253.06607918,267.11132265)(252.79394403,267.84537139)(252.79394531,268.77636882)
\curveto(252.79394403,269.90786933)(253.1949853,270.78156638)(253.99707031,271.39746257)
\curveto(254.79915036,272.01333598)(255.9378211,272.32127838)(257.41308594,272.3212907)
\curveto(258.14354806,272.32127838)(258.83104737,272.26756749)(259.47558594,272.16015789)
\curveto(260.12010858,272.05272396)(260.71450903,271.89159131)(261.25878906,271.67675945)
}
}
{
\newrgbcolor{curcolor}{0 0 0}
\pscustom[linestyle=none,fillstyle=solid,fillcolor=curcolor]
{
\newpath
\moveto(275.35253906,266.50976726)
\lineto(275.35253906,265.54297039)
\lineto(266.26464844,265.54297039)
\curveto(266.35058258,264.18228912)(266.75878529,263.1438787)(267.48925781,262.42773601)
\curveto(268.22688278,261.71874992)(269.2509703,261.36425808)(270.56152344,261.36425945)
\curveto(271.32062969,261.36425808)(272.05467843,261.45735695)(272.76367188,261.64355632)
\curveto(273.47980722,261.82975241)(274.18879088,262.10904901)(274.890625,262.48144695)
\lineto(274.890625,260.61230632)
\curveto(274.18162943,260.31152476)(273.45474214,260.08235832)(272.70996094,259.92480632)
\curveto(271.96516029,259.76725447)(271.2096272,259.68847851)(270.44335938,259.6884782)
\curveto(268.52408301,259.68847851)(267.00227463,260.2470717)(265.87792969,261.36425945)
\curveto(264.76074042,262.48144447)(264.20214722,263.99251066)(264.20214844,265.89746257)
\curveto(264.20214722,267.86685575)(264.73209461,269.4280521)(265.79199219,270.58105632)
\curveto(266.85904561,271.74120083)(268.29491657,272.32127838)(270.09960938,272.3212907)
\curveto(271.71809023,272.32127838)(272.99640926,271.79849244)(273.93457031,270.75293132)
\curveto(274.87987092,269.71451015)(275.3525267,268.30012354)(275.35253906,266.50976726)
\moveto(273.37597656,267.08984539)
\curveto(273.36164327,268.17121742)(273.0572816,269.03417229)(272.46289062,269.67871257)
\curveto(271.87564215,270.3232335)(271.09504398,270.6454988)(270.12109375,270.64550945)
\curveto(269.01822314,270.6454988)(268.13378392,270.33397568)(267.46777344,269.71093914)
\curveto(266.80891545,269.08788317)(266.42935854,268.2106054)(266.32910156,267.0791032)
\lineto(273.37597656,267.08984539)
}
}
{
\newrgbcolor{curcolor}{0 0 0}
\pscustom[linestyle=none,fillstyle=solid,fillcolor=curcolor]
{
\newpath
\moveto(144.16894531,224.80273601)
\lineto(144.16894531,222.51465007)
\curveto(143.43846312,223.19497542)(142.65786495,223.70343845)(141.82714844,224.0400407)
\curveto(141.00356973,224.37661486)(140.12629196,224.54490897)(139.1953125,224.54492351)
\curveto(137.3619718,224.54490897)(135.95832738,223.98273505)(134.984375,222.85840007)
\curveto(134.01041266,221.74120083)(133.52343398,220.12271287)(133.5234375,218.00293132)
\curveto(133.52343398,215.89029522)(134.01041266,214.27180726)(134.984375,213.14746257)
\curveto(135.95832738,212.03027304)(137.3619718,211.47167985)(139.1953125,211.47168132)
\curveto(140.12629196,211.47167985)(141.00356973,211.63997395)(141.82714844,211.97656414)
\curveto(142.65786495,212.31315036)(143.43846312,212.8216134)(144.16894531,213.50195476)
\lineto(144.16894531,211.2353532)
\curveto(143.40981732,210.71972748)(142.60415406,210.33300912)(141.75195312,210.07519695)
\curveto(140.90689013,209.81738463)(140.01170874,209.68847851)(139.06640625,209.6884782)
\curveto(136.63866524,209.68847851)(134.72655777,210.42968871)(133.33007812,211.91211101)
\curveto(131.93359182,213.40169094)(131.23535033,215.43196235)(131.23535156,218.00293132)
\curveto(131.23535033,220.58104574)(131.93359182,222.61131715)(133.33007812,224.09375164)
\curveto(134.72655777,225.58331939)(136.63866524,226.32811031)(139.06640625,226.32812664)
\curveto(140.02603164,226.32811031)(140.92837449,226.19920419)(141.7734375,225.94140789)
\curveto(142.62563842,225.69074115)(143.42414022,225.31118424)(144.16894531,224.80273601)
}
}
{
\newrgbcolor{curcolor}{0 0 0}
\pscustom[linestyle=none,fillstyle=solid,fillcolor=curcolor]
{
\newpath
\moveto(152.11816406,220.64550945)
\curveto(151.05826255,220.6454988)(150.22037277,220.23013463)(149.60449219,219.3994157)
\curveto(148.98860317,218.57583941)(148.68066076,217.44433013)(148.68066406,216.00488445)
\curveto(148.68066076,214.56542676)(148.98502244,213.43033675)(149.59375,212.59961101)
\curveto(150.20963059,211.77604153)(151.0511011,211.36425808)(152.11816406,211.36425945)
\curveto(153.17089065,211.36425808)(154.00519971,211.77962225)(154.62109375,212.6103532)
\curveto(155.23696931,213.44107892)(155.54491171,214.57258821)(155.54492188,216.00488445)
\curveto(155.54491171,217.43000723)(155.23696931,218.55793579)(154.62109375,219.38867351)
\curveto(154.00519971,220.22655391)(153.17089065,220.6454988)(152.11816406,220.64550945)
\moveto(152.11816406,222.3212907)
\curveto(153.83690561,222.32127838)(155.18683915,221.76268519)(156.16796875,220.64550945)
\curveto(157.14907678,219.52831242)(157.63963618,217.98143897)(157.63964844,216.00488445)
\curveto(157.63963618,214.03547937)(157.14907678,212.48860592)(156.16796875,211.36425945)
\curveto(155.18683915,210.2470717)(153.83690561,209.68847851)(152.11816406,209.6884782)
\curveto(150.39224759,209.68847851)(149.03873332,210.2470717)(148.05761719,211.36425945)
\curveto(147.08365715,212.48860592)(146.59667847,214.03547937)(146.59667969,216.00488445)
\curveto(146.59667847,217.98143897)(147.08365715,219.52831242)(148.05761719,220.64550945)
\curveto(149.03873332,221.76268519)(150.39224759,222.32127838)(152.11816406,222.3212907)
}
}
{
\newrgbcolor{curcolor}{0 0 0}
\pscustom[linestyle=none,fillstyle=solid,fillcolor=curcolor]
{
\newpath
\moveto(170.90625,217.26172039)
\lineto(170.90625,210.00000164)
\lineto(168.9296875,210.00000164)
\lineto(168.9296875,217.19726726)
\curveto(168.9296774,218.3359308)(168.70767242,219.18814349)(168.26367188,219.75390789)
\curveto(167.81965247,220.31965277)(167.15363751,220.6025301)(166.265625,220.6025407)
\curveto(165.19856134,220.6025301)(164.35709083,220.26236117)(163.74121094,219.58203289)
\curveto(163.12532123,218.90168544)(162.81737883,217.97427752)(162.81738281,216.79980632)
\lineto(162.81738281,210.00000164)
\lineto(160.83007812,210.00000164)
\lineto(160.83007812,222.03125164)
\lineto(162.81738281,222.03125164)
\lineto(162.81738281,220.16211101)
\curveto(163.2900346,220.88540742)(163.84504707,221.42609698)(164.48242188,221.78418132)
\curveto(165.12694683,222.1422421)(165.86815703,222.32127838)(166.70605469,222.3212907)
\curveto(168.08820689,222.32127838)(169.13377876,221.89159131)(169.84277344,221.0322282)
\curveto(170.55174609,220.18000448)(170.90623793,218.9231698)(170.90625,217.26172039)
}
}
{
\newrgbcolor{curcolor}{0 0 0}
\pscustom[linestyle=none,fillstyle=solid,fillcolor=curcolor]
{
\newpath
\moveto(184.87109375,217.26172039)
\lineto(184.87109375,210.00000164)
\lineto(182.89453125,210.00000164)
\lineto(182.89453125,217.19726726)
\curveto(182.89452115,218.3359308)(182.67251617,219.18814349)(182.22851562,219.75390789)
\curveto(181.78449622,220.31965277)(181.11848126,220.6025301)(180.23046875,220.6025407)
\curveto(179.16340509,220.6025301)(178.32193458,220.26236117)(177.70605469,219.58203289)
\curveto(177.09016498,218.90168544)(176.78222258,217.97427752)(176.78222656,216.79980632)
\lineto(176.78222656,210.00000164)
\lineto(174.79492188,210.00000164)
\lineto(174.79492188,222.03125164)
\lineto(176.78222656,222.03125164)
\lineto(176.78222656,220.16211101)
\curveto(177.25487835,220.88540742)(177.80989082,221.42609698)(178.44726562,221.78418132)
\curveto(179.09179058,222.1422421)(179.83300078,222.32127838)(180.67089844,222.3212907)
\curveto(182.05305064,222.32127838)(183.09862251,221.89159131)(183.80761719,221.0322282)
\curveto(184.51658984,220.18000448)(184.87108168,218.9231698)(184.87109375,217.26172039)
}
}
{
\newrgbcolor{curcolor}{0 0 0}
\pscustom[linestyle=none,fillstyle=solid,fillcolor=curcolor]
{
\newpath
\moveto(199.12597656,216.50976726)
\lineto(199.12597656,215.54297039)
\lineto(190.03808594,215.54297039)
\curveto(190.12402008,214.18228912)(190.53222279,213.1438787)(191.26269531,212.42773601)
\curveto(192.00032028,211.71874992)(193.0244078,211.36425808)(194.33496094,211.36425945)
\curveto(195.09406719,211.36425808)(195.82811593,211.45735695)(196.53710938,211.64355632)
\curveto(197.25324472,211.82975241)(197.96222838,212.10904901)(198.6640625,212.48144695)
\lineto(198.6640625,210.61230632)
\curveto(197.95506693,210.31152476)(197.22817964,210.08235832)(196.48339844,209.92480632)
\curveto(195.73859779,209.76725447)(194.9830647,209.68847851)(194.21679688,209.6884782)
\curveto(192.29752051,209.68847851)(190.77571213,210.2470717)(189.65136719,211.36425945)
\curveto(188.53417792,212.48144447)(187.97558472,213.99251066)(187.97558594,215.89746257)
\curveto(187.97558472,217.86685575)(188.50553211,219.4280521)(189.56542969,220.58105632)
\curveto(190.63248311,221.74120083)(192.06835407,222.32127838)(193.87304688,222.3212907)
\curveto(195.49152773,222.32127838)(196.76984676,221.79849244)(197.70800781,220.75293132)
\curveto(198.65330842,219.71451015)(199.1259642,218.30012354)(199.12597656,216.50976726)
\moveto(197.14941406,217.08984539)
\curveto(197.13508077,218.17121742)(196.8307191,219.03417229)(196.23632812,219.67871257)
\curveto(195.64907965,220.3232335)(194.86848148,220.6454988)(193.89453125,220.64550945)
\curveto(192.79166064,220.6454988)(191.90722142,220.33397568)(191.24121094,219.71093914)
\curveto(190.58235295,219.08788317)(190.20279604,218.2106054)(190.10253906,217.0791032)
\lineto(197.14941406,217.08984539)
}
}
{
\newrgbcolor{curcolor}{0 0 0}
\pscustom[linestyle=none,fillstyle=solid,fillcolor=curcolor]
{
\newpath
\moveto(211.984375,222.03125164)
\lineto(207.63378906,216.17675945)
\lineto(212.20996094,210.00000164)
\lineto(209.87890625,210.00000164)
\lineto(206.37695312,214.72656414)
\lineto(202.875,210.00000164)
\lineto(200.54394531,210.00000164)
\lineto(205.21679688,216.29492351)
\lineto(200.94140625,222.03125164)
\lineto(203.27246094,222.03125164)
\lineto(206.46289062,217.74511882)
\lineto(209.65332031,222.03125164)
\lineto(211.984375,222.03125164)
}
}
{
\newrgbcolor{curcolor}{0 0 0}
\pscustom[linestyle=none,fillstyle=solid,fillcolor=curcolor]
{
\newpath
\moveto(215.00292969,222.03125164)
\lineto(216.97949219,222.03125164)
\lineto(216.97949219,210.00000164)
\lineto(215.00292969,210.00000164)
\lineto(215.00292969,222.03125164)
\moveto(215.00292969,226.71484539)
\lineto(216.97949219,226.71484539)
\lineto(216.97949219,224.2119157)
\lineto(215.00292969,224.2119157)
\lineto(215.00292969,226.71484539)
}
}
{
\newrgbcolor{curcolor}{0 0 0}
\pscustom[linestyle=none,fillstyle=solid,fillcolor=curcolor]
{
\newpath
\moveto(225.76660156,220.64550945)
\curveto(224.70670005,220.6454988)(223.86881027,220.23013463)(223.25292969,219.3994157)
\curveto(222.63704067,218.57583941)(222.32909826,217.44433013)(222.32910156,216.00488445)
\curveto(222.32909826,214.56542676)(222.63345994,213.43033675)(223.2421875,212.59961101)
\curveto(223.85806809,211.77604153)(224.6995386,211.36425808)(225.76660156,211.36425945)
\curveto(226.81932815,211.36425808)(227.65363721,211.77962225)(228.26953125,212.6103532)
\curveto(228.88540681,213.44107892)(229.19334921,214.57258821)(229.19335938,216.00488445)
\curveto(229.19334921,217.43000723)(228.88540681,218.55793579)(228.26953125,219.38867351)
\curveto(227.65363721,220.22655391)(226.81932815,220.6454988)(225.76660156,220.64550945)
\moveto(225.76660156,222.3212907)
\curveto(227.48534311,222.32127838)(228.83527665,221.76268519)(229.81640625,220.64550945)
\curveto(230.79751428,219.52831242)(231.28807368,217.98143897)(231.28808594,216.00488445)
\curveto(231.28807368,214.03547937)(230.79751428,212.48860592)(229.81640625,211.36425945)
\curveto(228.83527665,210.2470717)(227.48534311,209.68847851)(225.76660156,209.6884782)
\curveto(224.04068509,209.68847851)(222.68717082,210.2470717)(221.70605469,211.36425945)
\curveto(220.73209465,212.48860592)(220.24511597,214.03547937)(220.24511719,216.00488445)
\curveto(220.24511597,217.98143897)(220.73209465,219.52831242)(221.70605469,220.64550945)
\curveto(222.68717082,221.76268519)(224.04068509,222.32127838)(225.76660156,222.3212907)
}
}
{
\newrgbcolor{curcolor}{0 0 0}
\pscustom[linestyle=none,fillstyle=solid,fillcolor=curcolor]
{
\newpath
\moveto(244.5546875,217.26172039)
\lineto(244.5546875,210.00000164)
\lineto(242.578125,210.00000164)
\lineto(242.578125,217.19726726)
\curveto(242.5781149,218.3359308)(242.35610992,219.18814349)(241.91210938,219.75390789)
\curveto(241.46808997,220.31965277)(240.80207501,220.6025301)(239.9140625,220.6025407)
\curveto(238.84699884,220.6025301)(238.00552833,220.26236117)(237.38964844,219.58203289)
\curveto(236.77375873,218.90168544)(236.46581633,217.97427752)(236.46582031,216.79980632)
\lineto(236.46582031,210.00000164)
\lineto(234.47851562,210.00000164)
\lineto(234.47851562,222.03125164)
\lineto(236.46582031,222.03125164)
\lineto(236.46582031,220.16211101)
\curveto(236.9384721,220.88540742)(237.49348457,221.42609698)(238.13085938,221.78418132)
\curveto(238.77538433,222.1422421)(239.51659453,222.32127838)(240.35449219,222.3212907)
\curveto(241.73664439,222.32127838)(242.78221626,221.89159131)(243.49121094,221.0322282)
\curveto(244.20018359,220.18000448)(244.55467543,218.9231698)(244.5546875,217.26172039)
}
}
{
\newrgbcolor{curcolor}{0 0 0}
\pscustom[linestyle=none,fillstyle=solid,fillcolor=curcolor]
{
}
}
{
\newrgbcolor{curcolor}{0 0 0}
\pscustom[linestyle=none,fillstyle=solid,fillcolor=curcolor]
{
\newpath
\moveto(260.99023438,216.0478532)
\curveto(259.39322322,216.04784715)(258.28677902,215.86523015)(257.67089844,215.50000164)
\curveto(257.05500942,215.13476213)(256.74706701,214.51171587)(256.74707031,213.63086101)
\curveto(256.74706701,212.92903517)(256.97623345,212.37044197)(257.43457031,211.95507976)
\curveto(257.90006065,211.54687509)(258.53026836,211.34277373)(259.32519531,211.34277507)
\curveto(260.42089147,211.34277373)(261.29816923,211.72949209)(261.95703125,212.50293132)
\curveto(262.6230377,213.283527)(262.95604518,214.31835669)(262.95605469,215.60742351)
\lineto(262.95605469,216.0478532)
\lineto(260.99023438,216.0478532)
\moveto(264.93261719,216.86425945)
\lineto(264.93261719,210.00000164)
\lineto(262.95605469,210.00000164)
\lineto(262.95605469,211.82617351)
\curveto(262.50487376,211.09570367)(261.94269984,210.5550141)(261.26953125,210.2041032)
\curveto(260.59634702,209.86035334)(259.77278013,209.68847851)(258.79882812,209.6884782)
\curveto(257.56705317,209.68847851)(256.58593436,210.03222817)(255.85546875,210.7197282)
\curveto(255.13215978,211.41438824)(254.77050649,212.34179617)(254.77050781,213.50195476)
\curveto(254.77050649,214.85546553)(255.22167792,215.87597232)(256.12402344,216.5634782)
\curveto(257.03352506,217.25097095)(258.38703933,217.5947206)(260.18457031,217.5947282)
\lineto(262.95605469,217.5947282)
\lineto(262.95605469,217.78808757)
\curveto(262.95604518,218.69758408)(262.65526423,219.3994063)(262.05371094,219.89355632)
\curveto(261.45930189,220.39484801)(260.6214121,220.6454988)(259.54003906,220.64550945)
\curveto(258.85253366,220.6454988)(258.18293797,220.56314211)(257.53125,220.39843914)
\curveto(256.87955386,220.23371536)(256.25292688,219.98664529)(255.65136719,219.6572282)
\lineto(255.65136719,221.48340007)
\curveto(256.37467155,221.76268519)(257.07649377,221.97036727)(257.75683594,222.10644695)
\curveto(258.43716949,222.24966387)(259.09960372,222.32127838)(259.74414062,222.3212907)
\curveto(261.48436696,222.32127838)(262.78417035,221.87010695)(263.64355469,220.96777507)
\curveto(264.50291863,220.06542126)(264.9326057,218.69758408)(264.93261719,216.86425945)
}
}
{
\newrgbcolor{curcolor}{0 0 0}
\pscustom[linestyle=none,fillstyle=solid,fillcolor=curcolor]
{
\newpath
\moveto(268.81054688,214.74804851)
\lineto(268.81054688,222.03125164)
\lineto(270.78710938,222.03125164)
\lineto(270.78710938,214.82324382)
\curveto(270.78710553,213.68456826)(271.00911052,212.82877485)(271.453125,212.25586101)
\curveto(271.89713046,211.69010411)(272.56314542,211.40722679)(273.45117188,211.4072282)
\curveto(274.51822159,211.40722679)(275.3596921,211.74739572)(275.97558594,212.42773601)
\curveto(276.59862316,213.10807144)(276.91014628,214.03547937)(276.91015625,215.20996257)
\lineto(276.91015625,222.03125164)
\lineto(278.88671875,222.03125164)
\lineto(278.88671875,210.00000164)
\lineto(276.91015625,210.00000164)
\lineto(276.91015625,211.84765789)
\curveto(276.43032905,211.11718802)(275.87173586,210.57291773)(275.234375,210.21484539)
\curveto(274.604159,209.86393406)(273.87011026,209.68847851)(273.03222656,209.6884782)
\curveto(271.6500604,209.68847851)(270.6009078,210.11816558)(269.88476562,210.9775407)
\curveto(269.16861756,211.83691386)(268.81054501,213.09374854)(268.81054688,214.74804851)
\moveto(273.78417969,222.3212907)
\lineto(273.78417969,222.3212907)
}
}
{
\newrgbcolor{curcolor}{0 0 0}
\pscustom[linestyle=none,fillstyle=solid,fillcolor=curcolor]
{
\newpath
\moveto(284.93457031,225.44726726)
\lineto(284.93457031,222.03125164)
\lineto(289.00585938,222.03125164)
\lineto(289.00585938,220.49511882)
\lineto(284.93457031,220.49511882)
\lineto(284.93457031,213.96386882)
\curveto(284.93456628,212.98274605)(285.06705313,212.35253835)(285.33203125,212.07324382)
\curveto(285.60416197,211.79394515)(286.15201298,211.65429686)(286.97558594,211.65429851)
\lineto(289.00585938,211.65429851)
\lineto(289.00585938,210.00000164)
\lineto(286.97558594,210.00000164)
\curveto(285.45019077,210.00000164)(284.39745745,210.28287896)(283.81738281,210.84863445)
\curveto(283.23730236,211.42154969)(282.94726358,212.45996011)(282.94726562,213.96386882)
\lineto(282.94726562,220.49511882)
\lineto(281.49707031,220.49511882)
\lineto(281.49707031,222.03125164)
\lineto(282.94726562,222.03125164)
\lineto(282.94726562,225.44726726)
\lineto(284.93457031,225.44726726)
}
}
{
\newrgbcolor{curcolor}{0 0 0}
\pscustom[linestyle=none,fillstyle=solid,fillcolor=curcolor]
{
\newpath
\moveto(296.27832031,220.64550945)
\curveto(295.2184188,220.6454988)(294.38052902,220.23013463)(293.76464844,219.3994157)
\curveto(293.14875942,218.57583941)(292.84081701,217.44433013)(292.84082031,216.00488445)
\curveto(292.84081701,214.56542676)(293.14517869,213.43033675)(293.75390625,212.59961101)
\curveto(294.36978684,211.77604153)(295.21125735,211.36425808)(296.27832031,211.36425945)
\curveto(297.3310469,211.36425808)(298.16535596,211.77962225)(298.78125,212.6103532)
\curveto(299.39712556,213.44107892)(299.70506796,214.57258821)(299.70507812,216.00488445)
\curveto(299.70506796,217.43000723)(299.39712556,218.55793579)(298.78125,219.38867351)
\curveto(298.16535596,220.22655391)(297.3310469,220.6454988)(296.27832031,220.64550945)
\moveto(296.27832031,222.3212907)
\curveto(297.99706186,222.32127838)(299.3469954,221.76268519)(300.328125,220.64550945)
\curveto(301.30923303,219.52831242)(301.79979243,217.98143897)(301.79980469,216.00488445)
\curveto(301.79979243,214.03547937)(301.30923303,212.48860592)(300.328125,211.36425945)
\curveto(299.3469954,210.2470717)(297.99706186,209.68847851)(296.27832031,209.6884782)
\curveto(294.55240384,209.68847851)(293.19888957,210.2470717)(292.21777344,211.36425945)
\curveto(291.2438134,212.48860592)(290.75683472,214.03547937)(290.75683594,216.00488445)
\curveto(290.75683472,217.98143897)(291.2438134,219.52831242)(292.21777344,220.64550945)
\curveto(293.19888957,221.76268519)(294.55240384,222.32127838)(296.27832031,222.3212907)
}
}
{
\newrgbcolor{curcolor}{0 0 0}
\pscustom[linestyle=none,fillstyle=solid,fillcolor=curcolor]
{
\newpath
\moveto(132.15917969,186.03807231)
\lineto(135.39257812,186.03807231)
\lineto(139.48535156,175.12400981)
\lineto(143.59960938,186.03807231)
\lineto(146.83300781,186.03807231)
\lineto(146.83300781,169.99998638)
\lineto(144.71679688,169.99998638)
\lineto(144.71679688,184.08299419)
\lineto(140.58105469,173.08299419)
\lineto(138.40039062,173.08299419)
\lineto(134.26464844,184.08299419)
\lineto(134.26464844,169.99998638)
\lineto(132.15917969,169.99998638)
\lineto(132.15917969,186.03807231)
}
}
{
\newrgbcolor{curcolor}{0 0 0}
\pscustom[linestyle=none,fillstyle=solid,fillcolor=curcolor]
{
\newpath
\moveto(161.35644531,176.509752)
\lineto(161.35644531,175.54295513)
\lineto(152.26855469,175.54295513)
\curveto(152.35448883,174.18227386)(152.76269154,173.14386344)(153.49316406,172.42772075)
\curveto(154.23078903,171.71873466)(155.25487655,171.36424283)(156.56542969,171.36424419)
\curveto(157.32453594,171.36424283)(158.05858468,171.45734169)(158.76757812,171.64354106)
\curveto(159.48371347,171.82973715)(160.19269713,172.10903375)(160.89453125,172.48143169)
\lineto(160.89453125,170.61229106)
\curveto(160.18553568,170.3115095)(159.45864839,170.08234307)(158.71386719,169.92479106)
\curveto(157.96906654,169.76723921)(157.21353345,169.68846325)(156.44726562,169.68846294)
\curveto(154.52798926,169.68846325)(153.00618088,170.24705644)(151.88183594,171.36424419)
\curveto(150.76464667,172.48142921)(150.20605347,173.99249541)(150.20605469,175.89744731)
\curveto(150.20605347,177.86684049)(150.73600086,179.42803684)(151.79589844,180.58104106)
\curveto(152.86295186,181.74118557)(154.29882282,182.32126312)(156.10351562,182.32127544)
\curveto(157.72199648,182.32126312)(159.00031551,181.79847718)(159.93847656,180.75291606)
\curveto(160.88377717,179.71449489)(161.35643295,178.30010829)(161.35644531,176.509752)
\moveto(159.37988281,177.08983013)
\curveto(159.36554952,178.17120216)(159.06118785,179.03415703)(158.46679688,179.67869731)
\curveto(157.8795484,180.32321824)(157.09895023,180.64548354)(156.125,180.64549419)
\curveto(155.02212939,180.64548354)(154.13769017,180.33396042)(153.47167969,179.71092388)
\curveto(152.8128217,179.08786791)(152.43326479,178.21059015)(152.33300781,177.07908794)
\lineto(159.37988281,177.08983013)
\moveto(157.47851562,187.5956895)
\lineto(159.61621094,187.5956895)
\lineto(156.11425781,183.556627)
\lineto(154.47070312,183.556627)
\lineto(157.47851562,187.5956895)
}
}
{
\newrgbcolor{curcolor}{0 0 0}
\pscustom[linestyle=none,fillstyle=solid,fillcolor=curcolor]
{
\newpath
\moveto(173.96777344,179.72166606)
\curveto(174.46190213,180.60967629)(175.05272185,181.26494907)(175.74023438,181.68748638)
\curveto(176.42772047,182.11000031)(177.23696446,182.32126312)(178.16796875,182.32127544)
\curveto(179.42120706,182.32126312)(180.38800297,181.88083387)(181.06835938,180.99998638)
\curveto(181.7486787,180.12627833)(182.08884763,178.88018583)(182.08886719,177.26170513)
\lineto(182.08886719,169.99998638)
\lineto(180.1015625,169.99998638)
\lineto(180.1015625,177.197252)
\curveto(180.10154493,178.35023844)(179.89744357,179.20603186)(179.48925781,179.76463481)
\curveto(179.08103813,180.32321824)(178.45799188,180.60251484)(177.62011719,180.60252544)
\curveto(176.59601458,180.60251484)(175.78677059,180.26234591)(175.19238281,179.58201763)
\curveto(174.5979697,178.90167018)(174.30076948,177.97426226)(174.30078125,176.79979106)
\lineto(174.30078125,169.99998638)
\lineto(172.31347656,169.99998638)
\lineto(172.31347656,177.197252)
\curveto(172.31346678,178.35739989)(172.10936542,179.21319331)(171.70117188,179.76463481)
\curveto(171.29295998,180.32321824)(170.66275228,180.60251484)(169.81054688,180.60252544)
\curveto(168.80077498,180.60251484)(167.99869245,180.25876518)(167.40429688,179.57127544)
\curveto(166.80989155,178.89092801)(166.51269133,177.96710081)(166.51269531,176.79979106)
\lineto(166.51269531,169.99998638)
\lineto(164.52539062,169.99998638)
\lineto(164.52539062,182.03123638)
\lineto(166.51269531,182.03123638)
\lineto(166.51269531,180.16209575)
\curveto(166.96386275,180.89971506)(167.50455231,181.44398535)(168.13476562,181.79490825)
\curveto(168.76496772,182.14580756)(169.51333937,182.32126312)(170.37988281,182.32127544)
\curveto(171.253572,182.32126312)(171.9947822,182.09925813)(172.60351562,181.65525981)
\curveto(173.21939035,181.21123819)(173.6741425,180.56670758)(173.96777344,179.72166606)
}
}
{
\newrgbcolor{curcolor}{0 0 0}
\pscustom[linestyle=none,fillstyle=solid,fillcolor=curcolor]
{
\newpath
\moveto(190.70410156,180.64549419)
\curveto(189.64420005,180.64548354)(188.80631027,180.23011938)(188.19042969,179.39940044)
\curveto(187.57454067,178.57582416)(187.26659826,177.44431487)(187.26660156,176.00486919)
\curveto(187.26659826,174.5654115)(187.57095994,173.43032149)(188.1796875,172.59959575)
\curveto(188.79556809,171.77602627)(189.6370386,171.36424283)(190.70410156,171.36424419)
\curveto(191.75682815,171.36424283)(192.59113721,171.77960699)(193.20703125,172.61033794)
\curveto(193.82290681,173.44106367)(194.13084921,174.57257295)(194.13085938,176.00486919)
\curveto(194.13084921,177.42999197)(193.82290681,178.55792053)(193.20703125,179.38865825)
\curveto(192.59113721,180.22653865)(191.75682815,180.64548354)(190.70410156,180.64549419)
\moveto(190.70410156,182.32127544)
\curveto(192.42284311,182.32126312)(193.77277665,181.76266993)(194.75390625,180.64549419)
\curveto(195.73501428,179.52829716)(196.22557368,177.98142371)(196.22558594,176.00486919)
\curveto(196.22557368,174.03546411)(195.73501428,172.48859066)(194.75390625,171.36424419)
\curveto(193.77277665,170.24705644)(192.42284311,169.68846325)(190.70410156,169.68846294)
\curveto(188.97818509,169.68846325)(187.62467082,170.24705644)(186.64355469,171.36424419)
\curveto(185.66959465,172.48859066)(185.18261597,174.03546411)(185.18261719,176.00486919)
\curveto(185.18261597,177.98142371)(185.66959465,179.52829716)(186.64355469,180.64549419)
\curveto(187.62467082,181.76266993)(188.97818509,182.32126312)(190.70410156,182.32127544)
}
}
{
\newrgbcolor{curcolor}{0 0 0}
\pscustom[linestyle=none,fillstyle=solid,fillcolor=curcolor]
{
\newpath
\moveto(206.46289062,180.18358013)
\curveto(206.24087659,180.31247606)(205.99738725,180.40557493)(205.73242188,180.462877)
\curveto(205.47460132,180.5273196)(205.18814327,180.55954613)(204.87304688,180.55955669)
\curveto(203.75585304,180.55954613)(202.8964789,180.19431212)(202.29492188,179.46385356)
\curveto(201.70051655,178.74053753)(201.40331633,177.69854639)(201.40332031,176.337877)
\lineto(201.40332031,169.99998638)
\lineto(199.41601562,169.99998638)
\lineto(199.41601562,182.03123638)
\lineto(201.40332031,182.03123638)
\lineto(201.40332031,180.16209575)
\curveto(201.8186805,180.89255361)(202.35937006,181.43324317)(203.02539062,181.78416606)
\curveto(203.69139998,182.14222684)(204.50064396,182.32126312)(205.453125,182.32127544)
\curveto(205.58918454,182.32126312)(205.73957501,182.31052094)(205.90429688,182.28904888)
\curveto(206.06900177,182.27471369)(206.25161877,182.24964861)(206.45214844,182.21385356)
\lineto(206.46289062,180.18358013)
}
}
{
\newrgbcolor{curcolor}{0 0 0}
\pscustom[linestyle=none,fillstyle=solid,fillcolor=curcolor]
{
\newpath
\moveto(208.55761719,182.03123638)
\lineto(210.53417969,182.03123638)
\lineto(210.53417969,169.99998638)
\lineto(208.55761719,169.99998638)
\lineto(208.55761719,182.03123638)
\moveto(208.55761719,186.71483013)
\lineto(210.53417969,186.71483013)
\lineto(210.53417969,184.21190044)
\lineto(208.55761719,184.21190044)
\lineto(208.55761719,186.71483013)
}
}
{
\newrgbcolor{curcolor}{0 0 0}
\pscustom[linestyle=none,fillstyle=solid,fillcolor=curcolor]
{
\newpath
\moveto(222.32910156,181.67674419)
\lineto(222.32910156,179.80760356)
\curveto(221.77049863,180.0940518)(221.19042108,180.30889534)(220.58886719,180.45213481)
\curveto(219.98729729,180.59535339)(219.36425103,180.6669679)(218.71972656,180.66697856)
\curveto(217.73860162,180.6669679)(217.00097215,180.51657742)(216.50683594,180.21580669)
\curveto(216.01985334,179.91501552)(215.776364,179.4638441)(215.77636719,178.86229106)
\curveto(215.776364,178.40394933)(215.95181955,178.04229604)(216.30273438,177.77733013)
\curveto(216.65364177,177.51951011)(217.35904471,177.27244004)(218.41894531,177.03611919)
\lineto(219.09570312,176.88572856)
\curveto(220.49934104,176.58494073)(221.49478276,176.15883438)(222.08203125,175.60740825)
\curveto(222.6764222,175.06313236)(222.97362242,174.30043781)(222.97363281,173.31932231)
\curveto(222.97362242,172.20213261)(222.52961245,171.31769339)(221.64160156,170.666002)
\curveto(220.76073401,170.01430928)(219.54686804,169.68846325)(218,169.68846294)
\curveto(217.35546398,169.68846325)(216.68228757,169.75291631)(215.98046875,169.88182231)
\curveto(215.28580459,170.0035671)(214.55175585,170.18976483)(213.77832031,170.44041606)
\lineto(213.77832031,172.48143169)
\curveto(214.50878714,172.1018723)(215.22851298,171.81541425)(215.9375,171.62205669)
\curveto(216.64648031,171.43585734)(217.34830253,171.34275847)(218.04296875,171.34275981)
\curveto(218.97395195,171.34275847)(219.69009706,171.5003104)(220.19140625,171.81541606)
\curveto(220.69270023,172.13767955)(220.94335102,172.58885098)(220.94335938,173.16893169)
\curveto(220.94335102,173.70603736)(220.76073401,174.1178208)(220.39550781,174.40428325)
\curveto(220.03742744,174.69073689)(219.24608709,174.96645276)(218.02148438,175.23143169)
\lineto(217.33398438,175.3925645)
\curveto(216.10937148,175.65037135)(215.22493226,176.04425117)(214.68066406,176.57420513)
\curveto(214.13639168,177.11130739)(213.86425653,177.84535614)(213.86425781,178.77635356)
\curveto(213.86425653,179.90785407)(214.2652978,180.78155112)(215.06738281,181.39744731)
\curveto(215.86946286,182.01332072)(217.0081336,182.32126312)(218.48339844,182.32127544)
\curveto(219.21386056,182.32126312)(219.90135987,182.26755223)(220.54589844,182.16014263)
\curveto(221.19042108,182.0527087)(221.78482153,181.89157605)(222.32910156,181.67674419)
}
}
{
\newrgbcolor{curcolor}{0 0 0}
\pscustom[linestyle=none,fillstyle=solid,fillcolor=curcolor]
{
\newpath
\moveto(236.42285156,176.509752)
\lineto(236.42285156,175.54295513)
\lineto(227.33496094,175.54295513)
\curveto(227.42089508,174.18227386)(227.82909779,173.14386344)(228.55957031,172.42772075)
\curveto(229.29719528,171.71873466)(230.3212828,171.36424283)(231.63183594,171.36424419)
\curveto(232.39094219,171.36424283)(233.12499093,171.45734169)(233.83398438,171.64354106)
\curveto(234.55011972,171.82973715)(235.25910338,172.10903375)(235.9609375,172.48143169)
\lineto(235.9609375,170.61229106)
\curveto(235.25194193,170.3115095)(234.52505464,170.08234307)(233.78027344,169.92479106)
\curveto(233.03547279,169.76723921)(232.2799397,169.68846325)(231.51367188,169.68846294)
\curveto(229.59439551,169.68846325)(228.07258713,170.24705644)(226.94824219,171.36424419)
\curveto(225.83105292,172.48142921)(225.27245972,173.99249541)(225.27246094,175.89744731)
\curveto(225.27245972,177.86684049)(225.80240711,179.42803684)(226.86230469,180.58104106)
\curveto(227.92935811,181.74118557)(229.36522907,182.32126312)(231.16992188,182.32127544)
\curveto(232.78840273,182.32126312)(234.06672176,181.79847718)(235.00488281,180.75291606)
\curveto(235.95018342,179.71449489)(236.4228392,178.30010829)(236.42285156,176.509752)
\moveto(234.44628906,177.08983013)
\curveto(234.43195577,178.17120216)(234.1275941,179.03415703)(233.53320312,179.67869731)
\curveto(232.94595465,180.32321824)(232.16535648,180.64548354)(231.19140625,180.64549419)
\curveto(230.08853564,180.64548354)(229.20409642,180.33396042)(228.53808594,179.71092388)
\curveto(227.87922795,179.08786791)(227.49967104,178.21059015)(227.39941406,177.07908794)
\lineto(234.44628906,177.08983013)
}
}
{
\newrgbcolor{curcolor}{0 0 0}
\pscustom[linestyle=none,fillstyle=solid,fillcolor=curcolor]
{
\newpath
\moveto(246.63867188,180.18358013)
\curveto(246.41665784,180.31247606)(246.1731685,180.40557493)(245.90820312,180.462877)
\curveto(245.65038257,180.5273196)(245.36392452,180.55954613)(245.04882812,180.55955669)
\curveto(243.93163429,180.55954613)(243.07226015,180.19431212)(242.47070312,179.46385356)
\curveto(241.8762978,178.74053753)(241.57909758,177.69854639)(241.57910156,176.337877)
\lineto(241.57910156,169.99998638)
\lineto(239.59179688,169.99998638)
\lineto(239.59179688,182.03123638)
\lineto(241.57910156,182.03123638)
\lineto(241.57910156,180.16209575)
\curveto(241.99446175,180.89255361)(242.53515131,181.43324317)(243.20117188,181.78416606)
\curveto(243.86718123,182.14222684)(244.67642521,182.32126312)(245.62890625,182.32127544)
\curveto(245.76496579,182.32126312)(245.91535626,182.31052094)(246.08007812,182.28904888)
\curveto(246.24478302,182.27471369)(246.42740002,182.24964861)(246.62792969,182.21385356)
\lineto(246.63867188,180.18358013)
}
}
{
\newrgbcolor{curcolor}{0 0 0}
\pscustom[linestyle=none,fillstyle=solid,fillcolor=curcolor]
{
}
}
{
\newrgbcolor{curcolor}{0 0 0}
\pscustom[linestyle=none,fillstyle=solid,fillcolor=curcolor]
{
\newpath
\moveto(265.10449219,179.72166606)
\curveto(265.59862088,180.60967629)(266.1894406,181.26494907)(266.87695312,181.68748638)
\curveto(267.56443922,182.11000031)(268.37368321,182.32126312)(269.3046875,182.32127544)
\curveto(270.55792581,182.32126312)(271.52472172,181.88083387)(272.20507812,180.99998638)
\curveto(272.88539745,180.12627833)(273.22556638,178.88018583)(273.22558594,177.26170513)
\lineto(273.22558594,169.99998638)
\lineto(271.23828125,169.99998638)
\lineto(271.23828125,177.197252)
\curveto(271.23826368,178.35023844)(271.03416232,179.20603186)(270.62597656,179.76463481)
\curveto(270.21775688,180.32321824)(269.59471063,180.60251484)(268.75683594,180.60252544)
\curveto(267.73273333,180.60251484)(266.92348934,180.26234591)(266.32910156,179.58201763)
\curveto(265.73468845,178.90167018)(265.43748823,177.97426226)(265.4375,176.79979106)
\lineto(265.4375,169.99998638)
\lineto(263.45019531,169.99998638)
\lineto(263.45019531,177.197252)
\curveto(263.45018553,178.35739989)(263.24608417,179.21319331)(262.83789062,179.76463481)
\curveto(262.42967873,180.32321824)(261.79947103,180.60251484)(260.94726562,180.60252544)
\curveto(259.93749373,180.60251484)(259.1354112,180.25876518)(258.54101562,179.57127544)
\curveto(257.9466103,178.89092801)(257.64941008,177.96710081)(257.64941406,176.79979106)
\lineto(257.64941406,169.99998638)
\lineto(255.66210938,169.99998638)
\lineto(255.66210938,182.03123638)
\lineto(257.64941406,182.03123638)
\lineto(257.64941406,180.16209575)
\curveto(258.1005815,180.89971506)(258.64127106,181.44398535)(259.27148438,181.79490825)
\curveto(259.90168647,182.14580756)(260.65005812,182.32126312)(261.51660156,182.32127544)
\curveto(262.39029075,182.32126312)(263.13150095,182.09925813)(263.74023438,181.65525981)
\curveto(264.3561091,181.21123819)(264.81086125,180.56670758)(265.10449219,179.72166606)
}
}
{
\newrgbcolor{curcolor}{0 0 0}
\pscustom[linestyle=none,fillstyle=solid,fillcolor=curcolor]
{
\newpath
\moveto(281.84082031,180.64549419)
\curveto(280.7809188,180.64548354)(279.94302902,180.23011938)(279.32714844,179.39940044)
\curveto(278.71125942,178.57582416)(278.40331701,177.44431487)(278.40332031,176.00486919)
\curveto(278.40331701,174.5654115)(278.70767869,173.43032149)(279.31640625,172.59959575)
\curveto(279.93228684,171.77602627)(280.77375735,171.36424283)(281.84082031,171.36424419)
\curveto(282.8935469,171.36424283)(283.72785596,171.77960699)(284.34375,172.61033794)
\curveto(284.95962556,173.44106367)(285.26756796,174.57257295)(285.26757812,176.00486919)
\curveto(285.26756796,177.42999197)(284.95962556,178.55792053)(284.34375,179.38865825)
\curveto(283.72785596,180.22653865)(282.8935469,180.64548354)(281.84082031,180.64549419)
\moveto(281.84082031,182.32127544)
\curveto(283.55956186,182.32126312)(284.9094954,181.76266993)(285.890625,180.64549419)
\curveto(286.87173303,179.52829716)(287.36229243,177.98142371)(287.36230469,176.00486919)
\curveto(287.36229243,174.03546411)(286.87173303,172.48859066)(285.890625,171.36424419)
\curveto(284.9094954,170.24705644)(283.55956186,169.68846325)(281.84082031,169.68846294)
\curveto(280.11490384,169.68846325)(278.76138957,170.24705644)(277.78027344,171.36424419)
\curveto(276.8063134,172.48859066)(276.31933472,174.03546411)(276.31933594,176.00486919)
\curveto(276.31933472,177.98142371)(276.8063134,179.52829716)(277.78027344,180.64549419)
\curveto(278.76138957,181.76266993)(280.11490384,182.32126312)(281.84082031,182.32127544)
}
}
{
\newrgbcolor{curcolor}{0 0 0}
\pscustom[linestyle=none,fillstyle=solid,fillcolor=curcolor]
{
\newpath
\moveto(292.58300781,185.447252)
\lineto(292.58300781,182.03123638)
\lineto(296.65429688,182.03123638)
\lineto(296.65429688,180.49510356)
\lineto(292.58300781,180.49510356)
\lineto(292.58300781,173.96385356)
\curveto(292.58300378,172.98273079)(292.71549063,172.35252309)(292.98046875,172.07322856)
\curveto(293.25259947,171.7939299)(293.80045048,171.6542816)(294.62402344,171.65428325)
\lineto(296.65429688,171.65428325)
\lineto(296.65429688,169.99998638)
\lineto(294.62402344,169.99998638)
\curveto(293.09862827,169.99998638)(292.04589495,170.2828637)(291.46582031,170.84861919)
\curveto(290.88573986,171.42153443)(290.59570108,172.45994485)(290.59570312,173.96385356)
\lineto(290.59570312,180.49510356)
\lineto(289.14550781,180.49510356)
\lineto(289.14550781,182.03123638)
\lineto(290.59570312,182.03123638)
\lineto(290.59570312,185.447252)
\lineto(292.58300781,185.447252)
}
}
{
\newrgbcolor{curcolor}{0 0 0}
\pscustom[linestyle=none,fillstyle=solid,fillcolor=curcolor]
{
}
}
{
\newrgbcolor{curcolor}{0 0 0}
\pscustom[linestyle=none,fillstyle=solid,fillcolor=curcolor]
{
\newpath
\moveto(314.18554688,180.2050645)
\lineto(314.18554688,186.71483013)
\lineto(316.16210938,186.71483013)
\lineto(316.16210938,169.99998638)
\lineto(314.18554688,169.99998638)
\lineto(314.18554688,171.80467388)
\curveto(313.77017272,171.08852696)(313.24380606,170.55499884)(312.60644531,170.20408794)
\curveto(311.9762292,169.86033808)(311.21711537,169.68846325)(310.32910156,169.68846294)
\curveto(308.87532084,169.68846325)(307.69010067,170.2685408)(306.7734375,171.42869731)
\curveto(305.86393062,172.58885098)(305.40917847,174.11424008)(305.40917969,176.00486919)
\curveto(305.40917847,177.89548629)(305.86393062,179.42087539)(306.7734375,180.58104106)
\curveto(307.69010067,181.74118557)(308.87532084,182.32126312)(310.32910156,182.32127544)
\curveto(311.21711537,182.32126312)(311.9762292,182.14580756)(312.60644531,181.79490825)
\curveto(313.24380606,181.4511468)(313.77017272,180.92119941)(314.18554688,180.2050645)
\moveto(307.45019531,176.00486919)
\curveto(307.45019206,174.5510886)(307.74739228,173.40883713)(308.34179688,172.57811138)
\curveto(308.94335463,171.75454191)(309.76692151,171.34275847)(310.8125,171.34275981)
\curveto(311.85806525,171.34275847)(312.68163214,171.75454191)(313.28320312,172.57811138)
\curveto(313.88475594,173.40883713)(314.18553688,174.5510886)(314.18554688,176.00486919)
\curveto(314.18553688,177.45863777)(313.88475594,178.59730851)(313.28320312,179.42088481)
\curveto(312.68163214,180.25160373)(311.85806525,180.6669679)(310.8125,180.66697856)
\curveto(309.76692151,180.6669679)(308.94335463,180.25160373)(308.34179688,179.42088481)
\curveto(307.74739228,178.59730851)(307.45019206,177.45863777)(307.45019531,176.00486919)
}
}
{
\newrgbcolor{curcolor}{0 0 0}
\pscustom[linestyle=none,fillstyle=solid,fillcolor=curcolor]
{
\newpath
\moveto(330.52441406,176.509752)
\lineto(330.52441406,175.54295513)
\lineto(321.43652344,175.54295513)
\curveto(321.52245758,174.18227386)(321.93066029,173.14386344)(322.66113281,172.42772075)
\curveto(323.39875778,171.71873466)(324.4228453,171.36424283)(325.73339844,171.36424419)
\curveto(326.49250469,171.36424283)(327.22655343,171.45734169)(327.93554688,171.64354106)
\curveto(328.65168222,171.82973715)(329.36066588,172.10903375)(330.0625,172.48143169)
\lineto(330.0625,170.61229106)
\curveto(329.35350443,170.3115095)(328.62661714,170.08234307)(327.88183594,169.92479106)
\curveto(327.13703529,169.76723921)(326.3815022,169.68846325)(325.61523438,169.68846294)
\curveto(323.69595801,169.68846325)(322.17414963,170.24705644)(321.04980469,171.36424419)
\curveto(319.93261542,172.48142921)(319.37402222,173.99249541)(319.37402344,175.89744731)
\curveto(319.37402222,177.86684049)(319.90396961,179.42803684)(320.96386719,180.58104106)
\curveto(322.03092061,181.74118557)(323.46679157,182.32126312)(325.27148438,182.32127544)
\curveto(326.88996523,182.32126312)(328.16828426,181.79847718)(329.10644531,180.75291606)
\curveto(330.05174592,179.71449489)(330.5244017,178.30010829)(330.52441406,176.509752)
\moveto(328.54785156,177.08983013)
\curveto(328.53351827,178.17120216)(328.2291566,179.03415703)(327.63476562,179.67869731)
\curveto(327.04751715,180.32321824)(326.26691898,180.64548354)(325.29296875,180.64549419)
\curveto(324.19009814,180.64548354)(323.30565892,180.33396042)(322.63964844,179.71092388)
\curveto(321.98079045,179.08786791)(321.60123354,178.21059015)(321.50097656,177.07908794)
\lineto(328.54785156,177.08983013)
}
}
{
\newrgbcolor{curcolor}{0 0 0}
\pscustom[linestyle=none,fillstyle=solid,fillcolor=curcolor]
{
}
}
{
\newrgbcolor{curcolor}{0 0 0}
\pscustom[linestyle=none,fillstyle=solid,fillcolor=curcolor]
{
\newpath
\moveto(342.68457031,171.80467388)
\lineto(342.68457031,165.4238145)
\lineto(340.69726562,165.4238145)
\lineto(340.69726562,182.03123638)
\lineto(342.68457031,182.03123638)
\lineto(342.68457031,180.2050645)
\curveto(343.0999305,180.92119941)(343.62271643,181.4511468)(344.25292969,181.79490825)
\curveto(344.89029329,182.14580756)(345.64940711,182.32126312)(346.53027344,182.32127544)
\curveto(347.99120165,182.32126312)(349.17642181,181.74118557)(350.0859375,180.58104106)
\curveto(351.00259186,179.42087539)(351.46092474,177.89548629)(351.4609375,176.00486919)
\curveto(351.46092474,174.11424008)(351.00259186,172.58885098)(350.0859375,171.42869731)
\curveto(349.17642181,170.2685408)(347.99120165,169.68846325)(346.53027344,169.68846294)
\curveto(345.64940711,169.68846325)(344.89029329,169.86033808)(344.25292969,170.20408794)
\curveto(343.62271643,170.55499884)(343.0999305,171.08852696)(342.68457031,171.80467388)
\moveto(349.40917969,176.00486919)
\curveto(349.40916898,177.45863777)(349.10838803,178.59730851)(348.50683594,179.42088481)
\curveto(347.91242568,180.25160373)(347.09243952,180.6669679)(346.046875,180.66697856)
\curveto(345.00129578,180.6669679)(344.1777289,180.25160373)(343.57617188,179.42088481)
\curveto(342.98176655,178.59730851)(342.68456633,177.45863777)(342.68457031,176.00486919)
\curveto(342.68456633,174.5510886)(342.98176655,173.40883713)(343.57617188,172.57811138)
\curveto(344.1777289,171.75454191)(345.00129578,171.34275847)(346.046875,171.34275981)
\curveto(347.09243952,171.34275847)(347.91242568,171.75454191)(348.50683594,172.57811138)
\curveto(349.10838803,173.40883713)(349.40916898,174.5510886)(349.40917969,176.00486919)
}
}
{
\newrgbcolor{curcolor}{0 0 0}
\pscustom[linestyle=none,fillstyle=solid,fillcolor=curcolor]
{
\newpath
\moveto(360.20507812,176.04783794)
\curveto(358.60806697,176.04783189)(357.50162277,175.86521489)(356.88574219,175.49998638)
\curveto(356.26985317,175.13474687)(355.96191076,174.51170062)(355.96191406,173.63084575)
\curveto(355.96191076,172.92901991)(356.1910772,172.37042671)(356.64941406,171.9550645)
\curveto(357.1149044,171.54685983)(357.74511211,171.34275847)(358.54003906,171.34275981)
\curveto(359.63573522,171.34275847)(360.51301298,171.72947683)(361.171875,172.50291606)
\curveto(361.83788145,173.28351174)(362.17088893,174.31834143)(362.17089844,175.60740825)
\lineto(362.17089844,176.04783794)
\lineto(360.20507812,176.04783794)
\moveto(364.14746094,176.86424419)
\lineto(364.14746094,169.99998638)
\lineto(362.17089844,169.99998638)
\lineto(362.17089844,171.82615825)
\curveto(361.71971751,171.09568841)(361.15754359,170.55499884)(360.484375,170.20408794)
\curveto(359.81119077,169.86033808)(358.98762388,169.68846325)(358.01367188,169.68846294)
\curveto(356.78189692,169.68846325)(355.80077811,170.03221291)(355.0703125,170.71971294)
\curveto(354.34700353,171.41437298)(353.98535024,172.34178091)(353.98535156,173.5019395)
\curveto(353.98535024,174.85545027)(354.43652167,175.87595706)(355.33886719,176.56346294)
\curveto(356.24836881,177.25095569)(357.60188308,177.59470534)(359.39941406,177.59471294)
\lineto(362.17089844,177.59471294)
\lineto(362.17089844,177.78807231)
\curveto(362.17088893,178.69756883)(361.87010798,179.39939104)(361.26855469,179.89354106)
\curveto(360.67414564,180.39483275)(359.83625585,180.64548354)(358.75488281,180.64549419)
\curveto(358.06737741,180.64548354)(357.39778172,180.56312686)(356.74609375,180.39842388)
\curveto(356.09439761,180.2337001)(355.46777063,179.98663004)(354.86621094,179.65721294)
\lineto(354.86621094,181.48338481)
\curveto(355.5895153,181.76266993)(356.29133752,181.97035201)(356.97167969,182.10643169)
\curveto(357.65201324,182.24964861)(358.31444747,182.32126312)(358.95898438,182.32127544)
\curveto(360.69921071,182.32126312)(361.9990141,181.87009169)(362.85839844,180.96775981)
\curveto(363.71776238,180.065406)(364.14744945,178.69756883)(364.14746094,176.86424419)
}
}
{
\newrgbcolor{curcolor}{0 0 0}
\pscustom[linestyle=none,fillstyle=solid,fillcolor=curcolor]
{
\newpath
\moveto(375.89941406,181.67674419)
\lineto(375.89941406,179.80760356)
\curveto(375.34081113,180.0940518)(374.76073358,180.30889534)(374.15917969,180.45213481)
\curveto(373.55760979,180.59535339)(372.93456353,180.6669679)(372.29003906,180.66697856)
\curveto(371.30891412,180.6669679)(370.57128465,180.51657742)(370.07714844,180.21580669)
\curveto(369.59016584,179.91501552)(369.3466765,179.4638441)(369.34667969,178.86229106)
\curveto(369.3466765,178.40394933)(369.52213205,178.04229604)(369.87304688,177.77733013)
\curveto(370.22395427,177.51951011)(370.92935721,177.27244004)(371.98925781,177.03611919)
\lineto(372.66601562,176.88572856)
\curveto(374.06965354,176.58494073)(375.06509526,176.15883438)(375.65234375,175.60740825)
\curveto(376.2467347,175.06313236)(376.54393492,174.30043781)(376.54394531,173.31932231)
\curveto(376.54393492,172.20213261)(376.09992495,171.31769339)(375.21191406,170.666002)
\curveto(374.33104651,170.01430928)(373.11718054,169.68846325)(371.5703125,169.68846294)
\curveto(370.92577648,169.68846325)(370.25260007,169.75291631)(369.55078125,169.88182231)
\curveto(368.85611709,170.0035671)(368.12206835,170.18976483)(367.34863281,170.44041606)
\lineto(367.34863281,172.48143169)
\curveto(368.07909964,172.1018723)(368.79882548,171.81541425)(369.5078125,171.62205669)
\curveto(370.21679281,171.43585734)(370.91861503,171.34275847)(371.61328125,171.34275981)
\curveto(372.54426445,171.34275847)(373.26040956,171.5003104)(373.76171875,171.81541606)
\curveto(374.26301273,172.13767955)(374.51366352,172.58885098)(374.51367188,173.16893169)
\curveto(374.51366352,173.70603736)(374.33104651,174.1178208)(373.96582031,174.40428325)
\curveto(373.60773994,174.69073689)(372.81639959,174.96645276)(371.59179688,175.23143169)
\lineto(370.90429688,175.3925645)
\curveto(369.67968398,175.65037135)(368.79524476,176.04425117)(368.25097656,176.57420513)
\curveto(367.70670418,177.11130739)(367.43456903,177.84535614)(367.43457031,178.77635356)
\curveto(367.43456903,179.90785407)(367.8356103,180.78155112)(368.63769531,181.39744731)
\curveto(369.43977536,182.01332072)(370.5784461,182.32126312)(372.05371094,182.32127544)
\curveto(372.78417306,182.32126312)(373.47167237,182.26755223)(374.11621094,182.16014263)
\curveto(374.76073358,182.0527087)(375.35513403,181.89157605)(375.89941406,181.67674419)
}
}
{
\newrgbcolor{curcolor}{0 0 0}
\pscustom[linestyle=none,fillstyle=solid,fillcolor=curcolor]
{
\newpath
\moveto(387.37207031,181.67674419)
\lineto(387.37207031,179.80760356)
\curveto(386.81346738,180.0940518)(386.23338983,180.30889534)(385.63183594,180.45213481)
\curveto(385.03026604,180.59535339)(384.40721978,180.6669679)(383.76269531,180.66697856)
\curveto(382.78157037,180.6669679)(382.0439409,180.51657742)(381.54980469,180.21580669)
\curveto(381.06282209,179.91501552)(380.81933275,179.4638441)(380.81933594,178.86229106)
\curveto(380.81933275,178.40394933)(380.9947883,178.04229604)(381.34570312,177.77733013)
\curveto(381.69661052,177.51951011)(382.40201346,177.27244004)(383.46191406,177.03611919)
\lineto(384.13867188,176.88572856)
\curveto(385.54230979,176.58494073)(386.53775151,176.15883438)(387.125,175.60740825)
\curveto(387.71939095,175.06313236)(388.01659117,174.30043781)(388.01660156,173.31932231)
\curveto(388.01659117,172.20213261)(387.5725812,171.31769339)(386.68457031,170.666002)
\curveto(385.80370276,170.01430928)(384.58983679,169.68846325)(383.04296875,169.68846294)
\curveto(382.39843273,169.68846325)(381.72525632,169.75291631)(381.0234375,169.88182231)
\curveto(380.32877334,170.0035671)(379.5947246,170.18976483)(378.82128906,170.44041606)
\lineto(378.82128906,172.48143169)
\curveto(379.55175589,172.1018723)(380.27148173,171.81541425)(380.98046875,171.62205669)
\curveto(381.68944906,171.43585734)(382.39127128,171.34275847)(383.0859375,171.34275981)
\curveto(384.0169207,171.34275847)(384.73306581,171.5003104)(385.234375,171.81541606)
\curveto(385.73566898,172.13767955)(385.98631977,172.58885098)(385.98632812,173.16893169)
\curveto(385.98631977,173.70603736)(385.80370276,174.1178208)(385.43847656,174.40428325)
\curveto(385.08039619,174.69073689)(384.28905584,174.96645276)(383.06445312,175.23143169)
\lineto(382.37695312,175.3925645)
\curveto(381.15234023,175.65037135)(380.26790101,176.04425117)(379.72363281,176.57420513)
\curveto(379.17936043,177.11130739)(378.90722528,177.84535614)(378.90722656,178.77635356)
\curveto(378.90722528,179.90785407)(379.30826655,180.78155112)(380.11035156,181.39744731)
\curveto(380.91243161,182.01332072)(382.05110235,182.32126312)(383.52636719,182.32127544)
\curveto(384.25682931,182.32126312)(384.94432862,182.26755223)(385.58886719,182.16014263)
\curveto(386.23338983,182.0527087)(386.82779028,181.89157605)(387.37207031,181.67674419)
}
}
{
\newrgbcolor{curcolor}{0 0 0}
\pscustom[linestyle=none,fillstyle=solid,fillcolor=curcolor]
{
\newpath
\moveto(401.46582031,176.509752)
\lineto(401.46582031,175.54295513)
\lineto(392.37792969,175.54295513)
\curveto(392.46386383,174.18227386)(392.87206654,173.14386344)(393.60253906,172.42772075)
\curveto(394.34016403,171.71873466)(395.36425155,171.36424283)(396.67480469,171.36424419)
\curveto(397.43391094,171.36424283)(398.16795968,171.45734169)(398.87695312,171.64354106)
\curveto(399.59308847,171.82973715)(400.30207213,172.10903375)(401.00390625,172.48143169)
\lineto(401.00390625,170.61229106)
\curveto(400.29491068,170.3115095)(399.56802339,170.08234307)(398.82324219,169.92479106)
\curveto(398.07844154,169.76723921)(397.32290845,169.68846325)(396.55664062,169.68846294)
\curveto(394.63736426,169.68846325)(393.11555588,170.24705644)(391.99121094,171.36424419)
\curveto(390.87402167,172.48142921)(390.31542847,173.99249541)(390.31542969,175.89744731)
\curveto(390.31542847,177.86684049)(390.84537586,179.42803684)(391.90527344,180.58104106)
\curveto(392.97232686,181.74118557)(394.40819782,182.32126312)(396.21289062,182.32127544)
\curveto(397.83137148,182.32126312)(399.10969051,181.79847718)(400.04785156,180.75291606)
\curveto(400.99315217,179.71449489)(401.46580795,178.30010829)(401.46582031,176.509752)
\moveto(399.48925781,177.08983013)
\curveto(399.47492452,178.17120216)(399.17056285,179.03415703)(398.57617188,179.67869731)
\curveto(397.9889234,180.32321824)(397.20832523,180.64548354)(396.234375,180.64549419)
\curveto(395.13150439,180.64548354)(394.24706517,180.33396042)(393.58105469,179.71092388)
\curveto(392.9221967,179.08786791)(392.54263979,178.21059015)(392.44238281,177.07908794)
\lineto(399.48925781,177.08983013)
}
}
\end{pspicture}
		
		\end{center}
		
		\begin{enumerate}
		  \item Fond d'écran (présent dans l'archive stocké sur le téléphone)
		  \item Pseudonyme du compte hors ligne utilisé
		  \item Bouton ``\hyperlink{Creation compte multi-joueurs}{Inscription}''
		  \item Cadre contenant l'erreur survenue
		  \item Champs de texte ``Nom utilisateur''
		  \item Champs de texte ``Mot de passe''
		  \item Check Box ``Connexion auto''
		  \item Bouton ``\hyperlink{Accueil}{Retour}''
		  \item Bouton ``\hyperlink{Accueil multi-joueurs}{Connexion}''
		\end{enumerate}

		\subsubsection{Description des zones}
		
			\begin{tabular}{|c|c|c|c|c|} \hline
				Numéro de zone & Type  & Description & Evènement &	Règle \\\hline
				2 & Label & Affiche le pseudonyme du compte & Chargement de la page & RG4-01 \\
				  &       & hors ligne en cours d'utilisation & & \\\hline
				3 & Bouton & Permet à l'utilisateur d'être redirigé & Cliqué & RG5-01\\
				  &        & sur la page de création  & & \\
				  &        & d'un compte multi-joueurs \footnotemark[1] & & \\\hline
				4 & Label & Affiche l'erreur & Serveur distant & RG4-02 \\\hline
				7 & Check box & Permet la connexion automatique & Cliqué & RG4-05 \\
				  &           & lors des prochaines utilisations&        & \\				
				  &           & du multi-joueurs                &        & \\\hline
				8 & Bouton & Permet de revenir à la page & Cliqué & RG5-02 \\
				  &        & de connexion multi-joueurs \footnotemark[2] & & \\\hline
				9 & Bouton & Valide les paramètres entrés & Cliqué & RG5-03 \\\hline
			\end{tabular}
			
		\subsubsection{Description des règles}

			\underline{RG5-01 :}
				\begin{quote}
					Charger la page de création d'un compte multi-joueurs%
						\footnote[1]{
							\hyperlink{Creation compte multi-joueurs}{Creation compte multi-joueurs}
							\og voir section \ref{Creation compte multi-joueurs}, page \pageref{Creation compte multi-joueurs}.\fg
						}.\\
					Afficher la page de création d'un compte multi-joueurs\footnotemark[1].\\
					Supprimer la page de connexion multi-joueurs%
						\footnote[2]{
							\hyperlink{Connexion multi-joueurs}{Connexion multi-joueurs}
							\og voir section \ref{Connexion multi-joueurs}, page \pageref{Connexion multi-joueurs}.\fg
						}.\\
				\end{quote}	

				
			\underline{RG5-02 :}
				\begin{quote}
					Afficher la page d'accueil%
						\footnote[3]{
							\hyperlink{Page d'accueil}{Page d'accueil}
							\og voir section \ref{Accueil}, page \pageref{Accueil}.\fg
						}.\\
					Supprimer la page de connexion multi-joueurs \footnotemark[2].		
				\end{quote}	
				
			$\,$
				
			\underline{RG5-03 :}
				\begin{quote}
					Vérification des paramètres passés.\\
					Connexion au serveur distant.\\
					Si le nom du compte et le password sont bons alors
					\begin{quote}					
						Charger la page d'accueil multi-joueurs%
							\footnote[4]{
								\hyperlink{Accueil multi-joueurs}{Accueil multi-joueurs}
								\og voir section \ref{Accueil multi-joueurs}, page \pageref{Accueil multi-joueurs}.\fg
							}.\\
						Afficher la page d'accueil multi-joueurs \footnotemark[4].\\
						Supprimer la page de connexion multi-joueurs \footnotemark[2].
					\end{quote}	
					Sinon
					\begin{quote}
						Afficher l'erreur dans le label d'erreur (4).
					\end{quote}					
				\end{quote}	

\newpage

	\subsection{Accueil multi-joueurs}

		\hypertarget{Accueil multi-joueurs}{}
		\label{Accueil multi-joueurs}

		\begin{center}
			\input{./tex/6_Accueil_multi_joueurs}
		\end{center}

		\begin{enumerate}
		  \item Fond d'écran (présent dans l'archive stocké sur le téléphone)
		  \item Champs de texte ``Filtrer''
		  \item Bouton ``Rafraichir''
		  \item Cadre contenant les différentes parties en ligne en cours
		  \item Bouton ``\hyperlink{Accueil}{Retour}''
		  \item Bouton ``\hyperlink{Creer partie multi-joueurs}{Créer partie}''
		\end{enumerate}

		\subsubsection{Description des zones}

			\begin{tabular}{|c|c|c|c|c|} \hline
				Numéro de zone & Type  & Description & Evènement &	Règle \\\hline
				2 & Champs de texte & Permet de filtrer les parties & Texte saisi & RG6-01 \\
				  &                 & selon le texte tapé par l'utilisateur & & \\\hline
				3 & Bouton & Rafraichi la liste des parties & Cliqué & RG6-02 \\
				  &        & en cours & & \\\hline
				4 & Label & Liste des parties en cours & Chargement de la page & RG6-03 \\
				  &       &                            & RG6-01 & RG6-04 \\
				  &       &                            & RG6-02 & RG6-03 \\
				  &       &                            & Cliqué & RG6-05 \\\hline
				5 & Bouton & Permet à l'utilisateur de & Cliqué & RG6-06 \\
				  &        & revenir à la page d'acceuil & & \\\hline
				6 & Bouton & Affiche la page de création de & Cliqué & RG6-07 \\
				  &        & création d'une partie multi-joueurs & & \\\hline    
			\end{tabular}

		\subsubsection{Description des règles}

			\underline{RG6-01 :}
				\begin{quote}
					Si le champs de texte (2) n'est pas vide alors
					\begin{quote}	
						Envoyer le contenu à RG6-04
					\end{quote}					
				\end{quote}	
		
			$\,$
				
			\underline{RG6-02 :}
				\begin{quote}	
					RG6-03
				\end{quote}

			$\,$	
				
			\underline{RG6-03 :}
				\begin{quote}
					Requête au serveur distant afin de récupérer la liste des parties en cours.\\
					Si la liste n'est pas vide
					\begin{quote}
						Afficher la liste.
					\end{quote}								
				\end{quote}

			$\,$

			\underline{RG6-04 :}
				\begin{quote}
					Filtrer l'affichage du label (4) selon le texte reçu.
				\end{quote}

			$\,$
				
			\underline{RG6-05 :}
				\begin{quote}
					Connexion au serveur distant afin de rejoindre la partie.
				\end{quote}

			$\,$
				
			\underline{RG6-06 :}
				\begin{quote}
					Charger la page d'accueil%
						\footnote[1]{
							\hyperlink{Page d'accueil}{Page d'accueil}
							\og voir section \ref{Accueil}, page \pageref{Accueil}.\fg
						}.\\
					Afficher la page d'accueil\footnotemark[1].\\
					Supprimer la page d'accueil multi-joueurs%
						\footnote[2]{
							\hyperlink{Accueil multi-joueurs}{Accueil multi-joueurs}
							\og voir section \ref{Accueil multi-joueurs}, page \pageref{Accueil multi-joueurs}.\fg
						}.
				\end{quote}

			$\,$

			\underline{RG6-07 :}
				\begin{quote}
					Charger la page de création d'une partie multi-joueurs.%
						\footnote[3]{
							\hyperlink{Creer partie multi-joueurs}{Créer partie multi-joueurs}
							\og voir section \ref{Creer partie multi-joueurs}, page \pageref{Creer partie multi-joueurs}.\fg
						}.\\
					Afficher la page de création d'une partie multi-joueurs\footnotemark[3].\\
					Supprimer la page d'accueil multi-joueurs\footnotemark[2].
				\end{quote}

\newpage

	\subsection{Créer une partie multi-joueurs}
	
		\hypertarget{Creer partie multi-joueurs}{}
		\label{Creer partie multi-joueurs}
	
		\begin{center}
			%LaTeX with PSTricks extensions
%%Creator: inkscape 0.48.0
%%Please note this file requires PSTricks extensions
\psset{xunit=.5pt,yunit=.5pt,runit=.5pt}
\begin{pspicture}(560,600)
{
\newrgbcolor{curcolor}{1 1 1}
\pscustom[linestyle=none,fillstyle=solid,fillcolor=curcolor]
{
\newpath
\moveto(133.12401581,597.52220273)
\lineto(426.87598419,597.52220273)
\curveto(443.85397169,597.52220273)(457.52217102,583.8540034)(457.52217102,566.8760159)
\lineto(457.52217102,33.124017)
\curveto(457.52217102,16.1460295)(443.85397169,2.47783017)(426.87598419,2.47783017)
\lineto(133.12401581,2.47783017)
\curveto(116.14602831,2.47783017)(102.47782898,16.1460295)(102.47782898,33.124017)
\lineto(102.47782898,566.8760159)
\curveto(102.47782898,583.8540034)(116.14602831,597.52220273)(133.12401581,597.52220273)
\closepath
}
}
{
\newrgbcolor{curcolor}{0 0 0}
\pscustom[linewidth=4.95566034,linecolor=curcolor]
{
\newpath
\moveto(133.12401581,597.52220273)
\lineto(426.87598419,597.52220273)
\curveto(443.85397169,597.52220273)(457.52217102,583.8540034)(457.52217102,566.8760159)
\lineto(457.52217102,33.124017)
\curveto(457.52217102,16.1460295)(443.85397169,2.47783017)(426.87598419,2.47783017)
\lineto(133.12401581,2.47783017)
\curveto(116.14602831,2.47783017)(102.47782898,16.1460295)(102.47782898,33.124017)
\lineto(102.47782898,566.8760159)
\curveto(102.47782898,583.8540034)(116.14602831,597.52220273)(133.12401581,597.52220273)
\closepath
}
}
{
\newrgbcolor{curcolor}{1 1 1}
\pscustom[linestyle=none,fillstyle=solid,fillcolor=curcolor]
{
\newpath
\moveto(133.87602234,229.91494751)
\lineto(422.23780823,229.91494751)
\curveto(436.20654425,229.91494751)(447.45213318,218.66935858)(447.45213318,204.70062256)
\lineto(447.45213318,95.27957153)
\curveto(447.45213318,81.31083551)(436.20654425,70.06524658)(422.23780823,70.06524658)
\lineto(133.87602234,70.06524658)
\curveto(119.90728632,70.06524658)(108.66169739,81.31083551)(108.66169739,95.27957153)
\lineto(108.66169739,204.70062256)
\curveto(108.66169739,218.66935858)(119.90728632,229.91494751)(133.87602234,229.91494751)
\closepath
}
}
{
\newrgbcolor{curcolor}{0 0 0}
\pscustom[linewidth=1.87744212,linecolor=curcolor]
{
\newpath
\moveto(133.87602234,229.91494751)
\lineto(422.23780823,229.91494751)
\curveto(436.20654425,229.91494751)(447.45213318,218.66935858)(447.45213318,204.70062256)
\lineto(447.45213318,95.27957153)
\curveto(447.45213318,81.31083551)(436.20654425,70.06524658)(422.23780823,70.06524658)
\lineto(133.87602234,70.06524658)
\curveto(119.90728632,70.06524658)(108.66169739,81.31083551)(108.66169739,95.27957153)
\lineto(108.66169739,204.70062256)
\curveto(108.66169739,218.66935858)(119.90728632,229.91494751)(133.87602234,229.91494751)
\closepath
}
}
{
\newrgbcolor{curcolor}{0 0 0}
\pscustom[linestyle=none,fillstyle=solid,fillcolor=curcolor,opacity=0.11935484]
{
\newpath
\moveto(322.58069324,500)
\lineto(427.16530895,500)
\curveto(434.27572779,500)(440,495.83353307)(440,490.6581459)
\lineto(440,479.6365757)
\curveto(440,474.46118853)(434.27572779,470.2947216)(427.16530895,470.2947216)
\lineto(322.58069324,470.2947216)
\curveto(315.4702744,470.2947216)(309.7460022,474.46118853)(309.7460022,479.6365757)
\lineto(309.7460022,490.6581459)
\curveto(309.7460022,495.83353307)(315.4702744,500)(322.58069324,500)
\closepath
}
}
{
\newrgbcolor{curcolor}{0 0 0}
\pscustom[linewidth=2,linecolor=curcolor]
{
\newpath
\moveto(322.58069324,500)
\lineto(427.16530895,500)
\curveto(434.27572779,500)(440,495.83353307)(440,490.6581459)
\lineto(440,479.6365757)
\curveto(440,474.46118853)(434.27572779,470.2947216)(427.16530895,470.2947216)
\lineto(322.58069324,470.2947216)
\curveto(315.4702744,470.2947216)(309.7460022,474.46118853)(309.7460022,479.6365757)
\lineto(309.7460022,490.6581459)
\curveto(309.7460022,495.83353307)(315.4702744,500)(322.58069324,500)
\closepath
}
}
{
\newrgbcolor{curcolor}{1 1 1}
\pscustom[linestyle=none,fillstyle=solid,fillcolor=curcolor]
{
\newpath
\moveto(188.83351135,280.19555664)
\lineto(370.00863647,280.19555664)
\lineto(370.00863647,249.90542412)
\lineto(188.83351135,249.90542412)
\closepath
}
}
{
\newrgbcolor{curcolor}{0 0 0}
\pscustom[linewidth=2,linecolor=curcolor]
{
\newpath
\moveto(188.83351135,280.19555664)
\lineto(370.00863647,280.19555664)
\lineto(370.00863647,249.90542412)
\lineto(188.83351135,249.90542412)
\closepath
}
}
{
\newrgbcolor{curcolor}{1 1 1}
\pscustom[linestyle=none,fillstyle=solid,fillcolor=curcolor]
{
\newpath
\moveto(218.86587524,409.90138245)
\lineto(338.92390442,409.90138245)
\lineto(338.92390442,289.90190125)
\lineto(218.86587524,289.90190125)
\closepath
}
}
{
\newrgbcolor{curcolor}{0 0 0}
\pscustom[linewidth=2,linecolor=curcolor]
{
\newpath
\moveto(218.86587524,409.90138245)
\lineto(338.92390442,409.90138245)
\lineto(338.92390442,289.90190125)
\lineto(218.86587524,289.90190125)
\closepath
}
}
{
\newrgbcolor{curcolor}{1 1 1}
\pscustom[linestyle=none,fillstyle=solid,fillcolor=curcolor]
{
\newpath
\moveto(358.87203979,390.10528564)
\lineto(438.87251282,390.10528564)
\lineto(438.87251282,310.14836884)
\lineto(358.87203979,310.14836884)
\closepath
}
}
{
\newrgbcolor{curcolor}{0 0 0}
\pscustom[linewidth=2,linecolor=curcolor]
{
\newpath
\moveto(358.87203979,390.10528564)
\lineto(438.87251282,390.10528564)
\lineto(438.87251282,310.14836884)
\lineto(358.87203979,310.14836884)
\closepath
}
}
{
\newrgbcolor{curcolor}{1 1 1}
\pscustom[linestyle=none,fillstyle=solid,fillcolor=curcolor]
{
\newpath
\moveto(118.66197968,390.10528564)
\lineto(198.66246033,390.10528564)
\lineto(198.66246033,310.14836884)
\lineto(118.66197968,310.14836884)
\closepath
}
}
{
\newrgbcolor{curcolor}{0 0 0}
\pscustom[linewidth=2,linecolor=curcolor]
{
\newpath
\moveto(118.66197968,390.10528564)
\lineto(198.66246033,390.10528564)
\lineto(198.66246033,310.14836884)
\lineto(118.66197968,310.14836884)
\closepath
}
}
{
\newrgbcolor{curcolor}{0 0 0}
\pscustom[linestyle=none,fillstyle=solid,fillcolor=curcolor,opacity=0.11935484]
{
\newpath
\moveto(332.67380142,170)
\lineto(418.2635994,170)
\curveto(424.08261341,170)(428.7672348,165.80657851)(428.7672348,160.59770966)
\lineto(428.7672348,149.50483704)
\curveto(428.7672348,144.29596819)(424.08261341,140.10254669)(418.2635994,140.10254669)
\lineto(332.67380142,140.10254669)
\curveto(326.85478741,140.10254669)(322.17016602,144.29596819)(322.17016602,149.50483704)
\lineto(322.17016602,160.59770966)
\curveto(322.17016602,165.80657851)(326.85478741,170)(332.67380142,170)
\closepath
}
}
{
\newrgbcolor{curcolor}{0 0 0}
\pscustom[linewidth=2,linecolor=curcolor]
{
\newpath
\moveto(332.67380142,170)
\lineto(418.2635994,170)
\curveto(424.08261341,170)(428.7672348,165.80657851)(428.7672348,160.59770966)
\lineto(428.7672348,149.50483704)
\curveto(428.7672348,144.29596819)(424.08261341,140.10254669)(418.2635994,140.10254669)
\lineto(332.67380142,140.10254669)
\curveto(326.85478741,140.10254669)(322.17016602,144.29596819)(322.17016602,149.50483704)
\lineto(322.17016602,160.59770966)
\curveto(322.17016602,165.80657851)(326.85478741,170)(332.67380142,170)
\closepath
}
}
{
\newrgbcolor{curcolor}{0 0 0}
\pscustom[linestyle=none,fillstyle=solid,fillcolor=curcolor,opacity=0.11935484]
{
\newpath
\moveto(311.47112656,119.96066284)
\lineto(416.01738358,119.96066284)
\curveto(423.12519402,119.96066284)(428.84736633,115.79414402)(428.84736633,110.6186924)
\lineto(428.84736633,99.59698677)
\curveto(428.84736633,94.42153514)(423.12519402,90.25501633)(416.01738358,90.25501633)
\lineto(311.47112656,90.25501633)
\curveto(304.36331611,90.25501633)(298.6411438,94.42153514)(298.6411438,99.59698677)
\lineto(298.6411438,110.6186924)
\curveto(298.6411438,115.79414402)(304.36331611,119.96066284)(311.47112656,119.96066284)
\closepath
}
}
{
\newrgbcolor{curcolor}{0 0 0}
\pscustom[linewidth=2,linecolor=curcolor]
{
\newpath
\moveto(311.47112656,119.96066284)
\lineto(416.01738358,119.96066284)
\curveto(423.12519402,119.96066284)(428.84736633,115.79414402)(428.84736633,110.6186924)
\lineto(428.84736633,99.59698677)
\curveto(428.84736633,94.42153514)(423.12519402,90.25501633)(416.01738358,90.25501633)
\lineto(311.47112656,90.25501633)
\curveto(304.36331611,90.25501633)(298.6411438,94.42153514)(298.6411438,99.59698677)
\lineto(298.6411438,110.6186924)
\curveto(298.6411438,115.79414402)(304.36331611,119.96066284)(311.47112656,119.96066284)
\closepath
}
}
{
\newrgbcolor{curcolor}{1 1 1}
\pscustom[linestyle=none,fillstyle=solid,fillcolor=curcolor]
{
\newpath
\moveto(329.26973534,49.81427002)
\lineto(420.7846241,49.81427002)
\curveto(425.93425914,49.81427002)(430.0799942,46.91140156)(430.0799942,43.30559635)
\lineto(430.0799942,26.42860699)
\curveto(430.0799942,22.82280177)(425.93425914,19.91993332)(420.7846241,19.91993332)
\lineto(329.26973534,19.91993332)
\curveto(324.1201003,19.91993332)(319.97436523,22.82280177)(319.97436523,26.42860699)
\lineto(319.97436523,43.30559635)
\curveto(319.97436523,46.91140156)(324.1201003,49.81427002)(329.26973534,49.81427002)
\closepath
}
}
{
\newrgbcolor{curcolor}{0 0 0}
\pscustom[linewidth=2,linecolor=curcolor]
{
\newpath
\moveto(329.26973534,49.81427002)
\lineto(420.7846241,49.81427002)
\curveto(425.93425914,49.81427002)(430.0799942,46.91140156)(430.0799942,43.30559635)
\lineto(430.0799942,26.42860699)
\curveto(430.0799942,22.82280177)(425.93425914,19.91993332)(420.7846241,19.91993332)
\lineto(329.26973534,19.91993332)
\curveto(324.1201003,19.91993332)(319.97436523,22.82280177)(319.97436523,26.42860699)
\lineto(319.97436523,43.30559635)
\curveto(319.97436523,46.91140156)(324.1201003,49.81427002)(329.26973534,49.81427002)
\closepath
}
}
{
\newrgbcolor{curcolor}{0 0 0}
\pscustom[linestyle=none,fillstyle=solid,fillcolor=curcolor]
{
\newpath
\moveto(332.35546875,47.49609375)
\lineto(334.72265625,47.49609375)
\lineto(334.72265625,31.9921875)
\lineto(343.2421875,31.9921875)
\lineto(343.2421875,30)
\lineto(332.35546875,30)
\lineto(332.35546875,47.49609375)
}
}
{
\newrgbcolor{curcolor}{0 0 0}
\pscustom[linestyle=none,fillstyle=solid,fillcolor=curcolor]
{
\newpath
\moveto(351.5859375,36.59765625)
\curveto(349.84374352,36.59764965)(348.63671347,36.3984311)(347.96484375,36)
\curveto(347.29296482,35.6015569)(346.95702765,34.92187008)(346.95703125,33.9609375)
\curveto(346.95702765,33.1953093)(347.2070274,32.58593491)(347.70703125,32.1328125)
\curveto(348.21483889,31.68749831)(348.90233821,31.46484229)(349.76953125,31.46484375)
\curveto(350.96483614,31.46484229)(351.92186644,31.88671686)(352.640625,32.73046875)
\curveto(353.36717749,33.58202767)(353.73045838,34.71093279)(353.73046875,36.1171875)
\lineto(353.73046875,36.59765625)
\lineto(351.5859375,36.59765625)
\moveto(355.88671875,37.48828125)
\lineto(355.88671875,30)
\lineto(353.73046875,30)
\lineto(353.73046875,31.9921875)
\curveto(353.23827137,31.1953113)(352.62499073,30.60546814)(351.890625,30.22265625)
\curveto(351.1562422,29.8476564)(350.2578056,29.66015659)(349.1953125,29.66015625)
\curveto(347.85155801,29.66015659)(346.78124658,30.03515621)(345.984375,30.78515625)
\curveto(345.19531066,31.54296721)(344.80077981,32.55468495)(344.80078125,33.8203125)
\curveto(344.80077981,35.2968697)(345.29296682,36.41014984)(346.27734375,37.16015625)
\curveto(347.26952734,37.91014834)(348.74608836,38.28514796)(350.70703125,38.28515625)
\lineto(353.73046875,38.28515625)
\lineto(353.73046875,38.49609375)
\curveto(353.73045838,39.48827176)(353.40233371,40.253896)(352.74609375,40.79296875)
\curveto(352.09764751,41.33983241)(351.18358593,41.61326964)(350.00390625,41.61328125)
\curveto(349.25390036,41.61326964)(348.52343234,41.52342598)(347.8125,41.34375)
\curveto(347.10155876,41.16405134)(346.41796569,40.89452036)(345.76171875,40.53515625)
\lineto(345.76171875,42.52734375)
\curveto(346.55077806,42.83201842)(347.31640229,43.05858069)(348.05859375,43.20703125)
\curveto(348.80077581,43.36326789)(349.52343134,43.44139281)(350.2265625,43.44140625)
\curveto(352.12499123,43.44139281)(353.54295857,42.9492058)(354.48046875,41.96484375)
\curveto(355.41795669,40.98045777)(355.88670622,39.48827176)(355.88671875,37.48828125)
}
}
{
\newrgbcolor{curcolor}{0 0 0}
\pscustom[linestyle=none,fillstyle=solid,fillcolor=curcolor]
{
\newpath
\moveto(371.25,37.921875)
\lineto(371.25,30)
\lineto(369.09375,30)
\lineto(369.09375,37.8515625)
\curveto(369.09373898,39.09374091)(368.85155173,40.02342748)(368.3671875,40.640625)
\curveto(367.8828027,41.25780124)(367.15624092,41.56639468)(366.1875,41.56640625)
\curveto(365.02343055,41.56639468)(364.10546272,41.1953013)(363.43359375,40.453125)
\curveto(362.76171407,39.71092779)(362.4257769,38.69921005)(362.42578125,37.41796875)
\lineto(362.42578125,30)
\lineto(360.2578125,30)
\lineto(360.2578125,43.125)
\lineto(362.42578125,43.125)
\lineto(362.42578125,41.0859375)
\curveto(362.94140139,41.87498812)(363.54686953,42.46483129)(364.2421875,42.85546875)
\curveto(364.94530563,43.2460805)(365.75389857,43.44139281)(366.66796875,43.44140625)
\curveto(368.17577115,43.44139281)(369.31639501,42.97264328)(370.08984375,42.03515625)
\curveto(370.86326846,41.10545764)(371.24998683,39.73436527)(371.25,37.921875)
}
}
{
\newrgbcolor{curcolor}{0 0 0}
\pscustom[linestyle=none,fillstyle=solid,fillcolor=curcolor]
{
\newpath
\moveto(385.01953125,42.62109375)
\lineto(385.01953125,40.60546875)
\curveto(384.41014515,40.94139531)(383.79686452,41.19139506)(383.1796875,41.35546875)
\curveto(382.57030324,41.52733222)(381.95311636,41.61326964)(381.328125,41.61328125)
\curveto(379.92968088,41.61326964)(378.84374447,41.16795758)(378.0703125,40.27734375)
\curveto(377.29687102,39.39452186)(376.91015265,38.1523356)(376.91015625,36.55078125)
\curveto(376.91015265,34.9492138)(377.29687102,33.7031213)(378.0703125,32.8125)
\curveto(378.84374447,31.92968557)(379.92968088,31.48827976)(381.328125,31.48828125)
\curveto(381.95311636,31.48827976)(382.57030324,31.57031093)(383.1796875,31.734375)
\curveto(383.79686452,31.90624809)(384.41014515,32.16015409)(385.01953125,32.49609375)
\lineto(385.01953125,30.50390625)
\curveto(384.41795764,30.22265603)(383.79295827,30.01171874)(383.14453125,29.87109375)
\curveto(382.50389706,29.73046902)(381.82030399,29.66015659)(381.09375,29.66015625)
\curveto(379.1171817,29.66015659)(377.54687077,30.28124972)(376.3828125,31.5234375)
\curveto(375.21874809,32.76562223)(374.63671743,34.44140181)(374.63671875,36.55078125)
\curveto(374.63671743,38.69139756)(375.22265434,40.37498963)(376.39453125,41.6015625)
\curveto(377.57421449,42.82811217)(379.18749413,43.44139281)(381.234375,43.44140625)
\curveto(381.89842891,43.44139281)(382.54686577,43.37108038)(383.1796875,43.23046875)
\curveto(383.8124895,43.09764315)(384.42577014,42.89451836)(385.01953125,42.62109375)
}
}
{
\newrgbcolor{curcolor}{0 0 0}
\pscustom[linestyle=none,fillstyle=solid,fillcolor=curcolor]
{
\newpath
\moveto(400.01953125,37.1015625)
\lineto(400.01953125,36.046875)
\lineto(390.10546875,36.046875)
\curveto(390.19921508,34.56249544)(390.64452714,33.42968407)(391.44140625,32.6484375)
\curveto(392.24608804,31.87499813)(393.36327442,31.48827976)(394.79296875,31.48828125)
\curveto(395.62108466,31.48827976)(396.42186511,31.58984216)(397.1953125,31.79296875)
\curveto(397.97655105,31.99609175)(398.74998778,32.30077895)(399.515625,32.70703125)
\lineto(399.515625,30.66796875)
\curveto(398.74217529,30.33984341)(397.94920733,30.08984366)(397.13671875,29.91796875)
\curveto(396.32420896,29.746094)(395.49999103,29.66015659)(394.6640625,29.66015625)
\curveto(392.57030646,29.66015659)(390.91015187,30.26953098)(389.68359375,31.48828125)
\curveto(388.46484182,32.70702854)(387.85546743,34.35546439)(387.85546875,36.43359375)
\curveto(387.85546743,38.58202267)(388.43359185,40.28514596)(389.58984375,41.54296875)
\curveto(390.75390203,42.80858094)(392.32030671,43.44139281)(394.2890625,43.44140625)
\curveto(396.05467798,43.44139281)(397.44920783,42.87108088)(398.47265625,41.73046875)
\curveto(399.50389328,40.59764565)(400.01951776,39.05467845)(400.01953125,37.1015625)
\moveto(397.86328125,37.734375)
\curveto(397.84764493,38.91405359)(397.51561402,39.85545889)(396.8671875,40.55859375)
\curveto(396.2265528,41.26170749)(395.37499116,41.61326964)(394.3125,41.61328125)
\curveto(393.10936842,41.61326964)(392.14452564,41.27342623)(391.41796875,40.59375)
\curveto(390.69921458,39.91405259)(390.2851525,38.95702229)(390.17578125,37.72265625)
\lineto(397.86328125,37.734375)
}
}
{
\newrgbcolor{curcolor}{0 0 0}
\pscustom[linestyle=none,fillstyle=solid,fillcolor=curcolor]
{
\newpath
\moveto(411.1640625,41.109375)
\curveto(410.92186537,41.24998875)(410.65624064,41.35155115)(410.3671875,41.4140625)
\curveto(410.08592871,41.48436352)(409.77342902,41.51951973)(409.4296875,41.51953125)
\curveto(408.21093059,41.51951973)(407.27343152,41.12108263)(406.6171875,40.32421875)
\curveto(405.96874533,39.53514671)(405.6445269,38.3984291)(405.64453125,36.9140625)
\lineto(405.64453125,30)
\lineto(403.4765625,30)
\lineto(403.4765625,43.125)
\lineto(405.64453125,43.125)
\lineto(405.64453125,41.0859375)
\curveto(406.09765145,41.88280062)(406.68749461,42.47264378)(407.4140625,42.85546875)
\curveto(408.14061816,43.2460805)(409.02342977,43.44139281)(410.0625,43.44140625)
\curveto(410.21092859,43.44139281)(410.37499092,43.42967407)(410.5546875,43.40625)
\curveto(410.73436556,43.39061161)(410.93358411,43.36326789)(411.15234375,43.32421875)
\lineto(411.1640625,41.109375)
}
}
{
\newrgbcolor{curcolor}{1 1 1}
\pscustom[linestyle=none,fillstyle=solid,fillcolor=curcolor]
{
\newpath
\moveto(139.20729923,49.62908936)
\lineto(230.67112637,49.62908936)
\curveto(235.81788884,49.62908936)(239.96131134,46.74393137)(239.96131134,43.16012526)
\lineto(239.96131134,26.38610125)
\curveto(239.96131134,22.80229513)(235.81788884,19.91713715)(230.67112637,19.91713715)
\lineto(139.20729923,19.91713715)
\curveto(134.06053676,19.91713715)(129.91711426,22.80229513)(129.91711426,26.38610125)
\lineto(129.91711426,43.16012526)
\curveto(129.91711426,46.74393137)(134.06053676,49.62908936)(139.20729923,49.62908936)
\closepath
}
}
{
\newrgbcolor{curcolor}{0 0 0}
\pscustom[linewidth=1.72500002,linecolor=curcolor]
{
\newpath
\moveto(139.20729923,49.62908936)
\lineto(230.67112637,49.62908936)
\curveto(235.81788884,49.62908936)(239.96131134,46.74393137)(239.96131134,43.16012526)
\lineto(239.96131134,26.38610125)
\curveto(239.96131134,22.80229513)(235.81788884,19.91713715)(230.67112637,19.91713715)
\lineto(139.20729923,19.91713715)
\curveto(134.06053676,19.91713715)(129.91711426,22.80229513)(129.91711426,26.38610125)
\lineto(129.91711426,43.16012526)
\curveto(129.91711426,46.74393137)(134.06053676,49.62908936)(139.20729923,49.62908936)
\closepath
}
}
{
\newrgbcolor{curcolor}{0 0 0}
\pscustom[linestyle=none,fillstyle=solid,fillcolor=curcolor]
{
\newpath
\moveto(160.65234375,38.203125)
\curveto(161.16014509,38.03124197)(161.6523321,37.66405484)(162.12890625,37.1015625)
\curveto(162.61326864,36.53905596)(163.09764315,35.76561923)(163.58203125,34.78125)
\lineto(165.984375,30)
\lineto(163.44140625,30)
\lineto(161.203125,34.48828125)
\curveto(160.62498938,35.66015059)(160.06248994,36.43749356)(159.515625,36.8203125)
\curveto(158.97655352,37.2031178)(158.23827301,37.39452386)(157.30078125,37.39453125)
\lineto(154.72265625,37.39453125)
\lineto(154.72265625,30)
\lineto(152.35546875,30)
\lineto(152.35546875,47.49609375)
\lineto(157.69921875,47.49609375)
\curveto(159.69920905,47.49607625)(161.19139506,47.07810792)(162.17578125,46.2421875)
\curveto(163.16014309,45.40623459)(163.6523301,44.14451711)(163.65234375,42.45703125)
\curveto(163.6523301,41.35545739)(163.39451786,40.44139581)(162.87890625,39.71484375)
\curveto(162.37108138,38.98827226)(161.62889462,38.48436652)(160.65234375,38.203125)
\moveto(154.72265625,45.55078125)
\lineto(154.72265625,39.33984375)
\lineto(157.69921875,39.33984375)
\curveto(158.83983491,39.33983441)(159.69920905,39.6015529)(160.27734375,40.125)
\curveto(160.86327039,40.65623934)(161.15623884,41.43358232)(161.15625,42.45703125)
\curveto(161.15623884,43.48045527)(160.86327039,44.24998575)(160.27734375,44.765625)
\curveto(159.69920905,45.28904721)(158.83983491,45.5507657)(157.69921875,45.55078125)
\lineto(154.72265625,45.55078125)
}
}
{
\newrgbcolor{curcolor}{0 0 0}
\pscustom[linestyle=none,fillstyle=solid,fillcolor=curcolor]
{
\newpath
\moveto(179.09765625,37.1015625)
\lineto(179.09765625,36.046875)
\lineto(169.18359375,36.046875)
\curveto(169.27734008,34.56249544)(169.72265214,33.42968407)(170.51953125,32.6484375)
\curveto(171.32421304,31.87499813)(172.44139942,31.48827976)(173.87109375,31.48828125)
\curveto(174.69920966,31.48827976)(175.49999011,31.58984216)(176.2734375,31.79296875)
\curveto(177.05467605,31.99609175)(177.82811278,32.30077895)(178.59375,32.70703125)
\lineto(178.59375,30.66796875)
\curveto(177.82030029,30.33984341)(177.02733233,30.08984366)(176.21484375,29.91796875)
\curveto(175.40233396,29.746094)(174.57811603,29.66015659)(173.7421875,29.66015625)
\curveto(171.64843146,29.66015659)(169.98827687,30.26953098)(168.76171875,31.48828125)
\curveto(167.54296682,32.70702854)(166.93359243,34.35546439)(166.93359375,36.43359375)
\curveto(166.93359243,38.58202267)(167.51171685,40.28514596)(168.66796875,41.54296875)
\curveto(169.83202703,42.80858094)(171.39843171,43.44139281)(173.3671875,43.44140625)
\curveto(175.13280298,43.44139281)(176.52733283,42.87108088)(177.55078125,41.73046875)
\curveto(178.58201828,40.59764565)(179.09764276,39.05467845)(179.09765625,37.1015625)
\moveto(176.94140625,37.734375)
\curveto(176.92576993,38.91405359)(176.59373902,39.85545889)(175.9453125,40.55859375)
\curveto(175.3046778,41.26170749)(174.45311616,41.61326964)(173.390625,41.61328125)
\curveto(172.18749342,41.61326964)(171.22265064,41.27342623)(170.49609375,40.59375)
\curveto(169.77733958,39.91405259)(169.3632775,38.95702229)(169.25390625,37.72265625)
\lineto(176.94140625,37.734375)
}
}
{
\newrgbcolor{curcolor}{0 0 0}
\pscustom[linestyle=none,fillstyle=solid,fillcolor=curcolor]
{
\newpath
\moveto(184.76953125,46.8515625)
\lineto(184.76953125,43.125)
\lineto(189.2109375,43.125)
\lineto(189.2109375,41.44921875)
\lineto(184.76953125,41.44921875)
\lineto(184.76953125,34.32421875)
\curveto(184.76952686,33.253903)(184.91405796,32.56640368)(185.203125,32.26171875)
\curveto(185.49999487,31.95702929)(186.09765053,31.8046857)(186.99609375,31.8046875)
\lineto(189.2109375,31.8046875)
\lineto(189.2109375,30)
\lineto(186.99609375,30)
\curveto(185.33202629,30)(184.18358994,30.30859344)(183.55078125,30.92578125)
\curveto(182.91796621,31.5507797)(182.60156027,32.68359107)(182.6015625,34.32421875)
\lineto(182.6015625,41.44921875)
\lineto(181.01953125,41.44921875)
\lineto(181.01953125,43.125)
\lineto(182.6015625,43.125)
\lineto(182.6015625,46.8515625)
\lineto(184.76953125,46.8515625)
}
}
{
\newrgbcolor{curcolor}{0 0 0}
\pscustom[linestyle=none,fillstyle=solid,fillcolor=curcolor]
{
\newpath
\moveto(197.14453125,41.61328125)
\curveto(195.98827506,41.61326964)(195.07421347,41.16014509)(194.40234375,40.25390625)
\curveto(193.73046482,39.35545939)(193.39452765,38.12108563)(193.39453125,36.55078125)
\curveto(193.39452765,34.98046377)(193.72655857,33.74218376)(194.390625,32.8359375)
\curveto(195.06249473,31.93749806)(195.98046257,31.48827976)(197.14453125,31.48828125)
\curveto(198.29296025,31.48827976)(199.20311559,31.94140431)(199.875,32.84765625)
\curveto(200.54686425,33.7539025)(200.88280141,34.98827626)(200.8828125,36.55078125)
\curveto(200.88280141,38.10546064)(200.54686425,39.33592816)(199.875,40.2421875)
\curveto(199.20311559,41.15623884)(198.29296025,41.61326964)(197.14453125,41.61328125)
\moveto(197.14453125,43.44140625)
\curveto(199.01952203,43.44139281)(200.4921768,42.83201842)(201.5625,41.61328125)
\curveto(202.63279966,40.39452086)(203.16795538,38.70702254)(203.16796875,36.55078125)
\curveto(203.16795538,34.40233935)(202.63279966,32.71484104)(201.5625,31.48828125)
\curveto(200.4921768,30.26953098)(199.01952203,29.66015659)(197.14453125,29.66015625)
\curveto(195.26171329,29.66015659)(193.78515226,30.26953098)(192.71484375,31.48828125)
\curveto(191.65234189,32.71484104)(191.12109243,34.40233935)(191.12109375,36.55078125)
\curveto(191.12109243,38.70702254)(191.65234189,40.39452086)(192.71484375,41.61328125)
\curveto(193.78515226,42.83201842)(195.26171329,43.44139281)(197.14453125,43.44140625)
}
}
{
\newrgbcolor{curcolor}{0 0 0}
\pscustom[linestyle=none,fillstyle=solid,fillcolor=curcolor]
{
\newpath
\moveto(206.5078125,35.1796875)
\lineto(206.5078125,43.125)
\lineto(208.6640625,43.125)
\lineto(208.6640625,35.26171875)
\curveto(208.6640583,34.01952723)(208.90624556,33.08593441)(209.390625,32.4609375)
\curveto(209.87499459,31.84374816)(210.60155637,31.53515471)(211.5703125,31.53515625)
\curveto(212.73436673,31.53515471)(213.65233457,31.90624809)(214.32421875,32.6484375)
\curveto(215.00389571,33.39062161)(215.34373913,34.40233935)(215.34375,35.68359375)
\lineto(215.34375,43.125)
\lineto(217.5,43.125)
\lineto(217.5,30)
\lineto(215.34375,30)
\lineto(215.34375,32.015625)
\curveto(214.82030215,31.21874878)(214.21092776,30.62499938)(213.515625,30.234375)
\curveto(212.82811664,29.85156265)(212.02733619,29.66015659)(211.11328125,29.66015625)
\curveto(209.60546361,29.66015659)(208.46093351,30.12890612)(207.6796875,31.06640625)
\curveto(206.89843507,32.00390425)(206.50781046,33.37499662)(206.5078125,35.1796875)
\moveto(211.93359375,43.44140625)
\lineto(211.93359375,43.44140625)
}
}
{
\newrgbcolor{curcolor}{0 0 0}
\pscustom[linestyle=none,fillstyle=solid,fillcolor=curcolor]
{
\newpath
\moveto(229.5703125,41.109375)
\curveto(229.32811538,41.24998875)(229.06249064,41.35155115)(228.7734375,41.4140625)
\curveto(228.49217871,41.48436352)(228.17967902,41.51951973)(227.8359375,41.51953125)
\curveto(226.61718059,41.51951973)(225.67968152,41.12108263)(225.0234375,40.32421875)
\curveto(224.37499533,39.53514671)(224.0507769,38.3984291)(224.05078125,36.9140625)
\lineto(224.05078125,30)
\lineto(221.8828125,30)
\lineto(221.8828125,43.125)
\lineto(224.05078125,43.125)
\lineto(224.05078125,41.0859375)
\curveto(224.50390145,41.88280062)(225.09374461,42.47264378)(225.8203125,42.85546875)
\curveto(226.54686816,43.2460805)(227.42967977,43.44139281)(228.46875,43.44140625)
\curveto(228.61717859,43.44139281)(228.78124092,43.42967407)(228.9609375,43.40625)
\curveto(229.14061556,43.39061161)(229.33983411,43.36326789)(229.55859375,43.32421875)
\lineto(229.5703125,41.109375)
}
}
{
\newrgbcolor{curcolor}{0 0 0}
\pscustom[linewidth=2,linecolor=curcolor,linestyle=dashed,dash=8 8]
{
\newpath
\moveto(140,40)
\lineto(70.11406,39.73606)
}
}
{
\newrgbcolor{curcolor}{0 0 0}
\pscustom[linestyle=none,fillstyle=solid,fillcolor=curcolor]
{
\newpath
\moveto(129.51948821,44.80093475)
\lineto(142.64865594,40.0292193)
\lineto(129.55590443,35.15847191)
\curveto(131.6397373,38.01283536)(131.61296003,41.9072312)(129.51948821,44.80093475)
\lineto(129.51948821,44.80093475)
\closepath
}
}
{
\newrgbcolor{curcolor}{0 0 0}
\pscustom[linewidth=2.06612587,linecolor=curcolor,linestyle=dashed,dash=8.26450373 8.26450373]
{
\newpath
\moveto(218.76726,260)
\lineto(60,260)
}
}
{
\newrgbcolor{curcolor}{0 0 0}
\pscustom[linestyle=none,fillstyle=solid,fillcolor=curcolor]
{
\newpath
\moveto(207.95904012,265.00052279)
\lineto(221.50358276,260.01985125)
\lineto(207.95903938,255.03918079)
\curveto(210.12289102,257.97976663)(210.11042285,262.00299841)(207.95904012,265.00052279)
\lineto(207.95904012,265.00052279)
\closepath
}
}
{
\newrgbcolor{curcolor}{0 0 0}
\pscustom[linewidth=2.21782422,linecolor=curcolor,linestyle=dashed,dash=8.87129678 8.87129678]
{
\newpath
\moveto(62.086063,350)
\lineto(158.76726,350)
}
}
{
\newrgbcolor{curcolor}{0 0 0}
\pscustom[linestyle=none,fillstyle=solid,fillcolor=curcolor]
{
\newpath
\moveto(147.16548288,355.36766938)
\lineto(161.70448807,350.02130875)
\lineto(147.16548209,344.67494931)
\curveto(149.48820726,347.83143778)(149.47482367,352.15006184)(147.16548288,355.36766938)
\closepath
}
}
{
\newrgbcolor{curcolor}{0 0 0}
\pscustom[linewidth=2,linecolor=curcolor,linestyle=dashed,dash=8 8]
{
\newpath
\moveto(498.76725,350)
\lineto(398.47533,350)
}
}
{
\newrgbcolor{curcolor}{0 0 0}
\pscustom[linestyle=none,fillstyle=solid,fillcolor=curcolor]
{
\newpath
\moveto(408.93763536,345.15951776)
\lineto(395.8265826,349.98078409)
\lineto(408.93763608,354.80204936)
\curveto(406.843038,351.95557628)(406.85510712,348.06110708)(408.93763536,345.15951776)
\closepath
}
}
{
\newrgbcolor{curcolor}{0 0 0}
\pscustom[linewidth=2,linecolor=curcolor,linestyle=dashed,dash=8 8]
{
\newpath
\moveto(498.76725,200)
\lineto(398.62788,200)
}
}
{
\newrgbcolor{curcolor}{0 0 0}
\pscustom[linestyle=none,fillstyle=solid,fillcolor=curcolor]
{
\newpath
\moveto(409.09018536,195.15951776)
\lineto(395.9791326,199.98078409)
\lineto(409.09018608,204.80204936)
\curveto(406.995588,201.95557628)(407.00765712,198.06110708)(409.09018536,195.15951776)
\closepath
}
}
{
\newrgbcolor{curcolor}{0 0 0}
\pscustom[linewidth=2.12666917,linecolor=curcolor,linestyle=dashed,dash=8.50667644 8.50667644]
{
\newpath
\moveto(499.2423,149.83946)
\lineto(412.44065,149.83946)
}
}
{
\newrgbcolor{curcolor}{0 0 0}
\pscustom[linestyle=none,fillstyle=solid,fillcolor=curcolor]
{
\newpath
\moveto(423.56558112,144.69240783)
\lineto(409.62414528,149.81902706)
\lineto(423.56558189,154.94564516)
\curveto(421.33832331,151.91889189)(421.35115682,147.7777681)(423.56558112,144.69240783)
\closepath
}
}
{
\newrgbcolor{curcolor}{0 0 0}
\pscustom[linewidth=2.10198331,linecolor=curcolor,linestyle=dashed,dash=8.40793346 8.40793346]
{
\newpath
\moveto(498.39405,101.30557)
\lineto(398.76725,100)
}
}
{
\newrgbcolor{curcolor}{0 0 0}
\pscustom[linestyle=none,fillstyle=solid,fillcolor=curcolor]
{
\newpath
\moveto(409.82876301,95.05721353)
\lineto(395.98394222,99.94332836)
\lineto(409.6959702,105.19056371)
\curveto(407.53395473,102.17035504)(407.60027145,98.07781803)(409.82876301,95.05721353)
\closepath
}
}
{
\newrgbcolor{curcolor}{0 0 0}
\pscustom[linewidth=2,linecolor=curcolor,linestyle=dashed,dash=8 8]
{
\newpath
\moveto(500,40)
\lineto(411.0551,41.27788)
}
}
{
\newrgbcolor{curcolor}{0 0 0}
\pscustom[linestyle=none,fillstyle=solid,fillcolor=curcolor]
{
\newpath
\moveto(421.44678925,36.28759981)
\lineto(408.40634988,41.29671696)
\lineto(421.58531086,45.92913638)
\curveto(419.45003758,43.11304722)(419.40615902,39.21880651)(421.44678925,36.28759981)
\closepath
}
}
{
\newrgbcolor{curcolor}{0 0 0}
\pscustom[linestyle=none,fillstyle=solid,fillcolor=curcolor]
{
\newpath
\moveto(52.984375,452.578125)
\curveto(54.49477717,452.25519608)(55.67185933,451.58332175)(56.515625,450.5625)
\curveto(57.3697743,449.54165713)(57.7968572,448.28124172)(57.796875,446.78125)
\curveto(57.7968572,444.47916219)(57.00519133,442.69791397)(55.421875,441.4375)
\curveto(53.83852783,440.17708316)(51.58853008,439.54687545)(48.671875,439.546875)
\curveto(47.69270064,439.54687545)(46.68228498,439.64583369)(45.640625,439.84375)
\curveto(44.60937039,440.03124997)(43.54166312,440.31770802)(42.4375,440.703125)
\lineto(42.4375,443.75)
\curveto(43.31249669,443.23958009)(44.27082906,442.85416381)(45.3125,442.59375)
\curveto(46.35416031,442.333331)(47.44270089,442.2031228)(48.578125,442.203125)
\curveto(50.55728111,442.2031228)(52.06248794,442.59374741)(53.09375,443.375)
\curveto(54.13540253,444.15624584)(54.65623534,445.29166137)(54.65625,446.78125)
\curveto(54.65623534,448.15624184)(54.17186083,449.22915744)(53.203125,450)
\curveto(52.24477942,450.78123922)(50.90623909,451.17186383)(49.1875,451.171875)
\lineto(46.46875,451.171875)
\lineto(46.46875,453.765625)
\lineto(49.3125,453.765625)
\curveto(50.86457247,453.76561123)(52.05207128,454.07290259)(52.875,454.6875)
\curveto(53.69790297,455.31248469)(54.10936089,456.20831713)(54.109375,457.375)
\curveto(54.10936089,458.57289809)(53.68227798,459.48956384)(52.828125,460.125)
\curveto(51.98436302,460.77081256)(50.77082256,461.09372891)(49.1875,461.09375)
\curveto(48.32290834,461.09372891)(47.39582594,460.999979)(46.40625,460.8125)
\curveto(45.41666125,460.62497938)(44.32812067,460.333313)(43.140625,459.9375)
\lineto(43.140625,462.75)
\curveto(44.33853733,463.08331025)(45.45832788,463.33331)(46.5,463.5)
\curveto(47.55207578,463.666643)(48.54165812,463.74997625)(49.46875,463.75)
\curveto(51.86457147,463.74997625)(53.76040291,463.2031018)(55.15625,462.109375)
\curveto(56.55206678,461.02602064)(57.24998275,459.55727211)(57.25,457.703125)
\curveto(57.24998275,456.41144192)(56.88019145,455.31769302)(56.140625,454.421875)
\curveto(55.40102627,453.5364448)(54.34894398,452.92186208)(52.984375,452.578125)
}
}
{
\newrgbcolor{curcolor}{0 0 0}
\pscustom[linestyle=none,fillstyle=solid,fillcolor=curcolor]
{
\newpath
\moveto(512.09375,420.578125)
\lineto(504.125,408.125)
\lineto(512.09375,408.125)
\lineto(512.09375,420.578125)
\moveto(511.265625,423.328125)
\lineto(515.234375,423.328125)
\lineto(515.234375,408.125)
\lineto(518.5625,408.125)
\lineto(518.5625,405.5)
\lineto(515.234375,405.5)
\lineto(515.234375,400)
\lineto(512.09375,400)
\lineto(512.09375,405.5)
\lineto(501.5625,405.5)
\lineto(501.5625,408.546875)
\lineto(511.265625,423.328125)
}
}
{
\newrgbcolor{curcolor}{0 0 0}
\pscustom[linestyle=none,fillstyle=solid,fillcolor=curcolor]
{
\newpath
\moveto(43.453125,363.328125)
\lineto(55.84375,363.328125)
\lineto(55.84375,360.671875)
\lineto(46.34375,360.671875)
\lineto(46.34375,354.953125)
\curveto(46.80207653,355.10935989)(47.26040941,355.22394311)(47.71875,355.296875)
\curveto(48.17707516,355.38019295)(48.63540803,355.42185958)(49.09375,355.421875)
\curveto(51.69790497,355.42185958)(53.76040291,354.70831863)(55.28125,353.28125)
\curveto(56.80206653,351.85415481)(57.56248244,349.92186508)(57.5625,347.484375)
\curveto(57.56248244,344.97395336)(56.78123322,343.02083031)(55.21875,341.625)
\curveto(53.65623634,340.23958309)(51.45311355,339.54687545)(48.609375,339.546875)
\curveto(47.6302007,339.54687545)(46.6302017,339.6302087)(45.609375,339.796875)
\curveto(44.59895373,339.9635417)(43.55207978,340.21354145)(42.46875,340.546875)
\lineto(42.46875,343.71875)
\curveto(43.40624659,343.20833012)(44.37499562,342.82812217)(45.375,342.578125)
\curveto(46.37499363,342.32812267)(47.43228423,342.2031228)(48.546875,342.203125)
\curveto(50.34894798,342.2031228)(51.77602989,342.67708066)(52.828125,343.625)
\curveto(53.88019445,344.57291209)(54.40623559,345.85936914)(54.40625,347.484375)
\curveto(54.40623559,349.10936589)(53.88019445,350.39582294)(52.828125,351.34375)
\curveto(51.77602989,352.29165438)(50.34894798,352.76561223)(48.546875,352.765625)
\curveto(47.7031173,352.76561223)(46.85936814,352.67186233)(46.015625,352.484375)
\curveto(45.18228648,352.2968627)(44.32812067,352.00519633)(43.453125,351.609375)
\lineto(43.453125,363.328125)
}
}
{
\newrgbcolor{curcolor}{0 0 0}
\pscustom[linestyle=none,fillstyle=solid,fillcolor=curcolor]
{
\newpath
\moveto(510.5625,352.921875)
\curveto(509.14582419,352.92186208)(508.02082531,352.43748756)(507.1875,351.46875)
\curveto(506.36457697,350.4999895)(505.95311905,349.17186583)(505.953125,347.484375)
\curveto(505.95311905,345.80728586)(506.36457697,344.47916219)(507.1875,343.5)
\curveto(508.02082531,342.53124747)(509.14582419,342.04687295)(510.5625,342.046875)
\curveto(511.97915469,342.04687295)(513.09894523,342.53124747)(513.921875,343.5)
\curveto(514.75519358,344.47916219)(515.17185983,345.80728586)(515.171875,347.484375)
\curveto(515.17185983,349.17186583)(514.75519358,350.4999895)(513.921875,351.46875)
\curveto(513.09894523,352.43748756)(511.97915469,352.92186208)(510.5625,352.921875)
\moveto(516.828125,362.8125)
\lineto(516.828125,359.9375)
\curveto(516.0364423,360.31247969)(515.23435977,360.59893773)(514.421875,360.796875)
\curveto(513.61977805,360.99477067)(512.82290384,361.09372891)(512.03125,361.09375)
\curveto(509.94790672,361.09372891)(508.35415831,360.39060461)(507.25,358.984375)
\curveto(506.15624384,357.57810742)(505.53124447,355.45310955)(505.375,352.609375)
\curveto(505.98957734,353.51561148)(506.76040991,354.20831913)(507.6875,354.6875)
\curveto(508.61457472,355.17706816)(509.63540703,355.42185958)(510.75,355.421875)
\curveto(513.09373691,355.42185958)(514.94269339,354.70831863)(516.296875,353.28125)
\curveto(517.66144067,351.86457147)(518.34373166,349.93228173)(518.34375,347.484375)
\curveto(518.34373166,345.08853658)(517.63539903,343.1666635)(516.21875,341.71875)
\curveto(514.80206853,340.27083306)(512.91665375,339.54687545)(510.5625,339.546875)
\curveto(507.86457547,339.54687545)(505.80207753,340.57812442)(504.375,342.640625)
\curveto(502.94791372,344.71353695)(502.23437277,347.71353395)(502.234375,351.640625)
\curveto(502.23437277,355.32810967)(503.10937189,358.26560673)(504.859375,360.453125)
\curveto(506.60936839,362.65101902)(508.95832438,363.74997625)(511.90625,363.75)
\curveto(512.69790397,363.74997625)(513.49477817,363.67185133)(514.296875,363.515625)
\curveto(515.10935989,363.35935164)(515.95310905,363.12497688)(516.828125,362.8125)
}
}
{
\newrgbcolor{curcolor}{0 0 0}
\pscustom[linestyle=none,fillstyle=solid,fillcolor=curcolor]
{
\newpath
\moveto(42.625,273.328125)
\lineto(57.625,273.328125)
\lineto(57.625,271.984375)
\lineto(49.15625,250)
\lineto(45.859375,250)
\lineto(53.828125,270.671875)
\lineto(42.625,270.671875)
\lineto(42.625,273.328125)
}
}
{
\newrgbcolor{curcolor}{0 0 0}
\pscustom[linestyle=none,fillstyle=solid,fillcolor=curcolor]
{
\newpath
\moveto(510.171875,201.078125)
\curveto(508.67186633,201.07811392)(507.48957584,200.67707266)(506.625,199.875)
\curveto(505.77082756,199.07290759)(505.34374466,197.96874203)(505.34375,196.5625)
\curveto(505.34374466,195.15624484)(505.77082756,194.05207928)(506.625,193.25)
\curveto(507.48957584,192.44791422)(508.67186633,192.04687295)(510.171875,192.046875)
\curveto(511.67186333,192.04687295)(512.85415381,192.44791422)(513.71875,193.25)
\curveto(514.58331875,194.06249594)(515.01560998,195.1666615)(515.015625,196.5625)
\curveto(515.01560998,197.96874203)(514.58331875,199.07290759)(513.71875,199.875)
\curveto(512.86457047,200.67707266)(511.68227998,201.07811392)(510.171875,201.078125)
\moveto(507.015625,202.421875)
\curveto(505.66145267,202.75519558)(504.60416206,203.38540328)(503.84375,204.3125)
\curveto(503.09374691,205.23956809)(502.71874728,206.3697753)(502.71875,207.703125)
\curveto(502.71874728,209.56768877)(503.38020495,211.04164563)(504.703125,212.125)
\curveto(506.0364523,213.20831013)(507.85936714,213.74997625)(510.171875,213.75)
\curveto(512.49477917,213.74997625)(514.31769402,213.20831013)(515.640625,212.125)
\curveto(516.9635247,211.04164563)(517.62498237,209.56768877)(517.625,207.703125)
\curveto(517.62498237,206.3697753)(517.24477442,205.23956809)(516.484375,204.3125)
\curveto(515.73435927,203.38540328)(514.68748531,202.75519558)(513.34375,202.421875)
\curveto(514.86456847,202.06769627)(516.04685895,201.37498863)(516.890625,200.34375)
\curveto(517.74477392,199.31249069)(518.17185683,198.05207528)(518.171875,196.5625)
\curveto(518.17185683,194.30207903)(517.47914919,192.56770577)(516.09375,191.359375)
\curveto(514.71873528,190.15104152)(512.74477892,189.54687545)(510.171875,189.546875)
\curveto(507.59895073,189.54687545)(505.61978605,190.15104152)(504.234375,191.359375)
\curveto(502.85937214,192.56770577)(502.17187283,194.30207903)(502.171875,196.5625)
\curveto(502.17187283,198.05207528)(502.59895573,199.31249069)(503.453125,200.34375)
\curveto(504.30728736,201.37498863)(505.49478617,202.06769627)(507.015625,202.421875)
\moveto(505.859375,207.40625)
\curveto(505.85936914,206.19790047)(506.23436877,205.25519308)(506.984375,204.578125)
\curveto(507.74478392,203.90102777)(508.80728286,203.56248644)(510.171875,203.5625)
\curveto(511.52603014,203.56248644)(512.58332075,203.90102777)(513.34375,204.578125)
\curveto(514.11456922,205.25519308)(514.4999855,206.19790047)(514.5,207.40625)
\curveto(514.4999855,208.61456472)(514.11456922,209.55727211)(513.34375,210.234375)
\curveto(512.58332075,210.91143742)(511.52603014,211.24997875)(510.171875,211.25)
\curveto(508.80728286,211.24997875)(507.74478392,210.91143742)(506.984375,210.234375)
\curveto(506.23436877,209.55727211)(505.85936914,208.61456472)(505.859375,207.40625)
}
}
{
\newrgbcolor{curcolor}{0 0 0}
\pscustom[linestyle=none,fillstyle=solid,fillcolor=curcolor]
{
\newpath
\moveto(503.515625,140.484375)
\lineto(503.515625,143.359375)
\curveto(504.30728736,142.98437202)(505.10936989,142.69791397)(505.921875,142.5)
\curveto(506.73436827,142.30208103)(507.53124247,142.2031228)(508.3125,142.203125)
\curveto(510.39582294,142.2031228)(511.98436302,142.90103877)(513.078125,144.296875)
\curveto(514.18227748,145.7031193)(514.81248519,147.8333255)(514.96875,150.6875)
\curveto(514.36456897,149.79165688)(513.59894473,149.10415756)(512.671875,148.625)
\curveto(511.74477992,148.14582519)(510.71873928,147.90624209)(509.59375,147.90625)
\curveto(507.26040941,147.90624209)(505.41145292,148.60936639)(504.046875,150.015625)
\curveto(502.69270564,151.43228023)(502.01562298,153.36456997)(502.015625,155.8125)
\curveto(502.01562298,158.20831513)(502.72395561,160.1301882)(504.140625,161.578125)
\curveto(505.55728611,163.02601864)(507.44270089,163.74997625)(509.796875,163.75)
\curveto(512.49477917,163.74997625)(514.55206878,162.71351895)(515.96875,160.640625)
\curveto(517.39581594,158.57810642)(518.10935689,155.57810942)(518.109375,151.640625)
\curveto(518.10935689,147.9635337)(517.23435777,145.02603664)(515.484375,142.828125)
\curveto(513.74477792,140.64062436)(511.40103027,139.54687545)(508.453125,139.546875)
\curveto(507.66145067,139.54687545)(506.85936814,139.62500038)(506.046875,139.78125)
\curveto(505.23436977,139.93750006)(504.39062061,140.17187483)(503.515625,140.484375)
\moveto(509.796875,150.375)
\curveto(511.21353045,150.37498963)(512.333321,150.85936414)(513.15625,151.828125)
\curveto(513.98956934,152.7968622)(514.40623559,154.12498587)(514.40625,155.8125)
\curveto(514.40623559,157.48956584)(513.98956934,158.81248119)(513.15625,159.78125)
\curveto(512.333321,160.76039591)(511.21353045,161.24997875)(509.796875,161.25)
\curveto(508.38019995,161.24997875)(507.25520108,160.76039591)(506.421875,159.78125)
\curveto(505.59895273,158.81248119)(505.18749481,157.48956584)(505.1875,155.8125)
\curveto(505.18749481,154.12498587)(505.59895273,152.7968622)(506.421875,151.828125)
\curveto(507.25520108,150.85936414)(508.38019995,150.37498963)(509.796875,150.375)
}
}
{
\newrgbcolor{curcolor}{0 0 0}
\pscustom[linewidth=2,linecolor=curcolor,linestyle=dashed,dash=8 8]
{
\newpath
\moveto(59.189408,539.999593)
\lineto(139.51115,539.999593)
}
}
{
\newrgbcolor{curcolor}{0 0 0}
\pscustom[linestyle=none,fillstyle=solid,fillcolor=curcolor]
{
\newpath
\moveto(129.04884464,544.84007524)
\lineto(142.1598974,540.01880891)
\lineto(129.04884392,535.19754364)
\curveto(131.143442,538.04401672)(131.13137288,541.93848592)(129.04884464,544.84007524)
\closepath
}
}
{
\newrgbcolor{curcolor}{0 0 0}
\pscustom[linestyle=none,fillstyle=solid,fillcolor=curcolor]
{
\newpath
\moveto(43.96875,532.65625)
\lineto(49.125,532.65625)
\lineto(49.125,550.453125)
\lineto(43.515625,549.328125)
\lineto(43.515625,552.203125)
\lineto(49.09375,553.328125)
\lineto(52.25,553.328125)
\lineto(52.25,532.65625)
\lineto(57.40625,532.65625)
\lineto(57.40625,530)
\lineto(43.96875,530)
\lineto(43.96875,532.65625)
}
}
{
\newrgbcolor{curcolor}{0 0 0}
\pscustom[linestyle=none,fillstyle=solid,fillcolor=curcolor]
{
\newpath
\moveto(506.140625,482.65625)
\lineto(517.15625,482.65625)
\lineto(517.15625,480)
\lineto(502.34375,480)
\lineto(502.34375,482.65625)
\curveto(503.54166313,483.89582944)(505.17186983,485.55728611)(507.234375,487.640625)
\curveto(509.30728236,489.73436527)(510.60936439,491.08332225)(511.140625,491.6875)
\curveto(512.15102952,492.82290384)(512.85415381,493.78123622)(513.25,494.5625)
\curveto(513.65623634,495.35415131)(513.85936114,496.1301922)(513.859375,496.890625)
\curveto(513.85936114,498.1301902)(513.42186158,499.14060586)(512.546875,499.921875)
\curveto(511.68227998,500.7031043)(510.55207278,501.09372891)(509.15625,501.09375)
\curveto(508.1666585,501.09372891)(507.11978455,500.92185408)(506.015625,500.578125)
\curveto(504.92187008,500.23435477)(503.74999625,499.71352195)(502.5,499.015625)
\lineto(502.5,502.203125)
\curveto(503.77082956,502.71351895)(504.95832838,503.09893523)(506.0625,503.359375)
\curveto(507.1666595,503.61976805)(508.17707516,503.74997625)(509.09375,503.75)
\curveto(511.51040516,503.74997625)(513.43748656,503.14581019)(514.875,501.9375)
\curveto(516.31248369,500.72914594)(517.03123297,499.11456422)(517.03125,497.09375)
\curveto(517.03123297,496.13540053)(516.84894148,495.22394311)(516.484375,494.359375)
\curveto(516.1301922,493.50519483)(515.47915119,492.49477917)(514.53125,491.328125)
\curveto(514.27081906,491.02603064)(513.44269489,490.15103152)(512.046875,488.703125)
\curveto(510.65103102,487.26561773)(508.68228298,485.24999475)(506.140625,482.65625)
}
}
{
\newrgbcolor{curcolor}{0 0 0}
\pscustom[linestyle=none,fillstyle=solid,fillcolor=curcolor]
{
\newpath
\moveto(503.96875,92.65625)
\lineto(509.125,92.65625)
\lineto(509.125,110.453125)
\lineto(503.515625,109.328125)
\lineto(503.515625,112.203125)
\lineto(509.09375,113.328125)
\lineto(512.25,113.328125)
\lineto(512.25,92.65625)
\lineto(517.40625,92.65625)
\lineto(517.40625,90)
\lineto(503.96875,90)
\lineto(503.96875,92.65625)
}
}
{
\newrgbcolor{curcolor}{0 0 0}
\pscustom[linestyle=none,fillstyle=solid,fillcolor=curcolor]
{
\newpath
\moveto(530.546875,111.25)
\curveto(528.92186645,111.24997875)(527.69790934,110.44789622)(526.875,108.84375)
\curveto(526.06249431,107.24998275)(525.65624472,104.84894348)(525.65625,101.640625)
\curveto(525.65624472,98.44269989)(526.06249431,96.04166063)(526.875,94.4375)
\curveto(527.69790934,92.84374716)(528.92186645,92.04687295)(530.546875,92.046875)
\curveto(532.18227986,92.04687295)(533.40623697,92.84374716)(534.21875,94.4375)
\curveto(535.041652,96.04166063)(535.45310992,98.44269989)(535.453125,101.640625)
\curveto(535.45310992,104.84894348)(535.041652,107.24998275)(534.21875,108.84375)
\curveto(533.40623697,110.44789622)(532.18227986,111.24997875)(530.546875,111.25)
\moveto(530.546875,113.75)
\curveto(533.16144555,113.74997625)(535.15623522,112.71351895)(536.53125,110.640625)
\curveto(537.91664913,108.57810642)(538.60935677,105.57810942)(538.609375,101.640625)
\curveto(538.60935677,97.71353395)(537.91664913,94.71353695)(536.53125,92.640625)
\curveto(535.15623522,90.57812442)(533.16144555,89.54687545)(530.546875,89.546875)
\curveto(527.93228411,89.54687545)(525.93228611,90.57812442)(524.546875,92.640625)
\curveto(523.1718722,94.71353695)(522.48437289,97.71353395)(522.484375,101.640625)
\curveto(522.48437289,105.57810942)(523.1718722,108.57810642)(524.546875,110.640625)
\curveto(525.93228611,112.71351895)(527.93228411,113.74997625)(530.546875,113.75)
}
}
{
\newrgbcolor{curcolor}{0 0 0}
\pscustom[linestyle=none,fillstyle=solid,fillcolor=curcolor,opacity=0.11935484]
{
\newpath
\moveto(312.83469105,220)
\lineto(417.41930676,220)
\curveto(424.5297256,220)(430.2539978,215.83353307)(430.2539978,210.6581459)
\lineto(430.2539978,199.6365757)
\curveto(430.2539978,194.46118853)(424.5297256,190.2947216)(417.41930676,190.2947216)
\lineto(312.83469105,190.2947216)
\curveto(305.72427221,190.2947216)(300,194.46118853)(300,199.6365757)
\lineto(300,210.6581459)
\curveto(300,215.83353307)(305.72427221,220)(312.83469105,220)
\closepath
}
}
{
\newrgbcolor{curcolor}{0 0 0}
\pscustom[linewidth=2,linecolor=curcolor]
{
\newpath
\moveto(312.83469105,220)
\lineto(417.41930676,220)
\curveto(424.5297256,220)(430.2539978,215.83353307)(430.2539978,210.6581459)
\lineto(430.2539978,199.6365757)
\curveto(430.2539978,194.46118853)(424.5297256,190.2947216)(417.41930676,190.2947216)
\lineto(312.83469105,190.2947216)
\curveto(305.72427221,190.2947216)(300,194.46118853)(300,199.6365757)
\lineto(300,210.6581459)
\curveto(300,215.83353307)(305.72427221,220)(312.83469105,220)
\closepath
}
}
{
\newrgbcolor{curcolor}{0 0 0}
\pscustom[linewidth=2,linecolor=curcolor,linestyle=dashed,dash=8 8]
{
\newpath
\moveto(420,490)
\lineto(500,490)
}
}
{
\newrgbcolor{curcolor}{0 0 0}
\pscustom[linestyle=none,fillstyle=solid,fillcolor=curcolor]
{
\newpath
\moveto(430.46230536,485.15951776)
\lineto(417.3512526,489.98078409)
\lineto(430.46230608,494.80204936)
\curveto(428.367708,491.95557628)(428.37977712,488.06110708)(430.46230536,485.15951776)
\lineto(430.46230536,485.15951776)
\closepath
}
}
{
\newrgbcolor{curcolor}{0 0 0}
\pscustom[linewidth=2,linecolor=curcolor,linestyle=dashed,dash=8 8]
{
\newpath
\moveto(310,390)
\lineto(500,410)
}
}
{
\newrgbcolor{curcolor}{0 0 0}
\pscustom[linestyle=none,fillstyle=solid,fillcolor=curcolor]
{
\newpath
\moveto(320.91154457,386.2813582)
\lineto(307.36781788,389.70360612)
\lineto(319.90211894,395.87090853)
\curveto(318.11701215,392.82080298)(318.53670663,388.94899558)(320.91154457,386.2813582)
\lineto(320.91154457,386.2813582)
\closepath
}
}
{
\newrgbcolor{curcolor}{0 0 0}
\pscustom[linestyle=none,fillstyle=solid,fillcolor=curcolor]
{
\newpath
\moveto(43.96875,32.65625)
\lineto(49.125,32.65625)
\lineto(49.125,50.453125)
\lineto(43.515625,49.328125)
\lineto(43.515625,52.203125)
\lineto(49.09375,53.328125)
\lineto(52.25,53.328125)
\lineto(52.25,32.65625)
\lineto(57.40625,32.65625)
\lineto(57.40625,30)
\lineto(43.96875,30)
\lineto(43.96875,32.65625)
}
}
{
\newrgbcolor{curcolor}{0 0 0}
\pscustom[linestyle=none,fillstyle=solid,fillcolor=curcolor]
{
\newpath
\moveto(64.34375,32.65625)
\lineto(69.5,32.65625)
\lineto(69.5,50.453125)
\lineto(63.890625,49.328125)
\lineto(63.890625,52.203125)
\lineto(69.46875,53.328125)
\lineto(72.625,53.328125)
\lineto(72.625,32.65625)
\lineto(77.78125,32.65625)
\lineto(77.78125,30)
\lineto(64.34375,30)
\lineto(64.34375,32.65625)
}
}
{
\newrgbcolor{curcolor}{1 1 1}
\pscustom[linestyle=none,fillstyle=solid,fillcolor=curcolor]
{
\newpath
\moveto(160,460)
\lineto(400,460)
\lineto(400,430)
\lineto(160,430)
\closepath
}
}
{
\newrgbcolor{curcolor}{1 0 0}
\pscustom[linewidth=2.04672694,linecolor=curcolor,linestyle=dashed,dash=8.18690742 8.18690742]
{
\newpath
\moveto(160,460)
\lineto(400,460)
\lineto(400,430)
\lineto(160,430)
\closepath
}
}
{
\newrgbcolor{curcolor}{0 0 0}
\pscustom[linewidth=2,linecolor=curcolor,linestyle=dashed,dash=8 8]
{
\newpath
\moveto(190,450)
\lineto(60,450)
}
}
{
\newrgbcolor{curcolor}{0 0 0}
\pscustom[linestyle=none,fillstyle=solid,fillcolor=curcolor]
{
\newpath
\moveto(179.53769464,454.84048224)
\lineto(192.6487474,450.01921591)
\lineto(179.53769392,445.19795064)
\curveto(181.632292,448.04442372)(181.62022288,451.93889292)(179.53769464,454.84048224)
\lineto(179.53769464,454.84048224)
\closepath
}
}
{
\newrgbcolor{curcolor}{0 0 0}
\pscustom[linestyle=none,fillstyle=solid,fillcolor=curcolor]
{
\newpath
\moveto(503.96875,32.65625)
\lineto(509.125,32.65625)
\lineto(509.125,50.453125)
\lineto(503.515625,49.328125)
\lineto(503.515625,52.203125)
\lineto(509.09375,53.328125)
\lineto(512.25,53.328125)
\lineto(512.25,32.65625)
\lineto(517.40625,32.65625)
\lineto(517.40625,30)
\lineto(503.96875,30)
\lineto(503.96875,32.65625)
}
}
{
\newrgbcolor{curcolor}{0 0 0}
\pscustom[linestyle=none,fillstyle=solid,fillcolor=curcolor]
{
\newpath
\moveto(526.515625,32.65625)
\lineto(537.53125,32.65625)
\lineto(537.53125,30)
\lineto(522.71875,30)
\lineto(522.71875,32.65625)
\curveto(523.91666313,33.89582944)(525.54686983,35.55728611)(527.609375,37.640625)
\curveto(529.68228236,39.73436527)(530.98436439,41.08332225)(531.515625,41.6875)
\curveto(532.52602952,42.82290384)(533.22915381,43.78123622)(533.625,44.5625)
\curveto(534.03123634,45.35415131)(534.23436114,46.1301922)(534.234375,46.890625)
\curveto(534.23436114,48.1301902)(533.79686158,49.14060586)(532.921875,49.921875)
\curveto(532.05727998,50.7031043)(530.92707278,51.09372891)(529.53125,51.09375)
\curveto(528.5416585,51.09372891)(527.49478455,50.92185408)(526.390625,50.578125)
\curveto(525.29687008,50.23435477)(524.12499625,49.71352195)(522.875,49.015625)
\lineto(522.875,52.203125)
\curveto(524.14582956,52.71351895)(525.33332837,53.09893523)(526.4375,53.359375)
\curveto(527.5416595,53.61976805)(528.55207516,53.74997625)(529.46875,53.75)
\curveto(531.88540516,53.74997625)(533.81248656,53.14581019)(535.25,51.9375)
\curveto(536.68748369,50.72914594)(537.40623297,49.11456422)(537.40625,47.09375)
\curveto(537.40623297,46.13540053)(537.22394148,45.22394311)(536.859375,44.359375)
\curveto(536.5051922,43.50519483)(535.85415119,42.49477917)(534.90625,41.328125)
\curveto(534.64581906,41.02603064)(533.81769489,40.15103152)(532.421875,38.703125)
\curveto(531.02603102,37.26561773)(529.05728298,35.24999475)(526.515625,32.65625)
}
}
{
\newrgbcolor{curcolor}{0 0 0}
\pscustom[linestyle=none,fillstyle=solid,fillcolor=curcolor]
{
\newpath
\moveto(112.15917969,496.03808594)
\lineto(115.08105469,496.03808594)
\lineto(122.19238281,482.62109375)
\lineto(122.19238281,496.03808594)
\lineto(124.29785156,496.03808594)
\lineto(124.29785156,480)
\lineto(121.37597656,480)
\lineto(114.26464844,493.41699219)
\lineto(114.26464844,480)
\lineto(112.15917969,480)
\lineto(112.15917969,496.03808594)
}
}
{
\newrgbcolor{curcolor}{0 0 0}
\pscustom[linestyle=none,fillstyle=solid,fillcolor=curcolor]
{
\newpath
\moveto(133.19238281,490.64550781)
\curveto(132.1324813,490.64549717)(131.29459152,490.230133)(130.67871094,489.39941406)
\curveto(130.06282192,488.57583778)(129.75487951,487.44432849)(129.75488281,486.00488281)
\curveto(129.75487951,484.56542512)(130.05924119,483.43033511)(130.66796875,482.59960938)
\curveto(131.28384934,481.77603989)(132.12531985,481.36425645)(133.19238281,481.36425781)
\curveto(134.2451094,481.36425645)(135.07941846,481.77962062)(135.6953125,482.61035156)
\curveto(136.31118806,483.44107729)(136.61913046,484.57258657)(136.61914062,486.00488281)
\curveto(136.61913046,487.43000559)(136.31118806,488.55793415)(135.6953125,489.38867188)
\curveto(135.07941846,490.22655227)(134.2451094,490.64549717)(133.19238281,490.64550781)
\moveto(133.19238281,492.32128906)
\curveto(134.91112436,492.32127674)(136.2610579,491.76268355)(137.2421875,490.64550781)
\curveto(138.22329553,489.52831078)(138.71385493,487.98143733)(138.71386719,486.00488281)
\curveto(138.71385493,484.03547774)(138.22329553,482.48860428)(137.2421875,481.36425781)
\curveto(136.2610579,480.24707007)(134.91112436,479.68847687)(133.19238281,479.68847656)
\curveto(131.46646634,479.68847687)(130.11295207,480.24707007)(129.13183594,481.36425781)
\curveto(128.1578759,482.48860428)(127.67089722,484.03547774)(127.67089844,486.00488281)
\curveto(127.67089722,487.98143733)(128.1578759,489.52831078)(129.13183594,490.64550781)
\curveto(130.11295207,491.76268355)(131.46646634,492.32127674)(133.19238281,492.32128906)
}
}
{
\newrgbcolor{curcolor}{0 0 0}
\pscustom[linestyle=none,fillstyle=solid,fillcolor=curcolor]
{
\newpath
\moveto(151.34667969,489.72167969)
\curveto(151.84080838,490.60968991)(152.4316281,491.26496269)(153.11914062,491.6875)
\curveto(153.80662672,492.11001393)(154.61587071,492.32127674)(155.546875,492.32128906)
\curveto(156.80011331,492.32127674)(157.76690922,491.88084749)(158.44726562,491)
\curveto(159.12758495,490.12629196)(159.46775388,488.88019945)(159.46777344,487.26171875)
\lineto(159.46777344,480)
\lineto(157.48046875,480)
\lineto(157.48046875,487.19726562)
\curveto(157.48045118,488.35025207)(157.27634982,489.20604548)(156.86816406,489.76464844)
\curveto(156.45994438,490.32323186)(155.83689813,490.60252846)(154.99902344,490.60253906)
\curveto(153.97492083,490.60252846)(153.16567684,490.26235953)(152.57128906,489.58203125)
\curveto(151.97687595,488.90168381)(151.67967573,487.97427588)(151.6796875,486.79980469)
\lineto(151.6796875,480)
\lineto(149.69238281,480)
\lineto(149.69238281,487.19726562)
\curveto(149.69237303,488.35741352)(149.48827167,489.21320693)(149.08007812,489.76464844)
\curveto(148.67186623,490.32323186)(148.04165853,490.60252846)(147.18945312,490.60253906)
\curveto(146.17968123,490.60252846)(145.3775987,490.2587788)(144.78320312,489.57128906)
\curveto(144.1887978,488.89094163)(143.89159758,487.96711443)(143.89160156,486.79980469)
\lineto(143.89160156,480)
\lineto(141.90429688,480)
\lineto(141.90429688,492.03125)
\lineto(143.89160156,492.03125)
\lineto(143.89160156,490.16210938)
\curveto(144.342769,490.89972868)(144.88345856,491.44399897)(145.51367188,491.79492188)
\curveto(146.14387397,492.14582119)(146.89224562,492.32127674)(147.75878906,492.32128906)
\curveto(148.63247825,492.32127674)(149.37368845,492.09927175)(149.98242188,491.65527344)
\curveto(150.5982966,491.21125181)(151.05304875,490.5667212)(151.34667969,489.72167969)
}
}
{
\newrgbcolor{curcolor}{0 0 0}
\pscustom[linestyle=none,fillstyle=solid,fillcolor=curcolor]
{
}
}
{
\newrgbcolor{curcolor}{0 0 0}
\pscustom[linestyle=none,fillstyle=solid,fillcolor=curcolor]
{
\newpath
\moveto(178.34179688,490.20507812)
\lineto(178.34179688,496.71484375)
\lineto(180.31835938,496.71484375)
\lineto(180.31835938,480)
\lineto(178.34179688,480)
\lineto(178.34179688,481.8046875)
\curveto(177.92642272,481.08854058)(177.40005606,480.55501247)(176.76269531,480.20410156)
\curveto(176.1324792,479.8603517)(175.37336537,479.68847687)(174.48535156,479.68847656)
\curveto(173.03157084,479.68847687)(171.84635067,480.26855442)(170.9296875,481.42871094)
\curveto(170.02018062,482.5888646)(169.56542847,484.1142537)(169.56542969,486.00488281)
\curveto(169.56542847,487.89549992)(170.02018062,489.42088902)(170.9296875,490.58105469)
\curveto(171.84635067,491.7411992)(173.03157084,492.32127674)(174.48535156,492.32128906)
\curveto(175.37336537,492.32127674)(176.1324792,492.14582119)(176.76269531,491.79492188)
\curveto(177.40005606,491.45116042)(177.92642272,490.92121304)(178.34179688,490.20507812)
\moveto(171.60644531,486.00488281)
\curveto(171.60644206,484.55110222)(171.90364228,483.40885076)(172.49804688,482.578125)
\curveto(173.09960463,481.75455554)(173.92317151,481.34277209)(174.96875,481.34277344)
\curveto(176.01431525,481.34277209)(176.83788214,481.75455554)(177.43945312,482.578125)
\curveto(178.04100594,483.40885076)(178.34178688,484.55110222)(178.34179688,486.00488281)
\curveto(178.34178688,487.4586514)(178.04100594,488.59732213)(177.43945312,489.42089844)
\curveto(176.83788214,490.25161735)(176.01431525,490.66698152)(174.96875,490.66699219)
\curveto(173.92317151,490.66698152)(173.09960463,490.25161735)(172.49804688,489.42089844)
\curveto(171.90364228,488.59732213)(171.60644206,487.4586514)(171.60644531,486.00488281)
}
}
{
\newrgbcolor{curcolor}{0 0 0}
\pscustom[linestyle=none,fillstyle=solid,fillcolor=curcolor]
{
\newpath
\moveto(194.68066406,486.50976562)
\lineto(194.68066406,485.54296875)
\lineto(185.59277344,485.54296875)
\curveto(185.67870758,484.18228748)(186.08691029,483.14387706)(186.81738281,482.42773438)
\curveto(187.55500778,481.71874828)(188.5790953,481.36425645)(189.88964844,481.36425781)
\curveto(190.64875469,481.36425645)(191.38280343,481.45735531)(192.09179688,481.64355469)
\curveto(192.80793222,481.82975077)(193.51691588,482.10904737)(194.21875,482.48144531)
\lineto(194.21875,480.61230469)
\curveto(193.50975443,480.31152313)(192.78286714,480.08235669)(192.03808594,479.92480469)
\curveto(191.29328529,479.76725284)(190.5377522,479.68847687)(189.77148438,479.68847656)
\curveto(187.85220801,479.68847687)(186.33039963,480.24707007)(185.20605469,481.36425781)
\curveto(184.08886542,482.48144283)(183.53027222,483.99250903)(183.53027344,485.89746094)
\curveto(183.53027222,487.86685411)(184.06021961,489.42805047)(185.12011719,490.58105469)
\curveto(186.18717061,491.7411992)(187.62304157,492.32127674)(189.42773438,492.32128906)
\curveto(191.04621523,492.32127674)(192.32453426,491.79849081)(193.26269531,490.75292969)
\curveto(194.20799592,489.71450851)(194.6806517,488.30012191)(194.68066406,486.50976562)
\moveto(192.70410156,487.08984375)
\curveto(192.68976827,488.17121579)(192.3854066,489.03417065)(191.79101562,489.67871094)
\curveto(191.20376715,490.32323186)(190.42316898,490.64549717)(189.44921875,490.64550781)
\curveto(188.34634814,490.64549717)(187.46190892,490.33397404)(186.79589844,489.7109375)
\curveto(186.13704045,489.08788154)(185.75748354,488.21060377)(185.65722656,487.07910156)
\lineto(192.70410156,487.08984375)
}
}
{
\newrgbcolor{curcolor}{0 0 0}
\pscustom[linestyle=none,fillstyle=solid,fillcolor=curcolor]
{
}
}
{
\newrgbcolor{curcolor}{0 0 0}
\pscustom[linestyle=none,fillstyle=solid,fillcolor=curcolor]
{
\newpath
\moveto(204.92871094,496.71484375)
\lineto(206.90527344,496.71484375)
\lineto(206.90527344,480)
\lineto(204.92871094,480)
\lineto(204.92871094,496.71484375)
}
}
{
\newrgbcolor{curcolor}{0 0 0}
\pscustom[linestyle=none,fillstyle=solid,fillcolor=curcolor]
{
\newpath
\moveto(216.49804688,486.04785156)
\curveto(214.90103572,486.04784551)(213.79459152,485.86522851)(213.17871094,485.5)
\curveto(212.56282192,485.13476049)(212.25487951,484.51171424)(212.25488281,483.63085938)
\curveto(212.25487951,482.92903353)(212.48404595,482.37044034)(212.94238281,481.95507812)
\curveto(213.40787315,481.54687345)(214.03808086,481.34277209)(214.83300781,481.34277344)
\curveto(215.92870397,481.34277209)(216.80598173,481.72949046)(217.46484375,482.50292969)
\curveto(218.1308502,483.28352536)(218.46385768,484.31835506)(218.46386719,485.60742188)
\lineto(218.46386719,486.04785156)
\lineto(216.49804688,486.04785156)
\moveto(220.44042969,486.86425781)
\lineto(220.44042969,480)
\lineto(218.46386719,480)
\lineto(218.46386719,481.82617188)
\curveto(218.01268626,481.09570203)(217.45051234,480.55501247)(216.77734375,480.20410156)
\curveto(216.10415952,479.8603517)(215.28059263,479.68847687)(214.30664062,479.68847656)
\curveto(213.07486567,479.68847687)(212.09374686,480.03222653)(211.36328125,480.71972656)
\curveto(210.63997228,481.41438661)(210.27831899,482.34179453)(210.27832031,483.50195312)
\curveto(210.27831899,484.85546389)(210.72949042,485.87597069)(211.63183594,486.56347656)
\curveto(212.54133756,487.25096931)(213.89485183,487.59471897)(215.69238281,487.59472656)
\lineto(218.46386719,487.59472656)
\lineto(218.46386719,487.78808594)
\curveto(218.46385768,488.69758245)(218.16307673,489.39940466)(217.56152344,489.89355469)
\curveto(216.96711439,490.39484638)(216.1292246,490.64549717)(215.04785156,490.64550781)
\curveto(214.36034616,490.64549717)(213.69075047,490.56314048)(213.0390625,490.3984375)
\curveto(212.38736636,490.23371372)(211.76073938,489.98664366)(211.15917969,489.65722656)
\lineto(211.15917969,491.48339844)
\curveto(211.88248405,491.76268355)(212.58430627,491.97036563)(213.26464844,492.10644531)
\curveto(213.94498199,492.24966223)(214.60741622,492.32127674)(215.25195312,492.32128906)
\curveto(216.99217946,492.32127674)(218.29198285,491.87010532)(219.15136719,490.96777344)
\curveto(220.01073113,490.06541962)(220.4404182,488.69758245)(220.44042969,486.86425781)
}
}
{
\newrgbcolor{curcolor}{0 0 0}
\pscustom[linestyle=none,fillstyle=solid,fillcolor=curcolor]
{
}
}
{
\newrgbcolor{curcolor}{0 0 0}
\pscustom[linestyle=none,fillstyle=solid,fillcolor=curcolor]
{
\newpath
\moveto(233.43847656,481.8046875)
\lineto(233.43847656,475.42382812)
\lineto(231.45117188,475.42382812)
\lineto(231.45117188,492.03125)
\lineto(233.43847656,492.03125)
\lineto(233.43847656,490.20507812)
\curveto(233.85383675,490.92121304)(234.37662268,491.45116042)(235.00683594,491.79492188)
\curveto(235.64419954,492.14582119)(236.40331336,492.32127674)(237.28417969,492.32128906)
\curveto(238.7451079,492.32127674)(239.93032806,491.7411992)(240.83984375,490.58105469)
\curveto(241.75649811,489.42088902)(242.21483099,487.89549992)(242.21484375,486.00488281)
\curveto(242.21483099,484.1142537)(241.75649811,482.5888646)(240.83984375,481.42871094)
\curveto(239.93032806,480.26855442)(238.7451079,479.68847687)(237.28417969,479.68847656)
\curveto(236.40331336,479.68847687)(235.64419954,479.8603517)(235.00683594,480.20410156)
\curveto(234.37662268,480.55501247)(233.85383675,481.08854058)(233.43847656,481.8046875)
\moveto(240.16308594,486.00488281)
\curveto(240.16307523,487.4586514)(239.86229428,488.59732213)(239.26074219,489.42089844)
\curveto(238.66633193,490.25161735)(237.84634577,490.66698152)(236.80078125,490.66699219)
\curveto(235.75520203,490.66698152)(234.93163515,490.25161735)(234.33007812,489.42089844)
\curveto(233.7356728,488.59732213)(233.43847258,487.4586514)(233.43847656,486.00488281)
\curveto(233.43847258,484.55110222)(233.7356728,483.40885076)(234.33007812,482.578125)
\curveto(234.93163515,481.75455554)(235.75520203,481.34277209)(236.80078125,481.34277344)
\curveto(237.84634577,481.34277209)(238.66633193,481.75455554)(239.26074219,482.578125)
\curveto(239.86229428,483.40885076)(240.16307523,484.55110222)(240.16308594,486.00488281)
}
}
{
\newrgbcolor{curcolor}{0 0 0}
\pscustom[linestyle=none,fillstyle=solid,fillcolor=curcolor]
{
\newpath
\moveto(250.95898438,486.04785156)
\curveto(249.36197322,486.04784551)(248.25552902,485.86522851)(247.63964844,485.5)
\curveto(247.02375942,485.13476049)(246.71581701,484.51171424)(246.71582031,483.63085938)
\curveto(246.71581701,482.92903353)(246.94498345,482.37044034)(247.40332031,481.95507812)
\curveto(247.86881065,481.54687345)(248.49901836,481.34277209)(249.29394531,481.34277344)
\curveto(250.38964147,481.34277209)(251.26691923,481.72949046)(251.92578125,482.50292969)
\curveto(252.5917877,483.28352536)(252.92479518,484.31835506)(252.92480469,485.60742188)
\lineto(252.92480469,486.04785156)
\lineto(250.95898438,486.04785156)
\moveto(254.90136719,486.86425781)
\lineto(254.90136719,480)
\lineto(252.92480469,480)
\lineto(252.92480469,481.82617188)
\curveto(252.47362376,481.09570203)(251.91144984,480.55501247)(251.23828125,480.20410156)
\curveto(250.56509702,479.8603517)(249.74153013,479.68847687)(248.76757812,479.68847656)
\curveto(247.53580317,479.68847687)(246.55468436,480.03222653)(245.82421875,480.71972656)
\curveto(245.10090978,481.41438661)(244.73925649,482.34179453)(244.73925781,483.50195312)
\curveto(244.73925649,484.85546389)(245.19042792,485.87597069)(246.09277344,486.56347656)
\curveto(247.00227506,487.25096931)(248.35578933,487.59471897)(250.15332031,487.59472656)
\lineto(252.92480469,487.59472656)
\lineto(252.92480469,487.78808594)
\curveto(252.92479518,488.69758245)(252.62401423,489.39940466)(252.02246094,489.89355469)
\curveto(251.42805189,490.39484638)(250.5901621,490.64549717)(249.50878906,490.64550781)
\curveto(248.82128366,490.64549717)(248.15168797,490.56314048)(247.5,490.3984375)
\curveto(246.84830386,490.23371372)(246.22167688,489.98664366)(245.62011719,489.65722656)
\lineto(245.62011719,491.48339844)
\curveto(246.34342155,491.76268355)(247.04524377,491.97036563)(247.72558594,492.10644531)
\curveto(248.40591949,492.24966223)(249.06835372,492.32127674)(249.71289062,492.32128906)
\curveto(251.45311696,492.32127674)(252.75292035,491.87010532)(253.61230469,490.96777344)
\curveto(254.47166863,490.06541962)(254.9013557,488.69758245)(254.90136719,486.86425781)
}
}
{
\newrgbcolor{curcolor}{0 0 0}
\pscustom[linestyle=none,fillstyle=solid,fillcolor=curcolor]
{
\newpath
\moveto(265.95507812,490.18359375)
\curveto(265.73306409,490.31248969)(265.48957475,490.40558855)(265.22460938,490.46289062)
\curveto(264.96678882,490.52733322)(264.68033077,490.55955975)(264.36523438,490.55957031)
\curveto(263.24804054,490.55955975)(262.3886664,490.19432574)(261.78710938,489.46386719)
\curveto(261.19270405,488.74055116)(260.89550383,487.69856001)(260.89550781,486.33789062)
\lineto(260.89550781,480)
\lineto(258.90820312,480)
\lineto(258.90820312,492.03125)
\lineto(260.89550781,492.03125)
\lineto(260.89550781,490.16210938)
\curveto(261.310868,490.89256723)(261.85155756,491.4332568)(262.51757812,491.78417969)
\curveto(263.18358748,492.14224046)(263.99283146,492.32127674)(264.9453125,492.32128906)
\curveto(265.08137204,492.32127674)(265.23176251,492.31053456)(265.39648438,492.2890625)
\curveto(265.56118927,492.27472731)(265.74380627,492.24966223)(265.94433594,492.21386719)
\lineto(265.95507812,490.18359375)
}
}
{
\newrgbcolor{curcolor}{0 0 0}
\pscustom[linestyle=none,fillstyle=solid,fillcolor=curcolor]
{
\newpath
\moveto(270.00488281,495.44726562)
\lineto(270.00488281,492.03125)
\lineto(274.07617188,492.03125)
\lineto(274.07617188,490.49511719)
\lineto(270.00488281,490.49511719)
\lineto(270.00488281,483.96386719)
\curveto(270.00487878,482.98274441)(270.13736563,482.35253671)(270.40234375,482.07324219)
\curveto(270.67447447,481.79394352)(271.22232548,481.65429522)(272.04589844,481.65429688)
\lineto(274.07617188,481.65429688)
\lineto(274.07617188,480)
\lineto(272.04589844,480)
\curveto(270.52050327,480)(269.46776995,480.28287732)(268.88769531,480.84863281)
\curveto(268.30761486,481.42154806)(268.01757608,482.45995848)(268.01757812,483.96386719)
\lineto(268.01757812,490.49511719)
\lineto(266.56738281,490.49511719)
\lineto(266.56738281,492.03125)
\lineto(268.01757812,492.03125)
\lineto(268.01757812,495.44726562)
\lineto(270.00488281,495.44726562)
}
}
{
\newrgbcolor{curcolor}{0 0 0}
\pscustom[linestyle=none,fillstyle=solid,fillcolor=curcolor]
{
\newpath
\moveto(276.68652344,492.03125)
\lineto(278.66308594,492.03125)
\lineto(278.66308594,480)
\lineto(276.68652344,480)
\lineto(276.68652344,492.03125)
\moveto(276.68652344,496.71484375)
\lineto(278.66308594,496.71484375)
\lineto(278.66308594,494.21191406)
\lineto(276.68652344,494.21191406)
\lineto(276.68652344,496.71484375)
}
}
{
\newrgbcolor{curcolor}{0 0 0}
\pscustom[linestyle=none,fillstyle=solid,fillcolor=curcolor]
{
\newpath
\moveto(293.07910156,486.50976562)
\lineto(293.07910156,485.54296875)
\lineto(283.99121094,485.54296875)
\curveto(284.07714508,484.18228748)(284.48534779,483.14387706)(285.21582031,482.42773438)
\curveto(285.95344528,481.71874828)(286.9775328,481.36425645)(288.28808594,481.36425781)
\curveto(289.04719219,481.36425645)(289.78124093,481.45735531)(290.49023438,481.64355469)
\curveto(291.20636972,481.82975077)(291.91535338,482.10904737)(292.6171875,482.48144531)
\lineto(292.6171875,480.61230469)
\curveto(291.90819193,480.31152313)(291.18130464,480.08235669)(290.43652344,479.92480469)
\curveto(289.69172279,479.76725284)(288.9361897,479.68847687)(288.16992188,479.68847656)
\curveto(286.25064551,479.68847687)(284.72883713,480.24707007)(283.60449219,481.36425781)
\curveto(282.48730292,482.48144283)(281.92870972,483.99250903)(281.92871094,485.89746094)
\curveto(281.92870972,487.86685411)(282.45865711,489.42805047)(283.51855469,490.58105469)
\curveto(284.58560811,491.7411992)(286.02147907,492.32127674)(287.82617188,492.32128906)
\curveto(289.44465273,492.32127674)(290.72297176,491.79849081)(291.66113281,490.75292969)
\curveto(292.60643342,489.71450851)(293.0790892,488.30012191)(293.07910156,486.50976562)
\moveto(291.10253906,487.08984375)
\curveto(291.08820577,488.17121579)(290.7838441,489.03417065)(290.18945312,489.67871094)
\curveto(289.60220465,490.32323186)(288.82160648,490.64549717)(287.84765625,490.64550781)
\curveto(286.74478564,490.64549717)(285.86034642,490.33397404)(285.19433594,489.7109375)
\curveto(284.53547795,489.08788154)(284.15592104,488.21060377)(284.05566406,487.07910156)
\lineto(291.10253906,487.08984375)
}
}
{
\newrgbcolor{curcolor}{0 0 0}
\pscustom[linestyle=none,fillstyle=solid,fillcolor=curcolor]
{
\newpath
\moveto(129.93554688,216.03808594)
\lineto(143.50292969,216.03808594)
\lineto(143.50292969,214.21191406)
\lineto(137.80957031,214.21191406)
\lineto(137.80957031,200)
\lineto(135.62890625,200)
\lineto(135.62890625,214.21191406)
\lineto(129.93554688,214.21191406)
\lineto(129.93554688,216.03808594)
}
}
{
\newrgbcolor{curcolor}{0 0 0}
\pscustom[linestyle=none,fillstyle=solid,fillcolor=curcolor]
{
\newpath
\moveto(147.09082031,198.8828125)
\curveto(146.53222004,197.45052338)(145.98794975,196.515954)(145.45800781,196.07910156)
\curveto(144.92805498,195.64225696)(144.21907131,195.4238327)(143.33105469,195.42382812)
\lineto(141.75195312,195.42382812)
\lineto(141.75195312,197.078125)
\lineto(142.91210938,197.078125)
\curveto(143.45637676,197.07812792)(143.87890238,197.20703404)(144.1796875,197.46484375)
\curveto(144.48046428,197.72265853)(144.81347176,198.33138188)(145.17871094,199.29101562)
\lineto(145.53320312,200.19335938)
\lineto(140.66699219,212.03125)
\lineto(142.76171875,212.03125)
\lineto(146.52148438,202.62109375)
\lineto(150.28125,212.03125)
\lineto(152.37597656,212.03125)
\lineto(147.09082031,198.8828125)
}
}
{
\newrgbcolor{curcolor}{0 0 0}
\pscustom[linestyle=none,fillstyle=solid,fillcolor=curcolor]
{
\newpath
\moveto(157.01660156,201.8046875)
\lineto(157.01660156,195.42382812)
\lineto(155.02929688,195.42382812)
\lineto(155.02929688,212.03125)
\lineto(157.01660156,212.03125)
\lineto(157.01660156,210.20507812)
\curveto(157.43196175,210.92121304)(157.95474768,211.45116042)(158.58496094,211.79492188)
\curveto(159.22232454,212.14582119)(159.98143836,212.32127674)(160.86230469,212.32128906)
\curveto(162.3232329,212.32127674)(163.50845306,211.7411992)(164.41796875,210.58105469)
\curveto(165.33462311,209.42088902)(165.79295599,207.89549992)(165.79296875,206.00488281)
\curveto(165.79295599,204.1142537)(165.33462311,202.5888646)(164.41796875,201.42871094)
\curveto(163.50845306,200.26855442)(162.3232329,199.68847687)(160.86230469,199.68847656)
\curveto(159.98143836,199.68847687)(159.22232454,199.8603517)(158.58496094,200.20410156)
\curveto(157.95474768,200.55501247)(157.43196175,201.08854058)(157.01660156,201.8046875)
\moveto(163.74121094,206.00488281)
\curveto(163.74120023,207.4586514)(163.44041928,208.59732213)(162.83886719,209.42089844)
\curveto(162.24445693,210.25161735)(161.42447077,210.66698152)(160.37890625,210.66699219)
\curveto(159.33332703,210.66698152)(158.50976015,210.25161735)(157.90820312,209.42089844)
\curveto(157.3137978,208.59732213)(157.01659758,207.4586514)(157.01660156,206.00488281)
\curveto(157.01659758,204.55110222)(157.3137978,203.40885076)(157.90820312,202.578125)
\curveto(158.50976015,201.75455554)(159.33332703,201.34277209)(160.37890625,201.34277344)
\curveto(161.42447077,201.34277209)(162.24445693,201.75455554)(162.83886719,202.578125)
\curveto(163.44041928,203.40885076)(163.74120023,204.55110222)(163.74121094,206.00488281)
}
}
{
\newrgbcolor{curcolor}{0 0 0}
\pscustom[linestyle=none,fillstyle=solid,fillcolor=curcolor]
{
\newpath
\moveto(179.36035156,206.50976562)
\lineto(179.36035156,205.54296875)
\lineto(170.27246094,205.54296875)
\curveto(170.35839508,204.18228748)(170.76659779,203.14387706)(171.49707031,202.42773438)
\curveto(172.23469528,201.71874828)(173.2587828,201.36425645)(174.56933594,201.36425781)
\curveto(175.32844219,201.36425645)(176.06249093,201.45735531)(176.77148438,201.64355469)
\curveto(177.48761972,201.82975077)(178.19660338,202.10904737)(178.8984375,202.48144531)
\lineto(178.8984375,200.61230469)
\curveto(178.18944193,200.31152313)(177.46255464,200.08235669)(176.71777344,199.92480469)
\curveto(175.97297279,199.76725284)(175.2174397,199.68847687)(174.45117188,199.68847656)
\curveto(172.53189551,199.68847687)(171.01008713,200.24707007)(169.88574219,201.36425781)
\curveto(168.76855292,202.48144283)(168.20995972,203.99250903)(168.20996094,205.89746094)
\curveto(168.20995972,207.86685411)(168.73990711,209.42805047)(169.79980469,210.58105469)
\curveto(170.86685811,211.7411992)(172.30272907,212.32127674)(174.10742188,212.32128906)
\curveto(175.72590273,212.32127674)(177.00422176,211.79849081)(177.94238281,210.75292969)
\curveto(178.88768342,209.71450851)(179.3603392,208.30012191)(179.36035156,206.50976562)
\moveto(177.38378906,207.08984375)
\curveto(177.36945577,208.17121579)(177.0650941,209.03417065)(176.47070312,209.67871094)
\curveto(175.88345465,210.32323186)(175.10285648,210.64549717)(174.12890625,210.64550781)
\curveto(173.02603564,210.64549717)(172.14159642,210.33397404)(171.47558594,209.7109375)
\curveto(170.81672795,209.08788154)(170.43717104,208.21060377)(170.33691406,207.07910156)
\lineto(177.38378906,207.08984375)
}
}
{
\newrgbcolor{curcolor}{0 0 0}
\pscustom[linestyle=none,fillstyle=solid,fillcolor=curcolor]
{
}
}
{
\newrgbcolor{curcolor}{0 0 0}
\pscustom[linestyle=none,fillstyle=solid,fillcolor=curcolor]
{
\newpath
\moveto(197.52539062,210.20507812)
\lineto(197.52539062,216.71484375)
\lineto(199.50195312,216.71484375)
\lineto(199.50195312,200)
\lineto(197.52539062,200)
\lineto(197.52539062,201.8046875)
\curveto(197.11001647,201.08854058)(196.58364981,200.55501247)(195.94628906,200.20410156)
\curveto(195.31607295,199.8603517)(194.55695912,199.68847687)(193.66894531,199.68847656)
\curveto(192.21516459,199.68847687)(191.02994442,200.26855442)(190.11328125,201.42871094)
\curveto(189.20377437,202.5888646)(188.74902222,204.1142537)(188.74902344,206.00488281)
\curveto(188.74902222,207.89549992)(189.20377437,209.42088902)(190.11328125,210.58105469)
\curveto(191.02994442,211.7411992)(192.21516459,212.32127674)(193.66894531,212.32128906)
\curveto(194.55695912,212.32127674)(195.31607295,212.14582119)(195.94628906,211.79492188)
\curveto(196.58364981,211.45116042)(197.11001647,210.92121304)(197.52539062,210.20507812)
\moveto(190.79003906,206.00488281)
\curveto(190.79003581,204.55110222)(191.08723603,203.40885076)(191.68164062,202.578125)
\curveto(192.28319838,201.75455554)(193.10676526,201.34277209)(194.15234375,201.34277344)
\curveto(195.197909,201.34277209)(196.02147589,201.75455554)(196.62304688,202.578125)
\curveto(197.22459969,203.40885076)(197.52538063,204.55110222)(197.52539062,206.00488281)
\curveto(197.52538063,207.4586514)(197.22459969,208.59732213)(196.62304688,209.42089844)
\curveto(196.02147589,210.25161735)(195.197909,210.66698152)(194.15234375,210.66699219)
\curveto(193.10676526,210.66698152)(192.28319838,210.25161735)(191.68164062,209.42089844)
\curveto(191.08723603,208.59732213)(190.79003581,207.4586514)(190.79003906,206.00488281)
}
}
{
\newrgbcolor{curcolor}{0 0 0}
\pscustom[linestyle=none,fillstyle=solid,fillcolor=curcolor]
{
\newpath
\moveto(213.86425781,206.50976562)
\lineto(213.86425781,205.54296875)
\lineto(204.77636719,205.54296875)
\curveto(204.86230133,204.18228748)(205.27050404,203.14387706)(206.00097656,202.42773438)
\curveto(206.73860153,201.71874828)(207.76268905,201.36425645)(209.07324219,201.36425781)
\curveto(209.83234844,201.36425645)(210.56639718,201.45735531)(211.27539062,201.64355469)
\curveto(211.99152597,201.82975077)(212.70050963,202.10904737)(213.40234375,202.48144531)
\lineto(213.40234375,200.61230469)
\curveto(212.69334818,200.31152313)(211.96646089,200.08235669)(211.22167969,199.92480469)
\curveto(210.47687904,199.76725284)(209.72134595,199.68847687)(208.95507812,199.68847656)
\curveto(207.03580176,199.68847687)(205.51399338,200.24707007)(204.38964844,201.36425781)
\curveto(203.27245917,202.48144283)(202.71386597,203.99250903)(202.71386719,205.89746094)
\curveto(202.71386597,207.86685411)(203.24381336,209.42805047)(204.30371094,210.58105469)
\curveto(205.37076436,211.7411992)(206.80663532,212.32127674)(208.61132812,212.32128906)
\curveto(210.22980898,212.32127674)(211.50812801,211.79849081)(212.44628906,210.75292969)
\curveto(213.39158967,209.71450851)(213.86424545,208.30012191)(213.86425781,206.50976562)
\moveto(211.88769531,207.08984375)
\curveto(211.87336202,208.17121579)(211.56900035,209.03417065)(210.97460938,209.67871094)
\curveto(210.3873609,210.32323186)(209.60676273,210.64549717)(208.6328125,210.64550781)
\curveto(207.52994189,210.64549717)(206.64550267,210.33397404)(205.97949219,209.7109375)
\curveto(205.3206342,209.08788154)(204.94107729,208.21060377)(204.84082031,207.07910156)
\lineto(211.88769531,207.08984375)
}
}
{
\newrgbcolor{curcolor}{0 0 0}
\pscustom[linestyle=none,fillstyle=solid,fillcolor=curcolor]
{
}
}
{
\newrgbcolor{curcolor}{0 0 0}
\pscustom[linestyle=none,fillstyle=solid,fillcolor=curcolor]
{
\newpath
\moveto(226.02441406,201.8046875)
\lineto(226.02441406,195.42382812)
\lineto(224.03710938,195.42382812)
\lineto(224.03710938,212.03125)
\lineto(226.02441406,212.03125)
\lineto(226.02441406,210.20507812)
\curveto(226.43977425,210.92121304)(226.96256018,211.45116042)(227.59277344,211.79492188)
\curveto(228.23013704,212.14582119)(228.98925086,212.32127674)(229.87011719,212.32128906)
\curveto(231.3310454,212.32127674)(232.51626556,211.7411992)(233.42578125,210.58105469)
\curveto(234.34243561,209.42088902)(234.80076849,207.89549992)(234.80078125,206.00488281)
\curveto(234.80076849,204.1142537)(234.34243561,202.5888646)(233.42578125,201.42871094)
\curveto(232.51626556,200.26855442)(231.3310454,199.68847687)(229.87011719,199.68847656)
\curveto(228.98925086,199.68847687)(228.23013704,199.8603517)(227.59277344,200.20410156)
\curveto(226.96256018,200.55501247)(226.43977425,201.08854058)(226.02441406,201.8046875)
\moveto(232.74902344,206.00488281)
\curveto(232.74901273,207.4586514)(232.44823178,208.59732213)(231.84667969,209.42089844)
\curveto(231.25226943,210.25161735)(230.43228327,210.66698152)(229.38671875,210.66699219)
\curveto(228.34113953,210.66698152)(227.51757265,210.25161735)(226.91601562,209.42089844)
\curveto(226.3216103,208.59732213)(226.02441008,207.4586514)(226.02441406,206.00488281)
\curveto(226.02441008,204.55110222)(226.3216103,203.40885076)(226.91601562,202.578125)
\curveto(227.51757265,201.75455554)(228.34113953,201.34277209)(229.38671875,201.34277344)
\curveto(230.43228327,201.34277209)(231.25226943,201.75455554)(231.84667969,202.578125)
\curveto(232.44823178,203.40885076)(232.74901273,204.55110222)(232.74902344,206.00488281)
}
}
{
\newrgbcolor{curcolor}{0 0 0}
\pscustom[linestyle=none,fillstyle=solid,fillcolor=curcolor]
{
\newpath
\moveto(243.54492188,206.04785156)
\curveto(241.94791072,206.04784551)(240.84146652,205.86522851)(240.22558594,205.5)
\curveto(239.60969692,205.13476049)(239.30175451,204.51171424)(239.30175781,203.63085938)
\curveto(239.30175451,202.92903353)(239.53092095,202.37044034)(239.98925781,201.95507812)
\curveto(240.45474815,201.54687345)(241.08495586,201.34277209)(241.87988281,201.34277344)
\curveto(242.97557897,201.34277209)(243.85285673,201.72949046)(244.51171875,202.50292969)
\curveto(245.1777252,203.28352536)(245.51073268,204.31835506)(245.51074219,205.60742188)
\lineto(245.51074219,206.04785156)
\lineto(243.54492188,206.04785156)
\moveto(247.48730469,206.86425781)
\lineto(247.48730469,200)
\lineto(245.51074219,200)
\lineto(245.51074219,201.82617188)
\curveto(245.05956126,201.09570203)(244.49738734,200.55501247)(243.82421875,200.20410156)
\curveto(243.15103452,199.8603517)(242.32746763,199.68847687)(241.35351562,199.68847656)
\curveto(240.12174067,199.68847687)(239.14062186,200.03222653)(238.41015625,200.71972656)
\curveto(237.68684728,201.41438661)(237.32519399,202.34179453)(237.32519531,203.50195312)
\curveto(237.32519399,204.85546389)(237.77636542,205.87597069)(238.67871094,206.56347656)
\curveto(239.58821256,207.25096931)(240.94172683,207.59471897)(242.73925781,207.59472656)
\lineto(245.51074219,207.59472656)
\lineto(245.51074219,207.78808594)
\curveto(245.51073268,208.69758245)(245.20995173,209.39940466)(244.60839844,209.89355469)
\curveto(244.01398939,210.39484638)(243.1760996,210.64549717)(242.09472656,210.64550781)
\curveto(241.40722116,210.64549717)(240.73762547,210.56314048)(240.0859375,210.3984375)
\curveto(239.43424136,210.23371372)(238.80761438,209.98664366)(238.20605469,209.65722656)
\lineto(238.20605469,211.48339844)
\curveto(238.92935905,211.76268355)(239.63118127,211.97036563)(240.31152344,212.10644531)
\curveto(240.99185699,212.24966223)(241.65429122,212.32127674)(242.29882812,212.32128906)
\curveto(244.03905446,212.32127674)(245.33885785,211.87010532)(246.19824219,210.96777344)
\curveto(247.05760613,210.06541962)(247.4872932,208.69758245)(247.48730469,206.86425781)
}
}
{
\newrgbcolor{curcolor}{0 0 0}
\pscustom[linestyle=none,fillstyle=solid,fillcolor=curcolor]
{
\newpath
\moveto(258.54101562,210.18359375)
\curveto(258.31900159,210.31248969)(258.07551225,210.40558855)(257.81054688,210.46289062)
\curveto(257.55272632,210.52733322)(257.26626827,210.55955975)(256.95117188,210.55957031)
\curveto(255.83397804,210.55955975)(254.9746039,210.19432574)(254.37304688,209.46386719)
\curveto(253.77864155,208.74055116)(253.48144133,207.69856001)(253.48144531,206.33789062)
\lineto(253.48144531,200)
\lineto(251.49414062,200)
\lineto(251.49414062,212.03125)
\lineto(253.48144531,212.03125)
\lineto(253.48144531,210.16210938)
\curveto(253.8968055,210.89256723)(254.43749506,211.4332568)(255.10351562,211.78417969)
\curveto(255.76952498,212.14224046)(256.57876896,212.32127674)(257.53125,212.32128906)
\curveto(257.66730954,212.32127674)(257.81770001,212.31053456)(257.98242188,212.2890625)
\curveto(258.14712677,212.27472731)(258.32974377,212.24966223)(258.53027344,212.21386719)
\lineto(258.54101562,210.18359375)
}
}
{
\newrgbcolor{curcolor}{0 0 0}
\pscustom[linestyle=none,fillstyle=solid,fillcolor=curcolor]
{
\newpath
\moveto(262.59082031,215.44726562)
\lineto(262.59082031,212.03125)
\lineto(266.66210938,212.03125)
\lineto(266.66210938,210.49511719)
\lineto(262.59082031,210.49511719)
\lineto(262.59082031,203.96386719)
\curveto(262.59081628,202.98274441)(262.72330313,202.35253671)(262.98828125,202.07324219)
\curveto(263.26041197,201.79394352)(263.80826298,201.65429522)(264.63183594,201.65429688)
\lineto(266.66210938,201.65429688)
\lineto(266.66210938,200)
\lineto(264.63183594,200)
\curveto(263.10644077,200)(262.05370745,200.28287732)(261.47363281,200.84863281)
\curveto(260.89355236,201.42154806)(260.60351358,202.45995848)(260.60351562,203.96386719)
\lineto(260.60351562,210.49511719)
\lineto(259.15332031,210.49511719)
\lineto(259.15332031,212.03125)
\lineto(260.60351562,212.03125)
\lineto(260.60351562,215.44726562)
\lineto(262.59082031,215.44726562)
}
}
{
\newrgbcolor{curcolor}{0 0 0}
\pscustom[linestyle=none,fillstyle=solid,fillcolor=curcolor]
{
\newpath
\moveto(269.27246094,212.03125)
\lineto(271.24902344,212.03125)
\lineto(271.24902344,200)
\lineto(269.27246094,200)
\lineto(269.27246094,212.03125)
\moveto(269.27246094,216.71484375)
\lineto(271.24902344,216.71484375)
\lineto(271.24902344,214.21191406)
\lineto(269.27246094,214.21191406)
\lineto(269.27246094,216.71484375)
}
}
{
\newrgbcolor{curcolor}{0 0 0}
\pscustom[linestyle=none,fillstyle=solid,fillcolor=curcolor]
{
\newpath
\moveto(285.66503906,206.50976562)
\lineto(285.66503906,205.54296875)
\lineto(276.57714844,205.54296875)
\curveto(276.66308258,204.18228748)(277.07128529,203.14387706)(277.80175781,202.42773438)
\curveto(278.53938278,201.71874828)(279.5634703,201.36425645)(280.87402344,201.36425781)
\curveto(281.63312969,201.36425645)(282.36717843,201.45735531)(283.07617188,201.64355469)
\curveto(283.79230722,201.82975077)(284.50129088,202.10904737)(285.203125,202.48144531)
\lineto(285.203125,200.61230469)
\curveto(284.49412943,200.31152313)(283.76724214,200.08235669)(283.02246094,199.92480469)
\curveto(282.27766029,199.76725284)(281.5221272,199.68847687)(280.75585938,199.68847656)
\curveto(278.83658301,199.68847687)(277.31477463,200.24707007)(276.19042969,201.36425781)
\curveto(275.07324042,202.48144283)(274.51464722,203.99250903)(274.51464844,205.89746094)
\curveto(274.51464722,207.86685411)(275.04459461,209.42805047)(276.10449219,210.58105469)
\curveto(277.17154561,211.7411992)(278.60741657,212.32127674)(280.41210938,212.32128906)
\curveto(282.03059023,212.32127674)(283.30890926,211.79849081)(284.24707031,210.75292969)
\curveto(285.19237092,209.71450851)(285.6650267,208.30012191)(285.66503906,206.50976562)
\moveto(283.68847656,207.08984375)
\curveto(283.67414327,208.17121579)(283.3697816,209.03417065)(282.77539062,209.67871094)
\curveto(282.18814215,210.32323186)(281.40754398,210.64549717)(280.43359375,210.64550781)
\curveto(279.33072314,210.64549717)(278.44628392,210.33397404)(277.78027344,209.7109375)
\curveto(277.12141545,209.08788154)(276.74185854,208.21060377)(276.64160156,207.07910156)
\lineto(283.68847656,207.08984375)
}
}
{
\newrgbcolor{curcolor}{0 0 0}
\pscustom[linestyle=none,fillstyle=solid,fillcolor=curcolor]
{
\newpath
\moveto(112.15917969,166.03808594)
\lineto(115.08105469,166.03808594)
\lineto(122.19238281,152.62109375)
\lineto(122.19238281,166.03808594)
\lineto(124.29785156,166.03808594)
\lineto(124.29785156,150)
\lineto(121.37597656,150)
\lineto(114.26464844,163.41699219)
\lineto(114.26464844,150)
\lineto(112.15917969,150)
\lineto(112.15917969,166.03808594)
}
}
{
\newrgbcolor{curcolor}{0 0 0}
\pscustom[linestyle=none,fillstyle=solid,fillcolor=curcolor]
{
\newpath
\moveto(133.19238281,160.64550781)
\curveto(132.1324813,160.64549717)(131.29459152,160.230133)(130.67871094,159.39941406)
\curveto(130.06282192,158.57583778)(129.75487951,157.44432849)(129.75488281,156.00488281)
\curveto(129.75487951,154.56542512)(130.05924119,153.43033511)(130.66796875,152.59960938)
\curveto(131.28384934,151.77603989)(132.12531985,151.36425645)(133.19238281,151.36425781)
\curveto(134.2451094,151.36425645)(135.07941846,151.77962062)(135.6953125,152.61035156)
\curveto(136.31118806,153.44107729)(136.61913046,154.57258657)(136.61914062,156.00488281)
\curveto(136.61913046,157.43000559)(136.31118806,158.55793415)(135.6953125,159.38867188)
\curveto(135.07941846,160.22655227)(134.2451094,160.64549717)(133.19238281,160.64550781)
\moveto(133.19238281,162.32128906)
\curveto(134.91112436,162.32127674)(136.2610579,161.76268355)(137.2421875,160.64550781)
\curveto(138.22329553,159.52831078)(138.71385493,157.98143733)(138.71386719,156.00488281)
\curveto(138.71385493,154.03547774)(138.22329553,152.48860428)(137.2421875,151.36425781)
\curveto(136.2610579,150.24707007)(134.91112436,149.68847687)(133.19238281,149.68847656)
\curveto(131.46646634,149.68847687)(130.11295207,150.24707007)(129.13183594,151.36425781)
\curveto(128.1578759,152.48860428)(127.67089722,154.03547774)(127.67089844,156.00488281)
\curveto(127.67089722,157.98143733)(128.1578759,159.52831078)(129.13183594,160.64550781)
\curveto(130.11295207,161.76268355)(131.46646634,162.32127674)(133.19238281,162.32128906)
}
}
{
\newrgbcolor{curcolor}{0 0 0}
\pscustom[linestyle=none,fillstyle=solid,fillcolor=curcolor]
{
\newpath
\moveto(151.34667969,159.72167969)
\curveto(151.84080838,160.60968991)(152.4316281,161.26496269)(153.11914062,161.6875)
\curveto(153.80662672,162.11001393)(154.61587071,162.32127674)(155.546875,162.32128906)
\curveto(156.80011331,162.32127674)(157.76690922,161.88084749)(158.44726562,161)
\curveto(159.12758495,160.12629196)(159.46775388,158.88019945)(159.46777344,157.26171875)
\lineto(159.46777344,150)
\lineto(157.48046875,150)
\lineto(157.48046875,157.19726562)
\curveto(157.48045118,158.35025207)(157.27634982,159.20604548)(156.86816406,159.76464844)
\curveto(156.45994438,160.32323186)(155.83689813,160.60252846)(154.99902344,160.60253906)
\curveto(153.97492083,160.60252846)(153.16567684,160.26235953)(152.57128906,159.58203125)
\curveto(151.97687595,158.90168381)(151.67967573,157.97427588)(151.6796875,156.79980469)
\lineto(151.6796875,150)
\lineto(149.69238281,150)
\lineto(149.69238281,157.19726562)
\curveto(149.69237303,158.35741352)(149.48827167,159.21320693)(149.08007812,159.76464844)
\curveto(148.67186623,160.32323186)(148.04165853,160.60252846)(147.18945312,160.60253906)
\curveto(146.17968123,160.60252846)(145.3775987,160.2587788)(144.78320312,159.57128906)
\curveto(144.1887978,158.89094163)(143.89159758,157.96711443)(143.89160156,156.79980469)
\lineto(143.89160156,150)
\lineto(141.90429688,150)
\lineto(141.90429688,162.03125)
\lineto(143.89160156,162.03125)
\lineto(143.89160156,160.16210938)
\curveto(144.342769,160.89972868)(144.88345856,161.44399897)(145.51367188,161.79492188)
\curveto(146.14387397,162.14582119)(146.89224562,162.32127674)(147.75878906,162.32128906)
\curveto(148.63247825,162.32127674)(149.37368845,162.09927175)(149.98242188,161.65527344)
\curveto(150.5982966,161.21125181)(151.05304875,160.5667212)(151.34667969,159.72167969)
}
}
{
\newrgbcolor{curcolor}{0 0 0}
\pscustom[linestyle=none,fillstyle=solid,fillcolor=curcolor]
{
\newpath
\moveto(172.05761719,156.00488281)
\curveto(172.05760648,157.4586514)(171.75682553,158.59732213)(171.15527344,159.42089844)
\curveto(170.56086318,160.25161735)(169.74087702,160.66698152)(168.6953125,160.66699219)
\curveto(167.64973328,160.66698152)(166.8261664,160.25161735)(166.22460938,159.42089844)
\curveto(165.63020405,158.59732213)(165.33300383,157.4586514)(165.33300781,156.00488281)
\curveto(165.33300383,154.55110222)(165.63020405,153.40885076)(166.22460938,152.578125)
\curveto(166.8261664,151.75455554)(167.64973328,151.34277209)(168.6953125,151.34277344)
\curveto(169.74087702,151.34277209)(170.56086318,151.75455554)(171.15527344,152.578125)
\curveto(171.75682553,153.40885076)(172.05760648,154.55110222)(172.05761719,156.00488281)
\moveto(165.33300781,160.20507812)
\curveto(165.748368,160.92121304)(166.27115393,161.45116042)(166.90136719,161.79492188)
\curveto(167.53873079,162.14582119)(168.29784461,162.32127674)(169.17871094,162.32128906)
\curveto(170.63963915,162.32127674)(171.82485931,161.7411992)(172.734375,160.58105469)
\curveto(173.65102936,159.42088902)(174.10936224,157.89549992)(174.109375,156.00488281)
\curveto(174.10936224,154.1142537)(173.65102936,152.5888646)(172.734375,151.42871094)
\curveto(171.82485931,150.26855442)(170.63963915,149.68847687)(169.17871094,149.68847656)
\curveto(168.29784461,149.68847687)(167.53873079,149.8603517)(166.90136719,150.20410156)
\curveto(166.27115393,150.55501247)(165.748368,151.08854058)(165.33300781,151.8046875)
\lineto(165.33300781,150)
\lineto(163.34570312,150)
\lineto(163.34570312,166.71484375)
\lineto(165.33300781,166.71484375)
\lineto(165.33300781,160.20507812)
}
}
{
\newrgbcolor{curcolor}{0 0 0}
\pscustom[linestyle=none,fillstyle=solid,fillcolor=curcolor]
{
\newpath
\moveto(184.35742188,160.18359375)
\curveto(184.13540784,160.31248969)(183.8919185,160.40558855)(183.62695312,160.46289062)
\curveto(183.36913257,160.52733322)(183.08267452,160.55955975)(182.76757812,160.55957031)
\curveto(181.65038429,160.55955975)(180.79101015,160.19432574)(180.18945312,159.46386719)
\curveto(179.5950478,158.74055116)(179.29784758,157.69856001)(179.29785156,156.33789062)
\lineto(179.29785156,150)
\lineto(177.31054688,150)
\lineto(177.31054688,162.03125)
\lineto(179.29785156,162.03125)
\lineto(179.29785156,160.16210938)
\curveto(179.71321175,160.89256723)(180.25390131,161.4332568)(180.91992188,161.78417969)
\curveto(181.58593123,162.14224046)(182.39517521,162.32127674)(183.34765625,162.32128906)
\curveto(183.48371579,162.32127674)(183.63410626,162.31053456)(183.79882812,162.2890625)
\curveto(183.96353302,162.27472731)(184.14615002,162.24966223)(184.34667969,162.21386719)
\lineto(184.35742188,160.18359375)
}
}
{
\newrgbcolor{curcolor}{0 0 0}
\pscustom[linestyle=none,fillstyle=solid,fillcolor=curcolor]
{
\newpath
\moveto(196.27050781,156.50976562)
\lineto(196.27050781,155.54296875)
\lineto(187.18261719,155.54296875)
\curveto(187.26855133,154.18228748)(187.67675404,153.14387706)(188.40722656,152.42773438)
\curveto(189.14485153,151.71874828)(190.16893905,151.36425645)(191.47949219,151.36425781)
\curveto(192.23859844,151.36425645)(192.97264718,151.45735531)(193.68164062,151.64355469)
\curveto(194.39777597,151.82975077)(195.10675963,152.10904737)(195.80859375,152.48144531)
\lineto(195.80859375,150.61230469)
\curveto(195.09959818,150.31152313)(194.37271089,150.08235669)(193.62792969,149.92480469)
\curveto(192.88312904,149.76725284)(192.12759595,149.68847687)(191.36132812,149.68847656)
\curveto(189.44205176,149.68847687)(187.92024338,150.24707007)(186.79589844,151.36425781)
\curveto(185.67870917,152.48144283)(185.12011597,153.99250903)(185.12011719,155.89746094)
\curveto(185.12011597,157.86685411)(185.65006336,159.42805047)(186.70996094,160.58105469)
\curveto(187.77701436,161.7411992)(189.21288532,162.32127674)(191.01757812,162.32128906)
\curveto(192.63605898,162.32127674)(193.91437801,161.79849081)(194.85253906,160.75292969)
\curveto(195.79783967,159.71450851)(196.27049545,158.30012191)(196.27050781,156.50976562)
\moveto(194.29394531,157.08984375)
\curveto(194.27961202,158.17121579)(193.97525035,159.03417065)(193.38085938,159.67871094)
\curveto(192.7936109,160.32323186)(192.01301273,160.64549717)(191.0390625,160.64550781)
\curveto(189.93619189,160.64549717)(189.05175267,160.33397404)(188.38574219,159.7109375)
\curveto(187.7268842,159.08788154)(187.34732729,158.21060377)(187.24707031,157.07910156)
\lineto(194.29394531,157.08984375)
}
}
{
\newrgbcolor{curcolor}{0 0 0}
\pscustom[linestyle=none,fillstyle=solid,fillcolor=curcolor]
{
}
}
{
\newrgbcolor{curcolor}{0 0 0}
\pscustom[linestyle=none,fillstyle=solid,fillcolor=curcolor]
{
\newpath
\moveto(214.43554688,160.20507812)
\lineto(214.43554688,166.71484375)
\lineto(216.41210938,166.71484375)
\lineto(216.41210938,150)
\lineto(214.43554688,150)
\lineto(214.43554688,151.8046875)
\curveto(214.02017272,151.08854058)(213.49380606,150.55501247)(212.85644531,150.20410156)
\curveto(212.2262292,149.8603517)(211.46711537,149.68847687)(210.57910156,149.68847656)
\curveto(209.12532084,149.68847687)(207.94010067,150.26855442)(207.0234375,151.42871094)
\curveto(206.11393062,152.5888646)(205.65917847,154.1142537)(205.65917969,156.00488281)
\curveto(205.65917847,157.89549992)(206.11393062,159.42088902)(207.0234375,160.58105469)
\curveto(207.94010067,161.7411992)(209.12532084,162.32127674)(210.57910156,162.32128906)
\curveto(211.46711537,162.32127674)(212.2262292,162.14582119)(212.85644531,161.79492188)
\curveto(213.49380606,161.45116042)(214.02017272,160.92121304)(214.43554688,160.20507812)
\moveto(207.70019531,156.00488281)
\curveto(207.70019206,154.55110222)(207.99739228,153.40885076)(208.59179688,152.578125)
\curveto(209.19335463,151.75455554)(210.01692151,151.34277209)(211.0625,151.34277344)
\curveto(212.10806525,151.34277209)(212.93163214,151.75455554)(213.53320312,152.578125)
\curveto(214.13475594,153.40885076)(214.43553688,154.55110222)(214.43554688,156.00488281)
\curveto(214.43553688,157.4586514)(214.13475594,158.59732213)(213.53320312,159.42089844)
\curveto(212.93163214,160.25161735)(212.10806525,160.66698152)(211.0625,160.66699219)
\curveto(210.01692151,160.66698152)(209.19335463,160.25161735)(208.59179688,159.42089844)
\curveto(207.99739228,158.59732213)(207.70019206,157.4586514)(207.70019531,156.00488281)
}
}
{
\newrgbcolor{curcolor}{0 0 0}
\pscustom[linestyle=none,fillstyle=solid,fillcolor=curcolor]
{
\newpath
\moveto(222.35253906,166.03808594)
\lineto(222.35253906,160.07617188)
\lineto(220.52636719,160.07617188)
\lineto(220.52636719,166.03808594)
\lineto(222.35253906,166.03808594)
}
}
{
\newrgbcolor{curcolor}{0 0 0}
\pscustom[linestyle=none,fillstyle=solid,fillcolor=curcolor]
{
\newpath
\moveto(236.83300781,156.50976562)
\lineto(236.83300781,155.54296875)
\lineto(227.74511719,155.54296875)
\curveto(227.83105133,154.18228748)(228.23925404,153.14387706)(228.96972656,152.42773438)
\curveto(229.70735153,151.71874828)(230.73143905,151.36425645)(232.04199219,151.36425781)
\curveto(232.80109844,151.36425645)(233.53514718,151.45735531)(234.24414062,151.64355469)
\curveto(234.96027597,151.82975077)(235.66925963,152.10904737)(236.37109375,152.48144531)
\lineto(236.37109375,150.61230469)
\curveto(235.66209818,150.31152313)(234.93521089,150.08235669)(234.19042969,149.92480469)
\curveto(233.44562904,149.76725284)(232.69009595,149.68847687)(231.92382812,149.68847656)
\curveto(230.00455176,149.68847687)(228.48274338,150.24707007)(227.35839844,151.36425781)
\curveto(226.24120917,152.48144283)(225.68261597,153.99250903)(225.68261719,155.89746094)
\curveto(225.68261597,157.86685411)(226.21256336,159.42805047)(227.27246094,160.58105469)
\curveto(228.33951436,161.7411992)(229.77538532,162.32127674)(231.58007812,162.32128906)
\curveto(233.19855898,162.32127674)(234.47687801,161.79849081)(235.41503906,160.75292969)
\curveto(236.36033967,159.71450851)(236.83299545,158.30012191)(236.83300781,156.50976562)
\moveto(234.85644531,157.08984375)
\curveto(234.84211202,158.17121579)(234.53775035,159.03417065)(233.94335938,159.67871094)
\curveto(233.3561109,160.32323186)(232.57551273,160.64549717)(231.6015625,160.64550781)
\curveto(230.49869189,160.64549717)(229.61425267,160.33397404)(228.94824219,159.7109375)
\curveto(228.2893842,159.08788154)(227.90982729,158.21060377)(227.80957031,157.07910156)
\lineto(234.85644531,157.08984375)
}
}
{
\newrgbcolor{curcolor}{0 0 0}
\pscustom[linestyle=none,fillstyle=solid,fillcolor=curcolor]
{
\newpath
\moveto(250.078125,157.26171875)
\lineto(250.078125,150)
\lineto(248.1015625,150)
\lineto(248.1015625,157.19726562)
\curveto(248.1015524,158.33592916)(247.87954742,159.18814185)(247.43554688,159.75390625)
\curveto(246.99152747,160.31965114)(246.32551251,160.60252846)(245.4375,160.60253906)
\curveto(244.37043634,160.60252846)(243.52896583,160.26235953)(242.91308594,159.58203125)
\curveto(242.29719623,158.90168381)(241.98925383,157.97427588)(241.98925781,156.79980469)
\lineto(241.98925781,150)
\lineto(240.00195312,150)
\lineto(240.00195312,162.03125)
\lineto(241.98925781,162.03125)
\lineto(241.98925781,160.16210938)
\curveto(242.4619096,160.88540578)(243.01692207,161.42609534)(243.65429688,161.78417969)
\curveto(244.29882183,162.14224046)(245.04003203,162.32127674)(245.87792969,162.32128906)
\curveto(247.26008189,162.32127674)(248.30565376,161.89158967)(249.01464844,161.03222656)
\curveto(249.72362109,160.18000284)(250.07811293,158.92316816)(250.078125,157.26171875)
}
}
{
\newrgbcolor{curcolor}{0 0 0}
\pscustom[linestyle=none,fillstyle=solid,fillcolor=curcolor]
{
\newpath
\moveto(264.04296875,157.26171875)
\lineto(264.04296875,150)
\lineto(262.06640625,150)
\lineto(262.06640625,157.19726562)
\curveto(262.06639615,158.33592916)(261.84439117,159.18814185)(261.40039062,159.75390625)
\curveto(260.95637122,160.31965114)(260.29035626,160.60252846)(259.40234375,160.60253906)
\curveto(258.33528009,160.60252846)(257.49380958,160.26235953)(256.87792969,159.58203125)
\curveto(256.26203998,158.90168381)(255.95409758,157.97427588)(255.95410156,156.79980469)
\lineto(255.95410156,150)
\lineto(253.96679688,150)
\lineto(253.96679688,162.03125)
\lineto(255.95410156,162.03125)
\lineto(255.95410156,160.16210938)
\curveto(256.42675335,160.88540578)(256.98176582,161.42609534)(257.61914062,161.78417969)
\curveto(258.26366558,162.14224046)(259.00487578,162.32127674)(259.84277344,162.32128906)
\curveto(261.22492564,162.32127674)(262.27049751,161.89158967)(262.97949219,161.03222656)
\curveto(263.68846484,160.18000284)(264.04295668,158.92316816)(264.04296875,157.26171875)
}
}
{
\newrgbcolor{curcolor}{0 0 0}
\pscustom[linestyle=none,fillstyle=solid,fillcolor=curcolor]
{
\newpath
\moveto(278.29785156,156.50976562)
\lineto(278.29785156,155.54296875)
\lineto(269.20996094,155.54296875)
\curveto(269.29589508,154.18228748)(269.70409779,153.14387706)(270.43457031,152.42773438)
\curveto(271.17219528,151.71874828)(272.1962828,151.36425645)(273.50683594,151.36425781)
\curveto(274.26594219,151.36425645)(274.99999093,151.45735531)(275.70898438,151.64355469)
\curveto(276.42511972,151.82975077)(277.13410338,152.10904737)(277.8359375,152.48144531)
\lineto(277.8359375,150.61230469)
\curveto(277.12694193,150.31152313)(276.40005464,150.08235669)(275.65527344,149.92480469)
\curveto(274.91047279,149.76725284)(274.1549397,149.68847687)(273.38867188,149.68847656)
\curveto(271.46939551,149.68847687)(269.94758713,150.24707007)(268.82324219,151.36425781)
\curveto(267.70605292,152.48144283)(267.14745972,153.99250903)(267.14746094,155.89746094)
\curveto(267.14745972,157.86685411)(267.67740711,159.42805047)(268.73730469,160.58105469)
\curveto(269.80435811,161.7411992)(271.24022907,162.32127674)(273.04492188,162.32128906)
\curveto(274.66340273,162.32127674)(275.94172176,161.79849081)(276.87988281,160.75292969)
\curveto(277.82518342,159.71450851)(278.2978392,158.30012191)(278.29785156,156.50976562)
\moveto(276.32128906,157.08984375)
\curveto(276.30695577,158.17121579)(276.0025941,159.03417065)(275.40820312,159.67871094)
\curveto(274.82095465,160.32323186)(274.04035648,160.64549717)(273.06640625,160.64550781)
\curveto(271.96353564,160.64549717)(271.07909642,160.33397404)(270.41308594,159.7109375)
\curveto(269.75422795,159.08788154)(269.37467104,158.21060377)(269.27441406,157.07910156)
\lineto(276.32128906,157.08984375)
}
}
{
\newrgbcolor{curcolor}{0 0 0}
\pscustom[linestyle=none,fillstyle=solid,fillcolor=curcolor]
{
\newpath
\moveto(290.90917969,159.72167969)
\curveto(291.40330838,160.60968991)(291.9941281,161.26496269)(292.68164062,161.6875)
\curveto(293.36912672,162.11001393)(294.17837071,162.32127674)(295.109375,162.32128906)
\curveto(296.36261331,162.32127674)(297.32940922,161.88084749)(298.00976562,161)
\curveto(298.69008495,160.12629196)(299.03025388,158.88019945)(299.03027344,157.26171875)
\lineto(299.03027344,150)
\lineto(297.04296875,150)
\lineto(297.04296875,157.19726562)
\curveto(297.04295118,158.35025207)(296.83884982,159.20604548)(296.43066406,159.76464844)
\curveto(296.02244438,160.32323186)(295.39939813,160.60252846)(294.56152344,160.60253906)
\curveto(293.53742083,160.60252846)(292.72817684,160.26235953)(292.13378906,159.58203125)
\curveto(291.53937595,158.90168381)(291.24217573,157.97427588)(291.2421875,156.79980469)
\lineto(291.2421875,150)
\lineto(289.25488281,150)
\lineto(289.25488281,157.19726562)
\curveto(289.25487303,158.35741352)(289.05077167,159.21320693)(288.64257812,159.76464844)
\curveto(288.23436623,160.32323186)(287.60415853,160.60252846)(286.75195312,160.60253906)
\curveto(285.74218123,160.60252846)(284.9400987,160.2587788)(284.34570312,159.57128906)
\curveto(283.7512978,158.89094163)(283.45409758,157.96711443)(283.45410156,156.79980469)
\lineto(283.45410156,150)
\lineto(281.46679688,150)
\lineto(281.46679688,162.03125)
\lineto(283.45410156,162.03125)
\lineto(283.45410156,160.16210938)
\curveto(283.905269,160.89972868)(284.44595856,161.44399897)(285.07617188,161.79492188)
\curveto(285.70637397,162.14582119)(286.45474562,162.32127674)(287.32128906,162.32128906)
\curveto(288.19497825,162.32127674)(288.93618845,162.09927175)(289.54492188,161.65527344)
\curveto(290.1607966,161.21125181)(290.61554875,160.5667212)(290.90917969,159.72167969)
}
}
{
\newrgbcolor{curcolor}{0 0 0}
\pscustom[linestyle=none,fillstyle=solid,fillcolor=curcolor]
{
\newpath
\moveto(302.98339844,162.03125)
\lineto(304.95996094,162.03125)
\lineto(304.95996094,150)
\lineto(302.98339844,150)
\lineto(302.98339844,162.03125)
\moveto(302.98339844,166.71484375)
\lineto(304.95996094,166.71484375)
\lineto(304.95996094,164.21191406)
\lineto(302.98339844,164.21191406)
\lineto(302.98339844,166.71484375)
}
}
{
\newrgbcolor{curcolor}{0 0 0}
\pscustom[linestyle=none,fillstyle=solid,fillcolor=curcolor]
{
\newpath
\moveto(316.75488281,161.67675781)
\lineto(316.75488281,159.80761719)
\curveto(316.19627988,160.09406543)(315.61620233,160.30890896)(315.01464844,160.45214844)
\curveto(314.41307854,160.59536701)(313.79003228,160.66698152)(313.14550781,160.66699219)
\curveto(312.16438287,160.66698152)(311.4267534,160.51659105)(310.93261719,160.21582031)
\curveto(310.44563459,159.91502915)(310.20214525,159.46385772)(310.20214844,158.86230469)
\curveto(310.20214525,158.40396295)(310.3776008,158.04230967)(310.72851562,157.77734375)
\curveto(311.07942302,157.51952373)(311.78482596,157.27245367)(312.84472656,157.03613281)
\lineto(313.52148438,156.88574219)
\curveto(314.92512229,156.58495435)(315.92056401,156.15884801)(316.5078125,155.60742188)
\curveto(317.10220345,155.06314598)(317.39940367,154.30045143)(317.39941406,153.31933594)
\curveto(317.39940367,152.20214624)(316.9553937,151.31770702)(316.06738281,150.66601562)
\curveto(315.18651526,150.0143229)(313.97264929,149.68847687)(312.42578125,149.68847656)
\curveto(311.78124523,149.68847687)(311.10806882,149.75292993)(310.40625,149.88183594)
\curveto(309.71158584,150.00358073)(308.9775371,150.18977846)(308.20410156,150.44042969)
\lineto(308.20410156,152.48144531)
\curveto(308.93456839,152.10188592)(309.65429423,151.81542787)(310.36328125,151.62207031)
\curveto(311.07226156,151.43587096)(311.77408378,151.34277209)(312.46875,151.34277344)
\curveto(313.3997332,151.34277209)(314.11587831,151.50032402)(314.6171875,151.81542969)
\curveto(315.11848148,152.13769317)(315.36913227,152.5888646)(315.36914062,153.16894531)
\curveto(315.36913227,153.70605098)(315.18651526,154.11783442)(314.82128906,154.40429688)
\curveto(314.46320869,154.69075052)(313.67186834,154.96646639)(312.44726562,155.23144531)
\lineto(311.75976562,155.39257812)
\curveto(310.53515273,155.65038497)(309.65071351,156.04426479)(309.10644531,156.57421875)
\curveto(308.56217293,157.11132101)(308.29003778,157.84536976)(308.29003906,158.77636719)
\curveto(308.29003778,159.9078677)(308.69107905,160.78156474)(309.49316406,161.39746094)
\curveto(310.29524411,162.01333434)(311.43391485,162.32127674)(312.90917969,162.32128906)
\curveto(313.63964181,162.32127674)(314.32714112,162.26756586)(314.97167969,162.16015625)
\curveto(315.61620233,162.05272232)(316.21060278,161.89158967)(316.75488281,161.67675781)
}
}
{
\newrgbcolor{curcolor}{0 0 0}
\pscustom[linestyle=none,fillstyle=solid,fillcolor=curcolor]
{
\newpath
\moveto(165.00097656,124.25488281)
\lineto(165.00097656,111.78320312)
\lineto(167.62207031,111.78320312)
\curveto(169.83495177,111.78320134)(171.45343974,112.28450292)(172.47753906,113.28710938)
\curveto(173.50877623,114.28970925)(174.02440071,115.87238996)(174.02441406,118.03515625)
\curveto(174.02440071,120.18358357)(173.50877623,121.7555221)(172.47753906,122.75097656)
\curveto(171.45343974,123.75356698)(169.83495177,124.25486856)(167.62207031,124.25488281)
\lineto(165.00097656,124.25488281)
\moveto(162.83105469,126.03808594)
\lineto(167.2890625,126.03808594)
\curveto(170.39712569,126.0380699)(172.67804789,125.38995857)(174.13183594,124.09375)
\curveto(175.58559707,122.8046747)(176.31248436,120.78514546)(176.3125,118.03515625)
\curveto(176.31248436,115.27082806)(175.58201634,113.24055666)(174.12109375,111.94433594)
\curveto(172.66014426,110.64811133)(170.38280279,110)(167.2890625,110)
\lineto(162.83105469,110)
\lineto(162.83105469,126.03808594)
}
}
{
\newrgbcolor{curcolor}{0 0 0}
\pscustom[linestyle=none,fillstyle=solid,fillcolor=curcolor]
{
\newpath
\moveto(179.67480469,122.03125)
\lineto(181.65136719,122.03125)
\lineto(181.65136719,110)
\lineto(179.67480469,110)
\lineto(179.67480469,122.03125)
\moveto(179.67480469,126.71484375)
\lineto(181.65136719,126.71484375)
\lineto(181.65136719,124.21191406)
\lineto(179.67480469,124.21191406)
\lineto(179.67480469,126.71484375)
}
}
{
\newrgbcolor{curcolor}{0 0 0}
\pscustom[linestyle=none,fillstyle=solid,fillcolor=curcolor]
{
\newpath
\moveto(191.8671875,126.71484375)
\lineto(191.8671875,125.07128906)
\lineto(189.9765625,125.07128906)
\curveto(189.26757256,125.07127399)(188.77343243,124.92804497)(188.49414062,124.64160156)
\curveto(188.22200069,124.35512887)(188.08593312,123.83950439)(188.0859375,123.09472656)
\lineto(188.0859375,122.03125)
\lineto(193.51074219,122.03125)
\lineto(193.51074219,122.86914062)
\curveto(193.51073238,124.20831913)(193.82225551,125.18227648)(194.4453125,125.79101562)
\curveto(194.64582239,125.99152047)(194.885731,126.15981457)(195.16503906,126.29589844)
\curveto(195.72362079,126.57517874)(196.47199244,126.71482704)(197.41015625,126.71484375)
\lineto(199.27929688,126.71484375)
\lineto(199.27929688,125.07128906)
\lineto(197.38867188,125.07128906)
\curveto(196.67967452,125.07127399)(196.18553439,124.92804497)(195.90625,124.64160156)
\curveto(195.63410265,124.35512887)(195.49803508,123.83950439)(195.49804688,123.09472656)
\lineto(195.49804688,122.03125)
\lineto(202.91015625,122.03125)
\lineto(202.91015625,110)
\lineto(200.92285156,110)
\lineto(200.92285156,120.49511719)
\lineto(195.49804688,120.49511719)
\lineto(195.49804688,110)
\lineto(193.51074219,110)
\lineto(193.51074219,120.49511719)
\lineto(188.0859375,120.49511719)
\lineto(188.0859375,110)
\lineto(186.09863281,110)
\lineto(186.09863281,120.49511719)
\lineto(184.20800781,120.49511719)
\lineto(184.20800781,122.03125)
\lineto(186.09863281,122.03125)
\lineto(186.09863281,122.86914062)
\curveto(186.09863042,124.20831913)(186.41015354,125.18227648)(187.03320312,125.79101562)
\curveto(187.65624605,126.40688463)(188.64452631,126.71482704)(189.99804688,126.71484375)
\lineto(191.8671875,126.71484375)
\moveto(200.92285156,126.69335938)
\lineto(202.91015625,126.69335938)
\lineto(202.91015625,124.19042969)
\lineto(200.92285156,124.19042969)
\lineto(200.92285156,126.69335938)
}
}
{
\newrgbcolor{curcolor}{0 0 0}
\pscustom[linestyle=none,fillstyle=solid,fillcolor=curcolor]
{
\newpath
\moveto(215.70410156,121.56933594)
\lineto(215.70410156,119.72167969)
\curveto(215.14549764,120.02961237)(214.58332372,120.2587788)(214.01757812,120.40917969)
\curveto(213.45897589,120.5667212)(212.89322125,120.64549717)(212.3203125,120.64550781)
\curveto(211.03840539,120.64549717)(210.04296368,120.23729445)(209.33398438,119.42089844)
\curveto(208.62499635,118.61164503)(208.27050451,117.4729743)(208.27050781,116.00488281)
\curveto(208.27050451,114.53677932)(208.62499635,113.39452786)(209.33398438,112.578125)
\curveto(210.04296368,111.76887844)(211.03840539,111.36425645)(212.3203125,111.36425781)
\curveto(212.89322125,111.36425645)(213.45897589,111.43945169)(214.01757812,111.58984375)
\curveto(214.58332372,111.74739409)(215.14549764,111.98014125)(215.70410156,112.28808594)
\lineto(215.70410156,110.46191406)
\curveto(215.15265909,110.20410136)(214.579743,110.01074218)(213.98535156,109.88183594)
\curveto(213.39810355,109.75292993)(212.77147658,109.68847687)(212.10546875,109.68847656)
\curveto(210.29361447,109.68847687)(208.85416279,110.25781224)(207.78710938,111.39648438)
\curveto(206.72005034,112.53515371)(206.18652222,114.07128499)(206.18652344,116.00488281)
\curveto(206.18652222,117.96711443)(206.72363106,119.51040716)(207.79785156,120.63476562)
\curveto(208.87922786,121.75910282)(210.35806753,122.32127674)(212.234375,122.32128906)
\curveto(212.84309109,122.32127674)(213.43749154,122.25682368)(214.01757812,122.12792969)
\curveto(214.59764663,122.00617289)(215.15982054,121.81997516)(215.70410156,121.56933594)
}
}
{
\newrgbcolor{curcolor}{0 0 0}
\pscustom[linestyle=none,fillstyle=solid,fillcolor=curcolor]
{
\newpath
\moveto(218.95898438,114.74804688)
\lineto(218.95898438,122.03125)
\lineto(220.93554688,122.03125)
\lineto(220.93554688,114.82324219)
\curveto(220.93554303,113.68456663)(221.15754802,112.82877321)(221.6015625,112.25585938)
\curveto(222.04556796,111.69010248)(222.71158292,111.40722516)(223.59960938,111.40722656)
\curveto(224.66665909,111.40722516)(225.5081296,111.74739409)(226.12402344,112.42773438)
\curveto(226.74706066,113.10806981)(227.05858378,114.03547774)(227.05859375,115.20996094)
\lineto(227.05859375,122.03125)
\lineto(229.03515625,122.03125)
\lineto(229.03515625,110)
\lineto(227.05859375,110)
\lineto(227.05859375,111.84765625)
\curveto(226.57876655,111.11718638)(226.02017336,110.57291609)(225.3828125,110.21484375)
\curveto(224.7525965,109.86393243)(224.01854776,109.68847687)(223.18066406,109.68847656)
\curveto(221.7984979,109.68847687)(220.7493453,110.11816394)(220.03320312,110.97753906)
\curveto(219.31705506,111.83691223)(218.95898251,113.09374691)(218.95898438,114.74804688)
\moveto(223.93261719,122.32128906)
\lineto(223.93261719,122.32128906)
}
}
{
\newrgbcolor{curcolor}{0 0 0}
\pscustom[linestyle=none,fillstyle=solid,fillcolor=curcolor]
{
\newpath
\moveto(233.12792969,126.71484375)
\lineto(235.10449219,126.71484375)
\lineto(235.10449219,110)
\lineto(233.12792969,110)
\lineto(233.12792969,126.71484375)
}
}
{
\newrgbcolor{curcolor}{0 0 0}
\pscustom[linestyle=none,fillstyle=solid,fillcolor=curcolor]
{
\newpath
\moveto(241.18457031,125.44726562)
\lineto(241.18457031,122.03125)
\lineto(245.25585938,122.03125)
\lineto(245.25585938,120.49511719)
\lineto(241.18457031,120.49511719)
\lineto(241.18457031,113.96386719)
\curveto(241.18456628,112.98274441)(241.31705313,112.35253671)(241.58203125,112.07324219)
\curveto(241.85416197,111.79394352)(242.40201298,111.65429522)(243.22558594,111.65429688)
\lineto(245.25585938,111.65429688)
\lineto(245.25585938,110)
\lineto(243.22558594,110)
\curveto(241.70019077,110)(240.64745745,110.28287732)(240.06738281,110.84863281)
\curveto(239.48730236,111.42154806)(239.19726358,112.45995848)(239.19726562,113.96386719)
\lineto(239.19726562,120.49511719)
\lineto(237.74707031,120.49511719)
\lineto(237.74707031,122.03125)
\lineto(239.19726562,122.03125)
\lineto(239.19726562,125.44726562)
\lineto(241.18457031,125.44726562)
}
}
{
\newrgbcolor{curcolor}{0 0 0}
\pscustom[linestyle=none,fillstyle=solid,fillcolor=curcolor]
{
\newpath
\moveto(258.15722656,116.50976562)
\lineto(258.15722656,115.54296875)
\lineto(249.06933594,115.54296875)
\curveto(249.15527008,114.18228748)(249.56347279,113.14387706)(250.29394531,112.42773438)
\curveto(251.03157028,111.71874828)(252.0556578,111.36425645)(253.36621094,111.36425781)
\curveto(254.12531719,111.36425645)(254.85936593,111.45735531)(255.56835938,111.64355469)
\curveto(256.28449472,111.82975077)(256.99347838,112.10904737)(257.6953125,112.48144531)
\lineto(257.6953125,110.61230469)
\curveto(256.98631693,110.31152313)(256.25942964,110.08235669)(255.51464844,109.92480469)
\curveto(254.76984779,109.76725284)(254.0143147,109.68847687)(253.24804688,109.68847656)
\curveto(251.32877051,109.68847687)(249.80696213,110.24707007)(248.68261719,111.36425781)
\curveto(247.56542792,112.48144283)(247.00683472,113.99250903)(247.00683594,115.89746094)
\curveto(247.00683472,117.86685411)(247.53678211,119.42805047)(248.59667969,120.58105469)
\curveto(249.66373311,121.7411992)(251.09960407,122.32127674)(252.90429688,122.32128906)
\curveto(254.52277773,122.32127674)(255.80109676,121.79849081)(256.73925781,120.75292969)
\curveto(257.68455842,119.71450851)(258.1572142,118.30012191)(258.15722656,116.50976562)
\moveto(256.18066406,117.08984375)
\curveto(256.16633077,118.17121579)(255.8619691,119.03417065)(255.26757812,119.67871094)
\curveto(254.68032965,120.32323186)(253.89973148,120.64549717)(252.92578125,120.64550781)
\curveto(251.82291064,120.64549717)(250.93847142,120.33397404)(250.27246094,119.7109375)
\curveto(249.61360295,119.08788154)(249.23404604,118.21060377)(249.13378906,117.07910156)
\lineto(256.18066406,117.08984375)
\moveto(254.27929688,127.59570312)
\lineto(256.41699219,127.59570312)
\lineto(252.91503906,123.55664062)
\lineto(251.27148438,123.55664062)
\lineto(254.27929688,127.59570312)
}
}
{
\newrgbcolor{curcolor}{0 0 0}
\pscustom[linestyle=none,fillstyle=solid,fillcolor=curcolor]
{
\newpath
\moveto(149.99414062,92.70507812)
\lineto(149.99414062,99.21484375)
\lineto(151.97070312,99.21484375)
\lineto(151.97070312,82.5)
\lineto(149.99414062,82.5)
\lineto(149.99414062,84.3046875)
\curveto(149.57876647,83.58854058)(149.05239981,83.05501247)(148.41503906,82.70410156)
\curveto(147.78482295,82.3603517)(147.02570912,82.18847687)(146.13769531,82.18847656)
\curveto(144.68391459,82.18847687)(143.49869442,82.76855442)(142.58203125,83.92871094)
\curveto(141.67252437,85.0888646)(141.21777222,86.6142537)(141.21777344,88.50488281)
\curveto(141.21777222,90.39549992)(141.67252437,91.92088902)(142.58203125,93.08105469)
\curveto(143.49869442,94.2411992)(144.68391459,94.82127674)(146.13769531,94.82128906)
\curveto(147.02570912,94.82127674)(147.78482295,94.64582119)(148.41503906,94.29492188)
\curveto(149.05239981,93.95116042)(149.57876647,93.42121304)(149.99414062,92.70507812)
\moveto(143.25878906,88.50488281)
\curveto(143.25878581,87.05110222)(143.55598603,85.90885076)(144.15039062,85.078125)
\curveto(144.75194838,84.25455554)(145.57551526,83.84277209)(146.62109375,83.84277344)
\curveto(147.666659,83.84277209)(148.49022589,84.25455554)(149.09179688,85.078125)
\curveto(149.69334969,85.90885076)(149.99413063,87.05110222)(149.99414062,88.50488281)
\curveto(149.99413063,89.9586514)(149.69334969,91.09732213)(149.09179688,91.92089844)
\curveto(148.49022589,92.75161735)(147.666659,93.16698152)(146.62109375,93.16699219)
\curveto(145.57551526,93.16698152)(144.75194838,92.75161735)(144.15039062,91.92089844)
\curveto(143.55598603,91.09732213)(143.25878581,89.9586514)(143.25878906,88.50488281)
}
}
{
\newrgbcolor{curcolor}{0 0 0}
\pscustom[linestyle=none,fillstyle=solid,fillcolor=curcolor]
{
\newpath
\moveto(166.33300781,89.00976562)
\lineto(166.33300781,88.04296875)
\lineto(157.24511719,88.04296875)
\curveto(157.33105133,86.68228748)(157.73925404,85.64387706)(158.46972656,84.92773438)
\curveto(159.20735153,84.21874828)(160.23143905,83.86425645)(161.54199219,83.86425781)
\curveto(162.30109844,83.86425645)(163.03514718,83.95735531)(163.74414062,84.14355469)
\curveto(164.46027597,84.32975077)(165.16925963,84.60904737)(165.87109375,84.98144531)
\lineto(165.87109375,83.11230469)
\curveto(165.16209818,82.81152313)(164.43521089,82.58235669)(163.69042969,82.42480469)
\curveto(162.94562904,82.26725284)(162.19009595,82.18847687)(161.42382812,82.18847656)
\curveto(159.50455176,82.18847687)(157.98274338,82.74707007)(156.85839844,83.86425781)
\curveto(155.74120917,84.98144283)(155.18261597,86.49250903)(155.18261719,88.39746094)
\curveto(155.18261597,90.36685411)(155.71256336,91.92805047)(156.77246094,93.08105469)
\curveto(157.83951436,94.2411992)(159.27538532,94.82127674)(161.08007812,94.82128906)
\curveto(162.69855898,94.82127674)(163.97687801,94.29849081)(164.91503906,93.25292969)
\curveto(165.86033967,92.21450851)(166.33299545,90.80012191)(166.33300781,89.00976562)
\moveto(164.35644531,89.58984375)
\curveto(164.34211202,90.67121579)(164.03775035,91.53417065)(163.44335938,92.17871094)
\curveto(162.8561109,92.82323186)(162.07551273,93.14549717)(161.1015625,93.14550781)
\curveto(159.99869189,93.14549717)(159.11425267,92.83397404)(158.44824219,92.2109375)
\curveto(157.7893842,91.58788154)(157.40982729,90.71060377)(157.30957031,89.57910156)
\lineto(164.35644531,89.58984375)
}
}
{
\newrgbcolor{curcolor}{0 0 0}
\pscustom[linestyle=none,fillstyle=solid,fillcolor=curcolor]
{
\newpath
\moveto(177.24707031,94.17675781)
\lineto(177.24707031,92.30761719)
\curveto(176.68846738,92.59406543)(176.10838983,92.80890896)(175.50683594,92.95214844)
\curveto(174.90526604,93.09536701)(174.28221978,93.16698152)(173.63769531,93.16699219)
\curveto(172.65657037,93.16698152)(171.9189409,93.01659105)(171.42480469,92.71582031)
\curveto(170.93782209,92.41502915)(170.69433275,91.96385772)(170.69433594,91.36230469)
\curveto(170.69433275,90.90396295)(170.8697883,90.54230967)(171.22070312,90.27734375)
\curveto(171.57161052,90.01952373)(172.27701346,89.77245367)(173.33691406,89.53613281)
\lineto(174.01367188,89.38574219)
\curveto(175.41730979,89.08495435)(176.41275151,88.65884801)(177,88.10742188)
\curveto(177.59439095,87.56314598)(177.89159117,86.80045143)(177.89160156,85.81933594)
\curveto(177.89159117,84.70214624)(177.4475812,83.81770702)(176.55957031,83.16601562)
\curveto(175.67870276,82.5143229)(174.46483679,82.18847687)(172.91796875,82.18847656)
\curveto(172.27343273,82.18847687)(171.60025632,82.25292993)(170.8984375,82.38183594)
\curveto(170.20377334,82.50358073)(169.4697246,82.68977846)(168.69628906,82.94042969)
\lineto(168.69628906,84.98144531)
\curveto(169.42675589,84.60188592)(170.14648173,84.31542787)(170.85546875,84.12207031)
\curveto(171.56444906,83.93587096)(172.26627128,83.84277209)(172.9609375,83.84277344)
\curveto(173.8919207,83.84277209)(174.60806581,84.00032402)(175.109375,84.31542969)
\curveto(175.61066898,84.63769317)(175.86131977,85.0888646)(175.86132812,85.66894531)
\curveto(175.86131977,86.20605098)(175.67870276,86.61783442)(175.31347656,86.90429688)
\curveto(174.95539619,87.19075052)(174.16405584,87.46646639)(172.93945312,87.73144531)
\lineto(172.25195312,87.89257812)
\curveto(171.02734023,88.15038497)(170.14290101,88.54426479)(169.59863281,89.07421875)
\curveto(169.05436043,89.61132101)(168.78222528,90.34536976)(168.78222656,91.27636719)
\curveto(168.78222528,92.4078677)(169.18326655,93.28156474)(169.98535156,93.89746094)
\curveto(170.78743161,94.51333434)(171.92610235,94.82127674)(173.40136719,94.82128906)
\curveto(174.13182931,94.82127674)(174.81932862,94.76756586)(175.46386719,94.66015625)
\curveto(176.10838983,94.55272232)(176.70279028,94.39158967)(177.24707031,94.17675781)
}
}
{
\newrgbcolor{curcolor}{0 0 0}
\pscustom[linestyle=none,fillstyle=solid,fillcolor=curcolor]
{
}
}
{
\newrgbcolor{curcolor}{0 0 0}
\pscustom[linestyle=none,fillstyle=solid,fillcolor=curcolor]
{
\newpath
\moveto(198.34472656,89.00976562)
\lineto(198.34472656,88.04296875)
\lineto(189.25683594,88.04296875)
\curveto(189.34277008,86.68228748)(189.75097279,85.64387706)(190.48144531,84.92773438)
\curveto(191.21907028,84.21874828)(192.2431578,83.86425645)(193.55371094,83.86425781)
\curveto(194.31281719,83.86425645)(195.04686593,83.95735531)(195.75585938,84.14355469)
\curveto(196.47199472,84.32975077)(197.18097838,84.60904737)(197.8828125,84.98144531)
\lineto(197.8828125,83.11230469)
\curveto(197.17381693,82.81152313)(196.44692964,82.58235669)(195.70214844,82.42480469)
\curveto(194.95734779,82.26725284)(194.2018147,82.18847687)(193.43554688,82.18847656)
\curveto(191.51627051,82.18847687)(189.99446213,82.74707007)(188.87011719,83.86425781)
\curveto(187.75292792,84.98144283)(187.19433472,86.49250903)(187.19433594,88.39746094)
\curveto(187.19433472,90.36685411)(187.72428211,91.92805047)(188.78417969,93.08105469)
\curveto(189.85123311,94.2411992)(191.28710407,94.82127674)(193.09179688,94.82128906)
\curveto(194.71027773,94.82127674)(195.98859676,94.29849081)(196.92675781,93.25292969)
\curveto(197.87205842,92.21450851)(198.3447142,90.80012191)(198.34472656,89.00976562)
\moveto(196.36816406,89.58984375)
\curveto(196.35383077,90.67121579)(196.0494691,91.53417065)(195.45507812,92.17871094)
\curveto(194.86782965,92.82323186)(194.08723148,93.14549717)(193.11328125,93.14550781)
\curveto(192.01041064,93.14549717)(191.12597142,92.83397404)(190.45996094,92.2109375)
\curveto(189.80110295,91.58788154)(189.42154604,90.71060377)(189.32128906,89.57910156)
\lineto(196.36816406,89.58984375)
}
}
{
\newrgbcolor{curcolor}{0 0 0}
\pscustom[linestyle=none,fillstyle=solid,fillcolor=curcolor]
{
\newpath
\moveto(211.58984375,89.76171875)
\lineto(211.58984375,82.5)
\lineto(209.61328125,82.5)
\lineto(209.61328125,89.69726562)
\curveto(209.61327115,90.83592916)(209.39126617,91.68814185)(208.94726562,92.25390625)
\curveto(208.50324622,92.81965114)(207.83723126,93.10252846)(206.94921875,93.10253906)
\curveto(205.88215509,93.10252846)(205.04068458,92.76235953)(204.42480469,92.08203125)
\curveto(203.80891498,91.40168381)(203.50097258,90.47427588)(203.50097656,89.29980469)
\lineto(203.50097656,82.5)
\lineto(201.51367188,82.5)
\lineto(201.51367188,94.53125)
\lineto(203.50097656,94.53125)
\lineto(203.50097656,92.66210938)
\curveto(203.97362835,93.38540578)(204.52864082,93.92609534)(205.16601562,94.28417969)
\curveto(205.81054058,94.64224046)(206.55175078,94.82127674)(207.38964844,94.82128906)
\curveto(208.77180064,94.82127674)(209.81737251,94.39158967)(210.52636719,93.53222656)
\curveto(211.23533984,92.68000284)(211.58983168,91.42316816)(211.58984375,89.76171875)
}
}
{
\newrgbcolor{curcolor}{0 0 0}
\pscustom[linestyle=none,fillstyle=solid,fillcolor=curcolor]
{
\newpath
\moveto(225.5546875,89.76171875)
\lineto(225.5546875,82.5)
\lineto(223.578125,82.5)
\lineto(223.578125,89.69726562)
\curveto(223.5781149,90.83592916)(223.35610992,91.68814185)(222.91210938,92.25390625)
\curveto(222.46808997,92.81965114)(221.80207501,93.10252846)(220.9140625,93.10253906)
\curveto(219.84699884,93.10252846)(219.00552833,92.76235953)(218.38964844,92.08203125)
\curveto(217.77375873,91.40168381)(217.46581633,90.47427588)(217.46582031,89.29980469)
\lineto(217.46582031,82.5)
\lineto(215.47851562,82.5)
\lineto(215.47851562,94.53125)
\lineto(217.46582031,94.53125)
\lineto(217.46582031,92.66210938)
\curveto(217.9384721,93.38540578)(218.49348457,93.92609534)(219.13085938,94.28417969)
\curveto(219.77538433,94.64224046)(220.51659453,94.82127674)(221.35449219,94.82128906)
\curveto(222.73664439,94.82127674)(223.78221626,94.39158967)(224.49121094,93.53222656)
\curveto(225.20018359,92.68000284)(225.55467543,91.42316816)(225.5546875,89.76171875)
}
}
{
\newrgbcolor{curcolor}{0 0 0}
\pscustom[linestyle=none,fillstyle=solid,fillcolor=curcolor]
{
\newpath
\moveto(239.80957031,89.00976562)
\lineto(239.80957031,88.04296875)
\lineto(230.72167969,88.04296875)
\curveto(230.80761383,86.68228748)(231.21581654,85.64387706)(231.94628906,84.92773438)
\curveto(232.68391403,84.21874828)(233.70800155,83.86425645)(235.01855469,83.86425781)
\curveto(235.77766094,83.86425645)(236.51170968,83.95735531)(237.22070312,84.14355469)
\curveto(237.93683847,84.32975077)(238.64582213,84.60904737)(239.34765625,84.98144531)
\lineto(239.34765625,83.11230469)
\curveto(238.63866068,82.81152313)(237.91177339,82.58235669)(237.16699219,82.42480469)
\curveto(236.42219154,82.26725284)(235.66665845,82.18847687)(234.90039062,82.18847656)
\curveto(232.98111426,82.18847687)(231.45930588,82.74707007)(230.33496094,83.86425781)
\curveto(229.21777167,84.98144283)(228.65917847,86.49250903)(228.65917969,88.39746094)
\curveto(228.65917847,90.36685411)(229.18912586,91.92805047)(230.24902344,93.08105469)
\curveto(231.31607686,94.2411992)(232.75194782,94.82127674)(234.55664062,94.82128906)
\curveto(236.17512148,94.82127674)(237.45344051,94.29849081)(238.39160156,93.25292969)
\curveto(239.33690217,92.21450851)(239.80955795,90.80012191)(239.80957031,89.00976562)
\moveto(237.83300781,89.58984375)
\curveto(237.81867452,90.67121579)(237.51431285,91.53417065)(236.91992188,92.17871094)
\curveto(236.3326734,92.82323186)(235.55207523,93.14549717)(234.578125,93.14550781)
\curveto(233.47525439,93.14549717)(232.59081517,92.83397404)(231.92480469,92.2109375)
\curveto(231.2659467,91.58788154)(230.88638979,90.71060377)(230.78613281,89.57910156)
\lineto(237.83300781,89.58984375)
}
}
{
\newrgbcolor{curcolor}{0 0 0}
\pscustom[linestyle=none,fillstyle=solid,fillcolor=curcolor]
{
\newpath
\moveto(252.42089844,92.22167969)
\curveto(252.91502713,93.10968991)(253.50584685,93.76496269)(254.19335938,94.1875)
\curveto(254.88084547,94.61001393)(255.69008946,94.82127674)(256.62109375,94.82128906)
\curveto(257.87433206,94.82127674)(258.84112797,94.38084749)(259.52148438,93.5)
\curveto(260.2018037,92.62629196)(260.54197263,91.38019945)(260.54199219,89.76171875)
\lineto(260.54199219,82.5)
\lineto(258.5546875,82.5)
\lineto(258.5546875,89.69726562)
\curveto(258.55466993,90.85025207)(258.35056857,91.70604548)(257.94238281,92.26464844)
\curveto(257.53416313,92.82323186)(256.91111688,93.10252846)(256.07324219,93.10253906)
\curveto(255.04913958,93.10252846)(254.23989559,92.76235953)(253.64550781,92.08203125)
\curveto(253.0510947,91.40168381)(252.75389448,90.47427588)(252.75390625,89.29980469)
\lineto(252.75390625,82.5)
\lineto(250.76660156,82.5)
\lineto(250.76660156,89.69726562)
\curveto(250.76659178,90.85741352)(250.56249042,91.71320693)(250.15429688,92.26464844)
\curveto(249.74608498,92.82323186)(249.11587728,93.10252846)(248.26367188,93.10253906)
\curveto(247.25389998,93.10252846)(246.45181745,92.7587788)(245.85742188,92.07128906)
\curveto(245.26301655,91.39094163)(244.96581633,90.46711443)(244.96582031,89.29980469)
\lineto(244.96582031,82.5)
\lineto(242.97851562,82.5)
\lineto(242.97851562,94.53125)
\lineto(244.96582031,94.53125)
\lineto(244.96582031,92.66210938)
\curveto(245.41698775,93.39972868)(245.95767731,93.94399897)(246.58789062,94.29492188)
\curveto(247.21809272,94.64582119)(247.96646437,94.82127674)(248.83300781,94.82128906)
\curveto(249.706697,94.82127674)(250.4479072,94.59927175)(251.05664062,94.15527344)
\curveto(251.67251535,93.71125181)(252.1272675,93.0667212)(252.42089844,92.22167969)
}
}
{
\newrgbcolor{curcolor}{0 0 0}
\pscustom[linestyle=none,fillstyle=solid,fillcolor=curcolor]
{
\newpath
\moveto(264.49511719,94.53125)
\lineto(266.47167969,94.53125)
\lineto(266.47167969,82.5)
\lineto(264.49511719,82.5)
\lineto(264.49511719,94.53125)
\moveto(264.49511719,99.21484375)
\lineto(266.47167969,99.21484375)
\lineto(266.47167969,96.71191406)
\lineto(264.49511719,96.71191406)
\lineto(264.49511719,99.21484375)
}
}
{
\newrgbcolor{curcolor}{0 0 0}
\pscustom[linestyle=none,fillstyle=solid,fillcolor=curcolor]
{
\newpath
\moveto(278.26660156,94.17675781)
\lineto(278.26660156,92.30761719)
\curveto(277.70799863,92.59406543)(277.12792108,92.80890896)(276.52636719,92.95214844)
\curveto(275.92479729,93.09536701)(275.30175103,93.16698152)(274.65722656,93.16699219)
\curveto(273.67610162,93.16698152)(272.93847215,93.01659105)(272.44433594,92.71582031)
\curveto(271.95735334,92.41502915)(271.713864,91.96385772)(271.71386719,91.36230469)
\curveto(271.713864,90.90396295)(271.88931955,90.54230967)(272.24023438,90.27734375)
\curveto(272.59114177,90.01952373)(273.29654471,89.77245367)(274.35644531,89.53613281)
\lineto(275.03320312,89.38574219)
\curveto(276.43684104,89.08495435)(277.43228276,88.65884801)(278.01953125,88.10742188)
\curveto(278.6139222,87.56314598)(278.91112242,86.80045143)(278.91113281,85.81933594)
\curveto(278.91112242,84.70214624)(278.46711245,83.81770702)(277.57910156,83.16601562)
\curveto(276.69823401,82.5143229)(275.48436804,82.18847687)(273.9375,82.18847656)
\curveto(273.29296398,82.18847687)(272.61978757,82.25292993)(271.91796875,82.38183594)
\curveto(271.22330459,82.50358073)(270.48925585,82.68977846)(269.71582031,82.94042969)
\lineto(269.71582031,84.98144531)
\curveto(270.44628714,84.60188592)(271.16601298,84.31542787)(271.875,84.12207031)
\curveto(272.58398031,83.93587096)(273.28580253,83.84277209)(273.98046875,83.84277344)
\curveto(274.91145195,83.84277209)(275.62759706,84.00032402)(276.12890625,84.31542969)
\curveto(276.63020023,84.63769317)(276.88085102,85.0888646)(276.88085938,85.66894531)
\curveto(276.88085102,86.20605098)(276.69823401,86.61783442)(276.33300781,86.90429688)
\curveto(275.97492744,87.19075052)(275.18358709,87.46646639)(273.95898438,87.73144531)
\lineto(273.27148438,87.89257812)
\curveto(272.04687148,88.15038497)(271.16243226,88.54426479)(270.61816406,89.07421875)
\curveto(270.07389168,89.61132101)(269.80175653,90.34536976)(269.80175781,91.27636719)
\curveto(269.80175653,92.4078677)(270.2027978,93.28156474)(271.00488281,93.89746094)
\curveto(271.80696286,94.51333434)(272.9456336,94.82127674)(274.42089844,94.82128906)
\curveto(275.15136056,94.82127674)(275.83885987,94.76756586)(276.48339844,94.66015625)
\curveto(277.12792108,94.55272232)(277.72232153,94.39158967)(278.26660156,94.17675781)
}
}
\end{pspicture}

		\end{center}
		
		\begin{enumerate}
		  \item Fond d'écran (présent dans l'archive stocké sur le téléphone)
		  \item Champs de texte ``Nom de la partie''
		  \item Carte sélectionnée
		  \item Carte précédente
		  \item Carte suivante
		  \item Nom de la carte sélectionnée
		  \item Liste déroulante ``Type de partie''
		  \item Liste déroulante ``Nombre d'ennemis''
		  \item Liste déroulante ``Difficulté des ennemis''
		  \item Bouton ``\hyperlink{Accueil multi-joueurs}{Retour}''
		  \item Bouton ``Lancer'' 
		\end{enumerate}

		\subsubsection{Description des zones}
		
			\begin{tabular}{|c|c|c|c|c|} \hline
				Numéro de zone & Type  & Description & Evènement &	Règle \\\hline
				2 & Champs de texte & Nom de la partie désiré & Perte du focus & RG7-01 \\\hline
				3,4,5 & Sélecteur & Permet de sélectionner la carte voulue & Cliqué & RG7-02 \\\hline
				6 & Label & Affiche le nom de la carte choisie & RG7-02 & RG7-03 \\\hline
				7 & Liste déroulante & Permet de choisir le type de partie & Cliqué & RG7-04 \\\hline
				8 & Liste déroulante & Permet de choisir le nombre d'ennemis & Cliqué & RG7-05 \\
				  &                  & (1, 2 ou 3) & & \\\hline
				9 & Liste      & Permet de choisir la difficulté  & Cliqué & RG7-06 \\
				  & déroulante & de l'intelligence artificielle   &        & \\
				  &            & (facile/moyenne/difficile) & & \\\hline				  
				10& Bouton & Affiche l'écran d'accueil & Cliqué & RG7-07 \\\hline
				11& Bouton & Lance une partie selon & Cliqué & RG7-08 \\
				  &        & les paramètres choisis &        & \\\hline
			\end{tabular}
			
		\subsubsection{Description des règles}

			\underline{RG7-02 :}
				\begin{quote}
					Si le champs est vide alors 
					\begin{quote}
						Afficher l'erreur.
					\end{quote}
				\end{quote}
				

			\underline{RG7-02 :}
				\begin{quote}
					Décaler les cartes vers la droite ou la gauche.\\
					RG3-03
				\end{quote}
				
			$\,$				
				
			\underline{RG7-03 :}
				\begin{quote}
					Afficher le nom de la carte choisie.
				\end{quote}

			$\,$

			\underline{RG7-04 :}
				\begin{quote}
					Dérouler la liste.
				\end{quote}

			$\,$

			\underline{RG7-05 :}
				\begin{quote}
					Dérouler la liste.			
				\end{quote}	
				
			$\,$

			\underline{RG7-06 :}
				\begin{quote}
					Dérouler la liste.		
				\end{quote}

			$\,$

			\underline{RG7-07 :}
				\begin{quote}
					Afficher la page d'accueil%
						\footnote[1]{
							\hyperlink{Page d'accueil}{Page d'accueil}
							\og voir section \ref{Accueil}, page \pageref{Accueil}.\fg
						}.\\
					Supprimer la page de création d'une partie multi-joueurs.
				\end{quote}
				
			$\,$

			\underline{RG7-08 :}
				\begin{quote}
					 Connexion au serveur distant.\\
					 Créer la partie.\\
					 Lancer la partie.\\
					 Supprimer la page de création d'une partie solitaire.
				\end{quote}
				
			$\,$
	 
\newpage

	\subsection{Options}
	
		\hypertarget{Options}{}
		\label{Options}
	
		\begin{center}
			\input{./tex/8_Options}
		\end{center}
				
		\begin{enumerate}
		  \item Fond d'écran (présent dans l'archive stocké sur le téléphone)
		  \item Barre de volume
		  \item Check box ``Muet''
		  \item Liste déroulante ``Langues''
		  \item Bouton ``\hyperlink{Gestion du profil}{Gestion du profil}''
		  \item Bouton ``\hyperlink{Accueil}{Retour}''
		\end{enumerate}

		\subsubsection{Description des zones}
		
			\begin{tabular}{|c|c|c|c|c|} \hline
				Numéro de zone & Type  & Description & Evènement &	Règle \\\hline
				2 & & Barre de volume & Cliqué & RG8-01 \\\hline
				3 & Check box & Permet de passer le volume à muet & Cliqué & RG8-02 \\\hline
				4 & Liste déroulante & Permet de choisir la langue voulu pour le jeu & Perte du focus & RG8-03 \\\hline 
				5 & Bouton & Ouvre la page de gestion de profil & Cliqué & RG8-04 \\\hline 
				6 & Bouton & Permet de revenir à l'accueil principal & Cliqué & RG8-05 \\\hline
			\end{tabular}
			
		\subsubsection{Description des règles}

			\underline{RG8-01 :}
				\begin{quote}
					Enregistrer le volume selectionné dans le fichier de paramètre.\\
				\end{quote}
				

			\underline{RG8-02 :}
				\begin{quote}
					Enregistrer le volume à 0 dans le fichier paramètre.\\
				\end{quote}
				

			\underline{RG8-03 :}
				\begin{quote}
					Si la langue a changé alors
					\begin{quote}
						Appliquer la nouvelle langue dans le fichier de paramètre.
					\end{quote}
				\end{quote}


			\underline{RG8-04 :}
				\begin{quote}
					Recharger la classe utilisateur.\\
					Charger la page de gestion du profil%
						\footnote[1]{
							\hyperlink{Gestion du profil}{Gestion du profil}
							\og voir section \ref{Gestion du profil}, page \pageref{Gestion du profil}.\fg
						}.\\
					Afficher la page de gestion du profil\footnotemark[1].\\
					Supprimer la page des options%
						\footnote[2]{
							\hyperlink{Options}{Options}
							\og voir section \ref{Options}, page \pageref{Options}.\fg
						}.\\						
				\end{quote}
				
				
			\underline{RG8-05 :}
				\begin{quote}
					Recharger la classe utilisateur.\\
					Charger la page d'accueil%
						\footnote[3]{
							\hyperlink{Page d'accueil}{Page d'accueil}
							\og voir section \ref{Accueil}, page \pageref{Accueil}.\fg
						}.\\
					Afficher la page d'accueil\footnotemark[3].\\
					Supprimer la page des options%
						\footnote[2]{
							\hyperlink{Options}{Options}
							\og voir section \ref{Options}, page \pageref{Options}.\fg
						}.\\		
				\end{quote}
\newpage

	\subsection{Gestion du profil}
	
		\hypertarget{Gestion du profil}{}
		\label{Gestion du profil}
		
		\begin{center}
			%LaTeX with PSTricks extensions
%%Creator: inkscape 0.48.0
%%Please note this file requires PSTricks extensions
\psset{xunit=.5pt,yunit=.5pt,runit=.5pt}
\begin{pspicture}(560,600)
{
\newrgbcolor{curcolor}{1 1 1}
\pscustom[linestyle=none,fillstyle=solid,fillcolor=curcolor]
{
\newpath
\moveto(133.12401581,597.52220317)
\lineto(426.87598419,597.52220317)
\curveto(443.85397169,597.52220317)(457.52217102,583.85400385)(457.52217102,566.87601635)
\lineto(457.52217102,33.12401744)
\curveto(457.52217102,16.14602994)(443.85397169,2.47783062)(426.87598419,2.47783062)
\lineto(133.12401581,2.47783062)
\curveto(116.14602831,2.47783062)(102.47782898,16.14602994)(102.47782898,33.12401744)
\lineto(102.47782898,566.87601635)
\curveto(102.47782898,583.85400385)(116.14602831,597.52220317)(133.12401581,597.52220317)
\closepath
}
}
{
\newrgbcolor{curcolor}{0 0 0}
\pscustom[linewidth=4.95566034,linecolor=curcolor]
{
\newpath
\moveto(133.12401581,597.52220317)
\lineto(426.87598419,597.52220317)
\curveto(443.85397169,597.52220317)(457.52217102,583.85400385)(457.52217102,566.87601635)
\lineto(457.52217102,33.12401744)
\curveto(457.52217102,16.14602994)(443.85397169,2.47783062)(426.87598419,2.47783062)
\lineto(133.12401581,2.47783062)
\curveto(116.14602831,2.47783062)(102.47782898,16.14602994)(102.47782898,33.12401744)
\lineto(102.47782898,566.87601635)
\curveto(102.47782898,583.85400385)(116.14602831,597.52220317)(133.12401581,597.52220317)
\closepath
}
}
{
\newrgbcolor{curcolor}{1 1 1}
\pscustom[linestyle=none,fillstyle=solid,fillcolor=curcolor]
{
\newpath
\moveto(133.82258224,500.11460277)
\lineto(426.10676956,500.11460277)
\curveto(433.79505596,500.11460277)(439.98454285,493.92511589)(439.98454285,486.23682949)
\lineto(439.98454285,383.9630334)
\curveto(439.98454285,376.274747)(433.79505596,370.08526012)(426.10676956,370.08526012)
\lineto(133.82258224,370.08526012)
\curveto(126.13429585,370.08526012)(119.94480896,376.274747)(119.94480896,383.9630334)
\lineto(119.94480896,486.23682949)
\curveto(119.94480896,493.92511589)(126.13429585,500.11460277)(133.82258224,500.11460277)
\closepath
}
}
{
\newrgbcolor{curcolor}{0 0 0}
\pscustom[linewidth=2.2812984,linecolor=curcolor]
{
\newpath
\moveto(133.82258224,500.11460277)
\lineto(426.10676956,500.11460277)
\curveto(433.79505596,500.11460277)(439.98454285,493.92511589)(439.98454285,486.23682949)
\lineto(439.98454285,383.9630334)
\curveto(439.98454285,376.274747)(433.79505596,370.08526012)(426.10676956,370.08526012)
\lineto(133.82258224,370.08526012)
\curveto(126.13429585,370.08526012)(119.94480896,376.274747)(119.94480896,383.9630334)
\lineto(119.94480896,486.23682949)
\curveto(119.94480896,493.92511589)(126.13429585,500.11460277)(133.82258224,500.11460277)
\closepath
}
}
{
\newrgbcolor{curcolor}{1 1 1}
\pscustom[linestyle=none,fillstyle=solid,fillcolor=curcolor]
{
\newpath
\moveto(139.94768143,330.00000164)
\lineto(414.99142075,330.00000164)
\curveto(427.40815835,330.00000164)(437.4043045,320.00385548)(437.4043045,307.58711788)
\lineto(437.4043045,142.41287014)
\curveto(437.4043045,129.99613253)(427.40815835,119.99998638)(414.99142075,119.99998638)
\lineto(139.94768143,119.99998638)
\curveto(127.53094382,119.99998638)(117.53479767,129.99613253)(117.53479767,142.41287014)
\lineto(117.53479767,307.58711788)
\curveto(117.53479767,320.00385548)(127.53094382,330.00000164)(139.94768143,330.00000164)
\closepath
}
}
{
\newrgbcolor{curcolor}{0 0 0}
\pscustom[linewidth=2.29226828,linecolor=curcolor]
{
\newpath
\moveto(139.94768143,330.00000164)
\lineto(414.99142075,330.00000164)
\curveto(427.40815835,330.00000164)(437.4043045,320.00385548)(437.4043045,307.58711788)
\lineto(437.4043045,142.41287014)
\curveto(437.4043045,129.99613253)(427.40815835,119.99998638)(414.99142075,119.99998638)
\lineto(139.94768143,119.99998638)
\curveto(127.53094382,119.99998638)(117.53479767,129.99613253)(117.53479767,142.41287014)
\lineto(117.53479767,307.58711788)
\curveto(117.53479767,320.00385548)(127.53094382,330.00000164)(139.94768143,330.00000164)
\closepath
}
}
{
\newrgbcolor{curcolor}{0 0 0}
\pscustom[linestyle=none,fillstyle=solid,fillcolor=curcolor]
{
\newpath
\moveto(130.1920166,350.49772808)
\lineto(134.30725098,350.49772808)
\lineto(139.51623535,336.60710308)
\lineto(144.75256348,350.49772808)
\lineto(148.86779785,350.49772808)
\lineto(148.86779785,330.0856187)
\lineto(146.17443848,330.0856187)
\lineto(146.17443848,348.00944683)
\lineto(140.9107666,334.00944683)
\lineto(138.13537598,334.00944683)
\lineto(132.8717041,348.00944683)
\lineto(132.8717041,330.0856187)
\lineto(130.1920166,330.0856187)
\lineto(130.1920166,350.49772808)
}
}
{
\newrgbcolor{curcolor}{0 0 0}
\pscustom[linestyle=none,fillstyle=solid,fillcolor=curcolor]
{
\newpath
\moveto(153.99475098,336.12858745)
\lineto(153.99475098,345.3981187)
\lineto(156.51037598,345.3981187)
\lineto(156.51037598,336.22429058)
\curveto(156.51037108,334.77506714)(156.79292288,333.68587552)(157.35803223,332.95671245)
\curveto(157.92313009,332.23665822)(158.77078549,331.87663254)(159.90100098,331.87663433)
\curveto(161.25906425,331.87663254)(162.33002672,332.30957481)(163.1138916,333.17546245)
\curveto(163.90684806,334.04134391)(164.30333204,335.22168127)(164.30334473,336.71647808)
\lineto(164.30334473,345.3981187)
\lineto(166.81896973,345.3981187)
\lineto(166.81896973,330.0856187)
\lineto(164.30334473,330.0856187)
\lineto(164.30334473,332.4371812)
\curveto(163.69265557,331.50749228)(162.98171878,330.81478464)(162.17053223,330.3590562)
\curveto(161.36843914,329.91244179)(160.43419528,329.68913472)(159.36779785,329.68913433)
\curveto(157.60867728,329.68913472)(156.27339215,330.23600918)(155.36193848,331.32975933)
\curveto(154.45047731,332.42350699)(153.9947486,334.02311476)(153.99475098,336.12858745)
\moveto(160.3248291,345.76725933)
\lineto(160.3248291,345.76725933)
}
}
{
\newrgbcolor{curcolor}{0 0 0}
\pscustom[linestyle=none,fillstyle=solid,fillcolor=curcolor]
{
\newpath
\moveto(172.0279541,351.3590562)
\lineto(174.5435791,351.3590562)
\lineto(174.5435791,330.0856187)
\lineto(172.0279541,330.0856187)
\lineto(172.0279541,351.3590562)
}
}
{
\newrgbcolor{curcolor}{0 0 0}
\pscustom[linestyle=none,fillstyle=solid,fillcolor=curcolor]
{
\newpath
\moveto(182.28186035,349.74577495)
\lineto(182.28186035,345.3981187)
\lineto(187.46350098,345.3981187)
\lineto(187.46350098,343.44304058)
\lineto(182.28186035,343.44304058)
\lineto(182.28186035,335.13054058)
\curveto(182.28185522,333.88183886)(182.45047485,333.07975633)(182.78771973,332.72429058)
\curveto(183.13406791,332.36881954)(183.83133284,332.19108535)(184.8795166,332.19108745)
\lineto(187.46350098,332.19108745)
\lineto(187.46350098,330.0856187)
\lineto(184.8795166,330.0856187)
\curveto(182.93810457,330.0856187)(181.59826216,330.44564438)(180.85998535,331.16569683)
\curveto(180.12170113,331.89486168)(179.75256088,333.21647495)(179.75256348,335.13054058)
\lineto(179.75256348,343.44304058)
\lineto(177.90686035,343.44304058)
\lineto(177.90686035,345.3981187)
\lineto(179.75256348,345.3981187)
\lineto(179.75256348,349.74577495)
\lineto(182.28186035,349.74577495)
}
}
{
\newrgbcolor{curcolor}{0 0 0}
\pscustom[linestyle=none,fillstyle=solid,fillcolor=curcolor]
{
\newpath
\moveto(190.7857666,345.3981187)
\lineto(193.3013916,345.3981187)
\lineto(193.3013916,330.0856187)
\lineto(190.7857666,330.0856187)
\lineto(190.7857666,345.3981187)
\moveto(190.7857666,351.3590562)
\lineto(193.3013916,351.3590562)
\lineto(193.3013916,348.17350933)
\lineto(190.7857666,348.17350933)
\lineto(190.7857666,351.3590562)
}
}
{
\newrgbcolor{curcolor}{0 0 0}
\pscustom[linestyle=none,fillstyle=solid,fillcolor=curcolor]
{
\newpath
\moveto(198.5513916,345.3981187)
\lineto(201.0670166,345.3981187)
\lineto(201.0670166,329.8121812)
\curveto(201.06701145,327.86166259)(200.6933139,326.44890359)(199.94592285,325.57389995)
\curveto(199.20763831,324.69890534)(198.01362908,324.26140578)(196.3638916,324.26139995)
\lineto(195.40686035,324.26139995)
\lineto(195.40686035,326.39421245)
\lineto(196.07678223,326.39421245)
\curveto(197.03381236,326.39421614)(197.68550441,326.61752321)(198.03186035,327.06413433)
\curveto(198.37821205,327.50163691)(198.55138896,328.41765162)(198.5513916,329.8121812)
\lineto(198.5513916,345.3981187)
\moveto(198.5513916,351.3590562)
\lineto(201.0670166,351.3590562)
\lineto(201.0670166,348.17350933)
\lineto(198.5513916,348.17350933)
\lineto(198.5513916,351.3590562)
}
}
{
\newrgbcolor{curcolor}{0 0 0}
\pscustom[linestyle=none,fillstyle=solid,fillcolor=curcolor]
{
\newpath
\moveto(212.25061035,343.63444683)
\curveto(210.90164479,343.63443328)(209.83523961,343.10578797)(209.0513916,342.04850933)
\curveto(208.26753285,341.00032133)(207.87560615,339.5602186)(207.87561035,337.72819683)
\curveto(207.87560615,335.89615977)(208.26297556,334.45149975)(209.03771973,333.39421245)
\curveto(209.82156775,332.34603311)(210.89253022,331.82194509)(212.25061035,331.82194683)
\curveto(213.59044419,331.82194509)(214.65229209,332.35059039)(215.43615723,333.40788433)
\curveto(216.21999885,334.46517161)(216.61192554,335.90527434)(216.61193848,337.72819683)
\curveto(216.61192554,339.54198945)(216.21999885,340.97753489)(215.43615723,342.03483745)
\curveto(214.65229209,343.10123069)(213.59044419,343.63443328)(212.25061035,343.63444683)
\moveto(212.25061035,345.76725933)
\curveto(214.43809959,345.76724364)(216.15619683,345.05630686)(217.40490723,343.63444683)
\curveto(218.65359017,342.2125597)(219.2779385,340.24381167)(219.2779541,337.72819683)
\curveto(219.2779385,335.22168127)(218.65359017,333.25293324)(217.40490723,331.82194683)
\curveto(216.15619683,330.40007151)(214.43809959,329.68913472)(212.25061035,329.68913433)
\curveto(210.05398939,329.68913472)(208.33133487,330.40007151)(207.0826416,331.82194683)
\curveto(205.8430561,333.25293324)(205.22326506,335.22168127)(205.2232666,337.72819683)
\curveto(205.22326506,340.24381167)(205.8430561,342.2125597)(207.0826416,343.63444683)
\curveto(208.33133487,345.05630686)(210.05398939,345.76724364)(212.25061035,345.76725933)
}
}
{
\newrgbcolor{curcolor}{0 0 0}
\pscustom[linestyle=none,fillstyle=solid,fillcolor=curcolor]
{
\newpath
\moveto(223.17443848,336.12858745)
\lineto(223.17443848,345.3981187)
\lineto(225.69006348,345.3981187)
\lineto(225.69006348,336.22429058)
\curveto(225.69005858,334.77506714)(225.97261038,333.68587552)(226.53771973,332.95671245)
\curveto(227.10281759,332.23665822)(227.95047299,331.87663254)(229.08068848,331.87663433)
\curveto(230.43875175,331.87663254)(231.50971422,332.30957481)(232.2935791,333.17546245)
\curveto(233.08653556,334.04134391)(233.48301954,335.22168127)(233.48303223,336.71647808)
\lineto(233.48303223,345.3981187)
\lineto(235.99865723,345.3981187)
\lineto(235.99865723,330.0856187)
\lineto(233.48303223,330.0856187)
\lineto(233.48303223,332.4371812)
\curveto(232.87234307,331.50749228)(232.16140628,330.81478464)(231.35021973,330.3590562)
\curveto(230.54812664,329.91244179)(229.61388278,329.68913472)(228.54748535,329.68913433)
\curveto(226.78836478,329.68913472)(225.45307965,330.23600918)(224.54162598,331.32975933)
\curveto(223.63016481,332.42350699)(223.1744361,334.02311476)(223.17443848,336.12858745)
\moveto(229.5045166,345.76725933)
\lineto(229.5045166,345.76725933)
}
}
{
\newrgbcolor{curcolor}{0 0 0}
\pscustom[linestyle=none,fillstyle=solid,fillcolor=curcolor]
{
\newpath
\moveto(254.30529785,338.37077495)
\lineto(254.30529785,337.1403062)
\lineto(242.7388916,337.1403062)
\curveto(242.84826232,335.40853004)(243.36779305,334.08691678)(244.29748535,333.17546245)
\curveto(245.23628077,332.27311651)(246.53966488,331.82194509)(248.2076416,331.82194683)
\curveto(249.17377683,331.82194509)(250.10802069,331.94043455)(251.01037598,332.17741558)
\curveto(251.92182096,332.41439241)(252.8241638,332.76986081)(253.71740723,333.24382183)
\lineto(253.71740723,330.86491558)
\curveto(252.81504923,330.48210268)(251.88991995,330.1904363)(250.9420166,329.98991558)
\curveto(249.99408851,329.78939504)(249.03250093,329.68913472)(248.05725098,329.68913433)
\curveto(245.6145356,329.68913472)(243.67768858,330.40007151)(242.2467041,331.82194683)
\curveto(240.82482685,333.24381867)(240.11389006,335.16699383)(240.1138916,337.59147808)
\curveto(240.11389006,340.09797848)(240.78836855,342.08495566)(242.1373291,343.55241558)
\curveto(243.49539709,345.02896313)(245.32286922,345.76724364)(247.61975098,345.76725933)
\curveto(249.6796357,345.76724364)(251.3065872,345.10187973)(252.50061035,343.77116558)
\curveto(253.70372022,342.44953863)(254.30528212,340.64941022)(254.30529785,338.37077495)
\moveto(251.78967285,339.1090562)
\curveto(251.77143048,340.48534788)(251.38406108,341.58365408)(250.62756348,342.40397808)
\curveto(249.88015633,343.22427744)(248.88666774,343.63443328)(247.64709473,343.63444683)
\curveto(246.24344122,343.63443328)(245.1177913,343.2379493)(244.2701416,342.4449937)
\curveto(243.43159507,341.65201338)(242.94852264,340.53547804)(242.82092285,339.09538433)
\lineto(251.78967285,339.1090562)
}
}
{
\newrgbcolor{curcolor}{0 0 0}
\pscustom[linestyle=none,fillstyle=solid,fillcolor=curcolor]
{
\newpath
\moveto(258.17443848,336.12858745)
\lineto(258.17443848,345.3981187)
\lineto(260.69006348,345.3981187)
\lineto(260.69006348,336.22429058)
\curveto(260.69005858,334.77506714)(260.97261038,333.68587552)(261.53771973,332.95671245)
\curveto(262.10281759,332.23665822)(262.95047299,331.87663254)(264.08068848,331.87663433)
\curveto(265.43875175,331.87663254)(266.50971422,332.30957481)(267.2935791,333.17546245)
\curveto(268.08653556,334.04134391)(268.48301954,335.22168127)(268.48303223,336.71647808)
\lineto(268.48303223,345.3981187)
\lineto(270.99865723,345.3981187)
\lineto(270.99865723,330.0856187)
\lineto(268.48303223,330.0856187)
\lineto(268.48303223,332.4371812)
\curveto(267.87234307,331.50749228)(267.16140628,330.81478464)(266.35021973,330.3590562)
\curveto(265.54812664,329.91244179)(264.61388278,329.68913472)(263.54748535,329.68913433)
\curveto(261.78836478,329.68913472)(260.45307965,330.23600918)(259.54162598,331.32975933)
\curveto(258.63016481,332.42350699)(258.1744361,334.02311476)(258.17443848,336.12858745)
\moveto(264.5045166,345.76725933)
\lineto(264.5045166,345.76725933)
}
}
{
\newrgbcolor{curcolor}{0 0 0}
\pscustom[linestyle=none,fillstyle=solid,fillcolor=curcolor]
{
\newpath
\moveto(285.08068848,343.0465562)
\curveto(284.79812516,343.21060558)(284.48822964,343.32909504)(284.15100098,343.40202495)
\curveto(283.82286572,343.4840428)(283.45828275,343.52505839)(283.05725098,343.52507183)
\curveto(281.63536791,343.52505839)(280.541619,343.0602151)(279.77600098,342.13054058)
\curveto(279.01948511,341.20995654)(278.64123028,339.88378599)(278.64123535,338.15202495)
\lineto(278.64123535,330.0856187)
\lineto(276.11193848,330.0856187)
\lineto(276.11193848,345.3981187)
\lineto(278.64123535,345.3981187)
\lineto(278.64123535,343.01921245)
\curveto(279.16987558,343.94888609)(279.85802594,344.63703644)(280.70568848,345.08366558)
\curveto(281.55333674,345.53937929)(282.58328363,345.76724364)(283.79553223,345.76725933)
\curveto(283.96869891,345.76724364)(284.16010497,345.75357178)(284.36975098,345.7262437)
\curveto(284.57937538,345.70799891)(284.81179703,345.6760979)(285.0670166,345.63054058)
\lineto(285.08068848,343.0465562)
}
}
{
\newrgbcolor{curcolor}{0 0 0}
\pscustom[linestyle=none,fillstyle=solid,fillcolor=curcolor,opacity=0.11935484]
{
\newpath
\moveto(295.13745689,310.08526775)
\lineto(412.38874245,310.08526775)
\curveto(420.78921357,310.08526775)(427.55204773,303.32243359)(427.55204773,294.92196247)
\curveto(427.55204773,286.52149134)(420.78921357,279.75865718)(412.38874245,279.75865718)
\lineto(295.13745689,279.75865718)
\curveto(286.73698577,279.75865718)(279.97415161,286.52149134)(279.97415161,294.92196247)
\curveto(279.97415161,303.32243359)(286.73698577,310.08526775)(295.13745689,310.08526775)
\closepath
}
}
{
\newrgbcolor{curcolor}{0 0 0}
\pscustom[linewidth=2,linecolor=curcolor]
{
\newpath
\moveto(295.13745689,310.08526775)
\lineto(412.38874245,310.08526775)
\curveto(420.78921357,310.08526775)(427.55204773,303.32243359)(427.55204773,294.92196247)
\curveto(427.55204773,286.52149134)(420.78921357,279.75865718)(412.38874245,279.75865718)
\lineto(295.13745689,279.75865718)
\curveto(286.73698577,279.75865718)(279.97415161,286.52149134)(279.97415161,294.92196247)
\curveto(279.97415161,303.32243359)(286.73698577,310.08526775)(295.13745689,310.08526775)
\closepath
}
}
{
\newrgbcolor{curcolor}{0 0 0}
\pscustom[linestyle=none,fillstyle=solid,fillcolor=curcolor,opacity=0.11935484]
{
\newpath
\moveto(254.73094368,489.95401546)
\lineto(415.03449821,489.95401546)
\curveto(423.36245423,489.95401546)(430.06690979,483.2495599)(430.06690979,474.92160388)
\curveto(430.06690979,466.59364787)(423.36245423,459.88919231)(415.03449821,459.88919231)
\lineto(254.73094368,459.88919231)
\curveto(246.40298767,459.88919231)(239.6985321,466.59364787)(239.6985321,474.92160388)
\curveto(239.6985321,483.2495599)(246.40298767,489.95401546)(254.73094368,489.95401546)
\closepath
}
}
{
\newrgbcolor{curcolor}{0 0 0}
\pscustom[linewidth=2,linecolor=curcolor]
{
\newpath
\moveto(254.73094368,489.95401546)
\lineto(415.03449821,489.95401546)
\curveto(423.36245423,489.95401546)(430.06690979,483.2495599)(430.06690979,474.92160388)
\curveto(430.06690979,466.59364787)(423.36245423,459.88919231)(415.03449821,459.88919231)
\lineto(254.73094368,459.88919231)
\curveto(246.40298767,459.88919231)(239.6985321,466.59364787)(239.6985321,474.92160388)
\curveto(239.6985321,483.2495599)(246.40298767,489.95401546)(254.73094368,489.95401546)
\closepath
}
}
{
\newrgbcolor{curcolor}{1 1 1}
\pscustom[linestyle=none,fillstyle=solid,fillcolor=curcolor]
{
\newpath
\moveto(141.63620281,49.84626934)
\lineto(228.64879322,49.84626934)
\curveto(235.04154532,49.84626934)(240.18805695,44.69975771)(240.18805695,38.30700561)
\lineto(240.18805695,31.46732971)
\curveto(240.18805695,25.0745776)(235.04154532,19.92806598)(228.64879322,19.92806598)
\lineto(141.63620281,19.92806598)
\curveto(135.24345071,19.92806598)(130.09693909,25.0745776)(130.09693909,31.46732971)
\lineto(130.09693909,38.30700561)
\curveto(130.09693909,44.69975771)(135.24345071,49.84626934)(141.63620281,49.84626934)
\closepath
}
}
{
\newrgbcolor{curcolor}{0 0 0}
\pscustom[linewidth=2,linecolor=curcolor]
{
\newpath
\moveto(141.63620281,49.84626934)
\lineto(228.64879322,49.84626934)
\curveto(235.04154532,49.84626934)(240.18805695,44.69975771)(240.18805695,38.30700561)
\lineto(240.18805695,31.46732971)
\curveto(240.18805695,25.0745776)(235.04154532,19.92806598)(228.64879322,19.92806598)
\lineto(141.63620281,19.92806598)
\curveto(135.24345071,19.92806598)(130.09693909,25.0745776)(130.09693909,31.46732971)
\lineto(130.09693909,38.30700561)
\curveto(130.09693909,44.69975771)(135.24345071,49.84626934)(141.63620281,49.84626934)
\closepath
}
}
{
\newrgbcolor{curcolor}{0 0 0}
\pscustom[linestyle=none,fillstyle=solid,fillcolor=curcolor]
{
\newpath
\moveto(160.65234375,38.20311138)
\curveto(161.16014509,38.03122835)(161.6523321,37.66404121)(162.12890625,37.10154888)
\curveto(162.61326864,36.53904234)(163.09764315,35.76560561)(163.58203125,34.78123638)
\lineto(165.984375,29.99998638)
\lineto(163.44140625,29.99998638)
\lineto(161.203125,34.48826763)
\curveto(160.62498938,35.66013697)(160.06248994,36.43747994)(159.515625,36.82029888)
\curveto(158.97655352,37.20310417)(158.23827301,37.39451023)(157.30078125,37.39451763)
\lineto(154.72265625,37.39451763)
\lineto(154.72265625,29.99998638)
\lineto(152.35546875,29.99998638)
\lineto(152.35546875,47.49608013)
\lineto(157.69921875,47.49608013)
\curveto(159.69920905,47.49606263)(161.19139506,47.0780943)(162.17578125,46.24217388)
\curveto(163.16014309,45.40622097)(163.6523301,44.14450348)(163.65234375,42.45701763)
\curveto(163.6523301,41.35544377)(163.39451786,40.44138219)(162.87890625,39.71483013)
\curveto(162.37108138,38.98825864)(161.62889462,38.48435289)(160.65234375,38.20311138)
\moveto(154.72265625,45.55076763)
\lineto(154.72265625,39.33983013)
\lineto(157.69921875,39.33983013)
\curveto(158.83983491,39.33982079)(159.69920905,39.60153928)(160.27734375,40.12498638)
\curveto(160.86327039,40.65622572)(161.15623884,41.43356869)(161.15625,42.45701763)
\curveto(161.15623884,43.48044165)(160.86327039,44.24997213)(160.27734375,44.76561138)
\curveto(159.69920905,45.28903359)(158.83983491,45.55075208)(157.69921875,45.55076763)
\lineto(154.72265625,45.55076763)
}
}
{
\newrgbcolor{curcolor}{0 0 0}
\pscustom[linestyle=none,fillstyle=solid,fillcolor=curcolor]
{
\newpath
\moveto(179.09765625,37.10154888)
\lineto(179.09765625,36.04686138)
\lineto(169.18359375,36.04686138)
\curveto(169.27734008,34.56248181)(169.72265214,33.42967045)(170.51953125,32.64842388)
\curveto(171.32421304,31.8749845)(172.44139942,31.48826614)(173.87109375,31.48826763)
\curveto(174.69920966,31.48826614)(175.49999011,31.58982854)(176.2734375,31.79295513)
\curveto(177.05467605,31.99607813)(177.82811278,32.30076533)(178.59375,32.70701763)
\lineto(178.59375,30.66795513)
\curveto(177.82030029,30.33982979)(177.02733233,30.08983004)(176.21484375,29.91795513)
\curveto(175.40233396,29.74608038)(174.57811603,29.66014297)(173.7421875,29.66014263)
\curveto(171.64843146,29.66014297)(169.98827687,30.26951736)(168.76171875,31.48826763)
\curveto(167.54296682,32.70701492)(166.93359243,34.35545077)(166.93359375,36.43358013)
\curveto(166.93359243,38.58200904)(167.51171685,40.28513234)(168.66796875,41.54295513)
\curveto(169.83202703,42.80856732)(171.39843171,43.44137919)(173.3671875,43.44139263)
\curveto(175.13280298,43.44137919)(176.52733283,42.87106726)(177.55078125,41.73045513)
\curveto(178.58201828,40.59763203)(179.09764276,39.05466482)(179.09765625,37.10154888)
\moveto(176.94140625,37.73436138)
\curveto(176.92576993,38.91403996)(176.59373902,39.85544527)(175.9453125,40.55858013)
\curveto(175.3046778,41.26169387)(174.45311616,41.61325601)(173.390625,41.61326763)
\curveto(172.18749342,41.61325601)(171.22265064,41.2734126)(170.49609375,40.59373638)
\curveto(169.77733958,39.91403896)(169.3632775,38.95700867)(169.25390625,37.72264263)
\lineto(176.94140625,37.73436138)
}
}
{
\newrgbcolor{curcolor}{0 0 0}
\pscustom[linestyle=none,fillstyle=solid,fillcolor=curcolor]
{
\newpath
\moveto(184.76953125,46.85154888)
\lineto(184.76953125,43.12498638)
\lineto(189.2109375,43.12498638)
\lineto(189.2109375,41.44920513)
\lineto(184.76953125,41.44920513)
\lineto(184.76953125,34.32420513)
\curveto(184.76952686,33.25388937)(184.91405796,32.56639006)(185.203125,32.26170513)
\curveto(185.49999487,31.95701567)(186.09765053,31.80467207)(186.99609375,31.80467388)
\lineto(189.2109375,31.80467388)
\lineto(189.2109375,29.99998638)
\lineto(186.99609375,29.99998638)
\curveto(185.33202629,29.99998638)(184.18358994,30.30857982)(183.55078125,30.92576763)
\curveto(182.91796621,31.55076608)(182.60156027,32.68357744)(182.6015625,34.32420513)
\lineto(182.6015625,41.44920513)
\lineto(181.01953125,41.44920513)
\lineto(181.01953125,43.12498638)
\lineto(182.6015625,43.12498638)
\lineto(182.6015625,46.85154888)
\lineto(184.76953125,46.85154888)
}
}
{
\newrgbcolor{curcolor}{0 0 0}
\pscustom[linestyle=none,fillstyle=solid,fillcolor=curcolor]
{
\newpath
\moveto(197.14453125,41.61326763)
\curveto(195.98827506,41.61325601)(195.07421347,41.16013147)(194.40234375,40.25389263)
\curveto(193.73046482,39.35544577)(193.39452765,38.12107201)(193.39453125,36.55076763)
\curveto(193.39452765,34.98045015)(193.72655857,33.74217013)(194.390625,32.83592388)
\curveto(195.06249473,31.93748444)(195.98046257,31.48826614)(197.14453125,31.48826763)
\curveto(198.29296025,31.48826614)(199.20311559,31.94139069)(199.875,32.84764263)
\curveto(200.54686425,33.75388887)(200.88280141,34.98826264)(200.8828125,36.55076763)
\curveto(200.88280141,38.10544702)(200.54686425,39.33591454)(199.875,40.24217388)
\curveto(199.20311559,41.15622522)(198.29296025,41.61325601)(197.14453125,41.61326763)
\moveto(197.14453125,43.44139263)
\curveto(199.01952203,43.44137919)(200.4921768,42.83200479)(201.5625,41.61326763)
\curveto(202.63279966,40.39450723)(203.16795538,38.70700892)(203.16796875,36.55076763)
\curveto(203.16795538,34.40232572)(202.63279966,32.71482741)(201.5625,31.48826763)
\curveto(200.4921768,30.26951736)(199.01952203,29.66014297)(197.14453125,29.66014263)
\curveto(195.26171329,29.66014297)(193.78515226,30.26951736)(192.71484375,31.48826763)
\curveto(191.65234189,32.71482741)(191.12109243,34.40232572)(191.12109375,36.55076763)
\curveto(191.12109243,38.70700892)(191.65234189,40.39450723)(192.71484375,41.61326763)
\curveto(193.78515226,42.83200479)(195.26171329,43.44137919)(197.14453125,43.44139263)
}
}
{
\newrgbcolor{curcolor}{0 0 0}
\pscustom[linestyle=none,fillstyle=solid,fillcolor=curcolor]
{
\newpath
\moveto(206.5078125,35.17967388)
\lineto(206.5078125,43.12498638)
\lineto(208.6640625,43.12498638)
\lineto(208.6640625,35.26170513)
\curveto(208.6640583,34.01951361)(208.90624556,33.08592079)(209.390625,32.46092388)
\curveto(209.87499459,31.84373453)(210.60155637,31.53514109)(211.5703125,31.53514263)
\curveto(212.73436673,31.53514109)(213.65233457,31.90623447)(214.32421875,32.64842388)
\curveto(215.00389571,33.39060799)(215.34373913,34.40232572)(215.34375,35.68358013)
\lineto(215.34375,43.12498638)
\lineto(217.5,43.12498638)
\lineto(217.5,29.99998638)
\lineto(215.34375,29.99998638)
\lineto(215.34375,32.01561138)
\curveto(214.82030215,31.21873516)(214.21092776,30.62498575)(213.515625,30.23436138)
\curveto(212.82811664,29.85154903)(212.02733619,29.66014297)(211.11328125,29.66014263)
\curveto(209.60546361,29.66014297)(208.46093351,30.1288925)(207.6796875,31.06639263)
\curveto(206.89843507,32.00389062)(206.50781046,33.374983)(206.5078125,35.17967388)
\moveto(211.93359375,43.44139263)
\lineto(211.93359375,43.44139263)
}
}
{
\newrgbcolor{curcolor}{0 0 0}
\pscustom[linestyle=none,fillstyle=solid,fillcolor=curcolor]
{
\newpath
\moveto(229.5703125,41.10936138)
\curveto(229.32811538,41.24997513)(229.06249064,41.35153753)(228.7734375,41.41404888)
\curveto(228.49217871,41.48434989)(228.17967902,41.51950611)(227.8359375,41.51951763)
\curveto(226.61718059,41.51950611)(225.67968152,41.12106901)(225.0234375,40.32420513)
\curveto(224.37499533,39.53513309)(224.0507769,38.39841548)(224.05078125,36.91404888)
\lineto(224.05078125,29.99998638)
\lineto(221.8828125,29.99998638)
\lineto(221.8828125,43.12498638)
\lineto(224.05078125,43.12498638)
\lineto(224.05078125,41.08592388)
\curveto(224.50390145,41.88278699)(225.09374461,42.47263015)(225.8203125,42.85545513)
\curveto(226.54686816,43.24606688)(227.42967977,43.44137919)(228.46875,43.44139263)
\curveto(228.61717859,43.44137919)(228.78124092,43.42966045)(228.9609375,43.40623638)
\curveto(229.14061556,43.39059799)(229.33983411,43.36325426)(229.55859375,43.32420513)
\lineto(229.5703125,41.10936138)
}
}
{
\newrgbcolor{curcolor}{1 1 1}
\pscustom[linestyle=none,fillstyle=solid,fillcolor=curcolor]
{
\newpath
\moveto(331.97983837,49.87309429)
\lineto(428.58592701,49.87309429)
\curveto(434.99059414,49.87309429)(440.146698,44.71699043)(440.146698,38.3123233)
\lineto(440.146698,31.48006507)
\curveto(440.146698,25.07539795)(434.99059414,19.91929409)(428.58592701,19.91929409)
\lineto(331.97983837,19.91929409)
\curveto(325.57517124,19.91929409)(320.41906738,25.07539795)(320.41906738,31.48006507)
\lineto(320.41906738,38.3123233)
\curveto(320.41906738,44.71699043)(325.57517124,49.87309429)(331.97983837,49.87309429)
\closepath
}
}
{
\newrgbcolor{curcolor}{0 0 0}
\pscustom[linewidth=2,linecolor=curcolor]
{
\newpath
\moveto(331.97983837,49.87309429)
\lineto(428.58592701,49.87309429)
\curveto(434.99059414,49.87309429)(440.146698,44.71699043)(440.146698,38.3123233)
\lineto(440.146698,31.48006507)
\curveto(440.146698,25.07539795)(434.99059414,19.91929409)(428.58592701,19.91929409)
\lineto(331.97983837,19.91929409)
\curveto(325.57517124,19.91929409)(320.41906738,25.07539795)(320.41906738,31.48006507)
\lineto(320.41906738,38.3123233)
\curveto(320.41906738,44.71699043)(325.57517124,49.87309429)(331.97983837,49.87309429)
\closepath
}
}
{
\newrgbcolor{curcolor}{0 0 0}
\pscustom[linestyle=none,fillstyle=solid,fillcolor=curcolor]
{
\newpath
\moveto(346.8671875,29.99998638)
\lineto(340.1875,47.49608013)
\lineto(342.66015625,47.49608013)
\lineto(348.203125,32.76561138)
\lineto(353.7578125,47.49608013)
\lineto(356.21875,47.49608013)
\lineto(349.55078125,29.99998638)
\lineto(346.8671875,29.99998638)
}
}
{
\newrgbcolor{curcolor}{0 0 0}
\pscustom[linestyle=none,fillstyle=solid,fillcolor=curcolor]
{
\newpath
\moveto(362.7578125,36.59764263)
\curveto(361.01561852,36.59763603)(359.80858847,36.39841748)(359.13671875,35.99998638)
\curveto(358.46483982,35.60154328)(358.12890265,34.92185646)(358.12890625,33.96092388)
\curveto(358.12890265,33.19529568)(358.3789024,32.58592129)(358.87890625,32.13279888)
\curveto(359.38671389,31.68748469)(360.07421321,31.46482866)(360.94140625,31.46483013)
\curveto(362.13671114,31.46482866)(363.09374144,31.88670324)(363.8125,32.73045513)
\curveto(364.53905249,33.58201404)(364.90233338,34.71091917)(364.90234375,36.11717388)
\lineto(364.90234375,36.59764263)
\lineto(362.7578125,36.59764263)
\moveto(367.05859375,37.48826763)
\lineto(367.05859375,29.99998638)
\lineto(364.90234375,29.99998638)
\lineto(364.90234375,31.99217388)
\curveto(364.41014637,31.19529768)(363.79686573,30.60545452)(363.0625,30.22264263)
\curveto(362.3281172,29.84764278)(361.4296806,29.66014297)(360.3671875,29.66014263)
\curveto(359.02343301,29.66014297)(357.95312158,30.03514259)(357.15625,30.78514263)
\curveto(356.36718566,31.54295358)(355.97265481,32.55467132)(355.97265625,33.82029888)
\curveto(355.97265481,35.29685608)(356.46484182,36.41013622)(357.44921875,37.16014263)
\curveto(358.44140234,37.91013472)(359.91796336,38.28513434)(361.87890625,38.28514263)
\lineto(364.90234375,38.28514263)
\lineto(364.90234375,38.49608013)
\curveto(364.90233338,39.48825814)(364.57420871,40.25388237)(363.91796875,40.79295513)
\curveto(363.26952251,41.33981879)(362.35546093,41.61325601)(361.17578125,41.61326763)
\curveto(360.42577536,41.61325601)(359.69530734,41.52341235)(358.984375,41.34373638)
\curveto(358.27343376,41.16403771)(357.58984069,40.89450673)(356.93359375,40.53514263)
\lineto(356.93359375,42.52733013)
\curveto(357.72265306,42.83200479)(358.48827729,43.05856707)(359.23046875,43.20701763)
\curveto(359.97265081,43.36325426)(360.69530634,43.44137919)(361.3984375,43.44139263)
\curveto(363.29686623,43.44137919)(364.71483357,42.94919218)(365.65234375,41.96483013)
\curveto(366.58983169,40.98044415)(367.05858122,39.48825814)(367.05859375,37.48826763)
}
}
{
\newrgbcolor{curcolor}{0 0 0}
\pscustom[linestyle=none,fillstyle=solid,fillcolor=curcolor]
{
\newpath
\moveto(371.51171875,48.23436138)
\lineto(373.66796875,48.23436138)
\lineto(373.66796875,29.99998638)
\lineto(371.51171875,29.99998638)
\lineto(371.51171875,48.23436138)
}
}
{
\newrgbcolor{curcolor}{0 0 0}
\pscustom[linestyle=none,fillstyle=solid,fillcolor=curcolor]
{
\newpath
\moveto(378.16796875,43.12498638)
\lineto(380.32421875,43.12498638)
\lineto(380.32421875,29.99998638)
\lineto(378.16796875,29.99998638)
\lineto(378.16796875,43.12498638)
\moveto(378.16796875,48.23436138)
\lineto(380.32421875,48.23436138)
\lineto(380.32421875,45.50389263)
\lineto(378.16796875,45.50389263)
\lineto(378.16796875,48.23436138)
}
}
{
\newrgbcolor{curcolor}{0 0 0}
\pscustom[linestyle=none,fillstyle=solid,fillcolor=curcolor]
{
\newpath
\moveto(393.4609375,41.13279888)
\lineto(393.4609375,48.23436138)
\lineto(395.6171875,48.23436138)
\lineto(395.6171875,29.99998638)
\lineto(393.4609375,29.99998638)
\lineto(393.4609375,31.96873638)
\curveto(393.00780205,31.18748519)(392.43358388,30.60545452)(391.73828125,30.22264263)
\curveto(391.05077276,29.84764278)(390.22264859,29.66014297)(389.25390625,29.66014263)
\curveto(387.66796364,29.66014297)(386.37499619,30.29295483)(385.375,31.55858013)
\curveto(384.38281068,32.8242023)(383.88671743,34.48826314)(383.88671875,36.55076763)
\curveto(383.88671743,38.61325901)(384.38281068,40.27731985)(385.375,41.54295513)
\curveto(386.37499619,42.80856732)(387.66796364,43.44137919)(389.25390625,43.44139263)
\curveto(390.22264859,43.44137919)(391.05077276,43.24997313)(391.73828125,42.86717388)
\curveto(392.43358388,42.49216138)(393.00780205,41.91403696)(393.4609375,41.13279888)
\moveto(386.11328125,36.55076763)
\curveto(386.1132777,34.96482516)(386.43749612,33.71873266)(387.0859375,32.81248638)
\curveto(387.74218232,31.91404696)(388.64061892,31.46482866)(389.78125,31.46483013)
\curveto(390.92186664,31.46482866)(391.82030324,31.91404696)(392.4765625,32.81248638)
\curveto(393.13280193,33.71873266)(393.4609266,34.96482516)(393.4609375,36.55076763)
\curveto(393.4609266,38.13669699)(393.13280193,39.37888325)(392.4765625,40.27733013)
\curveto(391.82030324,41.18356894)(390.92186664,41.63669349)(389.78125,41.63670513)
\curveto(388.64061892,41.63669349)(387.74218232,41.18356894)(387.0859375,40.27733013)
\curveto(386.43749612,39.37888325)(386.1132777,38.13669699)(386.11328125,36.55076763)
}
}
{
\newrgbcolor{curcolor}{0 0 0}
\pscustom[linestyle=none,fillstyle=solid,fillcolor=curcolor]
{
\newpath
\moveto(411.28515625,37.10154888)
\lineto(411.28515625,36.04686138)
\lineto(401.37109375,36.04686138)
\curveto(401.46484008,34.56248181)(401.91015214,33.42967045)(402.70703125,32.64842388)
\curveto(403.51171304,31.8749845)(404.62889942,31.48826614)(406.05859375,31.48826763)
\curveto(406.88670966,31.48826614)(407.68749011,31.58982854)(408.4609375,31.79295513)
\curveto(409.24217605,31.99607813)(410.01561278,32.30076533)(410.78125,32.70701763)
\lineto(410.78125,30.66795513)
\curveto(410.00780029,30.33982979)(409.21483233,30.08983004)(408.40234375,29.91795513)
\curveto(407.58983396,29.74608038)(406.76561603,29.66014297)(405.9296875,29.66014263)
\curveto(403.83593146,29.66014297)(402.17577687,30.26951736)(400.94921875,31.48826763)
\curveto(399.73046682,32.70701492)(399.12109243,34.35545077)(399.12109375,36.43358013)
\curveto(399.12109243,38.58200904)(399.69921685,40.28513234)(400.85546875,41.54295513)
\curveto(402.01952703,42.80856732)(403.58593171,43.44137919)(405.5546875,43.44139263)
\curveto(407.32030298,43.44137919)(408.71483283,42.87106726)(409.73828125,41.73045513)
\curveto(410.76951828,40.59763203)(411.28514276,39.05466482)(411.28515625,37.10154888)
\moveto(409.12890625,37.73436138)
\curveto(409.11326993,38.91403996)(408.78123902,39.85544527)(408.1328125,40.55858013)
\curveto(407.4921778,41.26169387)(406.64061616,41.61325601)(405.578125,41.61326763)
\curveto(404.37499342,41.61325601)(403.41015064,41.2734126)(402.68359375,40.59373638)
\curveto(401.96483958,39.91403896)(401.5507775,38.95700867)(401.44140625,37.72264263)
\lineto(409.12890625,37.73436138)
}
}
{
\newrgbcolor{curcolor}{0 0 0}
\pscustom[linestyle=none,fillstyle=solid,fillcolor=curcolor]
{
\newpath
\moveto(422.4296875,41.10936138)
\curveto(422.18749037,41.24997513)(421.92186564,41.35153753)(421.6328125,41.41404888)
\curveto(421.35155371,41.48434989)(421.03905402,41.51950611)(420.6953125,41.51951763)
\curveto(419.47655559,41.51950611)(418.53905652,41.12106901)(417.8828125,40.32420513)
\curveto(417.23437033,39.53513309)(416.9101519,38.39841548)(416.91015625,36.91404888)
\lineto(416.91015625,29.99998638)
\lineto(414.7421875,29.99998638)
\lineto(414.7421875,43.12498638)
\lineto(416.91015625,43.12498638)
\lineto(416.91015625,41.08592388)
\curveto(417.36327645,41.88278699)(417.95311961,42.47263015)(418.6796875,42.85545513)
\curveto(419.40624316,43.24606688)(420.28905477,43.44137919)(421.328125,43.44139263)
\curveto(421.47655359,43.44137919)(421.64061592,43.42966045)(421.8203125,43.40623638)
\curveto(421.99999056,43.39059799)(422.19920911,43.36325426)(422.41796875,43.32420513)
\lineto(422.4296875,41.10936138)
}
}
{
\newrgbcolor{curcolor}{0 0 0}
\pscustom[linestyle=none,fillstyle=solid,fillcolor=curcolor]
{
\newpath
\moveto(169.9296875,97.49608013)
\lineto(184.73046875,97.49608013)
\lineto(184.73046875,95.50389263)
\lineto(178.51953125,95.50389263)
\lineto(178.51953125,79.99998638)
\lineto(176.140625,79.99998638)
\lineto(176.140625,95.50389263)
\lineto(169.9296875,95.50389263)
\lineto(169.9296875,97.49608013)
}
}
{
\newrgbcolor{curcolor}{0 0 0}
\pscustom[linestyle=none,fillstyle=solid,fillcolor=curcolor]
{
\newpath
\moveto(188.64453125,78.78123638)
\curveto(188.03514914,77.21873916)(187.44139973,76.19920893)(186.86328125,75.72264263)
\curveto(186.28515089,75.24608488)(185.51171416,75.00780387)(184.54296875,75.00779888)
\lineto(182.8203125,75.00779888)
\lineto(182.8203125,76.81248638)
\lineto(184.0859375,76.81248638)
\curveto(184.67968374,76.81248956)(185.14062078,76.95311442)(185.46875,77.23436138)
\curveto(185.79687012,77.51561386)(186.16015101,78.1796757)(186.55859375,79.22654888)
\lineto(186.9453125,80.21092388)
\lineto(181.63671875,93.12498638)
\lineto(183.921875,93.12498638)
\lineto(188.0234375,82.85936138)
\lineto(192.125,93.12498638)
\lineto(194.41015625,93.12498638)
\lineto(188.64453125,78.78123638)
}
}
{
\newrgbcolor{curcolor}{0 0 0}
\pscustom[linestyle=none,fillstyle=solid,fillcolor=curcolor]
{
\newpath
\moveto(199.47265625,81.96873638)
\lineto(199.47265625,75.00779888)
\lineto(197.3046875,75.00779888)
\lineto(197.3046875,93.12498638)
\lineto(199.47265625,93.12498638)
\lineto(199.47265625,91.13279888)
\curveto(199.92577645,91.91403696)(200.49608838,92.49216138)(201.18359375,92.86717388)
\curveto(201.8788995,93.24997313)(202.70702367,93.44137919)(203.66796875,93.44139263)
\curveto(205.26170861,93.44137919)(206.55467607,92.80856732)(207.546875,91.54295513)
\curveto(208.54686158,90.27731985)(209.04686108,88.61325901)(209.046875,86.55076763)
\curveto(209.04686108,84.48826314)(208.54686158,82.8242023)(207.546875,81.55858013)
\curveto(206.55467607,80.29295483)(205.26170861,79.66014297)(203.66796875,79.66014263)
\curveto(202.70702367,79.66014297)(201.8788995,79.84764278)(201.18359375,80.22264263)
\curveto(200.49608838,80.60545452)(199.92577645,81.18748519)(199.47265625,81.96873638)
\moveto(206.80859375,86.55076763)
\curveto(206.80858207,88.13669699)(206.48045739,89.37888325)(205.82421875,90.27733013)
\curveto(205.1757712,91.18356894)(204.28124084,91.63669349)(203.140625,91.63670513)
\curveto(201.99999313,91.63669349)(201.10155652,91.18356894)(200.4453125,90.27733013)
\curveto(199.79687033,89.37888325)(199.4726519,88.13669699)(199.47265625,86.55076763)
\curveto(199.4726519,84.96482516)(199.79687033,83.71873266)(200.4453125,82.81248638)
\curveto(201.10155652,81.91404696)(201.99999313,81.46482866)(203.140625,81.46483013)
\curveto(204.28124084,81.46482866)(205.1757712,81.91404696)(205.82421875,82.81248638)
\curveto(206.48045739,83.71873266)(206.80858207,84.96482516)(206.80859375,86.55076763)
}
}
{
\newrgbcolor{curcolor}{0 0 0}
\pscustom[linestyle=none,fillstyle=solid,fillcolor=curcolor]
{
\newpath
\moveto(223.84765625,87.10154888)
\lineto(223.84765625,86.04686138)
\lineto(213.93359375,86.04686138)
\curveto(214.02734008,84.56248181)(214.47265214,83.42967045)(215.26953125,82.64842388)
\curveto(216.07421304,81.8749845)(217.19139942,81.48826614)(218.62109375,81.48826763)
\curveto(219.44920966,81.48826614)(220.24999011,81.58982854)(221.0234375,81.79295513)
\curveto(221.80467605,81.99607813)(222.57811278,82.30076533)(223.34375,82.70701763)
\lineto(223.34375,80.66795513)
\curveto(222.57030029,80.33982979)(221.77733233,80.08983004)(220.96484375,79.91795513)
\curveto(220.15233396,79.74608038)(219.32811603,79.66014297)(218.4921875,79.66014263)
\curveto(216.39843146,79.66014297)(214.73827687,80.26951736)(213.51171875,81.48826763)
\curveto(212.29296682,82.70701492)(211.68359243,84.35545077)(211.68359375,86.43358013)
\curveto(211.68359243,88.58200904)(212.26171685,90.28513234)(213.41796875,91.54295513)
\curveto(214.58202703,92.80856732)(216.14843171,93.44137919)(218.1171875,93.44139263)
\curveto(219.88280298,93.44137919)(221.27733283,92.87106726)(222.30078125,91.73045513)
\curveto(223.33201828,90.59763203)(223.84764276,89.05466482)(223.84765625,87.10154888)
\moveto(221.69140625,87.73436138)
\curveto(221.67576993,88.91403996)(221.34373902,89.85544527)(220.6953125,90.55858013)
\curveto(220.0546778,91.26169387)(219.20311616,91.61325601)(218.140625,91.61326763)
\curveto(216.93749342,91.61325601)(215.97265064,91.2734126)(215.24609375,90.59373638)
\curveto(214.52733958,89.91403896)(214.1132775,88.95700867)(214.00390625,87.72264263)
\lineto(221.69140625,87.73436138)
}
}
{
\newrgbcolor{curcolor}{0 0 0}
\pscustom[linestyle=none,fillstyle=solid,fillcolor=curcolor,opacity=0.11935484]
{
\newpath
\moveto(275.16365814,100.32731792)
\lineto(392.30652618,100.32731792)
\curveto(400.70719279,100.32731792)(407.47018433,93.56432639)(407.47018433,85.16365978)
\curveto(407.47018433,76.76299317)(400.70719279,70.00000164)(392.30652618,70.00000164)
\lineto(275.16365814,70.00000164)
\curveto(266.76299153,70.00000164)(260,76.76299317)(260,85.16365978)
\curveto(260,93.56432639)(266.76299153,100.32731792)(275.16365814,100.32731792)
\closepath
}
}
{
\newrgbcolor{curcolor}{0 0 0}
\pscustom[linewidth=2,linecolor=curcolor]
{
\newpath
\moveto(275.16365814,100.32731792)
\lineto(392.30652618,100.32731792)
\curveto(400.70719279,100.32731792)(407.47018433,93.56432639)(407.47018433,85.16365978)
\curveto(407.47018433,76.76299317)(400.70719279,70.00000164)(392.30652618,70.00000164)
\lineto(275.16365814,70.00000164)
\curveto(266.76299153,70.00000164)(260,76.76299317)(260,85.16365978)
\curveto(260,93.56432639)(266.76299153,100.32731792)(275.16365814,100.32731792)
\closepath
}
}
{
\newrgbcolor{curcolor}{0 0 0}
\pscustom[linestyle=none,fillstyle=solid,fillcolor=curcolor,opacity=0.11935484]
{
\newpath
\moveto(325.25841713,420.00000164)
\lineto(414.83723259,420.00000164)
\curveto(423.29039568,420.00000164)(430.09564972,413.1947476)(430.09564972,404.74158451)
\curveto(430.09564972,396.28842142)(423.29039568,389.48316738)(414.83723259,389.48316738)
\lineto(325.25841713,389.48316738)
\curveto(316.80525404,389.48316738)(310,396.28842142)(310,404.74158451)
\curveto(310,413.1947476)(316.80525404,420.00000164)(325.25841713,420.00000164)
\closepath
}
}
{
\newrgbcolor{curcolor}{0 0 0}
\pscustom[linewidth=2,linecolor=curcolor]
{
\newpath
\moveto(325.25841713,420.00000164)
\lineto(414.83723259,420.00000164)
\curveto(423.29039568,420.00000164)(430.09564972,413.1947476)(430.09564972,404.74158451)
\curveto(430.09564972,396.28842142)(423.29039568,389.48316738)(414.83723259,389.48316738)
\lineto(325.25841713,389.48316738)
\curveto(316.80525404,389.48316738)(310,396.28842142)(310,404.74158451)
\curveto(310,413.1947476)(316.80525404,420.00000164)(325.25841713,420.00000164)
\closepath
}
}
{
\newrgbcolor{curcolor}{0 0 0}
\pscustom[linestyle=none,fillstyle=solid,fillcolor=curcolor]
{
\newpath
\moveto(144.95852661,519.82745143)
\lineto(144.95852661,517.13409206)
\curveto(143.91033559,517.63537659)(142.92140429,518.00907414)(141.99172974,518.25518581)
\curveto(141.06203115,518.50126114)(140.16424559,518.6243079)(139.29837036,518.62432643)
\curveto(137.79445629,518.6243079)(136.63234808,518.33264152)(135.81204224,517.74932643)
\curveto(135.00083929,517.16597602)(134.59524074,516.33654977)(134.59524536,515.26104518)
\curveto(134.59524074,514.35868716)(134.86412068,513.67509409)(135.40188599,513.21026393)
\curveto(135.94875501,512.7545221)(136.9787019,512.38538184)(138.49172974,512.10284206)
\lineto(140.15969849,511.76104518)
\curveto(142.21958207,511.36910682)(143.73715868,510.67639918)(144.71243286,509.68292018)
\curveto(145.69679214,508.69853657)(146.18897915,507.37692331)(146.18899536,505.71807643)
\curveto(146.18897915,503.7402082)(145.52361523,502.24086074)(144.19290161,501.22002956)
\curveto(142.87127413,500.19919611)(140.92986982,499.68877996)(138.36868286,499.68877956)
\curveto(137.4025296,499.68877996)(136.37258271,499.79815485)(135.27883911,500.01690456)
\curveto(134.19419947,500.23565441)(133.06854956,500.55922179)(131.90188599,500.98760768)
\lineto(131.90188599,503.83135768)
\curveto(133.02297669,503.20244832)(134.12128288,502.72849046)(135.19680786,502.40948268)
\curveto(136.2723224,502.09047026)(137.32961301,501.93096521)(138.36868286,501.93096706)
\curveto(139.94549581,501.93096521)(141.16229146,502.24086074)(142.01907349,502.86065456)
\curveto(142.87583142,503.48044283)(143.30421641,504.36455653)(143.30422974,505.51299831)
\curveto(143.30421641,506.51559605)(142.99432088,507.29944943)(142.37454224,507.86456081)
\curveto(141.76385336,508.42965663)(140.75669291,508.85348433)(139.35305786,509.13604518)
\lineto(137.67141724,509.46417018)
\curveto(135.61151577,509.87431665)(134.12128288,510.51689413)(133.20071411,511.39190456)
\curveto(132.28013889,512.26689238)(131.81985289,513.48368804)(131.81985474,515.04229518)
\curveto(131.81985289,516.84696592)(132.4533158,518.2688395)(133.72024536,519.30792018)
\curveto(134.99628201,520.34696242)(136.75083754,520.86649315)(138.98391724,520.86651393)
\curveto(139.94093852,520.86649315)(140.91619796,520.7799047)(141.90969849,520.60674831)
\curveto(142.90317514,520.43355088)(143.91945017,520.17378551)(144.95852661,519.82745143)
}
}
{
\newrgbcolor{curcolor}{0 0 0}
\pscustom[linestyle=none,fillstyle=solid,fillcolor=curcolor]
{
\newpath
\moveto(156.31985474,513.63409206)
\curveto(154.97088918,513.63407851)(153.904484,513.10543321)(153.12063599,512.04815456)
\curveto(152.33677723,510.99996656)(151.94485054,509.55986383)(151.94485474,507.72784206)
\curveto(151.94485054,505.895805)(152.33221994,504.45114499)(153.10696411,503.39385768)
\curveto(153.89081213,502.34567834)(154.96177461,501.82159032)(156.31985474,501.82159206)
\curveto(157.65968857,501.82159032)(158.72153647,502.35023563)(159.50540161,503.40752956)
\curveto(160.28924324,504.46481685)(160.68116993,505.90491957)(160.68118286,507.72784206)
\curveto(160.68116993,509.54163469)(160.28924324,510.97718013)(159.50540161,512.03448268)
\curveto(158.72153647,513.10087592)(157.65968857,513.63407851)(156.31985474,513.63409206)
\moveto(156.31985474,515.76690456)
\curveto(158.50734398,515.76688888)(160.22544122,515.05595209)(161.47415161,513.63409206)
\curveto(162.72283455,512.21220493)(163.34718289,510.2434569)(163.34719849,507.72784206)
\curveto(163.34718289,505.22132651)(162.72283455,503.25257848)(161.47415161,501.82159206)
\curveto(160.22544122,500.39971674)(158.50734398,499.68877996)(156.31985474,499.68877956)
\curveto(154.12323378,499.68877996)(152.40057925,500.39971674)(151.15188599,501.82159206)
\curveto(149.91230049,503.25257848)(149.29250944,505.22132651)(149.29251099,507.72784206)
\curveto(149.29250944,510.2434569)(149.91230049,512.21220493)(151.15188599,513.63409206)
\curveto(152.40057925,515.05595209)(154.12323378,515.76688888)(156.31985474,515.76690456)
}
}
{
\newrgbcolor{curcolor}{0 0 0}
\pscustom[linestyle=none,fillstyle=solid,fillcolor=curcolor]
{
\newpath
\moveto(167.50344849,521.35870143)
\lineto(170.01907349,521.35870143)
\lineto(170.01907349,500.08526393)
\lineto(167.50344849,500.08526393)
\lineto(167.50344849,521.35870143)
}
}
{
\newrgbcolor{curcolor}{0 0 0}
\pscustom[linestyle=none,fillstyle=solid,fillcolor=curcolor]
{
\newpath
\moveto(181.20266724,513.63409206)
\curveto(179.85370168,513.63407851)(178.7872965,513.10543321)(178.00344849,512.04815456)
\curveto(177.21958973,510.99996656)(176.82766304,509.55986383)(176.82766724,507.72784206)
\curveto(176.82766304,505.895805)(177.21503244,504.45114499)(177.98977661,503.39385768)
\curveto(178.77362463,502.34567834)(179.84458711,501.82159032)(181.20266724,501.82159206)
\curveto(182.54250107,501.82159032)(183.60434897,502.35023563)(184.38821411,503.40752956)
\curveto(185.17205574,504.46481685)(185.56398243,505.90491957)(185.56399536,507.72784206)
\curveto(185.56398243,509.54163469)(185.17205574,510.97718013)(184.38821411,512.03448268)
\curveto(183.60434897,513.10087592)(182.54250107,513.63407851)(181.20266724,513.63409206)
\moveto(181.20266724,515.76690456)
\curveto(183.39015648,515.76688888)(185.10825372,515.05595209)(186.35696411,513.63409206)
\curveto(187.60564705,512.21220493)(188.22999539,510.2434569)(188.23001099,507.72784206)
\curveto(188.22999539,505.22132651)(187.60564705,503.25257848)(186.35696411,501.82159206)
\curveto(185.10825372,500.39971674)(183.39015648,499.68877996)(181.20266724,499.68877956)
\curveto(179.00604628,499.68877996)(177.28339175,500.39971674)(176.03469849,501.82159206)
\curveto(174.79511299,503.25257848)(174.17532194,505.22132651)(174.17532349,507.72784206)
\curveto(174.17532194,510.2434569)(174.79511299,512.21220493)(176.03469849,513.63409206)
\curveto(177.28339175,515.05595209)(179.00604628,515.76688888)(181.20266724,515.76690456)
}
}
{
\newrgbcolor{curcolor}{1 1 1}
\pscustom[linestyle=none,fillstyle=solid,fillcolor=curcolor]
{
\newpath
\moveto(150.00000191,260.00000164)
\lineto(259.99998283,260.00000164)
\curveto(271.07998389,260.00000164)(279.99998474,251.08000079)(279.99998474,239.99999973)
\lineto(279.99998474,219.9999921)
\curveto(279.99998474,208.91999104)(271.07998389,199.99999019)(259.99998283,199.99999019)
\lineto(150.00000191,199.99999019)
\curveto(138.92000085,199.99999019)(130,208.91999104)(130,219.9999921)
\lineto(130,239.99999973)
\curveto(130,251.08000079)(138.92000085,260.00000164)(150.00000191,260.00000164)
\closepath
}
}
{
\newrgbcolor{curcolor}{0 0 0}
\pscustom[linewidth=2,linecolor=curcolor]
{
\newpath
\moveto(150.00000191,260.00000164)
\lineto(259.99998283,260.00000164)
\curveto(271.07998389,260.00000164)(279.99998474,251.08000079)(279.99998474,239.99999973)
\lineto(279.99998474,219.9999921)
\curveto(279.99998474,208.91999104)(271.07998389,199.99999019)(259.99998283,199.99999019)
\lineto(150.00000191,199.99999019)
\curveto(138.92000085,199.99999019)(130,208.91999104)(130,219.9999921)
\lineto(130,239.99999973)
\curveto(130,251.08000079)(138.92000085,260.00000164)(150.00000191,260.00000164)
\closepath
}
}
{
\newrgbcolor{curcolor}{0 0 0}
\pscustom[linestyle=none,fillstyle=solid,fillcolor=curcolor]
{
\newpath
\moveto(43.96875,532.65625164)
\lineto(49.125,532.65625164)
\lineto(49.125,550.45312664)
\lineto(43.515625,549.32812664)
\lineto(43.515625,552.20312664)
\lineto(49.09375,553.32812664)
\lineto(52.25,553.32812664)
\lineto(52.25,532.65625164)
\lineto(57.40625,532.65625164)
\lineto(57.40625,530.00000164)
\lineto(43.96875,530.00000164)
\lineto(43.96875,532.65625164)
}
}
{
\newrgbcolor{curcolor}{0 0 0}
\pscustom[linestyle=none,fillstyle=solid,fillcolor=curcolor]
{
\newpath
\moveto(506.11477661,472.74151393)
\lineto(517.13040161,472.74151393)
\lineto(517.13040161,470.08526393)
\lineto(502.31790161,470.08526393)
\lineto(502.31790161,472.74151393)
\curveto(503.51581474,473.98109337)(505.14602144,475.64255004)(507.20852661,477.72588893)
\curveto(509.28143397,479.8196292)(510.583516,481.16858618)(511.11477661,481.77276393)
\curveto(512.12518113,482.90816778)(512.82830542,483.86650015)(513.22415161,484.64776393)
\curveto(513.63038796,485.43941525)(513.83351275,486.21545614)(513.83352661,486.97588893)
\curveto(513.83351275,488.21545414)(513.39601319,489.22586979)(512.52102661,490.00713893)
\curveto(511.6564316,490.78836823)(510.52622439,491.17899284)(509.13040161,491.17901393)
\curveto(508.14081011,491.17899284)(507.09393616,491.00711801)(505.98977661,490.66338893)
\curveto(504.89602169,490.3196187)(503.72414786,489.79878589)(502.47415161,489.10088893)
\lineto(502.47415161,492.28838893)
\curveto(503.74498117,492.79878289)(504.93247999,493.18419917)(506.03665161,493.44463893)
\curveto(507.14081111,493.70503198)(508.15122677,493.83524018)(509.06790161,493.83526393)
\curveto(511.48455677,493.83524018)(513.41163817,493.23107412)(514.84915161,492.02276393)
\curveto(516.2866353,490.81440987)(517.00538458,489.19982815)(517.00540161,487.17901393)
\curveto(517.00538458,486.22066447)(516.8230931,485.30920704)(516.45852661,484.44463893)
\curveto(516.10434381,483.59045876)(515.4533028,482.58004311)(514.50540161,481.41338893)
\curveto(514.24497067,481.11129457)(513.4168465,480.23629545)(512.02102661,478.78838893)
\curveto(510.62518263,477.35088167)(508.6564346,475.33525868)(506.11477661,472.74151393)
}
}
{
\newrgbcolor{curcolor}{0 0 0}
\pscustom[linestyle=none,fillstyle=solid,fillcolor=curcolor]
{
\newpath
\moveto(512.95852661,412.66339275)
\curveto(514.46892878,412.34046383)(515.64601094,411.6685895)(516.48977661,410.64776775)
\curveto(517.34392591,409.62692487)(517.77100881,408.36650947)(517.77102661,406.86651775)
\curveto(517.77100881,404.56442994)(516.97934294,402.78318172)(515.39602661,401.52276775)
\curveto(513.81267944,400.26235091)(511.56268169,399.6321432)(508.64602661,399.63214275)
\curveto(507.66685225,399.6321432)(506.6564366,399.73110144)(505.61477661,399.92901775)
\curveto(504.583522,400.11651772)(503.51581474,400.40297576)(502.41165161,400.78839275)
\lineto(502.41165161,403.83526775)
\curveto(503.2866483,403.32484784)(504.24498067,402.93943156)(505.28665161,402.67901775)
\curveto(506.32831192,402.41859875)(507.4168525,402.28839055)(508.55227661,402.28839275)
\curveto(510.53143272,402.28839055)(512.03663955,402.67901516)(513.06790161,403.46026775)
\curveto(514.10955414,404.24151359)(514.63038696,405.37692912)(514.63040161,406.86651775)
\curveto(514.63038696,408.24150959)(514.14601244,409.31442519)(513.17727661,410.08526775)
\curveto(512.21893103,410.86650697)(510.88039071,411.25713158)(509.16165161,411.25714275)
\lineto(506.44290161,411.25714275)
\lineto(506.44290161,413.85089275)
\lineto(509.28665161,413.85089275)
\curveto(510.83872408,413.85087898)(512.02622289,414.15817034)(512.84915161,414.77276775)
\curveto(513.67205458,415.39775244)(514.0835125,416.29358487)(514.08352661,417.46026775)
\curveto(514.0835125,418.65816584)(513.6564296,419.57483159)(512.80227661,420.21026775)
\curveto(511.95851463,420.85608031)(510.74497417,421.17899666)(509.16165161,421.17901775)
\curveto(508.29705996,421.17899666)(507.36997755,421.08524675)(506.38040161,420.89776775)
\curveto(505.39081286,420.71024712)(504.30227228,420.41858075)(503.11477661,420.02276775)
\lineto(503.11477661,422.83526775)
\curveto(504.31268894,423.168578)(505.43247949,423.41857775)(506.47415161,423.58526775)
\curveto(507.52622739,423.75191075)(508.51580974,423.835244)(509.44290161,423.83526775)
\curveto(511.83872308,423.835244)(513.73455452,423.28836955)(515.13040161,422.19464275)
\curveto(516.52621839,421.11128839)(517.22413436,419.64253986)(517.22415161,417.78839275)
\curveto(517.22413436,416.49670967)(516.85434306,415.40296076)(516.11477661,414.50714275)
\curveto(515.37517788,413.62171255)(514.3230956,413.00712983)(512.95852661,412.66339275)
}
}
{
\newrgbcolor{curcolor}{0 0 0}
\pscustom[linestyle=none,fillstyle=solid,fillcolor=curcolor]
{
\newpath
\moveto(509.53771973,310.6637437)
\lineto(501.56896973,298.2106187)
\lineto(509.53771973,298.2106187)
\lineto(509.53771973,310.6637437)
\moveto(508.70959473,313.4137437)
\lineto(512.67834473,313.4137437)
\lineto(512.67834473,298.2106187)
\lineto(516.00646973,298.2106187)
\lineto(516.00646973,295.5856187)
\lineto(512.67834473,295.5856187)
\lineto(512.67834473,290.0856187)
\lineto(509.53771973,290.0856187)
\lineto(509.53771973,295.5856187)
\lineto(499.00646973,295.5856187)
\lineto(499.00646973,298.6324937)
\lineto(508.70959473,313.4137437)
}
}
{
\newrgbcolor{curcolor}{0 0 0}
\pscustom[linestyle=none,fillstyle=solid,fillcolor=curcolor]
{
\newpath
\moveto(500.89709473,253.4137437)
\lineto(513.28771973,253.4137437)
\lineto(513.28771973,250.7574937)
\lineto(503.78771973,250.7574937)
\lineto(503.78771973,245.0387437)
\curveto(504.24604626,245.19497859)(504.70437913,245.30956181)(505.16271973,245.3824937)
\curveto(505.62104488,245.46581165)(506.07937776,245.50747828)(506.53771973,245.5074937)
\curveto(509.1418747,245.50747828)(511.20437263,244.79393733)(512.72521973,243.3668687)
\curveto(514.24603626,241.93977351)(515.00645216,240.00748378)(515.00646973,237.5699937)
\curveto(515.00645216,235.05957206)(514.22520295,233.10644901)(512.66271973,231.7106187)
\curveto(511.10020607,230.32520179)(508.89708327,229.63249415)(506.05334473,229.6324937)
\curveto(505.07417043,229.63249415)(504.07417143,229.7158274)(503.05334473,229.8824937)
\curveto(502.04292346,230.0491604)(500.99604951,230.29916015)(499.91271973,230.6324937)
\lineto(499.91271973,233.8043687)
\curveto(500.85021632,233.29394883)(501.81896535,232.91374087)(502.81896973,232.6637437)
\curveto(503.81896335,232.41374137)(504.87625396,232.2887415)(505.99084473,232.2887437)
\curveto(507.79291771,232.2887415)(509.21999962,232.76269936)(510.27209473,233.7106187)
\curveto(511.32416418,234.65853079)(511.85020532,235.94498784)(511.85021973,237.5699937)
\curveto(511.85020532,239.19498459)(511.32416418,240.48144164)(510.27209473,241.4293687)
\curveto(509.21999962,242.37727308)(507.79291771,242.85123094)(505.99084473,242.8512437)
\curveto(505.14708702,242.85123094)(504.30333787,242.75748103)(503.45959473,242.5699937)
\curveto(502.62625621,242.3824814)(501.7720904,242.09081503)(500.89709473,241.6949937)
\lineto(500.89709473,253.4137437)
}
}
{
\newrgbcolor{curcolor}{0 0 0}
\pscustom[linestyle=none,fillstyle=solid,fillcolor=curcolor]
{
\newpath
\moveto(510.5625,182.92186138)
\curveto(509.14582419,182.92184846)(508.02082531,182.43747394)(507.1875,181.46873638)
\curveto(506.36457697,180.49997588)(505.95311905,179.17185221)(505.953125,177.48436138)
\curveto(505.95311905,175.80727224)(506.36457697,174.47914856)(507.1875,173.49998638)
\curveto(508.02082531,172.53123385)(509.14582419,172.04685933)(510.5625,172.04686138)
\curveto(511.97915469,172.04685933)(513.09894523,172.53123385)(513.921875,173.49998638)
\curveto(514.75519358,174.47914856)(515.17185983,175.80727224)(515.171875,177.48436138)
\curveto(515.17185983,179.17185221)(514.75519358,180.49997588)(513.921875,181.46873638)
\curveto(513.09894523,182.43747394)(511.97915469,182.92184846)(510.5625,182.92186138)
\moveto(516.828125,192.81248638)
\lineto(516.828125,189.93748638)
\curveto(516.0364423,190.31246606)(515.23435977,190.59892411)(514.421875,190.79686138)
\curveto(513.61977805,190.99475705)(512.82290384,191.09371528)(512.03125,191.09373638)
\curveto(509.94790672,191.09371528)(508.35415831,190.39059099)(507.25,188.98436138)
\curveto(506.15624384,187.5780938)(505.53124447,185.45309592)(505.375,182.60936138)
\curveto(505.98957734,183.51559786)(506.76040991,184.2083055)(507.6875,184.68748638)
\curveto(508.61457472,185.17705453)(509.63540703,185.42184596)(510.75,185.42186138)
\curveto(513.09373691,185.42184596)(514.94269339,184.708305)(516.296875,183.28123638)
\curveto(517.66144067,181.86455785)(518.34373166,179.93226811)(518.34375,177.48436138)
\curveto(518.34373166,175.08852296)(517.63539903,173.16664988)(516.21875,171.71873638)
\curveto(514.80206853,170.27081944)(512.91665375,169.54686183)(510.5625,169.54686138)
\curveto(507.86457547,169.54686183)(505.80207753,170.5781108)(504.375,172.64061138)
\curveto(502.94791372,174.71352333)(502.23437277,177.71352033)(502.234375,181.64061138)
\curveto(502.23437277,185.32809605)(503.10937189,188.26559311)(504.859375,190.45311138)
\curveto(506.60936839,192.65100539)(508.95832438,193.74996263)(511.90625,193.74998638)
\curveto(512.69790397,193.74996263)(513.49477817,193.67183771)(514.296875,193.51561138)
\curveto(515.10935989,193.35933802)(515.95310905,193.12496325)(516.828125,192.81248638)
}
}
{
\newrgbcolor{curcolor}{0 0 0}
\pscustom[linestyle=none,fillstyle=solid,fillcolor=curcolor]
{
\newpath
\moveto(502.625,163.32811138)
\lineto(517.625,163.32811138)
\lineto(517.625,161.98436138)
\lineto(509.15625,139.99998638)
\lineto(505.859375,139.99998638)
\lineto(513.828125,160.67186138)
\lineto(502.625,160.67186138)
\lineto(502.625,163.32811138)
}
}
{
\newrgbcolor{curcolor}{0 0 0}
\pscustom[linestyle=none,fillstyle=solid,fillcolor=curcolor]
{
\newpath
\moveto(510.171875,91.07811138)
\curveto(508.67186633,91.0781003)(507.48957584,90.67705903)(506.625,89.87498638)
\curveto(505.77082756,89.07289397)(505.34374466,87.96872841)(505.34375,86.56248638)
\curveto(505.34374466,85.15623122)(505.77082756,84.05206566)(506.625,83.24998638)
\curveto(507.48957584,82.4479006)(508.67186633,82.04685933)(510.171875,82.04686138)
\curveto(511.67186333,82.04685933)(512.85415381,82.4479006)(513.71875,83.24998638)
\curveto(514.58331875,84.06248231)(515.01560998,85.16664788)(515.015625,86.56248638)
\curveto(515.01560998,87.96872841)(514.58331875,89.07289397)(513.71875,89.87498638)
\curveto(512.86457047,90.67705903)(511.68227998,91.0781003)(510.171875,91.07811138)
\moveto(507.015625,92.42186138)
\curveto(505.66145267,92.75518196)(504.60416206,93.38538966)(503.84375,94.31248638)
\curveto(503.09374691,95.23955447)(502.71874728,96.36976167)(502.71875,97.70311138)
\curveto(502.71874728,99.56767514)(503.38020495,101.041632)(504.703125,102.12498638)
\curveto(506.0364523,103.2082965)(507.85936714,103.74996263)(510.171875,103.74998638)
\curveto(512.49477917,103.74996263)(514.31769402,103.2082965)(515.640625,102.12498638)
\curveto(516.9635247,101.041632)(517.62498237,99.56767514)(517.625,97.70311138)
\curveto(517.62498237,96.36976167)(517.24477442,95.23955447)(516.484375,94.31248638)
\curveto(515.73435927,93.38538966)(514.68748531,92.75518196)(513.34375,92.42186138)
\curveto(514.86456847,92.06768264)(516.04685895,91.374975)(516.890625,90.34373638)
\curveto(517.74477392,89.31247706)(518.17185683,88.05206166)(518.171875,86.56248638)
\curveto(518.17185683,84.30206541)(517.47914919,82.56769214)(516.09375,81.35936138)
\curveto(514.71873528,80.15102789)(512.74477892,79.54686183)(510.171875,79.54686138)
\curveto(507.59895073,79.54686183)(505.61978605,80.15102789)(504.234375,81.35936138)
\curveto(502.85937214,82.56769214)(502.17187283,84.30206541)(502.171875,86.56248638)
\curveto(502.17187283,88.05206166)(502.59895573,89.31247706)(503.453125,90.34373638)
\curveto(504.30728736,91.374975)(505.49478617,92.06768264)(507.015625,92.42186138)
\moveto(505.859375,97.40623638)
\curveto(505.85936914,96.19788685)(506.23436877,95.25517946)(506.984375,94.57811138)
\curveto(507.74478392,93.90101414)(508.80728286,93.56247281)(510.171875,93.56248638)
\curveto(511.52603014,93.56247281)(512.58332075,93.90101414)(513.34375,94.57811138)
\curveto(514.11456922,95.25517946)(514.4999855,96.19788685)(514.5,97.40623638)
\curveto(514.4999855,98.6145511)(514.11456922,99.55725849)(513.34375,100.23436138)
\curveto(512.58332075,100.9114238)(511.52603014,101.24996513)(510.171875,101.24998638)
\curveto(508.80728286,101.24996513)(507.74478392,100.9114238)(506.984375,100.23436138)
\curveto(506.23436877,99.55725849)(505.85936914,98.6145511)(505.859375,97.40623638)
}
}
{
\newrgbcolor{curcolor}{1 1 1}
\pscustom[linestyle=none,fillstyle=solid,fillcolor=curcolor]
{
\newpath
\moveto(330,189.99998638)
\lineto(350,189.99998638)
\lineto(350,169.99998638)
\lineto(330,169.99998638)
\closepath
}
}
{
\newrgbcolor{curcolor}{0 0 0}
\pscustom[linewidth=2,linecolor=curcolor]
{
\newpath
\moveto(330,189.99998638)
\lineto(350,189.99998638)
\lineto(350,169.99998638)
\lineto(330,169.99998638)
\closepath
}
}
{
\newrgbcolor{curcolor}{0 0 0}
\pscustom[linewidth=2,linecolor=curcolor,linestyle=dashed,dash=8 8]
{
\newpath
\moveto(150,540)
\lineto(60,540)
}
}
{
\newrgbcolor{curcolor}{0 0 0}
\pscustom[linestyle=none,fillstyle=solid,fillcolor=curcolor]
{
\newpath
\moveto(139.53769464,544.84048224)
\lineto(152.6487474,540.01921591)
\lineto(139.53769392,535.19795064)
\curveto(141.632292,538.04442372)(141.62022288,541.93889292)(139.53769464,544.84048224)
\lineto(139.53769464,544.84048224)
\closepath
}
}
{
\newrgbcolor{curcolor}{0 0 0}
\pscustom[linewidth=2,linecolor=curcolor,linestyle=dashed,dash=8 8]
{
\newpath
\moveto(399.97415,480.085264)
\lineto(499.97415,480.085264)
}
}
{
\newrgbcolor{curcolor}{0 0 0}
\pscustom[linestyle=none,fillstyle=solid,fillcolor=curcolor]
{
\newpath
\moveto(410.43645536,475.24478176)
\lineto(397.3254026,480.06604809)
\lineto(410.43645608,484.88731336)
\curveto(408.341858,482.04084028)(408.35392712,478.14637108)(410.43645536,475.24478176)
\lineto(410.43645536,475.24478176)
\closepath
}
}
{
\newrgbcolor{curcolor}{0 0 0}
\pscustom[linewidth=2,linecolor=curcolor,linestyle=dashed,dash=8 8]
{
\newpath
\moveto(399.97415,410.085264)
\lineto(499.97415,410.085264)
}
}
{
\newrgbcolor{curcolor}{0 0 0}
\pscustom[linestyle=none,fillstyle=solid,fillcolor=curcolor]
{
\newpath
\moveto(410.43645536,405.24478176)
\lineto(397.3254026,410.06604809)
\lineto(410.43645608,414.88731336)
\curveto(408.341858,412.04084028)(408.35392712,408.14637108)(410.43645536,405.24478176)
\lineto(410.43645536,405.24478176)
\closepath
}
}
{
\newrgbcolor{curcolor}{0 0 0}
\pscustom[linewidth=2,linecolor=curcolor,linestyle=dashed,dash=8 8]
{
\newpath
\moveto(397.44397,300.08561)
\lineto(497.44397,300.08561)
}
}
{
\newrgbcolor{curcolor}{0 0 0}
\pscustom[linestyle=none,fillstyle=solid,fillcolor=curcolor]
{
\newpath
\moveto(407.90627536,295.24512776)
\lineto(394.7952226,300.06639409)
\lineto(407.90627608,304.88765936)
\curveto(405.811678,302.04118628)(405.82374712,298.14671708)(407.90627536,295.24512776)
\lineto(407.90627536,295.24512776)
\closepath
}
}
{
\newrgbcolor{curcolor}{1 1 1}
\pscustom[linestyle=none,fillstyle=solid,fillcolor=curcolor]
{
\newpath
\moveto(314.28493977,248.77482769)
\lineto(415.62935162,248.77482769)
\curveto(423.57636236,248.77482769)(429.97413635,242.37705369)(429.97413635,234.43004295)
\curveto(429.97413635,226.4830322)(423.57636236,220.08525821)(415.62935162,220.08525821)
\lineto(314.28493977,220.08525821)
\curveto(306.33792902,220.08525821)(299.94015503,226.4830322)(299.94015503,234.43004295)
\curveto(299.94015503,242.37705369)(306.33792902,248.77482769)(314.28493977,248.77482769)
\closepath
}
}
{
\newrgbcolor{curcolor}{0 0 0}
\pscustom[linewidth=2,linecolor=curcolor]
{
\newpath
\moveto(314.28493977,248.77482769)
\lineto(415.62935162,248.77482769)
\curveto(423.57636236,248.77482769)(429.97413635,242.37705369)(429.97413635,234.43004295)
\curveto(429.97413635,226.4830322)(423.57636236,220.08525821)(415.62935162,220.08525821)
\lineto(314.28493977,220.08525821)
\curveto(306.33792902,220.08525821)(299.94015503,226.4830322)(299.94015503,234.43004295)
\curveto(299.94015503,242.37705369)(306.33792902,248.77482769)(314.28493977,248.77482769)
\closepath
}
}
{
\newrgbcolor{curcolor}{0 0 0}
\pscustom[linewidth=2,linecolor=curcolor,linestyle=dashed,dash=8 8]
{
\newpath
\moveto(409.97415,240.08561)
\lineto(497.44397,240.08561)
}
}
{
\newrgbcolor{curcolor}{0 0 0}
\pscustom[linestyle=none,fillstyle=solid,fillcolor=curcolor]
{
\newpath
\moveto(420.43645536,235.24512776)
\lineto(407.3254026,240.06639409)
\lineto(420.43645608,244.88765936)
\curveto(418.341858,242.04118628)(418.35392712,238.14671708)(420.43645536,235.24512776)
\lineto(420.43645536,235.24512776)
\closepath
}
}
{
\newrgbcolor{curcolor}{0 0 0}
\pscustom[linewidth=2,linecolor=curcolor,linestyle=dashed,dash=8 8]
{
\newpath
\moveto(380,90)
\lineto(500,90)
}
}
{
\newrgbcolor{curcolor}{0 0 0}
\pscustom[linestyle=none,fillstyle=solid,fillcolor=curcolor]
{
\newpath
\moveto(390.46230536,85.15951776)
\lineto(377.3512526,89.98078409)
\lineto(390.46230608,94.80204936)
\curveto(388.367708,91.95557628)(388.37977712,88.06110708)(390.46230536,85.15951776)
\lineto(390.46230536,85.15951776)
\closepath
}
}
{
\newrgbcolor{curcolor}{0 0 0}
\pscustom[linewidth=2,linecolor=curcolor,linestyle=dashed,dash=8 8]
{
\newpath
\moveto(420,40)
\lineto(500,40)
}
}
{
\newrgbcolor{curcolor}{0 0 0}
\pscustom[linestyle=none,fillstyle=solid,fillcolor=curcolor]
{
\newpath
\moveto(430.46230536,35.15951776)
\lineto(417.3512526,39.98078409)
\lineto(430.46230608,44.80204936)
\curveto(428.367708,41.95557628)(428.37977712,38.06110708)(430.46230536,35.15951776)
\lineto(430.46230536,35.15951776)
\closepath
}
}
{
\newrgbcolor{curcolor}{0 0 0}
\pscustom[linewidth=2,linecolor=curcolor,linestyle=dashed,dash=8 8]
{
\newpath
\moveto(150,40)
\lineto(60,40)
}
}
{
\newrgbcolor{curcolor}{0 0 0}
\pscustom[linestyle=none,fillstyle=solid,fillcolor=curcolor]
{
\newpath
\moveto(139.53769464,44.84048224)
\lineto(152.6487474,40.01921591)
\lineto(139.53769392,35.19795064)
\curveto(141.632292,38.04442372)(141.62022288,41.93889292)(139.53769464,44.84048224)
\lineto(139.53769464,44.84048224)
\closepath
}
}
{
\newrgbcolor{curcolor}{0 0 0}
\pscustom[linewidth=2,linecolor=curcolor,linestyle=dashed,dash=8 8]
{
\newpath
\moveto(340,180)
\lineto(500,180)
}
}
{
\newrgbcolor{curcolor}{0 0 0}
\pscustom[linestyle=none,fillstyle=solid,fillcolor=curcolor]
{
\newpath
\moveto(350.46230536,175.15951776)
\lineto(337.3512526,179.98078409)
\lineto(350.46230608,184.80204936)
\curveto(348.367708,181.95557628)(348.37977712,178.06110708)(350.46230536,175.15951776)
\lineto(350.46230536,175.15951776)
\closepath
}
}
{
\newrgbcolor{curcolor}{0 0 0}
\pscustom[linestyle=none,fillstyle=solid,fillcolor=curcolor]
{
\newpath
\moveto(43.515625,30.48436138)
\lineto(43.515625,33.35936138)
\curveto(44.30728736,32.98435839)(45.10936989,32.69790035)(45.921875,32.49998638)
\curveto(46.73436827,32.30206741)(47.53124247,32.20310917)(48.3125,32.20311138)
\curveto(50.39582294,32.20310917)(51.98436302,32.90102514)(53.078125,34.29686138)
\curveto(54.18227748,35.70310567)(54.81248519,37.83331188)(54.96875,40.68748638)
\curveto(54.36456897,39.79164325)(53.59894473,39.10414394)(52.671875,38.62498638)
\curveto(51.74477992,38.14581156)(50.71873928,37.90622847)(49.59375,37.90623638)
\curveto(47.26040941,37.90622847)(45.41145292,38.60935277)(44.046875,40.01561138)
\curveto(42.69270564,41.43226661)(42.01562298,43.36455635)(42.015625,45.81248638)
\curveto(42.01562298,48.2083015)(42.72395561,50.13017458)(44.140625,51.57811138)
\curveto(45.55728611,53.02600502)(47.44270089,53.74996263)(49.796875,53.74998638)
\curveto(52.49477917,53.74996263)(54.55206878,52.71350533)(55.96875,50.64061138)
\curveto(57.39581594,48.5780928)(58.10935689,45.5780958)(58.109375,41.64061138)
\curveto(58.10935689,37.96352008)(57.23435777,35.02602302)(55.484375,32.82811138)
\curveto(53.74477792,30.64061074)(51.40103027,29.54686183)(48.453125,29.54686138)
\curveto(47.66145067,29.54686183)(46.85936814,29.62498675)(46.046875,29.78123638)
\curveto(45.23436977,29.93748644)(44.39062061,30.17186121)(43.515625,30.48436138)
\moveto(49.796875,40.37498638)
\curveto(51.21353045,40.374976)(52.333321,40.85935052)(53.15625,41.82811138)
\curveto(53.98956934,42.79684858)(54.40623559,44.12497225)(54.40625,45.81248638)
\curveto(54.40623559,47.48955222)(53.98956934,48.81246756)(53.15625,49.78123638)
\curveto(52.333321,50.76038228)(51.21353045,51.24996513)(49.796875,51.24998638)
\curveto(48.38019995,51.24996513)(47.25520108,50.76038228)(46.421875,49.78123638)
\curveto(45.59895273,48.81246756)(45.18749481,47.48955222)(45.1875,45.81248638)
\curveto(45.18749481,44.12497225)(45.59895273,42.79684858)(46.421875,41.82811138)
\curveto(47.25520108,40.85935052)(48.38019995,40.374976)(49.796875,40.37498638)
}
}
{
\newrgbcolor{curcolor}{1 1 1}
\pscustom[linestyle=none,fillstyle=solid,fillcolor=curcolor]
{
\newpath
\moveto(410,160)
\lineto(430,160)
\lineto(430,140)
\lineto(410,140)
\closepath
}
}
{
\newrgbcolor{curcolor}{0 0 0}
\pscustom[linewidth=2,linecolor=curcolor]
{
\newpath
\moveto(410,160)
\lineto(430,160)
\lineto(430,140)
\lineto(410,140)
\closepath
}
}
{
\newrgbcolor{curcolor}{0 0 0}
\pscustom[linewidth=2,linecolor=curcolor,linestyle=dashed,dash=8 8]
{
\newpath
\moveto(420,150)
\lineto(500,150)
}
}
{
\newrgbcolor{curcolor}{0 0 0}
\pscustom[linestyle=none,fillstyle=solid,fillcolor=curcolor]
{
\newpath
\moveto(430.46230536,145.15951776)
\lineto(417.3512526,149.98078409)
\lineto(430.46230608,154.80204936)
\curveto(428.367708,151.95557628)(428.37977712,148.06110708)(430.46230536,145.15951776)
\lineto(430.46230536,145.15951776)
\closepath
}
}
{
\newrgbcolor{curcolor}{0 0 0}
\pscustom[linestyle=none,fillstyle=solid,fillcolor=curcolor]
{
\newpath
\moveto(493.96875,32.65623638)
\lineto(499.125,32.65623638)
\lineto(499.125,50.45311138)
\lineto(493.515625,49.32811138)
\lineto(493.515625,52.20311138)
\lineto(499.09375,53.32811138)
\lineto(502.25,53.32811138)
\lineto(502.25,32.65623638)
\lineto(507.40625,32.65623638)
\lineto(507.40625,29.99998638)
\lineto(493.96875,29.99998638)
\lineto(493.96875,32.65623638)
}
}
{
\newrgbcolor{curcolor}{0 0 0}
\pscustom[linestyle=none,fillstyle=solid,fillcolor=curcolor]
{
\newpath
\moveto(520.546875,51.24998638)
\curveto(518.92186645,51.24996513)(517.69790934,50.4478826)(516.875,48.84373638)
\curveto(516.06249431,47.24996913)(515.65624472,44.84892986)(515.65625,41.64061138)
\curveto(515.65624472,38.44268627)(516.06249431,36.041647)(516.875,34.43748638)
\curveto(517.69790934,32.84373353)(518.92186645,32.04685933)(520.546875,32.04686138)
\curveto(522.18227986,32.04685933)(523.40623697,32.84373353)(524.21875,34.43748638)
\curveto(525.041652,36.041647)(525.45310992,38.44268627)(525.453125,41.64061138)
\curveto(525.45310992,44.84892986)(525.041652,47.24996913)(524.21875,48.84373638)
\curveto(523.40623697,50.4478826)(522.18227986,51.24996513)(520.546875,51.24998638)
\moveto(520.546875,53.74998638)
\curveto(523.16144555,53.74996263)(525.15623522,52.71350533)(526.53125,50.64061138)
\curveto(527.91664913,48.5780928)(528.60935677,45.5780958)(528.609375,41.64061138)
\curveto(528.60935677,37.71352033)(527.91664913,34.71352333)(526.53125,32.64061138)
\curveto(525.15623522,30.5781108)(523.16144555,29.54686183)(520.546875,29.54686138)
\curveto(517.93228411,29.54686183)(515.93228611,30.5781108)(514.546875,32.64061138)
\curveto(513.1718722,34.71352333)(512.48437289,37.71352033)(512.484375,41.64061138)
\curveto(512.48437289,45.5780958)(513.1718722,48.5780928)(514.546875,50.64061138)
\curveto(515.93228611,52.71350533)(517.93228411,53.74996263)(520.546875,53.74998638)
}
}
{
\newrgbcolor{curcolor}{0 0 0}
\pscustom[linestyle=none,fillstyle=solid,fillcolor=curcolor]
{
\newpath
\moveto(144.32910156,484.25488254)
\lineto(144.32910156,478.22851535)
\lineto(147.05761719,478.22851535)
\curveto(148.06737475,478.22850712)(148.84797292,478.48990009)(149.39941406,479.01269504)
\curveto(149.9508364,479.53547196)(150.22655227,480.28026289)(150.2265625,481.24707004)
\curveto(150.22655227,482.20669325)(149.9508364,482.94790345)(149.39941406,483.47070285)
\curveto(148.84797292,483.99347532)(148.06737475,484.25486829)(147.05761719,484.25488254)
\lineto(144.32910156,484.25488254)
\moveto(142.15917969,486.03808567)
\lineto(147.05761719,486.03808567)
\curveto(148.85513437,486.03806963)(150.21222937,485.62986691)(151.12890625,484.81347629)
\curveto(152.05272232,484.00421749)(152.51463592,482.8154166)(152.51464844,481.24707004)
\curveto(152.51463592,479.66437808)(152.05272232,478.46841574)(151.12890625,477.65917942)
\curveto(150.21222937,476.84992777)(148.85513437,476.44530578)(147.05761719,476.44531223)
\lineto(144.32910156,476.44531223)
\lineto(144.32910156,469.99999973)
\lineto(142.15917969,469.99999973)
\lineto(142.15917969,486.03808567)
}
}
{
\newrgbcolor{curcolor}{0 0 0}
\pscustom[linestyle=none,fillstyle=solid,fillcolor=curcolor]
{
\newpath
\moveto(162.63378906,481.67675754)
\lineto(162.63378906,479.80761692)
\curveto(162.07518613,480.09406516)(161.49510858,480.30890869)(160.89355469,480.45214817)
\curveto(160.29198479,480.59536674)(159.66893853,480.66698125)(159.02441406,480.66699192)
\curveto(158.04328912,480.66698125)(157.30565965,480.51659077)(156.81152344,480.21582004)
\curveto(156.32454084,479.91502888)(156.0810515,479.46385745)(156.08105469,478.86230442)
\curveto(156.0810515,478.40396268)(156.25650705,478.04230939)(156.60742188,477.77734348)
\curveto(156.95832927,477.51952346)(157.66373221,477.27245339)(158.72363281,477.03613254)
\lineto(159.40039062,476.88574192)
\curveto(160.80402854,476.58495408)(161.79947026,476.15884774)(162.38671875,475.6074216)
\curveto(162.9811097,475.06314571)(163.27830992,474.30045116)(163.27832031,473.31933567)
\curveto(163.27830992,472.20214596)(162.83429995,471.31770674)(161.94628906,470.66601535)
\curveto(161.06542151,470.01432263)(159.85155554,469.6884766)(158.3046875,469.68847629)
\curveto(157.66015148,469.6884766)(156.98697507,469.75292966)(156.28515625,469.88183567)
\curveto(155.59049209,470.00358045)(154.85644335,470.18977818)(154.08300781,470.44042942)
\lineto(154.08300781,472.48144504)
\curveto(154.81347464,472.10188565)(155.53320048,471.8154276)(156.2421875,471.62207004)
\curveto(156.95116781,471.43587069)(157.65299003,471.34277182)(158.34765625,471.34277317)
\curveto(159.27863945,471.34277182)(159.99478456,471.50032375)(160.49609375,471.81542942)
\curveto(160.99738773,472.1376929)(161.24803852,472.58886433)(161.24804688,473.16894504)
\curveto(161.24803852,473.70605071)(161.06542151,474.11783415)(160.70019531,474.4042966)
\curveto(160.34211494,474.69075025)(159.55077459,474.96646612)(158.32617188,475.23144504)
\lineto(157.63867188,475.39257785)
\curveto(156.41405898,475.6503847)(155.52961976,476.04426452)(154.98535156,476.57421848)
\curveto(154.44107918,477.11132074)(154.16894403,477.84536949)(154.16894531,478.77636692)
\curveto(154.16894403,479.90786742)(154.5699853,480.78156447)(155.37207031,481.39746067)
\curveto(156.17415036,482.01333407)(157.3128211,482.32127647)(158.78808594,482.32128879)
\curveto(159.51854806,482.32127647)(160.20604737,482.26756559)(160.85058594,482.16015598)
\curveto(161.49510858,482.05272205)(162.08950903,481.8915894)(162.63378906,481.67675754)
}
}
{
\newrgbcolor{curcolor}{0 0 0}
\pscustom[linestyle=none,fillstyle=solid,fillcolor=curcolor]
{
\newpath
\moveto(176.72753906,476.50976535)
\lineto(176.72753906,475.54296848)
\lineto(167.63964844,475.54296848)
\curveto(167.72558258,474.18228721)(168.13378529,473.14387679)(168.86425781,472.4277341)
\curveto(169.60188278,471.71874801)(170.6259703,471.36425618)(171.93652344,471.36425754)
\curveto(172.69562969,471.36425618)(173.42967843,471.45735504)(174.13867188,471.64355442)
\curveto(174.85480722,471.8297505)(175.56379088,472.1090471)(176.265625,472.48144504)
\lineto(176.265625,470.61230442)
\curveto(175.55662943,470.31152285)(174.82974214,470.08235642)(174.08496094,469.92480442)
\curveto(173.34016029,469.76725257)(172.5846272,469.6884766)(171.81835938,469.68847629)
\curveto(169.89908301,469.6884766)(168.37727463,470.24706979)(167.25292969,471.36425754)
\curveto(166.13574042,472.48144256)(165.57714722,473.99250876)(165.57714844,475.89746067)
\curveto(165.57714722,477.86685384)(166.10709461,479.4280502)(167.16699219,480.58105442)
\curveto(168.23404561,481.74119892)(169.66991657,482.32127647)(171.47460938,482.32128879)
\curveto(173.09309023,482.32127647)(174.37140926,481.79849053)(175.30957031,480.75292942)
\curveto(176.25487092,479.71450824)(176.7275267,478.30012164)(176.72753906,476.50976535)
\moveto(174.75097656,477.08984348)
\curveto(174.73664327,478.17121552)(174.4322816,479.03417038)(173.83789062,479.67871067)
\curveto(173.25064215,480.32323159)(172.47004398,480.6454969)(171.49609375,480.64550754)
\curveto(170.39322314,480.6454969)(169.50878392,480.33397377)(168.84277344,479.71093723)
\curveto(168.18391545,479.08788127)(167.80435854,478.2106035)(167.70410156,477.07910129)
\lineto(174.75097656,477.08984348)
}
}
{
\newrgbcolor{curcolor}{0 0 0}
\pscustom[linestyle=none,fillstyle=solid,fillcolor=curcolor]
{
\newpath
\moveto(179.76757812,474.7480466)
\lineto(179.76757812,482.03124973)
\lineto(181.74414062,482.03124973)
\lineto(181.74414062,474.82324192)
\curveto(181.74413678,473.68456636)(181.96614177,472.82877294)(182.41015625,472.2558591)
\curveto(182.85416171,471.6901022)(183.52017667,471.40722488)(184.40820312,471.40722629)
\curveto(185.47525284,471.40722488)(186.31672335,471.74739381)(186.93261719,472.4277341)
\curveto(187.55565441,473.10806954)(187.86717753,474.03547746)(187.8671875,475.20996067)
\lineto(187.8671875,482.03124973)
\lineto(189.84375,482.03124973)
\lineto(189.84375,469.99999973)
\lineto(187.8671875,469.99999973)
\lineto(187.8671875,471.84765598)
\curveto(187.3873603,471.11718611)(186.82876711,470.57291582)(186.19140625,470.21484348)
\curveto(185.56119025,469.86393216)(184.82714151,469.6884766)(183.98925781,469.68847629)
\curveto(182.60709165,469.6884766)(181.55793905,470.11816367)(180.84179688,470.97753879)
\curveto(180.12564881,471.83691195)(179.76757626,473.09374663)(179.76757812,474.7480466)
\moveto(184.74121094,482.32128879)
\lineto(184.74121094,482.32128879)
}
}
{
\newrgbcolor{curcolor}{0 0 0}
\pscustom[linestyle=none,fillstyle=solid,fillcolor=curcolor]
{
\newpath
\moveto(201.85351562,480.20507785)
\lineto(201.85351562,486.71484348)
\lineto(203.83007812,486.71484348)
\lineto(203.83007812,469.99999973)
\lineto(201.85351562,469.99999973)
\lineto(201.85351562,471.80468723)
\curveto(201.43814147,471.08854031)(200.91177481,470.55501219)(200.27441406,470.20410129)
\curveto(199.64419795,469.86035143)(198.88508412,469.6884766)(197.99707031,469.68847629)
\curveto(196.54328959,469.6884766)(195.35806942,470.26855415)(194.44140625,471.42871067)
\curveto(193.53189937,472.58886433)(193.07714722,474.11425343)(193.07714844,476.00488254)
\curveto(193.07714722,477.89549965)(193.53189937,479.42088874)(194.44140625,480.58105442)
\curveto(195.35806942,481.74119892)(196.54328959,482.32127647)(197.99707031,482.32128879)
\curveto(198.88508412,482.32127647)(199.64419795,482.14582092)(200.27441406,481.7949216)
\curveto(200.91177481,481.45116015)(201.43814147,480.92121277)(201.85351562,480.20507785)
\moveto(195.11816406,476.00488254)
\curveto(195.11816081,474.55110195)(195.41536103,473.40885049)(196.00976562,472.57812473)
\curveto(196.61132338,471.75455527)(197.43489026,471.34277182)(198.48046875,471.34277317)
\curveto(199.526034,471.34277182)(200.34960089,471.75455527)(200.95117188,472.57812473)
\curveto(201.55272469,473.40885049)(201.85350563,474.55110195)(201.85351562,476.00488254)
\curveto(201.85350563,477.45865112)(201.55272469,478.59732186)(200.95117188,479.42089817)
\curveto(200.34960089,480.25161708)(199.526034,480.66698125)(198.48046875,480.66699192)
\curveto(197.43489026,480.66698125)(196.61132338,480.25161708)(196.00976562,479.42089817)
\curveto(195.41536103,478.59732186)(195.11816081,477.45865112)(195.11816406,476.00488254)
}
}
{
\newrgbcolor{curcolor}{0 0 0}
\pscustom[linestyle=none,fillstyle=solid,fillcolor=curcolor]
{
\newpath
\moveto(212.56347656,480.64550754)
\curveto(211.50357505,480.6454969)(210.66568527,480.23013273)(210.04980469,479.39941379)
\curveto(209.43391567,478.57583751)(209.12597326,477.44432822)(209.12597656,476.00488254)
\curveto(209.12597326,474.56542485)(209.43033494,473.43033484)(210.0390625,472.5996091)
\curveto(210.65494309,471.77603962)(211.4964136,471.36425618)(212.56347656,471.36425754)
\curveto(213.61620315,471.36425618)(214.45051221,471.77962034)(215.06640625,472.61035129)
\curveto(215.68228181,473.44107702)(215.99022421,474.5725863)(215.99023438,476.00488254)
\curveto(215.99022421,477.43000532)(215.68228181,478.55793388)(215.06640625,479.3886716)
\curveto(214.45051221,480.226552)(213.61620315,480.6454969)(212.56347656,480.64550754)
\moveto(212.56347656,482.32128879)
\curveto(214.28221811,482.32127647)(215.63215165,481.76268328)(216.61328125,480.64550754)
\curveto(217.59438928,479.52831051)(218.08494868,477.98143706)(218.08496094,476.00488254)
\curveto(218.08494868,474.03547746)(217.59438928,472.48860401)(216.61328125,471.36425754)
\curveto(215.63215165,470.24706979)(214.28221811,469.6884766)(212.56347656,469.68847629)
\curveto(210.83756009,469.6884766)(209.48404582,470.24706979)(208.50292969,471.36425754)
\curveto(207.52896965,472.48860401)(207.04199097,474.03547746)(207.04199219,476.00488254)
\curveto(207.04199097,477.98143706)(207.52896965,479.52831051)(208.50292969,480.64550754)
\curveto(209.48404582,481.76268328)(210.83756009,482.32127647)(212.56347656,482.32128879)
}
}
{
\newrgbcolor{curcolor}{0 0 0}
\pscustom[linestyle=none,fillstyle=solid,fillcolor=curcolor]
{
\newpath
\moveto(144.16894531,414.80273601)
\lineto(144.16894531,412.51465007)
\curveto(143.43846312,413.19497542)(142.65786495,413.70343845)(141.82714844,414.0400407)
\curveto(141.00356973,414.37661486)(140.12629196,414.54490897)(139.1953125,414.54492351)
\curveto(137.3619718,414.54490897)(135.95832738,413.98273505)(134.984375,412.85840007)
\curveto(134.01041266,411.74120083)(133.52343398,410.12271287)(133.5234375,408.00293132)
\curveto(133.52343398,405.89029522)(134.01041266,404.27180726)(134.984375,403.14746257)
\curveto(135.95832738,402.03027304)(137.3619718,401.47167985)(139.1953125,401.47168132)
\curveto(140.12629196,401.47167985)(141.00356973,401.63997395)(141.82714844,401.97656414)
\curveto(142.65786495,402.31315036)(143.43846312,402.8216134)(144.16894531,403.50195476)
\lineto(144.16894531,401.2353532)
\curveto(143.40981732,400.71972748)(142.60415406,400.33300912)(141.75195312,400.07519695)
\curveto(140.90689013,399.81738463)(140.01170874,399.68847851)(139.06640625,399.6884782)
\curveto(136.63866524,399.68847851)(134.72655777,400.42968871)(133.33007812,401.91211101)
\curveto(131.93359182,403.40169094)(131.23535033,405.43196235)(131.23535156,408.00293132)
\curveto(131.23535033,410.58104574)(131.93359182,412.61131715)(133.33007812,414.09375164)
\curveto(134.72655777,415.58331939)(136.63866524,416.32811031)(139.06640625,416.32812664)
\curveto(140.02603164,416.32811031)(140.92837449,416.19920419)(141.7734375,415.94140789)
\curveto(142.62563842,415.69074115)(143.42414022,415.31118424)(144.16894531,414.80273601)
}
}
{
\newrgbcolor{curcolor}{0 0 0}
\pscustom[linestyle=none,fillstyle=solid,fillcolor=curcolor]
{
\newpath
\moveto(152.11816406,410.64550945)
\curveto(151.05826255,410.6454988)(150.22037277,410.23013463)(149.60449219,409.3994157)
\curveto(148.98860317,408.57583941)(148.68066076,407.44433013)(148.68066406,406.00488445)
\curveto(148.68066076,404.56542676)(148.98502244,403.43033675)(149.59375,402.59961101)
\curveto(150.20963059,401.77604153)(151.0511011,401.36425808)(152.11816406,401.36425945)
\curveto(153.17089065,401.36425808)(154.00519971,401.77962225)(154.62109375,402.6103532)
\curveto(155.23696931,403.44107892)(155.54491171,404.57258821)(155.54492188,406.00488445)
\curveto(155.54491171,407.43000723)(155.23696931,408.55793579)(154.62109375,409.38867351)
\curveto(154.00519971,410.22655391)(153.17089065,410.6454988)(152.11816406,410.64550945)
\moveto(152.11816406,412.3212907)
\curveto(153.83690561,412.32127838)(155.18683915,411.76268519)(156.16796875,410.64550945)
\curveto(157.14907678,409.52831242)(157.63963618,407.98143897)(157.63964844,406.00488445)
\curveto(157.63963618,404.03547937)(157.14907678,402.48860592)(156.16796875,401.36425945)
\curveto(155.18683915,400.2470717)(153.83690561,399.68847851)(152.11816406,399.6884782)
\curveto(150.39224759,399.68847851)(149.03873332,400.2470717)(148.05761719,401.36425945)
\curveto(147.08365715,402.48860592)(146.59667847,404.03547937)(146.59667969,406.00488445)
\curveto(146.59667847,407.98143897)(147.08365715,409.52831242)(148.05761719,410.64550945)
\curveto(149.03873332,411.76268519)(150.39224759,412.32127838)(152.11816406,412.3212907)
}
}
{
\newrgbcolor{curcolor}{0 0 0}
\pscustom[linestyle=none,fillstyle=solid,fillcolor=curcolor]
{
\newpath
\moveto(160.70117188,404.74804851)
\lineto(160.70117188,412.03125164)
\lineto(162.67773438,412.03125164)
\lineto(162.67773438,404.82324382)
\curveto(162.67773053,403.68456826)(162.89973552,402.82877485)(163.34375,402.25586101)
\curveto(163.78775546,401.69010411)(164.45377042,401.40722679)(165.34179688,401.4072282)
\curveto(166.40884659,401.40722679)(167.2503171,401.74739572)(167.86621094,402.42773601)
\curveto(168.48924816,403.10807144)(168.80077128,404.03547937)(168.80078125,405.20996257)
\lineto(168.80078125,412.03125164)
\lineto(170.77734375,412.03125164)
\lineto(170.77734375,400.00000164)
\lineto(168.80078125,400.00000164)
\lineto(168.80078125,401.84765789)
\curveto(168.32095405,401.11718802)(167.76236086,400.57291773)(167.125,400.21484539)
\curveto(166.494784,399.86393406)(165.76073526,399.68847851)(164.92285156,399.6884782)
\curveto(163.5406854,399.68847851)(162.4915328,400.11816558)(161.77539062,400.9775407)
\curveto(161.05924256,401.83691386)(160.70117001,403.09374854)(160.70117188,404.74804851)
\moveto(165.67480469,412.3212907)
\lineto(165.67480469,412.3212907)
}
}
{
\newrgbcolor{curcolor}{0 0 0}
\pscustom[linestyle=none,fillstyle=solid,fillcolor=curcolor]
{
\newpath
\moveto(174.87011719,416.71484539)
\lineto(176.84667969,416.71484539)
\lineto(176.84667969,400.00000164)
\lineto(174.87011719,400.00000164)
\lineto(174.87011719,416.71484539)
}
}
{
\newrgbcolor{curcolor}{0 0 0}
\pscustom[linestyle=none,fillstyle=solid,fillcolor=curcolor]
{
\newpath
\moveto(191.26269531,406.50976726)
\lineto(191.26269531,405.54297039)
\lineto(182.17480469,405.54297039)
\curveto(182.26073883,404.18228912)(182.66894154,403.1438787)(183.39941406,402.42773601)
\curveto(184.13703903,401.71874992)(185.16112655,401.36425808)(186.47167969,401.36425945)
\curveto(187.23078594,401.36425808)(187.96483468,401.45735695)(188.67382812,401.64355632)
\curveto(189.38996347,401.82975241)(190.09894713,402.10904901)(190.80078125,402.48144695)
\lineto(190.80078125,400.61230632)
\curveto(190.09178568,400.31152476)(189.36489839,400.08235832)(188.62011719,399.92480632)
\curveto(187.87531654,399.76725447)(187.11978345,399.68847851)(186.35351562,399.6884782)
\curveto(184.43423926,399.68847851)(182.91243088,400.2470717)(181.78808594,401.36425945)
\curveto(180.67089667,402.48144447)(180.11230347,403.99251066)(180.11230469,405.89746257)
\curveto(180.11230347,407.86685575)(180.64225086,409.4280521)(181.70214844,410.58105632)
\curveto(182.76920186,411.74120083)(184.20507282,412.32127838)(186.00976562,412.3212907)
\curveto(187.62824648,412.32127838)(188.90656551,411.79849244)(189.84472656,410.75293132)
\curveto(190.79002717,409.71451015)(191.26268295,408.30012354)(191.26269531,406.50976726)
\moveto(189.28613281,407.08984539)
\curveto(189.27179952,408.17121742)(188.96743785,409.03417229)(188.37304688,409.67871257)
\curveto(187.7857984,410.3232335)(187.00520023,410.6454988)(186.03125,410.64550945)
\curveto(184.92837939,410.6454988)(184.04394017,410.33397568)(183.37792969,409.71093914)
\curveto(182.7190717,409.08788317)(182.33951479,408.2106054)(182.23925781,407.0791032)
\lineto(189.28613281,407.08984539)
}
}
{
\newrgbcolor{curcolor}{0 0 0}
\pscustom[linestyle=none,fillstyle=solid,fillcolor=curcolor]
{
\newpath
\moveto(194.30273438,404.74804851)
\lineto(194.30273438,412.03125164)
\lineto(196.27929688,412.03125164)
\lineto(196.27929688,404.82324382)
\curveto(196.27929303,403.68456826)(196.50129802,402.82877485)(196.9453125,402.25586101)
\curveto(197.38931796,401.69010411)(198.05533292,401.40722679)(198.94335938,401.4072282)
\curveto(200.01040909,401.40722679)(200.8518796,401.74739572)(201.46777344,402.42773601)
\curveto(202.09081066,403.10807144)(202.40233378,404.03547937)(202.40234375,405.20996257)
\lineto(202.40234375,412.03125164)
\lineto(204.37890625,412.03125164)
\lineto(204.37890625,400.00000164)
\lineto(202.40234375,400.00000164)
\lineto(202.40234375,401.84765789)
\curveto(201.92251655,401.11718802)(201.36392336,400.57291773)(200.7265625,400.21484539)
\curveto(200.0963465,399.86393406)(199.36229776,399.68847851)(198.52441406,399.6884782)
\curveto(197.1422479,399.68847851)(196.0930953,400.11816558)(195.37695312,400.9775407)
\curveto(194.66080506,401.83691386)(194.30273251,403.09374854)(194.30273438,404.74804851)
\moveto(199.27636719,412.3212907)
\lineto(199.27636719,412.3212907)
}
}
{
\newrgbcolor{curcolor}{0 0 0}
\pscustom[linestyle=none,fillstyle=solid,fillcolor=curcolor]
{
\newpath
\moveto(215.44335938,410.18359539)
\curveto(215.22134534,410.31249132)(214.977856,410.40559019)(214.71289062,410.46289226)
\curveto(214.45507007,410.52733486)(214.16861202,410.55956139)(213.85351562,410.55957195)
\curveto(212.73632179,410.55956139)(211.87694765,410.19432738)(211.27539062,409.46386882)
\curveto(210.6809853,408.74055279)(210.38378508,407.69856165)(210.38378906,406.33789226)
\lineto(210.38378906,400.00000164)
\lineto(208.39648438,400.00000164)
\lineto(208.39648438,412.03125164)
\lineto(210.38378906,412.03125164)
\lineto(210.38378906,410.16211101)
\curveto(210.79914925,410.89256887)(211.33983881,411.43325843)(212.00585938,411.78418132)
\curveto(212.67186873,412.1422421)(213.48111271,412.32127838)(214.43359375,412.3212907)
\curveto(214.56965329,412.32127838)(214.72004376,412.3105362)(214.88476562,412.28906414)
\curveto(215.04947052,412.27472894)(215.23208752,412.24966387)(215.43261719,412.21386882)
\lineto(215.44335938,410.18359539)
}
}
{
\newrgbcolor{curcolor}{0 0 0}
\pscustom[linestyle=none,fillstyle=solid,fillcolor=curcolor]
{
}
}
{
\newrgbcolor{curcolor}{0 0 0}
\pscustom[linestyle=none,fillstyle=solid,fillcolor=curcolor]
{
\newpath
\moveto(224.54199219,412.03125164)
\lineto(226.51855469,412.03125164)
\lineto(226.51855469,399.78515789)
\curveto(226.51855064,398.25260755)(226.22493114,397.14258262)(225.63769531,396.45507976)
\curveto(225.0576146,395.76758399)(224.1194645,395.42383434)(222.82324219,395.42382976)
\lineto(222.07128906,395.42382976)
\lineto(222.07128906,397.09961101)
\lineto(222.59765625,397.09961101)
\curveto(223.34960849,397.09961391)(223.86165225,397.27506946)(224.13378906,397.6259782)
\curveto(224.40592254,397.96973023)(224.54199011,398.68945607)(224.54199219,399.78515789)
\lineto(224.54199219,412.03125164)
\moveto(224.54199219,416.71484539)
\lineto(226.51855469,416.71484539)
\lineto(226.51855469,414.2119157)
\lineto(224.54199219,414.2119157)
\lineto(224.54199219,416.71484539)
}
}
{
\newrgbcolor{curcolor}{0 0 0}
\pscustom[linestyle=none,fillstyle=solid,fillcolor=curcolor]
{
\newpath
\moveto(235.30566406,410.64550945)
\curveto(234.24576255,410.6454988)(233.40787277,410.23013463)(232.79199219,409.3994157)
\curveto(232.17610317,408.57583941)(231.86816076,407.44433013)(231.86816406,406.00488445)
\curveto(231.86816076,404.56542676)(232.17252244,403.43033675)(232.78125,402.59961101)
\curveto(233.39713059,401.77604153)(234.2386011,401.36425808)(235.30566406,401.36425945)
\curveto(236.35839065,401.36425808)(237.19269971,401.77962225)(237.80859375,402.6103532)
\curveto(238.42446931,403.44107892)(238.73241171,404.57258821)(238.73242188,406.00488445)
\curveto(238.73241171,407.43000723)(238.42446931,408.55793579)(237.80859375,409.38867351)
\curveto(237.19269971,410.22655391)(236.35839065,410.6454988)(235.30566406,410.64550945)
\moveto(235.30566406,412.3212907)
\curveto(237.02440561,412.32127838)(238.37433915,411.76268519)(239.35546875,410.64550945)
\curveto(240.33657678,409.52831242)(240.82713618,407.98143897)(240.82714844,406.00488445)
\curveto(240.82713618,404.03547937)(240.33657678,402.48860592)(239.35546875,401.36425945)
\curveto(238.37433915,400.2470717)(237.02440561,399.68847851)(235.30566406,399.6884782)
\curveto(233.57974759,399.68847851)(232.22623332,400.2470717)(231.24511719,401.36425945)
\curveto(230.27115715,402.48860592)(229.78417847,404.03547937)(229.78417969,406.00488445)
\curveto(229.78417847,407.98143897)(230.27115715,409.52831242)(231.24511719,410.64550945)
\curveto(232.22623332,411.76268519)(233.57974759,412.32127838)(235.30566406,412.3212907)
}
}
{
\newrgbcolor{curcolor}{0 0 0}
\pscustom[linestyle=none,fillstyle=solid,fillcolor=curcolor]
{
\newpath
\moveto(243.88867188,404.74804851)
\lineto(243.88867188,412.03125164)
\lineto(245.86523438,412.03125164)
\lineto(245.86523438,404.82324382)
\curveto(245.86523053,403.68456826)(246.08723552,402.82877485)(246.53125,402.25586101)
\curveto(246.97525546,401.69010411)(247.64127042,401.40722679)(248.52929688,401.4072282)
\curveto(249.59634659,401.40722679)(250.4378171,401.74739572)(251.05371094,402.42773601)
\curveto(251.67674816,403.10807144)(251.98827128,404.03547937)(251.98828125,405.20996257)
\lineto(251.98828125,412.03125164)
\lineto(253.96484375,412.03125164)
\lineto(253.96484375,400.00000164)
\lineto(251.98828125,400.00000164)
\lineto(251.98828125,401.84765789)
\curveto(251.50845405,401.11718802)(250.94986086,400.57291773)(250.3125,400.21484539)
\curveto(249.682284,399.86393406)(248.94823526,399.68847851)(248.11035156,399.6884782)
\curveto(246.7281854,399.68847851)(245.6790328,400.11816558)(244.96289062,400.9775407)
\curveto(244.24674256,401.83691386)(243.88867001,403.09374854)(243.88867188,404.74804851)
\moveto(248.86230469,412.3212907)
\lineto(248.86230469,412.3212907)
}
}
{
\newrgbcolor{curcolor}{0 0 0}
\pscustom[linestyle=none,fillstyle=solid,fillcolor=curcolor]
{
\newpath
\moveto(268.34863281,406.50976726)
\lineto(268.34863281,405.54297039)
\lineto(259.26074219,405.54297039)
\curveto(259.34667633,404.18228912)(259.75487904,403.1438787)(260.48535156,402.42773601)
\curveto(261.22297653,401.71874992)(262.24706405,401.36425808)(263.55761719,401.36425945)
\curveto(264.31672344,401.36425808)(265.05077218,401.45735695)(265.75976562,401.64355632)
\curveto(266.47590097,401.82975241)(267.18488463,402.10904901)(267.88671875,402.48144695)
\lineto(267.88671875,400.61230632)
\curveto(267.17772318,400.31152476)(266.45083589,400.08235832)(265.70605469,399.92480632)
\curveto(264.96125404,399.76725447)(264.20572095,399.68847851)(263.43945312,399.6884782)
\curveto(261.52017676,399.68847851)(259.99836838,400.2470717)(258.87402344,401.36425945)
\curveto(257.75683417,402.48144447)(257.19824097,403.99251066)(257.19824219,405.89746257)
\curveto(257.19824097,407.86685575)(257.72818836,409.4280521)(258.78808594,410.58105632)
\curveto(259.85513936,411.74120083)(261.29101032,412.32127838)(263.09570312,412.3212907)
\curveto(264.71418398,412.32127838)(265.99250301,411.79849244)(266.93066406,410.75293132)
\curveto(267.87596467,409.71451015)(268.34862045,408.30012354)(268.34863281,406.50976726)
\moveto(266.37207031,407.08984539)
\curveto(266.35773702,408.17121742)(266.05337535,409.03417229)(265.45898438,409.67871257)
\curveto(264.8717359,410.3232335)(264.09113773,410.6454988)(263.1171875,410.64550945)
\curveto(262.01431689,410.6454988)(261.12987767,410.33397568)(260.46386719,409.71093914)
\curveto(259.8050092,409.08788317)(259.42545229,408.2106054)(259.32519531,407.0791032)
\lineto(266.37207031,407.08984539)
}
}
{
\newrgbcolor{curcolor}{0 0 0}
\pscustom[linestyle=none,fillstyle=solid,fillcolor=curcolor]
{
\newpath
\moveto(271.38867188,404.74804851)
\lineto(271.38867188,412.03125164)
\lineto(273.36523438,412.03125164)
\lineto(273.36523438,404.82324382)
\curveto(273.36523053,403.68456826)(273.58723552,402.82877485)(274.03125,402.25586101)
\curveto(274.47525546,401.69010411)(275.14127042,401.40722679)(276.02929688,401.4072282)
\curveto(277.09634659,401.40722679)(277.9378171,401.74739572)(278.55371094,402.42773601)
\curveto(279.17674816,403.10807144)(279.48827128,404.03547937)(279.48828125,405.20996257)
\lineto(279.48828125,412.03125164)
\lineto(281.46484375,412.03125164)
\lineto(281.46484375,400.00000164)
\lineto(279.48828125,400.00000164)
\lineto(279.48828125,401.84765789)
\curveto(279.00845405,401.11718802)(278.44986086,400.57291773)(277.8125,400.21484539)
\curveto(277.182284,399.86393406)(276.44823526,399.68847851)(275.61035156,399.6884782)
\curveto(274.2281854,399.68847851)(273.1790328,400.11816558)(272.46289062,400.9775407)
\curveto(271.74674256,401.83691386)(271.38867001,403.09374854)(271.38867188,404.74804851)
\moveto(276.36230469,412.3212907)
\lineto(276.36230469,412.3212907)
}
}
{
\newrgbcolor{curcolor}{0 0 0}
\pscustom[linestyle=none,fillstyle=solid,fillcolor=curcolor]
{
\newpath
\moveto(292.52929688,410.18359539)
\curveto(292.30728284,410.31249132)(292.0637935,410.40559019)(291.79882812,410.46289226)
\curveto(291.54100757,410.52733486)(291.25454952,410.55956139)(290.93945312,410.55957195)
\curveto(289.82225929,410.55956139)(288.96288515,410.19432738)(288.36132812,409.46386882)
\curveto(287.7669228,408.74055279)(287.46972258,407.69856165)(287.46972656,406.33789226)
\lineto(287.46972656,400.00000164)
\lineto(285.48242188,400.00000164)
\lineto(285.48242188,412.03125164)
\lineto(287.46972656,412.03125164)
\lineto(287.46972656,410.16211101)
\curveto(287.88508675,410.89256887)(288.42577631,411.43325843)(289.09179688,411.78418132)
\curveto(289.75780623,412.1422421)(290.56705021,412.32127838)(291.51953125,412.3212907)
\curveto(291.65559079,412.32127838)(291.80598126,412.3105362)(291.97070312,412.28906414)
\curveto(292.13540802,412.27472894)(292.31802502,412.24966387)(292.51855469,412.21386882)
\lineto(292.52929688,410.18359539)
}
}
{
\newrgbcolor{curcolor}{0 0 0}
\pscustom[linestyle=none,fillstyle=solid,fillcolor=curcolor]
{
\newpath
\moveto(176.48535156,316.03808757)
\lineto(179.40722656,316.03808757)
\lineto(186.51855469,302.62109539)
\lineto(186.51855469,316.03808757)
\lineto(188.62402344,316.03808757)
\lineto(188.62402344,300.00000164)
\lineto(185.70214844,300.00000164)
\lineto(178.59082031,313.41699382)
\lineto(178.59082031,300.00000164)
\lineto(176.48535156,300.00000164)
\lineto(176.48535156,316.03808757)
}
}
{
\newrgbcolor{curcolor}{0 0 0}
\pscustom[linestyle=none,fillstyle=solid,fillcolor=curcolor]
{
\newpath
\moveto(197.51855469,310.64550945)
\curveto(196.45865318,310.6454988)(195.62076339,310.23013463)(195.00488281,309.3994157)
\curveto(194.38899379,308.57583941)(194.08105139,307.44433013)(194.08105469,306.00488445)
\curveto(194.08105139,304.56542676)(194.38541306,303.43033675)(194.99414062,302.59961101)
\curveto(195.61002121,301.77604153)(196.45149173,301.36425808)(197.51855469,301.36425945)
\curveto(198.57128127,301.36425808)(199.40559034,301.77962225)(200.02148438,302.6103532)
\curveto(200.63735994,303.44107892)(200.94530234,304.57258821)(200.9453125,306.00488445)
\curveto(200.94530234,307.43000723)(200.63735994,308.55793579)(200.02148438,309.38867351)
\curveto(199.40559034,310.22655391)(198.57128127,310.6454988)(197.51855469,310.64550945)
\moveto(197.51855469,312.3212907)
\curveto(199.23729623,312.32127838)(200.58722978,311.76268519)(201.56835938,310.64550945)
\curveto(202.5494674,309.52831242)(203.04002681,307.98143897)(203.04003906,306.00488445)
\curveto(203.04002681,304.03547937)(202.5494674,302.48860592)(201.56835938,301.36425945)
\curveto(200.58722978,300.2470717)(199.23729623,299.68847851)(197.51855469,299.6884782)
\curveto(195.79263822,299.68847851)(194.43912395,300.2470717)(193.45800781,301.36425945)
\curveto(192.48404778,302.48860592)(191.9970691,304.03547937)(191.99707031,306.00488445)
\curveto(191.9970691,307.98143897)(192.48404778,309.52831242)(193.45800781,310.64550945)
\curveto(194.43912395,311.76268519)(195.79263822,312.32127838)(197.51855469,312.3212907)
}
}
{
\newrgbcolor{curcolor}{0 0 0}
\pscustom[linestyle=none,fillstyle=solid,fillcolor=curcolor]
{
\newpath
\moveto(215.67285156,309.72168132)
\curveto(216.16698025,310.60969155)(216.75779997,311.26496433)(217.4453125,311.68750164)
\curveto(218.1327986,312.11001557)(218.94204258,312.32127838)(219.87304688,312.3212907)
\curveto(221.12628519,312.32127838)(222.0930811,311.88084913)(222.7734375,311.00000164)
\curveto(223.45375682,310.12629359)(223.79392575,308.88020109)(223.79394531,307.26172039)
\lineto(223.79394531,300.00000164)
\lineto(221.80664062,300.00000164)
\lineto(221.80664062,307.19726726)
\curveto(221.80662305,308.3502537)(221.60252169,309.20604712)(221.19433594,309.76465007)
\curveto(220.78611626,310.3232335)(220.16307001,310.6025301)(219.32519531,310.6025407)
\curveto(218.3010927,310.6025301)(217.49184872,310.26236117)(216.89746094,309.58203289)
\curveto(216.30304783,308.90168544)(216.0058476,307.97427752)(216.00585938,306.79980632)
\lineto(216.00585938,300.00000164)
\lineto(214.01855469,300.00000164)
\lineto(214.01855469,307.19726726)
\curveto(214.0185449,308.35741515)(213.81444354,309.21320857)(213.40625,309.76465007)
\curveto(212.99803811,310.3232335)(212.36783041,310.6025301)(211.515625,310.6025407)
\curveto(210.5058531,310.6025301)(209.70377057,310.25878044)(209.109375,309.5712907)
\curveto(208.51496968,308.89094327)(208.21776945,307.96711606)(208.21777344,306.79980632)
\lineto(208.21777344,300.00000164)
\lineto(206.23046875,300.00000164)
\lineto(206.23046875,312.03125164)
\lineto(208.21777344,312.03125164)
\lineto(208.21777344,310.16211101)
\curveto(208.66894088,310.89973032)(209.20963044,311.44400061)(209.83984375,311.79492351)
\curveto(210.47004585,312.14582282)(211.21841749,312.32127838)(212.08496094,312.3212907)
\curveto(212.95865013,312.32127838)(213.69986032,312.09927339)(214.30859375,311.65527507)
\curveto(214.92446847,311.21125345)(215.37922062,310.56672284)(215.67285156,309.72168132)
}
}
{
\newrgbcolor{curcolor}{0 0 0}
\pscustom[linestyle=none,fillstyle=solid,fillcolor=curcolor]
{
\newpath
\moveto(144.44140625,282.70507976)
\lineto(144.44140625,289.21484539)
\lineto(146.41796875,289.21484539)
\lineto(146.41796875,272.50000164)
\lineto(144.44140625,272.50000164)
\lineto(144.44140625,274.30468914)
\curveto(144.02603209,273.58854221)(143.49966543,273.0550141)(142.86230469,272.7041032)
\curveto(142.23208857,272.36035334)(141.47297475,272.18847851)(140.58496094,272.1884782)
\curveto(139.13118022,272.18847851)(137.94596005,272.76855605)(137.02929688,273.92871257)
\curveto(136.11979,275.08886623)(135.66503785,276.61425533)(135.66503906,278.50488445)
\curveto(135.66503785,280.39550155)(136.11979,281.92089065)(137.02929688,283.08105632)
\curveto(137.94596005,284.24120083)(139.13118022,284.82127838)(140.58496094,284.8212907)
\curveto(141.47297475,284.82127838)(142.23208857,284.64582282)(142.86230469,284.29492351)
\curveto(143.49966543,283.95116206)(144.02603209,283.42121467)(144.44140625,282.70507976)
\moveto(137.70605469,278.50488445)
\curveto(137.70605143,277.05110386)(138.00325166,275.90885239)(138.59765625,275.07812664)
\curveto(139.199214,274.25455717)(140.02278089,273.84277373)(141.06835938,273.84277507)
\curveto(142.11392463,273.84277373)(142.93749151,274.25455717)(143.5390625,275.07812664)
\curveto(144.14061531,275.90885239)(144.44139626,277.05110386)(144.44140625,278.50488445)
\curveto(144.44139626,279.95865303)(144.14061531,281.09732377)(143.5390625,281.92090007)
\curveto(142.93749151,282.75161899)(142.11392463,283.16698316)(141.06835938,283.16699382)
\curveto(140.02278089,283.16698316)(139.199214,282.75161899)(138.59765625,281.92090007)
\curveto(138.00325166,281.09732377)(137.70605143,279.95865303)(137.70605469,278.50488445)
}
}
{
\newrgbcolor{curcolor}{0 0 0}
\pscustom[linestyle=none,fillstyle=solid,fillcolor=curcolor]
{
\newpath
\moveto(152.35839844,288.53808757)
\lineto(152.35839844,282.57617351)
\lineto(150.53222656,282.57617351)
\lineto(150.53222656,288.53808757)
\lineto(152.35839844,288.53808757)
}
}
{
\newrgbcolor{curcolor}{0 0 0}
\pscustom[linestyle=none,fillstyle=solid,fillcolor=curcolor]
{
\newpath
\moveto(156.34375,277.24804851)
\lineto(156.34375,284.53125164)
\lineto(158.3203125,284.53125164)
\lineto(158.3203125,277.32324382)
\curveto(158.32030865,276.18456826)(158.54231364,275.32877485)(158.98632812,274.75586101)
\curveto(159.43033359,274.19010411)(160.09634854,273.90722679)(160.984375,273.9072282)
\curveto(162.05142471,273.90722679)(162.89289523,274.24739572)(163.50878906,274.92773601)
\curveto(164.13182628,275.60807144)(164.44334941,276.53547937)(164.44335938,277.70996257)
\lineto(164.44335938,284.53125164)
\lineto(166.41992188,284.53125164)
\lineto(166.41992188,272.50000164)
\lineto(164.44335938,272.50000164)
\lineto(164.44335938,274.34765789)
\curveto(163.96353218,273.61718802)(163.40493899,273.07291773)(162.76757812,272.71484539)
\curveto(162.13736213,272.36393406)(161.40331338,272.18847851)(160.56542969,272.1884782)
\curveto(159.18326352,272.18847851)(158.13411092,272.61816558)(157.41796875,273.4775407)
\curveto(156.70182069,274.33691386)(156.34374813,275.59374854)(156.34375,277.24804851)
\moveto(161.31738281,284.8212907)
\lineto(161.31738281,284.8212907)
}
}
{
\newrgbcolor{curcolor}{0 0 0}
\pscustom[linestyle=none,fillstyle=solid,fillcolor=curcolor]
{
\newpath
\moveto(172.46777344,287.94726726)
\lineto(172.46777344,284.53125164)
\lineto(176.5390625,284.53125164)
\lineto(176.5390625,282.99511882)
\lineto(172.46777344,282.99511882)
\lineto(172.46777344,276.46386882)
\curveto(172.46776941,275.48274605)(172.60025626,274.85253835)(172.86523438,274.57324382)
\curveto(173.13736509,274.29394515)(173.68521611,274.15429686)(174.50878906,274.15429851)
\lineto(176.5390625,274.15429851)
\lineto(176.5390625,272.50000164)
\lineto(174.50878906,272.50000164)
\curveto(172.98339389,272.50000164)(171.93066057,272.78287896)(171.35058594,273.34863445)
\curveto(170.77050548,273.92154969)(170.48046671,274.95996011)(170.48046875,276.46386882)
\lineto(170.48046875,282.99511882)
\lineto(169.03027344,282.99511882)
\lineto(169.03027344,284.53125164)
\lineto(170.48046875,284.53125164)
\lineto(170.48046875,287.94726726)
\lineto(172.46777344,287.94726726)
}
}
{
\newrgbcolor{curcolor}{0 0 0}
\pscustom[linestyle=none,fillstyle=solid,fillcolor=curcolor]
{
\newpath
\moveto(179.14941406,284.53125164)
\lineto(181.12597656,284.53125164)
\lineto(181.12597656,272.50000164)
\lineto(179.14941406,272.50000164)
\lineto(179.14941406,284.53125164)
\moveto(179.14941406,289.21484539)
\lineto(181.12597656,289.21484539)
\lineto(181.12597656,286.7119157)
\lineto(179.14941406,286.7119157)
\lineto(179.14941406,289.21484539)
}
}
{
\newrgbcolor{curcolor}{0 0 0}
\pscustom[linestyle=none,fillstyle=solid,fillcolor=curcolor]
{
\newpath
\moveto(185.25097656,289.21484539)
\lineto(187.22753906,289.21484539)
\lineto(187.22753906,272.50000164)
\lineto(185.25097656,272.50000164)
\lineto(185.25097656,289.21484539)
}
}
{
\newrgbcolor{curcolor}{0 0 0}
\pscustom[linestyle=none,fillstyle=solid,fillcolor=curcolor]
{
\newpath
\moveto(191.35253906,284.53125164)
\lineto(193.32910156,284.53125164)
\lineto(193.32910156,272.50000164)
\lineto(191.35253906,272.50000164)
\lineto(191.35253906,284.53125164)
\moveto(191.35253906,289.21484539)
\lineto(193.32910156,289.21484539)
\lineto(193.32910156,286.7119157)
\lineto(191.35253906,286.7119157)
\lineto(191.35253906,289.21484539)
}
}
{
\newrgbcolor{curcolor}{0 0 0}
\pscustom[linestyle=none,fillstyle=solid,fillcolor=curcolor]
{
\newpath
\moveto(205.12402344,284.17675945)
\lineto(205.12402344,282.30761882)
\curveto(204.5654205,282.59406706)(203.98534296,282.8089106)(203.38378906,282.95215007)
\curveto(202.78221916,283.09536864)(202.15917291,283.16698316)(201.51464844,283.16699382)
\curveto(200.53352349,283.16698316)(199.79589402,283.01659268)(199.30175781,282.71582195)
\curveto(198.81477521,282.41503078)(198.57128587,281.96385936)(198.57128906,281.36230632)
\curveto(198.57128587,280.90396459)(198.74674143,280.5423113)(199.09765625,280.27734539)
\curveto(199.44856364,280.01952537)(200.15396658,279.7724553)(201.21386719,279.53613445)
\lineto(201.890625,279.38574382)
\curveto(203.29426292,279.08495599)(204.28970463,278.65884964)(204.87695312,278.10742351)
\curveto(205.47134408,277.56314761)(205.7685443,276.80045306)(205.76855469,275.81933757)
\curveto(205.7685443,274.70214787)(205.32453433,273.81770865)(204.43652344,273.16601726)
\curveto(203.55565589,272.51432454)(202.34178991,272.18847851)(200.79492188,272.1884782)
\curveto(200.15038586,272.18847851)(199.47720945,272.25293157)(198.77539062,272.38183757)
\curveto(198.08072647,272.50358236)(197.34667772,272.68978009)(196.57324219,272.94043132)
\lineto(196.57324219,274.98144695)
\curveto(197.30370901,274.60188755)(198.02343486,274.31542951)(198.73242188,274.12207195)
\curveto(199.44140219,273.9358726)(200.1432244,273.84277373)(200.83789062,273.84277507)
\curveto(201.76887382,273.84277373)(202.48501894,274.00032566)(202.98632812,274.31543132)
\curveto(203.4876221,274.63769481)(203.73827289,275.08886623)(203.73828125,275.66894695)
\curveto(203.73827289,276.20605262)(203.55565589,276.61783606)(203.19042969,276.90429851)
\curveto(202.83234932,277.19075215)(202.04100896,277.46646802)(200.81640625,277.73144695)
\lineto(200.12890625,277.89257976)
\curveto(198.90429335,278.15038661)(198.01985413,278.54426642)(197.47558594,279.07422039)
\curveto(196.93131355,279.61132265)(196.65917841,280.34537139)(196.65917969,281.27636882)
\curveto(196.65917841,282.40786933)(197.06021967,283.28156638)(197.86230469,283.89746257)
\curveto(198.66438474,284.51333598)(199.80305547,284.82127838)(201.27832031,284.8212907)
\curveto(202.00878243,284.82127838)(202.69628175,284.76756749)(203.34082031,284.66015789)
\curveto(203.98534296,284.55272396)(204.57974341,284.39159131)(205.12402344,284.17675945)
}
}
{
\newrgbcolor{curcolor}{0 0 0}
\pscustom[linestyle=none,fillstyle=solid,fillcolor=curcolor]
{
\newpath
\moveto(214.39453125,278.5478532)
\curveto(212.7975201,278.54784715)(211.69107589,278.36523015)(211.07519531,278.00000164)
\curveto(210.45930629,277.63476213)(210.15136389,277.01171587)(210.15136719,276.13086101)
\curveto(210.15136389,275.42903517)(210.38053033,274.87044197)(210.83886719,274.45507976)
\curveto(211.30435753,274.04687509)(211.93456523,273.84277373)(212.72949219,273.84277507)
\curveto(213.82518834,273.84277373)(214.70246611,274.22949209)(215.36132812,275.00293132)
\curveto(216.02733458,275.783527)(216.36034206,276.81835669)(216.36035156,278.10742351)
\lineto(216.36035156,278.5478532)
\lineto(214.39453125,278.5478532)
\moveto(218.33691406,279.36425945)
\lineto(218.33691406,272.50000164)
\lineto(216.36035156,272.50000164)
\lineto(216.36035156,274.32617351)
\curveto(215.90917063,273.59570367)(215.34699671,273.0550141)(214.67382812,272.7041032)
\curveto(214.00064389,272.36035334)(213.17707701,272.18847851)(212.203125,272.1884782)
\curveto(210.97135005,272.18847851)(209.99023124,272.53222817)(209.25976562,273.2197282)
\curveto(208.53645665,273.91438824)(208.17480337,274.84179617)(208.17480469,276.00195476)
\curveto(208.17480337,277.35546553)(208.62597479,278.37597232)(209.52832031,279.0634782)
\curveto(210.43782194,279.75097095)(211.79133621,280.0947206)(213.58886719,280.0947282)
\lineto(216.36035156,280.0947282)
\lineto(216.36035156,280.28808757)
\curveto(216.36034206,281.19758408)(216.05956111,281.8994063)(215.45800781,282.39355632)
\curveto(214.86359876,282.89484801)(214.02570897,283.1454988)(212.94433594,283.14550945)
\curveto(212.25683053,283.1454988)(211.58723485,283.06314211)(210.93554688,282.89843914)
\curveto(210.28385074,282.73371536)(209.65722376,282.48664529)(209.05566406,282.1572282)
\lineto(209.05566406,283.98340007)
\curveto(209.77896843,284.26268519)(210.48079064,284.47036727)(211.16113281,284.60644695)
\curveto(211.84146637,284.74966387)(212.5039006,284.82127838)(213.1484375,284.8212907)
\curveto(214.88866384,284.82127838)(216.18846723,284.37010695)(217.04785156,283.46777507)
\curveto(217.90721551,282.56542126)(218.33690258,281.19758408)(218.33691406,279.36425945)
}
}
{
\newrgbcolor{curcolor}{0 0 0}
\pscustom[linestyle=none,fillstyle=solid,fillcolor=curcolor]
{
\newpath
\moveto(224.37402344,287.94726726)
\lineto(224.37402344,284.53125164)
\lineto(228.4453125,284.53125164)
\lineto(228.4453125,282.99511882)
\lineto(224.37402344,282.99511882)
\lineto(224.37402344,276.46386882)
\curveto(224.37401941,275.48274605)(224.50650626,274.85253835)(224.77148438,274.57324382)
\curveto(225.04361509,274.29394515)(225.59146611,274.15429686)(226.41503906,274.15429851)
\lineto(228.4453125,274.15429851)
\lineto(228.4453125,272.50000164)
\lineto(226.41503906,272.50000164)
\curveto(224.88964389,272.50000164)(223.83691057,272.78287896)(223.25683594,273.34863445)
\curveto(222.67675548,273.92154969)(222.38671671,274.95996011)(222.38671875,276.46386882)
\lineto(222.38671875,282.99511882)
\lineto(220.93652344,282.99511882)
\lineto(220.93652344,284.53125164)
\lineto(222.38671875,284.53125164)
\lineto(222.38671875,287.94726726)
\lineto(224.37402344,287.94726726)
}
}
{
\newrgbcolor{curcolor}{0 0 0}
\pscustom[linestyle=none,fillstyle=solid,fillcolor=curcolor]
{
\newpath
\moveto(241.34667969,279.00976726)
\lineto(241.34667969,278.04297039)
\lineto(232.25878906,278.04297039)
\curveto(232.3447232,276.68228912)(232.75292592,275.6438787)(233.48339844,274.92773601)
\curveto(234.22102341,274.21874992)(235.24511092,273.86425808)(236.55566406,273.86425945)
\curveto(237.31477031,273.86425808)(238.04881906,273.95735695)(238.7578125,274.14355632)
\curveto(239.47394784,274.32975241)(240.18293151,274.60904901)(240.88476562,274.98144695)
\lineto(240.88476562,273.11230632)
\curveto(240.17577006,272.81152476)(239.44888276,272.58235832)(238.70410156,272.42480632)
\curveto(237.95930092,272.26725447)(237.20376782,272.18847851)(236.4375,272.1884782)
\curveto(234.51822363,272.18847851)(232.99641526,272.7470717)(231.87207031,273.86425945)
\curveto(230.75488104,274.98144447)(230.19628785,276.49251066)(230.19628906,278.39746257)
\curveto(230.19628785,280.36685575)(230.72623524,281.9280521)(231.78613281,283.08105632)
\curveto(232.85318623,284.24120083)(234.28905719,284.82127838)(236.09375,284.8212907)
\curveto(237.71223085,284.82127838)(238.99054989,284.29849244)(239.92871094,283.25293132)
\curveto(240.87401155,282.21451015)(241.34666732,280.80012354)(241.34667969,279.00976726)
\moveto(239.37011719,279.58984539)
\curveto(239.3557839,280.67121742)(239.05142222,281.53417229)(238.45703125,282.17871257)
\curveto(237.86978278,282.8232335)(237.0891846,283.1454988)(236.11523438,283.14550945)
\curveto(235.01236376,283.1454988)(234.12792454,282.83397568)(233.46191406,282.21093914)
\curveto(232.80305608,281.58788317)(232.42349916,280.7106054)(232.32324219,279.5791032)
\lineto(239.37011719,279.58984539)
}
}
{
\newrgbcolor{curcolor}{0 0 0}
\pscustom[linestyle=none,fillstyle=solid,fillcolor=curcolor]
{
\newpath
\moveto(244.38671875,277.24804851)
\lineto(244.38671875,284.53125164)
\lineto(246.36328125,284.53125164)
\lineto(246.36328125,277.32324382)
\curveto(246.3632774,276.18456826)(246.58528239,275.32877485)(247.02929688,274.75586101)
\curveto(247.47330234,274.19010411)(248.13931729,273.90722679)(249.02734375,273.9072282)
\curveto(250.09439346,273.90722679)(250.93586398,274.24739572)(251.55175781,274.92773601)
\curveto(252.17479503,275.60807144)(252.48631816,276.53547937)(252.48632812,277.70996257)
\lineto(252.48632812,284.53125164)
\lineto(254.46289062,284.53125164)
\lineto(254.46289062,272.50000164)
\lineto(252.48632812,272.50000164)
\lineto(252.48632812,274.34765789)
\curveto(252.00650093,273.61718802)(251.44790774,273.07291773)(250.81054688,272.71484539)
\curveto(250.18033088,272.36393406)(249.44628213,272.18847851)(248.60839844,272.1884782)
\curveto(247.22623227,272.18847851)(246.17707967,272.61816558)(245.4609375,273.4775407)
\curveto(244.74478944,274.33691386)(244.38671688,275.59374854)(244.38671875,277.24804851)
\moveto(249.36035156,284.8212907)
\lineto(249.36035156,284.8212907)
}
}
{
\newrgbcolor{curcolor}{0 0 0}
\pscustom[linestyle=none,fillstyle=solid,fillcolor=curcolor]
{
\newpath
\moveto(265.52734375,282.68359539)
\curveto(265.30532972,282.81249132)(265.06184038,282.90559019)(264.796875,282.96289226)
\curveto(264.53905444,283.02733486)(264.2525964,283.05956139)(263.9375,283.05957195)
\curveto(262.82030616,283.05956139)(261.96093202,282.69432738)(261.359375,281.96386882)
\curveto(260.76496968,281.24055279)(260.46776945,280.19856165)(260.46777344,278.83789226)
\lineto(260.46777344,272.50000164)
\lineto(258.48046875,272.50000164)
\lineto(258.48046875,284.53125164)
\lineto(260.46777344,284.53125164)
\lineto(260.46777344,282.66211101)
\curveto(260.88313362,283.39256887)(261.42382318,283.93325843)(262.08984375,284.28418132)
\curveto(262.7558531,284.6422421)(263.56509708,284.82127838)(264.51757812,284.8212907)
\curveto(264.65363766,284.82127838)(264.80402814,284.8105362)(264.96875,284.78906414)
\curveto(265.13345489,284.77472894)(265.3160719,284.74966387)(265.51660156,284.71386882)
\lineto(265.52734375,282.68359539)
}
}
{
\newrgbcolor{curcolor}{0 0 0}
\pscustom[linestyle=none,fillstyle=solid,fillcolor=curcolor]
{
\newpath
\moveto(173.31672668,253.13606426)
\lineto(173.31672668,250.84797832)
\curveto(172.5862445,251.52830367)(171.80564632,252.0367667)(170.97492981,252.37336895)
\curveto(170.1513511,252.70994311)(169.27407333,252.87823721)(168.34309387,252.87825176)
\curveto(166.50975318,252.87823721)(165.10610875,252.3160633)(164.13215637,251.19172832)
\curveto(163.15819403,250.07452908)(162.67121535,248.45604111)(162.67121887,246.33625957)
\curveto(162.67121535,244.22362347)(163.15819403,242.60513551)(164.13215637,241.48079082)
\curveto(165.10610875,240.36360129)(166.50975318,239.8050081)(168.34309387,239.80500957)
\curveto(169.27407333,239.8050081)(170.1513511,239.9733022)(170.97492981,240.30989238)
\curveto(171.80564632,240.64647861)(172.5862445,241.15494164)(173.31672668,241.83528301)
\lineto(173.31672668,239.56868145)
\curveto(172.55759869,239.05305573)(171.75193543,238.66633736)(170.8997345,238.4085252)
\curveto(170.05467151,238.15071288)(169.15949011,238.02180676)(168.21418762,238.02180645)
\curveto(165.78644661,238.02180676)(163.87433915,238.76301695)(162.4778595,240.24543926)
\curveto(161.08137319,241.73501919)(160.3831317,243.7652906)(160.38313293,246.33625957)
\curveto(160.3831317,248.91437399)(161.08137319,250.9446454)(162.4778595,252.42707988)
\curveto(163.87433915,253.91664763)(165.78644661,254.66143855)(168.21418762,254.66145488)
\curveto(169.17381301,254.66143855)(170.07615586,254.53253243)(170.92121887,254.27473613)
\curveto(171.77341979,254.0240694)(172.57192159,253.64451249)(173.31672668,253.13606426)
}
}
{
\newrgbcolor{curcolor}{0 0 0}
\pscustom[linestyle=none,fillstyle=solid,fillcolor=curcolor]
{
\newpath
\moveto(186.60481262,245.59504863)
\lineto(186.60481262,238.33332988)
\lineto(184.62825012,238.33332988)
\lineto(184.62825012,245.53059551)
\curveto(184.62824002,246.66925905)(184.40623504,247.52147174)(183.9622345,248.08723613)
\curveto(183.51821509,248.65298102)(182.85220013,248.93585834)(181.96418762,248.93586895)
\curveto(180.89712396,248.93585834)(180.05565345,248.59568941)(179.43977356,247.91536113)
\curveto(178.82388385,247.23501369)(178.51594145,246.30760576)(178.51594543,245.13313457)
\lineto(178.51594543,238.33332988)
\lineto(176.52864075,238.33332988)
\lineto(176.52864075,255.04817363)
\lineto(178.51594543,255.04817363)
\lineto(178.51594543,248.49543926)
\curveto(178.98859723,249.21873566)(179.54360969,249.75942523)(180.1809845,250.11750957)
\curveto(180.82550945,250.47557034)(181.56671965,250.65460662)(182.40461731,250.65461895)
\curveto(183.78676951,250.65460662)(184.83234138,250.22491955)(185.54133606,249.36555645)
\curveto(186.25030871,248.51333272)(186.60480055,247.25649804)(186.60481262,245.59504863)
}
}
{
\newrgbcolor{curcolor}{0 0 0}
\pscustom[linestyle=none,fillstyle=solid,fillcolor=curcolor]
{
\newpath
\moveto(196.03645325,244.38118145)
\curveto(194.43944209,244.3811754)(193.33299789,244.19855839)(192.71711731,243.83332988)
\curveto(192.10122829,243.46809037)(191.79328589,242.84504412)(191.79328918,241.96418926)
\curveto(191.79328589,241.26236341)(192.02245232,240.70377022)(192.48078918,240.28840801)
\curveto(192.94627953,239.88020334)(193.57648723,239.67610198)(194.37141418,239.67610332)
\curveto(195.46711034,239.67610198)(196.34438811,240.06282034)(197.00325012,240.83625957)
\curveto(197.66925657,241.61685525)(198.00226405,242.65168494)(198.00227356,243.94075176)
\lineto(198.00227356,244.38118145)
\lineto(196.03645325,244.38118145)
\moveto(199.97883606,245.1975877)
\lineto(199.97883606,238.33332988)
\lineto(198.00227356,238.33332988)
\lineto(198.00227356,240.15950176)
\curveto(197.55109263,239.42903191)(196.98891871,238.88834235)(196.31575012,238.53743145)
\curveto(195.64256589,238.19368158)(194.81899901,238.02180676)(193.845047,238.02180645)
\curveto(192.61327205,238.02180676)(191.63215324,238.36555641)(190.90168762,239.05305645)
\curveto(190.17837865,239.74771649)(189.81672536,240.67512442)(189.81672668,241.83528301)
\curveto(189.81672536,243.18879378)(190.26789679,244.20930057)(191.17024231,244.89680645)
\curveto(192.07974393,245.58429919)(193.43325821,245.92804885)(195.23078918,245.92805645)
\lineto(198.00227356,245.92805645)
\lineto(198.00227356,246.12141582)
\curveto(198.00226405,247.03091233)(197.7014831,247.73273455)(197.09992981,248.22688457)
\curveto(196.50552076,248.72817626)(195.66763097,248.97882705)(194.58625793,248.9788377)
\curveto(193.89875253,248.97882705)(193.22915685,248.89647036)(192.57746887,248.73176738)
\curveto(191.92577273,248.56704361)(191.29914576,248.31997354)(190.69758606,247.99055645)
\lineto(190.69758606,249.81672832)
\curveto(191.42089043,250.09601343)(192.12271264,250.30369552)(192.80305481,250.4397752)
\curveto(193.48338836,250.58299211)(194.1458226,250.65460662)(194.7903595,250.65461895)
\curveto(196.53058584,250.65460662)(197.83038922,250.2034352)(198.68977356,249.30110332)
\curveto(199.54913751,248.3987495)(199.97882458,247.03091233)(199.97883606,245.1975877)
}
}
{
\newrgbcolor{curcolor}{0 0 0}
\pscustom[linestyle=none,fillstyle=solid,fillcolor=curcolor]
{
\newpath
\moveto(214.06184387,245.59504863)
\lineto(214.06184387,238.33332988)
\lineto(212.08528137,238.33332988)
\lineto(212.08528137,245.53059551)
\curveto(212.08527127,246.66925905)(211.86326629,247.52147174)(211.41926575,248.08723613)
\curveto(210.97524634,248.65298102)(210.30923138,248.93585834)(209.42121887,248.93586895)
\curveto(208.35415521,248.93585834)(207.5126847,248.59568941)(206.89680481,247.91536113)
\curveto(206.2809151,247.23501369)(205.9729727,246.30760576)(205.97297668,245.13313457)
\lineto(205.97297668,238.33332988)
\lineto(203.985672,238.33332988)
\lineto(203.985672,250.36457988)
\lineto(205.97297668,250.36457988)
\lineto(205.97297668,248.49543926)
\curveto(206.44562848,249.21873566)(207.00064094,249.75942523)(207.63801575,250.11750957)
\curveto(208.2825407,250.47557034)(209.0237509,250.65460662)(209.86164856,250.65461895)
\curveto(211.24380076,250.65460662)(212.28937263,250.22491955)(212.99836731,249.36555645)
\curveto(213.70733996,248.51333272)(214.0618318,247.25649804)(214.06184387,245.59504863)
}
}
{
\newrgbcolor{curcolor}{0 0 0}
\pscustom[linestyle=none,fillstyle=solid,fillcolor=curcolor]
{
\newpath
\moveto(225.94270325,244.48860332)
\curveto(225.94269326,245.9208874)(225.64549303,247.03091233)(225.05110168,247.81868145)
\curveto(224.46385359,248.60643159)(223.63670598,249.0003114)(222.56965637,249.00032207)
\curveto(221.50975498,249.0003114)(220.68260737,248.60643159)(220.08821106,247.81868145)
\curveto(219.50096793,247.03091233)(219.20734843,245.9208874)(219.20735168,244.48860332)
\curveto(219.20734843,243.06346838)(219.50096793,241.95702418)(220.08821106,241.16926738)
\curveto(220.68260737,240.38150492)(221.50975498,239.9876251)(222.56965637,239.98762676)
\curveto(223.63670598,239.9876251)(224.46385359,240.38150492)(225.05110168,241.16926738)
\curveto(225.64549303,241.95702418)(225.94269326,243.06346838)(225.94270325,244.48860332)
\moveto(227.91926575,239.82649395)
\curveto(227.91925378,237.77831742)(227.46450163,236.25650904)(226.55500793,235.26106426)
\curveto(225.64549303,234.25846417)(224.25259078,233.75716258)(222.376297,233.75715801)
\curveto(221.68162981,233.75716258)(221.02635703,233.81087347)(220.41047668,233.91829082)
\curveto(219.79458743,234.01855555)(219.19660625,234.17610748)(218.61653137,234.39094707)
\lineto(218.61653137,236.31379863)
\curveto(219.19660625,235.9986968)(219.76952235,235.76594964)(220.33528137,235.61555645)
\curveto(220.90103163,235.46516869)(221.47752845,235.38997345)(222.06477356,235.38997051)
\curveto(223.36099011,235.38997345)(224.33136674,235.73014238)(224.97590637,236.41047832)
\curveto(225.62042795,237.08365665)(225.94269326,238.10416345)(225.94270325,239.47200176)
\lineto(225.94270325,240.44954082)
\curveto(225.53449054,239.74055504)(225.0117046,239.21060765)(224.37434387,238.85969707)
\curveto(223.7369663,238.50878544)(222.97427175,238.33332988)(222.08625793,238.33332988)
\curveto(220.61099286,238.33332988)(219.42219196,238.8955038)(218.51985168,240.01985332)
\curveto(217.61750627,241.14419947)(217.16633485,242.63378131)(217.16633606,244.48860332)
\curveto(217.16633485,246.35057447)(217.61750627,247.84373704)(218.51985168,248.96809551)
\curveto(219.42219196,250.09243271)(220.61099286,250.65460662)(222.08625793,250.65461895)
\curveto(222.97427175,250.65460662)(223.7369663,250.47915107)(224.37434387,250.12825176)
\curveto(225.0117046,249.77732886)(225.53449054,249.24738147)(225.94270325,248.53840801)
\lineto(225.94270325,250.36457988)
\lineto(227.91926575,250.36457988)
\lineto(227.91926575,239.82649395)
}
}
{
\newrgbcolor{curcolor}{0 0 0}
\pscustom[linestyle=none,fillstyle=solid,fillcolor=curcolor]
{
\newpath
\moveto(242.28157043,244.84309551)
\lineto(242.28157043,243.87629863)
\lineto(233.19367981,243.87629863)
\curveto(233.27961395,242.51561737)(233.68781666,241.47720695)(234.41828918,240.76106426)
\curveto(235.15591415,240.05207816)(236.18000167,239.69758633)(237.49055481,239.6975877)
\curveto(238.24966106,239.69758633)(238.98370981,239.7906852)(239.69270325,239.97688457)
\curveto(240.40883859,240.16308066)(241.11782225,240.44237725)(241.81965637,240.8147752)
\lineto(241.81965637,238.94563457)
\curveto(241.1106608,238.64485301)(240.38377351,238.41568657)(239.63899231,238.25813457)
\curveto(238.89419167,238.10058272)(238.13865857,238.02180676)(237.37239075,238.02180645)
\curveto(235.45311438,238.02180676)(233.931306,238.58039995)(232.80696106,239.6975877)
\curveto(231.68977179,240.81477271)(231.1311786,242.32583891)(231.13117981,244.23079082)
\curveto(231.1311786,246.200184)(231.66112598,247.76138035)(232.72102356,248.91438457)
\curveto(233.78807698,250.07452908)(235.22394794,250.65460662)(237.02864075,250.65461895)
\curveto(238.6471216,250.65460662)(239.92544063,250.13182069)(240.86360168,249.08625957)
\curveto(241.80890229,248.0478384)(242.28155807,246.63345179)(242.28157043,244.84309551)
\moveto(240.30500793,245.42317363)
\curveto(240.29067464,246.50454567)(239.98631297,247.36750054)(239.391922,248.01204082)
\curveto(238.80467353,248.65656175)(238.02407535,248.97882705)(237.05012512,248.9788377)
\curveto(235.94725451,248.97882705)(235.06281529,248.66730392)(234.39680481,248.04426738)
\curveto(233.73794682,247.42121142)(233.35838991,246.54393365)(233.25813293,245.41243145)
\lineto(240.30500793,245.42317363)
}
}
{
\newrgbcolor{curcolor}{0 0 0}
\pscustom[linestyle=none,fillstyle=solid,fillcolor=curcolor]
{
\newpath
\moveto(252.49739075,248.51692363)
\curveto(252.27537672,248.64581957)(252.03188738,248.73891844)(251.766922,248.79622051)
\curveto(251.50910144,248.86066311)(251.22264339,248.89288964)(250.907547,248.8929002)
\curveto(249.79035316,248.89288964)(248.93097902,248.52765563)(248.329422,247.79719707)
\curveto(247.73501667,247.07388104)(247.43781645,246.03188989)(247.43782043,244.67122051)
\lineto(247.43782043,238.33332988)
\lineto(245.45051575,238.33332988)
\lineto(245.45051575,250.36457988)
\lineto(247.43782043,250.36457988)
\lineto(247.43782043,248.49543926)
\curveto(247.85318062,249.22589712)(248.39387018,249.76658668)(249.05989075,250.11750957)
\curveto(249.7259001,250.47557034)(250.53514408,250.65460662)(251.48762512,250.65461895)
\curveto(251.62368466,250.65460662)(251.77407513,250.64386445)(251.938797,250.62239238)
\curveto(252.10350189,250.60805719)(252.28611889,250.58299211)(252.48664856,250.54719707)
\lineto(252.49739075,248.51692363)
}
}
{
\newrgbcolor{curcolor}{0 0 0}
\pscustom[linestyle=none,fillstyle=solid,fillcolor=curcolor]
{
\newpath
\moveto(156.99934387,221.03840801)
\lineto(156.99934387,227.54817363)
\lineto(158.97590637,227.54817363)
\lineto(158.97590637,210.83332988)
\lineto(156.99934387,210.83332988)
\lineto(156.99934387,212.63801738)
\curveto(156.58396971,211.92187046)(156.05760305,211.38834235)(155.42024231,211.03743145)
\curveto(154.7900262,210.69368158)(154.03091237,210.52180676)(153.14289856,210.52180645)
\curveto(151.68911784,210.52180676)(150.50389767,211.1018843)(149.5872345,212.26204082)
\curveto(148.67772762,213.42219448)(148.22297547,214.94758358)(148.22297668,216.8382127)
\curveto(148.22297547,218.7288298)(148.67772762,220.2542189)(149.5872345,221.41438457)
\curveto(150.50389767,222.57452908)(151.68911784,223.15460662)(153.14289856,223.15461895)
\curveto(154.03091237,223.15460662)(154.7900262,222.97915107)(155.42024231,222.62825176)
\curveto(156.05760305,222.28449031)(156.58396971,221.75454292)(156.99934387,221.03840801)
\moveto(150.26399231,216.8382127)
\curveto(150.26398905,215.3844321)(150.56118928,214.24218064)(151.15559387,213.41145488)
\curveto(151.75715162,212.58788542)(152.58071851,212.17610198)(153.626297,212.17610332)
\curveto(154.67186225,212.17610198)(155.49542914,212.58788542)(156.09700012,213.41145488)
\curveto(156.69855293,214.24218064)(156.99933388,215.3844321)(156.99934387,216.8382127)
\curveto(156.99933388,218.29198128)(156.69855293,219.43065201)(156.09700012,220.25422832)
\curveto(155.49542914,221.08494724)(154.67186225,221.5003114)(153.626297,221.50032207)
\curveto(152.58071851,221.5003114)(151.75715162,221.08494724)(151.15559387,220.25422832)
\curveto(150.56118928,219.43065201)(150.26398905,218.29198128)(150.26399231,216.8382127)
}
}
{
\newrgbcolor{curcolor}{0 0 0}
\pscustom[linestyle=none,fillstyle=solid,fillcolor=curcolor]
{
\newpath
\moveto(173.33821106,217.34309551)
\lineto(173.33821106,216.37629863)
\lineto(164.25032043,216.37629863)
\curveto(164.33625457,215.01561737)(164.74445729,213.97720695)(165.47492981,213.26106426)
\curveto(166.21255478,212.55207816)(167.2366423,212.19758633)(168.54719543,212.1975877)
\curveto(169.30630169,212.19758633)(170.04035043,212.2906852)(170.74934387,212.47688457)
\curveto(171.46547921,212.66308066)(172.17446288,212.94237725)(172.876297,213.3147752)
\lineto(172.876297,211.44563457)
\curveto(172.16730143,211.14485301)(171.44041413,210.91568657)(170.69563293,210.75813457)
\curveto(169.95083229,210.60058272)(169.19529919,210.52180676)(168.42903137,210.52180645)
\curveto(166.509755,210.52180676)(164.98794663,211.08039995)(163.86360168,212.1975877)
\curveto(162.74641241,213.31477271)(162.18781922,214.82583891)(162.18782043,216.73079082)
\curveto(162.18781922,218.700184)(162.71776661,220.26138035)(163.77766418,221.41438457)
\curveto(164.84471761,222.57452908)(166.28058857,223.15460662)(168.08528137,223.15461895)
\curveto(169.70376223,223.15460662)(170.98208126,222.63182069)(171.92024231,221.58625957)
\curveto(172.86554292,220.5478384)(173.3381987,219.13345179)(173.33821106,217.34309551)
\moveto(171.36164856,217.92317363)
\curveto(171.34731527,219.00454567)(171.04295359,219.86750054)(170.44856262,220.51204082)
\curveto(169.86131415,221.15656175)(169.08071597,221.47882705)(168.10676575,221.4788377)
\curveto(167.00389513,221.47882705)(166.11945591,221.16730392)(165.45344543,220.54426738)
\curveto(164.79458745,219.92121142)(164.41503054,219.04393365)(164.31477356,217.91243145)
\lineto(171.36164856,217.92317363)
}
}
{
\newrgbcolor{curcolor}{0 0 0}
\pscustom[linestyle=none,fillstyle=solid,fillcolor=curcolor]
{
}
}
{
\newrgbcolor{curcolor}{0 0 0}
\pscustom[linestyle=none,fillstyle=solid,fillcolor=curcolor]
{
\newpath
\moveto(192.24446106,222.40266582)
\lineto(192.24446106,220.55500957)
\curveto(191.68585714,220.86294225)(191.12368322,221.09210869)(190.55793762,221.24250957)
\curveto(189.99933539,221.40005109)(189.43358074,221.47882705)(188.860672,221.4788377)
\curveto(187.57876489,221.47882705)(186.58332318,221.07062433)(185.87434387,220.25422832)
\curveto(185.16535584,219.44497492)(184.81086401,218.30630418)(184.81086731,216.8382127)
\curveto(184.81086401,215.3701092)(185.16535584,214.22785774)(185.87434387,213.41145488)
\curveto(186.58332318,212.60220832)(187.57876489,212.19758633)(188.860672,212.1975877)
\curveto(189.43358074,212.19758633)(189.99933539,212.27278157)(190.55793762,212.42317363)
\curveto(191.12368322,212.58072397)(191.68585714,212.81347113)(192.24446106,213.12141582)
\lineto(192.24446106,211.29524395)
\curveto(191.69301859,211.03743124)(191.12010249,210.84407206)(190.52571106,210.71516582)
\curveto(189.93846305,210.58625982)(189.31183607,210.52180676)(188.64582825,210.52180645)
\curveto(186.83397397,210.52180676)(185.39452228,211.09114213)(184.32746887,212.22981426)
\curveto(183.26040983,213.3684836)(182.72688172,214.90461487)(182.72688293,216.8382127)
\curveto(182.72688172,218.80044431)(183.26399056,220.34373704)(184.33821106,221.46809551)
\curveto(185.41958736,222.59243271)(186.89842703,223.15460662)(188.7747345,223.15461895)
\curveto(189.38345058,223.15460662)(189.97785103,223.09015356)(190.55793762,222.96125957)
\curveto(191.13800612,222.83950277)(191.70018004,222.65330504)(192.24446106,222.40266582)
}
}
{
\newrgbcolor{curcolor}{0 0 0}
\pscustom[linestyle=none,fillstyle=solid,fillcolor=curcolor]
{
\newpath
\moveto(200.36555481,221.4788377)
\curveto(199.3056533,221.47882705)(198.46776351,221.06346288)(197.85188293,220.23274395)
\curveto(197.23599391,219.40916766)(196.92805151,218.27765838)(196.92805481,216.8382127)
\curveto(196.92805151,215.398755)(197.23241319,214.26366499)(197.84114075,213.43293926)
\curveto(198.45702134,212.60936977)(199.29849185,212.19758633)(200.36555481,212.1975877)
\curveto(201.4182814,212.19758633)(202.25259046,212.6129505)(202.8684845,213.44368145)
\curveto(203.48436006,214.27440717)(203.79230246,215.40591646)(203.79231262,216.8382127)
\curveto(203.79230246,218.26333547)(203.48436006,219.39126403)(202.8684845,220.22200176)
\curveto(202.25259046,221.05988216)(201.4182814,221.47882705)(200.36555481,221.4788377)
\moveto(200.36555481,223.15461895)
\curveto(202.08429636,223.15460662)(203.4342299,222.59601343)(204.4153595,221.4788377)
\curveto(205.39646752,220.36164067)(205.88702693,218.81476721)(205.88703918,216.8382127)
\curveto(205.88702693,214.86880762)(205.39646752,213.32193417)(204.4153595,212.1975877)
\curveto(203.4342299,211.08039995)(202.08429636,210.52180676)(200.36555481,210.52180645)
\curveto(198.63963834,210.52180676)(197.28612407,211.08039995)(196.30500793,212.1975877)
\curveto(195.3310479,213.32193417)(194.84406922,214.86880762)(194.84407043,216.8382127)
\curveto(194.84406922,218.81476721)(195.3310479,220.36164067)(196.30500793,221.4788377)
\curveto(197.28612407,222.59601343)(198.63963834,223.15460662)(200.36555481,223.15461895)
}
}
{
\newrgbcolor{curcolor}{0 0 0}
\pscustom[linestyle=none,fillstyle=solid,fillcolor=curcolor]
{
\newpath
\moveto(218.51985168,220.55500957)
\curveto(219.01398037,221.44301979)(219.6048001,222.09829258)(220.29231262,222.52082988)
\curveto(220.97979872,222.94334381)(221.7890427,223.15460662)(222.720047,223.15461895)
\curveto(223.97328531,223.15460662)(224.94008122,222.71417738)(225.62043762,221.83332988)
\curveto(226.30075694,220.95962184)(226.64092587,219.71352934)(226.64094543,218.09504863)
\lineto(226.64094543,210.83332988)
\lineto(224.65364075,210.83332988)
\lineto(224.65364075,218.03059551)
\curveto(224.65362317,219.18358195)(224.44952181,220.03937536)(224.04133606,220.59797832)
\curveto(223.63311638,221.15656175)(223.01007013,221.43585834)(222.17219543,221.43586895)
\curveto(221.14809282,221.43585834)(220.33884884,221.09568941)(219.74446106,220.41536113)
\curveto(219.15004795,219.73501369)(218.85284772,218.80760576)(218.8528595,217.63313457)
\lineto(218.8528595,210.83332988)
\lineto(216.86555481,210.83332988)
\lineto(216.86555481,218.03059551)
\curveto(216.86554502,219.1907434)(216.66144367,220.04653682)(216.25325012,220.59797832)
\curveto(215.84503823,221.15656175)(215.21483053,221.43585834)(214.36262512,221.43586895)
\curveto(213.35285322,221.43585834)(212.55077069,221.09210869)(211.95637512,220.40461895)
\curveto(211.3619698,219.72427151)(211.06476957,218.80044431)(211.06477356,217.63313457)
\lineto(211.06477356,210.83332988)
\lineto(209.07746887,210.83332988)
\lineto(209.07746887,222.86457988)
\lineto(211.06477356,222.86457988)
\lineto(211.06477356,220.99543926)
\curveto(211.515941,221.73305857)(212.05663056,222.27732886)(212.68684387,222.62825176)
\curveto(213.31704597,222.97915107)(214.06541762,223.15460662)(214.93196106,223.15461895)
\curveto(215.80565025,223.15460662)(216.54686045,222.93260164)(217.15559387,222.48860332)
\curveto(217.7714686,222.04458169)(218.22622075,221.40005109)(218.51985168,220.55500957)
}
}
{
\newrgbcolor{curcolor}{0 0 0}
\pscustom[linestyle=none,fillstyle=solid,fillcolor=curcolor]
{
\newpath
\moveto(232.50617981,212.63801738)
\lineto(232.50617981,206.25715801)
\lineto(230.51887512,206.25715801)
\lineto(230.51887512,222.86457988)
\lineto(232.50617981,222.86457988)
\lineto(232.50617981,221.03840801)
\curveto(232.92153999,221.75454292)(233.44432593,222.28449031)(234.07453918,222.62825176)
\curveto(234.71190279,222.97915107)(235.47101661,223.15460662)(236.35188293,223.15461895)
\curveto(237.81281114,223.15460662)(238.99803131,222.57452908)(239.907547,221.41438457)
\curveto(240.82420136,220.2542189)(241.28253424,218.7288298)(241.282547,216.8382127)
\curveto(241.28253424,214.94758358)(240.82420136,213.42219448)(239.907547,212.26204082)
\curveto(238.99803131,211.1018843)(237.81281114,210.52180676)(236.35188293,210.52180645)
\curveto(235.47101661,210.52180676)(234.71190279,210.69368158)(234.07453918,211.03743145)
\curveto(233.44432593,211.38834235)(232.92153999,211.92187046)(232.50617981,212.63801738)
\moveto(239.23078918,216.8382127)
\curveto(239.23077847,218.29198128)(238.92999753,219.43065201)(238.32844543,220.25422832)
\curveto(237.73403518,221.08494724)(236.91404902,221.5003114)(235.8684845,221.50032207)
\curveto(234.82290528,221.5003114)(233.99933839,221.08494724)(233.39778137,220.25422832)
\curveto(232.80337605,219.43065201)(232.50617582,218.29198128)(232.50617981,216.8382127)
\curveto(232.50617582,215.3844321)(232.80337605,214.24218064)(233.39778137,213.41145488)
\curveto(233.99933839,212.58788542)(234.82290528,212.17610198)(235.8684845,212.17610332)
\curveto(236.91404902,212.17610198)(237.73403518,212.58788542)(238.32844543,213.41145488)
\curveto(238.92999753,214.24218064)(239.23077847,215.3844321)(239.23078918,216.8382127)
}
}
{
\newrgbcolor{curcolor}{0 0 0}
\pscustom[linestyle=none,fillstyle=solid,fillcolor=curcolor]
{
\newpath
\moveto(246.51399231,226.28059551)
\lineto(246.51399231,222.86457988)
\lineto(250.58528137,222.86457988)
\lineto(250.58528137,221.32844707)
\lineto(246.51399231,221.32844707)
\lineto(246.51399231,214.79719707)
\curveto(246.51398828,213.8160743)(246.64647513,213.18586659)(246.91145325,212.90657207)
\curveto(247.18358397,212.6272734)(247.73143498,212.4876251)(248.55500793,212.48762676)
\lineto(250.58528137,212.48762676)
\lineto(250.58528137,210.83332988)
\lineto(248.55500793,210.83332988)
\curveto(247.02961277,210.83332988)(245.97687944,211.1162072)(245.39680481,211.6819627)
\curveto(244.81672435,212.25487794)(244.52668558,213.29328836)(244.52668762,214.79719707)
\lineto(244.52668762,221.32844707)
\lineto(243.07649231,221.32844707)
\lineto(243.07649231,222.86457988)
\lineto(244.52668762,222.86457988)
\lineto(244.52668762,226.28059551)
\lineto(246.51399231,226.28059551)
}
}
{
\newrgbcolor{curcolor}{0 0 0}
\pscustom[linestyle=none,fillstyle=solid,fillcolor=curcolor]
{
\newpath
\moveto(263.48664856,217.34309551)
\lineto(263.48664856,216.37629863)
\lineto(254.39875793,216.37629863)
\curveto(254.48469207,215.01561737)(254.89289479,213.97720695)(255.62336731,213.26106426)
\curveto(256.36099228,212.55207816)(257.3850798,212.19758633)(258.69563293,212.1975877)
\curveto(259.45473919,212.19758633)(260.18878793,212.2906852)(260.89778137,212.47688457)
\curveto(261.61391671,212.66308066)(262.32290038,212.94237725)(263.0247345,213.3147752)
\lineto(263.0247345,211.44563457)
\curveto(262.31573893,211.14485301)(261.58885163,210.91568657)(260.84407043,210.75813457)
\curveto(260.09926979,210.60058272)(259.34373669,210.52180676)(258.57746887,210.52180645)
\curveto(256.6581925,210.52180676)(255.13638413,211.08039995)(254.01203918,212.1975877)
\curveto(252.89484991,213.31477271)(252.33625672,214.82583891)(252.33625793,216.73079082)
\curveto(252.33625672,218.700184)(252.86620411,220.26138035)(253.92610168,221.41438457)
\curveto(254.99315511,222.57452908)(256.42902607,223.15460662)(258.23371887,223.15461895)
\curveto(259.85219973,223.15460662)(261.13051876,222.63182069)(262.06867981,221.58625957)
\curveto(263.01398042,220.5478384)(263.4866362,219.13345179)(263.48664856,217.34309551)
\moveto(261.51008606,217.92317363)
\curveto(261.49575277,219.00454567)(261.19139109,219.86750054)(260.59700012,220.51204082)
\curveto(260.00975165,221.15656175)(259.22915347,221.47882705)(258.25520325,221.4788377)
\curveto(257.15233263,221.47882705)(256.26789341,221.16730392)(255.60188293,220.54426738)
\curveto(254.94302495,219.92121142)(254.56346804,219.04393365)(254.46321106,217.91243145)
\lineto(261.51008606,217.92317363)
}
}
{
\newrgbcolor{curcolor}{0 0 0}
\pscustom[linestyle=none,fillstyle=solid,fillcolor=curcolor]
{
\newpath
\moveto(313.82583618,242.70475933)
\lineto(317.05923462,242.70475933)
\lineto(321.15200806,231.79069683)
\lineto(325.26626587,242.70475933)
\lineto(328.49966431,242.70475933)
\lineto(328.49966431,226.66667339)
\lineto(326.38345337,226.66667339)
\lineto(326.38345337,240.7496812)
\lineto(322.24771118,229.7496812)
\lineto(320.06704712,229.7496812)
\lineto(315.93130493,240.7496812)
\lineto(315.93130493,226.66667339)
\lineto(313.82583618,226.66667339)
\lineto(313.82583618,242.70475933)
}
}
{
\newrgbcolor{curcolor}{0 0 0}
\pscustom[linestyle=none,fillstyle=solid,fillcolor=curcolor]
{
\newpath
\moveto(337.39419556,237.3121812)
\curveto(336.33429405,237.31217056)(335.49640426,236.89680639)(334.88052368,236.06608745)
\curveto(334.26463466,235.24251117)(333.95669226,234.11100188)(333.95669556,232.6715562)
\curveto(333.95669226,231.23209851)(334.26105393,230.0970085)(334.86978149,229.26628276)
\curveto(335.48566208,228.44271328)(336.3271326,228.03092984)(337.39419556,228.0309312)
\curveto(338.44692214,228.03092984)(339.28123121,228.446294)(339.89712524,229.27702495)
\curveto(340.51300081,230.10775068)(340.82094321,231.23925996)(340.82095337,232.6715562)
\curveto(340.82094321,234.09667898)(340.51300081,235.22460754)(339.89712524,236.05534526)
\curveto(339.28123121,236.89322566)(338.44692214,237.31217056)(337.39419556,237.3121812)
\moveto(337.39419556,238.98796245)
\curveto(339.1129371,238.98795013)(340.46287065,238.42935694)(341.44400024,237.3121812)
\curveto(342.42510827,236.19498417)(342.91566767,234.64811072)(342.91567993,232.6715562)
\curveto(342.91566767,230.70215112)(342.42510827,229.15527767)(341.44400024,228.0309312)
\curveto(340.46287065,226.91374345)(339.1129371,226.35515026)(337.39419556,226.35514995)
\curveto(335.66827909,226.35515026)(334.31476482,226.91374345)(333.33364868,228.0309312)
\curveto(332.35968865,229.15527767)(331.87270997,230.70215112)(331.87271118,232.6715562)
\curveto(331.87270997,234.64811072)(332.35968865,236.19498417)(333.33364868,237.3121812)
\curveto(334.31476482,238.42935694)(335.66827909,238.98795013)(337.39419556,238.98796245)
}
}
{
\newrgbcolor{curcolor}{0 0 0}
\pscustom[linestyle=none,fillstyle=solid,fillcolor=curcolor]
{
\newpath
\moveto(354.09829712,236.87175151)
\lineto(354.09829712,243.38151714)
\lineto(356.07485962,243.38151714)
\lineto(356.07485962,226.66667339)
\lineto(354.09829712,226.66667339)
\lineto(354.09829712,228.47136089)
\curveto(353.68292296,227.75521397)(353.1565563,227.22168585)(352.51919556,226.87077495)
\curveto(351.88897944,226.52702509)(351.12986562,226.35515026)(350.24185181,226.35514995)
\curveto(348.78807108,226.35515026)(347.60285092,226.93522781)(346.68618774,228.09538433)
\curveto(345.77668087,229.25553799)(345.32192872,230.78092709)(345.32192993,232.6715562)
\curveto(345.32192872,234.56217331)(345.77668087,236.08756241)(346.68618774,237.24772808)
\curveto(347.60285092,238.40787258)(348.78807108,238.98795013)(350.24185181,238.98796245)
\curveto(351.12986562,238.98795013)(351.88897944,238.81249458)(352.51919556,238.46159526)
\curveto(353.1565563,238.11783381)(353.68292296,237.58788643)(354.09829712,236.87175151)
\moveto(347.36294556,232.6715562)
\curveto(347.3629423,231.21777561)(347.66014253,230.07552415)(348.25454712,229.24479839)
\curveto(348.85610487,228.42122893)(349.67967176,228.00944548)(350.72525024,228.00944683)
\curveto(351.7708155,228.00944548)(352.59438238,228.42122893)(353.19595337,229.24479839)
\curveto(353.79750618,230.07552415)(354.09828713,231.21777561)(354.09829712,232.6715562)
\curveto(354.09828713,234.12532478)(353.79750618,235.26399552)(353.19595337,236.08757183)
\curveto(352.59438238,236.91829074)(351.7708155,237.33365491)(350.72525024,237.33366558)
\curveto(349.67967176,237.33365491)(348.85610487,236.91829074)(348.25454712,236.08757183)
\curveto(347.66014253,235.26399552)(347.3629423,234.12532478)(347.36294556,232.6715562)
}
}
{
\newrgbcolor{curcolor}{0 0 0}
\pscustom[linestyle=none,fillstyle=solid,fillcolor=curcolor]
{
\newpath
\moveto(360.14614868,238.69792339)
\lineto(362.12271118,238.69792339)
\lineto(362.12271118,226.66667339)
\lineto(360.14614868,226.66667339)
\lineto(360.14614868,238.69792339)
\moveto(360.14614868,243.38151714)
\lineto(362.12271118,243.38151714)
\lineto(362.12271118,240.87858745)
\lineto(360.14614868,240.87858745)
\lineto(360.14614868,243.38151714)
}
}
{
\newrgbcolor{curcolor}{0 0 0}
\pscustom[linestyle=none,fillstyle=solid,fillcolor=curcolor]
{
\newpath
\moveto(375.96939087,238.69792339)
\lineto(375.96939087,226.66667339)
\lineto(373.98208618,226.66667339)
\lineto(373.98208618,237.16179058)
\lineto(368.55728149,237.16179058)
\lineto(368.55728149,226.66667339)
\lineto(366.56997681,226.66667339)
\lineto(366.56997681,237.16179058)
\lineto(364.67935181,237.16179058)
\lineto(364.67935181,238.69792339)
\lineto(366.56997681,238.69792339)
\lineto(366.56997681,239.53581401)
\curveto(366.56997441,240.84634671)(366.87791681,241.81314262)(367.49380493,242.43620464)
\curveto(368.11684786,243.06639657)(369.06932087,243.38150042)(370.35122681,243.38151714)
\lineto(372.33853149,243.38151714)
\lineto(372.33853149,241.73796245)
\lineto(370.44790649,241.73796245)
\curveto(369.73891655,241.73794738)(369.24477642,241.59471836)(368.96548462,241.30827495)
\curveto(368.69334468,241.02180226)(368.55727711,240.50617778)(368.55728149,239.76139995)
\lineto(368.55728149,238.69792339)
\lineto(375.96939087,238.69792339)
\moveto(373.98208618,243.36003276)
\lineto(375.96939087,243.36003276)
\lineto(375.96939087,240.85710308)
\lineto(373.98208618,240.85710308)
\lineto(373.98208618,243.36003276)
}
}
{
\newrgbcolor{curcolor}{0 0 0}
\pscustom[linestyle=none,fillstyle=solid,fillcolor=curcolor]
{
\newpath
\moveto(390.41763306,233.17643901)
\lineto(390.41763306,232.20964214)
\lineto(381.32974243,232.20964214)
\curveto(381.41567657,230.84896087)(381.82387929,229.81055045)(382.55435181,229.09440776)
\curveto(383.29197678,228.38542167)(384.31606429,228.03092984)(385.62661743,228.0309312)
\curveto(386.38572368,228.03092984)(387.11977243,228.1240287)(387.82876587,228.31022808)
\curveto(388.54490121,228.49642416)(389.25388488,228.77572076)(389.95571899,229.1481187)
\lineto(389.95571899,227.27897808)
\curveto(389.24672343,226.97819651)(388.51983613,226.74903008)(387.77505493,226.59147808)
\curveto(387.03025429,226.43392623)(386.27472119,226.35515026)(385.50845337,226.35514995)
\curveto(383.589177,226.35515026)(382.06736863,226.91374345)(380.94302368,228.0309312)
\curveto(379.82583441,229.14811622)(379.26724122,230.65918242)(379.26724243,232.56413433)
\curveto(379.26724122,234.5335275)(379.7971886,236.09472386)(380.85708618,237.24772808)
\curveto(381.9241396,238.40787258)(383.36001056,238.98795013)(385.16470337,238.98796245)
\curveto(386.78318422,238.98795013)(388.06150326,238.46516419)(388.99966431,237.41960308)
\curveto(389.94496492,236.3811819)(390.41762069,234.9667953)(390.41763306,233.17643901)
\moveto(388.44107056,233.75651714)
\curveto(388.42673727,234.83788918)(388.12237559,235.70084404)(387.52798462,236.34538433)
\curveto(386.94073615,236.98990525)(386.16013797,237.31217056)(385.18618774,237.3121812)
\curveto(384.08331713,237.31217056)(383.19887791,237.00064743)(382.53286743,236.37761089)
\curveto(381.87400944,235.75455493)(381.49445253,234.87727716)(381.39419556,233.74577495)
\lineto(388.44107056,233.75651714)
}
}
{
\newrgbcolor{curcolor}{0 0 0}
\pscustom[linestyle=none,fillstyle=solid,fillcolor=curcolor]
{
\newpath
\moveto(400.63345337,236.85026714)
\curveto(400.41143934,236.97916308)(400.16795,237.07226194)(399.90298462,237.12956401)
\curveto(399.64516406,237.19400661)(399.35870602,237.22623314)(399.04360962,237.2262437)
\curveto(397.92641578,237.22623314)(397.06704164,236.86099913)(396.46548462,236.13054058)
\curveto(395.87107929,235.40722454)(395.57387907,234.3652334)(395.57388306,233.00456401)
\lineto(395.57388306,226.66667339)
\lineto(393.58657837,226.66667339)
\lineto(393.58657837,238.69792339)
\lineto(395.57388306,238.69792339)
\lineto(395.57388306,236.82878276)
\curveto(395.98924324,237.55924062)(396.5299328,238.09993018)(397.19595337,238.45085308)
\curveto(397.86196272,238.80891385)(398.6712067,238.98795013)(399.62368774,238.98796245)
\curveto(399.75974728,238.98795013)(399.91013776,238.97720795)(400.07485962,238.95573589)
\curveto(400.23956451,238.9414007)(400.42218151,238.91633562)(400.62271118,238.88054058)
\lineto(400.63345337,236.85026714)
}
}
{
\newrgbcolor{curcolor}{0 0 0}
\pscustom[linestyle=none,fillstyle=solid,fillcolor=curcolor]
{
\newpath
\moveto(144.16894531,184.80272075)
\lineto(144.16894531,182.51463481)
\curveto(143.43846312,183.19496016)(142.65786495,183.70342319)(141.82714844,184.04002544)
\curveto(141.00356973,184.3765996)(140.12629196,184.54489371)(139.1953125,184.54490825)
\curveto(137.3619718,184.54489371)(135.95832738,183.98271979)(134.984375,182.85838481)
\curveto(134.01041266,181.74118557)(133.52343398,180.12269761)(133.5234375,178.00291606)
\curveto(133.52343398,175.89027997)(134.01041266,174.271792)(134.984375,173.14744731)
\curveto(135.95832738,172.03025778)(137.3619718,171.47166459)(139.1953125,171.47166606)
\curveto(140.12629196,171.47166459)(141.00356973,171.6399587)(141.82714844,171.97654888)
\curveto(142.65786495,172.31313511)(143.43846312,172.82159814)(144.16894531,173.5019395)
\lineto(144.16894531,171.23533794)
\curveto(143.40981732,170.71971222)(142.60415406,170.33299386)(141.75195312,170.07518169)
\curveto(140.90689013,169.81736937)(140.01170874,169.68846325)(139.06640625,169.68846294)
\curveto(136.63866524,169.68846325)(134.72655777,170.42967345)(133.33007812,171.91209575)
\curveto(131.93359182,173.40167568)(131.23535033,175.43194709)(131.23535156,178.00291606)
\curveto(131.23535033,180.58103048)(131.93359182,182.61130189)(133.33007812,184.09373638)
\curveto(134.72655777,185.58330413)(136.63866524,186.32809505)(139.06640625,186.32811138)
\curveto(140.02603164,186.32809505)(140.92837449,186.19918893)(141.7734375,185.94139263)
\curveto(142.62563842,185.69072589)(143.42414022,185.31116898)(144.16894531,184.80272075)
}
}
{
\newrgbcolor{curcolor}{0 0 0}
\pscustom[linestyle=none,fillstyle=solid,fillcolor=curcolor]
{
\newpath
\moveto(152.11816406,180.64549419)
\curveto(151.05826255,180.64548354)(150.22037277,180.23011938)(149.60449219,179.39940044)
\curveto(148.98860317,178.57582416)(148.68066076,177.44431487)(148.68066406,176.00486919)
\curveto(148.68066076,174.5654115)(148.98502244,173.43032149)(149.59375,172.59959575)
\curveto(150.20963059,171.77602627)(151.0511011,171.36424283)(152.11816406,171.36424419)
\curveto(153.17089065,171.36424283)(154.00519971,171.77960699)(154.62109375,172.61033794)
\curveto(155.23696931,173.44106367)(155.54491171,174.57257295)(155.54492188,176.00486919)
\curveto(155.54491171,177.42999197)(155.23696931,178.55792053)(154.62109375,179.38865825)
\curveto(154.00519971,180.22653865)(153.17089065,180.64548354)(152.11816406,180.64549419)
\moveto(152.11816406,182.32127544)
\curveto(153.83690561,182.32126312)(155.18683915,181.76266993)(156.16796875,180.64549419)
\curveto(157.14907678,179.52829716)(157.63963618,177.98142371)(157.63964844,176.00486919)
\curveto(157.63963618,174.03546411)(157.14907678,172.48859066)(156.16796875,171.36424419)
\curveto(155.18683915,170.24705644)(153.83690561,169.68846325)(152.11816406,169.68846294)
\curveto(150.39224759,169.68846325)(149.03873332,170.24705644)(148.05761719,171.36424419)
\curveto(147.08365715,172.48859066)(146.59667847,174.03546411)(146.59667969,176.00486919)
\curveto(146.59667847,177.98142371)(147.08365715,179.52829716)(148.05761719,180.64549419)
\curveto(149.03873332,181.76266993)(150.39224759,182.32126312)(152.11816406,182.32127544)
}
}
{
\newrgbcolor{curcolor}{0 0 0}
\pscustom[linestyle=none,fillstyle=solid,fillcolor=curcolor]
{
\newpath
\moveto(170.90625,177.26170513)
\lineto(170.90625,169.99998638)
\lineto(168.9296875,169.99998638)
\lineto(168.9296875,177.197252)
\curveto(168.9296774,178.33591554)(168.70767242,179.18812823)(168.26367188,179.75389263)
\curveto(167.81965247,180.31963752)(167.15363751,180.60251484)(166.265625,180.60252544)
\curveto(165.19856134,180.60251484)(164.35709083,180.26234591)(163.74121094,179.58201763)
\curveto(163.12532123,178.90167018)(162.81737883,177.97426226)(162.81738281,176.79979106)
\lineto(162.81738281,169.99998638)
\lineto(160.83007812,169.99998638)
\lineto(160.83007812,182.03123638)
\lineto(162.81738281,182.03123638)
\lineto(162.81738281,180.16209575)
\curveto(163.2900346,180.88539216)(163.84504707,181.42608172)(164.48242188,181.78416606)
\curveto(165.12694683,182.14222684)(165.86815703,182.32126312)(166.70605469,182.32127544)
\curveto(168.08820689,182.32126312)(169.13377876,181.89157605)(169.84277344,181.03221294)
\curveto(170.55174609,180.17998922)(170.90623793,178.92315454)(170.90625,177.26170513)
}
}
{
\newrgbcolor{curcolor}{0 0 0}
\pscustom[linestyle=none,fillstyle=solid,fillcolor=curcolor]
{
\newpath
\moveto(184.87109375,177.26170513)
\lineto(184.87109375,169.99998638)
\lineto(182.89453125,169.99998638)
\lineto(182.89453125,177.197252)
\curveto(182.89452115,178.33591554)(182.67251617,179.18812823)(182.22851562,179.75389263)
\curveto(181.78449622,180.31963752)(181.11848126,180.60251484)(180.23046875,180.60252544)
\curveto(179.16340509,180.60251484)(178.32193458,180.26234591)(177.70605469,179.58201763)
\curveto(177.09016498,178.90167018)(176.78222258,177.97426226)(176.78222656,176.79979106)
\lineto(176.78222656,169.99998638)
\lineto(174.79492188,169.99998638)
\lineto(174.79492188,182.03123638)
\lineto(176.78222656,182.03123638)
\lineto(176.78222656,180.16209575)
\curveto(177.25487835,180.88539216)(177.80989082,181.42608172)(178.44726562,181.78416606)
\curveto(179.09179058,182.14222684)(179.83300078,182.32126312)(180.67089844,182.32127544)
\curveto(182.05305064,182.32126312)(183.09862251,181.89157605)(183.80761719,181.03221294)
\curveto(184.51658984,180.17998922)(184.87108168,178.92315454)(184.87109375,177.26170513)
}
}
{
\newrgbcolor{curcolor}{0 0 0}
\pscustom[linestyle=none,fillstyle=solid,fillcolor=curcolor]
{
\newpath
\moveto(199.12597656,176.509752)
\lineto(199.12597656,175.54295513)
\lineto(190.03808594,175.54295513)
\curveto(190.12402008,174.18227386)(190.53222279,173.14386344)(191.26269531,172.42772075)
\curveto(192.00032028,171.71873466)(193.0244078,171.36424283)(194.33496094,171.36424419)
\curveto(195.09406719,171.36424283)(195.82811593,171.45734169)(196.53710938,171.64354106)
\curveto(197.25324472,171.82973715)(197.96222838,172.10903375)(198.6640625,172.48143169)
\lineto(198.6640625,170.61229106)
\curveto(197.95506693,170.3115095)(197.22817964,170.08234307)(196.48339844,169.92479106)
\curveto(195.73859779,169.76723921)(194.9830647,169.68846325)(194.21679688,169.68846294)
\curveto(192.29752051,169.68846325)(190.77571213,170.24705644)(189.65136719,171.36424419)
\curveto(188.53417792,172.48142921)(187.97558472,173.99249541)(187.97558594,175.89744731)
\curveto(187.97558472,177.86684049)(188.50553211,179.42803684)(189.56542969,180.58104106)
\curveto(190.63248311,181.74118557)(192.06835407,182.32126312)(193.87304688,182.32127544)
\curveto(195.49152773,182.32126312)(196.76984676,181.79847718)(197.70800781,180.75291606)
\curveto(198.65330842,179.71449489)(199.1259642,178.30010829)(199.12597656,176.509752)
\moveto(197.14941406,177.08983013)
\curveto(197.13508077,178.17120216)(196.8307191,179.03415703)(196.23632812,179.67869731)
\curveto(195.64907965,180.32321824)(194.86848148,180.64548354)(193.89453125,180.64549419)
\curveto(192.79166064,180.64548354)(191.90722142,180.33396042)(191.24121094,179.71092388)
\curveto(190.58235295,179.08786791)(190.20279604,178.21059015)(190.10253906,177.07908794)
\lineto(197.14941406,177.08983013)
}
}
{
\newrgbcolor{curcolor}{0 0 0}
\pscustom[linestyle=none,fillstyle=solid,fillcolor=curcolor]
{
\newpath
\moveto(211.984375,182.03123638)
\lineto(207.63378906,176.17674419)
\lineto(212.20996094,169.99998638)
\lineto(209.87890625,169.99998638)
\lineto(206.37695312,174.72654888)
\lineto(202.875,169.99998638)
\lineto(200.54394531,169.99998638)
\lineto(205.21679688,176.29490825)
\lineto(200.94140625,182.03123638)
\lineto(203.27246094,182.03123638)
\lineto(206.46289062,177.74510356)
\lineto(209.65332031,182.03123638)
\lineto(211.984375,182.03123638)
}
}
{
\newrgbcolor{curcolor}{0 0 0}
\pscustom[linestyle=none,fillstyle=solid,fillcolor=curcolor]
{
\newpath
\moveto(215.00292969,182.03123638)
\lineto(216.97949219,182.03123638)
\lineto(216.97949219,169.99998638)
\lineto(215.00292969,169.99998638)
\lineto(215.00292969,182.03123638)
\moveto(215.00292969,186.71483013)
\lineto(216.97949219,186.71483013)
\lineto(216.97949219,184.21190044)
\lineto(215.00292969,184.21190044)
\lineto(215.00292969,186.71483013)
}
}
{
\newrgbcolor{curcolor}{0 0 0}
\pscustom[linestyle=none,fillstyle=solid,fillcolor=curcolor]
{
\newpath
\moveto(225.76660156,180.64549419)
\curveto(224.70670005,180.64548354)(223.86881027,180.23011938)(223.25292969,179.39940044)
\curveto(222.63704067,178.57582416)(222.32909826,177.44431487)(222.32910156,176.00486919)
\curveto(222.32909826,174.5654115)(222.63345994,173.43032149)(223.2421875,172.59959575)
\curveto(223.85806809,171.77602627)(224.6995386,171.36424283)(225.76660156,171.36424419)
\curveto(226.81932815,171.36424283)(227.65363721,171.77960699)(228.26953125,172.61033794)
\curveto(228.88540681,173.44106367)(229.19334921,174.57257295)(229.19335938,176.00486919)
\curveto(229.19334921,177.42999197)(228.88540681,178.55792053)(228.26953125,179.38865825)
\curveto(227.65363721,180.22653865)(226.81932815,180.64548354)(225.76660156,180.64549419)
\moveto(225.76660156,182.32127544)
\curveto(227.48534311,182.32126312)(228.83527665,181.76266993)(229.81640625,180.64549419)
\curveto(230.79751428,179.52829716)(231.28807368,177.98142371)(231.28808594,176.00486919)
\curveto(231.28807368,174.03546411)(230.79751428,172.48859066)(229.81640625,171.36424419)
\curveto(228.83527665,170.24705644)(227.48534311,169.68846325)(225.76660156,169.68846294)
\curveto(224.04068509,169.68846325)(222.68717082,170.24705644)(221.70605469,171.36424419)
\curveto(220.73209465,172.48859066)(220.24511597,174.03546411)(220.24511719,176.00486919)
\curveto(220.24511597,177.98142371)(220.73209465,179.52829716)(221.70605469,180.64549419)
\curveto(222.68717082,181.76266993)(224.04068509,182.32126312)(225.76660156,182.32127544)
}
}
{
\newrgbcolor{curcolor}{0 0 0}
\pscustom[linestyle=none,fillstyle=solid,fillcolor=curcolor]
{
\newpath
\moveto(244.5546875,177.26170513)
\lineto(244.5546875,169.99998638)
\lineto(242.578125,169.99998638)
\lineto(242.578125,177.197252)
\curveto(242.5781149,178.33591554)(242.35610992,179.18812823)(241.91210938,179.75389263)
\curveto(241.46808997,180.31963752)(240.80207501,180.60251484)(239.9140625,180.60252544)
\curveto(238.84699884,180.60251484)(238.00552833,180.26234591)(237.38964844,179.58201763)
\curveto(236.77375873,178.90167018)(236.46581633,177.97426226)(236.46582031,176.79979106)
\lineto(236.46582031,169.99998638)
\lineto(234.47851562,169.99998638)
\lineto(234.47851562,182.03123638)
\lineto(236.46582031,182.03123638)
\lineto(236.46582031,180.16209575)
\curveto(236.9384721,180.88539216)(237.49348457,181.42608172)(238.13085938,181.78416606)
\curveto(238.77538433,182.14222684)(239.51659453,182.32126312)(240.35449219,182.32127544)
\curveto(241.73664439,182.32126312)(242.78221626,181.89157605)(243.49121094,181.03221294)
\curveto(244.20018359,180.17998922)(244.55467543,178.92315454)(244.5546875,177.26170513)
}
}
{
\newrgbcolor{curcolor}{0 0 0}
\pscustom[linestyle=none,fillstyle=solid,fillcolor=curcolor]
{
}
}
{
\newrgbcolor{curcolor}{0 0 0}
\pscustom[linestyle=none,fillstyle=solid,fillcolor=curcolor]
{
\newpath
\moveto(260.99023438,176.04783794)
\curveto(259.39322322,176.04783189)(258.28677902,175.86521489)(257.67089844,175.49998638)
\curveto(257.05500942,175.13474687)(256.74706701,174.51170062)(256.74707031,173.63084575)
\curveto(256.74706701,172.92901991)(256.97623345,172.37042671)(257.43457031,171.9550645)
\curveto(257.90006065,171.54685983)(258.53026836,171.34275847)(259.32519531,171.34275981)
\curveto(260.42089147,171.34275847)(261.29816923,171.72947683)(261.95703125,172.50291606)
\curveto(262.6230377,173.28351174)(262.95604518,174.31834143)(262.95605469,175.60740825)
\lineto(262.95605469,176.04783794)
\lineto(260.99023438,176.04783794)
\moveto(264.93261719,176.86424419)
\lineto(264.93261719,169.99998638)
\lineto(262.95605469,169.99998638)
\lineto(262.95605469,171.82615825)
\curveto(262.50487376,171.09568841)(261.94269984,170.55499884)(261.26953125,170.20408794)
\curveto(260.59634702,169.86033808)(259.77278013,169.68846325)(258.79882812,169.68846294)
\curveto(257.56705317,169.68846325)(256.58593436,170.03221291)(255.85546875,170.71971294)
\curveto(255.13215978,171.41437298)(254.77050649,172.34178091)(254.77050781,173.5019395)
\curveto(254.77050649,174.85545027)(255.22167792,175.87595706)(256.12402344,176.56346294)
\curveto(257.03352506,177.25095569)(258.38703933,177.59470534)(260.18457031,177.59471294)
\lineto(262.95605469,177.59471294)
\lineto(262.95605469,177.78807231)
\curveto(262.95604518,178.69756883)(262.65526423,179.39939104)(262.05371094,179.89354106)
\curveto(261.45930189,180.39483275)(260.6214121,180.64548354)(259.54003906,180.64549419)
\curveto(258.85253366,180.64548354)(258.18293797,180.56312686)(257.53125,180.39842388)
\curveto(256.87955386,180.2337001)(256.25292688,179.98663004)(255.65136719,179.65721294)
\lineto(255.65136719,181.48338481)
\curveto(256.37467155,181.76266993)(257.07649377,181.97035201)(257.75683594,182.10643169)
\curveto(258.43716949,182.24964861)(259.09960372,182.32126312)(259.74414062,182.32127544)
\curveto(261.48436696,182.32126312)(262.78417035,181.87009169)(263.64355469,180.96775981)
\curveto(264.50291863,180.065406)(264.9326057,178.69756883)(264.93261719,176.86424419)
}
}
{
\newrgbcolor{curcolor}{0 0 0}
\pscustom[linestyle=none,fillstyle=solid,fillcolor=curcolor]
{
\newpath
\moveto(268.81054688,174.74803325)
\lineto(268.81054688,182.03123638)
\lineto(270.78710938,182.03123638)
\lineto(270.78710938,174.82322856)
\curveto(270.78710553,173.684553)(271.00911052,172.82875959)(271.453125,172.25584575)
\curveto(271.89713046,171.69008885)(272.56314542,171.40721153)(273.45117188,171.40721294)
\curveto(274.51822159,171.40721153)(275.3596921,171.74738046)(275.97558594,172.42772075)
\curveto(276.59862316,173.10805619)(276.91014628,174.03546411)(276.91015625,175.20994731)
\lineto(276.91015625,182.03123638)
\lineto(278.88671875,182.03123638)
\lineto(278.88671875,169.99998638)
\lineto(276.91015625,169.99998638)
\lineto(276.91015625,171.84764263)
\curveto(276.43032905,171.11717276)(275.87173586,170.57290247)(275.234375,170.21483013)
\curveto(274.604159,169.8639188)(273.87011026,169.68846325)(273.03222656,169.68846294)
\curveto(271.6500604,169.68846325)(270.6009078,170.11815032)(269.88476562,170.97752544)
\curveto(269.16861756,171.8368986)(268.81054501,173.09373328)(268.81054688,174.74803325)
\moveto(273.78417969,182.32127544)
\lineto(273.78417969,182.32127544)
}
}
{
\newrgbcolor{curcolor}{0 0 0}
\pscustom[linestyle=none,fillstyle=solid,fillcolor=curcolor]
{
\newpath
\moveto(284.93457031,185.447252)
\lineto(284.93457031,182.03123638)
\lineto(289.00585938,182.03123638)
\lineto(289.00585938,180.49510356)
\lineto(284.93457031,180.49510356)
\lineto(284.93457031,173.96385356)
\curveto(284.93456628,172.98273079)(285.06705313,172.35252309)(285.33203125,172.07322856)
\curveto(285.60416197,171.7939299)(286.15201298,171.6542816)(286.97558594,171.65428325)
\lineto(289.00585938,171.65428325)
\lineto(289.00585938,169.99998638)
\lineto(286.97558594,169.99998638)
\curveto(285.45019077,169.99998638)(284.39745745,170.2828637)(283.81738281,170.84861919)
\curveto(283.23730236,171.42153443)(282.94726358,172.45994485)(282.94726562,173.96385356)
\lineto(282.94726562,180.49510356)
\lineto(281.49707031,180.49510356)
\lineto(281.49707031,182.03123638)
\lineto(282.94726562,182.03123638)
\lineto(282.94726562,185.447252)
\lineto(284.93457031,185.447252)
}
}
{
\newrgbcolor{curcolor}{0 0 0}
\pscustom[linestyle=none,fillstyle=solid,fillcolor=curcolor]
{
\newpath
\moveto(296.27832031,180.64549419)
\curveto(295.2184188,180.64548354)(294.38052902,180.23011938)(293.76464844,179.39940044)
\curveto(293.14875942,178.57582416)(292.84081701,177.44431487)(292.84082031,176.00486919)
\curveto(292.84081701,174.5654115)(293.14517869,173.43032149)(293.75390625,172.59959575)
\curveto(294.36978684,171.77602627)(295.21125735,171.36424283)(296.27832031,171.36424419)
\curveto(297.3310469,171.36424283)(298.16535596,171.77960699)(298.78125,172.61033794)
\curveto(299.39712556,173.44106367)(299.70506796,174.57257295)(299.70507812,176.00486919)
\curveto(299.70506796,177.42999197)(299.39712556,178.55792053)(298.78125,179.38865825)
\curveto(298.16535596,180.22653865)(297.3310469,180.64548354)(296.27832031,180.64549419)
\moveto(296.27832031,182.32127544)
\curveto(297.99706186,182.32126312)(299.3469954,181.76266993)(300.328125,180.64549419)
\curveto(301.30923303,179.52829716)(301.79979243,177.98142371)(301.79980469,176.00486919)
\curveto(301.79979243,174.03546411)(301.30923303,172.48859066)(300.328125,171.36424419)
\curveto(299.3469954,170.24705644)(297.99706186,169.68846325)(296.27832031,169.68846294)
\curveto(294.55240384,169.68846325)(293.19888957,170.24705644)(292.21777344,171.36424419)
\curveto(291.2438134,172.48859066)(290.75683472,174.03546411)(290.75683594,176.00486919)
\curveto(290.75683472,177.98142371)(291.2438134,179.52829716)(292.21777344,180.64549419)
\curveto(293.19888957,181.76266993)(294.55240384,182.32126312)(296.27832031,182.32127544)
}
}
{
\newrgbcolor{curcolor}{0 0 0}
\pscustom[linestyle=none,fillstyle=solid,fillcolor=curcolor]
{
\newpath
\moveto(132.15917969,156.03807231)
\lineto(135.39257812,156.03807231)
\lineto(139.48535156,145.12400981)
\lineto(143.59960938,156.03807231)
\lineto(146.83300781,156.03807231)
\lineto(146.83300781,139.99998638)
\lineto(144.71679688,139.99998638)
\lineto(144.71679688,154.08299419)
\lineto(140.58105469,143.08299419)
\lineto(138.40039062,143.08299419)
\lineto(134.26464844,154.08299419)
\lineto(134.26464844,139.99998638)
\lineto(132.15917969,139.99998638)
\lineto(132.15917969,156.03807231)
}
}
{
\newrgbcolor{curcolor}{0 0 0}
\pscustom[linestyle=none,fillstyle=solid,fillcolor=curcolor]
{
\newpath
\moveto(161.35644531,146.509752)
\lineto(161.35644531,145.54295513)
\lineto(152.26855469,145.54295513)
\curveto(152.35448883,144.18227386)(152.76269154,143.14386344)(153.49316406,142.42772075)
\curveto(154.23078903,141.71873466)(155.25487655,141.36424283)(156.56542969,141.36424419)
\curveto(157.32453594,141.36424283)(158.05858468,141.45734169)(158.76757812,141.64354106)
\curveto(159.48371347,141.82973715)(160.19269713,142.10903375)(160.89453125,142.48143169)
\lineto(160.89453125,140.61229106)
\curveto(160.18553568,140.3115095)(159.45864839,140.08234307)(158.71386719,139.92479106)
\curveto(157.96906654,139.76723921)(157.21353345,139.68846325)(156.44726562,139.68846294)
\curveto(154.52798926,139.68846325)(153.00618088,140.24705644)(151.88183594,141.36424419)
\curveto(150.76464667,142.48142921)(150.20605347,143.99249541)(150.20605469,145.89744731)
\curveto(150.20605347,147.86684049)(150.73600086,149.42803684)(151.79589844,150.58104106)
\curveto(152.86295186,151.74118557)(154.29882282,152.32126312)(156.10351562,152.32127544)
\curveto(157.72199648,152.32126312)(159.00031551,151.79847718)(159.93847656,150.75291606)
\curveto(160.88377717,149.71449489)(161.35643295,148.30010829)(161.35644531,146.509752)
\moveto(159.37988281,147.08983013)
\curveto(159.36554952,148.17120216)(159.06118785,149.03415703)(158.46679688,149.67869731)
\curveto(157.8795484,150.32321824)(157.09895023,150.64548354)(156.125,150.64549419)
\curveto(155.02212939,150.64548354)(154.13769017,150.33396042)(153.47167969,149.71092388)
\curveto(152.8128217,149.08786791)(152.43326479,148.21059015)(152.33300781,147.07908794)
\lineto(159.37988281,147.08983013)
\moveto(157.47851562,157.5956895)
\lineto(159.61621094,157.5956895)
\lineto(156.11425781,153.556627)
\lineto(154.47070312,153.556627)
\lineto(157.47851562,157.5956895)
}
}
{
\newrgbcolor{curcolor}{0 0 0}
\pscustom[linestyle=none,fillstyle=solid,fillcolor=curcolor]
{
\newpath
\moveto(173.96777344,149.72166606)
\curveto(174.46190213,150.60967629)(175.05272185,151.26494907)(175.74023438,151.68748638)
\curveto(176.42772047,152.11000031)(177.23696446,152.32126312)(178.16796875,152.32127544)
\curveto(179.42120706,152.32126312)(180.38800297,151.88083387)(181.06835938,150.99998638)
\curveto(181.7486787,150.12627833)(182.08884763,148.88018583)(182.08886719,147.26170513)
\lineto(182.08886719,139.99998638)
\lineto(180.1015625,139.99998638)
\lineto(180.1015625,147.197252)
\curveto(180.10154493,148.35023844)(179.89744357,149.20603186)(179.48925781,149.76463481)
\curveto(179.08103813,150.32321824)(178.45799188,150.60251484)(177.62011719,150.60252544)
\curveto(176.59601458,150.60251484)(175.78677059,150.26234591)(175.19238281,149.58201763)
\curveto(174.5979697,148.90167018)(174.30076948,147.97426226)(174.30078125,146.79979106)
\lineto(174.30078125,139.99998638)
\lineto(172.31347656,139.99998638)
\lineto(172.31347656,147.197252)
\curveto(172.31346678,148.35739989)(172.10936542,149.21319331)(171.70117188,149.76463481)
\curveto(171.29295998,150.32321824)(170.66275228,150.60251484)(169.81054688,150.60252544)
\curveto(168.80077498,150.60251484)(167.99869245,150.25876518)(167.40429688,149.57127544)
\curveto(166.80989155,148.89092801)(166.51269133,147.96710081)(166.51269531,146.79979106)
\lineto(166.51269531,139.99998638)
\lineto(164.52539062,139.99998638)
\lineto(164.52539062,152.03123638)
\lineto(166.51269531,152.03123638)
\lineto(166.51269531,150.16209575)
\curveto(166.96386275,150.89971506)(167.50455231,151.44398535)(168.13476562,151.79490825)
\curveto(168.76496772,152.14580756)(169.51333937,152.32126312)(170.37988281,152.32127544)
\curveto(171.253572,152.32126312)(171.9947822,152.09925813)(172.60351562,151.65525981)
\curveto(173.21939035,151.21123819)(173.6741425,150.56670758)(173.96777344,149.72166606)
}
}
{
\newrgbcolor{curcolor}{0 0 0}
\pscustom[linestyle=none,fillstyle=solid,fillcolor=curcolor]
{
\newpath
\moveto(190.70410156,150.64549419)
\curveto(189.64420005,150.64548354)(188.80631027,150.23011938)(188.19042969,149.39940044)
\curveto(187.57454067,148.57582416)(187.26659826,147.44431487)(187.26660156,146.00486919)
\curveto(187.26659826,144.5654115)(187.57095994,143.43032149)(188.1796875,142.59959575)
\curveto(188.79556809,141.77602627)(189.6370386,141.36424283)(190.70410156,141.36424419)
\curveto(191.75682815,141.36424283)(192.59113721,141.77960699)(193.20703125,142.61033794)
\curveto(193.82290681,143.44106367)(194.13084921,144.57257295)(194.13085938,146.00486919)
\curveto(194.13084921,147.42999197)(193.82290681,148.55792053)(193.20703125,149.38865825)
\curveto(192.59113721,150.22653865)(191.75682815,150.64548354)(190.70410156,150.64549419)
\moveto(190.70410156,152.32127544)
\curveto(192.42284311,152.32126312)(193.77277665,151.76266993)(194.75390625,150.64549419)
\curveto(195.73501428,149.52829716)(196.22557368,147.98142371)(196.22558594,146.00486919)
\curveto(196.22557368,144.03546411)(195.73501428,142.48859066)(194.75390625,141.36424419)
\curveto(193.77277665,140.24705644)(192.42284311,139.68846325)(190.70410156,139.68846294)
\curveto(188.97818509,139.68846325)(187.62467082,140.24705644)(186.64355469,141.36424419)
\curveto(185.66959465,142.48859066)(185.18261597,144.03546411)(185.18261719,146.00486919)
\curveto(185.18261597,147.98142371)(185.66959465,149.52829716)(186.64355469,150.64549419)
\curveto(187.62467082,151.76266993)(188.97818509,152.32126312)(190.70410156,152.32127544)
}
}
{
\newrgbcolor{curcolor}{0 0 0}
\pscustom[linestyle=none,fillstyle=solid,fillcolor=curcolor]
{
\newpath
\moveto(206.46289062,150.18358013)
\curveto(206.24087659,150.31247606)(205.99738725,150.40557493)(205.73242188,150.462877)
\curveto(205.47460132,150.5273196)(205.18814327,150.55954613)(204.87304688,150.55955669)
\curveto(203.75585304,150.55954613)(202.8964789,150.19431212)(202.29492188,149.46385356)
\curveto(201.70051655,148.74053753)(201.40331633,147.69854639)(201.40332031,146.337877)
\lineto(201.40332031,139.99998638)
\lineto(199.41601562,139.99998638)
\lineto(199.41601562,152.03123638)
\lineto(201.40332031,152.03123638)
\lineto(201.40332031,150.16209575)
\curveto(201.8186805,150.89255361)(202.35937006,151.43324317)(203.02539062,151.78416606)
\curveto(203.69139998,152.14222684)(204.50064396,152.32126312)(205.453125,152.32127544)
\curveto(205.58918454,152.32126312)(205.73957501,152.31052094)(205.90429688,152.28904888)
\curveto(206.06900177,152.27471369)(206.25161877,152.24964861)(206.45214844,152.21385356)
\lineto(206.46289062,150.18358013)
}
}
{
\newrgbcolor{curcolor}{0 0 0}
\pscustom[linestyle=none,fillstyle=solid,fillcolor=curcolor]
{
\newpath
\moveto(208.55761719,152.03123638)
\lineto(210.53417969,152.03123638)
\lineto(210.53417969,139.99998638)
\lineto(208.55761719,139.99998638)
\lineto(208.55761719,152.03123638)
\moveto(208.55761719,156.71483013)
\lineto(210.53417969,156.71483013)
\lineto(210.53417969,154.21190044)
\lineto(208.55761719,154.21190044)
\lineto(208.55761719,156.71483013)
}
}
{
\newrgbcolor{curcolor}{0 0 0}
\pscustom[linestyle=none,fillstyle=solid,fillcolor=curcolor]
{
\newpath
\moveto(222.32910156,151.67674419)
\lineto(222.32910156,149.80760356)
\curveto(221.77049863,150.0940518)(221.19042108,150.30889534)(220.58886719,150.45213481)
\curveto(219.98729729,150.59535339)(219.36425103,150.6669679)(218.71972656,150.66697856)
\curveto(217.73860162,150.6669679)(217.00097215,150.51657742)(216.50683594,150.21580669)
\curveto(216.01985334,149.91501552)(215.776364,149.4638441)(215.77636719,148.86229106)
\curveto(215.776364,148.40394933)(215.95181955,148.04229604)(216.30273438,147.77733013)
\curveto(216.65364177,147.51951011)(217.35904471,147.27244004)(218.41894531,147.03611919)
\lineto(219.09570312,146.88572856)
\curveto(220.49934104,146.58494073)(221.49478276,146.15883438)(222.08203125,145.60740825)
\curveto(222.6764222,145.06313236)(222.97362242,144.30043781)(222.97363281,143.31932231)
\curveto(222.97362242,142.20213261)(222.52961245,141.31769339)(221.64160156,140.666002)
\curveto(220.76073401,140.01430928)(219.54686804,139.68846325)(218,139.68846294)
\curveto(217.35546398,139.68846325)(216.68228757,139.75291631)(215.98046875,139.88182231)
\curveto(215.28580459,140.0035671)(214.55175585,140.18976483)(213.77832031,140.44041606)
\lineto(213.77832031,142.48143169)
\curveto(214.50878714,142.1018723)(215.22851298,141.81541425)(215.9375,141.62205669)
\curveto(216.64648031,141.43585734)(217.34830253,141.34275847)(218.04296875,141.34275981)
\curveto(218.97395195,141.34275847)(219.69009706,141.5003104)(220.19140625,141.81541606)
\curveto(220.69270023,142.13767955)(220.94335102,142.58885098)(220.94335938,143.16893169)
\curveto(220.94335102,143.70603736)(220.76073401,144.1178208)(220.39550781,144.40428325)
\curveto(220.03742744,144.69073689)(219.24608709,144.96645276)(218.02148438,145.23143169)
\lineto(217.33398438,145.3925645)
\curveto(216.10937148,145.65037135)(215.22493226,146.04425117)(214.68066406,146.57420513)
\curveto(214.13639168,147.11130739)(213.86425653,147.84535614)(213.86425781,148.77635356)
\curveto(213.86425653,149.90785407)(214.2652978,150.78155112)(215.06738281,151.39744731)
\curveto(215.86946286,152.01332072)(217.0081336,152.32126312)(218.48339844,152.32127544)
\curveto(219.21386056,152.32126312)(219.90135987,152.26755223)(220.54589844,152.16014263)
\curveto(221.19042108,152.0527087)(221.78482153,151.89157605)(222.32910156,151.67674419)
}
}
{
\newrgbcolor{curcolor}{0 0 0}
\pscustom[linestyle=none,fillstyle=solid,fillcolor=curcolor]
{
\newpath
\moveto(236.42285156,146.509752)
\lineto(236.42285156,145.54295513)
\lineto(227.33496094,145.54295513)
\curveto(227.42089508,144.18227386)(227.82909779,143.14386344)(228.55957031,142.42772075)
\curveto(229.29719528,141.71873466)(230.3212828,141.36424283)(231.63183594,141.36424419)
\curveto(232.39094219,141.36424283)(233.12499093,141.45734169)(233.83398438,141.64354106)
\curveto(234.55011972,141.82973715)(235.25910338,142.10903375)(235.9609375,142.48143169)
\lineto(235.9609375,140.61229106)
\curveto(235.25194193,140.3115095)(234.52505464,140.08234307)(233.78027344,139.92479106)
\curveto(233.03547279,139.76723921)(232.2799397,139.68846325)(231.51367188,139.68846294)
\curveto(229.59439551,139.68846325)(228.07258713,140.24705644)(226.94824219,141.36424419)
\curveto(225.83105292,142.48142921)(225.27245972,143.99249541)(225.27246094,145.89744731)
\curveto(225.27245972,147.86684049)(225.80240711,149.42803684)(226.86230469,150.58104106)
\curveto(227.92935811,151.74118557)(229.36522907,152.32126312)(231.16992188,152.32127544)
\curveto(232.78840273,152.32126312)(234.06672176,151.79847718)(235.00488281,150.75291606)
\curveto(235.95018342,149.71449489)(236.4228392,148.30010829)(236.42285156,146.509752)
\moveto(234.44628906,147.08983013)
\curveto(234.43195577,148.17120216)(234.1275941,149.03415703)(233.53320312,149.67869731)
\curveto(232.94595465,150.32321824)(232.16535648,150.64548354)(231.19140625,150.64549419)
\curveto(230.08853564,150.64548354)(229.20409642,150.33396042)(228.53808594,149.71092388)
\curveto(227.87922795,149.08786791)(227.49967104,148.21059015)(227.39941406,147.07908794)
\lineto(234.44628906,147.08983013)
}
}
{
\newrgbcolor{curcolor}{0 0 0}
\pscustom[linestyle=none,fillstyle=solid,fillcolor=curcolor]
{
\newpath
\moveto(246.63867188,150.18358013)
\curveto(246.41665784,150.31247606)(246.1731685,150.40557493)(245.90820312,150.462877)
\curveto(245.65038257,150.5273196)(245.36392452,150.55954613)(245.04882812,150.55955669)
\curveto(243.93163429,150.55954613)(243.07226015,150.19431212)(242.47070312,149.46385356)
\curveto(241.8762978,148.74053753)(241.57909758,147.69854639)(241.57910156,146.337877)
\lineto(241.57910156,139.99998638)
\lineto(239.59179688,139.99998638)
\lineto(239.59179688,152.03123638)
\lineto(241.57910156,152.03123638)
\lineto(241.57910156,150.16209575)
\curveto(241.99446175,150.89255361)(242.53515131,151.43324317)(243.20117188,151.78416606)
\curveto(243.86718123,152.14222684)(244.67642521,152.32126312)(245.62890625,152.32127544)
\curveto(245.76496579,152.32126312)(245.91535626,152.31052094)(246.08007812,152.28904888)
\curveto(246.24478302,152.27471369)(246.42740002,152.24964861)(246.62792969,152.21385356)
\lineto(246.63867188,150.18358013)
}
}
{
\newrgbcolor{curcolor}{0 0 0}
\pscustom[linestyle=none,fillstyle=solid,fillcolor=curcolor]
{
}
}
{
\newrgbcolor{curcolor}{0 0 0}
\pscustom[linestyle=none,fillstyle=solid,fillcolor=curcolor]
{
\newpath
\moveto(265.10449219,149.72166606)
\curveto(265.59862088,150.60967629)(266.1894406,151.26494907)(266.87695312,151.68748638)
\curveto(267.56443922,152.11000031)(268.37368321,152.32126312)(269.3046875,152.32127544)
\curveto(270.55792581,152.32126312)(271.52472172,151.88083387)(272.20507812,150.99998638)
\curveto(272.88539745,150.12627833)(273.22556638,148.88018583)(273.22558594,147.26170513)
\lineto(273.22558594,139.99998638)
\lineto(271.23828125,139.99998638)
\lineto(271.23828125,147.197252)
\curveto(271.23826368,148.35023844)(271.03416232,149.20603186)(270.62597656,149.76463481)
\curveto(270.21775688,150.32321824)(269.59471063,150.60251484)(268.75683594,150.60252544)
\curveto(267.73273333,150.60251484)(266.92348934,150.26234591)(266.32910156,149.58201763)
\curveto(265.73468845,148.90167018)(265.43748823,147.97426226)(265.4375,146.79979106)
\lineto(265.4375,139.99998638)
\lineto(263.45019531,139.99998638)
\lineto(263.45019531,147.197252)
\curveto(263.45018553,148.35739989)(263.24608417,149.21319331)(262.83789062,149.76463481)
\curveto(262.42967873,150.32321824)(261.79947103,150.60251484)(260.94726562,150.60252544)
\curveto(259.93749373,150.60251484)(259.1354112,150.25876518)(258.54101562,149.57127544)
\curveto(257.9466103,148.89092801)(257.64941008,147.96710081)(257.64941406,146.79979106)
\lineto(257.64941406,139.99998638)
\lineto(255.66210938,139.99998638)
\lineto(255.66210938,152.03123638)
\lineto(257.64941406,152.03123638)
\lineto(257.64941406,150.16209575)
\curveto(258.1005815,150.89971506)(258.64127106,151.44398535)(259.27148438,151.79490825)
\curveto(259.90168647,152.14580756)(260.65005812,152.32126312)(261.51660156,152.32127544)
\curveto(262.39029075,152.32126312)(263.13150095,152.09925813)(263.74023438,151.65525981)
\curveto(264.3561091,151.21123819)(264.81086125,150.56670758)(265.10449219,149.72166606)
}
}
{
\newrgbcolor{curcolor}{0 0 0}
\pscustom[linestyle=none,fillstyle=solid,fillcolor=curcolor]
{
\newpath
\moveto(281.84082031,150.64549419)
\curveto(280.7809188,150.64548354)(279.94302902,150.23011938)(279.32714844,149.39940044)
\curveto(278.71125942,148.57582416)(278.40331701,147.44431487)(278.40332031,146.00486919)
\curveto(278.40331701,144.5654115)(278.70767869,143.43032149)(279.31640625,142.59959575)
\curveto(279.93228684,141.77602627)(280.77375735,141.36424283)(281.84082031,141.36424419)
\curveto(282.8935469,141.36424283)(283.72785596,141.77960699)(284.34375,142.61033794)
\curveto(284.95962556,143.44106367)(285.26756796,144.57257295)(285.26757812,146.00486919)
\curveto(285.26756796,147.42999197)(284.95962556,148.55792053)(284.34375,149.38865825)
\curveto(283.72785596,150.22653865)(282.8935469,150.64548354)(281.84082031,150.64549419)
\moveto(281.84082031,152.32127544)
\curveto(283.55956186,152.32126312)(284.9094954,151.76266993)(285.890625,150.64549419)
\curveto(286.87173303,149.52829716)(287.36229243,147.98142371)(287.36230469,146.00486919)
\curveto(287.36229243,144.03546411)(286.87173303,142.48859066)(285.890625,141.36424419)
\curveto(284.9094954,140.24705644)(283.55956186,139.68846325)(281.84082031,139.68846294)
\curveto(280.11490384,139.68846325)(278.76138957,140.24705644)(277.78027344,141.36424419)
\curveto(276.8063134,142.48859066)(276.31933472,144.03546411)(276.31933594,146.00486919)
\curveto(276.31933472,147.98142371)(276.8063134,149.52829716)(277.78027344,150.64549419)
\curveto(278.76138957,151.76266993)(280.11490384,152.32126312)(281.84082031,152.32127544)
}
}
{
\newrgbcolor{curcolor}{0 0 0}
\pscustom[linestyle=none,fillstyle=solid,fillcolor=curcolor]
{
\newpath
\moveto(292.58300781,155.447252)
\lineto(292.58300781,152.03123638)
\lineto(296.65429688,152.03123638)
\lineto(296.65429688,150.49510356)
\lineto(292.58300781,150.49510356)
\lineto(292.58300781,143.96385356)
\curveto(292.58300378,142.98273079)(292.71549063,142.35252309)(292.98046875,142.07322856)
\curveto(293.25259947,141.7939299)(293.80045048,141.6542816)(294.62402344,141.65428325)
\lineto(296.65429688,141.65428325)
\lineto(296.65429688,139.99998638)
\lineto(294.62402344,139.99998638)
\curveto(293.09862827,139.99998638)(292.04589495,140.2828637)(291.46582031,140.84861919)
\curveto(290.88573986,141.42153443)(290.59570108,142.45994485)(290.59570312,143.96385356)
\lineto(290.59570312,150.49510356)
\lineto(289.14550781,150.49510356)
\lineto(289.14550781,152.03123638)
\lineto(290.59570312,152.03123638)
\lineto(290.59570312,155.447252)
\lineto(292.58300781,155.447252)
}
}
{
\newrgbcolor{curcolor}{0 0 0}
\pscustom[linestyle=none,fillstyle=solid,fillcolor=curcolor]
{
}
}
{
\newrgbcolor{curcolor}{0 0 0}
\pscustom[linestyle=none,fillstyle=solid,fillcolor=curcolor]
{
\newpath
\moveto(314.18554688,150.2050645)
\lineto(314.18554688,156.71483013)
\lineto(316.16210938,156.71483013)
\lineto(316.16210938,139.99998638)
\lineto(314.18554688,139.99998638)
\lineto(314.18554688,141.80467388)
\curveto(313.77017272,141.08852696)(313.24380606,140.55499884)(312.60644531,140.20408794)
\curveto(311.9762292,139.86033808)(311.21711537,139.68846325)(310.32910156,139.68846294)
\curveto(308.87532084,139.68846325)(307.69010067,140.2685408)(306.7734375,141.42869731)
\curveto(305.86393062,142.58885098)(305.40917847,144.11424008)(305.40917969,146.00486919)
\curveto(305.40917847,147.89548629)(305.86393062,149.42087539)(306.7734375,150.58104106)
\curveto(307.69010067,151.74118557)(308.87532084,152.32126312)(310.32910156,152.32127544)
\curveto(311.21711537,152.32126312)(311.9762292,152.14580756)(312.60644531,151.79490825)
\curveto(313.24380606,151.4511468)(313.77017272,150.92119941)(314.18554688,150.2050645)
\moveto(307.45019531,146.00486919)
\curveto(307.45019206,144.5510886)(307.74739228,143.40883713)(308.34179688,142.57811138)
\curveto(308.94335463,141.75454191)(309.76692151,141.34275847)(310.8125,141.34275981)
\curveto(311.85806525,141.34275847)(312.68163214,141.75454191)(313.28320312,142.57811138)
\curveto(313.88475594,143.40883713)(314.18553688,144.5510886)(314.18554688,146.00486919)
\curveto(314.18553688,147.45863777)(313.88475594,148.59730851)(313.28320312,149.42088481)
\curveto(312.68163214,150.25160373)(311.85806525,150.6669679)(310.8125,150.66697856)
\curveto(309.76692151,150.6669679)(308.94335463,150.25160373)(308.34179688,149.42088481)
\curveto(307.74739228,148.59730851)(307.45019206,147.45863777)(307.45019531,146.00486919)
}
}
{
\newrgbcolor{curcolor}{0 0 0}
\pscustom[linestyle=none,fillstyle=solid,fillcolor=curcolor]
{
\newpath
\moveto(330.52441406,146.509752)
\lineto(330.52441406,145.54295513)
\lineto(321.43652344,145.54295513)
\curveto(321.52245758,144.18227386)(321.93066029,143.14386344)(322.66113281,142.42772075)
\curveto(323.39875778,141.71873466)(324.4228453,141.36424283)(325.73339844,141.36424419)
\curveto(326.49250469,141.36424283)(327.22655343,141.45734169)(327.93554688,141.64354106)
\curveto(328.65168222,141.82973715)(329.36066588,142.10903375)(330.0625,142.48143169)
\lineto(330.0625,140.61229106)
\curveto(329.35350443,140.3115095)(328.62661714,140.08234307)(327.88183594,139.92479106)
\curveto(327.13703529,139.76723921)(326.3815022,139.68846325)(325.61523438,139.68846294)
\curveto(323.69595801,139.68846325)(322.17414963,140.24705644)(321.04980469,141.36424419)
\curveto(319.93261542,142.48142921)(319.37402222,143.99249541)(319.37402344,145.89744731)
\curveto(319.37402222,147.86684049)(319.90396961,149.42803684)(320.96386719,150.58104106)
\curveto(322.03092061,151.74118557)(323.46679157,152.32126312)(325.27148438,152.32127544)
\curveto(326.88996523,152.32126312)(328.16828426,151.79847718)(329.10644531,150.75291606)
\curveto(330.05174592,149.71449489)(330.5244017,148.30010829)(330.52441406,146.509752)
\moveto(328.54785156,147.08983013)
\curveto(328.53351827,148.17120216)(328.2291566,149.03415703)(327.63476562,149.67869731)
\curveto(327.04751715,150.32321824)(326.26691898,150.64548354)(325.29296875,150.64549419)
\curveto(324.19009814,150.64548354)(323.30565892,150.33396042)(322.63964844,149.71092388)
\curveto(321.98079045,149.08786791)(321.60123354,148.21059015)(321.50097656,147.07908794)
\lineto(328.54785156,147.08983013)
}
}
{
\newrgbcolor{curcolor}{0 0 0}
\pscustom[linestyle=none,fillstyle=solid,fillcolor=curcolor]
{
}
}
{
\newrgbcolor{curcolor}{0 0 0}
\pscustom[linestyle=none,fillstyle=solid,fillcolor=curcolor]
{
\newpath
\moveto(342.68457031,141.80467388)
\lineto(342.68457031,135.4238145)
\lineto(340.69726562,135.4238145)
\lineto(340.69726562,152.03123638)
\lineto(342.68457031,152.03123638)
\lineto(342.68457031,150.2050645)
\curveto(343.0999305,150.92119941)(343.62271643,151.4511468)(344.25292969,151.79490825)
\curveto(344.89029329,152.14580756)(345.64940711,152.32126312)(346.53027344,152.32127544)
\curveto(347.99120165,152.32126312)(349.17642181,151.74118557)(350.0859375,150.58104106)
\curveto(351.00259186,149.42087539)(351.46092474,147.89548629)(351.4609375,146.00486919)
\curveto(351.46092474,144.11424008)(351.00259186,142.58885098)(350.0859375,141.42869731)
\curveto(349.17642181,140.2685408)(347.99120165,139.68846325)(346.53027344,139.68846294)
\curveto(345.64940711,139.68846325)(344.89029329,139.86033808)(344.25292969,140.20408794)
\curveto(343.62271643,140.55499884)(343.0999305,141.08852696)(342.68457031,141.80467388)
\moveto(349.40917969,146.00486919)
\curveto(349.40916898,147.45863777)(349.10838803,148.59730851)(348.50683594,149.42088481)
\curveto(347.91242568,150.25160373)(347.09243952,150.6669679)(346.046875,150.66697856)
\curveto(345.00129578,150.6669679)(344.1777289,150.25160373)(343.57617188,149.42088481)
\curveto(342.98176655,148.59730851)(342.68456633,147.45863777)(342.68457031,146.00486919)
\curveto(342.68456633,144.5510886)(342.98176655,143.40883713)(343.57617188,142.57811138)
\curveto(344.1777289,141.75454191)(345.00129578,141.34275847)(346.046875,141.34275981)
\curveto(347.09243952,141.34275847)(347.91242568,141.75454191)(348.50683594,142.57811138)
\curveto(349.10838803,143.40883713)(349.40916898,144.5510886)(349.40917969,146.00486919)
}
}
{
\newrgbcolor{curcolor}{0 0 0}
\pscustom[linestyle=none,fillstyle=solid,fillcolor=curcolor]
{
\newpath
\moveto(360.20507812,146.04783794)
\curveto(358.60806697,146.04783189)(357.50162277,145.86521489)(356.88574219,145.49998638)
\curveto(356.26985317,145.13474687)(355.96191076,144.51170062)(355.96191406,143.63084575)
\curveto(355.96191076,142.92901991)(356.1910772,142.37042671)(356.64941406,141.9550645)
\curveto(357.1149044,141.54685983)(357.74511211,141.34275847)(358.54003906,141.34275981)
\curveto(359.63573522,141.34275847)(360.51301298,141.72947683)(361.171875,142.50291606)
\curveto(361.83788145,143.28351174)(362.17088893,144.31834143)(362.17089844,145.60740825)
\lineto(362.17089844,146.04783794)
\lineto(360.20507812,146.04783794)
\moveto(364.14746094,146.86424419)
\lineto(364.14746094,139.99998638)
\lineto(362.17089844,139.99998638)
\lineto(362.17089844,141.82615825)
\curveto(361.71971751,141.09568841)(361.15754359,140.55499884)(360.484375,140.20408794)
\curveto(359.81119077,139.86033808)(358.98762388,139.68846325)(358.01367188,139.68846294)
\curveto(356.78189692,139.68846325)(355.80077811,140.03221291)(355.0703125,140.71971294)
\curveto(354.34700353,141.41437298)(353.98535024,142.34178091)(353.98535156,143.5019395)
\curveto(353.98535024,144.85545027)(354.43652167,145.87595706)(355.33886719,146.56346294)
\curveto(356.24836881,147.25095569)(357.60188308,147.59470534)(359.39941406,147.59471294)
\lineto(362.17089844,147.59471294)
\lineto(362.17089844,147.78807231)
\curveto(362.17088893,148.69756883)(361.87010798,149.39939104)(361.26855469,149.89354106)
\curveto(360.67414564,150.39483275)(359.83625585,150.64548354)(358.75488281,150.64549419)
\curveto(358.06737741,150.64548354)(357.39778172,150.56312686)(356.74609375,150.39842388)
\curveto(356.09439761,150.2337001)(355.46777063,149.98663004)(354.86621094,149.65721294)
\lineto(354.86621094,151.48338481)
\curveto(355.5895153,151.76266993)(356.29133752,151.97035201)(356.97167969,152.10643169)
\curveto(357.65201324,152.24964861)(358.31444747,152.32126312)(358.95898438,152.32127544)
\curveto(360.69921071,152.32126312)(361.9990141,151.87009169)(362.85839844,150.96775981)
\curveto(363.71776238,150.065406)(364.14744945,148.69756883)(364.14746094,146.86424419)
}
}
{
\newrgbcolor{curcolor}{0 0 0}
\pscustom[linestyle=none,fillstyle=solid,fillcolor=curcolor]
{
\newpath
\moveto(375.89941406,151.67674419)
\lineto(375.89941406,149.80760356)
\curveto(375.34081113,150.0940518)(374.76073358,150.30889534)(374.15917969,150.45213481)
\curveto(373.55760979,150.59535339)(372.93456353,150.6669679)(372.29003906,150.66697856)
\curveto(371.30891412,150.6669679)(370.57128465,150.51657742)(370.07714844,150.21580669)
\curveto(369.59016584,149.91501552)(369.3466765,149.4638441)(369.34667969,148.86229106)
\curveto(369.3466765,148.40394933)(369.52213205,148.04229604)(369.87304688,147.77733013)
\curveto(370.22395427,147.51951011)(370.92935721,147.27244004)(371.98925781,147.03611919)
\lineto(372.66601562,146.88572856)
\curveto(374.06965354,146.58494073)(375.06509526,146.15883438)(375.65234375,145.60740825)
\curveto(376.2467347,145.06313236)(376.54393492,144.30043781)(376.54394531,143.31932231)
\curveto(376.54393492,142.20213261)(376.09992495,141.31769339)(375.21191406,140.666002)
\curveto(374.33104651,140.01430928)(373.11718054,139.68846325)(371.5703125,139.68846294)
\curveto(370.92577648,139.68846325)(370.25260007,139.75291631)(369.55078125,139.88182231)
\curveto(368.85611709,140.0035671)(368.12206835,140.18976483)(367.34863281,140.44041606)
\lineto(367.34863281,142.48143169)
\curveto(368.07909964,142.1018723)(368.79882548,141.81541425)(369.5078125,141.62205669)
\curveto(370.21679281,141.43585734)(370.91861503,141.34275847)(371.61328125,141.34275981)
\curveto(372.54426445,141.34275847)(373.26040956,141.5003104)(373.76171875,141.81541606)
\curveto(374.26301273,142.13767955)(374.51366352,142.58885098)(374.51367188,143.16893169)
\curveto(374.51366352,143.70603736)(374.33104651,144.1178208)(373.96582031,144.40428325)
\curveto(373.60773994,144.69073689)(372.81639959,144.96645276)(371.59179688,145.23143169)
\lineto(370.90429688,145.3925645)
\curveto(369.67968398,145.65037135)(368.79524476,146.04425117)(368.25097656,146.57420513)
\curveto(367.70670418,147.11130739)(367.43456903,147.84535614)(367.43457031,148.77635356)
\curveto(367.43456903,149.90785407)(367.8356103,150.78155112)(368.63769531,151.39744731)
\curveto(369.43977536,152.01332072)(370.5784461,152.32126312)(372.05371094,152.32127544)
\curveto(372.78417306,152.32126312)(373.47167237,152.26755223)(374.11621094,152.16014263)
\curveto(374.76073358,152.0527087)(375.35513403,151.89157605)(375.89941406,151.67674419)
}
}
{
\newrgbcolor{curcolor}{0 0 0}
\pscustom[linestyle=none,fillstyle=solid,fillcolor=curcolor]
{
\newpath
\moveto(387.37207031,151.67674419)
\lineto(387.37207031,149.80760356)
\curveto(386.81346738,150.0940518)(386.23338983,150.30889534)(385.63183594,150.45213481)
\curveto(385.03026604,150.59535339)(384.40721978,150.6669679)(383.76269531,150.66697856)
\curveto(382.78157037,150.6669679)(382.0439409,150.51657742)(381.54980469,150.21580669)
\curveto(381.06282209,149.91501552)(380.81933275,149.4638441)(380.81933594,148.86229106)
\curveto(380.81933275,148.40394933)(380.9947883,148.04229604)(381.34570312,147.77733013)
\curveto(381.69661052,147.51951011)(382.40201346,147.27244004)(383.46191406,147.03611919)
\lineto(384.13867188,146.88572856)
\curveto(385.54230979,146.58494073)(386.53775151,146.15883438)(387.125,145.60740825)
\curveto(387.71939095,145.06313236)(388.01659117,144.30043781)(388.01660156,143.31932231)
\curveto(388.01659117,142.20213261)(387.5725812,141.31769339)(386.68457031,140.666002)
\curveto(385.80370276,140.01430928)(384.58983679,139.68846325)(383.04296875,139.68846294)
\curveto(382.39843273,139.68846325)(381.72525632,139.75291631)(381.0234375,139.88182231)
\curveto(380.32877334,140.0035671)(379.5947246,140.18976483)(378.82128906,140.44041606)
\lineto(378.82128906,142.48143169)
\curveto(379.55175589,142.1018723)(380.27148173,141.81541425)(380.98046875,141.62205669)
\curveto(381.68944906,141.43585734)(382.39127128,141.34275847)(383.0859375,141.34275981)
\curveto(384.0169207,141.34275847)(384.73306581,141.5003104)(385.234375,141.81541606)
\curveto(385.73566898,142.13767955)(385.98631977,142.58885098)(385.98632812,143.16893169)
\curveto(385.98631977,143.70603736)(385.80370276,144.1178208)(385.43847656,144.40428325)
\curveto(385.08039619,144.69073689)(384.28905584,144.96645276)(383.06445312,145.23143169)
\lineto(382.37695312,145.3925645)
\curveto(381.15234023,145.65037135)(380.26790101,146.04425117)(379.72363281,146.57420513)
\curveto(379.17936043,147.11130739)(378.90722528,147.84535614)(378.90722656,148.77635356)
\curveto(378.90722528,149.90785407)(379.30826655,150.78155112)(380.11035156,151.39744731)
\curveto(380.91243161,152.01332072)(382.05110235,152.32126312)(383.52636719,152.32127544)
\curveto(384.25682931,152.32126312)(384.94432862,152.26755223)(385.58886719,152.16014263)
\curveto(386.23338983,152.0527087)(386.82779028,151.89157605)(387.37207031,151.67674419)
}
}
{
\newrgbcolor{curcolor}{0 0 0}
\pscustom[linestyle=none,fillstyle=solid,fillcolor=curcolor]
{
\newpath
\moveto(401.46582031,146.509752)
\lineto(401.46582031,145.54295513)
\lineto(392.37792969,145.54295513)
\curveto(392.46386383,144.18227386)(392.87206654,143.14386344)(393.60253906,142.42772075)
\curveto(394.34016403,141.71873466)(395.36425155,141.36424283)(396.67480469,141.36424419)
\curveto(397.43391094,141.36424283)(398.16795968,141.45734169)(398.87695312,141.64354106)
\curveto(399.59308847,141.82973715)(400.30207213,142.10903375)(401.00390625,142.48143169)
\lineto(401.00390625,140.61229106)
\curveto(400.29491068,140.3115095)(399.56802339,140.08234307)(398.82324219,139.92479106)
\curveto(398.07844154,139.76723921)(397.32290845,139.68846325)(396.55664062,139.68846294)
\curveto(394.63736426,139.68846325)(393.11555588,140.24705644)(391.99121094,141.36424419)
\curveto(390.87402167,142.48142921)(390.31542847,143.99249541)(390.31542969,145.89744731)
\curveto(390.31542847,147.86684049)(390.84537586,149.42803684)(391.90527344,150.58104106)
\curveto(392.97232686,151.74118557)(394.40819782,152.32126312)(396.21289062,152.32127544)
\curveto(397.83137148,152.32126312)(399.10969051,151.79847718)(400.04785156,150.75291606)
\curveto(400.99315217,149.71449489)(401.46580795,148.30010829)(401.46582031,146.509752)
\moveto(399.48925781,147.08983013)
\curveto(399.47492452,148.17120216)(399.17056285,149.03415703)(398.57617188,149.67869731)
\curveto(397.9889234,150.32321824)(397.20832523,150.64548354)(396.234375,150.64549419)
\curveto(395.13150439,150.64548354)(394.24706517,150.33396042)(393.58105469,149.71092388)
\curveto(392.9221967,149.08786791)(392.54263979,148.21059015)(392.44238281,147.07908794)
\lineto(399.48925781,147.08983013)
}
}
\end{pspicture}

		\end{center}
				
		\begin{enumerate}
		  \item Fond d'écran (présent dans l'archive stocké sur le téléphone)
		  \item Champs de texte ``Pseudo''
		  \item Liste déroulante ``Couleur joueur''
		  \item Champs de texte ``Nom d'utilisateur''
		  \item Bouton ``Modifier''
		  \item Bouton ``Connexion automatique''
		  \item Liste déroulante ``Type''
		  \item Bouton ``\hyperlink{Options}{Valider}''
		  \item Bouton ``\hyperlink{Options}{Retour}''
		\end{enumerate}

		\subsubsection{Description des zones}
		
			\begin{tabular}{|c|c|c|c|c|} \hline
				Numéro de zone & Type  & Description & Evènement &	Règle \\\hline
				2 & Champs de texte & Pseudonyme de l'utilisateur & Chargement de la page & RG4-02 \\
				  &                 &                             & Perte du focus & RG1-01 \\\hline
				3 & Champs de texte & Couleur du personnage souhaitée & Chargement de la page & RG9-01 \\
				  &                 &                             & Perte du focus & RG9-02 \\\hline
				4 & Champs de texte & Nom de l'utilisateur pour & Chargement de la page & RG9-03 \\
				  &                 & les parties en ligne      & Perte du focus & RG9-04 \\\hline
				5 & Bouton & Permet la modification du mot de & Cliqué & RG9-05 \\
				  &        & passe pour les parties multi-joueurs & & \\\hline
				6 & Check box & Permet la connexion automatique & Chargement de la page & RG9-06 \\
				  &           & lors des parties multi-joueurs  & Cliquée               & RG4-05 \\\hline
				7 & Liste déroulante & Permet de choisir le type & Chargement de la page & RG9-07 \\
				  &				     & de l'affichage            & Perte du focus & RG9-08 \\
				  &				     & (droitier/gaucher)        & & \\\hline
				8 & Bouton & Renvoie à l'écran des options & Cliqué & RG9-09 \\\hline

			\end{tabular}
			
		\subsubsection{Description des règles}

			\underline{RG9-01 :}
				\begin{quote}
					Récuperer dans la classe utilisateur la couleur utilisée.\\
				\end{quote}


			\underline{RG9-02 :}
				\begin{quote}
					Modifier la couleur dans le fichier utilisateur.\\
				\end{quote}

				
			\underline{RG9-03 :}
				\begin{quote}
					Récuperer dans la classe utilisateur le nom du compte hors ligne utilisé.\\
				\end{quote}		


			\underline{RG9-04 :}
				\begin{quote}
					Modifier le nom d'utilisateur pour les parties en ligne dans le fichier utilisateur.\\
				\end{quote}
				

			\underline{RG9-05 :}
				\begin{quote}
					Lancer une sous activité pour la modification du mot de passe.\\
				\end{quote}


			\underline{RG9-06 :}
				\begin{quote}
					Récuperer dans la classe utilisateur la valeur de la connexion automatique.\\
				\end{quote}
			 

			\underline{RG9-07 :}
				\begin{quote}
					Récuperer dans la classe utilisateur le type de l'affichage.\\
				\end{quote}


			\underline{RG9-08 :}
				\begin{quote}
					Modifier le type de l'affichage dans le fichier utilisateur.\\
				\end{quote}		
				
				
			\underline{RG9-09 :}
				\begin{quote}
					Recharger la classe utilisateur.\\
					Charger la page des options%
						\footnote[1]{
							\hyperlink{Options}{Options}
							\og voir section \ref{Options}, page \pageref{Options}.\fg
						}.\\
					Afficher la page des options\footnotemark[1].\\
					Supprimer la page de gestion du profil%
						\footnote[2]{
							\hyperlink{Gestion du profil}{Gestion du profil}
							\og voir section \ref{Gestion du profil}, page \pageref{Gestion du profil}.\fg
						}.\\
				\end{quote}
				

\newpage

	\subsection{Statistiques}
		
		\hypertarget{Statistiques}{}
		\label{Statistiques}
			
		\begin{center}
			\input{./tex/10_Mes_statistiques}
		\end{center}
		
		\begin{enumerate}
		  \item Fond d'écran (présent dans l'archive stocké sur le téléphone)
		  \item Pseudonyme du compte hors ligne utilisé
		  \item Text Box ``Statistiques''
		  \item Bouton ``\hyperlink{Accueil}{Retour}''
		\end{enumerate}

		\subsubsection{Description des zones}
		
			\begin{tabular}{|c|c|c|c|c|} \hline
				Numéro de zone & Type  & Description & Evènement &	Règle \\\hline
				2 & Label & Pseudonyme hors ligne utilisé & Chargement de la page & RG4-01 \\\hline
				3 & Label & Statistiques du joueur & Chargement de la page & RG10-01 \\\hline
				4 & Bouton & Permet de revenir à la page d'accueil & Cliqué & RG10-02 \\\hline 
			\end{tabular}
			
		\subsubsection{Description des règles}

			\underline{RG10-01 :}
				\begin{quote}
					Récuperer dans la classe utilisateur la valeur de la connexion automatique.\\
				\end{quote}
				
			\underline{RG10-02 :}
				\begin{quote}
					Charger la page d'accueil%
						\footnote[3]{
							\hyperlink{Page d'accueil}{Page d'accueil}
							\og voir section \ref{Accueil}, page \pageref{Accueil}.\fg
						}.\\
					Afficher la page d'accueil\footnotemark[3].\\
					Supprimer la page des statistiques%
						\footnote[2]{
							\hyperlink{Statistiques}{Statistiques}
							\og voir section \ref{Statistiques}, page \pageref{Statistiques}.\fg
						}.\\		
				\end{quote}
	
\newpage
	
	\subsection{Créer niveau}
		
		\hypertarget{Creer niveau}{}
		\label{Creer niveau}
			
		\begin{center}
			%LaTeX with PSTricks extensions
%%Creator: inkscape 0.47
%%Please note this file requires PSTricks extensions
\psset{xunit=.5pt,yunit=.5pt,runit=.5pt}
\begin{pspicture}(560,600)
{
\newrgbcolor{curcolor}{1 1 1}
\pscustom[linestyle=none,fillstyle=solid,fillcolor=curcolor]
{
\newpath
\moveto(133.12401716,597.5221417)
\lineto(426.87598554,597.5221417)
\curveto(443.85397304,597.5221417)(457.52217237,583.85394237)(457.52217237,566.87595487)
\lineto(457.52217237,33.124017)
\curveto(457.52217237,16.1460295)(443.85397304,2.47783017)(426.87598554,2.47783017)
\lineto(133.12401716,2.47783017)
\curveto(116.14602965,2.47783017)(102.47783033,16.1460295)(102.47783033,33.124017)
\lineto(102.47783033,566.87595487)
\curveto(102.47783033,583.85394237)(116.14602965,597.5221417)(133.12401716,597.5221417)
\closepath
}
}
{
\newrgbcolor{curcolor}{0 0 0}
\pscustom[linewidth=4.95566034,linecolor=curcolor]
{
\newpath
\moveto(133.12401716,597.5221417)
\lineto(426.87598554,597.5221417)
\curveto(443.85397304,597.5221417)(457.52217237,583.85394237)(457.52217237,566.87595487)
\lineto(457.52217237,33.124017)
\curveto(457.52217237,16.1460295)(443.85397304,2.47783017)(426.87598554,2.47783017)
\lineto(133.12401716,2.47783017)
\curveto(116.14602965,2.47783017)(102.47783033,16.1460295)(102.47783033,33.124017)
\lineto(102.47783033,566.87595487)
\curveto(102.47783033,583.85394237)(116.14602965,597.5221417)(133.12401716,597.5221417)
\closepath
}
}
{
\newrgbcolor{curcolor}{0 0 0}
\pscustom[linestyle=none,fillstyle=solid,fillcolor=curcolor]
{
\newpath
\moveto(53.96874944,532.65625119)
\lineto(59.12499944,532.65625119)
\lineto(59.12499944,550.45312619)
\lineto(53.51562444,549.32812619)
\lineto(53.51562444,552.20312619)
\lineto(59.09374944,553.32812619)
\lineto(62.24999944,553.32812619)
\lineto(62.24999944,532.65625119)
\lineto(67.40624944,532.65625119)
\lineto(67.40624944,530.00000119)
\lineto(53.96874944,530.00000119)
\lineto(53.96874944,532.65625119)
}
}
{
\newrgbcolor{curcolor}{0 0 0}
\pscustom[linestyle=none,fillstyle=solid,fillcolor=curcolor]
{
\newpath
\moveto(505.14062254,432.65625119)
\lineto(516.15624754,432.65625119)
\lineto(516.15624754,430.00000119)
\lineto(501.34374754,430.00000119)
\lineto(501.34374754,432.65625119)
\curveto(502.54166066,433.89583063)(504.17186736,435.5572873)(506.23437254,437.64062619)
\curveto(508.30727989,439.73436646)(509.60936193,441.08332344)(510.14062254,441.68750119)
\curveto(511.15102705,442.82290504)(511.85415135,443.78123741)(512.24999754,444.56250119)
\curveto(512.65623388,445.35415251)(512.85935868,446.1301934)(512.85937254,446.89062619)
\curveto(512.85935868,448.1301914)(512.42185911,449.14060705)(511.54687254,449.92187619)
\curveto(510.68227752,450.70310549)(509.55207032,451.0937301)(508.15624754,451.09375119)
\curveto(507.16665604,451.0937301)(506.11978208,450.92185527)(505.01562254,450.57812619)
\curveto(503.92186761,450.23435596)(502.74999379,449.71352315)(501.49999754,449.01562619)
\lineto(501.49999754,452.20312619)
\curveto(502.7708271,452.71352015)(503.95832591,453.09893643)(505.06249754,453.35937619)
\curveto(506.16665704,453.61976924)(507.17707269,453.74997744)(508.09374754,453.75000119)
\curveto(510.51040269,453.74997744)(512.4374841,453.14581138)(513.87499754,451.93750119)
\curveto(515.31248122,450.72914713)(516.0312305,449.11456541)(516.03124754,447.09375119)
\curveto(516.0312305,446.13540172)(515.84893902,445.2239443)(515.48437254,444.35937619)
\curveto(515.13018974,443.50519602)(514.47914872,442.49478037)(513.53124754,441.32812619)
\curveto(513.2708166,441.02603183)(512.44269243,440.15103271)(511.04687254,438.70312619)
\curveto(509.65102855,437.26561893)(507.68228052,435.24999594)(505.14062254,432.65625119)
}
}
{
\newrgbcolor{curcolor}{0 0 0}
\pscustom[linestyle=none,fillstyle=solid,fillcolor=curcolor]
{
\newpath
\moveto(322.38072721,451.00000119)
\lineto(425.61926786,451.00000119)
\curveto(433.0321921,451.00000119)(438.99999754,445.03219576)(438.99999754,437.61927152)
\lineto(438.99999754,433.38072896)
\curveto(438.99999754,425.96780472)(433.0321921,419.99999929)(425.61926786,419.99999929)
\lineto(322.38072721,419.99999929)
\curveto(314.96780297,419.99999929)(308.99999754,425.96780472)(308.99999754,433.38072896)
\lineto(308.99999754,437.61927152)
\curveto(308.99999754,445.03219576)(314.96780297,451.00000119)(322.38072721,451.00000119)
\closepath
}
}
{
\newrgbcolor{curcolor}{0 0 0}
\pscustom[linewidth=1.5675832,linecolor=curcolor]
{
\newpath
\moveto(322.38072721,451.00000119)
\lineto(425.61926786,451.00000119)
\curveto(433.0321921,451.00000119)(438.99999754,445.03219576)(438.99999754,437.61927152)
\lineto(438.99999754,433.38072896)
\curveto(438.99999754,425.96780472)(433.0321921,419.99999929)(425.61926786,419.99999929)
\lineto(322.38072721,419.99999929)
\curveto(314.96780297,419.99999929)(308.99999754,425.96780472)(308.99999754,433.38072896)
\lineto(308.99999754,437.61927152)
\curveto(308.99999754,445.03219576)(314.96780297,451.00000119)(322.38072721,451.00000119)
\closepath
}
}
{
\newrgbcolor{curcolor}{0 0 0}
\pscustom[linestyle=none,fillstyle=solid,fillcolor=curcolor]
{
\newpath
\moveto(111.35547391,447.49609494)
\lineto(114.54297391,447.49609494)
\lineto(122.30078641,432.85937619)
\lineto(122.30078641,447.49609494)
\lineto(124.59766141,447.49609494)
\lineto(124.59766141,430.00000119)
\lineto(121.41016141,430.00000119)
\lineto(113.65234891,444.63671994)
\lineto(113.65234891,430.00000119)
\lineto(111.35547391,430.00000119)
\lineto(111.35547391,447.49609494)
}
}
{
\newrgbcolor{curcolor}{0 0 0}
\pscustom[linestyle=none,fillstyle=solid,fillcolor=curcolor]
{
\newpath
\moveto(134.30078641,441.61328244)
\curveto(133.14453022,441.61327083)(132.23046864,441.16014628)(131.55859891,440.25390744)
\curveto(130.88671998,439.35546059)(130.55078282,438.12108682)(130.55078641,436.55078244)
\curveto(130.55078282,434.98046496)(130.88281373,433.74218495)(131.54688016,432.83593869)
\curveto(132.2187499,431.93749926)(133.13671773,431.48828096)(134.30078641,431.48828244)
\curveto(135.44921542,431.48828096)(136.35937076,431.9414055)(137.03125516,432.84765744)
\curveto(137.70311941,433.75390369)(138.03905658,434.98827746)(138.03906766,436.55078244)
\curveto(138.03905658,438.10546184)(137.70311941,439.33592936)(137.03125516,440.24218869)
\curveto(136.35937076,441.15624004)(135.44921542,441.61327083)(134.30078641,441.61328244)
\moveto(134.30078641,443.44140744)
\curveto(136.17577719,443.441394)(137.64843197,442.83201961)(138.71875516,441.61328244)
\curveto(139.78905483,440.39452205)(140.32421054,438.70702374)(140.32422391,436.55078244)
\curveto(140.32421054,434.40234054)(139.78905483,432.71484223)(138.71875516,431.48828244)
\curveto(137.64843197,430.26953217)(136.17577719,429.66015778)(134.30078641,429.66015744)
\curveto(132.41796845,429.66015778)(130.94140743,430.26953217)(129.87109891,431.48828244)
\curveto(128.80859706,432.71484223)(128.27734759,434.40234054)(128.27734891,436.55078244)
\curveto(128.27734759,438.70702374)(128.80859706,440.39452205)(129.87109891,441.61328244)
\curveto(130.94140743,442.83201961)(132.41796845,443.441394)(134.30078641,443.44140744)
}
}
{
\newrgbcolor{curcolor}{0 0 0}
\pscustom[linestyle=none,fillstyle=solid,fillcolor=curcolor]
{
\newpath
\moveto(154.10547391,440.60546994)
\curveto(154.6445234,441.57420837)(155.289054,442.2890514)(156.03906766,442.75000119)
\curveto(156.7890525,443.21092548)(157.67186412,443.441394)(158.68750516,443.44140744)
\curveto(160.05467423,443.441394)(161.10936068,442.96092573)(161.85156766,442.00000119)
\curveto(162.5937342,441.04686515)(162.96482757,439.68749151)(162.96484891,437.92187619)
\lineto(162.96484891,430.00000119)
\lineto(160.79688016,430.00000119)
\lineto(160.79688016,437.85156369)
\curveto(160.79686099,439.10936708)(160.57420497,440.0429599)(160.12891141,440.65234494)
\curveto(159.68358086,441.26170868)(159.00389404,441.56639588)(158.08984891,441.56640744)
\curveto(156.97264607,441.56639588)(156.08983445,441.1953025)(155.44141141,440.45312619)
\curveto(154.79296075,439.71092898)(154.46874232,438.69921124)(154.46875516,437.41796994)
\lineto(154.46875516,430.00000119)
\lineto(152.30078641,430.00000119)
\lineto(152.30078641,437.85156369)
\curveto(152.30077574,439.11717958)(152.07811971,440.05077239)(151.63281766,440.65234494)
\curveto(151.1874956,441.26170868)(150.49999629,441.56639588)(149.57031766,441.56640744)
\curveto(148.46874832,441.56639588)(147.5937492,441.19139625)(146.94531766,440.44140744)
\curveto(146.29687549,439.69921024)(145.97265707,438.69139875)(145.97266141,437.41796994)
\lineto(145.97266141,430.00000119)
\lineto(143.80469266,430.00000119)
\lineto(143.80469266,443.12500119)
\lineto(145.97266141,443.12500119)
\lineto(145.97266141,441.08593869)
\curveto(146.46484407,441.8906143)(147.05468723,442.48436371)(147.74219266,442.86718869)
\curveto(148.42968586,443.24998794)(149.24609129,443.441394)(150.19141141,443.44140744)
\curveto(151.1445269,443.441394)(151.95311984,443.19920674)(152.61719266,442.71484494)
\curveto(153.289056,442.23045771)(153.78514925,441.52733342)(154.10547391,440.60546994)
}
}
{
\newrgbcolor{curcolor}{0 0 0}
\pscustom[linestyle=none,fillstyle=solid,fillcolor=curcolor]
{
}
}
{
\newrgbcolor{curcolor}{0 0 0}
\pscustom[linestyle=none,fillstyle=solid,fillcolor=curcolor]
{
\newpath
\moveto(183.55469266,441.13281369)
\lineto(183.55469266,448.23437619)
\lineto(185.71094266,448.23437619)
\lineto(185.71094266,430.00000119)
\lineto(183.55469266,430.00000119)
\lineto(183.55469266,431.96875119)
\curveto(183.10155722,431.18750001)(182.52733904,430.60546934)(181.83203641,430.22265744)
\curveto(181.14452793,429.8476576)(180.31640375,429.66015778)(179.34766141,429.66015744)
\curveto(177.76171881,429.66015778)(176.46875135,430.29296965)(175.46875516,431.55859494)
\curveto(174.47656584,432.82421712)(173.98047259,434.48827796)(173.98047391,436.55078244)
\curveto(173.98047259,438.61327383)(174.47656584,440.27733467)(175.46875516,441.54296994)
\curveto(176.46875135,442.80858213)(177.76171881,443.441394)(179.34766141,443.44140744)
\curveto(180.31640375,443.441394)(181.14452793,443.24998794)(181.83203641,442.86718869)
\curveto(182.52733904,442.4921762)(183.10155722,441.91405178)(183.55469266,441.13281369)
\moveto(176.20703641,436.55078244)
\curveto(176.20703286,434.96483998)(176.53125129,433.71874747)(177.17969266,432.81250119)
\curveto(177.83593748,431.91406178)(178.73437409,431.46484348)(179.87500516,431.46484494)
\curveto(181.01562181,431.46484348)(181.91405841,431.91406178)(182.57031766,432.81250119)
\curveto(183.22655709,433.71874747)(183.55468177,434.96483998)(183.55469266,436.55078244)
\curveto(183.55468177,438.13671181)(183.22655709,439.37889806)(182.57031766,440.27734494)
\curveto(181.91405841,441.18358376)(181.01562181,441.63670831)(179.87500516,441.63671994)
\curveto(178.73437409,441.63670831)(177.83593748,441.18358376)(177.17969266,440.27734494)
\curveto(176.53125129,439.37889806)(176.20703286,438.13671181)(176.20703641,436.55078244)
}
}
{
\newrgbcolor{curcolor}{0 0 0}
\pscustom[linestyle=none,fillstyle=solid,fillcolor=curcolor]
{
\newpath
\moveto(189.92969266,435.17968869)
\lineto(189.92969266,443.12500119)
\lineto(192.08594266,443.12500119)
\lineto(192.08594266,435.26171994)
\curveto(192.08593847,434.01952842)(192.32812573,433.08593561)(192.81250516,432.46093869)
\curveto(193.29687476,431.84374935)(194.02343653,431.53515591)(194.99219266,431.53515744)
\curveto(196.1562469,431.53515591)(197.07421473,431.90624929)(197.74609891,432.64843869)
\curveto(198.42577588,433.3906228)(198.76561929,434.40234054)(198.76563016,435.68359494)
\lineto(198.76563016,443.12500119)
\lineto(200.92188016,443.12500119)
\lineto(200.92188016,430.00000119)
\lineto(198.76563016,430.00000119)
\lineto(198.76563016,432.01562619)
\curveto(198.24218231,431.21874997)(197.63280792,430.62500057)(196.93750516,430.23437619)
\curveto(196.24999681,429.85156384)(195.44921636,429.66015778)(194.53516141,429.66015744)
\curveto(193.02734378,429.66015778)(191.88281367,430.12890731)(191.10156766,431.06640744)
\curveto(190.32031523,432.00390544)(189.92969063,433.37499782)(189.92969266,435.17968869)
\moveto(195.35547391,443.44140744)
\lineto(195.35547391,443.44140744)
}
}
{
\newrgbcolor{curcolor}{0 0 0}
\pscustom[linestyle=none,fillstyle=solid,fillcolor=curcolor]
{
}
}
{
\newrgbcolor{curcolor}{0 0 0}
\pscustom[linestyle=none,fillstyle=solid,fillcolor=curcolor]
{
\newpath
\moveto(223.93750516,437.92187619)
\lineto(223.93750516,430.00000119)
\lineto(221.78125516,430.00000119)
\lineto(221.78125516,437.85156369)
\curveto(221.78124415,439.0937421)(221.53905689,440.02342867)(221.05469266,440.64062619)
\curveto(220.57030786,441.25780244)(219.84374609,441.56639588)(218.87500516,441.56640744)
\curveto(217.71093572,441.56639588)(216.79296789,441.1953025)(216.12109891,440.45312619)
\curveto(215.44921923,439.71092898)(215.11328207,438.69921124)(215.11328641,437.41796994)
\lineto(215.11328641,430.00000119)
\lineto(212.94531766,430.00000119)
\lineto(212.94531766,443.12500119)
\lineto(215.11328641,443.12500119)
\lineto(215.11328641,441.08593869)
\curveto(215.62890655,441.87498932)(216.2343747,442.46483248)(216.92969266,442.85546994)
\curveto(217.6328108,443.2460817)(218.44140374,443.441394)(219.35547391,443.44140744)
\curveto(220.86327632,443.441394)(222.00390018,442.97264447)(222.77734891,442.03515744)
\curveto(223.55077363,441.10545884)(223.93749199,439.73436646)(223.93750516,437.92187619)
}
}
{
\newrgbcolor{curcolor}{0 0 0}
\pscustom[linestyle=none,fillstyle=solid,fillcolor=curcolor]
{
\newpath
\moveto(228.26172391,443.12500119)
\lineto(230.41797391,443.12500119)
\lineto(230.41797391,430.00000119)
\lineto(228.26172391,430.00000119)
\lineto(228.26172391,443.12500119)
\moveto(228.26172391,448.23437619)
\lineto(230.41797391,448.23437619)
\lineto(230.41797391,445.50390744)
\lineto(228.26172391,445.50390744)
\lineto(228.26172391,448.23437619)
}
}
{
\newrgbcolor{curcolor}{0 0 0}
\pscustom[linestyle=none,fillstyle=solid,fillcolor=curcolor]
{
\newpath
\moveto(233.37109891,443.12500119)
\lineto(235.65625516,443.12500119)
\lineto(239.75781766,432.10937619)
\lineto(243.85938016,443.12500119)
\lineto(246.14453641,443.12500119)
\lineto(241.22266141,430.00000119)
\lineto(238.29297391,430.00000119)
\lineto(233.37109891,443.12500119)
}
}
{
\newrgbcolor{curcolor}{0 0 0}
\pscustom[linestyle=none,fillstyle=solid,fillcolor=curcolor]
{
\newpath
\moveto(260.34766141,437.10156369)
\lineto(260.34766141,436.04687619)
\lineto(250.43359891,436.04687619)
\curveto(250.52734525,434.56249663)(250.9726573,433.42968526)(251.76953641,432.64843869)
\curveto(252.5742182,431.87499932)(253.69140458,431.48828096)(255.12109891,431.48828244)
\curveto(255.94921482,431.48828096)(256.74999527,431.58984335)(257.52344266,431.79296994)
\curveto(258.30468122,431.99609295)(259.07811795,432.30078014)(259.84375516,432.70703244)
\lineto(259.84375516,430.66796994)
\curveto(259.07030545,430.3398446)(258.2773375,430.08984485)(257.46484891,429.91796994)
\curveto(256.65233912,429.7460952)(255.8281212,429.66015778)(254.99219266,429.66015744)
\curveto(252.89843663,429.66015778)(251.23828204,430.26953217)(250.01172391,431.48828244)
\curveto(248.79297198,432.70702974)(248.18359759,434.35546559)(248.18359891,436.43359494)
\curveto(248.18359759,438.58202386)(248.76172201,440.28514716)(249.91797391,441.54296994)
\curveto(251.08203219,442.80858213)(252.64843688,443.441394)(254.61719266,443.44140744)
\curveto(256.38280814,443.441394)(257.777338,442.87108207)(258.80078641,441.73046994)
\curveto(259.83202344,440.59764685)(260.34764793,439.05467964)(260.34766141,437.10156369)
\moveto(258.19141141,437.73437619)
\curveto(258.1757751,438.91405478)(257.84374418,439.85546009)(257.19531766,440.55859494)
\curveto(256.55468297,441.26170868)(255.70312132,441.61327083)(254.64063016,441.61328244)
\curveto(253.43749859,441.61327083)(252.4726558,441.27342742)(251.74609891,440.59375119)
\curveto(251.02734475,439.91405378)(250.61328266,438.95702349)(250.50391141,437.72265744)
\lineto(258.19141141,437.73437619)
}
}
{
\newrgbcolor{curcolor}{0 0 0}
\pscustom[linestyle=none,fillstyle=solid,fillcolor=curcolor]
{
\newpath
\moveto(269.85156766,436.59765744)
\curveto(268.10937368,436.59765085)(266.90234364,436.39843229)(266.23047391,436.00000119)
\curveto(265.55859498,435.60155809)(265.22265782,434.92187127)(265.22266141,433.96093869)
\curveto(265.22265782,433.1953105)(265.47265757,432.58593611)(265.97266141,432.13281369)
\curveto(266.48046906,431.68749951)(267.16796837,431.46484348)(268.03516141,431.46484494)
\curveto(269.23046631,431.46484348)(270.1874966,431.88671806)(270.90625516,432.73046994)
\curveto(271.63280766,433.58202886)(271.99608854,434.71093398)(271.99609891,436.11718869)
\lineto(271.99609891,436.59765744)
\lineto(269.85156766,436.59765744)
\moveto(274.15234891,437.48828244)
\lineto(274.15234891,430.00000119)
\lineto(271.99609891,430.00000119)
\lineto(271.99609891,431.99218869)
\curveto(271.50390154,431.1953125)(270.8906209,430.60546934)(270.15625516,430.22265744)
\curveto(269.42187237,429.8476576)(268.52343577,429.66015778)(267.46094266,429.66015744)
\curveto(266.11718817,429.66015778)(265.04687674,430.03515741)(264.25000516,430.78515744)
\curveto(263.46094083,431.5429684)(263.06640997,432.55468614)(263.06641141,433.82031369)
\curveto(263.06640997,435.2968709)(263.55859698,436.41015103)(264.54297391,437.16015744)
\curveto(265.5351575,437.91014953)(267.01171853,438.28514916)(268.97266141,438.28515744)
\lineto(271.99609891,438.28515744)
\lineto(271.99609891,438.49609494)
\curveto(271.99608854,439.48827296)(271.66796387,440.25389719)(271.01172391,440.79296994)
\curveto(270.36327768,441.3398336)(269.44921609,441.61327083)(268.26953641,441.61328244)
\curveto(267.51953052,441.61327083)(266.7890625,441.52342717)(266.07813016,441.34375119)
\curveto(265.36718892,441.16405253)(264.68359586,440.89452155)(264.02734891,440.53515744)
\lineto(264.02734891,442.52734494)
\curveto(264.81640822,442.83201961)(265.58203246,443.05858188)(266.32422391,443.20703244)
\curveto(267.06640597,443.36326908)(267.7890615,443.441394)(268.49219266,443.44140744)
\curveto(270.3906214,443.441394)(271.80858873,442.94920699)(272.74609891,441.96484494)
\curveto(273.68358686,440.98045896)(274.15233639,439.48827296)(274.15234891,437.48828244)
}
}
{
\newrgbcolor{curcolor}{0 0 0}
\pscustom[linestyle=none,fillstyle=solid,fillcolor=curcolor]
{
\newpath
\moveto(278.38281766,435.17968869)
\lineto(278.38281766,443.12500119)
\lineto(280.53906766,443.12500119)
\lineto(280.53906766,435.26171994)
\curveto(280.53906347,434.01952842)(280.78125073,433.08593561)(281.26563016,432.46093869)
\curveto(281.74999976,431.84374935)(282.47656153,431.53515591)(283.44531766,431.53515744)
\curveto(284.6093719,431.53515591)(285.52733973,431.90624929)(286.19922391,432.64843869)
\curveto(286.87890088,433.3906228)(287.21874429,434.40234054)(287.21875516,435.68359494)
\lineto(287.21875516,443.12500119)
\lineto(289.37500516,443.12500119)
\lineto(289.37500516,430.00000119)
\lineto(287.21875516,430.00000119)
\lineto(287.21875516,432.01562619)
\curveto(286.69530731,431.21874997)(286.08593292,430.62500057)(285.39063016,430.23437619)
\curveto(284.70312181,429.85156384)(283.90234136,429.66015778)(282.98828641,429.66015744)
\curveto(281.48046878,429.66015778)(280.33593867,430.12890731)(279.55469266,431.06640744)
\curveto(278.77344023,432.00390544)(278.38281563,433.37499782)(278.38281766,435.17968869)
\moveto(283.80859891,443.44140744)
\lineto(283.80859891,443.44140744)
}
}
{
\newrgbcolor{curcolor}{0 0 0}
\pscustom[linestyle=none,fillstyle=solid,fillcolor=curcolor]
{
\newpath
\moveto(142.35546629,397.49609494)
\lineto(145.54296629,397.49609494)
\lineto(153.30077879,382.85937619)
\lineto(153.30077879,397.49609494)
\lineto(155.59765379,397.49609494)
\lineto(155.59765379,380.00000119)
\lineto(152.41015379,380.00000119)
\lineto(144.65234129,394.63671994)
\lineto(144.65234129,380.00000119)
\lineto(142.35546629,380.00000119)
\lineto(142.35546629,397.49609494)
}
}
{
\newrgbcolor{curcolor}{0 0 0}
\pscustom[linestyle=none,fillstyle=solid,fillcolor=curcolor]
{
\newpath
\moveto(165.30077879,391.61328244)
\curveto(164.14452259,391.61327083)(163.23046101,391.16014628)(162.55859129,390.25390744)
\curveto(161.88671235,389.35546059)(161.55077519,388.12108682)(161.55077879,386.55078244)
\curveto(161.55077519,384.98046496)(161.88280611,383.74218495)(162.54687254,382.83593869)
\curveto(163.21874227,381.93749926)(164.1367101,381.48828096)(165.30077879,381.48828244)
\curveto(166.44920779,381.48828096)(167.35936313,381.9414055)(168.03124754,382.84765744)
\curveto(168.70311179,383.75390369)(169.03904895,384.98827746)(169.03906004,386.55078244)
\curveto(169.03904895,388.10546184)(168.70311179,389.33592936)(168.03124754,390.24218869)
\curveto(167.35936313,391.15624004)(166.44920779,391.61327083)(165.30077879,391.61328244)
\moveto(165.30077879,393.44140744)
\curveto(167.17576956,393.441394)(168.64842434,392.83201961)(169.71874754,391.61328244)
\curveto(170.7890472,390.39452205)(171.32420291,388.70702374)(171.32421629,386.55078244)
\curveto(171.32420291,384.40234054)(170.7890472,382.71484223)(169.71874754,381.48828244)
\curveto(168.64842434,380.26953217)(167.17576956,379.66015778)(165.30077879,379.66015744)
\curveto(163.41796082,379.66015778)(161.9413998,380.26953217)(160.87109129,381.48828244)
\curveto(159.80858943,382.71484223)(159.27733996,384.40234054)(159.27734129,386.55078244)
\curveto(159.27733996,388.70702374)(159.80858943,390.39452205)(160.87109129,391.61328244)
\curveto(161.9413998,392.83201961)(163.41796082,393.441394)(165.30077879,393.44140744)
}
}
{
\newrgbcolor{curcolor}{0 0 0}
\pscustom[linestyle=none,fillstyle=solid,fillcolor=curcolor]
{
\newpath
\moveto(185.10546629,390.60546994)
\curveto(185.64451577,391.57420837)(186.28904637,392.2890514)(187.03906004,392.75000119)
\curveto(187.78904487,393.21092548)(188.67185649,393.441394)(189.68749754,393.44140744)
\curveto(191.05466661,393.441394)(192.10935305,392.96092573)(192.85156004,392.00000119)
\curveto(193.59372657,391.04686515)(193.96481995,389.68749151)(193.96484129,387.92187619)
\lineto(193.96484129,380.00000119)
\lineto(191.79687254,380.00000119)
\lineto(191.79687254,387.85156369)
\curveto(191.79685336,389.10936708)(191.57419734,390.0429599)(191.12890379,390.65234494)
\curveto(190.68357323,391.26170868)(190.00388641,391.56639588)(189.08984129,391.56640744)
\curveto(187.97263844,391.56639588)(187.08982682,391.1953025)(186.44140379,390.45312619)
\curveto(185.79295312,389.71092898)(185.46873469,388.69921124)(185.46874754,387.41796994)
\lineto(185.46874754,380.00000119)
\lineto(183.30077879,380.00000119)
\lineto(183.30077879,387.85156369)
\curveto(183.30076811,389.11717958)(183.07811208,390.05077239)(182.63281004,390.65234494)
\curveto(182.18748797,391.26170868)(181.49998866,391.56639588)(180.57031004,391.56640744)
\curveto(179.46874069,391.56639588)(178.59374157,391.19139625)(177.94531004,390.44140744)
\curveto(177.29686786,389.69921024)(176.97264944,388.69139875)(176.97265379,387.41796994)
\lineto(176.97265379,380.00000119)
\lineto(174.80468504,380.00000119)
\lineto(174.80468504,393.12500119)
\lineto(176.97265379,393.12500119)
\lineto(176.97265379,391.08593869)
\curveto(177.46483645,391.8906143)(178.05467961,392.48436371)(178.74218504,392.86718869)
\curveto(179.42967823,393.24998794)(180.24608366,393.441394)(181.19140379,393.44140744)
\curveto(182.14451927,393.441394)(182.95311221,393.19920674)(183.61718504,392.71484494)
\curveto(184.28904837,392.23045771)(184.78514162,391.52733342)(185.10546629,390.60546994)
}
}
{
\newrgbcolor{curcolor}{0 0 0}
\pscustom[linestyle=none,fillstyle=solid,fillcolor=curcolor]
{
\newpath
\moveto(207.69921629,386.55078244)
\curveto(207.6992046,388.13671181)(207.37107993,389.37889806)(206.71484129,390.27734494)
\curveto(206.06639373,391.18358376)(205.17186338,391.63670831)(204.03124754,391.63671994)
\curveto(202.89061566,391.63670831)(201.99217906,391.18358376)(201.33593504,390.27734494)
\curveto(200.68749286,389.37889806)(200.36327444,388.13671181)(200.36327879,386.55078244)
\curveto(200.36327444,384.96483998)(200.68749286,383.71874747)(201.33593504,382.81250119)
\curveto(201.99217906,381.91406178)(202.89061566,381.46484348)(204.03124754,381.46484494)
\curveto(205.17186338,381.46484348)(206.06639373,381.91406178)(206.71484129,382.81250119)
\curveto(207.37107993,383.71874747)(207.6992046,384.96483998)(207.69921629,386.55078244)
\moveto(200.36327879,391.13281369)
\curveto(200.81639898,391.91405178)(201.38671091,392.4921762)(202.07421629,392.86718869)
\curveto(202.76952203,393.24998794)(203.5976462,393.441394)(204.55859129,393.44140744)
\curveto(206.15233115,393.441394)(207.44529861,392.80858213)(208.43749754,391.54296994)
\curveto(209.43748411,390.27733467)(209.93748361,388.61327383)(209.93749754,386.55078244)
\curveto(209.93748361,384.48827796)(209.43748411,382.82421712)(208.43749754,381.55859494)
\curveto(207.44529861,380.29296965)(206.15233115,379.66015778)(204.55859129,379.66015744)
\curveto(203.5976462,379.66015778)(202.76952203,379.8476576)(202.07421629,380.22265744)
\curveto(201.38671091,380.60546934)(200.81639898,381.18750001)(200.36327879,381.96875119)
\lineto(200.36327879,380.00000119)
\lineto(198.19531004,380.00000119)
\lineto(198.19531004,398.23437619)
\lineto(200.36327879,398.23437619)
\lineto(200.36327879,391.13281369)
}
}
{
\newrgbcolor{curcolor}{0 0 0}
\pscustom[linestyle=none,fillstyle=solid,fillcolor=curcolor]
{
\newpath
\moveto(221.11718504,391.10937619)
\curveto(220.87498791,391.24998994)(220.60936318,391.35155234)(220.32031004,391.41406369)
\curveto(220.03905125,391.48436471)(219.72655156,391.51952092)(219.38281004,391.51953244)
\curveto(218.16405312,391.51952092)(217.22655406,391.12108382)(216.57031004,390.32421994)
\curveto(215.92186786,389.53514791)(215.59764944,388.39843029)(215.59765379,386.91406369)
\lineto(215.59765379,380.00000119)
\lineto(213.42968504,380.00000119)
\lineto(213.42968504,393.12500119)
\lineto(215.59765379,393.12500119)
\lineto(215.59765379,391.08593869)
\curveto(216.05077398,391.88280181)(216.64061714,392.47264497)(217.36718504,392.85546994)
\curveto(218.09374069,393.2460817)(218.97655231,393.441394)(220.01562254,393.44140744)
\curveto(220.16405112,393.441394)(220.32811346,393.42967526)(220.50781004,393.40625119)
\curveto(220.6874881,393.3906128)(220.88670665,393.36326908)(221.10546629,393.32421994)
\lineto(221.11718504,391.10937619)
}
}
{
\newrgbcolor{curcolor}{0 0 0}
\pscustom[linestyle=none,fillstyle=solid,fillcolor=curcolor]
{
\newpath
\moveto(234.11327879,387.10156369)
\lineto(234.11327879,386.04687619)
\lineto(224.19921629,386.04687619)
\curveto(224.29296262,384.56249663)(224.73827467,383.42968526)(225.53515379,382.64843869)
\curveto(226.33983557,381.87499932)(227.45702195,381.48828096)(228.88671629,381.48828244)
\curveto(229.7148322,381.48828096)(230.51561264,381.58984335)(231.28906004,381.79296994)
\curveto(232.07029859,381.99609295)(232.84373532,382.30078014)(233.60937254,382.70703244)
\lineto(233.60937254,380.66796994)
\curveto(232.83592282,380.3398446)(232.04295487,380.08984485)(231.23046629,379.91796994)
\curveto(230.41795649,379.7460952)(229.59373857,379.66015778)(228.75781004,379.66015744)
\curveto(226.664054,379.66015778)(225.00389941,380.26953217)(223.77734129,381.48828244)
\curveto(222.55858935,382.70702974)(221.94921496,384.35546559)(221.94921629,386.43359494)
\curveto(221.94921496,388.58202386)(222.52733938,390.28514716)(223.68359129,391.54296994)
\curveto(224.84764956,392.80858213)(226.41405425,393.441394)(228.38281004,393.44140744)
\curveto(230.14842551,393.441394)(231.54295537,392.87108207)(232.56640379,391.73046994)
\curveto(233.59764081,390.59764685)(234.1132653,389.05467964)(234.11327879,387.10156369)
\moveto(231.95702879,387.73437619)
\curveto(231.94139247,388.91405478)(231.60936155,389.85546009)(230.96093504,390.55859494)
\curveto(230.32030034,391.26170868)(229.46873869,391.61327083)(228.40624754,391.61328244)
\curveto(227.20311596,391.61327083)(226.23827317,391.27342742)(225.51171629,390.59375119)
\curveto(224.79296212,389.91405378)(224.37890003,388.95702349)(224.26952879,387.72265744)
\lineto(231.95702879,387.73437619)
}
}
{
\newrgbcolor{curcolor}{0 0 0}
\pscustom[linestyle=none,fillstyle=solid,fillcolor=curcolor]
{
}
}
{
\newrgbcolor{curcolor}{0 0 0}
\pscustom[linestyle=none,fillstyle=solid,fillcolor=curcolor]
{
\newpath
\moveto(253.92968504,391.13281369)
\lineto(253.92968504,398.23437619)
\lineto(256.08593504,398.23437619)
\lineto(256.08593504,380.00000119)
\lineto(253.92968504,380.00000119)
\lineto(253.92968504,381.96875119)
\curveto(253.47654959,381.18750001)(252.90233141,380.60546934)(252.20702879,380.22265744)
\curveto(251.5195203,379.8476576)(250.69139612,379.66015778)(249.72265379,379.66015744)
\curveto(248.13671118,379.66015778)(246.84374372,380.29296965)(245.84374754,381.55859494)
\curveto(244.85155821,382.82421712)(244.35546496,384.48827796)(244.35546629,386.55078244)
\curveto(244.35546496,388.61327383)(244.85155821,390.27733467)(245.84374754,391.54296994)
\curveto(246.84374372,392.80858213)(248.13671118,393.441394)(249.72265379,393.44140744)
\curveto(250.69139612,393.441394)(251.5195203,393.24998794)(252.20702879,392.86718869)
\curveto(252.90233141,392.4921762)(253.47654959,391.91405178)(253.92968504,391.13281369)
\moveto(246.58202879,386.55078244)
\curveto(246.58202523,384.96483998)(246.90624366,383.71874747)(247.55468504,382.81250119)
\curveto(248.21092986,381.91406178)(249.10936646,381.46484348)(250.24999754,381.46484494)
\curveto(251.39061418,381.46484348)(252.28905078,381.91406178)(252.94531004,382.81250119)
\curveto(253.60154946,383.71874747)(253.92967414,384.96483998)(253.92968504,386.55078244)
\curveto(253.92967414,388.13671181)(253.60154946,389.37889806)(252.94531004,390.27734494)
\curveto(252.28905078,391.18358376)(251.39061418,391.63670831)(250.24999754,391.63671994)
\curveto(249.10936646,391.63670831)(248.21092986,391.18358376)(247.55468504,390.27734494)
\curveto(246.90624366,389.37889806)(246.58202523,388.13671181)(246.58202879,386.55078244)
}
}
{
\newrgbcolor{curcolor}{0 0 0}
\pscustom[linestyle=none,fillstyle=solid,fillcolor=curcolor]
{
\newpath
\moveto(271.75390379,387.10156369)
\lineto(271.75390379,386.04687619)
\lineto(261.83984129,386.04687619)
\curveto(261.93358762,384.56249663)(262.37889967,383.42968526)(263.17577879,382.64843869)
\curveto(263.98046057,381.87499932)(265.09764695,381.48828096)(266.52734129,381.48828244)
\curveto(267.3554572,381.48828096)(268.15623764,381.58984335)(268.92968504,381.79296994)
\curveto(269.71092359,381.99609295)(270.48436032,382.30078014)(271.24999754,382.70703244)
\lineto(271.24999754,380.66796994)
\curveto(270.47654782,380.3398446)(269.68357987,380.08984485)(268.87109129,379.91796994)
\curveto(268.05858149,379.7460952)(267.23436357,379.66015778)(266.39843504,379.66015744)
\curveto(264.304679,379.66015778)(262.64452441,380.26953217)(261.41796629,381.48828244)
\curveto(260.19921435,382.70702974)(259.58983996,384.35546559)(259.58984129,386.43359494)
\curveto(259.58983996,388.58202386)(260.16796438,390.28514716)(261.32421629,391.54296994)
\curveto(262.48827456,392.80858213)(264.05467925,393.441394)(266.02343504,393.44140744)
\curveto(267.78905051,393.441394)(269.18358037,392.87108207)(270.20702879,391.73046994)
\curveto(271.23826581,390.59764685)(271.7538903,389.05467964)(271.75390379,387.10156369)
\moveto(269.59765379,387.73437619)
\curveto(269.58201747,388.91405478)(269.24998655,389.85546009)(268.60156004,390.55859494)
\curveto(267.96092534,391.26170868)(267.10936369,391.61327083)(266.04687254,391.61328244)
\curveto(264.84374096,391.61327083)(263.87889817,391.27342742)(263.15234129,390.59375119)
\curveto(262.43358712,389.91405378)(262.01952503,388.95702349)(261.91015379,387.72265744)
\lineto(269.59765379,387.73437619)
}
}
{
\newrgbcolor{curcolor}{0 0 0}
\pscustom[linestyle=none,fillstyle=solid,fillcolor=curcolor]
{
}
}
{
\newrgbcolor{curcolor}{0 0 0}
\pscustom[linestyle=none,fillstyle=solid,fillcolor=curcolor]
{
\newpath
\moveto(292.37890379,392.62109494)
\lineto(292.37890379,390.60546994)
\curveto(291.76951769,390.9413965)(291.15623705,391.19139625)(290.53906004,391.35546994)
\curveto(289.92967578,391.52733342)(289.31248889,391.61327083)(288.68749754,391.61328244)
\curveto(287.28905342,391.61327083)(286.203117,391.16795878)(285.42968504,390.27734494)
\curveto(284.65624355,389.39452305)(284.26952519,388.15233679)(284.26952879,386.55078244)
\curveto(284.26952519,384.94921499)(284.65624355,383.70312249)(285.42968504,382.81250119)
\curveto(286.203117,381.92968676)(287.28905342,381.48828096)(288.68749754,381.48828244)
\curveto(289.31248889,381.48828096)(289.92967578,381.57031212)(290.53906004,381.73437619)
\curveto(291.15623705,381.90624929)(291.76951769,382.16015528)(292.37890379,382.49609494)
\lineto(292.37890379,380.50390744)
\curveto(291.77733018,380.22265722)(291.1523308,380.01171993)(290.50390379,379.87109494)
\curveto(289.86326959,379.73047021)(289.17967653,379.66015778)(288.45312254,379.66015744)
\curveto(286.47655423,379.66015778)(284.9062433,380.28125091)(283.74218504,381.52343869)
\curveto(282.57812063,382.76562343)(281.99608996,384.441403)(281.99609129,386.55078244)
\curveto(281.99608996,388.69139875)(282.58202687,390.37499082)(283.75390379,391.60156369)
\curveto(284.93358702,392.82811337)(286.54686666,393.441394)(288.59374754,393.44140744)
\curveto(289.25780145,393.441394)(289.9062383,393.37108157)(290.53906004,393.23046994)
\curveto(291.17186204,393.09764435)(291.78514267,392.89451955)(292.37890379,392.62109494)
}
}
{
\newrgbcolor{curcolor}{0 0 0}
\pscustom[linestyle=none,fillstyle=solid,fillcolor=curcolor]
{
\newpath
\moveto(302.11718504,386.59765744)
\curveto(300.37499105,386.59765085)(299.16796101,386.39843229)(298.49609129,386.00000119)
\curveto(297.82421235,385.60155809)(297.48827519,384.92187127)(297.48827879,383.96093869)
\curveto(297.48827519,383.1953105)(297.73827494,382.58593611)(298.23827879,382.13281369)
\curveto(298.74608643,381.68749951)(299.43358574,381.46484348)(300.30077879,381.46484494)
\curveto(301.49608368,381.46484348)(302.45311397,381.88671806)(303.17187254,382.73046994)
\curveto(303.89842503,383.58202886)(304.26170591,384.71093398)(304.26171629,386.11718869)
\lineto(304.26171629,386.59765744)
\lineto(302.11718504,386.59765744)
\moveto(306.41796629,387.48828244)
\lineto(306.41796629,380.00000119)
\lineto(304.26171629,380.00000119)
\lineto(304.26171629,381.99218869)
\curveto(303.76951891,381.1953125)(303.15623827,380.60546934)(302.42187254,380.22265744)
\curveto(301.68748974,379.8476576)(300.78905314,379.66015778)(299.72656004,379.66015744)
\curveto(298.38280554,379.66015778)(297.31249411,380.03515741)(296.51562254,380.78515744)
\curveto(295.7265582,381.5429684)(295.33202734,382.55468614)(295.33202879,383.82031369)
\curveto(295.33202734,385.2968709)(295.82421435,386.41015103)(296.80859129,387.16015744)
\curveto(297.80077488,387.91014953)(299.2773359,388.28514916)(301.23827879,388.28515744)
\lineto(304.26171629,388.28515744)
\lineto(304.26171629,388.49609494)
\curveto(304.26170591,389.48827296)(303.93358124,390.25389719)(303.27734129,390.79296994)
\curveto(302.62889505,391.3398336)(301.71483346,391.61327083)(300.53515379,391.61328244)
\curveto(299.78514789,391.61327083)(299.05467987,391.52342717)(298.34374754,391.34375119)
\curveto(297.63280629,391.16405253)(296.94921323,390.89452155)(296.29296629,390.53515744)
\lineto(296.29296629,392.52734494)
\curveto(297.08202559,392.83201961)(297.84764983,393.05858188)(298.58984129,393.20703244)
\curveto(299.33202334,393.36326908)(300.05467887,393.441394)(300.75781004,393.44140744)
\curveto(302.65623877,393.441394)(304.0742061,392.94920699)(305.01171629,391.96484494)
\curveto(305.94920423,390.98045896)(306.41795376,389.48827296)(306.41796629,387.48828244)
}
}
{
\newrgbcolor{curcolor}{0 0 0}
\pscustom[linestyle=none,fillstyle=solid,fillcolor=curcolor]
{
\newpath
\moveto(319.23827879,392.73828244)
\lineto(319.23827879,390.69921994)
\curveto(318.62889377,391.01170893)(317.9960819,391.2460837)(317.33984129,391.40234494)
\curveto(316.68358321,391.55858338)(316.00389639,391.63670831)(315.30077879,391.63671994)
\curveto(314.23046066,391.63670831)(313.42577397,391.47264597)(312.88671629,391.14453244)
\curveto(312.35546254,390.81639663)(312.0898378,390.32420962)(312.08984129,389.66796994)
\curveto(312.0898378,389.16796078)(312.28124386,388.77342992)(312.66406004,388.48437619)
\curveto(313.0468681,388.20311799)(313.81639858,387.93358701)(314.97265379,387.67578244)
\lineto(315.71093504,387.51171994)
\curveto(317.2421764,387.18358776)(318.32811282,386.71874447)(318.96874754,386.11718869)
\curveto(319.61717403,385.52343317)(319.94139245,384.69140275)(319.94140379,383.62109494)
\curveto(319.94139245,382.40234254)(319.45701794,381.43749976)(318.48827879,380.72656369)
\curveto(317.52733237,380.01562618)(316.20311494,379.66015778)(314.51562254,379.66015744)
\curveto(313.81249233,379.66015778)(313.07811807,379.73047021)(312.31249754,379.87109494)
\curveto(311.55468209,380.00390744)(310.75390164,380.20703224)(309.91015379,380.48046994)
\lineto(309.91015379,382.70703244)
\curveto(310.70702669,382.29296765)(311.49218215,381.98046796)(312.26562254,381.76953244)
\curveto(313.03905561,381.56640588)(313.80467984,381.46484348)(314.56249754,381.46484494)
\curveto(315.57811557,381.46484348)(316.35936479,381.63671831)(316.90624754,381.98046994)
\curveto(317.45311369,382.33203011)(317.72655092,382.82421712)(317.72656004,383.45703244)
\curveto(317.72655092,384.0429659)(317.52733237,384.4921842)(317.12890379,384.80468869)
\curveto(316.73827066,385.11718358)(315.87499027,385.41796453)(314.53906004,385.70703244)
\lineto(313.78906004,385.88281369)
\curveto(312.45311869,386.16405753)(311.48827591,386.5937446)(310.89452879,387.17187619)
\curveto(310.30077709,387.75780594)(310.00390239,388.55858638)(310.00390379,389.57421994)
\curveto(310.00390239,390.80858413)(310.44140195,391.76170818)(311.31640379,392.43359494)
\curveto(312.1914002,393.10545684)(313.43358646,393.441394)(315.04296629,393.44140744)
\curveto(315.83983405,393.441394)(316.5898333,393.38280031)(317.29296629,393.26562619)
\curveto(317.9960819,393.14842554)(318.64451875,392.97264447)(319.23827879,392.73828244)
}
}
{
\newrgbcolor{curcolor}{0 0 0}
\pscustom[linestyle=none,fillstyle=solid,fillcolor=curcolor]
{
\newpath
\moveto(334.61327879,387.10156369)
\lineto(334.61327879,386.04687619)
\lineto(324.69921629,386.04687619)
\curveto(324.79296262,384.56249663)(325.23827467,383.42968526)(326.03515379,382.64843869)
\curveto(326.83983557,381.87499932)(327.95702195,381.48828096)(329.38671629,381.48828244)
\curveto(330.2148322,381.48828096)(331.01561264,381.58984335)(331.78906004,381.79296994)
\curveto(332.57029859,381.99609295)(333.34373532,382.30078014)(334.10937254,382.70703244)
\lineto(334.10937254,380.66796994)
\curveto(333.33592282,380.3398446)(332.54295487,380.08984485)(331.73046629,379.91796994)
\curveto(330.91795649,379.7460952)(330.09373857,379.66015778)(329.25781004,379.66015744)
\curveto(327.164054,379.66015778)(325.50389941,380.26953217)(324.27734129,381.48828244)
\curveto(323.05858935,382.70702974)(322.44921496,384.35546559)(322.44921629,386.43359494)
\curveto(322.44921496,388.58202386)(323.02733938,390.28514716)(324.18359129,391.54296994)
\curveto(325.34764956,392.80858213)(326.91405425,393.441394)(328.88281004,393.44140744)
\curveto(330.64842551,393.441394)(332.04295537,392.87108207)(333.06640379,391.73046994)
\curveto(334.09764081,390.59764685)(334.6132653,389.05467964)(334.61327879,387.10156369)
\moveto(332.45702879,387.73437619)
\curveto(332.44139247,388.91405478)(332.10936155,389.85546009)(331.46093504,390.55859494)
\curveto(330.82030034,391.26170868)(329.96873869,391.61327083)(328.90624754,391.61328244)
\curveto(327.70311596,391.61327083)(326.73827317,391.27342742)(326.01171629,390.59375119)
\curveto(325.29296212,389.91405378)(324.87890003,388.95702349)(324.76952879,387.72265744)
\lineto(332.45702879,387.73437619)
}
}
{
\newrgbcolor{curcolor}{0 0 0}
\pscustom[linestyle=none,fillstyle=solid,fillcolor=curcolor]
{
\newpath
\moveto(346.51952879,392.73828244)
\lineto(346.51952879,390.69921994)
\curveto(345.91014377,391.01170893)(345.2773319,391.2460837)(344.62109129,391.40234494)
\curveto(343.96483321,391.55858338)(343.28514639,391.63670831)(342.58202879,391.63671994)
\curveto(341.51171066,391.63670831)(340.70702397,391.47264597)(340.16796629,391.14453244)
\curveto(339.63671254,390.81639663)(339.3710878,390.32420962)(339.37109129,389.66796994)
\curveto(339.3710878,389.16796078)(339.56249386,388.77342992)(339.94531004,388.48437619)
\curveto(340.3281181,388.20311799)(341.09764858,387.93358701)(342.25390379,387.67578244)
\lineto(342.99218504,387.51171994)
\curveto(344.5234264,387.18358776)(345.60936282,386.71874447)(346.24999754,386.11718869)
\curveto(346.89842403,385.52343317)(347.22264245,384.69140275)(347.22265379,383.62109494)
\curveto(347.22264245,382.40234254)(346.73826794,381.43749976)(345.76952879,380.72656369)
\curveto(344.80858237,380.01562618)(343.48436494,379.66015778)(341.79687254,379.66015744)
\curveto(341.09374233,379.66015778)(340.35936807,379.73047021)(339.59374754,379.87109494)
\curveto(338.83593209,380.00390744)(338.03515164,380.20703224)(337.19140379,380.48046994)
\lineto(337.19140379,382.70703244)
\curveto(337.98827669,382.29296765)(338.77343215,381.98046796)(339.54687254,381.76953244)
\curveto(340.32030561,381.56640588)(341.08592984,381.46484348)(341.84374754,381.46484494)
\curveto(342.85936557,381.46484348)(343.64061479,381.63671831)(344.18749754,381.98046994)
\curveto(344.73436369,382.33203011)(345.00780092,382.82421712)(345.00781004,383.45703244)
\curveto(345.00780092,384.0429659)(344.80858237,384.4921842)(344.41015379,384.80468869)
\curveto(344.01952066,385.11718358)(343.15624027,385.41796453)(341.82031004,385.70703244)
\lineto(341.07031004,385.88281369)
\curveto(339.73436869,386.16405753)(338.76952591,386.5937446)(338.17577879,387.17187619)
\curveto(337.58202709,387.75780594)(337.28515239,388.55858638)(337.28515379,389.57421994)
\curveto(337.28515239,390.80858413)(337.72265195,391.76170818)(338.59765379,392.43359494)
\curveto(339.4726502,393.10545684)(340.71483646,393.441394)(342.32421629,393.44140744)
\curveto(343.12108405,393.441394)(343.8710833,393.38280031)(344.57421629,393.26562619)
\curveto(345.2773319,393.14842554)(345.92576875,392.97264447)(346.51952879,392.73828244)
}
}
{
\newrgbcolor{curcolor}{1 1 1}
\pscustom[linestyle=none,fillstyle=solid,fillcolor=curcolor]
{
\newpath
\moveto(135.42911665,380.00000119)
\lineto(422.57089368,380.00000119)
\curveto(432.22662144,380.00000119)(440.00000516,372.22661747)(440.00000516,362.57088971)
\lineto(440.00000516,277.42911267)
\curveto(440.00000516,267.77338491)(432.22662144,260.00000119)(422.57089368,260.00000119)
\lineto(135.42911665,260.00000119)
\curveto(125.77338888,260.00000119)(118.00000516,267.77338491)(118.00000516,277.42911267)
\lineto(118.00000516,362.57088971)
\curveto(118.00000516,372.22661747)(125.77338888,380.00000119)(135.42911665,380.00000119)
\closepath
}
}
{
\newrgbcolor{curcolor}{0 0 0}
\pscustom[linewidth=1.50827098,linecolor=curcolor]
{
\newpath
\moveto(135.42911665,380.00000119)
\lineto(422.57089368,380.00000119)
\curveto(432.22662144,380.00000119)(440.00000516,372.22661747)(440.00000516,362.57088971)
\lineto(440.00000516,277.42911267)
\curveto(440.00000516,267.77338491)(432.22662144,260.00000119)(422.57089368,260.00000119)
\lineto(135.42911665,260.00000119)
\curveto(125.77338888,260.00000119)(118.00000516,267.77338491)(118.00000516,277.42911267)
\lineto(118.00000516,362.57088971)
\curveto(118.00000516,372.22661747)(125.77338888,380.00000119)(135.42911665,380.00000119)
\closepath
}
}
{
\newrgbcolor{curcolor}{0 0 0}
\pscustom[linestyle=none,fillstyle=solid,fillcolor=curcolor]
{
\newpath
\moveto(284.99999849,310.00000119)
\lineto(404.99999658,310.00000119)
\curveto(413.30999711,310.00000119)(419.99999754,303.31000077)(419.99999754,295.00000024)
\curveto(419.99999754,286.68999971)(413.30999711,279.99999929)(404.99999658,279.99999929)
\lineto(284.99999849,279.99999929)
\curveto(276.68999796,279.99999929)(269.99999754,286.68999971)(269.99999754,295.00000024)
\curveto(269.99999754,303.31000077)(276.68999796,310.00000119)(284.99999849,310.00000119)
\closepath
}
}
{
\newrgbcolor{curcolor}{0 0 0}
\pscustom[linewidth=2,linecolor=curcolor]
{
\newpath
\moveto(284.99999849,310.00000119)
\lineto(404.99999658,310.00000119)
\curveto(413.30999711,310.00000119)(419.99999754,303.31000077)(419.99999754,295.00000024)
\curveto(419.99999754,286.68999971)(413.30999711,279.99999929)(404.99999658,279.99999929)
\lineto(284.99999849,279.99999929)
\curveto(276.68999796,279.99999929)(269.99999754,286.68999971)(269.99999754,295.00000024)
\curveto(269.99999754,303.31000077)(276.68999796,310.00000119)(284.99999849,310.00000119)
\closepath
}
}
{
\newrgbcolor{curcolor}{0 0 0}
\pscustom[linestyle=none,fillstyle=solid,fillcolor=curcolor]
{
\newpath
\moveto(284.99999849,360.00000119)
\lineto(404.99999658,360.00000119)
\curveto(413.30999711,360.00000119)(419.99999754,353.31000077)(419.99999754,345.00000024)
\curveto(419.99999754,336.68999971)(413.30999711,329.99999929)(404.99999658,329.99999929)
\lineto(284.99999849,329.99999929)
\curveto(276.68999796,329.99999929)(269.99999754,336.68999971)(269.99999754,345.00000024)
\curveto(269.99999754,353.31000077)(276.68999796,360.00000119)(284.99999849,360.00000119)
\closepath
}
}
{
\newrgbcolor{curcolor}{0 0 0}
\pscustom[linewidth=2.28416109,linecolor=curcolor]
{
\newpath
\moveto(284.99999849,360.00000119)
\lineto(404.99999658,360.00000119)
\curveto(413.30999711,360.00000119)(419.99999754,353.31000077)(419.99999754,345.00000024)
\curveto(419.99999754,336.68999971)(413.30999711,329.99999929)(404.99999658,329.99999929)
\lineto(284.99999849,329.99999929)
\curveto(276.68999796,329.99999929)(269.99999754,336.68999971)(269.99999754,345.00000024)
\curveto(269.99999754,353.31000077)(276.68999796,360.00000119)(284.99999849,360.00000119)
\closepath
}
}
{
\newrgbcolor{curcolor}{0 0 0}
\pscustom[linestyle=none,fillstyle=solid,fillcolor=curcolor]
{
\newpath
\moveto(152.35546629,357.49609494)
\lineto(154.72265379,357.49609494)
\lineto(154.72265379,341.99218869)
\lineto(163.24218504,341.99218869)
\lineto(163.24218504,340.00000119)
\lineto(152.35546629,340.00000119)
\lineto(152.35546629,357.49609494)
}
}
{
\newrgbcolor{curcolor}{0 0 0}
\pscustom[linestyle=none,fillstyle=solid,fillcolor=curcolor]
{
\newpath
\moveto(171.58593504,346.59765744)
\curveto(169.84374105,346.59765085)(168.63671101,346.39843229)(167.96484129,346.00000119)
\curveto(167.29296235,345.60155809)(166.95702519,344.92187127)(166.95702879,343.96093869)
\curveto(166.95702519,343.1953105)(167.20702494,342.58593611)(167.70702879,342.13281369)
\curveto(168.21483643,341.68749951)(168.90233574,341.46484348)(169.76952879,341.46484494)
\curveto(170.96483368,341.46484348)(171.92186397,341.88671806)(172.64062254,342.73046994)
\curveto(173.36717503,343.58202886)(173.73045591,344.71093398)(173.73046629,346.11718869)
\lineto(173.73046629,346.59765744)
\lineto(171.58593504,346.59765744)
\moveto(175.88671629,347.48828244)
\lineto(175.88671629,340.00000119)
\lineto(173.73046629,340.00000119)
\lineto(173.73046629,341.99218869)
\curveto(173.23826891,341.1953125)(172.62498827,340.60546934)(171.89062254,340.22265744)
\curveto(171.15623974,339.8476576)(170.25780314,339.66015778)(169.19531004,339.66015744)
\curveto(167.85155554,339.66015778)(166.78124411,340.03515741)(165.98437254,340.78515744)
\curveto(165.1953082,341.5429684)(164.80077734,342.55468614)(164.80077879,343.82031369)
\curveto(164.80077734,345.2968709)(165.29296435,346.41015103)(166.27734129,347.16015744)
\curveto(167.26952488,347.91014953)(168.7460859,348.28514916)(170.70702879,348.28515744)
\lineto(173.73046629,348.28515744)
\lineto(173.73046629,348.49609494)
\curveto(173.73045591,349.48827296)(173.40233124,350.25389719)(172.74609129,350.79296994)
\curveto(172.09764505,351.3398336)(171.18358346,351.61327083)(170.00390379,351.61328244)
\curveto(169.25389789,351.61327083)(168.52342987,351.52342717)(167.81249754,351.34375119)
\curveto(167.10155629,351.16405253)(166.41796323,350.89452155)(165.76171629,350.53515744)
\lineto(165.76171629,352.52734494)
\curveto(166.55077559,352.83201961)(167.31639983,353.05858188)(168.05859129,353.20703244)
\curveto(168.80077334,353.36326908)(169.52342887,353.441394)(170.22656004,353.44140744)
\curveto(172.12498877,353.441394)(173.5429561,352.94920699)(174.48046629,351.96484494)
\curveto(175.41795423,350.98045896)(175.88670376,349.48827296)(175.88671629,347.48828244)
}
}
{
\newrgbcolor{curcolor}{0 0 0}
\pscustom[linestyle=none,fillstyle=solid,fillcolor=curcolor]
{
\newpath
\moveto(187.94531004,351.10937619)
\curveto(187.70311291,351.24998994)(187.43748818,351.35155234)(187.14843504,351.41406369)
\curveto(186.86717625,351.48436471)(186.55467656,351.51952092)(186.21093504,351.51953244)
\curveto(184.99217812,351.51952092)(184.05467906,351.12108382)(183.39843504,350.32421994)
\curveto(182.74999286,349.53514791)(182.42577444,348.39843029)(182.42577879,346.91406369)
\lineto(182.42577879,340.00000119)
\lineto(180.25781004,340.00000119)
\lineto(180.25781004,353.12500119)
\lineto(182.42577879,353.12500119)
\lineto(182.42577879,351.08593869)
\curveto(182.87889898,351.88280181)(183.46874214,352.47264497)(184.19531004,352.85546994)
\curveto(184.92186569,353.2460817)(185.80467731,353.441394)(186.84374754,353.44140744)
\curveto(186.99217612,353.441394)(187.15623846,353.42967526)(187.33593504,353.40625119)
\curveto(187.5156131,353.3906128)(187.71483165,353.36326908)(187.93359129,353.32421994)
\lineto(187.94531004,351.10937619)
}
}
{
\newrgbcolor{curcolor}{0 0 0}
\pscustom[linestyle=none,fillstyle=solid,fillcolor=curcolor]
{
\newpath
\moveto(198.44531004,346.71484494)
\curveto(198.44529914,348.27733667)(198.12108071,349.48827296)(197.47265379,350.34765744)
\curveto(196.8320195,351.20702124)(195.92967665,351.63670831)(194.76562254,351.63671994)
\curveto(193.60936647,351.63670831)(192.70702363,351.20702124)(192.05859129,350.34765744)
\curveto(191.41796241,349.48827296)(191.09765023,348.27733667)(191.09765379,346.71484494)
\curveto(191.09765023,345.16015228)(191.41796241,343.95312224)(192.05859129,343.09375119)
\curveto(192.70702363,342.23437396)(193.60936647,341.80468689)(194.76562254,341.80468869)
\curveto(195.92967665,341.80468689)(196.8320195,342.23437396)(197.47265379,343.09375119)
\curveto(198.12108071,343.95312224)(198.44529914,345.16015228)(198.44531004,346.71484494)
\moveto(200.60156004,341.62890744)
\curveto(200.60154698,339.39453305)(200.10545373,337.73437846)(199.11327879,336.64843869)
\curveto(198.12108071,335.55469314)(196.60155098,335.00781869)(194.55468504,335.00781369)
\curveto(193.79686629,335.00781869)(193.08202325,335.06641238)(192.41015379,335.18359494)
\curveto(191.73827459,335.29297465)(191.0859315,335.46484948)(190.45312254,335.69921994)
\lineto(190.45312254,337.79687619)
\curveto(191.0859315,337.45312874)(191.71093087,337.19922274)(192.32812254,337.03515744)
\curveto(192.94530464,336.87109807)(193.57421026,336.7890669)(194.21484129,336.78906369)
\curveto(195.6288957,336.7890669)(196.68748839,337.16016028)(197.39062254,337.90234494)
\curveto(198.09373699,338.63672131)(198.44529914,339.75000144)(198.44531004,341.24218869)
\lineto(198.44531004,342.30859494)
\curveto(197.99998708,341.53515591)(197.42967515,340.95703149)(196.73437254,340.57421994)
\curveto(196.03905154,340.19140725)(195.20702112,340.00000119)(194.23827879,340.00000119)
\curveto(192.6288987,340.00000119)(191.332025,340.61328183)(190.34765379,341.83984494)
\curveto(189.36327697,343.06640438)(188.87108996,344.69140275)(188.87109129,346.71484494)
\curveto(188.87108996,348.7460862)(189.36327697,350.37499082)(190.34765379,351.60156369)
\curveto(191.332025,352.82811337)(192.6288987,353.441394)(194.23827879,353.44140744)
\curveto(195.20702112,353.441394)(196.03905154,353.24998794)(196.73437254,352.86718869)
\curveto(197.42967515,352.48436371)(197.99998708,351.90623929)(198.44531004,351.13281369)
\lineto(198.44531004,353.12500119)
\lineto(200.60156004,353.12500119)
\lineto(200.60156004,341.62890744)
}
}
{
\newrgbcolor{curcolor}{0 0 0}
\pscustom[linestyle=none,fillstyle=solid,fillcolor=curcolor]
{
\newpath
\moveto(216.26952879,347.10156369)
\lineto(216.26952879,346.04687619)
\lineto(206.35546629,346.04687619)
\curveto(206.44921262,344.56249663)(206.89452467,343.42968526)(207.69140379,342.64843869)
\curveto(208.49608557,341.87499932)(209.61327195,341.48828096)(211.04296629,341.48828244)
\curveto(211.8710822,341.48828096)(212.67186264,341.58984335)(213.44531004,341.79296994)
\curveto(214.22654859,341.99609295)(214.99998532,342.30078014)(215.76562254,342.70703244)
\lineto(215.76562254,340.66796994)
\curveto(214.99217282,340.3398446)(214.19920487,340.08984485)(213.38671629,339.91796994)
\curveto(212.57420649,339.7460952)(211.74998857,339.66015778)(210.91406004,339.66015744)
\curveto(208.820304,339.66015778)(207.16014941,340.26953217)(205.93359129,341.48828244)
\curveto(204.71483935,342.70702974)(204.10546496,344.35546559)(204.10546629,346.43359494)
\curveto(204.10546496,348.58202386)(204.68358938,350.28514716)(205.83984129,351.54296994)
\curveto(207.00389956,352.80858213)(208.57030425,353.441394)(210.53906004,353.44140744)
\curveto(212.30467551,353.441394)(213.69920537,352.87108207)(214.72265379,351.73046994)
\curveto(215.75389081,350.59764685)(216.2695153,349.05467964)(216.26952879,347.10156369)
\moveto(214.11327879,347.73437619)
\curveto(214.09764247,348.91405478)(213.76561155,349.85546009)(213.11718504,350.55859494)
\curveto(212.47655034,351.26170868)(211.62498869,351.61327083)(210.56249754,351.61328244)
\curveto(209.35936596,351.61327083)(208.39452317,351.27342742)(207.66796629,350.59375119)
\curveto(206.94921212,349.91405378)(206.53515003,348.95702349)(206.42577879,347.72265744)
\lineto(214.11327879,347.73437619)
}
}
{
\newrgbcolor{curcolor}{0 0 0}
\pscustom[linestyle=none,fillstyle=solid,fillcolor=curcolor]
{
\newpath
\moveto(219.58593504,345.17968869)
\lineto(219.58593504,353.12500119)
\lineto(221.74218504,353.12500119)
\lineto(221.74218504,345.26171994)
\curveto(221.74218084,344.01952842)(221.9843681,343.08593561)(222.46874754,342.46093869)
\curveto(222.95311713,341.84374935)(223.6796789,341.53515591)(224.64843504,341.53515744)
\curveto(225.81248927,341.53515591)(226.7304571,341.90624929)(227.40234129,342.64843869)
\curveto(228.08201825,343.3906228)(228.42186166,344.40234054)(228.42187254,345.68359494)
\lineto(228.42187254,353.12500119)
\lineto(230.57812254,353.12500119)
\lineto(230.57812254,340.00000119)
\lineto(228.42187254,340.00000119)
\lineto(228.42187254,342.01562619)
\curveto(227.89842468,341.21874997)(227.28905029,340.62500057)(226.59374754,340.23437619)
\curveto(225.90623918,339.85156384)(225.10545873,339.66015778)(224.19140379,339.66015744)
\curveto(222.68358615,339.66015778)(221.53905604,340.12890731)(220.75781004,341.06640744)
\curveto(219.97655761,342.00390544)(219.585933,343.37499782)(219.58593504,345.17968869)
\moveto(225.01171629,353.44140744)
\lineto(225.01171629,353.44140744)
}
}
{
\newrgbcolor{curcolor}{0 0 0}
\pscustom[linestyle=none,fillstyle=solid,fillcolor=curcolor]
{
\newpath
\moveto(242.64843504,351.10937619)
\curveto(242.40623791,351.24998994)(242.14061318,351.35155234)(241.85156004,351.41406369)
\curveto(241.57030125,351.48436471)(241.25780156,351.51952092)(240.91406004,351.51953244)
\curveto(239.69530312,351.51952092)(238.75780406,351.12108382)(238.10156004,350.32421994)
\curveto(237.45311786,349.53514791)(237.12889944,348.39843029)(237.12890379,346.91406369)
\lineto(237.12890379,340.00000119)
\lineto(234.96093504,340.00000119)
\lineto(234.96093504,353.12500119)
\lineto(237.12890379,353.12500119)
\lineto(237.12890379,351.08593869)
\curveto(237.58202398,351.88280181)(238.17186714,352.47264497)(238.89843504,352.85546994)
\curveto(239.62499069,353.2460817)(240.50780231,353.441394)(241.54687254,353.44140744)
\curveto(241.69530112,353.441394)(241.85936346,353.42967526)(242.03906004,353.40625119)
\curveto(242.2187381,353.3906128)(242.41795665,353.36326908)(242.63671629,353.32421994)
\lineto(242.64843504,351.10937619)
}
}
{
\newrgbcolor{curcolor}{0 0 0}
\pscustom[linestyle=none,fillstyle=solid,fillcolor=curcolor]
{
\newpath
\moveto(152.35546629,307.49609494)
\lineto(154.72265379,307.49609494)
\lineto(154.72265379,300.32421994)
\lineto(163.32421629,300.32421994)
\lineto(163.32421629,307.49609494)
\lineto(165.69140379,307.49609494)
\lineto(165.69140379,290.00000119)
\lineto(163.32421629,290.00000119)
\lineto(163.32421629,298.33203244)
\lineto(154.72265379,298.33203244)
\lineto(154.72265379,290.00000119)
\lineto(152.35546629,290.00000119)
\lineto(152.35546629,307.49609494)
}
}
{
\newrgbcolor{curcolor}{0 0 0}
\pscustom[linestyle=none,fillstyle=solid,fillcolor=curcolor]
{
\newpath
\moveto(176.27343504,296.59765744)
\curveto(174.53124105,296.59765085)(173.32421101,296.39843229)(172.65234129,296.00000119)
\curveto(171.98046235,295.60155809)(171.64452519,294.92187127)(171.64452879,293.96093869)
\curveto(171.64452519,293.1953105)(171.89452494,292.58593611)(172.39452879,292.13281369)
\curveto(172.90233643,291.68749951)(173.58983574,291.46484348)(174.45702879,291.46484494)
\curveto(175.65233368,291.46484348)(176.60936397,291.88671806)(177.32812254,292.73046994)
\curveto(178.05467503,293.58202886)(178.41795591,294.71093398)(178.41796629,296.11718869)
\lineto(178.41796629,296.59765744)
\lineto(176.27343504,296.59765744)
\moveto(180.57421629,297.48828244)
\lineto(180.57421629,290.00000119)
\lineto(178.41796629,290.00000119)
\lineto(178.41796629,291.99218869)
\curveto(177.92576891,291.1953125)(177.31248827,290.60546934)(176.57812254,290.22265744)
\curveto(175.84373974,289.8476576)(174.94530314,289.66015778)(173.88281004,289.66015744)
\curveto(172.53905554,289.66015778)(171.46874411,290.03515741)(170.67187254,290.78515744)
\curveto(169.8828082,291.5429684)(169.48827734,292.55468614)(169.48827879,293.82031369)
\curveto(169.48827734,295.2968709)(169.98046435,296.41015103)(170.96484129,297.16015744)
\curveto(171.95702488,297.91014953)(173.4335859,298.28514916)(175.39452879,298.28515744)
\lineto(178.41796629,298.28515744)
\lineto(178.41796629,298.49609494)
\curveto(178.41795591,299.48827296)(178.08983124,300.25389719)(177.43359129,300.79296994)
\curveto(176.78514505,301.3398336)(175.87108346,301.61327083)(174.69140379,301.61328244)
\curveto(173.94139789,301.61327083)(173.21092987,301.52342717)(172.49999754,301.34375119)
\curveto(171.78905629,301.16405253)(171.10546323,300.89452155)(170.44921629,300.53515744)
\lineto(170.44921629,302.52734494)
\curveto(171.23827559,302.83201961)(172.00389983,303.05858188)(172.74609129,303.20703244)
\curveto(173.48827334,303.36326908)(174.21092887,303.441394)(174.91406004,303.44140744)
\curveto(176.81248877,303.441394)(178.2304561,302.94920699)(179.16796629,301.96484494)
\curveto(180.10545423,300.98045896)(180.57420376,299.48827296)(180.57421629,297.48828244)
}
}
{
\newrgbcolor{curcolor}{0 0 0}
\pscustom[linestyle=none,fillstyle=solid,fillcolor=curcolor]
{
\newpath
\moveto(184.80468504,295.17968869)
\lineto(184.80468504,303.12500119)
\lineto(186.96093504,303.12500119)
\lineto(186.96093504,295.26171994)
\curveto(186.96093084,294.01952842)(187.2031181,293.08593561)(187.68749754,292.46093869)
\curveto(188.17186713,291.84374935)(188.8984289,291.53515591)(189.86718504,291.53515744)
\curveto(191.03123927,291.53515591)(191.9492071,291.90624929)(192.62109129,292.64843869)
\curveto(193.30076825,293.3906228)(193.64061166,294.40234054)(193.64062254,295.68359494)
\lineto(193.64062254,303.12500119)
\lineto(195.79687254,303.12500119)
\lineto(195.79687254,290.00000119)
\lineto(193.64062254,290.00000119)
\lineto(193.64062254,292.01562619)
\curveto(193.11717468,291.21874997)(192.50780029,290.62500057)(191.81249754,290.23437619)
\curveto(191.12498918,289.85156384)(190.32420873,289.66015778)(189.41015379,289.66015744)
\curveto(187.90233615,289.66015778)(186.75780604,290.12890731)(185.97656004,291.06640744)
\curveto(185.19530761,292.00390544)(184.804683,293.37499782)(184.80468504,295.17968869)
\moveto(190.23046629,303.44140744)
\lineto(190.23046629,303.44140744)
}
}
{
\newrgbcolor{curcolor}{0 0 0}
\pscustom[linestyle=none,fillstyle=solid,fillcolor=curcolor]
{
\newpath
\moveto(202.39452879,306.85156369)
\lineto(202.39452879,303.12500119)
\lineto(206.83593504,303.12500119)
\lineto(206.83593504,301.44921994)
\lineto(202.39452879,301.44921994)
\lineto(202.39452879,294.32421994)
\curveto(202.39452439,293.25390419)(202.5390555,292.56640488)(202.82812254,292.26171994)
\curveto(203.12499241,291.95703049)(203.72264806,291.80468689)(204.62109129,291.80468869)
\lineto(206.83593504,291.80468869)
\lineto(206.83593504,290.00000119)
\lineto(204.62109129,290.00000119)
\curveto(202.95702383,290.00000119)(201.80858748,290.30859463)(201.17577879,290.92578244)
\curveto(200.54296374,291.55078089)(200.22655781,292.68359226)(200.22656004,294.32421994)
\lineto(200.22656004,301.44921994)
\lineto(198.64452879,301.44921994)
\lineto(198.64452879,303.12500119)
\lineto(200.22656004,303.12500119)
\lineto(200.22656004,306.85156369)
\lineto(202.39452879,306.85156369)
}
}
{
\newrgbcolor{curcolor}{0 0 0}
\pscustom[linestyle=none,fillstyle=solid,fillcolor=curcolor]
{
\newpath
\moveto(220.91015379,297.10156369)
\lineto(220.91015379,296.04687619)
\lineto(210.99609129,296.04687619)
\curveto(211.08983762,294.56249663)(211.53514967,293.42968526)(212.33202879,292.64843869)
\curveto(213.13671057,291.87499932)(214.25389695,291.48828096)(215.68359129,291.48828244)
\curveto(216.5117072,291.48828096)(217.31248764,291.58984335)(218.08593504,291.79296994)
\curveto(218.86717359,291.99609295)(219.64061032,292.30078014)(220.40624754,292.70703244)
\lineto(220.40624754,290.66796994)
\curveto(219.63279782,290.3398446)(218.83982987,290.08984485)(218.02734129,289.91796994)
\curveto(217.21483149,289.7460952)(216.39061357,289.66015778)(215.55468504,289.66015744)
\curveto(213.460929,289.66015778)(211.80077441,290.26953217)(210.57421629,291.48828244)
\curveto(209.35546435,292.70702974)(208.74608996,294.35546559)(208.74609129,296.43359494)
\curveto(208.74608996,298.58202386)(209.32421438,300.28514716)(210.48046629,301.54296994)
\curveto(211.64452456,302.80858213)(213.21092925,303.441394)(215.17968504,303.44140744)
\curveto(216.94530051,303.441394)(218.33983037,302.87108207)(219.36327879,301.73046994)
\curveto(220.39451581,300.59764685)(220.9101403,299.05467964)(220.91015379,297.10156369)
\moveto(218.75390379,297.73437619)
\curveto(218.73826747,298.91405478)(218.40623655,299.85546009)(217.75781004,300.55859494)
\curveto(217.11717534,301.26170868)(216.26561369,301.61327083)(215.20312254,301.61328244)
\curveto(213.99999096,301.61327083)(213.03514817,301.27342742)(212.30859129,300.59375119)
\curveto(211.58983712,299.91405378)(211.17577503,298.95702349)(211.06640379,297.72265744)
\lineto(218.75390379,297.73437619)
}
}
{
\newrgbcolor{curcolor}{0 0 0}
\pscustom[linestyle=none,fillstyle=solid,fillcolor=curcolor]
{
\newpath
\moveto(224.22656004,295.17968869)
\lineto(224.22656004,303.12500119)
\lineto(226.38281004,303.12500119)
\lineto(226.38281004,295.26171994)
\curveto(226.38280584,294.01952842)(226.6249931,293.08593561)(227.10937254,292.46093869)
\curveto(227.59374213,291.84374935)(228.3203039,291.53515591)(229.28906004,291.53515744)
\curveto(230.45311427,291.53515591)(231.3710821,291.90624929)(232.04296629,292.64843869)
\curveto(232.72264325,293.3906228)(233.06248666,294.40234054)(233.06249754,295.68359494)
\lineto(233.06249754,303.12500119)
\lineto(235.21874754,303.12500119)
\lineto(235.21874754,290.00000119)
\lineto(233.06249754,290.00000119)
\lineto(233.06249754,292.01562619)
\curveto(232.53904968,291.21874997)(231.92967529,290.62500057)(231.23437254,290.23437619)
\curveto(230.54686418,289.85156384)(229.74608373,289.66015778)(228.83202879,289.66015744)
\curveto(227.32421115,289.66015778)(226.17968104,290.12890731)(225.39843504,291.06640744)
\curveto(224.61718261,292.00390544)(224.226558,293.37499782)(224.22656004,295.17968869)
\moveto(229.65234129,303.44140744)
\lineto(229.65234129,303.44140744)
}
}
{
\newrgbcolor{curcolor}{0 0 0}
\pscustom[linestyle=none,fillstyle=solid,fillcolor=curcolor]
{
\newpath
\moveto(247.28906004,301.10937619)
\curveto(247.04686291,301.24998994)(246.78123818,301.35155234)(246.49218504,301.41406369)
\curveto(246.21092625,301.48436471)(245.89842656,301.51952092)(245.55468504,301.51953244)
\curveto(244.33592812,301.51952092)(243.39842906,301.12108382)(242.74218504,300.32421994)
\curveto(242.09374286,299.53514791)(241.76952444,298.39843029)(241.76952879,296.91406369)
\lineto(241.76952879,290.00000119)
\lineto(239.60156004,290.00000119)
\lineto(239.60156004,303.12500119)
\lineto(241.76952879,303.12500119)
\lineto(241.76952879,301.08593869)
\curveto(242.22264898,301.88280181)(242.81249214,302.47264497)(243.53906004,302.85546994)
\curveto(244.26561569,303.2460817)(245.14842731,303.441394)(246.18749754,303.44140744)
\curveto(246.33592612,303.441394)(246.49998846,303.42967526)(246.67968504,303.40625119)
\curveto(246.8593631,303.3906128)(247.05858165,303.36326908)(247.27734129,303.32421994)
\lineto(247.28906004,301.10937619)
}
}
{
\newrgbcolor{curcolor}{1 1 1}
\pscustom[linestyle=none,fillstyle=solid,fillcolor=curcolor]
{
\newpath
\moveto(145.00000516,49.99997068)
\lineto(235.00000516,49.99997068)
\curveto(243.31000516,49.99997068)(250.00000516,43.30997068)(250.00000516,34.99997068)
\curveto(250.00000516,26.68997068)(243.31000516,19.99997068)(235.00000516,19.99997068)
\lineto(145.00000516,19.99997068)
\curveto(136.69000516,19.99997068)(130.00000516,26.68997068)(130.00000516,34.99997068)
\curveto(130.00000516,43.30997068)(136.69000516,49.99997068)(145.00000516,49.99997068)
\closepath
}
}
{
\newrgbcolor{curcolor}{0 0 0}
\pscustom[linewidth=1.85164022,linecolor=curcolor]
{
\newpath
\moveto(145.00000516,49.99997068)
\lineto(235.00000516,49.99997068)
\curveto(243.31000516,49.99997068)(250.00000516,43.30997068)(250.00000516,34.99997068)
\curveto(250.00000516,26.68997068)(243.31000516,19.99997068)(235.00000516,19.99997068)
\lineto(145.00000516,19.99997068)
\curveto(136.69000516,19.99997068)(130.00000516,26.68997068)(130.00000516,34.99997068)
\curveto(130.00000516,43.30997068)(136.69000516,49.99997068)(145.00000516,49.99997068)
\closepath
}
}
{
\newrgbcolor{curcolor}{0 0 0}
\pscustom[linestyle=none,fillstyle=solid,fillcolor=curcolor]
{
\newpath
\moveto(160.65234129,38.20309568)
\curveto(161.16014262,38.03121264)(161.65232963,37.66402551)(162.12890379,37.10153318)
\curveto(162.61326617,36.53902664)(163.09764069,35.76558991)(163.58202879,34.78122068)
\lineto(165.98437254,29.99997068)
\lineto(163.44140379,29.99997068)
\lineto(161.20312254,34.48825193)
\curveto(160.62498691,35.66012127)(160.06248747,36.43746424)(159.51562254,36.82028318)
\curveto(158.97655106,37.20308847)(158.23827055,37.39449453)(157.30077879,37.39450193)
\lineto(154.72265379,37.39450193)
\lineto(154.72265379,29.99997068)
\lineto(152.35546629,29.99997068)
\lineto(152.35546629,47.49606443)
\lineto(157.69921629,47.49606443)
\curveto(159.69920659,47.49604693)(161.19139259,47.0780786)(162.17577879,46.24215818)
\curveto(163.16014062,45.40620527)(163.65232763,44.14448778)(163.65234129,42.45700193)
\curveto(163.65232763,41.35542807)(163.39451539,40.44136648)(162.87890379,39.71481443)
\curveto(162.37107891,38.98824294)(161.62889216,38.48433719)(160.65234129,38.20309568)
\moveto(154.72265379,45.55075193)
\lineto(154.72265379,39.33981443)
\lineto(157.69921629,39.33981443)
\curveto(158.83983245,39.33980509)(159.69920659,39.60152357)(160.27734129,40.12497068)
\curveto(160.86326792,40.65621002)(161.15623638,41.43355299)(161.15624754,42.45700193)
\curveto(161.15623638,43.48042595)(160.86326792,44.24995643)(160.27734129,44.76559568)
\curveto(159.69920659,45.28901789)(158.83983245,45.55073638)(157.69921629,45.55075193)
\lineto(154.72265379,45.55075193)
}
}
{
\newrgbcolor{curcolor}{0 0 0}
\pscustom[linestyle=none,fillstyle=solid,fillcolor=curcolor]
{
\newpath
\moveto(179.09765379,37.10153318)
\lineto(179.09765379,36.04684568)
\lineto(169.18359129,36.04684568)
\curveto(169.27733762,34.56246611)(169.72264967,33.42965475)(170.51952879,32.64840818)
\curveto(171.32421057,31.8749688)(172.44139695,31.48825044)(173.87109129,31.48825193)
\curveto(174.6992072,31.48825044)(175.49998764,31.58981284)(176.27343504,31.79293943)
\curveto(177.05467359,31.99606243)(177.82811032,32.30074962)(178.59374754,32.70700193)
\lineto(178.59374754,30.66793943)
\curveto(177.82029782,30.33981409)(177.02732987,30.08981434)(176.21484129,29.91793943)
\curveto(175.40233149,29.74606468)(174.57811357,29.66012727)(173.74218504,29.66012693)
\curveto(171.648429,29.66012727)(169.98827441,30.26950166)(168.76171629,31.48825193)
\curveto(167.54296435,32.70699922)(166.93358996,34.35543507)(166.93359129,36.43356443)
\curveto(166.93358996,38.58199334)(167.51171438,40.28511664)(168.66796629,41.54293943)
\curveto(169.83202456,42.80855162)(171.39842925,43.44136348)(173.36718504,43.44137693)
\curveto(175.13280051,43.44136348)(176.52733037,42.87105155)(177.55077879,41.73043943)
\curveto(178.58201581,40.59761633)(179.0976403,39.05464912)(179.09765379,37.10153318)
\moveto(176.94140379,37.73434568)
\curveto(176.92576747,38.91402426)(176.59373655,39.85542957)(175.94531004,40.55856443)
\curveto(175.30467534,41.26167816)(174.45311369,41.61324031)(173.39062254,41.61325193)
\curveto(172.18749096,41.61324031)(171.22264817,41.2733969)(170.49609129,40.59372068)
\curveto(169.77733712,39.91402326)(169.36327503,38.95699297)(169.25390379,37.72262693)
\lineto(176.94140379,37.73434568)
}
}
{
\newrgbcolor{curcolor}{0 0 0}
\pscustom[linestyle=none,fillstyle=solid,fillcolor=curcolor]
{
\newpath
\moveto(184.76952879,46.85153318)
\lineto(184.76952879,43.12497068)
\lineto(189.21093504,43.12497068)
\lineto(189.21093504,41.44918943)
\lineto(184.76952879,41.44918943)
\lineto(184.76952879,34.32418943)
\curveto(184.76952439,33.25387367)(184.9140555,32.56637436)(185.20312254,32.26168943)
\curveto(185.49999241,31.95699997)(186.09764806,31.80465637)(186.99609129,31.80465818)
\lineto(189.21093504,31.80465818)
\lineto(189.21093504,29.99997068)
\lineto(186.99609129,29.99997068)
\curveto(185.33202383,29.99997068)(184.18358748,30.30856412)(183.55077879,30.92575193)
\curveto(182.91796374,31.55075038)(182.60155781,32.68356174)(182.60156004,34.32418943)
\lineto(182.60156004,41.44918943)
\lineto(181.01952879,41.44918943)
\lineto(181.01952879,43.12497068)
\lineto(182.60156004,43.12497068)
\lineto(182.60156004,46.85153318)
\lineto(184.76952879,46.85153318)
}
}
{
\newrgbcolor{curcolor}{0 0 0}
\pscustom[linestyle=none,fillstyle=solid,fillcolor=curcolor]
{
\newpath
\moveto(197.14452879,41.61325193)
\curveto(195.98827259,41.61324031)(195.07421101,41.16011577)(194.40234129,40.25387693)
\curveto(193.73046235,39.35543007)(193.39452519,38.1210563)(193.39452879,36.55075193)
\curveto(193.39452519,34.98043445)(193.72655611,33.74215443)(194.39062254,32.83590818)
\curveto(195.06249227,31.93746874)(195.9804601,31.48825044)(197.14452879,31.48825193)
\curveto(198.29295779,31.48825044)(199.20311313,31.94137498)(199.87499754,32.84762693)
\curveto(200.54686179,33.75387317)(200.88279895,34.98824694)(200.88281004,36.55075193)
\curveto(200.88279895,38.10543132)(200.54686179,39.33589884)(199.87499754,40.24215818)
\curveto(199.20311313,41.15620952)(198.29295779,41.61324031)(197.14452879,41.61325193)
\moveto(197.14452879,43.44137693)
\curveto(199.01951956,43.44136348)(200.49217434,42.83198909)(201.56249754,41.61325193)
\curveto(202.6327972,40.39449153)(203.16795291,38.70699322)(203.16796629,36.55075193)
\curveto(203.16795291,34.40231002)(202.6327972,32.71481171)(201.56249754,31.48825193)
\curveto(200.49217434,30.26950166)(199.01951956,29.66012727)(197.14452879,29.66012693)
\curveto(195.26171082,29.66012727)(193.7851498,30.26950166)(192.71484129,31.48825193)
\curveto(191.65233943,32.71481171)(191.12108996,34.40231002)(191.12109129,36.55075193)
\curveto(191.12108996,38.70699322)(191.65233943,40.39449153)(192.71484129,41.61325193)
\curveto(193.7851498,42.83198909)(195.26171082,43.44136348)(197.14452879,43.44137693)
}
}
{
\newrgbcolor{curcolor}{0 0 0}
\pscustom[linestyle=none,fillstyle=solid,fillcolor=curcolor]
{
\newpath
\moveto(206.50781004,35.17965818)
\lineto(206.50781004,43.12497068)
\lineto(208.66406004,43.12497068)
\lineto(208.66406004,35.26168943)
\curveto(208.66405584,34.01949791)(208.9062431,33.08590509)(209.39062254,32.46090818)
\curveto(209.87499213,31.84371883)(210.6015539,31.53512539)(211.57031004,31.53512693)
\curveto(212.73436427,31.53512539)(213.6523321,31.90621877)(214.32421629,32.64840818)
\curveto(215.00389325,33.39059229)(215.34373666,34.40231002)(215.34374754,35.68356443)
\lineto(215.34374754,43.12497068)
\lineto(217.49999754,43.12497068)
\lineto(217.49999754,29.99997068)
\lineto(215.34374754,29.99997068)
\lineto(215.34374754,32.01559568)
\curveto(214.82029968,31.21871946)(214.21092529,30.62497005)(213.51562254,30.23434568)
\curveto(212.82811418,29.85153332)(212.02733373,29.66012727)(211.11327879,29.66012693)
\curveto(209.60546115,29.66012727)(208.46093104,30.1288768)(207.67968504,31.06637693)
\curveto(206.89843261,32.00387492)(206.507808,33.3749673)(206.50781004,35.17965818)
\moveto(211.93359129,43.44137693)
\lineto(211.93359129,43.44137693)
}
}
{
\newrgbcolor{curcolor}{0 0 0}
\pscustom[linestyle=none,fillstyle=solid,fillcolor=curcolor]
{
\newpath
\moveto(229.57031004,41.10934568)
\curveto(229.32811291,41.24995943)(229.06248818,41.35152182)(228.77343504,41.41403318)
\curveto(228.49217625,41.48433419)(228.17967656,41.51949041)(227.83593504,41.51950193)
\curveto(226.61717812,41.51949041)(225.67967906,41.1210533)(225.02343504,40.32418943)
\curveto(224.37499286,39.53511739)(224.05077444,38.39839978)(224.05077879,36.91403318)
\lineto(224.05077879,29.99997068)
\lineto(221.88281004,29.99997068)
\lineto(221.88281004,43.12497068)
\lineto(224.05077879,43.12497068)
\lineto(224.05077879,41.08590818)
\curveto(224.50389898,41.88277129)(225.09374214,42.47261445)(225.82031004,42.85543943)
\curveto(226.54686569,43.24605118)(227.42967731,43.44136348)(228.46874754,43.44137693)
\curveto(228.61717612,43.44136348)(228.78123846,43.42964475)(228.96093504,43.40622068)
\curveto(229.1406131,43.39058229)(229.33983165,43.36323856)(229.55859129,43.32418943)
\lineto(229.57031004,41.10934568)
}
}
{
\newrgbcolor{curcolor}{1 1 1}
\pscustom[linestyle=none,fillstyle=solid,fillcolor=curcolor]
{
\newpath
\moveto(324.99999754,49.99997068)
\lineto(414.99999754,49.99997068)
\curveto(423.30999754,49.99997068)(429.99999754,43.30997068)(429.99999754,34.99997068)
\curveto(429.99999754,26.68997068)(423.30999754,19.99997068)(414.99999754,19.99997068)
\lineto(324.99999754,19.99997068)
\curveto(316.68999754,19.99997068)(309.99999754,26.68997068)(309.99999754,34.99997068)
\curveto(309.99999754,43.30997068)(316.68999754,49.99997068)(324.99999754,49.99997068)
\closepath
}
}
{
\newrgbcolor{curcolor}{0 0 0}
\pscustom[linewidth=2,linecolor=curcolor]
{
\newpath
\moveto(324.99999754,49.99997068)
\lineto(414.99999754,49.99997068)
\curveto(423.30999754,49.99997068)(429.99999754,43.30997068)(429.99999754,34.99997068)
\curveto(429.99999754,26.68997068)(423.30999754,19.99997068)(414.99999754,19.99997068)
\lineto(324.99999754,19.99997068)
\curveto(316.68999754,19.99997068)(309.99999754,26.68997068)(309.99999754,34.99997068)
\curveto(309.99999754,43.30997068)(316.68999754,49.99997068)(324.99999754,49.99997068)
\closepath
}
}
{
\newrgbcolor{curcolor}{0 0 0}
\pscustom[linestyle=none,fillstyle=solid,fillcolor=curcolor]
{
\newpath
\moveto(336.86718504,29.99997068)
\lineto(330.18749754,47.49606443)
\lineto(332.66015379,47.49606443)
\lineto(338.20312254,32.76559568)
\lineto(343.75781004,47.49606443)
\lineto(346.21874754,47.49606443)
\lineto(339.55077879,29.99997068)
\lineto(336.86718504,29.99997068)
}
}
{
\newrgbcolor{curcolor}{0 0 0}
\pscustom[linestyle=none,fillstyle=solid,fillcolor=curcolor]
{
\newpath
\moveto(352.75781004,36.59762693)
\curveto(351.01561605,36.59762033)(349.80858601,36.39840178)(349.13671629,35.99997068)
\curveto(348.46483735,35.60152757)(348.12890019,34.92184075)(348.12890379,33.96090818)
\curveto(348.12890019,33.19527998)(348.37889994,32.58590559)(348.87890379,32.13278318)
\curveto(349.38671143,31.68746899)(350.07421074,31.46481296)(350.94140379,31.46481443)
\curveto(352.13670868,31.46481296)(353.09373897,31.88668754)(353.81249754,32.73043943)
\curveto(354.53905003,33.58199834)(354.90233091,34.71090346)(354.90234129,36.11715818)
\lineto(354.90234129,36.59762693)
\lineto(352.75781004,36.59762693)
\moveto(357.05859129,37.48825193)
\lineto(357.05859129,29.99997068)
\lineto(354.90234129,29.99997068)
\lineto(354.90234129,31.99215818)
\curveto(354.41014391,31.19528198)(353.79686327,30.60543882)(353.06249754,30.22262693)
\curveto(352.32811474,29.84762708)(351.42967814,29.66012727)(350.36718504,29.66012693)
\curveto(349.02343054,29.66012727)(347.95311911,30.03512689)(347.15624754,30.78512693)
\curveto(346.3671832,31.54293788)(345.97265234,32.55465562)(345.97265379,33.82028318)
\curveto(345.97265234,35.29684038)(346.46483935,36.41012052)(347.44921629,37.16012693)
\curveto(348.44139988,37.91011902)(349.9179609,38.28511864)(351.87890379,38.28512693)
\lineto(354.90234129,38.28512693)
\lineto(354.90234129,38.49606443)
\curveto(354.90233091,39.48824244)(354.57420624,40.25386667)(353.91796629,40.79293943)
\curveto(353.26952005,41.33980309)(352.35545846,41.61324031)(351.17577879,41.61325193)
\curveto(350.42577289,41.61324031)(349.69530487,41.52339665)(348.98437254,41.34372068)
\curveto(348.27343129,41.16402201)(347.58983823,40.89449103)(346.93359129,40.53512693)
\lineto(346.93359129,42.52731443)
\curveto(347.72265059,42.83198909)(348.48827483,43.05855137)(349.23046629,43.20700193)
\curveto(349.97264834,43.36323856)(350.69530387,43.44136348)(351.39843504,43.44137693)
\curveto(353.29686377,43.44136348)(354.7148311,42.94917648)(355.65234129,41.96481443)
\curveto(356.58982923,40.98042845)(357.05857876,39.48824244)(357.05859129,37.48825193)
}
}
{
\newrgbcolor{curcolor}{0 0 0}
\pscustom[linestyle=none,fillstyle=solid,fillcolor=curcolor]
{
\newpath
\moveto(361.51171629,48.23434568)
\lineto(363.66796629,48.23434568)
\lineto(363.66796629,29.99997068)
\lineto(361.51171629,29.99997068)
\lineto(361.51171629,48.23434568)
}
}
{
\newrgbcolor{curcolor}{0 0 0}
\pscustom[linestyle=none,fillstyle=solid,fillcolor=curcolor]
{
\newpath
\moveto(368.16796629,43.12497068)
\lineto(370.32421629,43.12497068)
\lineto(370.32421629,29.99997068)
\lineto(368.16796629,29.99997068)
\lineto(368.16796629,43.12497068)
\moveto(368.16796629,48.23434568)
\lineto(370.32421629,48.23434568)
\lineto(370.32421629,45.50387693)
\lineto(368.16796629,45.50387693)
\lineto(368.16796629,48.23434568)
}
}
{
\newrgbcolor{curcolor}{0 0 0}
\pscustom[linestyle=none,fillstyle=solid,fillcolor=curcolor]
{
\newpath
\moveto(383.46093504,41.13278318)
\lineto(383.46093504,48.23434568)
\lineto(385.61718504,48.23434568)
\lineto(385.61718504,29.99997068)
\lineto(383.46093504,29.99997068)
\lineto(383.46093504,31.96872068)
\curveto(383.00779959,31.18746949)(382.43358141,30.60543882)(381.73827879,30.22262693)
\curveto(381.0507703,29.84762708)(380.22264612,29.66012727)(379.25390379,29.66012693)
\curveto(377.66796118,29.66012727)(376.37499372,30.29293913)(375.37499754,31.55856443)
\curveto(374.38280821,32.8241866)(373.88671496,34.48824744)(373.88671629,36.55075193)
\curveto(373.88671496,38.61324331)(374.38280821,40.27730415)(375.37499754,41.54293943)
\curveto(376.37499372,42.80855162)(377.66796118,43.44136348)(379.25390379,43.44137693)
\curveto(380.22264612,43.44136348)(381.0507703,43.24995743)(381.73827879,42.86715818)
\curveto(382.43358141,42.49214568)(383.00779959,41.91402126)(383.46093504,41.13278318)
\moveto(376.11327879,36.55075193)
\curveto(376.11327523,34.96480946)(376.43749366,33.71871696)(377.08593504,32.81247068)
\curveto(377.74217986,31.91403126)(378.64061646,31.46481296)(379.78124754,31.46481443)
\curveto(380.92186418,31.46481296)(381.82030078,31.91403126)(382.47656004,32.81247068)
\curveto(383.13279946,33.71871696)(383.46092414,34.96480946)(383.46093504,36.55075193)
\curveto(383.46092414,38.13668129)(383.13279946,39.37886755)(382.47656004,40.27731443)
\curveto(381.82030078,41.18355324)(380.92186418,41.63667779)(379.78124754,41.63668943)
\curveto(378.64061646,41.63667779)(377.74217986,41.18355324)(377.08593504,40.27731443)
\curveto(376.43749366,39.37886755)(376.11327523,38.13668129)(376.11327879,36.55075193)
}
}
{
\newrgbcolor{curcolor}{0 0 0}
\pscustom[linestyle=none,fillstyle=solid,fillcolor=curcolor]
{
\newpath
\moveto(401.28515379,37.10153318)
\lineto(401.28515379,36.04684568)
\lineto(391.37109129,36.04684568)
\curveto(391.46483762,34.56246611)(391.91014967,33.42965475)(392.70702879,32.64840818)
\curveto(393.51171057,31.8749688)(394.62889695,31.48825044)(396.05859129,31.48825193)
\curveto(396.8867072,31.48825044)(397.68748764,31.58981284)(398.46093504,31.79293943)
\curveto(399.24217359,31.99606243)(400.01561032,32.30074962)(400.78124754,32.70700193)
\lineto(400.78124754,30.66793943)
\curveto(400.00779782,30.33981409)(399.21482987,30.08981434)(398.40234129,29.91793943)
\curveto(397.58983149,29.74606468)(396.76561357,29.66012727)(395.92968504,29.66012693)
\curveto(393.835929,29.66012727)(392.17577441,30.26950166)(390.94921629,31.48825193)
\curveto(389.73046435,32.70699922)(389.12108996,34.35543507)(389.12109129,36.43356443)
\curveto(389.12108996,38.58199334)(389.69921438,40.28511664)(390.85546629,41.54293943)
\curveto(392.01952456,42.80855162)(393.58592925,43.44136348)(395.55468504,43.44137693)
\curveto(397.32030051,43.44136348)(398.71483037,42.87105155)(399.73827879,41.73043943)
\curveto(400.76951581,40.59761633)(401.2851403,39.05464912)(401.28515379,37.10153318)
\moveto(399.12890379,37.73434568)
\curveto(399.11326747,38.91402426)(398.78123655,39.85542957)(398.13281004,40.55856443)
\curveto(397.49217534,41.26167816)(396.64061369,41.61324031)(395.57812254,41.61325193)
\curveto(394.37499096,41.61324031)(393.41014817,41.2733969)(392.68359129,40.59372068)
\curveto(391.96483712,39.91402326)(391.55077503,38.95699297)(391.44140379,37.72262693)
\lineto(399.12890379,37.73434568)
}
}
{
\newrgbcolor{curcolor}{0 0 0}
\pscustom[linestyle=none,fillstyle=solid,fillcolor=curcolor]
{
\newpath
\moveto(412.42968504,41.10934568)
\curveto(412.18748791,41.24995943)(411.92186318,41.35152182)(411.63281004,41.41403318)
\curveto(411.35155125,41.48433419)(411.03905156,41.51949041)(410.69531004,41.51950193)
\curveto(409.47655312,41.51949041)(408.53905406,41.1210533)(407.88281004,40.32418943)
\curveto(407.23436786,39.53511739)(406.91014944,38.39839978)(406.91015379,36.91403318)
\lineto(406.91015379,29.99997068)
\lineto(404.74218504,29.99997068)
\lineto(404.74218504,43.12497068)
\lineto(406.91015379,43.12497068)
\lineto(406.91015379,41.08590818)
\curveto(407.36327398,41.88277129)(407.95311714,42.47261445)(408.67968504,42.85543943)
\curveto(409.40624069,43.24605118)(410.28905231,43.44136348)(411.32812254,43.44137693)
\curveto(411.47655112,43.44136348)(411.64061346,43.42964475)(411.82031004,43.40622068)
\curveto(411.9999881,43.39058229)(412.19920665,43.36323856)(412.41796629,43.32418943)
\lineto(412.42968504,41.10934568)
}
}
{
\newrgbcolor{curcolor}{0 0 0}
\pscustom[linewidth=2,linecolor=curcolor,linestyle=dashed,dash=8 8]
{
\newpath
\moveto(409.999999,439.999998)
\lineto(498.999999,439.999998)
}
}
{
\newrgbcolor{curcolor}{0 0 0}
\pscustom[linestyle=none,fillstyle=solid,fillcolor=curcolor]
{
\newpath
\moveto(420.46230436,435.15951576)
\lineto(407.3512516,439.98078209)
\lineto(420.46230508,444.80204736)
\curveto(418.367707,441.95557428)(418.37977612,438.06110508)(420.46230436,435.15951576)
\lineto(420.46230436,435.15951576)
\closepath
}
}
{
\newrgbcolor{curcolor}{0 0 0}
\pscustom[linewidth=2,linecolor=curcolor,linestyle=dashed,dash=8 8]
{
\newpath
\moveto(389.999999,349.999998)
\lineto(499.999999,349.999998)
}
}
{
\newrgbcolor{curcolor}{0 0 0}
\pscustom[linestyle=none,fillstyle=solid,fillcolor=curcolor]
{
\newpath
\moveto(400.46230436,345.15951576)
\lineto(387.3512516,349.98078209)
\lineto(400.46230508,354.80204736)
\curveto(398.367707,351.95557428)(398.37977612,348.06110508)(400.46230436,345.15951576)
\lineto(400.46230436,345.15951576)
\closepath
}
}
{
\newrgbcolor{curcolor}{0 0 0}
\pscustom[linewidth=2,linecolor=curcolor,linestyle=dashed,dash=8 8]
{
\newpath
\moveto(399.999999,299.999998)
\lineto(499.999999,299.999998)
}
}
{
\newrgbcolor{curcolor}{0 0 0}
\pscustom[linestyle=none,fillstyle=solid,fillcolor=curcolor]
{
\newpath
\moveto(410.46230436,295.15951576)
\lineto(397.3512516,299.98078209)
\lineto(410.46230508,304.80204736)
\curveto(408.367707,301.95557428)(408.37977612,298.06110508)(410.46230436,295.15951576)
\lineto(410.46230436,295.15951576)
\closepath
}
}
{
\newrgbcolor{curcolor}{0 0 0}
\pscustom[linewidth=2,linecolor=curcolor,linestyle=dashed,dash=8 8]
{
\newpath
\moveto(410,40)
\lineto(500,40)
}
}
{
\newrgbcolor{curcolor}{0 0 0}
\pscustom[linestyle=none,fillstyle=solid,fillcolor=curcolor]
{
\newpath
\moveto(420.46230536,35.15951776)
\lineto(407.3512526,39.98078409)
\lineto(420.46230608,44.80204936)
\curveto(418.367708,41.95557628)(418.37977712,38.06110708)(420.46230536,35.15951776)
\lineto(420.46230536,35.15951776)
\closepath
}
}
{
\newrgbcolor{curcolor}{0 0 0}
\pscustom[linewidth=2,linecolor=curcolor,linestyle=dashed,dash=8 8]
{
\newpath
\moveto(150,40)
\lineto(60,40)
}
}
{
\newrgbcolor{curcolor}{0 0 0}
\pscustom[linestyle=none,fillstyle=solid,fillcolor=curcolor]
{
\newpath
\moveto(139.53769464,44.84048224)
\lineto(152.6487474,40.01921591)
\lineto(139.53769392,35.19795064)
\curveto(141.632292,38.04442372)(141.62022288,41.93889292)(139.53769464,44.84048224)
\lineto(139.53769464,44.84048224)
\closepath
}
}
{
\newrgbcolor{curcolor}{0 0 0}
\pscustom[linewidth=2,linecolor=curcolor,linestyle=dashed,dash=8 8]
{
\newpath
\moveto(140,540)
\lineto(60,540)
}
}
{
\newrgbcolor{curcolor}{0 0 0}
\pscustom[linestyle=none,fillstyle=solid,fillcolor=curcolor]
{
\newpath
\moveto(129.53769464,544.84048224)
\lineto(142.6487474,540.01921591)
\lineto(129.53769392,535.19795064)
\curveto(131.632292,538.04442372)(131.62022288,541.93889292)(129.53769464,544.84048224)
\lineto(129.53769464,544.84048224)
\closepath
}
}
{
\newrgbcolor{curcolor}{0 0 0}
\pscustom[linestyle=none,fillstyle=solid,fillcolor=curcolor]
{
\newpath
\moveto(509.73827879,349.43359494)
\curveto(510.87108041,349.19139825)(511.75389203,348.68749251)(512.38671629,347.92187619)
\curveto(513.02732826,347.15624404)(513.34764044,346.21093248)(513.34765379,345.08593869)
\curveto(513.34764044,343.35937283)(512.75389103,342.02343667)(511.56640379,341.07812619)
\curveto(510.37889341,340.13281356)(508.69139509,339.66015778)(506.50390379,339.66015744)
\curveto(505.76952302,339.66015778)(505.01171127,339.73437646)(504.23046629,339.88281369)
\curveto(503.45702533,340.02343867)(502.65624488,340.23828221)(501.82812254,340.52734494)
\lineto(501.82812254,342.81250119)
\curveto(502.48437005,342.42968626)(503.20311933,342.14062405)(503.98437254,341.94531369)
\curveto(504.76561777,341.74999944)(505.5820232,341.65234329)(506.43359129,341.65234494)
\curveto(507.91795837,341.65234329)(509.04686349,341.94531175)(509.82031004,342.53125119)
\curveto(510.60154943,343.11718558)(510.99217404,343.96874722)(510.99218504,345.08593869)
\curveto(510.99217404,346.11718258)(510.62889316,346.92186927)(509.90234129,347.50000119)
\curveto(509.1835821,348.08593061)(508.17967686,348.37889906)(506.89062254,348.37890744)
\lineto(504.85156004,348.37890744)
\lineto(504.85156004,350.32421994)
\lineto(506.98437254,350.32421994)
\curveto(508.14842689,350.32420962)(509.039051,350.55467814)(509.65624754,351.01562619)
\curveto(510.27342476,351.48436471)(510.5820182,352.15623904)(510.58202879,353.03125119)
\curveto(510.5820182,353.92967476)(510.26170602,354.61717408)(509.62109129,355.09375119)
\curveto(508.9882698,355.57811062)(508.07811446,355.82029787)(506.89062254,355.82031369)
\curveto(506.24217879,355.82029787)(505.54686699,355.74998544)(504.80468504,355.60937619)
\curveto(504.06249347,355.46873572)(503.24608804,355.24998594)(502.35546629,354.95312619)
\lineto(502.35546629,357.06250119)
\curveto(503.25390053,357.31248388)(504.09374344,357.49998369)(504.87499754,357.62500119)
\curveto(505.66405437,357.74998344)(506.40624113,357.81248338)(507.10156004,357.81250119)
\curveto(508.89842614,357.81248338)(510.32029971,357.40232754)(511.36718504,356.58203244)
\curveto(512.41404762,355.76951667)(512.9374846,354.66795528)(512.93749754,353.27734494)
\curveto(512.9374846,352.30858263)(512.66014112,351.48827096)(512.10546629,350.81640744)
\curveto(511.55076723,350.15233479)(510.76170552,349.69139775)(509.73827879,349.43359494)
}
}
{
\newrgbcolor{curcolor}{0 0 0}
\pscustom[linestyle=none,fillstyle=solid,fillcolor=curcolor]
{
\newpath
\moveto(509.07031004,305.43359494)
\lineto(503.09374754,296.09375119)
\lineto(509.07031004,296.09375119)
\lineto(509.07031004,305.43359494)
\moveto(508.44921629,307.49609494)
\lineto(511.42577879,307.49609494)
\lineto(511.42577879,296.09375119)
\lineto(513.92187254,296.09375119)
\lineto(513.92187254,294.12500119)
\lineto(511.42577879,294.12500119)
\lineto(511.42577879,290.00000119)
\lineto(509.07031004,290.00000119)
\lineto(509.07031004,294.12500119)
\lineto(501.17187254,294.12500119)
\lineto(501.17187254,296.41015744)
\lineto(508.44921629,307.49609494)
}
}
{
\newrgbcolor{curcolor}{0 0 0}
\pscustom[linestyle=none,fillstyle=solid,fillcolor=curcolor]
{
\newpath
\moveto(52.58984319,47.49606443)
\lineto(61.88281194,47.49606443)
\lineto(61.88281194,45.50387693)
\lineto(54.75781194,45.50387693)
\lineto(54.75781194,41.21481443)
\curveto(55.10155684,41.33199059)(55.4453065,41.41792801)(55.78906194,41.47262693)
\curveto(56.13280581,41.53511539)(56.47655547,41.56636536)(56.82031194,41.56637693)
\curveto(58.77342817,41.56636536)(60.32030162,41.03120964)(61.46093694,39.96090818)
\curveto(62.60154934,38.89058679)(63.17186127,37.44136948)(63.17187444,35.61325193)
\curveto(63.17186127,33.7304357)(62.58592436,32.26559341)(61.41406194,31.21872068)
\curveto(60.2421767,30.179658)(58.5898346,29.66012727)(56.45703069,29.66012693)
\curveto(55.72264997,29.66012727)(54.97265072,29.7226272)(54.20703069,29.84762693)
\curveto(53.44921474,29.97262695)(52.66405928,30.16012677)(51.85156194,30.41012693)
\lineto(51.85156194,32.78903318)
\curveto(52.55468439,32.40621827)(53.28124616,32.1210623)(54.03124944,31.93356443)
\curveto(54.78124466,31.74606268)(55.57421262,31.65231277)(56.41015569,31.65231443)
\curveto(57.76171043,31.65231277)(58.83202186,32.00778117)(59.62109319,32.71872068)
\curveto(60.41014528,33.42965475)(60.80467614,34.39449753)(60.80468694,35.61325193)
\curveto(60.80467614,36.83199509)(60.41014528,37.79683788)(59.62109319,38.50778318)
\curveto(58.83202186,39.21871146)(57.76171043,39.57417985)(56.41015569,39.57418943)
\curveto(55.77733742,39.57417985)(55.14452555,39.50386742)(54.51171819,39.36325193)
\curveto(53.88671431,39.2226177)(53.24608995,39.00386792)(52.58984319,38.70700193)
\lineto(52.58984319,47.49606443)
}
}
{
\newrgbcolor{curcolor}{0 0 0}
\pscustom[linestyle=none,fillstyle=solid,fillcolor=curcolor]
{
\newpath
\moveto(507.92187254,39.69137693)
\curveto(506.85936568,39.69136723)(506.01561652,39.32808635)(505.39062254,38.60153318)
\curveto(504.77343026,37.8749628)(504.46483682,36.87887005)(504.46484129,35.61325193)
\curveto(504.46483682,34.35543507)(504.77343026,33.35934232)(505.39062254,32.62497068)
\curveto(506.01561652,31.89840628)(506.85936568,31.53512539)(507.92187254,31.53512693)
\curveto(508.98436355,31.53512539)(509.82420646,31.89840628)(510.44140379,32.62497068)
\curveto(511.06639272,33.35934232)(511.37889241,34.35543507)(511.37890379,35.61325193)
\curveto(511.37889241,36.87887005)(511.06639272,37.8749628)(510.44140379,38.60153318)
\curveto(509.82420646,39.32808635)(508.98436355,39.69136723)(507.92187254,39.69137693)
\moveto(512.62109129,47.10934568)
\lineto(512.62109129,44.95309568)
\curveto(512.02732926,45.23433044)(511.42576736,45.44917398)(510.81640379,45.59762693)
\curveto(510.21483107,45.74604868)(509.61717542,45.82026736)(509.02343504,45.82028318)
\curveto(507.46092757,45.82026736)(506.26561627,45.29292413)(505.43749754,44.23825193)
\curveto(504.61718042,43.18355124)(504.14843089,41.58980284)(504.03124754,39.45700193)
\curveto(504.49218054,40.13667929)(505.07030496,40.65621002)(505.76562254,41.01559568)
\curveto(506.46092857,41.38277179)(507.22655281,41.56636536)(508.06249754,41.56637693)
\curveto(509.82030021,41.56636536)(511.20701758,41.03120964)(512.22265379,39.96090818)
\curveto(513.24607804,38.89839928)(513.75779628,37.44918198)(513.75781004,35.61325193)
\curveto(513.75779628,33.81637311)(513.22654681,32.3749683)(512.16406004,31.28903318)
\curveto(511.10154893,30.20309547)(509.68748785,29.66012727)(507.92187254,29.66012693)
\curveto(505.89842914,29.66012727)(504.35155568,30.43356399)(503.28124754,31.98043943)
\curveto(502.21093282,33.53512339)(501.67577711,35.78512114)(501.67577879,38.73043943)
\curveto(501.67577711,41.49605293)(502.33202645,43.69917573)(503.64452879,45.33981443)
\curveto(504.95702383,46.98823494)(506.71874082,47.81245286)(508.92968504,47.81247068)
\curveto(509.52342551,47.81245286)(510.12108116,47.75385917)(510.72265379,47.63668943)
\curveto(511.33201745,47.51948441)(511.96482932,47.34370333)(512.62109129,47.10934568)
}
}
{
\newrgbcolor{curcolor}{0 0 0}
\pscustom[linewidth=2,linecolor=curcolor]
{
\newpath
\moveto(150.000002,229.999998)
\lineto(399.999999,229.999998)
}
}
{
\newrgbcolor{curcolor}{1 1 1}
\pscustom[linestyle=none,fillstyle=solid,fillcolor=curcolor]
{
\newpath
\moveto(184.99999754,200.00000119)
\lineto(364.99999754,200.00000119)
\curveto(373.30999754,200.00000119)(379.99999754,193.31000119)(379.99999754,185.00000119)
\curveto(379.99999754,176.69000119)(373.30999754,170.00000119)(364.99999754,170.00000119)
\lineto(184.99999754,170.00000119)
\curveto(176.68999754,170.00000119)(169.99999754,176.69000119)(169.99999754,185.00000119)
\curveto(169.99999754,193.31000119)(176.68999754,200.00000119)(184.99999754,200.00000119)
\closepath
}
}
{
\newrgbcolor{curcolor}{0 0 0}
\pscustom[linewidth=2,linecolor=curcolor]
{
\newpath
\moveto(184.99999754,200.00000119)
\lineto(364.99999754,200.00000119)
\curveto(373.30999754,200.00000119)(379.99999754,193.31000119)(379.99999754,185.00000119)
\curveto(379.99999754,176.69000119)(373.30999754,170.00000119)(364.99999754,170.00000119)
\lineto(184.99999754,170.00000119)
\curveto(176.68999754,170.00000119)(169.99999754,176.69000119)(169.99999754,185.00000119)
\curveto(169.99999754,193.31000119)(176.68999754,200.00000119)(184.99999754,200.00000119)
\closepath
}
}
{
\newrgbcolor{curcolor}{0 0 0}
\pscustom[linestyle=none,fillstyle=solid,fillcolor=curcolor]
{
\newpath
\moveto(205.45702879,196.14843869)
\lineto(205.45702879,193.65234494)
\curveto(204.66013913,194.39451805)(203.80857748,194.94920499)(202.90234129,195.31640744)
\curveto(202.00389178,195.68357926)(201.04686149,195.86717283)(200.03124754,195.86718869)
\curveto(198.0312395,195.86717283)(196.49999104,195.25389219)(195.43749754,194.02734494)
\curveto(194.37499316,192.80858213)(193.84374369,191.0429589)(193.84374754,188.73046994)
\curveto(193.84374369,186.42577602)(194.37499316,184.66015278)(195.43749754,183.43359494)
\curveto(196.49999104,182.21484273)(198.0312395,181.60546834)(200.03124754,181.60546994)
\curveto(201.04686149,181.60546834)(202.00389178,181.7890619)(202.90234129,182.15625119)
\curveto(203.80857748,182.52343617)(204.66013913,183.07812312)(205.45702879,183.82031369)
\lineto(205.45702879,181.34765744)
\curveto(204.62888916,180.78515666)(203.74998379,180.36328208)(202.82031004,180.08203244)
\curveto(201.89842314,179.80078264)(200.92186161,179.66015778)(199.89062254,179.66015744)
\curveto(197.24217779,179.66015778)(195.15624238,180.46875072)(193.63281004,182.08593869)
\curveto(192.10937043,183.71093498)(191.34765244,185.92577652)(191.34765379,188.73046994)
\curveto(191.34765244,191.5429584)(192.10937043,193.75779994)(193.63281004,195.37500119)
\curveto(195.15624238,196.99998419)(197.24217779,197.81248338)(199.89062254,197.81250119)
\curveto(200.9374866,197.81248338)(201.92186061,197.67185852)(202.84374754,197.39062619)
\curveto(203.77342126,197.11717158)(204.64451414,196.70310949)(205.45702879,196.14843869)
}
}
{
\newrgbcolor{curcolor}{0 0 0}
\pscustom[linestyle=none,fillstyle=solid,fillcolor=curcolor]
{
\newpath
\moveto(219.95312254,187.92187619)
\lineto(219.95312254,180.00000119)
\lineto(217.79687254,180.00000119)
\lineto(217.79687254,187.85156369)
\curveto(217.79686152,189.0937421)(217.55467426,190.02342867)(217.07031004,190.64062619)
\curveto(216.58592523,191.25780244)(215.85936346,191.56639588)(214.89062254,191.56640744)
\curveto(213.72655309,191.56639588)(212.80858526,191.1953025)(212.13671629,190.45312619)
\curveto(211.4648366,189.71092898)(211.12889944,188.69921124)(211.12890379,187.41796994)
\lineto(211.12890379,180.00000119)
\lineto(208.96093504,180.00000119)
\lineto(208.96093504,198.23437619)
\lineto(211.12890379,198.23437619)
\lineto(211.12890379,191.08593869)
\curveto(211.64452392,191.87498932)(212.24999207,192.46483248)(212.94531004,192.85546994)
\curveto(213.64842817,193.2460817)(214.45702111,193.441394)(215.37109129,193.44140744)
\curveto(216.87889369,193.441394)(218.01951755,192.97264447)(218.79296629,192.03515744)
\curveto(219.566391,191.10545884)(219.95310936,189.73436646)(219.95312254,187.92187619)
}
}
{
\newrgbcolor{curcolor}{0 0 0}
\pscustom[linestyle=none,fillstyle=solid,fillcolor=curcolor]
{
\newpath
\moveto(230.24218504,186.59765744)
\curveto(228.49999105,186.59765085)(227.29296101,186.39843229)(226.62109129,186.00000119)
\curveto(225.94921235,185.60155809)(225.61327519,184.92187127)(225.61327879,183.96093869)
\curveto(225.61327519,183.1953105)(225.86327494,182.58593611)(226.36327879,182.13281369)
\curveto(226.87108643,181.68749951)(227.55858574,181.46484348)(228.42577879,181.46484494)
\curveto(229.62108368,181.46484348)(230.57811397,181.88671806)(231.29687254,182.73046994)
\curveto(232.02342503,183.58202886)(232.38670591,184.71093398)(232.38671629,186.11718869)
\lineto(232.38671629,186.59765744)
\lineto(230.24218504,186.59765744)
\moveto(234.54296629,187.48828244)
\lineto(234.54296629,180.00000119)
\lineto(232.38671629,180.00000119)
\lineto(232.38671629,181.99218869)
\curveto(231.89451891,181.1953125)(231.28123827,180.60546934)(230.54687254,180.22265744)
\curveto(229.81248974,179.8476576)(228.91405314,179.66015778)(227.85156004,179.66015744)
\curveto(226.50780554,179.66015778)(225.43749411,180.03515741)(224.64062254,180.78515744)
\curveto(223.8515582,181.5429684)(223.45702734,182.55468614)(223.45702879,183.82031369)
\curveto(223.45702734,185.2968709)(223.94921435,186.41015103)(224.93359129,187.16015744)
\curveto(225.92577488,187.91014953)(227.4023359,188.28514916)(229.36327879,188.28515744)
\lineto(232.38671629,188.28515744)
\lineto(232.38671629,188.49609494)
\curveto(232.38670591,189.48827296)(232.05858124,190.25389719)(231.40234129,190.79296994)
\curveto(230.75389505,191.3398336)(229.83983346,191.61327083)(228.66015379,191.61328244)
\curveto(227.91014789,191.61327083)(227.17967987,191.52342717)(226.46874754,191.34375119)
\curveto(225.75780629,191.16405253)(225.07421323,190.89452155)(224.41796629,190.53515744)
\lineto(224.41796629,192.52734494)
\curveto(225.20702559,192.83201961)(225.97264983,193.05858188)(226.71484129,193.20703244)
\curveto(227.45702334,193.36326908)(228.17967887,193.441394)(228.88281004,193.44140744)
\curveto(230.78123877,193.441394)(232.1992061,192.94920699)(233.13671629,191.96484494)
\curveto(234.07420423,190.98045896)(234.54295376,189.48827296)(234.54296629,187.48828244)
}
}
{
\newrgbcolor{curcolor}{0 0 0}
\pscustom[linestyle=none,fillstyle=solid,fillcolor=curcolor]
{
\newpath
\moveto(246.60156004,191.10937619)
\curveto(246.35936291,191.24998994)(246.09373818,191.35155234)(245.80468504,191.41406369)
\curveto(245.52342625,191.48436471)(245.21092656,191.51952092)(244.86718504,191.51953244)
\curveto(243.64842812,191.51952092)(242.71092906,191.12108382)(242.05468504,190.32421994)
\curveto(241.40624286,189.53514791)(241.08202444,188.39843029)(241.08202879,186.91406369)
\lineto(241.08202879,180.00000119)
\lineto(238.91406004,180.00000119)
\lineto(238.91406004,193.12500119)
\lineto(241.08202879,193.12500119)
\lineto(241.08202879,191.08593869)
\curveto(241.53514898,191.88280181)(242.12499214,192.47264497)(242.85156004,192.85546994)
\curveto(243.57811569,193.2460817)(244.46092731,193.441394)(245.49999754,193.44140744)
\curveto(245.64842612,193.441394)(245.81248846,193.42967526)(245.99218504,193.40625119)
\curveto(246.1718631,193.3906128)(246.37108165,193.36326908)(246.58984129,193.32421994)
\lineto(246.60156004,191.10937619)
}
}
{
\newrgbcolor{curcolor}{0 0 0}
\pscustom[linestyle=none,fillstyle=solid,fillcolor=curcolor]
{
\newpath
\moveto(257.10156004,186.71484494)
\curveto(257.10154914,188.27733667)(256.77733071,189.48827296)(256.12890379,190.34765744)
\curveto(255.4882695,191.20702124)(254.58592665,191.63670831)(253.42187254,191.63671994)
\curveto(252.26561647,191.63670831)(251.36327362,191.20702124)(250.71484129,190.34765744)
\curveto(250.07421241,189.48827296)(249.75390023,188.27733667)(249.75390379,186.71484494)
\curveto(249.75390023,185.16015228)(250.07421241,183.95312224)(250.71484129,183.09375119)
\curveto(251.36327362,182.23437396)(252.26561647,181.80468689)(253.42187254,181.80468869)
\curveto(254.58592665,181.80468689)(255.4882695,182.23437396)(256.12890379,183.09375119)
\curveto(256.77733071,183.95312224)(257.10154914,185.16015228)(257.10156004,186.71484494)
\moveto(259.25781004,181.62890744)
\curveto(259.25779698,179.39453305)(258.76170373,177.73437846)(257.76952879,176.64843869)
\curveto(256.77733071,175.55469314)(255.25780098,175.00781869)(253.21093504,175.00781369)
\curveto(252.45311629,175.00781869)(251.73827325,175.06641238)(251.06640379,175.18359494)
\curveto(250.39452459,175.29297465)(249.7421815,175.46484948)(249.10937254,175.69921994)
\lineto(249.10937254,177.79687619)
\curveto(249.7421815,177.45312874)(250.36718087,177.19922274)(250.98437254,177.03515744)
\curveto(251.60155464,176.87109807)(252.23046026,176.7890669)(252.87109129,176.78906369)
\curveto(254.2851457,176.7890669)(255.34373839,177.16016028)(256.04687254,177.90234494)
\curveto(256.74998699,178.63672131)(257.10154914,179.75000144)(257.10156004,181.24218869)
\lineto(257.10156004,182.30859494)
\curveto(256.65623708,181.53515591)(256.08592515,180.95703149)(255.39062254,180.57421994)
\curveto(254.69530154,180.19140725)(253.86327112,180.00000119)(252.89452879,180.00000119)
\curveto(251.2851487,180.00000119)(249.988275,180.61328183)(249.00390379,181.83984494)
\curveto(248.01952697,183.06640438)(247.52733996,184.69140275)(247.52734129,186.71484494)
\curveto(247.52733996,188.7460862)(248.01952697,190.37499082)(249.00390379,191.60156369)
\curveto(249.988275,192.82811337)(251.2851487,193.441394)(252.89452879,193.44140744)
\curveto(253.86327112,193.441394)(254.69530154,193.24998794)(255.39062254,192.86718869)
\curveto(256.08592515,192.48436371)(256.65623708,191.90623929)(257.10156004,191.13281369)
\lineto(257.10156004,193.12500119)
\lineto(259.25781004,193.12500119)
\lineto(259.25781004,181.62890744)
}
}
{
\newrgbcolor{curcolor}{0 0 0}
\pscustom[linestyle=none,fillstyle=solid,fillcolor=curcolor]
{
\newpath
\moveto(274.92577879,187.10156369)
\lineto(274.92577879,186.04687619)
\lineto(265.01171629,186.04687619)
\curveto(265.10546262,184.56249663)(265.55077467,183.42968526)(266.34765379,182.64843869)
\curveto(267.15233557,181.87499932)(268.26952195,181.48828096)(269.69921629,181.48828244)
\curveto(270.5273322,181.48828096)(271.32811264,181.58984335)(272.10156004,181.79296994)
\curveto(272.88279859,181.99609295)(273.65623532,182.30078014)(274.42187254,182.70703244)
\lineto(274.42187254,180.66796994)
\curveto(273.64842282,180.3398446)(272.85545487,180.08984485)(272.04296629,179.91796994)
\curveto(271.23045649,179.7460952)(270.40623857,179.66015778)(269.57031004,179.66015744)
\curveto(267.476554,179.66015778)(265.81639941,180.26953217)(264.58984129,181.48828244)
\curveto(263.37108935,182.70702974)(262.76171496,184.35546559)(262.76171629,186.43359494)
\curveto(262.76171496,188.58202386)(263.33983938,190.28514716)(264.49609129,191.54296994)
\curveto(265.66014956,192.80858213)(267.22655425,193.441394)(269.19531004,193.44140744)
\curveto(270.96092551,193.441394)(272.35545537,192.87108207)(273.37890379,191.73046994)
\curveto(274.41014081,190.59764685)(274.9257653,189.05467964)(274.92577879,187.10156369)
\moveto(272.76952879,187.73437619)
\curveto(272.75389247,188.91405478)(272.42186155,189.85546009)(271.77343504,190.55859494)
\curveto(271.13280034,191.26170868)(270.28123869,191.61327083)(269.21874754,191.61328244)
\curveto(268.01561596,191.61327083)(267.05077317,191.27342742)(266.32421629,190.59375119)
\curveto(265.60546212,189.91405378)(265.19140003,188.95702349)(265.08202879,187.72265744)
\lineto(272.76952879,187.73437619)
}
}
{
\newrgbcolor{curcolor}{0 0 0}
\pscustom[linestyle=none,fillstyle=solid,fillcolor=curcolor]
{
\newpath
\moveto(286.07031004,191.10937619)
\curveto(285.82811291,191.24998994)(285.56248818,191.35155234)(285.27343504,191.41406369)
\curveto(284.99217625,191.48436471)(284.67967656,191.51952092)(284.33593504,191.51953244)
\curveto(283.11717812,191.51952092)(282.17967906,191.12108382)(281.52343504,190.32421994)
\curveto(280.87499286,189.53514791)(280.55077444,188.39843029)(280.55077879,186.91406369)
\lineto(280.55077879,180.00000119)
\lineto(278.38281004,180.00000119)
\lineto(278.38281004,193.12500119)
\lineto(280.55077879,193.12500119)
\lineto(280.55077879,191.08593869)
\curveto(281.00389898,191.88280181)(281.59374214,192.47264497)(282.32031004,192.85546994)
\curveto(283.04686569,193.2460817)(283.92967731,193.441394)(284.96874754,193.44140744)
\curveto(285.11717612,193.441394)(285.28123846,193.42967526)(285.46093504,193.40625119)
\curveto(285.6406131,193.3906128)(285.83983165,193.36326908)(286.05859129,193.32421994)
\lineto(286.07031004,191.10937619)
}
}
{
\newrgbcolor{curcolor}{0 0 0}
\pscustom[linestyle=none,fillstyle=solid,fillcolor=curcolor]
{
}
}
{
\newrgbcolor{curcolor}{0 0 0}
\pscustom[linestyle=none,fillstyle=solid,fillcolor=curcolor]
{
\newpath
\moveto(306.90624754,187.92187619)
\lineto(306.90624754,180.00000119)
\lineto(304.74999754,180.00000119)
\lineto(304.74999754,187.85156369)
\curveto(304.74998652,189.0937421)(304.50779926,190.02342867)(304.02343504,190.64062619)
\curveto(303.53905023,191.25780244)(302.81248846,191.56639588)(301.84374754,191.56640744)
\curveto(300.67967809,191.56639588)(299.76171026,191.1953025)(299.08984129,190.45312619)
\curveto(298.4179616,189.71092898)(298.08202444,188.69921124)(298.08202879,187.41796994)
\lineto(298.08202879,180.00000119)
\lineto(295.91406004,180.00000119)
\lineto(295.91406004,193.12500119)
\lineto(298.08202879,193.12500119)
\lineto(298.08202879,191.08593869)
\curveto(298.59764892,191.87498932)(299.20311707,192.46483248)(299.89843504,192.85546994)
\curveto(300.60155317,193.2460817)(301.41014611,193.441394)(302.32421629,193.44140744)
\curveto(303.83201869,193.441394)(304.97264255,192.97264447)(305.74609129,192.03515744)
\curveto(306.519516,191.10545884)(306.90623436,189.73436646)(306.90624754,187.92187619)
}
}
{
\newrgbcolor{curcolor}{0 0 0}
\pscustom[linestyle=none,fillstyle=solid,fillcolor=curcolor]
{
\newpath
\moveto(311.23046629,193.12500119)
\lineto(313.38671629,193.12500119)
\lineto(313.38671629,180.00000119)
\lineto(311.23046629,180.00000119)
\lineto(311.23046629,193.12500119)
\moveto(311.23046629,198.23437619)
\lineto(313.38671629,198.23437619)
\lineto(313.38671629,195.50390744)
\lineto(311.23046629,195.50390744)
\lineto(311.23046629,198.23437619)
}
}
{
\newrgbcolor{curcolor}{0 0 0}
\pscustom[linestyle=none,fillstyle=solid,fillcolor=curcolor]
{
\newpath
\moveto(316.33984129,193.12500119)
\lineto(318.62499754,193.12500119)
\lineto(322.72656004,182.10937619)
\lineto(326.82812254,193.12500119)
\lineto(329.11327879,193.12500119)
\lineto(324.19140379,180.00000119)
\lineto(321.26171629,180.00000119)
\lineto(316.33984129,193.12500119)
}
}
{
\newrgbcolor{curcolor}{0 0 0}
\pscustom[linestyle=none,fillstyle=solid,fillcolor=curcolor]
{
\newpath
\moveto(343.31640379,187.10156369)
\lineto(343.31640379,186.04687619)
\lineto(333.40234129,186.04687619)
\curveto(333.49608762,184.56249663)(333.94139967,183.42968526)(334.73827879,182.64843869)
\curveto(335.54296057,181.87499932)(336.66014695,181.48828096)(338.08984129,181.48828244)
\curveto(338.9179572,181.48828096)(339.71873764,181.58984335)(340.49218504,181.79296994)
\curveto(341.27342359,181.99609295)(342.04686032,182.30078014)(342.81249754,182.70703244)
\lineto(342.81249754,180.66796994)
\curveto(342.03904782,180.3398446)(341.24607987,180.08984485)(340.43359129,179.91796994)
\curveto(339.62108149,179.7460952)(338.79686357,179.66015778)(337.96093504,179.66015744)
\curveto(335.867179,179.66015778)(334.20702441,180.26953217)(332.98046629,181.48828244)
\curveto(331.76171435,182.70702974)(331.15233996,184.35546559)(331.15234129,186.43359494)
\curveto(331.15233996,188.58202386)(331.73046438,190.28514716)(332.88671629,191.54296994)
\curveto(334.05077456,192.80858213)(335.61717925,193.441394)(337.58593504,193.44140744)
\curveto(339.35155051,193.441394)(340.74608037,192.87108207)(341.76952879,191.73046994)
\curveto(342.80076581,190.59764685)(343.3163903,189.05467964)(343.31640379,187.10156369)
\moveto(341.16015379,187.73437619)
\curveto(341.14451747,188.91405478)(340.81248655,189.85546009)(340.16406004,190.55859494)
\curveto(339.52342534,191.26170868)(338.67186369,191.61327083)(337.60937254,191.61328244)
\curveto(336.40624096,191.61327083)(335.44139817,191.27342742)(334.71484129,190.59375119)
\curveto(333.99608712,189.91405378)(333.58202503,188.95702349)(333.47265379,187.72265744)
\lineto(341.16015379,187.73437619)
}
}
{
\newrgbcolor{curcolor}{0 0 0}
\pscustom[linestyle=none,fillstyle=solid,fillcolor=curcolor]
{
\newpath
\moveto(352.82031004,186.59765744)
\curveto(351.07811605,186.59765085)(349.87108601,186.39843229)(349.19921629,186.00000119)
\curveto(348.52733735,185.60155809)(348.19140019,184.92187127)(348.19140379,183.96093869)
\curveto(348.19140019,183.1953105)(348.44139994,182.58593611)(348.94140379,182.13281369)
\curveto(349.44921143,181.68749951)(350.13671074,181.46484348)(351.00390379,181.46484494)
\curveto(352.19920868,181.46484348)(353.15623897,181.88671806)(353.87499754,182.73046994)
\curveto(354.60155003,183.58202886)(354.96483091,184.71093398)(354.96484129,186.11718869)
\lineto(354.96484129,186.59765744)
\lineto(352.82031004,186.59765744)
\moveto(357.12109129,187.48828244)
\lineto(357.12109129,180.00000119)
\lineto(354.96484129,180.00000119)
\lineto(354.96484129,181.99218869)
\curveto(354.47264391,181.1953125)(353.85936327,180.60546934)(353.12499754,180.22265744)
\curveto(352.39061474,179.8476576)(351.49217814,179.66015778)(350.42968504,179.66015744)
\curveto(349.08593054,179.66015778)(348.01561911,180.03515741)(347.21874754,180.78515744)
\curveto(346.4296832,181.5429684)(346.03515234,182.55468614)(346.03515379,183.82031369)
\curveto(346.03515234,185.2968709)(346.52733935,186.41015103)(347.51171629,187.16015744)
\curveto(348.50389988,187.91014953)(349.9804609,188.28514916)(351.94140379,188.28515744)
\lineto(354.96484129,188.28515744)
\lineto(354.96484129,188.49609494)
\curveto(354.96483091,189.48827296)(354.63670624,190.25389719)(353.98046629,190.79296994)
\curveto(353.33202005,191.3398336)(352.41795846,191.61327083)(351.23827879,191.61328244)
\curveto(350.48827289,191.61327083)(349.75780487,191.52342717)(349.04687254,191.34375119)
\curveto(348.33593129,191.16405253)(347.65233823,190.89452155)(346.99609129,190.53515744)
\lineto(346.99609129,192.52734494)
\curveto(347.78515059,192.83201961)(348.55077483,193.05858188)(349.29296629,193.20703244)
\curveto(350.03514834,193.36326908)(350.75780387,193.441394)(351.46093504,193.44140744)
\curveto(353.35936377,193.441394)(354.7773311,192.94920699)(355.71484129,191.96484494)
\curveto(356.65232923,190.98045896)(357.12107876,189.48827296)(357.12109129,187.48828244)
}
}
{
\newrgbcolor{curcolor}{0 0 0}
\pscustom[linestyle=none,fillstyle=solid,fillcolor=curcolor]
{
\newpath
\moveto(361.35156004,185.17968869)
\lineto(361.35156004,193.12500119)
\lineto(363.50781004,193.12500119)
\lineto(363.50781004,185.26171994)
\curveto(363.50780584,184.01952842)(363.7499931,183.08593561)(364.23437254,182.46093869)
\curveto(364.71874213,181.84374935)(365.4453039,181.53515591)(366.41406004,181.53515744)
\curveto(367.57811427,181.53515591)(368.4960821,181.90624929)(369.16796629,182.64843869)
\curveto(369.84764325,183.3906228)(370.18748666,184.40234054)(370.18749754,185.68359494)
\lineto(370.18749754,193.12500119)
\lineto(372.34374754,193.12500119)
\lineto(372.34374754,180.00000119)
\lineto(370.18749754,180.00000119)
\lineto(370.18749754,182.01562619)
\curveto(369.66404968,181.21874997)(369.05467529,180.62500057)(368.35937254,180.23437619)
\curveto(367.67186418,179.85156384)(366.87108373,179.66015778)(365.95702879,179.66015744)
\curveto(364.44921115,179.66015778)(363.30468104,180.12890731)(362.52343504,181.06640744)
\curveto(361.74218261,182.00390544)(361.351558,183.37499782)(361.35156004,185.17968869)
\moveto(366.77734129,193.44140744)
\lineto(366.77734129,193.44140744)
}
}
\end{pspicture}

		\end{center}

		\begin{enumerate}
		  \item Fond d'écran (présent dans l'archive stocké sur le téléphone)
		  \item Champs de texte ``Nom du niveau''
		  \item Liste déroulante ``Largeur''
		  \item Liste déroulante ``Hauteur''
		  \item Bouton ``\hyperlink{Charger}{Charger niveau}''
		  \item Bouton ``\hyperlink{Accueil}{Retour}''
		  \item Bouton ``\hyperlink{Editeur}{Valider}''
		\end{enumerate}

		\subsubsection{Description des zones}
		
			\begin{tabular}{|c|c|c|c|c|} \hline
				Numéro de zone & Type  & Description & Evènement &	Règle \\\hline
				5 & Bouton & Affiche une sous activée pour la selection de niveau & Cliqué & RG11-01 \\
				  &        & Seuls les niveaux non officiels peuvent être chargés &        &         \\\hline
				6 & Bouton & Permet de revenir à la page d'accueil & Cliqué & RG11-02 \\\hline
				7 & Bouton & Lance l'éditeur de niveau selon les paramètres choisis & Cliqué & RG11-03 \\\hline 
			\end{tabular}
			
		\subsubsection{Description des règles}

			\underline{RG11-01 :}
				\begin{quote}
					Charger la page de selection de niveau%
						\footnote[1]{
							\hyperlink{Charger un niveau}{Charger un niveau}
							\og voir section \ref{Charger}, page \pageref{Charger}.\fg
						}.\\
					Afficher la page de selection de niveau\footnotemark[1].
					Supprimer la page
				\end{quote}
				
			$\,$

			\underline{RG11-02 :}
				\begin{quote}
					Charger la page d'accueil%
						\footnote[2]{
							\hyperlink{Page d'accueil}{Page d'accueil}
							\og voir section \ref{Accueil}, page \pageref{Accueil}.\fg
						}.\\
					Afficher la page d'accueil\footnotemark[2].\\
					Supprimer la page de création de niveau%
					\footnote[3]{
						\hyperlink{Creer niveau}{Créer niveau}
						\og voir section \ref{Creer niveau}, page \pageref{Creer niveau}.\fg
					}.
				\end{quote}
				
			$\,$
			
			\underline{RG11-03 :}
				\begin{quote}
					Charger la page d'edition de carte%
					\footnote[2]{
							\hyperlink{Editeur}{Editeur de niveau}
							\og voir section \ref{Editeur}, page \pageref{Editeur}.\fg
						}.\\
					Afficher la page d'editeur de carte\footnotemark[2].\\
					Supprimer la page de création de niveau%
					\footnote[3]{
						\hyperlink{Creer niveau}{Créer niveau}
						\og voir section \ref{Creer niveau}, page \pageref{Creer niveau}.\fg
					}.
				\end{quote}
				
\newpage
	
	\subsection{Charger un niveau}
		
		\hypertarget{Charger niveau}{}
		\label{Charger niveau}
			
		\begin{center}

		\end{center}

		\begin{enumerate}
		  \item 
		\end{enumerate}

		\subsubsection{Description des zones}
		
			\begin{tabular}{|c|c|c|c|c|} \hline
				Numéro de zone & Type  & Description & Evènement &	Règle \\\hline
			\end{tabular}
			
		\subsubsection{Description des règles}

			\underline{RG12-01 :}
				\begin{quote}
				
				\end{quote}
				
\newpage
	
	\subsection{Aide}
		
		\hypertarget{Aide}{}
		\label{Aide}
			
		\begin{center}
			%LaTeX with PSTricks extensions
%%Creator: inkscape 0.47
%%Please note this file requires PSTricks extensions
\psset{xunit=.4pt,yunit=.4pt,runit=.4pt}
\begin{pspicture}(560,600)
{
\newrgbcolor{curcolor}{1 1 1}
\pscustom[linestyle=none,fillstyle=solid,fillcolor=curcolor]
{
\newpath
\moveto(133.12401716,597.5221417)
\lineto(426.87598554,597.5221417)
\curveto(443.85397304,597.5221417)(457.52217237,583.85394237)(457.52217237,566.87595487)
\lineto(457.52217237,33.124017)
\curveto(457.52217237,16.1460295)(443.85397304,2.47783017)(426.87598554,2.47783017)
\lineto(133.12401716,2.47783017)
\curveto(116.14602965,2.47783017)(102.47783033,16.1460295)(102.47783033,33.124017)
\lineto(102.47783033,566.87595487)
\curveto(102.47783033,583.85394237)(116.14602965,597.5221417)(133.12401716,597.5221417)
\closepath
}
}
{
\newrgbcolor{curcolor}{0 0 0}
\pscustom[linewidth=4.95566034,linecolor=curcolor]
{
\newpath
\moveto(133.12401716,597.5221417)
\lineto(426.87598554,597.5221417)
\curveto(443.85397304,597.5221417)(457.52217237,583.85394237)(457.52217237,566.87595487)
\lineto(457.52217237,33.124017)
\curveto(457.52217237,16.1460295)(443.85397304,2.47783017)(426.87598554,2.47783017)
\lineto(133.12401716,2.47783017)
\curveto(116.14602965,2.47783017)(102.47783033,16.1460295)(102.47783033,33.124017)
\lineto(102.47783033,566.87595487)
\curveto(102.47783033,583.85394237)(116.14602965,597.5221417)(133.12401716,597.5221417)
\closepath
}
}
{
\newrgbcolor{curcolor}{0 0 0}
\pscustom[linestyle=none,fillstyle=solid,fillcolor=curcolor]
{
\newpath
\moveto(43.96874944,502.65625119)
\lineto(49.12499944,502.65625119)
\lineto(49.12499944,520.45312619)
\lineto(43.51562444,519.32812619)
\lineto(43.51562444,522.20312619)
\lineto(49.09374944,523.32812619)
\lineto(52.24999944,523.32812619)
\lineto(52.24999944,502.65625119)
\lineto(57.40624944,502.65625119)
\lineto(57.40624944,500.00000119)
\lineto(43.96874944,500.00000119)
\lineto(43.96874944,502.65625119)
}
}
{
\newrgbcolor{curcolor}{0 0 0}
\pscustom[linestyle=none,fillstyle=solid,fillcolor=curcolor]
{
\newpath
\moveto(46.14062444,402.65625119)
\lineto(57.15624944,402.65625119)
\lineto(57.15624944,400.00000119)
\lineto(42.34374944,400.00000119)
\lineto(42.34374944,402.65625119)
\curveto(43.54166257,403.89583063)(45.17186927,405.5572873)(47.23437444,407.64062619)
\curveto(49.3072818,409.73436646)(50.60936383,411.08332344)(51.14062444,411.68750119)
\curveto(52.15102896,412.82290504)(52.85415326,413.78123741)(53.24999944,414.56250119)
\curveto(53.65623579,415.35415251)(53.85936058,416.1301934)(53.85937444,416.89062619)
\curveto(53.85936058,418.1301914)(53.42186102,419.14060705)(52.54687444,419.92187619)
\curveto(51.68227943,420.70310549)(50.55207222,421.0937301)(49.15624944,421.09375119)
\curveto(48.16665794,421.0937301)(47.11978399,420.92185527)(46.01562444,420.57812619)
\curveto(44.92186952,420.23435596)(43.74999569,419.71352315)(42.49999944,419.01562619)
\lineto(42.49999944,422.20312619)
\curveto(43.77082901,422.71352015)(44.95832782,423.09893643)(46.06249944,423.35937619)
\curveto(47.16665894,423.61976924)(48.1770746,423.74997744)(49.09374944,423.75000119)
\curveto(51.5104046,423.74997744)(53.43748601,423.14581138)(54.87499944,421.93750119)
\curveto(56.31248313,420.72914713)(57.03123241,419.11456541)(57.03124944,417.09375119)
\curveto(57.03123241,416.13540172)(56.84894093,415.2239443)(56.48437444,414.35937619)
\curveto(56.13019165,413.50519602)(55.47915063,412.49478037)(54.53124944,411.32812619)
\curveto(54.27081851,411.02603183)(53.44269433,410.15103271)(52.04687444,408.70312619)
\curveto(50.65103046,407.26561893)(48.68228243,405.24999594)(46.14062444,402.65625119)
}
}
{
\newrgbcolor{curcolor}{0 0 0}
\pscustom[linewidth=2,linecolor=curcolor,linestyle=dashed,dash=8 8]
{
\newpath
\moveto(150,510)
\lineto(60,510)
}
}
{
\newrgbcolor{curcolor}{0 0 0}
\pscustom[linestyle=none,fillstyle=solid,fillcolor=curcolor]
{
\newpath
\moveto(139.53769464,514.84048224)
\lineto(152.6487474,510.01921591)
\lineto(139.53769392,505.19795064)
\curveto(141.632292,508.04442372)(141.62022288,511.93889292)(139.53769464,514.84048224)
\lineto(139.53769464,514.84048224)
\closepath
}
}
{
\newrgbcolor{curcolor}{1 1 1}
\pscustom[linestyle=none,fillstyle=solid,fillcolor=curcolor]
{
\newpath
\moveto(212.47129861,420.00000119)
\lineto(339.52869646,420.00000119)
\curveto(345.32979725,420.00000119)(349.99999754,415.32980091)(349.99999754,409.52870011)
\lineto(349.99999754,400.47130036)
\curveto(349.99999754,394.67019957)(345.32979725,389.99999929)(339.52869646,389.99999929)
\lineto(212.47129861,389.99999929)
\curveto(206.67019782,389.99999929)(201.99999754,394.67019957)(201.99999754,400.47130036)
\lineto(201.99999754,409.52870011)
\curveto(201.99999754,415.32980091)(206.67019782,420.00000119)(212.47129861,420.00000119)
\closepath
}
}
{
\newrgbcolor{curcolor}{0 0 0}
\pscustom[linewidth=1.26719654,linecolor=curcolor]
{
\newpath
\moveto(212.47129861,420.00000119)
\lineto(339.52869646,420.00000119)
\curveto(345.32979725,420.00000119)(349.99999754,415.32980091)(349.99999754,409.52870011)
\lineto(349.99999754,400.47130036)
\curveto(349.99999754,394.67019957)(345.32979725,389.99999929)(339.52869646,389.99999929)
\lineto(212.47129861,389.99999929)
\curveto(206.67019782,389.99999929)(201.99999754,394.67019957)(201.99999754,400.47130036)
\lineto(201.99999754,409.52870011)
\curveto(201.99999754,415.32980091)(206.67019782,420.00000119)(212.47129861,420.00000119)
\closepath
}
}
{
\newrgbcolor{curcolor}{1 1 1}
\pscustom[linestyle=none,fillstyle=solid,fillcolor=curcolor]
{
\newpath
\moveto(202.79825346,370.00000119)
\lineto(359.20174161,370.00000119)
\curveto(366.29197539,370.00000119)(371.99999754,364.29197905)(371.99999754,357.20174527)
\lineto(371.99999754,349.79825711)
\curveto(371.99999754,342.70802333)(366.29197539,337.00000119)(359.20174161,337.00000119)
\lineto(202.79825346,337.00000119)
\curveto(195.70801968,337.00000119)(189.99999754,342.70802333)(189.99999754,349.79825711)
\lineto(189.99999754,357.20174527)
\curveto(189.99999754,364.29197905)(195.70801968,370.00000119)(202.79825346,370.00000119)
\closepath
}
}
{
\newrgbcolor{curcolor}{0 0 0}
\pscustom[linewidth=1.48897755,linecolor=curcolor]
{
\newpath
\moveto(202.79825346,370.00000119)
\lineto(359.20174161,370.00000119)
\curveto(366.29197539,370.00000119)(371.99999754,364.29197905)(371.99999754,357.20174527)
\lineto(371.99999754,349.79825711)
\curveto(371.99999754,342.70802333)(366.29197539,337.00000119)(359.20174161,337.00000119)
\lineto(202.79825346,337.00000119)
\curveto(195.70801968,337.00000119)(189.99999754,342.70802333)(189.99999754,349.79825711)
\lineto(189.99999754,357.20174527)
\curveto(189.99999754,364.29197905)(195.70801968,370.00000119)(202.79825346,370.00000119)
\closepath
}
}
{
\newrgbcolor{curcolor}{1 1 1}
\pscustom[linestyle=none,fillstyle=solid,fillcolor=curcolor]
{
\newpath
\moveto(234.99999849,320.00000119)
\lineto(314.99999658,320.00000119)
\curveto(323.30999711,320.00000119)(329.99999754,313.31000077)(329.99999754,305.00000024)
\curveto(329.99999754,296.68999971)(323.30999711,289.99999929)(314.99999658,289.99999929)
\lineto(234.99999849,289.99999929)
\curveto(226.68999796,289.99999929)(219.99999754,296.68999971)(219.99999754,305.00000024)
\curveto(219.99999754,313.31000077)(226.68999796,320.00000119)(234.99999849,320.00000119)
\closepath
}
}
{
\newrgbcolor{curcolor}{0 0 0}
\pscustom[linewidth=1.61259818,linecolor=curcolor]
{
\newpath
\moveto(234.99999849,320.00000119)
\lineto(314.99999658,320.00000119)
\curveto(323.30999711,320.00000119)(329.99999754,313.31000077)(329.99999754,305.00000024)
\curveto(329.99999754,296.68999971)(323.30999711,289.99999929)(314.99999658,289.99999929)
\lineto(234.99999849,289.99999929)
\curveto(226.68999796,289.99999929)(219.99999754,296.68999971)(219.99999754,305.00000024)
\curveto(219.99999754,313.31000077)(226.68999796,320.00000119)(234.99999849,320.00000119)
\closepath
}
}
{
\newrgbcolor{curcolor}{1 1 1}
\pscustom[linestyle=none,fillstyle=solid,fillcolor=curcolor]
{
\newpath
\moveto(218.70967428,270.00000119)
\lineto(341.29032079,270.00000119)
\curveto(346.11548171,270.00000119)(349.99999754,266.11548537)(349.99999754,261.29032445)
\lineto(349.99999754,248.70967984)
\curveto(349.99999754,243.88451893)(346.11548171,240.0000031)(341.29032079,240.0000031)
\lineto(218.70967428,240.0000031)
\curveto(213.88451336,240.0000031)(209.99999754,243.88451893)(209.99999754,248.70967984)
\lineto(209.99999754,261.29032445)
\curveto(209.99999754,266.11548537)(213.88451336,270.00000119)(218.70967428,270.00000119)
\closepath
}
}
{
\newrgbcolor{curcolor}{0 0 0}
\pscustom[linewidth=1.36701977,linecolor=curcolor]
{
\newpath
\moveto(218.70967428,270.00000119)
\lineto(341.29032079,270.00000119)
\curveto(346.11548171,270.00000119)(349.99999754,266.11548537)(349.99999754,261.29032445)
\lineto(349.99999754,248.70967984)
\curveto(349.99999754,243.88451893)(346.11548171,240.0000031)(341.29032079,240.0000031)
\lineto(218.70967428,240.0000031)
\curveto(213.88451336,240.0000031)(209.99999754,243.88451893)(209.99999754,248.70967984)
\lineto(209.99999754,261.29032445)
\curveto(209.99999754,266.11548537)(213.88451336,270.00000119)(218.70967428,270.00000119)
\closepath
}
}
{
\newrgbcolor{curcolor}{1 1 1}
\pscustom[linestyle=none,fillstyle=solid,fillcolor=curcolor]
{
\newpath
\moveto(135,50)
\lineto(225,50)
\curveto(233.31,50)(240,43.31)(240,35)
\curveto(240,26.69)(233.31,20)(225,20)
\lineto(135,20)
\curveto(126.69,20)(120,26.69)(120,35)
\curveto(120,43.31)(126.69,50)(135,50)
\closepath
}
}
{
\newrgbcolor{curcolor}{0 0 0}
\pscustom[linewidth=2,linecolor=curcolor]
{
\newpath
\moveto(135,50)
\lineto(225,50)
\curveto(233.31,50)(240,43.31)(240,35)
\curveto(240,26.69)(233.31,20)(225,20)
\lineto(135,20)
\curveto(126.69,20)(120,26.69)(120,35)
\curveto(120,43.31)(126.69,50)(135,50)
\closepath
}
}
{
\newrgbcolor{curcolor}{0 0 0}
\pscustom[linestyle=none,fillstyle=solid,fillcolor=curcolor]
{
\newpath
\moveto(224.72265379,408.35546994)
\lineto(224.72265379,401.94531369)
\lineto(228.51952879,401.94531369)
\curveto(229.79295649,401.94531175)(230.7343618,402.20703024)(231.34374754,402.73046994)
\curveto(231.96092307,403.26171668)(232.26951652,404.07030962)(232.26952879,405.15625119)
\curveto(232.26951652,406.24999494)(231.96092307,407.05468164)(231.34374754,407.57031369)
\curveto(230.7343618,408.0937431)(229.79295649,408.35546159)(228.51952879,408.35546994)
\lineto(224.72265379,408.35546994)
\moveto(224.72265379,415.55078244)
\lineto(224.72265379,410.27734494)
\lineto(228.22656004,410.27734494)
\curveto(229.38280065,410.27733467)(230.24217479,410.4921782)(230.80468504,410.92187619)
\curveto(231.37498616,411.35936483)(231.66014212,412.02342667)(231.66015379,412.91406369)
\curveto(231.66014212,413.7968624)(231.37498616,414.45701799)(230.80468504,414.89453244)
\curveto(230.24217479,415.33201711)(229.38280065,415.55076689)(228.22656004,415.55078244)
\lineto(224.72265379,415.55078244)
\moveto(222.35546629,417.49609494)
\lineto(228.40234129,417.49609494)
\curveto(230.20701858,417.49607745)(231.59764219,417.12107782)(232.57421629,416.37109494)
\curveto(233.55076523,415.62107932)(234.039046,414.55467414)(234.03906004,413.17187619)
\curveto(234.039046,412.10155159)(233.78904625,411.24998994)(233.28906004,410.61718869)
\curveto(232.78904725,409.98436621)(232.05467298,409.58983535)(231.08593504,409.43359494)
\curveto(232.24998529,409.18358576)(233.15232813,408.66014878)(233.79296629,407.86328244)
\curveto(234.44138934,407.07421287)(234.76560777,406.08593261)(234.76562254,404.89843869)
\curveto(234.76560777,403.33593536)(234.2343583,402.12890531)(233.17187254,401.27734494)
\curveto(232.10936043,400.42578202)(230.59764319,400.00000119)(228.63671629,400.00000119)
\lineto(222.35546629,400.00000119)
\lineto(222.35546629,417.49609494)
}
}
{
\newrgbcolor{curcolor}{0 0 0}
\pscustom[linestyle=none,fillstyle=solid,fillcolor=curcolor]
{
\newpath
\moveto(238.49218504,405.17968869)
\lineto(238.49218504,413.12500119)
\lineto(240.64843504,413.12500119)
\lineto(240.64843504,405.26171994)
\curveto(240.64843084,404.01952842)(240.8906181,403.08593561)(241.37499754,402.46093869)
\curveto(241.85936713,401.84374935)(242.5859289,401.53515591)(243.55468504,401.53515744)
\curveto(244.71873927,401.53515591)(245.6367071,401.90624929)(246.30859129,402.64843869)
\curveto(246.98826825,403.3906228)(247.32811166,404.40234054)(247.32812254,405.68359494)
\lineto(247.32812254,413.12500119)
\lineto(249.48437254,413.12500119)
\lineto(249.48437254,400.00000119)
\lineto(247.32812254,400.00000119)
\lineto(247.32812254,402.01562619)
\curveto(246.80467468,401.21874997)(246.19530029,400.62500057)(245.49999754,400.23437619)
\curveto(244.81248918,399.85156384)(244.01170873,399.66015778)(243.09765379,399.66015744)
\curveto(241.58983615,399.66015778)(240.44530604,400.12890731)(239.66406004,401.06640744)
\curveto(238.88280761,402.00390544)(238.492183,403.37499782)(238.49218504,405.17968869)
\moveto(243.91796629,413.44140744)
\lineto(243.91796629,413.44140744)
}
}
{
\newrgbcolor{curcolor}{0 0 0}
\pscustom[linestyle=none,fillstyle=solid,fillcolor=curcolor]
{
\newpath
\moveto(256.08202879,416.85156369)
\lineto(256.08202879,413.12500119)
\lineto(260.52343504,413.12500119)
\lineto(260.52343504,411.44921994)
\lineto(256.08202879,411.44921994)
\lineto(256.08202879,404.32421994)
\curveto(256.08202439,403.25390419)(256.2265555,402.56640488)(256.51562254,402.26171994)
\curveto(256.81249241,401.95703049)(257.41014806,401.80468689)(258.30859129,401.80468869)
\lineto(260.52343504,401.80468869)
\lineto(260.52343504,400.00000119)
\lineto(258.30859129,400.00000119)
\curveto(256.64452383,400.00000119)(255.49608748,400.30859463)(254.86327879,400.92578244)
\curveto(254.23046374,401.55078089)(253.91405781,402.68359226)(253.91406004,404.32421994)
\lineto(253.91406004,411.44921994)
\lineto(252.33202879,411.44921994)
\lineto(252.33202879,413.12500119)
\lineto(253.91406004,413.12500119)
\lineto(253.91406004,416.85156369)
\lineto(256.08202879,416.85156369)
}
}
{
\newrgbcolor{curcolor}{0 0 0}
\pscustom[linestyle=none,fillstyle=solid,fillcolor=curcolor]
{
}
}
{
\newrgbcolor{curcolor}{0 0 0}
\pscustom[linestyle=none,fillstyle=solid,fillcolor=curcolor]
{
\newpath
\moveto(279.64843504,411.13281369)
\lineto(279.64843504,418.23437619)
\lineto(281.80468504,418.23437619)
\lineto(281.80468504,400.00000119)
\lineto(279.64843504,400.00000119)
\lineto(279.64843504,401.96875119)
\curveto(279.19529959,401.18750001)(278.62108141,400.60546934)(277.92577879,400.22265744)
\curveto(277.2382703,399.8476576)(276.41014612,399.66015778)(275.44140379,399.66015744)
\curveto(273.85546118,399.66015778)(272.56249372,400.29296965)(271.56249754,401.55859494)
\curveto(270.57030821,402.82421712)(270.07421496,404.48827796)(270.07421629,406.55078244)
\curveto(270.07421496,408.61327383)(270.57030821,410.27733467)(271.56249754,411.54296994)
\curveto(272.56249372,412.80858213)(273.85546118,413.441394)(275.44140379,413.44140744)
\curveto(276.41014612,413.441394)(277.2382703,413.24998794)(277.92577879,412.86718869)
\curveto(278.62108141,412.4921762)(279.19529959,411.91405178)(279.64843504,411.13281369)
\moveto(272.30077879,406.55078244)
\curveto(272.30077523,404.96483998)(272.62499366,403.71874747)(273.27343504,402.81250119)
\curveto(273.92967986,401.91406178)(274.82811646,401.46484348)(275.96874754,401.46484494)
\curveto(277.10936418,401.46484348)(278.00780078,401.91406178)(278.66406004,402.81250119)
\curveto(279.32029946,403.71874747)(279.64842414,404.96483998)(279.64843504,406.55078244)
\curveto(279.64842414,408.13671181)(279.32029946,409.37889806)(278.66406004,410.27734494)
\curveto(278.00780078,411.18358376)(277.10936418,411.63670831)(275.96874754,411.63671994)
\curveto(274.82811646,411.63670831)(273.92967986,411.18358376)(273.27343504,410.27734494)
\curveto(272.62499366,409.37889806)(272.30077523,408.13671181)(272.30077879,406.55078244)
}
}
{
\newrgbcolor{curcolor}{0 0 0}
\pscustom[linestyle=none,fillstyle=solid,fillcolor=curcolor]
{
\newpath
\moveto(286.02343504,405.17968869)
\lineto(286.02343504,413.12500119)
\lineto(288.17968504,413.12500119)
\lineto(288.17968504,405.26171994)
\curveto(288.17968084,404.01952842)(288.4218681,403.08593561)(288.90624754,402.46093869)
\curveto(289.39061713,401.84374935)(290.1171789,401.53515591)(291.08593504,401.53515744)
\curveto(292.24998927,401.53515591)(293.1679571,401.90624929)(293.83984129,402.64843869)
\curveto(294.51951825,403.3906228)(294.85936166,404.40234054)(294.85937254,405.68359494)
\lineto(294.85937254,413.12500119)
\lineto(297.01562254,413.12500119)
\lineto(297.01562254,400.00000119)
\lineto(294.85937254,400.00000119)
\lineto(294.85937254,402.01562619)
\curveto(294.33592468,401.21874997)(293.72655029,400.62500057)(293.03124754,400.23437619)
\curveto(292.34373918,399.85156384)(291.54295873,399.66015778)(290.62890379,399.66015744)
\curveto(289.12108615,399.66015778)(287.97655604,400.12890731)(287.19531004,401.06640744)
\curveto(286.41405761,402.00390544)(286.023433,403.37499782)(286.02343504,405.17968869)
\moveto(291.44921629,413.44140744)
\lineto(291.44921629,413.44140744)
}
}
{
\newrgbcolor{curcolor}{0 0 0}
\pscustom[linestyle=none,fillstyle=solid,fillcolor=curcolor]
{
}
}
{
\newrgbcolor{curcolor}{0 0 0}
\pscustom[linestyle=none,fillstyle=solid,fillcolor=curcolor]
{
\newpath
\moveto(309.12109129,413.12500119)
\lineto(311.27734129,413.12500119)
\lineto(311.27734129,399.76562619)
\curveto(311.27733687,398.0937531)(310.95702469,396.88281681)(310.31640379,396.13281369)
\curveto(309.68358846,395.38281831)(308.66015198,395.00781869)(307.24609129,395.00781369)
\lineto(306.42577879,395.00781369)
\lineto(306.42577879,396.83593869)
\lineto(306.99999754,396.83593869)
\curveto(307.82030907,396.83594186)(308.37890227,397.02734792)(308.67577879,397.41015744)
\curveto(308.97265167,397.78515966)(309.12108902,398.57031512)(309.12109129,399.76562619)
\lineto(309.12109129,413.12500119)
\moveto(309.12109129,418.23437619)
\lineto(311.27734129,418.23437619)
\lineto(311.27734129,415.50390744)
\lineto(309.12109129,415.50390744)
\lineto(309.12109129,418.23437619)
}
}
{
\newrgbcolor{curcolor}{0 0 0}
\pscustom[linestyle=none,fillstyle=solid,fillcolor=curcolor]
{
\newpath
\moveto(327.00390379,407.10156369)
\lineto(327.00390379,406.04687619)
\lineto(317.08984129,406.04687619)
\curveto(317.18358762,404.56249663)(317.62889967,403.42968526)(318.42577879,402.64843869)
\curveto(319.23046057,401.87499932)(320.34764695,401.48828096)(321.77734129,401.48828244)
\curveto(322.6054572,401.48828096)(323.40623764,401.58984335)(324.17968504,401.79296994)
\curveto(324.96092359,401.99609295)(325.73436032,402.30078014)(326.49999754,402.70703244)
\lineto(326.49999754,400.66796994)
\curveto(325.72654782,400.3398446)(324.93357987,400.08984485)(324.12109129,399.91796994)
\curveto(323.30858149,399.7460952)(322.48436357,399.66015778)(321.64843504,399.66015744)
\curveto(319.554679,399.66015778)(317.89452441,400.26953217)(316.66796629,401.48828244)
\curveto(315.44921435,402.70702974)(314.83983996,404.35546559)(314.83984129,406.43359494)
\curveto(314.83983996,408.58202386)(315.41796438,410.28514716)(316.57421629,411.54296994)
\curveto(317.73827456,412.80858213)(319.30467925,413.441394)(321.27343504,413.44140744)
\curveto(323.03905051,413.441394)(324.43358037,412.87108207)(325.45702879,411.73046994)
\curveto(326.48826581,410.59764685)(327.0038903,409.05467964)(327.00390379,407.10156369)
\moveto(324.84765379,407.73437619)
\curveto(324.83201747,408.91405478)(324.49998655,409.85546009)(323.85156004,410.55859494)
\curveto(323.21092534,411.26170868)(322.35936369,411.61327083)(321.29687254,411.61328244)
\curveto(320.09374096,411.61327083)(319.12889817,411.27342742)(318.40234129,410.59375119)
\curveto(317.68358712,409.91405378)(317.26952503,408.95702349)(317.16015379,407.72265744)
\lineto(324.84765379,407.73437619)
}
}
{
\newrgbcolor{curcolor}{0 0 0}
\pscustom[linestyle=none,fillstyle=solid,fillcolor=curcolor]
{
\newpath
\moveto(330.32031004,405.17968869)
\lineto(330.32031004,413.12500119)
\lineto(332.47656004,413.12500119)
\lineto(332.47656004,405.26171994)
\curveto(332.47655584,404.01952842)(332.7187431,403.08593561)(333.20312254,402.46093869)
\curveto(333.68749213,401.84374935)(334.4140539,401.53515591)(335.38281004,401.53515744)
\curveto(336.54686427,401.53515591)(337.4648321,401.90624929)(338.13671629,402.64843869)
\curveto(338.81639325,403.3906228)(339.15623666,404.40234054)(339.15624754,405.68359494)
\lineto(339.15624754,413.12500119)
\lineto(341.31249754,413.12500119)
\lineto(341.31249754,400.00000119)
\lineto(339.15624754,400.00000119)
\lineto(339.15624754,402.01562619)
\curveto(338.63279968,401.21874997)(338.02342529,400.62500057)(337.32812254,400.23437619)
\curveto(336.64061418,399.85156384)(335.83983373,399.66015778)(334.92577879,399.66015744)
\curveto(333.41796115,399.66015778)(332.27343104,400.12890731)(331.49218504,401.06640744)
\curveto(330.71093261,402.00390544)(330.320308,403.37499782)(330.32031004,405.17968869)
\moveto(335.74609129,413.44140744)
\lineto(335.74609129,413.44140744)
}
}
{
\newrgbcolor{curcolor}{0 0 0}
\pscustom[linestyle=none,fillstyle=solid,fillcolor=curcolor]
{
\newpath
\moveto(214.35546629,364.49609494)
\lineto(216.72265379,364.49609494)
\lineto(216.72265379,347.00000119)
\lineto(214.35546629,347.00000119)
\lineto(214.35546629,364.49609494)
}
}
{
\newrgbcolor{curcolor}{0 0 0}
\pscustom[linestyle=none,fillstyle=solid,fillcolor=curcolor]
{
\newpath
\moveto(232.24999754,354.92187619)
\lineto(232.24999754,347.00000119)
\lineto(230.09374754,347.00000119)
\lineto(230.09374754,354.85156369)
\curveto(230.09373652,356.0937421)(229.85154926,357.02342867)(229.36718504,357.64062619)
\curveto(228.88280023,358.25780244)(228.15623846,358.56639588)(227.18749754,358.56640744)
\curveto(226.02342809,358.56639588)(225.10546026,358.1953025)(224.43359129,357.45312619)
\curveto(223.7617116,356.71092898)(223.42577444,355.69921124)(223.42577879,354.41796994)
\lineto(223.42577879,347.00000119)
\lineto(221.25781004,347.00000119)
\lineto(221.25781004,360.12500119)
\lineto(223.42577879,360.12500119)
\lineto(223.42577879,358.08593869)
\curveto(223.94139892,358.87498932)(224.54686707,359.46483248)(225.24218504,359.85546994)
\curveto(225.94530317,360.2460817)(226.75389611,360.441394)(227.66796629,360.44140744)
\curveto(229.17576869,360.441394)(230.31639255,359.97264447)(231.08984129,359.03515744)
\curveto(231.863266,358.10545884)(232.24998436,356.73436646)(232.24999754,354.92187619)
}
}
{
\newrgbcolor{curcolor}{0 0 0}
\pscustom[linestyle=none,fillstyle=solid,fillcolor=curcolor]
{
\newpath
\moveto(244.94140379,359.73828244)
\lineto(244.94140379,357.69921994)
\curveto(244.33201877,358.01170893)(243.6992069,358.2460837)(243.04296629,358.40234494)
\curveto(242.38670821,358.55858338)(241.70702139,358.63670831)(241.00390379,358.63671994)
\curveto(239.93358566,358.63670831)(239.12889897,358.47264597)(238.58984129,358.14453244)
\curveto(238.05858754,357.81639663)(237.7929628,357.32420962)(237.79296629,356.66796994)
\curveto(237.7929628,356.16796078)(237.98436886,355.77342992)(238.36718504,355.48437619)
\curveto(238.7499931,355.20311799)(239.51952358,354.93358701)(240.67577879,354.67578244)
\lineto(241.41406004,354.51171994)
\curveto(242.9453014,354.18358776)(244.03123782,353.71874447)(244.67187254,353.11718869)
\curveto(245.32029903,352.52343317)(245.64451745,351.69140275)(245.64452879,350.62109494)
\curveto(245.64451745,349.40234254)(245.16014294,348.43749976)(244.19140379,347.72656369)
\curveto(243.23045737,347.01562618)(241.90623994,346.66015778)(240.21874754,346.66015744)
\curveto(239.51561733,346.66015778)(238.78124307,346.73047021)(238.01562254,346.87109494)
\curveto(237.25780709,347.00390744)(236.45702664,347.20703224)(235.61327879,347.48046994)
\lineto(235.61327879,349.70703244)
\curveto(236.41015169,349.29296765)(237.19530715,348.98046796)(237.96874754,348.76953244)
\curveto(238.74218061,348.56640588)(239.50780484,348.46484348)(240.26562254,348.46484494)
\curveto(241.28124057,348.46484348)(242.06248979,348.63671831)(242.60937254,348.98046994)
\curveto(243.15623869,349.33203011)(243.42967592,349.82421712)(243.42968504,350.45703244)
\curveto(243.42967592,351.0429659)(243.23045737,351.4921842)(242.83202879,351.80468869)
\curveto(242.44139566,352.11718358)(241.57811527,352.41796453)(240.24218504,352.70703244)
\lineto(239.49218504,352.88281369)
\curveto(238.15624369,353.16405753)(237.19140091,353.5937446)(236.59765379,354.17187619)
\curveto(236.00390209,354.75780594)(235.70702739,355.55858638)(235.70702879,356.57421994)
\curveto(235.70702739,357.80858413)(236.14452695,358.76170818)(237.01952879,359.43359494)
\curveto(237.8945252,360.10545684)(239.13671146,360.441394)(240.74609129,360.44140744)
\curveto(241.54295905,360.441394)(242.2929583,360.38280031)(242.99609129,360.26562619)
\curveto(243.6992069,360.14842554)(244.34764375,359.97264447)(244.94140379,359.73828244)
}
}
{
\newrgbcolor{curcolor}{0 0 0}
\pscustom[linestyle=none,fillstyle=solid,fillcolor=curcolor]
{
\newpath
\moveto(251.22265379,363.85156369)
\lineto(251.22265379,360.12500119)
\lineto(255.66406004,360.12500119)
\lineto(255.66406004,358.44921994)
\lineto(251.22265379,358.44921994)
\lineto(251.22265379,351.32421994)
\curveto(251.22264939,350.25390419)(251.3671805,349.56640488)(251.65624754,349.26171994)
\curveto(251.95311741,348.95703049)(252.55077306,348.80468689)(253.44921629,348.80468869)
\lineto(255.66406004,348.80468869)
\lineto(255.66406004,347.00000119)
\lineto(253.44921629,347.00000119)
\curveto(251.78514883,347.00000119)(250.63671248,347.30859463)(250.00390379,347.92578244)
\curveto(249.37108874,348.55078089)(249.05468281,349.68359226)(249.05468504,351.32421994)
\lineto(249.05468504,358.44921994)
\lineto(247.47265379,358.44921994)
\lineto(247.47265379,360.12500119)
\lineto(249.05468504,360.12500119)
\lineto(249.05468504,363.85156369)
\lineto(251.22265379,363.85156369)
}
}
{
\newrgbcolor{curcolor}{0 0 0}
\pscustom[linestyle=none,fillstyle=solid,fillcolor=curcolor]
{
\newpath
\moveto(266.11718504,358.10937619)
\curveto(265.87498791,358.24998994)(265.60936318,358.35155234)(265.32031004,358.41406369)
\curveto(265.03905125,358.48436471)(264.72655156,358.51952092)(264.38281004,358.51953244)
\curveto(263.16405312,358.51952092)(262.22655406,358.12108382)(261.57031004,357.32421994)
\curveto(260.92186786,356.53514791)(260.59764944,355.39843029)(260.59765379,353.91406369)
\lineto(260.59765379,347.00000119)
\lineto(258.42968504,347.00000119)
\lineto(258.42968504,360.12500119)
\lineto(260.59765379,360.12500119)
\lineto(260.59765379,358.08593869)
\curveto(261.05077398,358.88280181)(261.64061714,359.47264497)(262.36718504,359.85546994)
\curveto(263.09374069,360.2460817)(263.97655231,360.441394)(265.01562254,360.44140744)
\curveto(265.16405112,360.441394)(265.32811346,360.42967526)(265.50781004,360.40625119)
\curveto(265.6874881,360.3906128)(265.88670665,360.36326908)(266.10546629,360.32421994)
\lineto(266.11718504,358.10937619)
}
}
{
\newrgbcolor{curcolor}{0 0 0}
\pscustom[linestyle=none,fillstyle=solid,fillcolor=curcolor]
{
\newpath
\moveto(268.17968504,352.17968869)
\lineto(268.17968504,360.12500119)
\lineto(270.33593504,360.12500119)
\lineto(270.33593504,352.26171994)
\curveto(270.33593084,351.01952842)(270.5781181,350.08593561)(271.06249754,349.46093869)
\curveto(271.54686713,348.84374935)(272.2734289,348.53515591)(273.24218504,348.53515744)
\curveto(274.40623927,348.53515591)(275.3242071,348.90624929)(275.99609129,349.64843869)
\curveto(276.67576825,350.3906228)(277.01561166,351.40234054)(277.01562254,352.68359494)
\lineto(277.01562254,360.12500119)
\lineto(279.17187254,360.12500119)
\lineto(279.17187254,347.00000119)
\lineto(277.01562254,347.00000119)
\lineto(277.01562254,349.01562619)
\curveto(276.49217468,348.21874997)(275.88280029,347.62500057)(275.18749754,347.23437619)
\curveto(274.49998918,346.85156384)(273.69920873,346.66015778)(272.78515379,346.66015744)
\curveto(271.27733615,346.66015778)(270.13280604,347.12890731)(269.35156004,348.06640744)
\curveto(268.57030761,349.00390544)(268.179683,350.37499782)(268.17968504,352.17968869)
\moveto(273.60546629,360.44140744)
\lineto(273.60546629,360.44140744)
}
}
{
\newrgbcolor{curcolor}{0 0 0}
\pscustom[linestyle=none,fillstyle=solid,fillcolor=curcolor]
{
\newpath
\moveto(293.08202879,359.62109494)
\lineto(293.08202879,357.60546994)
\curveto(292.47264269,357.9413965)(291.85936205,358.19139625)(291.24218504,358.35546994)
\curveto(290.63280078,358.52733342)(290.01561389,358.61327083)(289.39062254,358.61328244)
\curveto(287.99217842,358.61327083)(286.906242,358.16795878)(286.13281004,357.27734494)
\curveto(285.35936855,356.39452305)(284.97265019,355.15233679)(284.97265379,353.55078244)
\curveto(284.97265019,351.94921499)(285.35936855,350.70312249)(286.13281004,349.81250119)
\curveto(286.906242,348.92968676)(287.99217842,348.48828096)(289.39062254,348.48828244)
\curveto(290.01561389,348.48828096)(290.63280078,348.57031212)(291.24218504,348.73437619)
\curveto(291.85936205,348.90624929)(292.47264269,349.16015528)(293.08202879,349.49609494)
\lineto(293.08202879,347.50390744)
\curveto(292.48045518,347.22265722)(291.8554558,347.01171993)(291.20702879,346.87109494)
\curveto(290.56639459,346.73047021)(289.88280153,346.66015778)(289.15624754,346.66015744)
\curveto(287.17967923,346.66015778)(285.6093683,347.28125091)(284.44531004,348.52343869)
\curveto(283.28124563,349.76562343)(282.69921496,351.441403)(282.69921629,353.55078244)
\curveto(282.69921496,355.69139875)(283.28515187,357.37499082)(284.45702879,358.60156369)
\curveto(285.63671202,359.82811337)(287.24999166,360.441394)(289.29687254,360.44140744)
\curveto(289.96092645,360.441394)(290.6093633,360.37108157)(291.24218504,360.23046994)
\curveto(291.87498704,360.09764435)(292.48826767,359.89451955)(293.08202879,359.62109494)
}
}
{
\newrgbcolor{curcolor}{0 0 0}
\pscustom[linestyle=none,fillstyle=solid,fillcolor=curcolor]
{
\newpath
\moveto(298.98827879,363.85156369)
\lineto(298.98827879,360.12500119)
\lineto(303.42968504,360.12500119)
\lineto(303.42968504,358.44921994)
\lineto(298.98827879,358.44921994)
\lineto(298.98827879,351.32421994)
\curveto(298.98827439,350.25390419)(299.1328055,349.56640488)(299.42187254,349.26171994)
\curveto(299.71874241,348.95703049)(300.31639806,348.80468689)(301.21484129,348.80468869)
\lineto(303.42968504,348.80468869)
\lineto(303.42968504,347.00000119)
\lineto(301.21484129,347.00000119)
\curveto(299.55077383,347.00000119)(298.40233748,347.30859463)(297.76952879,347.92578244)
\curveto(297.13671374,348.55078089)(296.82030781,349.68359226)(296.82031004,351.32421994)
\lineto(296.82031004,358.44921994)
\lineto(295.23827879,358.44921994)
\lineto(295.23827879,360.12500119)
\lineto(296.82031004,360.12500119)
\lineto(296.82031004,363.85156369)
\lineto(298.98827879,363.85156369)
}
}
{
\newrgbcolor{curcolor}{0 0 0}
\pscustom[linestyle=none,fillstyle=solid,fillcolor=curcolor]
{
\newpath
\moveto(306.27734129,360.12500119)
\lineto(308.43359129,360.12500119)
\lineto(308.43359129,347.00000119)
\lineto(306.27734129,347.00000119)
\lineto(306.27734129,360.12500119)
\moveto(306.27734129,365.23437619)
\lineto(308.43359129,365.23437619)
\lineto(308.43359129,362.50390744)
\lineto(306.27734129,362.50390744)
\lineto(306.27734129,365.23437619)
}
}
{
\newrgbcolor{curcolor}{0 0 0}
\pscustom[linestyle=none,fillstyle=solid,fillcolor=curcolor]
{
\newpath
\moveto(318.01952879,358.61328244)
\curveto(316.86327259,358.61327083)(315.94921101,358.16014628)(315.27734129,357.25390744)
\curveto(314.60546235,356.35546059)(314.26952519,355.12108682)(314.26952879,353.55078244)
\curveto(314.26952519,351.98046496)(314.60155611,350.74218495)(315.26562254,349.83593869)
\curveto(315.93749227,348.93749926)(316.8554601,348.48828096)(318.01952879,348.48828244)
\curveto(319.16795779,348.48828096)(320.07811313,348.9414055)(320.74999754,349.84765744)
\curveto(321.42186179,350.75390369)(321.75779895,351.98827746)(321.75781004,353.55078244)
\curveto(321.75779895,355.10546184)(321.42186179,356.33592936)(320.74999754,357.24218869)
\curveto(320.07811313,358.15624004)(319.16795779,358.61327083)(318.01952879,358.61328244)
\moveto(318.01952879,360.44140744)
\curveto(319.89451956,360.441394)(321.36717434,359.83201961)(322.43749754,358.61328244)
\curveto(323.5077972,357.39452205)(324.04295291,355.70702374)(324.04296629,353.55078244)
\curveto(324.04295291,351.40234054)(323.5077972,349.71484223)(322.43749754,348.48828244)
\curveto(321.36717434,347.26953217)(319.89451956,346.66015778)(318.01952879,346.66015744)
\curveto(316.13671082,346.66015778)(314.6601498,347.26953217)(313.58984129,348.48828244)
\curveto(312.52733943,349.71484223)(311.99608996,351.40234054)(311.99609129,353.55078244)
\curveto(311.99608996,355.70702374)(312.52733943,357.39452205)(313.58984129,358.61328244)
\curveto(314.6601498,359.83201961)(316.13671082,360.441394)(318.01952879,360.44140744)
}
}
{
\newrgbcolor{curcolor}{0 0 0}
\pscustom[linestyle=none,fillstyle=solid,fillcolor=curcolor]
{
\newpath
\moveto(338.51562254,354.92187619)
\lineto(338.51562254,347.00000119)
\lineto(336.35937254,347.00000119)
\lineto(336.35937254,354.85156369)
\curveto(336.35936152,356.0937421)(336.11717426,357.02342867)(335.63281004,357.64062619)
\curveto(335.14842523,358.25780244)(334.42186346,358.56639588)(333.45312254,358.56640744)
\curveto(332.28905309,358.56639588)(331.37108526,358.1953025)(330.69921629,357.45312619)
\curveto(330.0273366,356.71092898)(329.69139944,355.69921124)(329.69140379,354.41796994)
\lineto(329.69140379,347.00000119)
\lineto(327.52343504,347.00000119)
\lineto(327.52343504,360.12500119)
\lineto(329.69140379,360.12500119)
\lineto(329.69140379,358.08593869)
\curveto(330.20702392,358.87498932)(330.81249207,359.46483248)(331.50781004,359.85546994)
\curveto(332.21092817,360.2460817)(333.01952111,360.441394)(333.93359129,360.44140744)
\curveto(335.44139369,360.441394)(336.58201755,359.97264447)(337.35546629,359.03515744)
\curveto(338.128891,358.10545884)(338.51560936,356.73436646)(338.51562254,354.92187619)
}
}
{
\newrgbcolor{curcolor}{0 0 0}
\pscustom[linestyle=none,fillstyle=solid,fillcolor=curcolor]
{
\newpath
\moveto(351.20702879,359.73828244)
\lineto(351.20702879,357.69921994)
\curveto(350.59764377,358.01170893)(349.9648319,358.2460837)(349.30859129,358.40234494)
\curveto(348.65233321,358.55858338)(347.97264639,358.63670831)(347.26952879,358.63671994)
\curveto(346.19921066,358.63670831)(345.39452397,358.47264597)(344.85546629,358.14453244)
\curveto(344.32421254,357.81639663)(344.0585878,357.32420962)(344.05859129,356.66796994)
\curveto(344.0585878,356.16796078)(344.24999386,355.77342992)(344.63281004,355.48437619)
\curveto(345.0156181,355.20311799)(345.78514858,354.93358701)(346.94140379,354.67578244)
\lineto(347.67968504,354.51171994)
\curveto(349.2109264,354.18358776)(350.29686282,353.71874447)(350.93749754,353.11718869)
\curveto(351.58592403,352.52343317)(351.91014245,351.69140275)(351.91015379,350.62109494)
\curveto(351.91014245,349.40234254)(351.42576794,348.43749976)(350.45702879,347.72656369)
\curveto(349.49608237,347.01562618)(348.17186494,346.66015778)(346.48437254,346.66015744)
\curveto(345.78124233,346.66015778)(345.04686807,346.73047021)(344.28124754,346.87109494)
\curveto(343.52343209,347.00390744)(342.72265164,347.20703224)(341.87890379,347.48046994)
\lineto(341.87890379,349.70703244)
\curveto(342.67577669,349.29296765)(343.46093215,348.98046796)(344.23437254,348.76953244)
\curveto(345.00780561,348.56640588)(345.77342984,348.46484348)(346.53124754,348.46484494)
\curveto(347.54686557,348.46484348)(348.32811479,348.63671831)(348.87499754,348.98046994)
\curveto(349.42186369,349.33203011)(349.69530092,349.82421712)(349.69531004,350.45703244)
\curveto(349.69530092,351.0429659)(349.49608237,351.4921842)(349.09765379,351.80468869)
\curveto(348.70702066,352.11718358)(347.84374027,352.41796453)(346.50781004,352.70703244)
\lineto(345.75781004,352.88281369)
\curveto(344.42186869,353.16405753)(343.45702591,353.5937446)(342.86327879,354.17187619)
\curveto(342.26952709,354.75780594)(341.97265239,355.55858638)(341.97265379,356.57421994)
\curveto(341.97265239,357.80858413)(342.41015195,358.76170818)(343.28515379,359.43359494)
\curveto(344.1601502,360.10545684)(345.40233646,360.441394)(347.01171629,360.44140744)
\curveto(347.80858405,360.441394)(348.5585833,360.38280031)(349.26171629,360.26562619)
\curveto(349.9648319,360.14842554)(350.61326875,359.97264447)(351.20702879,359.73828244)
}
}
{
\newrgbcolor{curcolor}{0 0 0}
\pscustom[linestyle=none,fillstyle=solid,fillcolor=curcolor]
{
\newpath
\moveto(244.72265379,308.35546994)
\lineto(244.72265379,301.94531369)
\lineto(248.51952879,301.94531369)
\curveto(249.79295649,301.94531175)(250.7343618,302.20703024)(251.34374754,302.73046994)
\curveto(251.96092307,303.26171668)(252.26951652,304.07030962)(252.26952879,305.15625119)
\curveto(252.26951652,306.24999494)(251.96092307,307.05468164)(251.34374754,307.57031369)
\curveto(250.7343618,308.0937431)(249.79295649,308.35546159)(248.51952879,308.35546994)
\lineto(244.72265379,308.35546994)
\moveto(244.72265379,315.55078244)
\lineto(244.72265379,310.27734494)
\lineto(248.22656004,310.27734494)
\curveto(249.38280065,310.27733467)(250.24217479,310.4921782)(250.80468504,310.92187619)
\curveto(251.37498616,311.35936483)(251.66014212,312.02342667)(251.66015379,312.91406369)
\curveto(251.66014212,313.7968624)(251.37498616,314.45701799)(250.80468504,314.89453244)
\curveto(250.24217479,315.33201711)(249.38280065,315.55076689)(248.22656004,315.55078244)
\lineto(244.72265379,315.55078244)
\moveto(242.35546629,317.49609494)
\lineto(248.40234129,317.49609494)
\curveto(250.20701858,317.49607745)(251.59764219,317.12107782)(252.57421629,316.37109494)
\curveto(253.55076523,315.62107932)(254.039046,314.55467414)(254.03906004,313.17187619)
\curveto(254.039046,312.10155159)(253.78904625,311.24998994)(253.28906004,310.61718869)
\curveto(252.78904725,309.98436621)(252.05467298,309.58983535)(251.08593504,309.43359494)
\curveto(252.24998529,309.18358576)(253.15232813,308.66014878)(253.79296629,307.86328244)
\curveto(254.44138934,307.07421287)(254.76560777,306.08593261)(254.76562254,304.89843869)
\curveto(254.76560777,303.33593536)(254.2343583,302.12890531)(253.17187254,301.27734494)
\curveto(252.10936043,300.42578202)(250.59764319,300.00000119)(248.63671629,300.00000119)
\lineto(242.35546629,300.00000119)
\lineto(242.35546629,317.49609494)
}
}
{
\newrgbcolor{curcolor}{0 0 0}
\pscustom[linestyle=none,fillstyle=solid,fillcolor=curcolor]
{
\newpath
\moveto(263.80077879,311.61328244)
\curveto(262.64452259,311.61327083)(261.73046101,311.16014628)(261.05859129,310.25390744)
\curveto(260.38671235,309.35546059)(260.05077519,308.12108682)(260.05077879,306.55078244)
\curveto(260.05077519,304.98046496)(260.38280611,303.74218495)(261.04687254,302.83593869)
\curveto(261.71874227,301.93749926)(262.6367101,301.48828096)(263.80077879,301.48828244)
\curveto(264.94920779,301.48828096)(265.85936313,301.9414055)(266.53124754,302.84765744)
\curveto(267.20311179,303.75390369)(267.53904895,304.98827746)(267.53906004,306.55078244)
\curveto(267.53904895,308.10546184)(267.20311179,309.33592936)(266.53124754,310.24218869)
\curveto(265.85936313,311.15624004)(264.94920779,311.61327083)(263.80077879,311.61328244)
\moveto(263.80077879,313.44140744)
\curveto(265.67576956,313.441394)(267.14842434,312.83201961)(268.21874754,311.61328244)
\curveto(269.2890472,310.39452205)(269.82420291,308.70702374)(269.82421629,306.55078244)
\curveto(269.82420291,304.40234054)(269.2890472,302.71484223)(268.21874754,301.48828244)
\curveto(267.14842434,300.26953217)(265.67576956,299.66015778)(263.80077879,299.66015744)
\curveto(261.91796082,299.66015778)(260.4413998,300.26953217)(259.37109129,301.48828244)
\curveto(258.30858943,302.71484223)(257.77733996,304.40234054)(257.77734129,306.55078244)
\curveto(257.77733996,308.70702374)(258.30858943,310.39452205)(259.37109129,311.61328244)
\curveto(260.4413998,312.83201961)(261.91796082,313.441394)(263.80077879,313.44140744)
}
}
{
\newrgbcolor{curcolor}{0 0 0}
\pscustom[linestyle=none,fillstyle=solid,fillcolor=curcolor]
{
\newpath
\moveto(284.29687254,307.92187619)
\lineto(284.29687254,300.00000119)
\lineto(282.14062254,300.00000119)
\lineto(282.14062254,307.85156369)
\curveto(282.14061152,309.0937421)(281.89842426,310.02342867)(281.41406004,310.64062619)
\curveto(280.92967523,311.25780244)(280.20311346,311.56639588)(279.23437254,311.56640744)
\curveto(278.07030309,311.56639588)(277.15233526,311.1953025)(276.48046629,310.45312619)
\curveto(275.8085866,309.71092898)(275.47264944,308.69921124)(275.47265379,307.41796994)
\lineto(275.47265379,300.00000119)
\lineto(273.30468504,300.00000119)
\lineto(273.30468504,313.12500119)
\lineto(275.47265379,313.12500119)
\lineto(275.47265379,311.08593869)
\curveto(275.98827392,311.87498932)(276.59374207,312.46483248)(277.28906004,312.85546994)
\curveto(277.99217817,313.2460817)(278.80077111,313.441394)(279.71484129,313.44140744)
\curveto(281.22264369,313.441394)(282.36326755,312.97264447)(283.13671629,312.03515744)
\curveto(283.910141,311.10545884)(284.29685936,309.73436646)(284.29687254,307.92187619)
}
}
{
\newrgbcolor{curcolor}{0 0 0}
\pscustom[linestyle=none,fillstyle=solid,fillcolor=curcolor]
{
\newpath
\moveto(288.39843504,305.17968869)
\lineto(288.39843504,313.12500119)
\lineto(290.55468504,313.12500119)
\lineto(290.55468504,305.26171994)
\curveto(290.55468084,304.01952842)(290.7968681,303.08593561)(291.28124754,302.46093869)
\curveto(291.76561713,301.84374935)(292.4921789,301.53515591)(293.46093504,301.53515744)
\curveto(294.62498927,301.53515591)(295.5429571,301.90624929)(296.21484129,302.64843869)
\curveto(296.89451825,303.3906228)(297.23436166,304.40234054)(297.23437254,305.68359494)
\lineto(297.23437254,313.12500119)
\lineto(299.39062254,313.12500119)
\lineto(299.39062254,300.00000119)
\lineto(297.23437254,300.00000119)
\lineto(297.23437254,302.01562619)
\curveto(296.71092468,301.21874997)(296.10155029,300.62500057)(295.40624754,300.23437619)
\curveto(294.71873918,299.85156384)(293.91795873,299.66015778)(293.00390379,299.66015744)
\curveto(291.49608615,299.66015778)(290.35155604,300.12890731)(289.57031004,301.06640744)
\curveto(288.78905761,302.00390544)(288.398433,303.37499782)(288.39843504,305.17968869)
\moveto(293.82421629,313.44140744)
\lineto(293.82421629,313.44140744)
}
}
{
\newrgbcolor{curcolor}{0 0 0}
\pscustom[linestyle=none,fillstyle=solid,fillcolor=curcolor]
{
\newpath
\moveto(312.22265379,312.73828244)
\lineto(312.22265379,310.69921994)
\curveto(311.61326877,311.01170893)(310.9804569,311.2460837)(310.32421629,311.40234494)
\curveto(309.66795821,311.55858338)(308.98827139,311.63670831)(308.28515379,311.63671994)
\curveto(307.21483566,311.63670831)(306.41014897,311.47264597)(305.87109129,311.14453244)
\curveto(305.33983754,310.81639663)(305.0742128,310.32420962)(305.07421629,309.66796994)
\curveto(305.0742128,309.16796078)(305.26561886,308.77342992)(305.64843504,308.48437619)
\curveto(306.0312431,308.20311799)(306.80077358,307.93358701)(307.95702879,307.67578244)
\lineto(308.69531004,307.51171994)
\curveto(310.2265514,307.18358776)(311.31248782,306.71874447)(311.95312254,306.11718869)
\curveto(312.60154903,305.52343317)(312.92576745,304.69140275)(312.92577879,303.62109494)
\curveto(312.92576745,302.40234254)(312.44139294,301.43749976)(311.47265379,300.72656369)
\curveto(310.51170737,300.01562618)(309.18748994,299.66015778)(307.49999754,299.66015744)
\curveto(306.79686733,299.66015778)(306.06249307,299.73047021)(305.29687254,299.87109494)
\curveto(304.53905709,300.00390744)(303.73827664,300.20703224)(302.89452879,300.48046994)
\lineto(302.89452879,302.70703244)
\curveto(303.69140169,302.29296765)(304.47655715,301.98046796)(305.24999754,301.76953244)
\curveto(306.02343061,301.56640588)(306.78905484,301.46484348)(307.54687254,301.46484494)
\curveto(308.56249057,301.46484348)(309.34373979,301.63671831)(309.89062254,301.98046994)
\curveto(310.43748869,302.33203011)(310.71092592,302.82421712)(310.71093504,303.45703244)
\curveto(310.71092592,304.0429659)(310.51170737,304.4921842)(310.11327879,304.80468869)
\curveto(309.72264566,305.11718358)(308.85936527,305.41796453)(307.52343504,305.70703244)
\lineto(306.77343504,305.88281369)
\curveto(305.43749369,306.16405753)(304.47265091,306.5937446)(303.87890379,307.17187619)
\curveto(303.28515209,307.75780594)(302.98827739,308.55858638)(302.98827879,309.57421994)
\curveto(302.98827739,310.80858413)(303.42577695,311.76170818)(304.30077879,312.43359494)
\curveto(305.1757752,313.10545684)(306.41796146,313.441394)(308.02734129,313.44140744)
\curveto(308.82420905,313.441394)(309.5742083,313.38280031)(310.27734129,313.26562619)
\curveto(310.9804569,313.14842554)(311.62889375,312.97264447)(312.22265379,312.73828244)
}
}
{
\newrgbcolor{curcolor}{0 0 0}
\pscustom[linestyle=none,fillstyle=solid,fillcolor=curcolor]
{
\newpath
\moveto(238.20312254,265.16406369)
\lineto(234.99218504,256.45703244)
\lineto(241.42577879,256.45703244)
\lineto(238.20312254,265.16406369)
\moveto(236.86718504,267.49609494)
\lineto(239.55077879,267.49609494)
\lineto(246.21874754,250.00000119)
\lineto(243.75781004,250.00000119)
\lineto(242.16406004,254.48828244)
\lineto(234.27734129,254.48828244)
\lineto(232.68359129,250.00000119)
\lineto(230.18749754,250.00000119)
\lineto(236.86718504,267.49609494)
}
}
{
\newrgbcolor{curcolor}{0 0 0}
\pscustom[linestyle=none,fillstyle=solid,fillcolor=curcolor]
{
}
}
{
\newrgbcolor{curcolor}{0 0 0}
\pscustom[linestyle=none,fillstyle=solid,fillcolor=curcolor]
{
\newpath
\moveto(258.39452879,251.96875119)
\lineto(258.39452879,245.00781369)
\lineto(256.22656004,245.00781369)
\lineto(256.22656004,263.12500119)
\lineto(258.39452879,263.12500119)
\lineto(258.39452879,261.13281369)
\curveto(258.84764898,261.91405178)(259.41796091,262.4921762)(260.10546629,262.86718869)
\curveto(260.80077203,263.24998794)(261.6288962,263.441394)(262.58984129,263.44140744)
\curveto(264.18358115,263.441394)(265.47654861,262.80858213)(266.46874754,261.54296994)
\curveto(267.46873411,260.27733467)(267.96873361,258.61327383)(267.96874754,256.55078244)
\curveto(267.96873361,254.48827796)(267.46873411,252.82421712)(266.46874754,251.55859494)
\curveto(265.47654861,250.29296965)(264.18358115,249.66015778)(262.58984129,249.66015744)
\curveto(261.6288962,249.66015778)(260.80077203,249.8476576)(260.10546629,250.22265744)
\curveto(259.41796091,250.60546934)(258.84764898,251.18750001)(258.39452879,251.96875119)
\moveto(265.73046629,256.55078244)
\curveto(265.7304546,258.13671181)(265.40232993,259.37889806)(264.74609129,260.27734494)
\curveto(264.09764373,261.18358376)(263.20311338,261.63670831)(262.06249754,261.63671994)
\curveto(260.92186566,261.63670831)(260.02342906,261.18358376)(259.36718504,260.27734494)
\curveto(258.71874286,259.37889806)(258.39452444,258.13671181)(258.39452879,256.55078244)
\curveto(258.39452444,254.96483998)(258.71874286,253.71874747)(259.36718504,252.81250119)
\curveto(260.02342906,251.91406178)(260.92186566,251.46484348)(262.06249754,251.46484494)
\curveto(263.20311338,251.46484348)(264.09764373,251.91406178)(264.74609129,252.81250119)
\curveto(265.40232993,253.71874747)(265.7304546,254.96483998)(265.73046629,256.55078244)
}
}
{
\newrgbcolor{curcolor}{0 0 0}
\pscustom[linestyle=none,fillstyle=solid,fillcolor=curcolor]
{
\newpath
\moveto(279.14843504,261.10937619)
\curveto(278.90623791,261.24998994)(278.64061318,261.35155234)(278.35156004,261.41406369)
\curveto(278.07030125,261.48436471)(277.75780156,261.51952092)(277.41406004,261.51953244)
\curveto(276.19530312,261.51952092)(275.25780406,261.12108382)(274.60156004,260.32421994)
\curveto(273.95311786,259.53514791)(273.62889944,258.39843029)(273.62890379,256.91406369)
\lineto(273.62890379,250.00000119)
\lineto(271.46093504,250.00000119)
\lineto(271.46093504,263.12500119)
\lineto(273.62890379,263.12500119)
\lineto(273.62890379,261.08593869)
\curveto(274.08202398,261.88280181)(274.67186714,262.47264497)(275.39843504,262.85546994)
\curveto(276.12499069,263.2460817)(277.00780231,263.441394)(278.04687254,263.44140744)
\curveto(278.19530112,263.441394)(278.35936346,263.42967526)(278.53906004,263.40625119)
\curveto(278.7187381,263.3906128)(278.91795665,263.36326908)(279.13671629,263.32421994)
\lineto(279.14843504,261.10937619)
}
}
{
\newrgbcolor{curcolor}{0 0 0}
\pscustom[linestyle=none,fillstyle=solid,fillcolor=curcolor]
{
\newpath
\moveto(286.00390379,261.61328244)
\curveto(284.84764759,261.61327083)(283.93358601,261.16014628)(283.26171629,260.25390744)
\curveto(282.58983735,259.35546059)(282.25390019,258.12108682)(282.25390379,256.55078244)
\curveto(282.25390019,254.98046496)(282.58593111,253.74218495)(283.24999754,252.83593869)
\curveto(283.92186727,251.93749926)(284.8398351,251.48828096)(286.00390379,251.48828244)
\curveto(287.15233279,251.48828096)(288.06248813,251.9414055)(288.73437254,252.84765744)
\curveto(289.40623679,253.75390369)(289.74217395,254.98827746)(289.74218504,256.55078244)
\curveto(289.74217395,258.10546184)(289.40623679,259.33592936)(288.73437254,260.24218869)
\curveto(288.06248813,261.15624004)(287.15233279,261.61327083)(286.00390379,261.61328244)
\moveto(286.00390379,263.44140744)
\curveto(287.87889456,263.441394)(289.35154934,262.83201961)(290.42187254,261.61328244)
\curveto(291.4921722,260.39452205)(292.02732791,258.70702374)(292.02734129,256.55078244)
\curveto(292.02732791,254.40234054)(291.4921722,252.71484223)(290.42187254,251.48828244)
\curveto(289.35154934,250.26953217)(287.87889456,249.66015778)(286.00390379,249.66015744)
\curveto(284.12108582,249.66015778)(282.6445248,250.26953217)(281.57421629,251.48828244)
\curveto(280.51171443,252.71484223)(279.98046496,254.40234054)(279.98046629,256.55078244)
\curveto(279.98046496,258.70702374)(280.51171443,260.39452205)(281.57421629,261.61328244)
\curveto(282.6445248,262.83201961)(284.12108582,263.441394)(286.00390379,263.44140744)
}
}
{
\newrgbcolor{curcolor}{0 0 0}
\pscustom[linestyle=none,fillstyle=solid,fillcolor=curcolor]
{
\newpath
\moveto(297.67577879,251.96875119)
\lineto(297.67577879,245.00781369)
\lineto(295.50781004,245.00781369)
\lineto(295.50781004,263.12500119)
\lineto(297.67577879,263.12500119)
\lineto(297.67577879,261.13281369)
\curveto(298.12889898,261.91405178)(298.69921091,262.4921762)(299.38671629,262.86718869)
\curveto(300.08202203,263.24998794)(300.9101462,263.441394)(301.87109129,263.44140744)
\curveto(303.46483115,263.441394)(304.75779861,262.80858213)(305.74999754,261.54296994)
\curveto(306.74998411,260.27733467)(307.24998361,258.61327383)(307.24999754,256.55078244)
\curveto(307.24998361,254.48827796)(306.74998411,252.82421712)(305.74999754,251.55859494)
\curveto(304.75779861,250.29296965)(303.46483115,249.66015778)(301.87109129,249.66015744)
\curveto(300.9101462,249.66015778)(300.08202203,249.8476576)(299.38671629,250.22265744)
\curveto(298.69921091,250.60546934)(298.12889898,251.18750001)(297.67577879,251.96875119)
\moveto(305.01171629,256.55078244)
\curveto(305.0117046,258.13671181)(304.68357993,259.37889806)(304.02734129,260.27734494)
\curveto(303.37889373,261.18358376)(302.48436338,261.63670831)(301.34374754,261.63671994)
\curveto(300.20311566,261.63670831)(299.30467906,261.18358376)(298.64843504,260.27734494)
\curveto(297.99999286,259.37889806)(297.67577444,258.13671181)(297.67577879,256.55078244)
\curveto(297.67577444,254.96483998)(297.99999286,253.71874747)(298.64843504,252.81250119)
\curveto(299.30467906,251.91406178)(300.20311566,251.46484348)(301.34374754,251.46484494)
\curveto(302.48436338,251.46484348)(303.37889373,251.91406178)(304.02734129,252.81250119)
\curveto(304.68357993,253.71874747)(305.0117046,254.96483998)(305.01171629,256.55078244)
}
}
{
\newrgbcolor{curcolor}{0 0 0}
\pscustom[linestyle=none,fillstyle=solid,fillcolor=curcolor]
{
\newpath
\moveto(315.91015379,261.61328244)
\curveto(314.75389759,261.61327083)(313.83983601,261.16014628)(313.16796629,260.25390744)
\curveto(312.49608735,259.35546059)(312.16015019,258.12108682)(312.16015379,256.55078244)
\curveto(312.16015019,254.98046496)(312.49218111,253.74218495)(313.15624754,252.83593869)
\curveto(313.82811727,251.93749926)(314.7460851,251.48828096)(315.91015379,251.48828244)
\curveto(317.05858279,251.48828096)(317.96873813,251.9414055)(318.64062254,252.84765744)
\curveto(319.31248679,253.75390369)(319.64842395,254.98827746)(319.64843504,256.55078244)
\curveto(319.64842395,258.10546184)(319.31248679,259.33592936)(318.64062254,260.24218869)
\curveto(317.96873813,261.15624004)(317.05858279,261.61327083)(315.91015379,261.61328244)
\moveto(315.91015379,263.44140744)
\curveto(317.78514456,263.441394)(319.25779934,262.83201961)(320.32812254,261.61328244)
\curveto(321.3984222,260.39452205)(321.93357791,258.70702374)(321.93359129,256.55078244)
\curveto(321.93357791,254.40234054)(321.3984222,252.71484223)(320.32812254,251.48828244)
\curveto(319.25779934,250.26953217)(317.78514456,249.66015778)(315.91015379,249.66015744)
\curveto(314.02733582,249.66015778)(312.5507748,250.26953217)(311.48046629,251.48828244)
\curveto(310.41796443,252.71484223)(309.88671496,254.40234054)(309.88671629,256.55078244)
\curveto(309.88671496,258.70702374)(310.41796443,260.39452205)(311.48046629,261.61328244)
\curveto(312.5507748,262.83201961)(314.02733582,263.441394)(315.91015379,263.44140744)
}
}
{
\newrgbcolor{curcolor}{0 0 0}
\pscustom[linestyle=none,fillstyle=solid,fillcolor=curcolor]
{
\newpath
\moveto(333.86327879,262.73828244)
\lineto(333.86327879,260.69921994)
\curveto(333.25389377,261.01170893)(332.6210819,261.2460837)(331.96484129,261.40234494)
\curveto(331.30858321,261.55858338)(330.62889639,261.63670831)(329.92577879,261.63671994)
\curveto(328.85546066,261.63670831)(328.05077397,261.47264597)(327.51171629,261.14453244)
\curveto(326.98046254,260.81639663)(326.7148378,260.32420962)(326.71484129,259.66796994)
\curveto(326.7148378,259.16796078)(326.90624386,258.77342992)(327.28906004,258.48437619)
\curveto(327.6718681,258.20311799)(328.44139858,257.93358701)(329.59765379,257.67578244)
\lineto(330.33593504,257.51171994)
\curveto(331.8671764,257.18358776)(332.95311282,256.71874447)(333.59374754,256.11718869)
\curveto(334.24217403,255.52343317)(334.56639245,254.69140275)(334.56640379,253.62109494)
\curveto(334.56639245,252.40234254)(334.08201794,251.43749976)(333.11327879,250.72656369)
\curveto(332.15233237,250.01562618)(330.82811494,249.66015778)(329.14062254,249.66015744)
\curveto(328.43749233,249.66015778)(327.70311807,249.73047021)(326.93749754,249.87109494)
\curveto(326.17968209,250.00390744)(325.37890164,250.20703224)(324.53515379,250.48046994)
\lineto(324.53515379,252.70703244)
\curveto(325.33202669,252.29296765)(326.11718215,251.98046796)(326.89062254,251.76953244)
\curveto(327.66405561,251.56640588)(328.42967984,251.46484348)(329.18749754,251.46484494)
\curveto(330.20311557,251.46484348)(330.98436479,251.63671831)(331.53124754,251.98046994)
\curveto(332.07811369,252.33203011)(332.35155092,252.82421712)(332.35156004,253.45703244)
\curveto(332.35155092,254.0429659)(332.15233237,254.4921842)(331.75390379,254.80468869)
\curveto(331.36327066,255.11718358)(330.49999027,255.41796453)(329.16406004,255.70703244)
\lineto(328.41406004,255.88281369)
\curveto(327.07811869,256.16405753)(326.11327591,256.5937446)(325.51952879,257.17187619)
\curveto(324.92577709,257.75780594)(324.62890239,258.55858638)(324.62890379,259.57421994)
\curveto(324.62890239,260.80858413)(325.06640195,261.76170818)(325.94140379,262.43359494)
\curveto(326.8164002,263.10545684)(328.05858646,263.441394)(329.66796629,263.44140744)
\curveto(330.46483405,263.441394)(331.2148333,263.38280031)(331.91796629,263.26562619)
\curveto(332.6210819,263.14842554)(333.26951875,262.97264447)(333.86327879,262.73828244)
}
}
{
\newrgbcolor{curcolor}{0 0 0}
\pscustom[linestyle=none,fillstyle=solid,fillcolor=curcolor]
{
\newpath
\moveto(150.65234129,38.20309568)
\curveto(151.16014262,38.03121264)(151.65232963,37.66402551)(152.12890379,37.10153318)
\curveto(152.61326617,36.53902664)(153.09764069,35.76558991)(153.58202879,34.78122068)
\lineto(155.98437254,29.99997068)
\lineto(153.44140379,29.99997068)
\lineto(151.20312254,34.48825193)
\curveto(150.62498691,35.66012127)(150.06248747,36.43746424)(149.51562254,36.82028318)
\curveto(148.97655106,37.20308847)(148.23827055,37.39449453)(147.30077879,37.39450193)
\lineto(144.72265379,37.39450193)
\lineto(144.72265379,29.99997068)
\lineto(142.35546629,29.99997068)
\lineto(142.35546629,47.49606443)
\lineto(147.69921629,47.49606443)
\curveto(149.69920659,47.49604693)(151.19139259,47.0780786)(152.17577879,46.24215818)
\curveto(153.16014062,45.40620527)(153.65232763,44.14448778)(153.65234129,42.45700193)
\curveto(153.65232763,41.35542807)(153.39451539,40.44136648)(152.87890379,39.71481443)
\curveto(152.37107891,38.98824294)(151.62889216,38.48433719)(150.65234129,38.20309568)
\moveto(144.72265379,45.55075193)
\lineto(144.72265379,39.33981443)
\lineto(147.69921629,39.33981443)
\curveto(148.83983245,39.33980509)(149.69920659,39.60152357)(150.27734129,40.12497068)
\curveto(150.86326792,40.65621002)(151.15623638,41.43355299)(151.15624754,42.45700193)
\curveto(151.15623638,43.48042595)(150.86326792,44.24995643)(150.27734129,44.76559568)
\curveto(149.69920659,45.28901789)(148.83983245,45.55073638)(147.69921629,45.55075193)
\lineto(144.72265379,45.55075193)
}
}
{
\newrgbcolor{curcolor}{0 0 0}
\pscustom[linestyle=none,fillstyle=solid,fillcolor=curcolor]
{
\newpath
\moveto(169.09765379,37.10153318)
\lineto(169.09765379,36.04684568)
\lineto(159.18359129,36.04684568)
\curveto(159.27733762,34.56246611)(159.72264967,33.42965475)(160.51952879,32.64840818)
\curveto(161.32421057,31.8749688)(162.44139695,31.48825044)(163.87109129,31.48825193)
\curveto(164.6992072,31.48825044)(165.49998764,31.58981284)(166.27343504,31.79293943)
\curveto(167.05467359,31.99606243)(167.82811032,32.30074962)(168.59374754,32.70700193)
\lineto(168.59374754,30.66793943)
\curveto(167.82029782,30.33981409)(167.02732987,30.08981434)(166.21484129,29.91793943)
\curveto(165.40233149,29.74606468)(164.57811357,29.66012727)(163.74218504,29.66012693)
\curveto(161.648429,29.66012727)(159.98827441,30.26950166)(158.76171629,31.48825193)
\curveto(157.54296435,32.70699922)(156.93358996,34.35543507)(156.93359129,36.43356443)
\curveto(156.93358996,38.58199334)(157.51171438,40.28511664)(158.66796629,41.54293943)
\curveto(159.83202456,42.80855162)(161.39842925,43.44136348)(163.36718504,43.44137693)
\curveto(165.13280051,43.44136348)(166.52733037,42.87105155)(167.55077879,41.73043943)
\curveto(168.58201581,40.59761633)(169.0976403,39.05464912)(169.09765379,37.10153318)
\moveto(166.94140379,37.73434568)
\curveto(166.92576747,38.91402426)(166.59373655,39.85542957)(165.94531004,40.55856443)
\curveto(165.30467534,41.26167816)(164.45311369,41.61324031)(163.39062254,41.61325193)
\curveto(162.18749096,41.61324031)(161.22264817,41.2733969)(160.49609129,40.59372068)
\curveto(159.77733712,39.91402326)(159.36327503,38.95699297)(159.25390379,37.72262693)
\lineto(166.94140379,37.73434568)
}
}
{
\newrgbcolor{curcolor}{0 0 0}
\pscustom[linestyle=none,fillstyle=solid,fillcolor=curcolor]
{
\newpath
\moveto(174.76952879,46.85153318)
\lineto(174.76952879,43.12497068)
\lineto(179.21093504,43.12497068)
\lineto(179.21093504,41.44918943)
\lineto(174.76952879,41.44918943)
\lineto(174.76952879,34.32418943)
\curveto(174.76952439,33.25387367)(174.9140555,32.56637436)(175.20312254,32.26168943)
\curveto(175.49999241,31.95699997)(176.09764806,31.80465637)(176.99609129,31.80465818)
\lineto(179.21093504,31.80465818)
\lineto(179.21093504,29.99997068)
\lineto(176.99609129,29.99997068)
\curveto(175.33202383,29.99997068)(174.18358748,30.30856412)(173.55077879,30.92575193)
\curveto(172.91796374,31.55075038)(172.60155781,32.68356174)(172.60156004,34.32418943)
\lineto(172.60156004,41.44918943)
\lineto(171.01952879,41.44918943)
\lineto(171.01952879,43.12497068)
\lineto(172.60156004,43.12497068)
\lineto(172.60156004,46.85153318)
\lineto(174.76952879,46.85153318)
}
}
{
\newrgbcolor{curcolor}{0 0 0}
\pscustom[linestyle=none,fillstyle=solid,fillcolor=curcolor]
{
\newpath
\moveto(187.14452879,41.61325193)
\curveto(185.98827259,41.61324031)(185.07421101,41.16011577)(184.40234129,40.25387693)
\curveto(183.73046235,39.35543007)(183.39452519,38.1210563)(183.39452879,36.55075193)
\curveto(183.39452519,34.98043445)(183.72655611,33.74215443)(184.39062254,32.83590818)
\curveto(185.06249227,31.93746874)(185.9804601,31.48825044)(187.14452879,31.48825193)
\curveto(188.29295779,31.48825044)(189.20311313,31.94137498)(189.87499754,32.84762693)
\curveto(190.54686179,33.75387317)(190.88279895,34.98824694)(190.88281004,36.55075193)
\curveto(190.88279895,38.10543132)(190.54686179,39.33589884)(189.87499754,40.24215818)
\curveto(189.20311313,41.15620952)(188.29295779,41.61324031)(187.14452879,41.61325193)
\moveto(187.14452879,43.44137693)
\curveto(189.01951956,43.44136348)(190.49217434,42.83198909)(191.56249754,41.61325193)
\curveto(192.6327972,40.39449153)(193.16795291,38.70699322)(193.16796629,36.55075193)
\curveto(193.16795291,34.40231002)(192.6327972,32.71481171)(191.56249754,31.48825193)
\curveto(190.49217434,30.26950166)(189.01951956,29.66012727)(187.14452879,29.66012693)
\curveto(185.26171082,29.66012727)(183.7851498,30.26950166)(182.71484129,31.48825193)
\curveto(181.65233943,32.71481171)(181.12108996,34.40231002)(181.12109129,36.55075193)
\curveto(181.12108996,38.70699322)(181.65233943,40.39449153)(182.71484129,41.61325193)
\curveto(183.7851498,42.83198909)(185.26171082,43.44136348)(187.14452879,43.44137693)
}
}
{
\newrgbcolor{curcolor}{0 0 0}
\pscustom[linestyle=none,fillstyle=solid,fillcolor=curcolor]
{
\newpath
\moveto(196.50781004,35.17965818)
\lineto(196.50781004,43.12497068)
\lineto(198.66406004,43.12497068)
\lineto(198.66406004,35.26168943)
\curveto(198.66405584,34.01949791)(198.9062431,33.08590509)(199.39062254,32.46090818)
\curveto(199.87499213,31.84371883)(200.6015539,31.53512539)(201.57031004,31.53512693)
\curveto(202.73436427,31.53512539)(203.6523321,31.90621877)(204.32421629,32.64840818)
\curveto(205.00389325,33.39059229)(205.34373666,34.40231002)(205.34374754,35.68356443)
\lineto(205.34374754,43.12497068)
\lineto(207.49999754,43.12497068)
\lineto(207.49999754,29.99997068)
\lineto(205.34374754,29.99997068)
\lineto(205.34374754,32.01559568)
\curveto(204.82029968,31.21871946)(204.21092529,30.62497005)(203.51562254,30.23434568)
\curveto(202.82811418,29.85153332)(202.02733373,29.66012727)(201.11327879,29.66012693)
\curveto(199.60546115,29.66012727)(198.46093104,30.1288768)(197.67968504,31.06637693)
\curveto(196.89843261,32.00387492)(196.507808,33.3749673)(196.50781004,35.17965818)
\moveto(201.93359129,43.44137693)
\lineto(201.93359129,43.44137693)
}
}
{
\newrgbcolor{curcolor}{0 0 0}
\pscustom[linestyle=none,fillstyle=solid,fillcolor=curcolor]
{
\newpath
\moveto(219.57031004,41.10934568)
\curveto(219.32811291,41.24995943)(219.06248818,41.35152182)(218.77343504,41.41403318)
\curveto(218.49217625,41.48433419)(218.17967656,41.51949041)(217.83593504,41.51950193)
\curveto(216.61717812,41.51949041)(215.67967906,41.1210533)(215.02343504,40.32418943)
\curveto(214.37499286,39.53511739)(214.05077444,38.39839978)(214.05077879,36.91403318)
\lineto(214.05077879,29.99997068)
\lineto(211.88281004,29.99997068)
\lineto(211.88281004,43.12497068)
\lineto(214.05077879,43.12497068)
\lineto(214.05077879,41.08590818)
\curveto(214.50389898,41.88277129)(215.09374214,42.47261445)(215.82031004,42.85543943)
\curveto(216.54686569,43.24605118)(217.42967731,43.44136348)(218.46874754,43.44137693)
\curveto(218.61717612,43.44136348)(218.78123846,43.42964475)(218.96093504,43.40622068)
\curveto(219.1406131,43.39058229)(219.33983165,43.36323856)(219.55859129,43.32418943)
\lineto(219.57031004,41.10934568)
}
}
{
\newrgbcolor{curcolor}{0 0 0}
\pscustom[linewidth=2,linecolor=curcolor,linestyle=dashed,dash=8 8]
{
\newpath
\moveto(210,410)
\lineto(60,410)
}
}
{
\newrgbcolor{curcolor}{0 0 0}
\pscustom[linestyle=none,fillstyle=solid,fillcolor=curcolor]
{
\newpath
\moveto(199.53769464,414.84048224)
\lineto(212.6487474,410.01921591)
\lineto(199.53769392,405.19795064)
\curveto(201.632292,408.04442372)(201.62022288,411.93889292)(199.53769464,414.84048224)
\lineto(199.53769464,414.84048224)
\closepath
}
}
{
\newrgbcolor{curcolor}{0 0 0}
\pscustom[linewidth=2,linecolor=curcolor,linestyle=dashed,dash=8 8]
{
\newpath
\moveto(359,359)
\lineto(500,360)
}
}
{
\newrgbcolor{curcolor}{0 0 0}
\pscustom[linestyle=none,fillstyle=solid,fillcolor=curcolor]
{
\newpath
\moveto(369.49637104,354.23383837)
\lineto(356.35145549,358.9619996)
\lineto(369.42798674,363.87612748)
\curveto(367.35362858,361.01487105)(367.39331705,357.12058539)(369.49637104,354.23383837)
\lineto(369.49637104,354.23383837)
\closepath
}
}
{
\newrgbcolor{curcolor}{0 0 0}
\pscustom[linewidth=2.06155276,linecolor=curcolor,linestyle=dashed,dash=8.24621105 8.24621105]
{
\newpath
\moveto(229.999999,309.999998)
\lineto(60,309.999998)
}
}
{
\newrgbcolor{curcolor}{0 0 0}
\pscustom[linestyle=none,fillstyle=solid,fillcolor=curcolor]
{
\newpath
\moveto(219.21570174,314.98945277)
\lineto(232.73026526,310.01980531)
\lineto(219.215701,305.05015894)
\curveto(221.37476323,307.98423616)(221.36232266,311.99856303)(219.21570174,314.98945277)
\lineto(219.21570174,314.98945277)
\closepath
}
}
{
\newrgbcolor{curcolor}{0 0 0}
\pscustom[linewidth=2.0655911,linecolor=curcolor,linestyle=dashed,dash=8.26236439 8.26236439]
{
\newpath
\moveto(219.999999,259.999998)
\lineto(60,259.999998)
}
}
{
\newrgbcolor{curcolor}{0 0 0}
\pscustom[linestyle=none,fillstyle=solid,fillcolor=curcolor]
{
\newpath
\moveto(209.1945766,264.99922651)
\lineto(222.73561352,260.01984411)
\lineto(209.19457585,255.0404628)
\curveto(211.35786743,257.98028752)(211.34540249,262.00247798)(209.1945766,264.99922651)
\lineto(209.1945766,264.99922651)
\closepath
}
}
{
\newrgbcolor{curcolor}{0 0 0}
\pscustom[linewidth=2,linecolor=curcolor,linestyle=dashed,dash=8 8]
{
\newpath
\moveto(140,40)
\lineto(60,40)
}
}
{
\newrgbcolor{curcolor}{0 0 0}
\pscustom[linestyle=none,fillstyle=solid,fillcolor=curcolor]
{
\newpath
\moveto(129.53769464,44.84048224)
\lineto(142.6487474,40.01921591)
\lineto(129.53769392,35.19795064)
\curveto(131.632292,38.04442372)(131.62022288,41.93889292)(129.53769464,44.84048224)
\lineto(129.53769464,44.84048224)
\closepath
}
}
{
\newrgbcolor{curcolor}{0 0 0}
\pscustom[linestyle=none,fillstyle=solid,fillcolor=curcolor]
{
\newpath
\moveto(512.98437254,362.57812619)
\curveto(514.49477471,362.25519727)(515.67185686,361.58332294)(516.51562254,360.56250119)
\curveto(517.36977183,359.54165832)(517.79685474,358.28124291)(517.79687254,356.78125119)
\curveto(517.79685474,354.47916338)(517.00518886,352.69791516)(515.42187254,351.43750119)
\curveto(513.83852536,350.17708435)(511.58852761,349.54687665)(508.67187254,349.54687619)
\curveto(507.69269818,349.54687665)(506.68228252,349.64583488)(505.64062254,349.84375119)
\curveto(504.60936793,350.03125116)(503.54166066,350.31770921)(502.43749754,350.70312619)
\lineto(502.43749754,353.75000119)
\curveto(503.31249422,353.23958129)(504.2708266,352.85416501)(505.31249754,352.59375119)
\curveto(506.35415785,352.33333219)(507.44269843,352.20312399)(508.57812254,352.20312619)
\curveto(510.55727864,352.20312399)(512.06248547,352.5937486)(513.09374754,353.37500119)
\curveto(514.13540007,354.15624704)(514.65623288,355.29166257)(514.65624754,356.78125119)
\curveto(514.65623288,358.15624304)(514.17185836,359.22915863)(513.20312254,360.00000119)
\curveto(512.24477696,360.78124041)(510.90623663,361.17186502)(509.18749754,361.17187619)
\lineto(506.46874754,361.17187619)
\lineto(506.46874754,363.76562619)
\lineto(509.31249754,363.76562619)
\curveto(510.86457,363.76561243)(512.05206882,364.07290379)(512.87499754,364.68750119)
\curveto(513.6979005,365.31248588)(514.10935843,366.20831832)(514.10937254,367.37500119)
\curveto(514.10935843,368.57289929)(513.68227552,369.48956504)(512.82812254,370.12500119)
\curveto(511.98436055,370.77081376)(510.7708201,371.0937301)(509.18749754,371.09375119)
\curveto(508.32290588,371.0937301)(507.39582347,370.99998019)(506.40624754,370.81250119)
\curveto(505.41665879,370.62498057)(504.32811821,370.33331419)(503.14062254,369.93750119)
\lineto(503.14062254,372.75000119)
\curveto(504.33853486,373.08331144)(505.45832541,373.33331119)(506.49999754,373.50000119)
\curveto(507.55207332,373.66664419)(508.54165566,373.74997744)(509.46874754,373.75000119)
\curveto(511.864569,373.74997744)(513.76040044,373.20310299)(515.15624754,372.10937619)
\curveto(516.55206432,371.02602183)(517.24998029,369.5572733)(517.24999754,367.70312619)
\curveto(517.24998029,366.41144312)(516.88018899,365.31769421)(516.14062254,364.42187619)
\curveto(515.4010238,363.53644599)(514.34894152,362.92186327)(512.98437254,362.57812619)
}
}
{
\newrgbcolor{curcolor}{0 0 0}
\pscustom[linestyle=none,fillstyle=solid,fillcolor=curcolor]
{
\newpath
\moveto(52.09374944,320.57812619)
\lineto(44.12499944,308.12500119)
\lineto(52.09374944,308.12500119)
\lineto(52.09374944,320.57812619)
\moveto(51.26562444,323.32812619)
\lineto(55.23437444,323.32812619)
\lineto(55.23437444,308.12500119)
\lineto(58.56249944,308.12500119)
\lineto(58.56249944,305.50000119)
\lineto(55.23437444,305.50000119)
\lineto(55.23437444,300.00000119)
\lineto(52.09374944,300.00000119)
\lineto(52.09374944,305.50000119)
\lineto(41.56249944,305.50000119)
\lineto(41.56249944,308.54687619)
\lineto(51.26562444,323.32812619)
}
}
{
\newrgbcolor{curcolor}{0 0 0}
\pscustom[linestyle=none,fillstyle=solid,fillcolor=curcolor]
{
\newpath
\moveto(43.45312444,273.32812619)
\lineto(55.84374944,273.32812619)
\lineto(55.84374944,270.67187619)
\lineto(46.34374944,270.67187619)
\lineto(46.34374944,264.95312619)
\curveto(46.80207597,265.10936108)(47.26040885,265.2239443)(47.71874944,265.29687619)
\curveto(48.1770746,265.38019415)(48.63540747,265.42186077)(49.09374944,265.42187619)
\curveto(51.69790441,265.42186077)(53.76040235,264.70831982)(55.28124944,263.28125119)
\curveto(56.80206597,261.85415601)(57.56248188,259.92186627)(57.56249944,257.48437619)
\curveto(57.56248188,254.97395455)(56.78123266,253.02083151)(55.21874944,251.62500119)
\curveto(53.65623579,250.23958429)(51.45311299,249.54687665)(48.60937444,249.54687619)
\curveto(47.63020015,249.54687665)(46.63020115,249.6302099)(45.60937444,249.79687619)
\curveto(44.59895318,249.9635429)(43.55207922,250.21354265)(42.46874944,250.54687619)
\lineto(42.46874944,253.71875119)
\curveto(43.40624604,253.20833132)(44.37499507,252.82812337)(45.37499944,252.57812619)
\curveto(46.37499307,252.32812387)(47.43228368,252.20312399)(48.54687444,252.20312619)
\curveto(50.34894743,252.20312399)(51.77602933,252.67708185)(52.82812444,253.62500119)
\curveto(53.8801939,254.57291329)(54.40623504,255.85937033)(54.40624944,257.48437619)
\curveto(54.40623504,259.10936708)(53.8801939,260.39582413)(52.82812444,261.34375119)
\curveto(51.77602933,262.29165557)(50.34894743,262.76561343)(48.54687444,262.76562619)
\curveto(47.70311674,262.76561343)(46.85936758,262.67186352)(46.01562444,262.48437619)
\curveto(45.18228593,262.2968639)(44.32812011,262.00519752)(43.45312444,261.60937619)
\lineto(43.45312444,273.32812619)
}
}
{
\newrgbcolor{curcolor}{0 0 0}
\pscustom[linestyle=none,fillstyle=solid,fillcolor=curcolor]
{
\newpath
\moveto(50.56249944,42.92184568)
\curveto(49.14582363,42.92183275)(48.02082476,42.43745824)(47.18749944,41.46872068)
\curveto(46.36457641,40.49996018)(45.95311849,39.1718365)(45.95312444,37.48434568)
\curveto(45.95311849,35.80725654)(46.36457641,34.47913286)(47.18749944,33.49997068)
\curveto(48.02082476,32.53121814)(49.14582363,32.04684363)(50.56249944,32.04684568)
\curveto(51.97915413,32.04684363)(53.09894468,32.53121814)(53.92187444,33.49997068)
\curveto(54.75519302,34.47913286)(55.17185927,35.80725654)(55.17187444,37.48434568)
\curveto(55.17185927,39.1718365)(54.75519302,40.49996018)(53.92187444,41.46872068)
\curveto(53.09894468,42.43745824)(51.97915413,42.92183275)(50.56249944,42.92184568)
\moveto(56.82812444,52.81247068)
\lineto(56.82812444,49.93747068)
\curveto(56.03644174,50.31245036)(55.23435921,50.59890841)(54.42187444,50.79684568)
\curveto(53.61977749,50.99474135)(52.82290329,51.09369958)(52.03124944,51.09372068)
\curveto(49.94790616,51.09369958)(48.35415776,50.39057529)(47.24999944,48.98434568)
\curveto(46.15624329,47.5780781)(45.53124391,45.45308022)(45.37499944,42.60934568)
\curveto(45.98957679,43.51558216)(46.76040935,44.2082898)(47.68749944,44.68747068)
\curveto(48.61457416,45.17703883)(49.63540647,45.42183025)(50.74999944,45.42184568)
\curveto(53.09373635,45.42183025)(54.94269283,44.7082893)(56.29687444,43.28122068)
\curveto(57.66144011,41.86454214)(58.3437311,39.93225241)(58.34374944,37.48434568)
\curveto(58.3437311,35.08850725)(57.63539847,33.16663418)(56.21874944,31.71872068)
\curveto(54.80206797,30.27080374)(52.91665319,29.54684613)(50.56249944,29.54684568)
\curveto(47.86457491,29.54684613)(45.80207697,30.5780951)(44.37499944,32.64059568)
\curveto(42.94791316,34.71350763)(42.23437221,37.71350463)(42.23437444,41.64059568)
\curveto(42.23437221,45.32808035)(43.10937133,48.26557741)(44.85937444,50.45309568)
\curveto(46.60936783,52.65098969)(48.95832382,53.74994693)(51.90624944,53.74997068)
\curveto(52.69790341,53.74994693)(53.49477761,53.671822)(54.29687444,53.51559568)
\curveto(55.10935933,53.35932232)(55.95310849,53.12494755)(56.82812444,52.81247068)
}
}
\end{pspicture}

		\end{center}

		\begin{enumerate}
		  \item Fond d'écran (présent dans l'archive stocké sur le téléphone)
		  \item Bouton ``But du jeu''
		  \item Bouton ``Instructions''
		  \item Bouton ``Bonus''
		  \item Bouton ``A propos''
		  \item Bouton ``\hyperlink{Accueil}{Retour}''
		\end{enumerate}

		\subsubsection{Description des zones}
		
			\begin{tabular}{|c|c|c|c|c|} \hline
				Numéro de zone & Type  & Description & Evènement &	Règle \\\hline
				2 & Bouton & Ouvre la page ``But du jeu'' & Cliqué & RG13-01 \\\hline
				3 & Bouton & Ouvre la page ``Instructions'' & Cliqué & RG13-02 \\\hline
				4 & Bouton & Ouvre la page ``Bonus'' & Cliqué & RG13-03 \\\hline
				5 & Bouton & Ouvre la page ``A propos'' & Cliqué & RG13-04 \\\hline
				6 & Bouton & Permet de revenir à la page d'accueil & Cliqué & RG13-05 \\\hline
			\end{tabular}
			
		\subsubsection{Description des règles}

			\underline{RG13-01 :}
				\begin{quote}
				
				\end{quote}	
\end{document}
